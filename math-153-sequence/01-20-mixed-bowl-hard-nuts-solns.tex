\documentclass[10pt]{amsart}

%Packages in use
\usepackage{fullpage, hyperref, vipul, enumerate}

%Title details
\title{Class quiz solutions: January 20: Mixed bowl of hard nuts}
\author{Math 153, Section 55 (Vipul Naik)}
%List of new commands

\begin{document}
\maketitle

\section{Performance review}

$11$ people took this $14$-question quiz. The score distribution was
as follows:

\begin{itemize}
\item Score of $4$: $1$ person
\item Score of $7$: $2$ people
\item Score of $10$: $1$ person
\item Score of $11$: $1$ person
\item Score of $12$: $1$ person
\item Score of $13$: $1$ person
\item Score of $14$: $4$ people
\end{itemize}

The mean score was $10.9$. The problem wise answers as as follows:

\begin{enumerate}
\item Option (B): $9$ people
\item Option (B): $9$ people
\item Option (B): $9$ people
\item Option (D): $7$ people
\item Option (D): $10$ people
\item Option (B): $9$ people
\item Option (B): $8$ people
\item Option (A): $7$ people
\item Option (C): $8$ people
\item Option (A): $8$ people
\item Option (C): $9$ people
\item Option (E): $10$ people
\item Option (C): $10$ people
\item Option (E): $7$ people
\end{enumerate}
\section{Solutions}

\begin{enumerate}
\item Which of the following statements is {\bf always true}? 

  \begin{enumerate}[(A)]
  \item The range of a continuous nonconstant function on an open
    bounded interval (i.e., an interval of the form $(a,b)$) is an
    open bounded interval (i.e., an interval of the form $(m,M)$).
  \item The range of a continuous nonconstant function on a closed
    bounded interval (i.e., an interval of the form $[a,b]$) is a
    closed bounded interval (i.e., an interval of the form $[m,M]$).
  \item The range of a continuous nonconstant function on an open
    interval that may be bounded or unbounded (i.e., an interval of
    the form $(a,b)$,$(a,\infty)$, $(-\infty,a)$, or
    $(-\infty,\infty)$), is also an open interval that may be bounded
    or unbounded.
  \item The range of a continuous nonconstant function on a closed
    interval that may be bounded or unbounded (i.e., an interval of
    the form $[a,b]$, $[a,\infty)$, $(-\infty,a]$, or
    $(-\infty,\infty)$) is also a closed interval that may be bounded
    or unbounded.
  \item None of the above.
  \end{enumerate}

  {\em Answer}: Option (B)

  {\em Explanation}: This is a combination of the extreme-value
  theorem and the intermediate-value theorem. By the extreme-value
  theorem, the continuous function attains a minimum value $m$ and a
  maximum value $M$. By the intermediate-value theorem, it attains
  every value between $m$ and $M$. Further, it can attain no other
  values because $m$ is after all the minimum and $M$ the maximum.

  {\em The other choices}:

  Option (A): Think of a function that increases first and then
  decreases. For instance, the function $f(x) := \sqrt{1 - x^2}$ on
  $(-1,1)$ has range $(0,1]$, which is not open. Or, the function
  $\sin x$ on the interval $(0,2\pi)$ has range $[-1,1]$.

  Option (C): The same counterexample as for option (A) works.

  Option (D): We can get counterexamples for unbounded intervals. For
  instance, consider the function $f(x) := 1/x$ on $[1,\infty)$. The
  range of this function is $(0,1]$, which is not closed. The idea is
  that we make the function approach but not reach a finite value as
  $x \to \infty$ (we'll talk more about this when we deal with
  asymptotes).

  {\em Performance review}: $9$ out of $11$ got this. $1$ chose (A)
  and $1$ chose (C).

  {\em Historical note (last year)}: $16$ out of $28$ people got this
  correct. $7$ people chose (A), $2$ people each chose (C) and (E),
  and $1$ person chose (D).
\item For which of the following specifications is there {\bf no
  continuous function} satisfying the specifications? 

  \begin{enumerate}[(A)]
  \item Domain $(0,1)$ and range $(0,1)$
  \item Domain $[0,1]$ and range $(0,1)$
  \item Domain $(0,1)$ and range $[0,1]$
  \item Domain $[0,1]$ and range $[0,1]$
  \item None of the above, i.e., we can get a continuous function for
    each of the specifications.
  \end{enumerate}

  {\em Answer}: Option (B)

  {\em Explanation}: By the extreme value theorem, any continuous
  function on a closed bounded interval must attain its maximum and
  minimum, and hence its image cannot be an open interval.

  {\em The other choices}:

  For options (A) and (D), we can pick the identity functions $f(x) :=
  x$ on the respective domains.

  For option (C), we can pick the function $f(x) := \sin^2(2\pi x)$ on
  the domain $(0,1)$. 

  {\em Performance review}: $9$ out of $11$ got this. $2$ chose (C).

  {\em Historical note (last year)}: $21$ out of $28$ people got this
  correct. $5$ people chose (C) and $2$ people chose (E).

\item Suppose $f$ is a continuously differentiable function from the
  open interval $(0,1)$ to $\R$. Suppose, further, that there are
  exactly $14$ values of $c$ in $(0,1)$ for which $f(c) = 0$. What can
  we say is {\bf definitely true} about the number of values of $c$ in
  the open interval $(0,1)$ for which $f'(c) = 0$? 

  \begin{enumerate}[(A)]
  \item It is at least $13$ and at most $15$.
  \item It is at least $13$, but we cannot put any upper bound on it
    based on the given information.
  \item It is at most $15$, but we cannot put any lower bound (other
  than the meaningless bound of $0$) based on the given information.
  \item It is at most $13$.
  \item It is at least $15$.
  \end{enumerate}

  {\em Answer}: Option (B)

  {\em Explanation}: Suppose the zeros of $f$ are $a_1 < a_2 < \dots <
  a_{14}$. By Rolle's theorem, there is {\em at least one} zero of
  $f'$ between each $a_i$ and $a_{i+1}$. There may be more, since
  Rolle's theorem gives only a {\em lower} bound. This gives thirteen
  solutions $c$ to $f'(c) = 0$.

  Note that in order to apply Rolle's theorem, it is enough to be
  given that $f$ is differentiable, so the additional hypothesis that
  $f'$ is continuous is not necessary.

  {\em Performance review}: $9$ out of $11$ got this. $2$ chose (A).

  {\em Historical note (last year)}: $19$ out of $28$ people got this
  correct. $5$ people chose (A), $2$ people chose (C), and $1$ person
  each chose (D) and (E).
\item Consider the function $f(x) := \lbrace\begin{array}{rl} x, & 0
  \le x \le 1/2 \\ x - (1/7), & 1/2 < x \le 1 \\\end{array}$. Define by
  $f^{[n]}$ the function obtained by iterating $f$ $n$ times, i.e.,
  the function $f \circ f \circ f \circ \dots \circ f$ where $f$
  occurs $n$ times. What is the smallest $n$ for which $f^{[n]} =
  f^{[n + 1]}$? {\em Earlier score: $3/16$}

  \begin{enumerate}[(A)]
  \item $1$
  \item $2$
  \item $3$
  \item $4$
  \item $5$
  \end{enumerate}

  {\em Answer}: Option (D)

  {\em Explanation}: We need to iterate $f$ enough times that
  everything gets inside $[0,1/2]$, after which it becomes
  stable. Note that each time, the value goes down by $1/7$. Thus, for
  any $x \le 1$, we need at most four steps to bring it in $[0,1/2]$,
  with the upper bound of $4$ being attained for $1$.

  {\em Performance review}: $7$ out of $11$ got this. $3$ chose (C),
  $1$ chose (A).

  {\em Historical note (last year)}: $10$ out of $28$ people got this
  correct. $9$ people chose (C), $5$ people chose (B), $2$ people
  chose (A), and $2$ people left the question blank.
\item Suppose $f$ and $g$ are functions $(0,1)$ to $(0,1)$ that are
  both right continuous on $(0,1)$. Which of the following is {\em
  not} guaranteed to be right continuous on $(0,1)$? {\em Earlier
  scores: $3/11$, $9/14$}

  \begin{enumerate}[(A)]
  \item $f + g$, i.e., the function $x \mapsto f(x) + g(x)$
  \item $f - g$, i.e., the function $x \mapsto f(x) - g(x)$
  \item $f \cdot g$, i.e., the function $x \mapsto f(x)g(x)$
  \item $f \circ g$, i.e., the function $x \mapsto f(g(x))$
  \item None of the above, i.e., they are all guaranteed to be right
    continuous functions
  \end{enumerate}

  {\em Answer}: Option (D)

  {\em Explanation}: See the explanation for Question 2 on the October
  1 quiz. Note that that quiz uses left continuity, but the example
  can be adapted to right continuity.

  {\em Performance review}: $10$ out of $11$ got this. $1$ chose (B).

  {\em Historical note (last year)}: $20$ out of $28$ people got this
  correct. $4$ people chose (C), $2$ people chose (B), $1$ person
  chose (E), and $1$ person left the question blank.
\item Suppose $f$ and $g$ are increasing functions from $\R$ to
  $\R$. Which of the following functions is {\em not} guaranteed to be
  an increasing functions from $\R$ to $\R$? {\em Earlier scores:
  $1/15$, $9/16$}

  \begin{enumerate}[(A)]

  \item $f + g$
  \item $f \cdot g$
  \item $f \circ g$
  \item All of the above, i.e., none of them is guaranteed to be increasing.
  \item None of the above, i.e., they are all guaranteed to be increasing.
  \end{enumerate}

      {\em Answer}: Option (B)

  {\em Explanation}: The problem with option (B) arises when one or
  both functions take negative values. For instance, consider the case
  $f(x) := x$ and $g(x) := x$. Both are increasing functions on all of
  $\R$. However, the pointwise product is the function $x \mapsto
  x^2$, which is a decreasing function for negative $x$.

  Formally, the issue is that we cannot multiply inequalities of the
  form $A < B$ and $C < D$ unless we are guaranteed to be working with
  positive numbers.

  {\em The other choices}:

  Option (A): For any $x_1 <
  x_2$, we have $f(x_1) < f(x_2)$ and $g(x_1) < g(x_2)$. Adding up, we
  get $f(x_1) + g(x_1) < f(x_2) + g(x_2)$, so $(f + g)(x_1) < (f + g)(x_2)$.

  Option (C): For any $x_1 < x_2$, we have $g(x_1) < g(x_2)$ since $g$
  is increasing. Now, we use the factthat $f$ is increasing to compare
  its values at the two points $g(x_1)$ and $g(x_2)$, and we get
  $f(g(x_1)) < f(g(x_2))$. We thus get $(f \circ g)(x_1) < (f \circ
  g)(x_2)$.

  {\em Performance review}: $9$ out of $11$ got this. $2$ chose (C).

  {\em Historical note (last year)}: $18$ out of $28$ people got this
  correct. $6$ people chose (E) and $4$ people chose (C).

\item Suppose $F$ and $G$ are two functions defined on $\R$ and $k$ is
  a natural number such that the $k^{th}$ derivatives of $F$ and $G$
  exist and are equal on all of $\R$. Then, $F - G$ must be a
  polynomial function. What is the {\bf maximum possible degree} of $F
  - G$?  (Note: Assume constant polynomials to have degree zero) {\em
  Earlier score: $6/16$}

  \begin{enumerate}[(A)]
  \item $k - 2$
  \item $k - 1$
  \item $k$
  \item $k + 1$
  \item There is no bound in terms of $k$.
  \end{enumerate}

  {\em Answer}: Option (B)

  {\em Explanation}: $F$ and $G$ having the same $k^{th}$ derivative
  is equivalent to requiring that $F - G$ have $k^{th}$ derivative
  equal to zero. For $k = 1$, this gives constant functions
  (polynomials of degree $0$). Each time we increment $k$, the degree
  of the polynomial could potentially go up by $1$. Thus, the answer
  is $k - 1$.

  {\em Performance review}: $8$ out of $11$ got this. $2$ chose (E),
  $1$ chose (D).

  {\em Historical note (last year)}: $10$ out of $28$ people got this
  correct. $5$ people each chose (D) and (E) and $4$ people each chose
  (A) and (C).

  {\em Action point}: This is a question you really {\em should} get
  correct!
\item Suppose $f$ is a continuous function on $\R$. Clearly, $f$ has
  antiderivatives on $\R$. For all but one of the following
  conditions, it is possible to guarantee, without any further
  information about $f$, that there exists an antiderivative $F$
  satisfying that condition. {\bf Identify the exceptional condition}
  (i.e., the condition that it may not always be possible to
  satisfy). {\em Earlier score: $3/16$}

  \begin{enumerate}[(A)]
  \item $F(1) = F(0)$.
  \item $F(1) + F(0) = 0$.
  \item $F(1) + F(0) = 1$.
  \item $F(1) = 2F(0)$.
  \item $F(1)F(0) = 0$.
  \end{enumerate}

  {\em Answer}: Option (A)

  {\em Explanation}: Suppose $G$ is an antiderivative for $f$. The
  general expression for an antiderivative is $G + C$, where $C$ is
  constant. We see that for options (b), (c), and (d), it is always
  possible to solve the equation we obtain to get one or more real
  values of $C$. However, (a) simplifies to $G(1) + C = G(0) + C$,
  whereby $C$ is canceled, and we are left with the statement $G(1) =
  G(0)$. If this statement is true, then {\em all} choices of $C$
  work, and if it is false, then {\em none} works. Since we cannot
  guarantee the truth of the statement, (a) is the exceptional
  condition.

  Another way of thinking about this is that $F(1) - F(0) = \int_0^1
  f(x) \, dx$, regardless of the choice of $F$. If this integral is
  $0$, then any antiderivative works. If it is not zero, no
  antiderivative works.

  {\em Performance review}: $7$ out of $11$ got this. $3$ chose (D),
  $1$ chose (E).

  {\em Historical note (last year)}: $10$ out of $28$ people got this
  correct. $6$ people chose (B), $5$ people chose (E), $4$ people
  chose (D), $2$ people chose (C), and $1$ person left the question
  blank.
\item Suppose $F$ is a function defined on $\R \setminus \{ 0 \}$ such
  that $F'(x) = -1/x^2$ for all $x \in \R \setminus \{ 0 \}$. Which of
  the following pieces of information is/are {\bf sufficient} to determine
  $F$ completely? {\em Earlier score: $4/16$}
  \begin{enumerate}[(A)]
  \item The value of $F$ at any two positive numbers.
  \item The value of $F$ at any two negative numbers.
  \item The value of $F$ at a positive number and a negative number.
  \item Any of the above pieces of information is sufficient, i.e., we
    need to know the value of $F$ at any two numbers.
  \item None of the above pieces of information is sufficient.
  \end{enumerate}

  {\em Answer}: Option (C)

  {\em Explanation}: There are two open intervals: $(-\infty,0)$ and
  $(0,\infty)$, on which we can look at $F$. On each of these
  intervals, $F(x) = 1/x + $ a constant, but the constant for
  $(-\infty,0)$ may differ from the constant for $(0,\infty)$. Thus,
  we need the initial value information at one positive number and one
  negative number.

  {\em Performance review}: $8$ out of $11$ got this. $3$ chose (D).

  {\em Historical note (last year)}: $15$ out of $28$ people got this
  correct. $9$ people chose (D), $2$ people chose (E), and $1$ person
  each chose (A) and (B).

\item Suppose $F$ and $G$ are continuously differentiable functions on
  all of $\R$ (i.e., both $F'$ and $G'$ are continuous). Which of the
  following is {\bf not necessarily true}? {\em Earlier scores: $0$,
  $10/16$}

  \begin{enumerate}[(A)]
  \item If $F'(x) = G'(x)$ for all integers $x$, then $F - G$ is a
    constant function when restricted to integers, i.e., it takes the
    same value at all integers.
  \item If $F'(x) = G'(x)$ for all numbers $x$ that are not integers,
    then $F - G$ is a constant function when restricted to the set of
    numbers $x$ that are not integers.
  \item If $F'(x) = G'(x)$ for all rational numbers $x$, then $F - G$
    is a constant function when restricted to the set of rational
    numbers.
  \item If $F'(x) = G'(x)$ for all irrational numbers $x$, then $F -
    G$ is a constant function when restricted to the set of irrational
    numbers.
  \item None of the above, i.e., they are all necessarily true.
  \end{enumerate}

  {\em Answer}: Option (A).

  {\em Explanation}: The fact that the derivatives of two functions
  agree at integers says nothing about how the derivatives behave
  elsewhere -- they could differ quite a bit at other places. Hence,
  (A) is not necessarily true, and hence must be the right option. All
  the other options are correct as statements and hence cannot be the
  right option. This is because in all of them, the set of points
  where the derivatives agree is {\em dense} -- it intersects every
  open interval. So, continuity forces the functions $F'$ and $G'$ to
  be equal everywhere, forcing $F - G$ to be constant everywhere.

  {\em Performance review}: $8$ out of $11$ got this. $2$ chose (E),
  $1$ chose (D).

  {\em Historical note (last year)}: $11$ out of $28$ people got this
  correct. $8$ people chose (E), $5$ people chose (C), and $2$ people
  each chose (B) and (D).
\item Consider the four functions $\sin(\sin x)$, $\sin(\cos x)$,
  $\cos(\sin x)$, and $\cos(\cos x)$. Which of the following
  statements are true about their periodicity? {\em Earlier score:
  $5/16$}

  \begin{enumerate}[(A)]
  \item All four functions are periodic with a period of $\pi$.
  \item All four functions are periodic with a period of $2\pi$.
  \item $\cos(\sin x)$ and $\cos(\cos x)$ have a period of $\pi$,
    whereas $\sin(\sin x)$ and $\sin(\cos x)$ have a period of $2\pi$.
  \item $\sin(\sin x)$ and $\sin(\cos x)$ have a period of $\pi$,
    whereas $\cos(\sin x)$ and $\cos(\cos x)$ have a period of $2\pi$.
  \item $\sin(\sin x)$ has a period of $2\pi$, the other three
    functions have a period of $\pi$.
  \end{enumerate}
  
  {\em Answer}: Option (C)

  {\em Explanation}: Since the inner functions in all cases have a
  period of $2\pi$, it is clear that all the four functions have a
  period of at most $2\pi$, in fact, the period of each divides
  $2\pi$. The crucial question is which of them have the smaller
  period $\pi$.

  Let's look at $\sin \circ \sin$ first. We have:

  $$\sin(\sin(x + \pi)) = \sin(-\sin x) = - \sin(\sin x)$$

  So, we see that that value at $x + \pi$ is the negative, and hence
  usually not the equal, of the value at $x$. Similarly:

  $$\sin(\cos(x + \pi)) = \sin(-\cos x) = -\sin(\cos x)$$

  On the other hand, for the functions that have a $\cos$ on the
  outside, the negative sign on the inside gets eaten up by the even
  nature of the outer function. For instance:

  $$\cos(\sin(x + \pi)) = \cos(-\sin x) = \cos(\sin x)$$

  and:

  $$\cos(\cos(x + \pi)) = \cos(-\cos x) = \cos(\cos x)$$

  Now, this is not a proof that $\pi$ is strictly the smallest period
  for these functions, but that can be proved using other methods. In
  any case, given the choices presented, it is now easy to single out
  (D) as the only correct answer.

  The key feature here is that both $\sin$ and $\cos$ (viewed as the
  inner functions of the composition) have {\em anti-period} $\pi$:
  their value gets negated after an interval of $\pi$.

  The outer function $\cos$ is even, hence it converts an anti-period
  for the inner function into a period for the overall function. The
  outer function $\sin$ is odd, so it keeps anti-periods anti-periods.

  {\em Performance review}: $9$ out of $11$ got this. $1$ chose (B),
  $1$ chose (D).

  {\em Historical note (last year)}: $14$ out of $28$ people got this
  correct. $6$ people chose (D), $3$ people chose (B), $2$ people each
  chose (A) and (E), and $1$ person left the question blank.
\item Suppose $f$ is a one-to-one function with domain a closed
  interval $[a,b]$ and range a closed interval $[c,d]$. Suppose $t$ is
  a point in $(a,b)$ such that $f$ has left hand derivative $l$ and
  right-hand derivative $r$ at $t$, with both $l$ and $r$
  nonzero. What is the left hand derivative and right hand derivative
  to $f^{-1}$ at $f(t)$? {\em Earlier score: $6/15$}

  \begin{enumerate}[(A)]
  \item The left hand derivative is $1/l$ and the right hand
    derivative is $1/r$.
  \item The left hand derivative is $-1/l$ and the right hand
    derivative is $-1/r$.
  \item The left hand derivative is $1/r$ and the right hand
    derivative is $1/l$.
  \item The left hand derivative is $-1/r$ and the right hand
    derivative is $-1/l$.
  \item The left hand derivative is $1/l$ and the right hand
    derivative is $1/r$ if $l > 0$, otherwise the left hand derivative
    is $1/r$ and the right hand derivative is $1/l$.
  \end{enumerate}

  {\em Answer}: Option (E)

  {\em Explanation}: Although it isn't necessary to note this, a
  one-to-one function that satisfies the intermediate value property
  is continuous, so even though $f$ is not explicitly given to be
  continuous, it is in fact continuous on its domain.

  If $l > 0$, then, since we are dealing with a one-to-one function,
  the function is increasing throughout, and so $r \ge 0$ as
  well. Since we know $r \ne 0$, we conclude that $r > 0$
  strictly. The upshot is that as $x \to t^-$, $f(x) \to f(t)^-$ and
  as $x \to t^+$, $f(x) \to f(t)^+$. Thus, when we pass to the inverse
  function, the roles of left and right remain the same.

  On the other hand, if $l < 0$, then as $x \to t^-$, $f(x) \to
  f(t)^+$, and hence the roles of left and right get interchanged.

  {\em Performance review}: $10$ out of $11$ got this. $1$ chose (C).

  {\em Historical note (last year)}: $16$ out of $28$ people got this
  correct. $5$ people chose (A), $3$ people chose (B), and $2$ people
  each chose (C) and (D).
\item Which of these functions is one-to-one? {\em Earlier score:
  $2/15$}

  \begin{enumerate}[(A)]
  \item $f_1(x) := \lbrace \begin{array}{rl} x, & x \text{ rational} \\ x^2, & x \text{ irrational}\\\end{array}$ 
  \item $f_2(x) := \lbrace \begin{array}{rl} x, & x \text{ rational} \\ x^3, & x \text{ irrational}\\\end{array}$
  \item $f_3(x) := \lbrace\begin{array}{rl} x, & x \text{ rational} \\ 1/(x - 1), & x \text{ irrational}\\\end{array}$
  \item All of the above
  \item None of the above
  \end{enumerate}

  {\em Answer}: Option (C)

  {\em Explanation}: Option (A) is easy to rule out: $\sqrt{2}$ and
  $-\sqrt{2}$ map to the same thing. Option (B) is a little harder to
  rule out, because the function is one-to-one within each piece,
  i.e., no two rationals map to the same thing and no two irrationals
  map to the same thing. However, a rational and an irrational can map
  to the same thing. For instance, $2$ and $2^{1/3}$ both map to $2$.

  For option (C), note that not only is the map one-to-one in each
  piece, but also, the image of the rationals stays inside the
  rationals and the image of the irrationals stays inside the
  irrationals. In particular, this means that a rational number and an
  irrational number cannot map to the same thing, so the function is
  globally one-to-one.

  {\em Performance review}: $10$ out of $11$ got this. $1$ chose (B).

  {\em Historical note (last year)}: $13$ out of $28$ people got this
  correct. $6$ people each chose (B) and (D), $2$ people chose (E),
  and $1$ person chose (A).

\item Consider the following function $f:[0,1] \to [0,1]$ given by
  $f(x) := \lbrace\begin{array}{rl} \sin(\pi x/2), & 0 \le x \le 1/2 \\
  \sqrt{x}, & 1/2 < x \le 1\\\end{array}$. What is the correct
  expression for $(f^{-1})'(1/2)$? {\em Earlier score: $1/15$}

  \begin{enumerate}[(A)]
  \item It does not exist, since the two-sided derivatives of $f$ at
    $1/2$ do not match.
  \item $\sqrt{2}$
  \item $2\sqrt{2}/\pi$
  \item $4/\pi$
  \item $4/(\sqrt{3}\pi)$
  \end{enumerate}

  {\em Answer}: Option (E)

  {\em Explanation}: We use:

  $$(f^{-1})'(1/2) = \frac{1}{f'(f^{-1}(1/2))}$$

  By inspection, we see that $f^{-1}(1/2)$ must be between $0$ and
  $1/2$. Thus, we must solve $\sin(\pi x/2) = 1/2$. This gives $\pi x
  / 2 = \pi/6$ (considering domain restrictions) so $x = 1/3$. Thus, we get:

  $$(f^{-1})'(1/2) = \frac{1}{f'(1/3)}$$

  The expression for the derivative is $(\pi/2)\cos(\pi x/2)$, which
  evaluated at $1/3$ gives $(\pi\sqrt{3})/4$. Taking the reciprocal,
  we get $4/(\pi\sqrt{3})$.

  Note that (A) is a sophisticated distractor in the sense that if you
  naively consider:

  $$(f^{-1})'(1/2) = \frac{1}{f'(1/2)}$$

  You will wrongly conclude (A). (B) and (C) are the one-sided
  derivative at $f(1/2)$, so these too are attractive propositions for
  the naive.

  {\em Performance review}: $7$ out of $11$ got this. $2$ chose (C),
  $1$ each chose (A) and (B).

  {\em Historical note (last year)}: $12$ people got this correct. $9$
  people chose (A), $4$ people chose (B), and $3$ people chose (C).
\end{enumerate}

\end{document}
