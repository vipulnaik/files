\documentclass[10pt]{amsart}

%Packages in use
\usepackage{fullpage, hyperref, vipul, enumerate}

%Title details
\title{Take-home class quiz solutions: due Monday February 18: Partial derivatives}
\author{Math 195, Section 59 (Vipul Naik)}
%List of new commands

\begin{document}
\maketitle

\section{Performance review}

$26$ people took this quiz. The score distribution was as follows:

\begin{itemize}
\item Score of $3$: $1$ person.
\item Score of $4$: $1$ person.
\item Score of $5$: $2$ people.
\item Score of $6$: $1$ person.
\item Score of $7$: $2$ people.
\item Score of $8$: $3$ people.
\item Score of $9$: $5$ people.
\item Score of $11$: $5$ people.
\item Score of $12$: $3$ people.
\item Score of $13$: $3$ people.
\end{itemize}

The question wise answers and performance review were as follows:

\begin{enumerate}
\item Option (D): $17$ people.
\item Option (A): $3$ people.
\item Option (A): $15$ people.
\item Option (E): $16$ people.
\item Option (A): $18$ people.
\item Option (E): $4$ people.
\item Option (C): $2$ people.
\item Option (B): $0$ people.
\item Option (C): $8$ people.
\item Option (C): $20$ people.
\item Option (E): $19$ people.
\item Option (D): $21$ people.
\item Option (B): $24$ people.
\item Option (D): $25$ people.
\item Option (B): $14$ people.
\item Option (D): $15$ people.
\item Option (E): $15$ people.
\end{enumerate}
\section{Solutions}

\begin{enumerate}

\item For this and the next question, consider the function on $\R^2$
  given as:

  $$f(x,y) := \left\lbrace \begin{array}{rl} 1, & x \text{ rational  or } y \text { rational }\\0, & x \text{ and } y \text{ both irrational } \\\end{array}\right.$$

  What can we say about the subset $S$ of $\R^2$ defined as the set of
  points where $f_x$ is defined?

  \begin{enumerate}[(A)]
  \item $S$ is the set of points for which at least one coordinate is rational.
  \item $S$ is the set of points for which both coordinates are rational.
  \item $S$ is the set of points for which the $x$-coordinate is rational.
  \item $S$ is the set of points for which the $y$-coordinate is rational.
  \item $S$ is the set of points for which at least one coordinate is
    irrational.
  \end{enumerate}

  {\em Answer}: Option (D)

  {\em Explanation}: $f_x$ means we take the derivative with respect
  to $x$ holding $y$ constant. Consider the point $(x_0,y_0)$. If
  $y_0$ is rational, then $f(x,y_0) = 1$ on the entire line $y =
  y_0$. Thus, $f_x(x_0,y_0)$ is the derivative of a constant function,
  hence is $0$. In particular, it is well defined.

  On the other hand, if $y_0$ is irrational, then there are points
  $(x,y_0)$ for $x$ arbitrarily close to $x_0$ for which $x$ is
  rational, giving $f(x,y_0) = 1$, and also points where $x$ is
  irrational, giving $f(x,y_0) = 0$. Thus, $f$ is not continuous in
  $x$ at $(x_0,y_0)$, and hence $f_x(x_0,y_0)$ does not exist.

  {\em Performance review}: $17$ out of $26$ people got this. $4$
  chose (A), $2$ each chose (C) and (E), $1$ chose (B).

\item With $f$ as in the previous question, what is the subset $T$ of
  $\R^2$ at which the second-order mixed partial derivative $f_{xy}$
  is defined?
  
  \begin{enumerate}[(A)]
  \item $T$ is the empty subset.
  \item $T$ is the set of points for which both coordinates are rational.
  \item $T$ is the set of points for which the $x$-coordinate is rational.
  \item $T$ is the set of points for which the $y$-coordinate is rational.
  \item $T$ is the set of points for which both coordinates are
    irrational.
  \end{enumerate}

  {\em Answer}: Option (A)

  {\em Explanation}: We have $f_{xy} = (f_x)_y$. In order for this to
  be defined at $(x_0,y_0)$, a necessary condition is that $f_x$ be
  defined at $(x_0,y_0)$, and also that it be defined at $(x_0,y)$ for
  $y$ close to $y_0$. Note that the former condition holds only if
  $y_0$ is rational. However, the latter condition is never true,
  because for any value of $y_0$, there are values $y$ arbitrarily
  close to $y_0$ that are rational, and also values $y$ that are
  irrational.

  {\em Performance review}: $3$ out of $26$ people got this. $19$ chose
  (B), $3$ chose (D), $1$ chose (C).

\item For this and the next three questions, consider the function on $\R^2$
  given as:

  $$g(x,y) := \left\lbrace \begin{array}{rl} 1, & x \text{ rational}\\0, & x \text{ irrational } \\\end{array}\right.$$

  What can we say about the subset $U$ of $\R^2$ defined as the set of
  points where $g_x$ is defined?

  \begin{enumerate}[(A)]
  \item $U$ is the empty subset.
  \item $U$ is the set of points for which both coordinates are rational.
  \item $U$ is the set of points for which the $x$-coordinate is rational.
  \item $U$ is the set of points for which the $y$-coordinate is rational.
  \item $U$ is the whole plane $\R^2$.
  \end{enumerate}

  {\em Answer}: Option (A)

  {\em Explanation}: At any point $(x_0,y_0)$, there are $x$-values
  arbitrarily close to $x_0$ that are rational, and $x$-values
  arbitrarily close to $x_0$ that are irrational. Thus, $g$ is not
  continuous in $x$ at any point, so $g_x$ does not exist anywhere.

  {\em Performance review}: $15$ out of $26$ got this. $10$ chose (C),
  $1$ chose (E).

\item With $g$ as in the preceding question, what can we say about the
  subset $V$ of $\R^2$ defined as the set of points where $g_y$ is
  defined?

  \begin{enumerate}[(A)]
  \item $V$ is the empty subset.
  \item $V$ is the set of points for which both coordinates are rational.
  \item $V$ is the set of points for which the $x$-coordinate is rational.
  \item $V$ is the set of points for which the $y$-coordinate is rational.
  \item $V$ is the whole plane $\R^2$.
  \end{enumerate}

  {\em Answer}: Option (E)

  {\em Explanation}: Note that $g$ depends {\em only} on $x$, hence it
  is independent of $y$. Thus, $g_y$ is identically the zero function,
  and is defined everywhere.

  {\em Performance review}: $16$ out of $26$ got this. $5$ chose (D),
  $2$ each chose (A) and (C), $1$ chose (B).

\item With $g$ as in the preceding question, what can we say about the
  subset $W$ of $\R^2$ defined as the set of points where $g_{xy}$ is
  defined?

  \begin{enumerate}[(A)]
  \item $W$ is the empty subset.
  \item $W$ is the set of points for which both coordinates are rational.
  \item $W$ is the set of points for which the $x$-coordinate is rational.
  \item $W$ is the set of points for which the $y$-coordinate is rational.
  \item $W$ is the whole plane $\R^2$.
  \end{enumerate}

  {\em Answer}: Option (A)

  {\em Explanation}: This follows from $g_x$ not being defined anywhere.

  {\em Performance review}: $18$ out of $26$ got this. $4$ chose (B),
  $2$ chose (E), $1$ each chose (C) and (D).

\item With $g$ as in the preceding question, what can we say about the
  subset $X$ of $\R^2$ defined as the set of points where $g_{yx}$ is
  defined?

  \begin{enumerate}[(A)]
  \item $X$ is the empty subset.
  \item $X$ is the set of points for which both coordinates are rational.
  \item $X$ is the set of points for which the $x$-coordinate is rational.
  \item $X$ is the set of points for which the $y$-coordinate is rational.
  \item $X$ is the whole plane $\R^2$.
  \end{enumerate}

  {\em Answer}: Option (E)

  {\em Explanation}: This follows from $g_y$ being the zero function
  everywhere.

  {\em Performance review}: $18$ out of $26$ got this. $4$ chose (E),
  $3$ chose (B), $1$ chose (C).
\item For this and the next two questions, consider the function on
  $\R^2$ given as:

  $$h(x,y) := \left\lbrace \begin{array}{rl} 1, & x \text{ an integer or } y \text{ an integer }\\0, & x \text{ not an integer and } y \text{ not an integer } \\\end{array}\right.$$

  What can we say about the subset $A$ of $\R^2$ defined as the set of
  points where $h_{xy}$ is defined?

  \begin{enumerate}[(A)]
  \item $A$ is the empty set.
  \item $A$ is the set of points whose $x$-coordinate is an integer.
  \item $A$ is the set of points whose $x$-coordinate is not an integer.
  \item $A$ is the set of points whose $y$-coordinate is an integer.
  \item $A$ is the set of points whose $y$-coordinate is not an integer.
  \end{enumerate}

  {\em Answer}: Option (C)

  {\em Explanation}: We first note that if both coordinates are
  non-integers, then $h$ is identiically the zero function {\em at and
  around} the point, so all first and higher order partials are
  zero. In particular, $h_x = h_{xy} = 0$ at all points with both
  coordinate non-integers.

  Suppose now that the $y$-coordinate is an integer. In this case, $h$
  is identically $1$ on lines of the form $y = y_0$, $y_0$ an
  integer. Thus, $h_x$ is zero on these lines.

  Suppose now that the $x$-coordinate is an integer but the
  $y$-coordinate is a non-integer. In this case, the function $h$
  takes the value $1$ at the point, but takes the value $0$ if we vary
  the $x$-coordinate even slightly. Thus, $h_x$ is not defined at such
  points.
  
  Thus, overall, $h_x$ is defined and equal to zero at all points
  where either the $y$-coordinate is an integer or both coordinates
  are non-integers. It is not defined precisely at the points where
  the $x$-coordinate is an integer and the $y$-coordinate is a
  non-integer.

  Of the points where $h_x$ is defined, $h_{xy}$ is not defined at
  points where both coordinates are integers, because slightly
  perturbing the $y$-coordinate gets a point where $h_x$ is
  undefined. $h_{xy}$ is defined and equal to zero at all other
  points, namely the points where the $x$-coordinate is a non-integer.

  {\em Performance review}: $2$ out of $26$ got this. $11$ chose (D),
  $8$ chose (A), $4$ chose (B), $1$ left the question blank.
\item With $h$ as defined in the previous question, what can we say
  about the subset $B$ of $\R^2$ defined as the set of points where
  $h_x$ is defined but $h_{xy}$ is not defined?

  \begin{enumerate}[(A)]
  \item $B$ is the empty set.
  \item $B$ is the set of points for which both coordinates are integers.
  \item $B$ is the set of points for which both coordinates are non-integers.
  \item $B$ is the set of points for which at least one coordinate is an integer.
  \item $B$ is the set of points for which at least one coordinate is
    a non-integer.
  \end{enumerate}

  {\em Answer}: Option (B)

  {\em Explanation}: See the explanation for the preceding question.

  {\em Performance review}: Nobody got this correct. $8$ chose (D),
  $7$ chose (E), $5$ each chose (A) and (C). $1$ left the question
  blank.
\item With $h$ as defined in the previous question, what can we say
  about the subset $C$ of $\R^2$ defined as the set of points where
  both $h_{xy}$ and $h_{yx}$ are defined?

  \begin{enumerate}[(A)]
  \item $C$ is the empty set.
  \item $C$ is the set of points for which both coordinates are integers.
  \item $C$ is the set of points for which both coordinates are non-integers.
  \item $C$ is the set of points for which at least one coordinate is an integer.
  \item $C$ is the set of points for which at least one coordinate is
    a non-integer.
  \end{enumerate}

  {\em Answer}: Option (C)

  {\em Explanation}: By the question before last, the set of points
  where $h_{xy}$ is defined is the set of points where the
  $x$-coordinate is a non-integer. The function is symmetric in $x$
  and $y$, so analogous reasoning yields that the set of points where
  $h{yx}$ is defined is the set of points where the $y$-coordinate is
  a non-integer. Intersecting the two sets, we get the set of points
  where both coordinates are non-integers.

  {\em Performance review}: $8$ out of $26$ got this correct. $14$
  chose (B), $2$ chose (A), $1$ chose (D), $1$ left the question blank.

\item Students training for an examination can spend money either on
  purchasing textbooks or on private tuitions. A student's expected
  performance on the examination is a function of the money the
  student spends on textbooks and on tuition (viewed as separate
  variables). Two researchers want to consider the question of whether
  increased expenditure on textbooks leads to improved performance on
  the examination, and if so, by how much.

  One researcher decides to measure the increase in the examination
  score for a marginal increase in textbook expenditure {\em holding
  constant the expenditure on tuitions}, arguing that in order to
  determine the effect of changes in textbook expenditures, the other
  expenditures need to be kept constant.

  The other researcher believes that since the student has a limited
  budget, it would be more realistic to measure the increase in the
  examination score for a marginal increase in textbook expenditure
  {\em holding constant the total expenditure on both textbook and
  tuitions}. This is because the student is likely to allocate money
  away from tuition expenditures in order to spend money on textbooks.

  Which of the following best describes what's happening?

  \begin{enumerate}[(A)]
  \item Both researchers are effectively computing the same quantity.
  \item The two quantities that the researchers are computing have a
    simple linear relationship, i.e., their sum or difference is a
    constant.
  \item The two quantities that the researchers are computing are
    meaningfully different and there is a relationship between them
    but that relationship involves other partial derivatives.
  \end{enumerate}

  {\em Answer}: Option (C)

  {\em Explanation}: Too tricky to review here, but you might want to watch this video and subsequent ones in the playlist:

  \url{http://www.youtube.com/watch?v=tfH2iqt2E0E&list=PLC0bHnWu122kC1WBgr0H9PEbHTYnYev27&index=4}

  {\em Performance review}: $20$ out of $26$ got this correct. $3$
  each chose (A) and (B).
\item $F$ is an everywhere twice differentiable function of two
  variables $x$ and $y$. Which of the following captures the manner in
  which the inputs $x$ and $y$ {\em interact} with each other in the
  description of $F$?

  \begin{enumerate}[(A)]
  \item The difference $F_x - F_y$
  \item The quotient $F_x/F_y$.
  \item The product $F_xF_y$.
  \item The product $F_{xx}F_{yy}$.
  \item The mixed partial $F_{xy}$
  \end{enumerate}
 
  {\em Answer}: Option (E)

  {\em Explanation}: One extreme way of seeing this is that if $F$ is
  additively separable (i.e., it is the sum of a function of $x$ and a
  function of $y$) then $F_{xy}(x,y) = 0$. Thus, the stereotypical
  case in which the variables don't interact with each other is the
  case that the second-order mixed partial is zero.

  {\em Performance review}: $19$ out of $26$ got this correct. $4$
  chose (D), $2$ chose (B), $1$ chose (C).

\item $F$ is a function of two variables $x$ and $y$ such that both
  $F_x$ and $F_y$ exist. Which of the following is generically true?

  \begin{enumerate}[(A)]
  \item In general, $F_x$ depends only on $x$ (i.e., it is independent
    of $y$) and $F_y$ depends only on $y$. An exception is if $F$ is
    multiplicatively separable.
  \item In general, $F_x$ depends only on $y$ (i.e., it is independent
    of $x$) and $F_y$ depends only on $x$ (i.e., it is independent of
    $y$). An exception is if $F$ is multiplicatively separable.
  \item In general, both $F_x$ and $F_y$ could each depend on both $x$
    and $y$. An exception is if $F$ is additively separable, in which
    case $F_x$ depends only on $y$ and $F_y$ depends only on $x$.
  \item In general, both $F_x$ and $F_y$ could each depend on both $x$
    and $y$. An exception is if $F$ is additively separable, in which
    case $F_x$ depends only on $x$ and $F_y$ depends only on $y$.
  \item In general, either both $F_x$ and $F_y$ depend only on $x$ or
    both $F_x$ and $F_y$ depend only on $y$.
  \end{enumerate}
  
  {\em Answer}: Option (D)

  {\em Explanation}: This should be straightforward if you understand
  what's going on. Otherwise, watch this video and the subsequent one:

  \url{http://www.youtube.com/watch?v=2T7iFZVLtn0&list=PLC0bHnWu122kC1WBgr0H9PEbHTYnYev27&index=1}

  {\em Performance review}: $21$ out of $26$ got this correct. $3$
  chose (A), $1$ each chose (C) and (E).

\item Consider a production function $f(L,K,T)$ of three inputs $L$
  (labor expenditure), $K$ (capital expenditure), and $T$ (technology
  expenditure). Suppose all mixed partials of $f$ with respect to $L$,
  $K$, and $T$ are continuous. Suppose we have the following signs of
  partial derivatives: $\partial f/\partial L > 0$, $\partial
  f/\partial K > 0$, $\partial^2f/(\partial L \partial K) < 0$, and
  $\partial^3f/(\partial L\partial K \partial T) > 0$. What does this
  mean?

  \begin{enumerate}[(A)]
  \item Increasing labor increases production, increasing capital
    increases production, and labor and capital substitute for each
    other to some extent. Increasing the expenditure on technology
    increases the degree to which labor and capital substitute for
    each other.
  \item Increasing labor increases production, increasing capital
    increases production, and labor and capital substitute for each
    other to some extent. Increasing the expenditure on technology
    decreases the degree to which labor and capital substitute for
    each other, i.e., with more technology investment, labor and
    capital become more complementary.
  \item Increasing labor increases production, increasing capital
    increases production, and labor and capital complement each other
    to some extent. Increasing the expenditure on technology increases
    the degree to which labor and capital complement for each other.
  \item Increasing labor increases production, increasing capital
    increases production, and labor and capital complement each other
    to some extent. Increasing the expenditure on technology decreases
    the degree to which labor and capital complement for each other.
  \item Increasing labor or capital decreases production.
  \end{enumerate}

  {\em Answer}: Option (B)

  {\em Explanation}: $\partial f/\partial L > 0$ shows that increasing
  labor increases production. $\partial f/\partial K > 0$ shows that
  increasing capital increases production. $\partial^2f/\partial L
  \partial K < 0$ indicates that labor and capital substitute for each
  other, i.e., a small increase in capital reduces the marginal
  product of labor. Finally, $\partial^3f/\partial L\partial K
  \partial T > 0$ indicates that $\partial^2f/\partial L \partial K$
  is increasing with $T$, i.e., getting less negative. So, although
  labor and capital substitute for each other, the degree to which
  they do so reduces as $T$ increases. Roughly speaking, more
  technology reduces the antagonism between labor and capital.

  {\em Performance review}: $24$ out of $26$ got this correct. $1$
  each chose (A) and (C).

  {\em Historical note (last time)}: $12$ out of $20$ people got this
  correct. $7$ chose (A) and $1$ chose (D). The people who chose (A)
  probably didn't note that an increase in the degree of substitution
  would mean a decrease in the derivative, rather than an increase.

\item Analysis of usage of an online social network finds that the
  total time spent by people on the social network is $P^{1.3}L^{0.5}$
  where $P$ is the total number of people on the network and $L$ is a
  number of processors used at the social network's server
  facility. Which of these is true?

  \begin{enumerate}[(A)]
  \item Increasing returns both on persons and on processors: every
    new person joining the network increases the average time spent
    {\em per person} (and not just the total time), and every new
    processor added to the server facility increases the average time
    spent per processor.
  \item Constant returns on persons, increasing returns on processors
  \item Constant returns on persons, decreasing returns on processors
  \item Increasing returns on persons, decreasing returns on processors
  \item Decreasing returns on persons, increasing returns on processors
  \end{enumerate}

  {\em Answer}: Option (D)

  {\em Explanation}: The short explanation is that the exponent on $P$
  is greater than $1$, so the second partial derivative is positive,
  and the exponent of $L$ is between $0$ and $1$, so the second
  partial derivative is negative.

  The long explanation is just working it out.

  {\em Performance review}: $25$ out of $26$ got this. $1$ chose (C).

  {\em Historical note (last time)}: $14$ out of $20$ got this correct. $5$
  chose (A), $1$ chose (B).

\item {\em Not a calculus question, but has deep calculus
  interpretations -- it is basically measuring the derivative of the
  $1/x$ function with respect to $x$}: A person travels fifty miles
  every day by car and the travel distance is fixed. The price of
  gasoline, which she uses to fuel her car, is also fixed. Which of
  the following increases in fuel efficiency result in the maximum
  amount of savings for her?

  \begin{enumerate}[(A)]
  \item From $11$ to $12$ miles per gallon
  \item From $12$ to $14$ miles per gallon
  \item From $20$ to $25$ miles per gallon
  \item From $36$ to $54$ miles per gallon
  \item From $50$ to $100$ miles per gallon
  \end{enumerate}

  {\em Answer}: Option (B)

  {\em Explanation}: The gain achieved by upgrading from $a$ miles per
  gallon to $b$ miles per gallon is $(50/a - 50/b)$ times the cost of
  a gallon. In particular, it is proportional to $1/a - 1/b$. It
  remains to compute the case where this difference is largest.

  The values of $1/a - 1/b$ are: Option (A):/ $1/132$, Option (B):
  $1/84$, Option (C): $1/100$, Option (D): $1/108$, Option (E):
  $1/100$. Of these, the largest is the one with smallest denominator,
  i.e., $1/84$. In other words, the maximum gain happens in going from
  $12$ to $14$.

  This seems a little counter-intuitive at first. Looked at in terms
  of ratios, the gain from $50$ to $100$ is most impressive. Looked at
  in terms of differences in MPG values, again the gain from $50$ to
  $100$ is more impressive. However, these gains are not what we are
  measuring, because in the question, it is specified that the
  distance of travel is {\em fixed} and hence what matters is the
  absolute savings in cost.

  Intuitively, what's happening is that while a gain from $50$ to
  $100$ halves the cost, that halving is occurring from an already
  fairly small cost base, so the quantitative savings are little. On
  the other hand, a jump from $12$ to $14$ is small in proportion but
  large in absolute terms because the base from which the savings are
  occurring is much larger. In fact, even a gain from $100$ miles per
  gallon to infinite miles per gallon produces less in cost savings
  assuming fixed distance and fixed cost per gallon than a gain from
  $12$ to $14$.

  Another way of thinking of this is in terms of the derivative of
  $1/x$. We know that as $x$ increases, $1/x$ decreases. However, the
  derivative is not constant. When $x$ is small, the derivative
  $-1/x^2$ is huge in magnitude, which means that small changes in $x$
  lead to large changes in $1/x$. When $x$ is large, the derivative
  $-1/x^2$ is small in magnitude, which means that large changes in
  $x$ lead to only small changes in $1/x$.

  Thus, we can get three fairly different pictures depending on
  whether we measure things using $x$, $1/x$, or $\ln x$.

  {\em Performance review}: $14$ out of $26$ people got this. $6$
  chose (E), $3$ chose (A), $2$ chose (D), $1$ chose (C).

  {\em Historical note (last time)}: $2$ out of $20$ people got this
  correct. $7$ chose (C), $6$ chose (E), $4$ chose (D), and $1$ chose
  (A).

\item For which of the following production functions $f(L,K)$ of
  labor and capital is it true that labor and capital can be
  complementary for some choices of $(L,K)$, and substitutes for
  others? In other words, for which of these are labor and capital
  neither globally complements nor globally substitutes? Assume the
  domain $L > 0, K > 0$.

  \begin{enumerate}[(A)]
  \item $L^2 + LK + K^2$
  \item $L^2 - LK + K^2$
  \item $L^3 + L^2K + LK^2 + K^3$
  \item $L^3 + L^2K - LK^2 + K^3$
  \item $L^3 - L^2K - LK^2 + K^3$
  \end{enumerate}

  {\em Answer}: Option (D)

  {\em Explanation}: For option (D), the second mixed partial is $2(L
  - K)$ which is positive if $L > K$ and negative if $L < K$. For
  options (A) and (C), the second mixed partial is always positive,
  while for options (B) and (E), the second mixed partial is always
  negative.

  {\em Performance review}: $15$ out of $26$ got this. $6$ chose (E),
  $3$ chose (B), $2$ chose (C).

  {\em Historical note (last time)}: $8$ out of $20$ people got this
  correct. $4$ each chose (B), (C) and (E).


\item Consider the following Leontief-like production function $f(L,K)
  = (\min \{ L, K \})^2$. Assume the domain $L > 0$, $K > 0$. What is
  the nature of returns and complementarity here?

  \begin{enumerate}[(A)]
  \item Positive increasing returns on the smaller of the inputs,
    positive constant returns on the larger of the inputs
  \item Positive constant returns of the smaller of the inputs,
    positive increasing returns on the larger of the inputs
  \item Zero returns on the smaller of the inputs, positive constant
    returns on the larger of the inputs
  \item Positive decreasing returns on the smaller of the inputs, zero
    returns on the larger of the inputs
  \item Positive increasing returns on the smaller of the inputs, zero
    returns on the larger of the inputs
  \end{enumerate}

  {\em Answer}: Option (E)

  {\em Explanation}: This is a ``weak link'' type of production
  function in the sense that the weakest link in the labor-capital
  nexus determines output. If $L < K$, then output is $L^2$, and if $K
  < L$, then output is $K^2$. This means that, at the margin,
  increasing the one which is already larger produces no gain in
  output. However, increasing the one which is smaller increases
  output as the square thereof. Since $2 > 1$, there are positive
  increasing returns on the smaller input.

  A practical example of this is where ``it takes two to tango'' --
  for instance, if each unit of labor is a person and each unit if
  capital is a machine, and if there are more machines than people or
  vice versa, the extra machines/people are completely unused.

  {\em Performance review}: $15$ out of $26$ got this. $5$ chose (A),
  $4$ chose (D), $1$ each chose (B) and (C).

  {\em Historical note (last time)}: $10$ out of $20$ people got this
  correct. $3$ each chose (A) and (D), $2$ each chose (B) and (C).

\end{enumerate}
\end{document}