\documentclass[10pt]{amsart}
\usepackage{fullpage,hyperref,vipul}
\title{Chain rule and second derivatives}
\author{Math 195, Section 59 (Vipul Naik)}

\begin{document}

\maketitle

The homework question is as follows: Suppose $z = f(x,y)$ where $x =
g(s,t)$ and $y = h(s,t)$.

\begin{enumerate}
\item Show the formula for $\frac{\partial^2z}{\partial t^2}$ as
  given in the book. It is a somewhat long formula.
\item Find a similar formula for $\partial^2z/(\partial s\partial
  t)$.
\end{enumerate}

Last year, many people got it correct, but some people didn't clearly
understand how to proceed.  In anticipation of students facing similar
difficulties this year, I include a detailed outline of the solution
below. Even those students who are able to arrive at the correct
answer by themselves may benefit from reading through this.

\subsection*{Similar one-variable question}

Let's consider a simple situation, where we have $x = g(t)$ and $z =
f(x)$, with no $y$ and $s$ appearing. We want a formula for
$d^2z/dt^2$. In other words, we want to compute $(f \circ g)''$

You've all seen this, by the way, in a past quiz (January 9), where
almost all of you got the question correct. The idea now is to
carefully understand what we're doing so that it can readily be
replicated in the multiple variable case.

We have, by the chain rule:

$$\frac{dz}{dt} = \frac{dz}{dx}\frac{dx}{dt} = f'(x)g'(t) = f'(g(t))g'(t)$$

We now want to differentiate this with respect to $t$ again:

$$\frac{d^2z}{dt^2} = \frac{d}{dt}[f'(g(t))g'(t)]$$

Note that the expression to be differentiated on the right side is a
product, so we use the product rule, and obtain:

$$\frac{d^2z}{dt^2} = \frac{d}{dt}[f'(g(t))]g'(t) + f'(g(t))g''(t)$$

For the derivative of $f'(g(t))$ with respect to $t$, we {\em again}
use the chain rule, and get $f''(g(t))g'(t)$, so overall:

$$\frac{d^2z}{dt^2} = f''(g(t))(g'(t))^2 + f'(g(t))g''(t)$$

Let's understand this step by step:

\begin{enumerate}
\item We calculated the first derivative using the chain rule, and
  got a product of a derivative with respect to the intermediate value
  $x$ (which became $f'(g(t))$) and a derivative with respect to the
  initial variable $t$ (which became $g'(t)$).
\item To differentiate this, we use the product rule.
\item For one of the pieces in the product rule, we are
  differentiating $g'(t)$, which becomes $g''(t)$. This is the
  straightforward piece (second summand in our description above).
\item For the other piece, we need to differentiate the composite
  $f'(g(t))$ with respect to $t$. For this, we {\em use the chain rule
  again}.
\end{enumerate}

\subsection*{Second partial with a single variable}

Return to the original question: Suppose $z = f(x,y)$ where $x = g(s,t)$
and $y = h(s,t)$.

We have:

$$\frac{\partial z}{\partial t} = \frac{\partial z}{\partial x}\frac{\partial x}{\partial t} + \frac{\partial z}{\partial y}\frac{\partial y}{\partial t}$$

With the subscript notation, this becomes:

$$\frac{\partial z}{\partial t} = f_xg_t + f_yh_t$$

So far, we are done with the equivalent of Step (1) for one
variable. Note that instead of just a single produce, we have a sum of
two products, indicating the two paths of dependence (via $x$ and via
$y$).

Now, we want to differentiate both sides with respect to $t$:

$$\frac{\partial^2z}{\partial t^2} = \frac{\partial}{\partial t}\left(f_xg_t + f_yh_t\right)$$

We break additively and use the product rule on each piece (analogous
to Step (2)), and get:

$$\frac{\partial^2z}{\partial t^2} = \frac{\partial(f_x)}{\partial t} g_t + f_x\frac{\partial g_t}{\partial t} + \frac{\partial f_y}{\partial t}h_t + f_y\frac{\partial h_t}{\partial t}$$

Within each product rule, the second summand is easy to rewrite:
$\partial g_t/\partial t$ becomes $g_{tt}$ and $\partial h_t/\partial
t = h_{tt}$. These simpler parts are analogous to Step (3) in the
one-variable scenario.

The harder pieces are $\partial f_x/\partial t$ and $\partial
f_y/\partial t$. These are analogous to Step (4). Let's look at these
more carefully.

$f_x(x,y)$ is a function of the variables $x$ and $y$, each of which
in turn depends on $s$ and $t$. Thus:

$$\frac{\partial f_x}{\partial t} = \frac{\partial f_x}{\partial x}\frac{\partial x}{\partial t} + \frac{\partial f_x}{\partial y} \frac{\partial y}{\partial t} = f_{xx}g_t + f_{xy}h_t$$

Similarly:

$$\frac{\partial f_y}{\partial t} = \frac{\partial f_y}{\partial x}\frac{\partial x}{\partial t} + \frac{\partial f_y}{\partial y}\frac{\partial y}{\partial t} = f_{yx}g_t + f_{yy}h_t$$

Plugging these all back in the original expression, we get:

$$\frac{\partial^2z}{\partial t^2} = (f_{xx}g_t + f_{xy}h_t)g_t + f_xg_{tt} + (f_{yx}g_t + f_{yy}h_t)h_t + f_yh_{tt}$$

There is a total of six terms, but using Clairaut's theorem, we see
that the term $f_{xy}g_th_t$ appears twice, so combining, we get a sum
of five terms with a coefficient of $2$ appearing on one of them.

The key difference relative to the situation with one variable: {\em
dependence on two intermediate variables means that at Step (1), we
have a sum of two products, and in Step (4), again each of the chain
derivatives is a sum of two products. Hence we get a total of $6$
terms.}

\subsection*{Mixed partial with both variables}

Let's try to compute $\partial^2 z/(\partial s\partial t)$. We already have:

$$\frac{\partial z}{\partial t} = f_xg_t + f_yh_t$$

We differentiate both sides with respect to $s$, and get:

$$\frac{\partial^2z}{\partial s \partial t} = \frac{\partial}{\partial s}[f_xg_t + f_yh_t]$$

Using additive splitting and the product rule (analogous to Step (2)), we get:

$$\frac{\partial^2z}{\partial s \partial t} = \frac{\partial f_x}{\partial s}g_t + f_x\frac{\partial g_t}{\partial s} + \frac{\partial f_y}{\partial s}h_t + f_y \frac{\partial h_t}{\partial s}$$

The second and fourth summand are easy to simply: $\partial
g_t/\partial s = g_{ts}$ and $\partial h_t/\partial s = h_{ts}$. The
harder ones are $\partial f_x/\partial s$ and $\partial f_y/\partial
s$. As before, we use the chain rule:

$$\frac{\partial f_x}{\partial s} = \frac{\partial f_x}{\partial x}\frac{\partial x}{\partial s} + \frac{\partial f_x}{\partial y}\frac{\partial y}{\partial s} = f_{xx}g_s + f_{xy}h_s$$

Similarly:

$$\frac{\partial f_y}{\partial s} = \frac{\partial f_y}{\partial x}\frac{\partial x}{\partial s} + \frac{\partial f_y}{\partial y}\frac{\partial y}{\partial s} = f_{yx}g_s + f_{yy}h_s$$

Plugging these in, we get:

$$\frac{\partial^2z}{\partial s\partial t} = (f_{xx}g_s + f_{xy}h_s)g_t + f_xg_{ts} + (f_{yx}g_s + f_{yy}h_s)h_t + f_yh_{ts}$$

As with the previous case, this is a sum of six terms.
\end{document}