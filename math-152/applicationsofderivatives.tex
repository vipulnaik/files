\section{Show off your mathematical literacy!}

Ideas and terminology from differential calculus have a healthy
interaction with popular culture. In particular, the study of {\em
trends over time} uses a lot of the terminology and notions of
derivatives. Even though most people studying these trends don't need
to differentiate rational functions in $15$ seconds, they do need to
have a fairly thorough grasp of various concepts.

\subsection{Exhibit one of popular culture: kinematics}

For those of you who have seen some mechanics in physics, the Leibniz
notation can be used to describe velocity, acceleration, etc. in terms
of position for motion in one dimension.

If we consider the position $x$ as a function of time $t$, then:

\begin{enumerate}
\item The velocity $v$ is defined by $v := dx/dt$. In other words, the
  velocity $v$ at time $t_0$ is $\frac{dx}{dt}|_{t = t_0}$. The
  magnitude of $v$ is sometimes called the {\em speed}.
\item The acceleration $a$ is given by $a := dv/dt$. Thus, $a =
  d^2x/dt^2$.
\end{enumerate}

If you have driven or seen driven a car, you may have noticed that
there's an {\em accelerator} pedal. Pressing the accelerator pedal
basically accelerates, or increases the speed (I'll assume you're
going in the forward direction). If you leave the accelerator pedal,
then your car will not be accelerating, which means it will go at a
constant speed. Actually, what happens is that due to what's called
rolling resistance or rolling friction, your car will actually have
negative acceleration and will stop after some time.

Now, you may ask -- {\em how does the acceleration change with time}?
If you start out from a car at rest, that's not moving, acceleration
till you press the pedal is zero. If you press the accelerator pedal
sudddenly, the car {\em jerks} forward, which is bad, so you probably
press the accelerator pedal gradually. So, you might be interested in
$da/dt$ -- how your acceleration changes with time. This is the {\em
third derivative} $d^3x/dt^3$, and is called the {\em jerk}.

Now, here are some points worth noting:

\begin{enumerate}
\item Position $x$ being continuous in $t$ simply means that the car
  does not teleport from one point to another. This is expected.
\item Velocity $v$ being continuous in $t$ means that the car does not
  {\em suddenly} brake or start moving. In particular, the velocity,
  when changing, must go through all intermediate values. You cannot
  go from a speed of $5$ to a speed of $50$ without going through all
  intermediate speeds. This need not always be true, but it should be
  true for good driving.
\item Acceleration being continuous in $t$ means that you change speed
  very gradually and smoothly.
\end{enumerate}

Basically, we see that making higher derivatives exist and be
continuous in $t$ corresponds to having a progressively smoother ride.

\subsection{Instantaneous rules: what do they mean in science?}

There are two broad kinds of mathematical laws and rules you're like
to see inthe physical sciences:

\begin{itemize}
\item Laws that operate at an instantaneous level, i.e., they relate
  various quantities that all operate at an instant or point in
  time. For instance, Newton's second law relates the instantaneous
  force at a point in time to the acceleration at that same point in
  time. These laws typically involve derivatives.
\item Laws that operate across a time interval. These are usually
  conservation laws, which state something like: the total energy in
  the system remains the same over a time interval, or the loss in
  total energy over a time interval equals the net energy exchanged
  with the environment, and so on.
\end{itemize}

These two kinds of laws are typically related in the following way:
for every law that operates across a time interval, there is a
corresponding instantaneous law that is obtained by differentiating
the relevant quantities with respect to time. For instance, Newton's
laws are the instantaneous counterparts (and hence obtained by
differentiation) of the conservation of momentum and momentum-impulse
laws.

Conversely, the laws over a time interval are obtained by a process
called (definite) integration from the instantaneous laws. Integration
is the reverse of differentiation.

It's a moot point whether instantaneous laws or laws operating over
time intervals (i.e., conservation laws) are more fundamental. In many
cases, both formulations are roughly. For less fundamental physics and
chemistry laws, however, the instantaneous version of the law is more
basic and fundamental than the version of the law that operates across
a time interval. We turn to some examples of this in the next
subsection.

\subsection{Exhibit two from popular culture: kinetics}

A chemical reaction takes some input chemicals (called reactants) and
gives some output chemicals (called products). A typical chemical
reaction may look like:

$$A + B \to C + D$$

where $A$ and $B$ are reactants and $C$ and $D$ are products. The
number of reactants and products may vary, and we may put nonnegative
integer coefficients to balance the reaction equation stoichiometrically.

Assuming that the reaction occurs in the forward direction, the
quantity of $A$ (which is typically measured using concentration when
the reaction is in an aqueous or other liquid solution) decreases with
time. In fact, the quantity of each reactant decreases with time. The
rate of change of the concentration of $A$ is written as $d[A]/dt$,
and this is negative. On the other hand, $d[C]/dt$ is positive.

Kinetics is the branch of chemistry that tries to express these rates
$d[A]/dt$, $d[B]/dt$, etc. as functions of $t$. An unholy alliance
between kinetics and the theory of integration/differential equations
(depending on the complexity of the situation) then allows us to
predict what happens to the concentrations of the reactants and
products over time.

\subsection{Exhibit three from popular culture: supply and demand}
