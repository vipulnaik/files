\documentclass[10pt]{amsart}

%Packages in use
\usepackage{fullpage, hyperref, vipul, enumerate}

%Title details
\title{Take-home class quiz solutions: due Wednesday January 23: Vectors, 3D, and parametric stuff -- miscellanea}
\author{Math 195, Section 59 (Vipul Naik)}
%List of new commands

\begin{document}
\maketitle

\section{Performance review}

$23$ people took this $12$-question quiz. The score distribution was
as follows:

\begin{itemize}
\item Score of $7$: $2$ people.
\item Score of $8$: $4$ people.
\item Score of $9$: $5$ people.
\item Score of $10$: $9$ people.
\item Score of $11$: $3$ people.
\end{itemize}

The question-wise answers and performance review were as follows:

\begin{enumerate}
\item Option (E): $20$ people.
\item Option (B): $23$ people.
\item Option (E): $22$ people.
\item Option (E): $20$ people.
\item Option (D): $21$ people.
\item Option (C): $23$ people.
\item Option (E): $12$ people.
\item Option (E): $20$ people.
\item Option (A): $10$ people.
\item Option (D): $22$ people.
\item Option (D): $7$ people.
\item Option (A): $14$ people.
\end{enumerate}

\section{Solutions}

\begin{enumerate}

\item Suppose we are given three subsets $\Gamma_1$, $\Gamma_2$, and
  $\Gamma_3$ of $\R^3$ where $\Gamma_1$ is the set of solutions to
  $F_1(x,y,z) = 0$, $\Gamma_2$ is the set of solutions to $F_2(x,y,z)
  = 0$, and $\Gamma_3$ is the set of solutions to $F_3(x,y,z) =
  0$. Which of the following equations gives precisely the set of
  points that lie in {\em at least two} of the subsets $\Gamma_1$,
  $\Gamma_2$, and $\Gamma_3$?

  \begin{enumerate}[(A)]
  \item $F_1(x,y,z)F_2(x,y,z)F_3(x,y,z) = 0$
  \item $(F_1(x,y,z))^2 + (F_2(x,y,z))^2 + (F_3(x,y,z))^2 = 0$
  \item $(F_1(x,y,z) + F_2(x,y,z) + F_3(x,y,z))^2 = 0$
  \item $(F_1(x,y,z)F_2(x,y,z)) + (F_2(x,y,z)F_3(x,y,z)) +
    (F_3(x,y,z)F_1(x,y,z)) = 0$
  \item $(F_1(x,y,z)F_2(x,y,z))^2 + (F_2(x,y,z)F_3(x,y,z))^2 +
    (F_3(x,y,z)F_1(x,y,z))^2 = 0$
  \end{enumerate}

  {\em Answer}: Option (E)

  {\em Explanation}: Option (E) means that all the statements
  $F_1(x,y,z)F_2(x,y,z) = 0$, $F_2(x,y,z)F_3(x,y,z) = 0$, and
  $F_3(x,y,z)F_1(x,y,z) = 0$ must simultaneously be true. This forces
  at least two of the values $F_1(x,y,z)$, $F_2(x,y,z)$, and
  $F_3(x,y,z)$ to equal zero. Conversely, if two or more of these are
  zero, so are all the pair products. Thus, a point satisfies this
  equation if and only if it lies in at least two of the three subsets
  $\Gamma_1$, $\Gamma_2$, $\Gamma_3$.

  {\em Performance review}: $20$ out of $23$ got this. $2$ chose (B),
  $1$ chose (D).

\item Suppose we are given three subsets $\Gamma_1$, $\Gamma_2$, and
  $\Gamma_3$ of $\R^3$ where $\Gamma_1$ is the set of solutions to
  $F_1(x,y,z) = 0$, $\Gamma_2$ is the set of solutions to $F_2(x,y,z)
  = 0$, and $\Gamma_3$ is the set of solutions to $F_3(x,y,z) =
  0$. Which of the following equations gives precisely the set
  $\Gamma_1 \cap (\Gamma_2 \cup \Gamma_3)$?

  \begin{enumerate}[(A)]
  \item $F_1(x,y,z) + F_2(x,y,z)F_3(x,y,z) = 0$
  \item $(F_1(x,y,z))^2 + (F_2(x,y,z)F_3(x,y,z))^2 = 0$
  \item $(F_1(x,y,z) + F_2(x,y,z)F_3(x,y,z))^2 = 0$
  \item $F_1(x,y,z)(F_2(x,y,z) + F_3(x,y,z))^2 = 0$
  \item $F_1(x,y,z)((F_2(x,y,z))^2 + (F_3(x,y,z))^2) = 0$
  \end{enumerate}

  {\em Answer}: Option (B)

  {\em Explanation}: Option (B) involves a sum of two squares being
  zero, so both the things being squared must equal zero. Thus,
  $F_1(x,y,z) = 0$ and $F_2(x,y,z)F_3(x,y,z) = 0$. The solution set to
  the latter is the union of the solution sets for $F_2$ and $F_3$, so
  is $\Gamma_2 \cup \Gamma_3$. So the overall solution set is
  $\Gamma_1 \cap (\Gamma_2 \cup \Gamma_3)$.

  {\em Performance review}: All $23$ got this correct.

\item Suppose we are given three subsets $\Gamma_1$, $\Gamma_2$, and
  $\Gamma_3$ of $\R^3$ where $\Gamma_1$ is the set of solutions to
  $F_1(x,y,z) = 0$, $\Gamma_2$ is the set of solutions to $F_2(x,y,z)
  = 0$, and $\Gamma_3$ is the set of solutions to $F_3(x,y,z) =
  0$. Which of the following equations gives precisely the set
  $\Gamma_1 \cup (\Gamma_2 \cap \Gamma_3)$?

  \begin{enumerate}[(A)]
  \item $F_1(x,y,z) + F_2(x,y,z)F_3(x,y,z) = 0$
  \item $(F_1(x,y,z))^2 + (F_2(x,y,z)F_3(x,y,z))^2 = 0$
  \item $(F_1(x,y,z) + F_2(x,y,z)F_3(x,y,z))^2 = 0$
  \item $F_1(x,y,z)(F_2(x,y,z) + F_3(x,y,z))^2 = 0$
  \item $F_1(x,y,z)((F_2(x,y,z))^2 + (F_3(x,y,z))^2) = 0$
  \end{enumerate}

  {\em Answer}: Option (E)

  {\em Explanation}: The solution set to Option (E) is the union of
  the solution sets for $F_1$ and for $F_2^2 + F_3^2$. The latter is
  precisely the set of points for which $F_2 = F_3 = 0$, so is
  $\Gamma_2 \cap \Gamma_3$. The overall solution is thus $\Gamma_1
  \cup (\Gamma_2 \cap \Gamma_3)$.

  {\em Performance review}: $22$ out of $23$ got this. $1$ chose (D).

\item Start with two vectors $a$ and $b$ in $\R^3$ such that $a \times
  b \ne 0$. Consider a sequence of vectors $c_1, c_2, \dots, c_n,
  \dots$ in $\R^3$ (note: each $c_n$ is a three-dimensional vector)
  defined as follows: $c_1 = a \times b$ and $c_{n+1} = a \times c_n$
  for $n \ge 1$. Which {\em one} of the following statements is {\bf
  false} about the $c_n$s? (5 points)

  \begin{enumerate}[(A)]
  \item All the vectors $c_n$ are nonzero vectors.
  \item $c_n$ and $c_{n+1}$ are orthogonal for every $n$.
  \item $c_n$ and $c_{n+2}$ are parallel for every $n$.
  \item $c_n$ and $a$ are orthogonal for every $n$.
  \item $c_n$ and $b$ are orthogonal for every $n$.
  \end{enumerate}

  {\em Answer}: Option (E)

  {\em Explanation}: Note first that since $a \times b \ne 0$, both
  $a$ and $b$ are nonzero vectors.

  In fact, although $c_1$ is orthogonal to both $a$
  and $b$, $c_2$, being orthogonal to $c_1$ and $a$, is in the plane
  of $a$ and $b$ and is orthogonal to $a$. Since $a$ and $b$ are not
  parallel, $c_2$ is not orthogonal to $b$.

  For the other options:

  Options (A) and (D): $c_1$ is orthogonal to $a$ because it is a
  cross product involving $a$ and a nonzero vector. At each stage, we
  are taking a cross product of a nonzero vector orthogonal to $a$
  with the nonzero vector $a$, so we get a nonzero vector orthogonal
  to $a$.

  Option (B): $c_{n+1}$ is a cross product of $a$ and $c_n$, and $a$
  and $c_n$ are both nonzero, so $c_{n+1}$ is orthogonal to $c_n$.

  Option (C): All the $c_n$s are orthogonal to $a$, so they are all in
  the plane orthogonal to $a$. Within this plane, each is
  perpendicular to its predecessor. Thus, $c_n$ and $c_{n+2}$ must be
  collinear. In fact, they points in opposite directions to each
  other, but are in the same line.

  {\em Performance review}: $20$ out of $23$ got this. $2$ chose (B),
  $1$ chose (D).

  {\em Historical note (last time)}: $9$ out of $23$ people got this
  correct. $5$ chose (C), $4$ chose (D), $3$ chose (A), $2$ chose (B).

\item As a general rule, what would you expect should be the
  dimensionality of the set of solutions to $m$ independent and
  consistent equations in $n$ variables? By solution, we mean here
  that the solution should be the $n$-tuples with coordinates in $\R$
  (or elements of $\R^n$) that satisfy all the $m$ equations. Assume
  $n \ge m \ge 1$.

  \begin{enumerate}[(A)]
  \item $n$
  \item $m$
  \item $n - 1$
  \item $n - m$
  \item $1$
  \end{enumerate} 

  {\em Answer}: Option (D)

  {\em Explanation}: As a general rule, we start out with the whole
  space, and each new constraint, if independent of prior constraints,
  whittles down the dimension by $1$. Thus, introducing $m$
  constraints in $n$-dimensional space gives a dimension of $n - m$.

  Note that this is not a hard-and-fast rule, because we can use
  tricks like the {\em sum of squares} trick to combine multiple
  equations into a single equation. However, it is a good rule of
  thumb for generic equations.

  {\em Performance review}: $21$ out of $23$ got this. $2$ chose (B).

\item As a general rule, what would you expect should be the
  dimensionality of the set of points in $\R^n$ that satisfy at least
  one of $m$ independent and consistent equations in $n$ variables?
  Assume $n \ge m \ge 1$.

  \begin{enumerate}[(A)]
  \item $n$
  \item $m$
  \item $n - 1$
  \item $n - m$
  \item $1$ 
  \end{enumerate} 

  {\em Answer}: Option (C)

  {\em Explanation}: We get a union of solution sets for equations,
  and each of these solution sets is of dimension $n - 1$ (since it is
  obtained by imposing a single constraint on $n$-dimensional
  space). The union should thus also have dimension $n - 1$.

  {\em Performance review}: All $23$ got this correct.

\item Measuring time $t$ in seconds since the beginning of the year
  2013, and stock prices on a $24 \times 7$ stock exchange in
  predetermined units, the stock prices of companies $A$, $B$, and $C$
  were found to be given by $30 + t/5000000 - \sin(t/10000)$, $16 +
  7t/3000000$, and $40 + t/1000000 - 5\sin(t/10000)$. To what extent
  can we deduce the stock prices of the companies from each other at a
  given point in time, without knowing what the time is?

  \begin{enumerate}[(A)]
  \item The stock price of any of the three companies can be used to
    deduce the other stock prices.
  \item The stock price of company $A$ can be used to deduce the stock
    prices of companies $B$ and $C$, but no other deductions are possible.
  \item The stock price of company $A$ can be used to deduce the stock
    prices of companies $B$ and $C$, and the stock price of company
    $C$ can be used to deduce the stock prices of companies $A$ and
    $B$.
  \item The stock price of company $B$ can be used to determine the
    stock prices of companies $A$ and $C$, and no other deductions are
    possible.
  \item The stock price of company $B$ can be used to determine the
    stock prices of companies $A$ and $C$, and the stock prices of
    companies $A$ and $C$ can be used to deduce each other but cannot
    be used to uniquely deduce the stock price of company $B$.
  \end{enumerate}

  {\em Answer}: Option (E)

  {\em Explanation}: Since the stock price of company $B$ is linear in
  $t$, we can determine $t$ from it, and use this to determine the
  stock prices of $A$ and $C$. This means that the only options we
  need to consider are (D) and (E).

  The stock prices of $A$ and $C$ are related by a linear relation
  $5(A - 30) = C - 40$, or equivalently, $5A = C + 110$. Thus, either
  is expressible in terms of the other.

  {\em Performance review}: $12$ out of $23$ got this correct. $10$
  chose (D), $1$ chose (C).

\item Lushanna is coaching $30$ young athletes for a 100 meter
  sprint. Every day, at the beginning of the day, she asks the athlete
  to run 100 meters as fast as they can and notes the time taken. She
  thus gets a vector with $30$ coordinates (measuring the time taken
  by all the athletes) everyday. Lushanna then plots a graph in
  thirty-dimensional space that includes all the points for her daily
  measurements. Each of the following is a sign that Lushanna's young
  athletes are improving. Which of these signs is {\bf strongest}, in
  the sense that it would imply all the others?

  \begin{enumerate}[(A)]
  \item The norm (length) of the vector every day (after the first) is
    less than the norm of the vector the previous day.
  \item The sum of the coordinates of the vector every day (after the
    first) is less than the sum of the coordinates of the vector the
    previous day.
  \item The minimum of the coordinates of the vector every day (after
    the first) is less than the minimum of the coordinates of the
    vector the previous day.
  \item The maximum of the coordinates of the vector every day (after
    the first) is less than the maximum of the coordinates of the
    vector the previous day.
  \item Each of the coordinates of the vector every day (after the
    first) is less than the corresponding coordinate of the vector the
    previous day.
  \end{enumerate}

  {\em Answer}: Option (E)

  {\em Explanation}: If each of the coordinates goes down, then all
  the measures (the maximum, minimum and various average measures) go
  down. However, it is possible for any one of these measures to go
  down while the others don't.

  {\em Performance review}: $20$ out of $23$ got this. $3$ chose (B).

\item In a closed system (no mass exchanged with the surroundings) a
  reversible chemical reaction $A + B \to C + D$, and its reverse, are
  proceeding. There are no other chemicals in the system, and no other
  reactions are proceeding in the system. A chemist studying the
  reaction decides to track the masses of $A$, $B$, $C$, and $D$ in
  the system as a function of time, and plots a parametric curve in
  four-dimensional space. What can we say about the nature of the
  curve, ignoring the parametrization (i.e., just looking at the set
  of points covered)?


  \begin{enumerate}[(A)]
  \item It is a part of a straight line.
  \item It is a part of a circle.
  \item It is a part of a parabola.
  \item It is a part of an astroid.
  \item It is a part of a cissoid.
  \end{enumerate}

  {\em Answer}: Option (A)

  {\em Explanation}: This follows from the law of constant proportions
  in chemistry! Basically, the amounts of gain/loss in each coordinate
  are fixed in proportion based on the stoichiometry of the reaction.

  {\em Performance review}: $10$ out of $23$ got this correct. $10$
 chose (B), $2$ chose (D), $1$ chose (C).

  {\em Lobbying special}: Casa is a lobbyist for a special interest
  group. There are three politicians $P_1,P_2,P_3$ competing for a
  general election. Casa has computed that the probabilities of the
  politicians winning are $p_1$ for $P_1$, $p_2$ for $P_2$, and $p_3$
  for $P_3$, with $p_1,p_2,p_3 \in [0,1]$ and $p_1 + p_2 + p_3 =
  1$. Casa estimates a payoff of $m_1$ money units to her special
  interest group if $P_1$ wins, $m_2$ money units if $P_2$ wins, and
  $m_3$ money units if $P_3$ wins. (These payoffs may be in terms of
  passage of favorable laws, repeal of unfavorable laws, or
  enforcement of laws unfavorable to competitors).

\item What is the expected payoff to the special interest group that
  Casa represents?

  \begin{enumerate}[(A)]
  \item $m_1 + m_2 + m_3$
  \item $(m_1 + m_2 + m_3)/3$
  \item $(p_1 + p_2 + p_3)(m_1 + m_2 + m_3)$
  \item $p_1m_1 + p_2m_2 + p_3m_3$
  \item $\sqrt{m_1^2 + m_2^2 + m_3^2}$
  \end{enumerate}

  {\em Answer}: Option (D)

  {\em Explanation}: The expected payoff contributions for each
  victory are $p_1m_1$, $p_2m_2$, and $p_3m_3$ respectively.

  {\em Performance review}: $22$ out of $23$ got this. $1$ chose (C).

\item Casa has discovered that the bribe multipliers of the
  politicians are the positive reals $b_1$, $b_2$, and $b_3$
  respectively. In other words, if Casa donates $u_i$ money units to
  $P_i$, then the expected payoff from politician $P_i$ winning is now
  $m_i + b_iu_i$. Consider the vectors $p = \langle p_1,p_2,p_3
  \rangle$, $m = \langle m_1,m_2,m_3 \rangle$, $c = \langle p_1b_1,
  p_2b_2, p_3b_3 \rangle$, and $f = \langle p_1/b_1, p_2/b_2, p_3/b_3
  \rangle$ and let $u = \langle u_1,u_2,u_3\rangle$ be the vector of
  the bribe quantities Casa gives to the politicians
  respectively. Assume that bribing politicians does not affect the
  relative probabilities of winning the election. Which of the
  following describes Casa's expected payoff from the election, once
  the bribe is made (if you want to include the cost of bribes, you'd
  need to subtract $u_1 + u_2 + u_3$ from this answer, but we're not
  doing that. {\em Note: Some of the answer options may not make sense
  from a dimensions/units viewpoint, but the correct answer does make
  sense}.

  \begin{enumerate}[(A)]
  \item $p \cdot (m + u)$
  \item $p \cdot (m + (b \cdot u))$
  \item $p \cdot (m + (f \cdot u))$
  \item $(p \cdot m) + (c \cdot u)$
  \item $p \cdot (f \cdot m + u)$
  \end{enumerate}

  {\em Answer}: Option (D)

  {\em Explanation}: The expected payoff for each politician is $m_i +
  b_iu_i$, so the expected payoff accounting for probability is
  $p_im_i + p_ib_iu_i = p_im_i + c_iu_i$. The sum thus becomes $(p
  \cdot m) + (c \cdot u)$.

  {\em Performance review}: $7$ out of $23$ got this. $14$ chose (B),
  $1$ chose (E), $1$ left the question blank.

\item Continuing with the full setup of the preceing question, what is
  Casa's optimal bribing strategy on a fixed budget of money to be
  used for bribes?

  \begin{enumerate}[(A)]
  \item Donate all the money to the politician with the maximum value
    of $p_ib_i$, i.e., to the politician corresponding to the largest
    coordinate of the vector $c$.
  \item Donate all the money to the politician with the minimum value
    of $p_ib_i$, i.e., to the politician corresponding to the smallest
    coordinate of the vector $c$.
  \item Donate all the money to the politician with the maximum value
    of $p_i/b_i$, i.e., to the politician corresponding to the largest
    coordinate of the vector $f$.
  \item Donate all the money to the politician with the minimum value
    of $p_i/b_i$, i.e., to the politician corresponding to the smallest
    coordinate of the vector $f$.
  \item Split the bribery budget between the politicians in the ratio
    $p_1b_1:p_2b_2:p_3b_3$.
  \end{enumerate}

  {\em Answer}: Option (A)

  {\em Explanation}: $p_ib_i$ is the overall return on investment
  multiplier for bribing any politician. It makes sense to spend
  scarce bribery resources on the politician with the highest return
  on investment.

  {\em Performance review}: $14$ out of $23$ got this. $6$ chose (E),
  $3$ chose (C).
\end{enumerate}
\end{document}
