\documentclass[10pt]{amsart}

%Packages in use
\usepackage{fullpage, hyperref, vipul}

%Title details
\title{Lie cring theory}
\author{Vipul Naik}

%List of new commands
\newcommand{\Skew}{\operatorname{Skew}}

\begin{document}
\maketitle

\section*{Motivational note}

There are two broad ways of motivating crings. First, they
generalize from the very strong assumption of distributivity to the
somewhat weaker assumption of $2$-cocycles, and hence are a natural
generalization.

Second, there are many cases where we have Lie rings or associative
rings or Jordan rings where we want to ``half'' the multiplication,
but there is no way of doing this halving while still remaining a Lie
ring or associative ring or Jordan ring. Cring theory gives us a
larger space in which to search for halves, and its advantages are
most notable in the case of $2$-torsion. Even though the halves
arising thus are not unique, they are unique up to coboundary, i.e.,
they are homologous.

Cring theory thus provides one special case of the approach of
managing division by $p$ when such division is not possible --
specifically, it is tailored to the case $p = 2$.
\section{Crings: basics}

\subsection{Ring recall}

We will use the term {\em ring} for a set with an abelian group
structure denoted additively and a binary operation, denoted
multiplicatively, that distributes both ways over the addition. Thus,
a ring need not be commutative, associative, or unital. Of particular
interest is a {\em Lie ring}, where the multiplication, denoted by a
Lie bracket, is alternating and satisfies the Jacobi identity.

\subsection{Cring: definition}

A {\em cring} is a set $L$ endowed with the structure of an abelian
group denoted additively, and a binary operation $*: L \times L \to L$
satisfying the following two conditions:

\begin{enumerate}
\item The zero condition: $x * 0 = 0 * x = 0$ for all $x$.
\item The $2$-cocycle condition with respect to addition, given below.
\end{enumerate}

\begin{equation}
  (x * (y + z)) + (y * z) = ((x + y) * z) + (x * y)
\end{equation}

Equivalently the cocycle condition can be written as:

\begin{equation}
  (x * (y + z)) - (x * y) = ((x + y) * z) - (y * z)
\end{equation}

Any ring is a cring.

We define the following useful triple operation:

\begin{equation}\label{eq:starone}
  x *_y z := (x * (y + z)) - (x * y)
\end{equation}

By the cocycle condition, we also have:

\begin{equation}\label{eq:startwo}
  x *_y z = ((x + y) * z) - (y * z)
\end{equation}

By the zero condition, we have $x *_0 z = x * z$ for all $x$, $z$.

For three elements $x$, $y$, and $z$, we define the {\em triple
  distributor} as:

\begin{equation}\label{eq:tripledistributordef}
  \delta(x,y,z) := ((x + y) * z) - (x * z) - (y * z) = (x * (y + z)) - (x * z) - (y * z)
\end{equation}
The equivalence of these again follows from the cocycle condition.

It can also be characterized as $x *_y z - x * z$.

Thus, we can also write:

\begin{equation}
  x *_y z = (x * z) + \delta(x,y,z)
\end{equation}
We say that $(x,y,z)$ is a {\em distributive triple} if its triple
distributor is zero.
\subsection{Lemmas}

\begin{lemma}[Triple distributor]
  In any cring, the triple distributor of three elements is symmetric
  in all variables. In other words, $\delta(x,y,z)$ equals
  $\delta(x,z,y)$, $\delta(y,x,z)$, $\delta(y,z,x)$, $\delta(z,x,y)$,
  and $\delta(z,y,x)$.
\end{lemma}

\begin{proof}
  It suffices to show that the triple distributor remains invariant
  under a change of the second and third coordinate, and also under a
  change of the first and second coordinate. But these follow
  respectively from the first and second definitions of triple
  distributor.
\end{proof}

Here is another useful identity:

\begin{lemma}[Shift identity]\label{lemma:shift}
  If $L$ is a cring, then for $x_1,x_2,y,z \in L$, we
  have:

  $$(x_1 + x_2) *_y z = x_1 *_{x_2 + y} z + x_2 *_y z$$
\end{lemma}

\begin{lemma}[Shear property of twisted product]\label{lemma:sheartwisted}
  $(-x) *_y z = -(x *_{y - x} z)$. 
\end{lemma}

\begin{proof}
  Set $y' = y - x$, so $y = y' + x$. We have, by equation
  \ref{eq:startwo}, that:

  $$(-x) *_y z = (y - x) * z - (y * z)$$

  Using the substitution, we get:

  $$(-x) *_y z = (y' * z) - ((x + y') * z)$$

  The right side is the negative of $x *_{y'} z$, as required.
\end{proof}

The next fact is crucial to much that will follow.

\begin{theorem}[Skew of cring is ring]\label{thm:skewofcringisring}
  If $L$ is a cring, and we define a new operation $\cdot$ by:

  $$x \cdot y = (x * y) - (y * x)$$

  Then $L$ is an alternating ring with the multiplication $\cdot$.
\end{theorem}

This is related to the fact that the skew of any $2$-cocycle is an
alternating biadditive map.

\subsection{Centralizing and distributing}

We say that two elements $x$, $z$ in a cring $L$
{\em centralize} each other if we have:

\begin{equation}
  x *_y z = z *_y x = 0 \ \forall \ y \in L
\end{equation}

We denote this by $x \perp z$.  

For a subset $S$ of $L$, define the {\em centralizer}:

\begin{equation}
  C_L(S) := \{ x \in L \mid x \perp z \ \forall \ z \in S \}
\end{equation}

We say that $x$ and $z$ form a {\em distributive pair} if:

\begin{equation}
  \delta(x,y,z) = 0 \ \forall \ y \in L
\end{equation}

Note that since $\delta$ is symmetric in all three inputs, $x$ and $z$
form a distributive pair if and only if $z$ and $x$ form a
distributive pair.

For a subset $S$ of $L$, we define the {\em distributor} $D_L(S)$ as:

\begin{equation}
  D_L(S) := \{ x \in L \mid \delta(x,y,z) = 0 \ \forall \ z \in S,y \in L \}
\end{equation}

We now state an important lemma:

\begin{lemma}[Centralizer is additive subgroup]
  Suppose $L$ is a cring. Then, for any subset $S$ of $L$,
  $C_L(S)$ is an additive subgroup of $L$.
\end{lemma}

\begin{proof}
  By the zero condition, $0 \in C_L(S)$. Thus, we need to show
  that $C_L(S)$ is closed under negation and under addition. Let's
  deal with negation first.

  {\em To prove}: $(-x) *_y z = z*_y (-x) = 0$ for all $x \in C_L(S)$,
  $y \in L$, $z \in S$.

  {\em Proof}: By lemma \ref{lemma:sheartwisted}, we have $(-x) *_y z
  = -(x *_{y-x} z)$, and the latter is $0$, hence so is the
  former. Thus, $(-x) *_y z = 0$. The proof that $z *_y (-x) = 0$ is
  analogous.

  We now deal with addition.

  {\em To prove}: If $x_1, x_2 \in C_L(S)$, then $(x_1 + x_2) *_y z =
  0$ and $z *_y (x_1 + x_2) = 0$ for all $y \in L$, $z \in S$.

  {\em Proof}: By lemma \ref{lemma:shift}, $(x_1 + x_2) *_y z = x_1 *_{x_2 +
    y} z + x_2 *_y z$. Both the quantities being added on the right
  side are zero, hence the left side is also zero.

  The proof for $z *_y (x_1 + x_2)$ is analogous.
\end{proof}

The next lemma is fairly similar:

\begin{lemma}[Distributor is additive subgroup]
  Suppose $L$ is a cring. Then, for any subset $S$ of $L$,
  $D_L(S)$ is an additive subgroup of $L$.
\end{lemma}

\begin{proof}
 By the zero condition, $0 \in D_L(S)$. Thus, we need to show
  that $D_L(S)$ is closed under negation and under addition. Let's
  deal with negation first.

  {\em To prove}: $(-x) *_y z = (-x) * z$ for $x \in D_L(S)$, $y \in
  L$ and $z \in S$.

  {\em Proof}: From lemma \ref{lemma:sheartwisted}, we have $(-x) *_y
  z = -(x *_{y-x} z)$, and the latter is $0$. 

  We now deal with addition.

  {\em To prove}: If $x_1, x_2 \in D_L(S)$, $(x_1 + x_2) *_y z = (x_1
  + x_2) * z$.

  {\em Proof}: By lemma \ref{lemma:shift}, $(x_1 + x_2) *_y z = x_1
  *_{x_2 + y} z + x_2 *_y z$. From the given condition, the right side
  becomes $(x_1 * z) + (x_2 * z)$. In particular, the right side is
  independent of $y$, hence is the same as the right side when $y =
  0$, which is precisely $(x_1 + x_2) * z$.
\end{proof}

\subsection{Distributor and center}

For a cring $L$, we define two important subsets.

The {\em distributive kernel} of $L$, which we denote by $\Delta(L)$,
is defined as the set:

\begin{equation}
  \Delta(L) := \{ x \in L \mid \delta(x,y,z) = 0 \ \forall y,z \in L \}
\end{equation}

In other words, $\Delta(L) = D_L(L)$ by the previous notation.

The {\em center} of $L$, which we denote by $Z(L)$, is defined as the set:

\begin{equation}
  Z(L) := \{ x \in L \mid x \perp z \ \forall \ z \in L \}
\end{equation}

The center can also be defined as:

\begin{equation}
  Z(L) := \{ x \in L \mid x * z = z * x = 0 \ \forall \ z \in L \}
\end{equation}

In other words, $Z(L) = C_L(L)$. Note that $Z(L)$ is a subset of
$\Delta(L)$.

\begin{lemma}
  $Z(L)$ and $\Delta(L)$ are additive subgroups of $L$.
\end{lemma}

Although the next lemma says nothing new per se, the results in it
are crucial to building our intuition about manipulating expressions
in crings.

\begin{lemma}
  Let $L$ be a cring. The following hold:

  \begin{enumerate}
  \item If $x \in Z(L)$ and $y,z \in L$, then $(x + y) * z = y * z$
    and $x * y = x * z = 0$. 
  \item As a corollary to (1), the product of two elements in $L$
    depends only on the cosets of $Z(L)$ in which they land (we'll 
    make this more precise when we introduce ideals and quotients).
  \item if $x \in \Delta(L)$ and $y,z \in L$ then $(x + y) * z = (x *
    z) + (y * z)$ and $x * (y + z) = (x * y) + (x * z)$.
  \item If $x_1,x_2 \in \Delta(L)$, then $(x_1 + y_1) * (x_2 + y_2)$
    can be distributed out completely (using FOIL, for instance).
  \end{enumerate}
\end{lemma}

\subsection{Definition of ideal}

In a cring $L$, a {\em left ideal} is a subset $I$ that is an
abelian subgroup of $L$ under addition and such that $x * z \in I$ for
all $x \in L$ and $z \in I$. Note that this is equivalent to requiring
that $x *_y z \in I$ for all $x \in I$ and $y,z \in L$. This is
because $x *_y z = ((x + y) * z) - (y * z)$, and both of them are
guaranteed to be in $I$, hence, since $I$ is an additive subgroup of
$L$, their difference is also guaranteed to be in $I$.

A {\em right ideal} is a subset $J$ that is an abelian subgroup of $L$
under addition and such that $x * z \in J$ for all $x \in J$ and $z
\in L$. This is equivalent to requiring that $x *_y z \in J$ for all
$x \in J$ and $y,z \in L$.

A {\em two-sided ideal} is a subset that is both a left
ideal and a right ideal.

If $I$ is a two-sided ideal in a cring $L$, then $L/I$ has the
natural quotient cring structure. Many of the basic isomorphism
theorem results hold.

In particular, we note the following.

\begin{lemma}
  In any cring $L$, the center $Z(L)$ is a characteristic two-sided
  ideal.
\end{lemma}

\subsection{Subcrings and strong subcrings}

A {\em subcring} of a cring $L$ is a subset $S$ that is a
subgroup of $S$ under addition and that is also closed under $*$.

A {\em strong subcring} of a Lie ring $L$ is a subset $S$ that is a
subgroup of $S$ under addition and that is also closed under $*_y$ for
any $y \in L$.

Note that any left ideal or right ideal is a strong subcring.

\section{Alternating, skew-symmetric, and near-alternating crings}

\subsection{Definitions}

A cring $L$ is termed:

\begin{itemize}
\item {\em skew symmetric} if $(x * y) + (y * x) = 0$ for all $x,y \in
L$.
\item {\em near-alternating} if $x * y = 0$ whenever $x$ and $y$
  generate (additively) a cyclic subgroup of $L$. In particular $x *
  x$, and $x * (-x)$, are all $0$ for all $x \in L$.
\item {\em alternating} if it is both skew symmetric and
  near-alternating.
\end{itemize}

\subsection{Attempted generalization: didn't quite work}

Define a {\em $2$-form} in two variables $x$ and $y$ to be a formal
expression obtained by taking the sum of expressions of the form $(ax
+ by) * (cx + dy)$ where $a$, $b$, $c$ and $d$ are integers. Define a
{\em pointed $2$-form} as a $2$-form expressible as a sum of
expressions $(ax) * (bx)$, $(ay) * (by)$, $(ax) * (by)$, and $(ay) *
(bx)$, where $a$, $b$ are both integers. A pointed $2$-form is termed
symmetric if interchanging the variables $x$ and $y$ gives the same
pointed $2$-form.

A cring is alternating if and only if 
\subsection{Basic lemmas}

\begin{lemma}[Alternation of twisted product]
  If $L$ is a skew symmetric cring, then $z *_y x = -(x *_y z)$ for
  all $x,y,z \in L$. Also, as an easy corollary, $2\delta(x,y,z) = 0$.
\end{lemma}

\begin{proof}
  By the first interpretation (equation \ref{eq:starone}), we have

  \begin{equation*}
    z *_y x = (z * (y + x)) - (z * y)
  \end{equation*}

  We now use the commutativity of addition and skew symmetry to write:

  \begin{equation*}
    z *_y x = -[((x + y) * z) - (y * z)]
  \end{equation*}

  Now using the second interpretation (equation \ref{eq:startwo}) on
  the right side, we get the result. Note that switching between the
  interpretations corresponds to using the cocycle condition.
\end{proof}

\begin{lemma}[Shear property of product]\label{lemma:shear}
  In a near-alternating cring $L$, if $x + y + z = 0$, then $x * y = y
  * z = z * x$.
\end{lemma}

\begin{proof}
  We apply the cocycle condition on $x$, $y$, $z$:

  $$(x * (y + z)) + (y * z) = ((x + y) * z) + (x * y)$$

  Since $y + z = -x$ and $x + y = -z$, the near-alternating condition
  causes the products $x * (y + z)$ and $(x + y) * z$ to become
  $0$. We thus get:

  $$y * z = x * y$$

  Because of the cyclic symmetry in $x$, $y$, and $z$ of the
  statement, we can now deduce the full result
\end{proof}


\subsection{Ideals and centralizers in alternating crings}

In an alternating cring, the following additional features are observed:

\begin{itemize}
\item To check if $x \perp z$, it suffices to check that $x *_y z =
  0$; the other condition follows automatically.
\item Left ideals, right ideals, and two-sided ideals coincide.
\item It is still {\em not} necessarily true that every element
  centralizes itself. This is because even though $x * x = 0$ always,
  we do not have $x *_y x = 0$ for all $x$.
\end{itemize}

\section{Other interesting conditions on crings}

\subsection{Higher additivity}

A cring $L$ with multiplication $*$ is termed:

\begin{itemize}
\item {\em outer 3-additive} if it satisfies:

  \begin{equation}
    (x_1 + x_2) * (y * z) = (x_1 * (y * z)) + (x_2 * (y * z))
  \end{equation}

  This is equivalent to:

  \begin{equation}
    (x * y) * (z_1 + z_2) = ((x * y) * z_1) + ((x * y) * z_2)
  \end{equation}

  It is also equivalent to the assertion that any element obtained as
  a product of two elements is in $\Delta(L)$.
\item {\em inner 3-additive} if it satisfies:

  \begin{eqnarray*}
    x * ((y_1 + y_2) * z) & = & (x * (y_1 * z)) + (x * (y_2 * z))\\
    x * (y * (z_1 + z_2)) & = & (x * (y * z_1)) + (x * (y * z_2))\\
    ((x_1 + x_2) * y) * z & = & ((x_1 * y) * z) + ((x_2 * y) * z)\\
    (x * (y_1 + y_2)) * z & = & ((x * y_1) * z) + ((x * y_2) * z)
  \end{eqnarray*}
\item {\em 3-additive} if it is both outer and inner 3-additive.
\end{itemize}

When combined with the various shift identities, 3-additivity is a
powerful restriction. We summarize some obvious conclusions here:

\begin{lemma}
  The following hold for any outer 3-additive cring:

  \begin{enumerate}
  \item Any product of two elements of $L$ is also in $D_L(L)$.
  \item Suppose $L$ is alternating. Then, if $S$ is a two-sided ideal
    in $L$, so is $D_L(S)$.
  \item $\Delta(L)$ is a two-sided ideal in $L$.
  \end{enumerate}
\end{lemma}

\subsection{$3$-skew symmetry and flip-flop symmetry}

We say that a cring $L$ has:

\begin{itemize}
\item {\em inner $3$-skew symmetry} if we have:

  \begin{eqnarray*}
    x * (y * z) & = & -(x * (z * y))\\
    (x * y) * z & = & - ((y * x) * z)
  \end{eqnarray*}
\item {\em outer $3$-skew symmetry} if we have:
  
  $$x * (y * z) = -((y * z) * x)$$
\item {\em flip-flop symmetry} if we have:

  $$x * (y * z) = (z * y) * x$$
\end{itemize}

Any two of these three kinds of symmmetry together imply the third. If
all conditions hold, we say that the cring has {$3$-skew symmetry}.

Skew symmetry implies outer $3$-skew symmetry, and if combined with
$3$-additivity, also implies inner $3$-skew symmetry.

\section{Nilpotent and solvable crings, derived and central series}

Suppose $L$ is a cring. We say that $L$ is an {\em abelian cring} if
$x * y = 0$ for all $x,y \in L$. Note that an abelian cring is
completely determined by its additive group.

\subsection{Derived and lower central series}

The {\em derived subcring} of $L$ is defined as the additive subgroup
generated by all elements of the form $x * y$, where $x,y \in
L$. Further, we define the {\em derived series} and {\em lower central
series} as follows:

\begin{itemize}
\item The $k^{th}$ member of the derived series ($k \ge 0$) is defined
  inductively as the derived subcring of the $(k - 1)^{th}$
  member. The $0^{th}$ member is taken to be the whole cring.
\item The $k^{th}$ member of the {\em all-parenthesization lower
  central series} is defined as the additive subgroup generated by all
  products of length $k$ of elements in $L$, where the products could
  be parenthesized in any manner whatsoever. In particular, the first
  member of the lower central series is the whole cring, and the
  second member is the derived subcring.
\item We can also define the {\em inner-to-outer-parenthesized lower
  central series}, which is obtained by building parentheses from the
  inside to the outside, though each new term being multipled may be
  added on the left or the right.
\end{itemize}

We note that all members of the derived series and of the lower
central series are characteristic two-sided ideals. Further, we define:

\begin{itemize}
\item A cring $L$ is {\em nilpotent} (in the all-parenthesization
  sense) of class $c$ if any product (however parenthesized) of length
  $c + 1$ in $L$ is zero. A cring is nilpotent if it is nilpotent of
  class $c$ for some positive $c$. The {\em nilpotency class} of a
  nilpotent cring is the smallest $c$ such that the cring has class
  $c$.
\item A cring $L$ is {\em nilpotent} (in the inner-to-outer-parenthesization
  sense) of class $c$ if any inner-to-outer parenthesized product of length
  $c + 1$ in $L$ is zero. 

  Note that although the two definitions of nilpotency class given
  here differ, the underlying notions of nilpotency itself are the
  same. Specifically, if a cring has class $c$ in the
  all-parenthesization sense, it has class $c$ in the inner-to-outer
  parenthesization sense as well. Conversely, if it has class $c$ in
  the inner-to-outer parenthesization sense, then the class in the
  all-parenthesization sense is at most $2^c$.

  For Lie crings, not only do the two notions of nilpotency coincide,
  the two notions of nilpotency class also coincide. This is because
  of the alternating nature {\em and} the Jacobi identity.
\item A cring $L$ is {\em solvable} of length $l$ if the $l^{th}$
  member of its derived series is zero. A cring is solvable if it is
  solvable of length $l$ for some positive $l$. The {\em derived
  length} of a solvable cring is the smallest $l$ such that the cring
  has class $l$.
\end{itemize}

Note that for the definition of lower central series, we allow {\em
all possible parenthesizations}, rather than restricting to left
normed or right normed parenthesizations. For Lie crings, it is
possible to restrict attention to left normed parenthesizations.

\section{Lie crings: basics}

\subsection{Definition}

We define a {\em near-Lie cring} as a near-alternating cring that
satisfies the following two additional conditions, which are known as
the {\em Jacobi identities}:

\begin{eqnarray*}
  ((x * y) * z) + ((y * z)* x) + ((z * x) * y) & = & 0\\
  (x * (y * z)) + (y * (z * x)) + (z * (x * y)) & = & 0
\end{eqnarray*}

A {\em Lie cring} is a near-Lie cring that is also skew
symmetric. Note that under skew symmetry, and more generally, under
flip-flop symmetry, the two Jacobi identities are equivalent.

\subsection{Proving that things are ideals}

We prove some important lemmas.

\begin{lemma}
  If $S$ is an ideal in a $3$-additive Lie cring $L$, then
  $D_L(S)$ and $C_L(S)$ are ideals.
\end{lemma}

(Do we really need $3$-additivity? Probably not, but the proof becomes
a little worse).

\begin{proof}
  We've already proved that $D_L(S)$ is an additive subgroup. The fact
  that it is an ideal follows from the trivial observation that, by
  $3$-additivity, any expression of the form $x * y$ is in $D_L(L)$,
  and hence in $D_L(S)$.
  
  We've already proved that $C_L(S)$ is an additive subgroup. Thus, we
  only want to establish that $C_L(S)$ satisfies the additional
  condition: if $x \in C_L(S)$ and $y \in L$, then $x * y \in
  C_L(S)$. For this, we need to show that if $w \in L$, and $z \in S$,
  then $(x * y) *_w z = 0$. Because of $3$-additivity, it suffices to
  prove that $(x * y) * z = 0$. We note that by the Jacobi identity, we
  have:

  $$((x * y) * z) + ((y * z) * x) + ((z * x) * y)  =  0$$

  Since $z \in S$ and $x \in C_L(S)$, $z * x = 0$, so $((z * x) * y =
  0$. Also, since $S$ is an ideal, and $z \in S$, we have $y * z \in
  S$. Since $x \in C_L(S)$, we get $(y * z) * x = 0$. Thus, the second
  and thir terms in the Jacobi identity are zero, and we thus get $(x
  * y) * z= 0$, completing the proof.
\end{proof}

\section{Relationship with Lie ring theory}

\subsection{The double Lie ring of a Lie cring}

Given any near-Lie cring $L$, we can define a
corresponding Lie ring $\stackrel{\Skew}{L}$ by the operation:

$$[x,y] := (x * y) - (y * x)$$

In other words, the new operation is obtained as the skew of the old
operation. That this is a Lie ring basically follows from Theorem
\ref{thm:skewofcringisring}, which asserts that the skew of any cring
is an alternating ring.

In particular, if $L$ is a Lie cring, we define the {\em double Lie
ring} $\stackrel{2}{L}$ as the following Lie ring: the additive
structure is the same as that of $L$, and the multiplication is
defined as:

$$[x,y] := 2(x * y)$$

\subsection{Homologous crings}

Suppose $L$ is an abelian group and $*$ and $\cdot$ define two Lie
cring operations on $L$. We say that $*$ and $\cdot$ are {\em
homologous} if $* - \cdot$ is a coboundary, i.e., there exists a
function $q:L \to L$ such that, for all $x,y \in L$, we have:

$$q(x + y) - q(x) - q(y) = (x * y) - (x \cdot y)$$

In particular, a cring is {\em nullhomologous} if its multiplication
is a coboundary.

\subsection{Cring halves of a Lie ring}

We can reverse the question now as follows: given a Lie ring $L$, can
we define a Lie cring structure $*$ such that the double of $*$ is
the Lie bracket? Or more generally, can we find a near-Lie cring
structure $*$ whose skew is the Lie bracket? We have a uniqueness
theorem in this regard:

\begin{theorem}
  Two near-Lie cring structures on a finitely generated abelian group
  have the same skew if and only if they are homologous. %this needs
  %to be strengthened. In particular, what is key is how deeply they
  %are homologous, i.e., what is the image of the gauge function whose
  %coboundary they are?
\end{theorem}

The result probably also holds for arbitrary abelian groups, but the
construction of the homology is probably trickier, so we restrict to
the finitely generated case.

\begin{proof}
  By basic group theory, this is equivalent to the assertion that a
  Lie cring has skew zero if and only if it is nullhomologous. In
  other words, for a Lie cring $L$ with multiplication $*$, we want to
  show that:

  $$x * y = y * x \ \forall \ x,y \in L \iff \exists q:L \to L \text{ s.t. } x * y = q(x + y) - q(x) - q(y) \ \forall x,y \in L$$

  The $\Leftarrow$ direction: If a function $q$ exists with the desired
  properties, then we have:

  \begin{eqnarray*}
    x * y & = & q(x + y) - q(x) - q(y) \\
    y * x & = & q(y + x) - q(y) - q(x)
  \end{eqnarray*}

  The right sides are equal because addition is commutative, so we get
  $x * y = y * x$ for all $x,y \in L$.

  The $\implies$ direction: Here, we need to use the additional
  information that we have a near-Lie cring. In particular, we have
  the near-alternating condition: $x * y = 0$ whenever $x$ and $y$
  generate a cyclic subgroup of $L$. Also, we have the symmetry
  condition: $x * y = y * x$ for all $x,y \in L$.

  We consider an extension of $L$ by itself with respect to the
  $2$-cocycle $*$. We get an extension $M$ with base $L$ and quotient
  $L$. Showing that $*$ is nillhomologous is equivalent to showing
  that this extension is congruent to the direct product extension $L
  \times L$. We now use that $L$ is the direct product of finitely
  many cyclic subgroups. (Rest of the proof is basically a copy of the
  proof of the uniqueness theorem found in the baer2cocycles file,
  need to adjust notation).
\end{proof}

\section{Nilpotent Lie crings and their exponential groups}

\subsection{Class two near-Lie crings}

A {\em class two near-Lie cring} is a near-Lie cring $(L,*)$ of
nilpotency class two. In other words, it has the property that $(x *
y) * z = 0$ for all $x,y,z \in L$. 

Note first that a class two near-alternating cring is
automatically near-Lie, because the Jacobi identity provides no
additional information. Further, it is automatically $3$-additive.

\subsection{Exponential group of a near-Lie cring}

For a class two near-Lie cring $(L,*)$, we can define the corresponding
{\em exponential group} by the following multiplication $\cdot$:

$$x \cdot y = x + y + (x * y)$$

\begin{lemma}
  The operation $\cdot$ above defines a group structure on $L$, such
  that:

  \begin{enumerate}
  \item The identity and inverse operations are the same as for the
    additive group of $L$. In fact, on any cyclic subgroup of $L$,
    $\cdot$ coincides with $+$. Hence, the group is $1$-isomorphic to
    the additive group of $L$.
  \item The group has nilpotency class two. The commutator map in the
    group is the same as the Lie bracket in $\stackrel{Skew}{L}$,
    i.e., it is the skew of $*$.
  \end{enumerate}
\end{lemma}

\begin{proof}
  {\em Identity element}: We want to show that $x \cdot 0 = 0 \cdot x = 0$ for
  all $x$. We do this by computing explicitly:

  $$x \cdot 0 = x + 0 + (x * 0) = x + 0 + 0 = x$$

  Similarly, $0 \cdot x = x$.

  {\em Inverses and cyclic subgroups}: This is easy, because $x * y =
  0$ forces $x \cdot y = x + y$.

  {\em Associativity}: We finally need to prove that $x \cdot (y \cdot
  z) = (x \cdot y) \cdot z$. We first simplify the left side:

  $$x \cdot (y \cdot z) = x  + (y + z + (y * z)) + (x * (y + z + (y * z)))$$

  Now, note that since the cring has class two, $y * z$ is in the
  center. Thus, $x *_{y + z} (y * z) = 0$ (note: we're jugging between
  the two definitions of center here, which implicitly uses the
  cocycle condition). We thus obtain that $x * (y + z + (y * z)) = x *
  (y + z)$. Thus, we get:

  $$x \cdot (y \cdot z) = x + y + z + (y * z) + (x * (y + z))$$

  Similarly, we obtain that:

  $$(x \cdot y) \cdot z = x + y + z + (x * y) + ((x + y) * z)$$

  The cocycle condition on $*$ now tells us that the right sides are
  equal, and we are done.

  Next, we need to show that the commutator for the group operation is
  $(x * y) - (y * x)$. We do the calculation:

  $$[x,y] = (x \cdot y) \cdot (y \cdot x)^{-1} = (x \cdot y) + [- (y \cdot x)] + (x \cdot y) * (-(y \cdot x))$$

  Expanding each of the $\cdot$ expressions on the right, we get:

  $$[x,y] = (x + y + (x * y)) - (y + x + (y * x)) + ((x + y + (x * y)) * (-y - x - (y * x)))$$

  When we open parentheses, and note that the $x * y$ and $y * x$
  terms can be dropped since the terms are central, a lot of
  cancellation occurs and we are left with:

  $$[x,y] = (x * y) - (y * x)$$

\end{proof}

These facts are closely connected with the way we use cocycles and
second cohomology to classify group extensions, and we will explore
this connection later. The key idea will be to view $*$, not as a
cocycle from $L \times L$ to $L$, but as a coycle from the quotient of
$L$ by a central subgroup to that same central subgroup. This cocycle
is then closely related to the defining cocycle for the corresponding
extension group, with the difference of the two cocycles being a
cocycle defining the abelian group extension part.

\subsection{Uniqueness theorem in class two}

We currently have the following scenario in class two:

$$\text{Lie ring obtained by Skew} \leftarrow \text{near-Lie cring}
\rightarrow \text{Group obtained by exponential map}$$

We would like to reverse these arrows. The key result here is a
uniqueness theorem, which asserts that distinct near-Lie cring
structures on the same abelian group give rise to the same Lie ring
iff they are homologous. A similar result holds for the groups side:
distinct near-Lie cring structures on the same abelian group give
rise to isomorphic groups iff they are homologous.

Note that in the case that $L$ is uniquely $2$-divisible, any homology
class of near-Lie crings has a unique representative that is a {\em
Lie ring}, namely the representative we get by halving the skew of
that cohomology class. Thus, the uniqueness theorem goes a lot farther
in the uniquely 2-divisible case. In the $2$-torsion case, however:
(i) there exist cohomology classes of near-Lie crings that do not
contain any Lie cring, and (ii) those that do may contain multiple Lie
crings. Hence, the uniqueness theorem is only {\em up to} the
cohomology class relationship.

\subsection{Class three $3$-additive Lie crings and exponential group}

We understand what it means to be a class three Lie cring. In
particular, it means that any left-normed or right-normed (or
otherwise parenthesized) product of four elements is zero. For this,
we make the further assumption of $3$-additivity.

We now try to define an exponential group.

Our take-off point is the class three truncation of the
Baker-Campell-Hausdorff formula:

$$x \cdot y = x + y + \frac{1}{2}[x,y] + \frac{1}{12}[x,[x,y]] - \frac{1}{12}[y,[x,y]]$$

We want to replace the occurrence of $\frac{1}{2}[x,y]$ with the Lie
cring operation $x * y$. Thus, our proposed new formula is:

$$x \cdot y = x + y + (x * y) + \frac{1}{3}(x * (x * y)) - \frac{1}{3}(y * (x * y))$$

Note that this formula assumes that $L$ is uniquely $3$-divisible. In
general, the cring setup is ideally geeared to handling $2$-torsion,
so we assume $p$-divisibility for all other primes $p$. There are
other very similar generalizations that work to tackle the issue of
non-unique $p$-divisibility for higher $p$.

For simplicity, we define $t_3(x,y) = \frac{1}{3}(x * (x * y)) -
\frac{1}{3}(y * (x * y))$.

Question: Does this define an associative group operation? The answer
is {\em yes assuming 3-additivity}, but we need some computational
machinery in order to arrive at it.

\begin{lemma}[Distribution of product terms]\label{lemma:distprod}
  Assume $L$ is an outer $3$-additive Lie cring. If $p,q,r,s \in L$, then
  $(p + (q * r)) * s = (p * s) + ((q * r) * s)$. A similar identity
  holds on the right. In other words, we can use distributivity when
  one of the terms involved is a product.
\end{lemma}

\begin{proof}
  Applying the cocycle condition on $(q * r)$, $p$, and $s$, we see
  that:

  $$(((q * r) + p) * s) - (p * s) = ((q * r) * (p + s)) -  ((q * r) * p)$$

  By $3$-additivity, the right side becomes $(q * r) * s$, and hence
  we get the desired inequality.
\end{proof}

Based on this lemma, we note that the following hold in class three
(note that some terms become zero, hence are not written in the
expressions on the right side):

\begin{align*}
  (x + y + (x * y) + t_3(x,y)) * z & = ((x + y) * z) + ((x * y) * z) \tag{T1}\\
  t_3(x + y + (x * y) + t_3(x,y),z) & = t_3(x + y,z) \tag{T2}\\
  x * (y + z + (y * z) + t_3(y,z)) & = (x * (y + z)) + (x * (y * z)) \tag{T3}\\
  t_3(x,y + z + (y * z) + t_3(y,z)) & = t_3(x,y + z) \tag{T4}
\end{align*}

We now prove another minor computational lemma.

\begin{lemma}
  Suppose $L$ is a $3$-additive Lie cring. Consider $t_3$ as a
  $2$-cochain from $L$ to $L$. Then $\partial t_3$ is the function
  $(x,y,z) \mapsto (y * (z * x))$.
\end{lemma}

\begin{proof}
  The boundary of the function $(x,y) \mapsto x * (x * y)$ is $(x,y,z)
  \mapsto -((x * (y * z)) + (y * (x * z))$. The boundary of the
  function $(x,y) \mapsto y * (x * y)$ is $(x,y,z) \mapsto (y * (x *
  z)) + (z * (x * y))$. Combining these, using the Jacobi identity and
  skew symmetry, and dividing by $3$, gives the final result.
\end{proof}

We can now prove the main lemma.

\begin{lemma}\label{3additiveclass3expoisgroup}
  The operation $\cdot$ defined above gives a group operation on $L$,
  satisfying the following:

  \begin{enumerate}
  \item The identity element is the $0$ of $L$. The restriction to
    cyclic subgroups of $L$ coincides with the addition operation on
    $L$. In particular, the group is $1$-isomorphic to $L$.
  \item If $x * y = 0$ for a particular choice of $x,y \in L$, then
    $x$, $y$ commute in the group.
  \end{enumerate}
\end{lemma}

\begin{proof}
  (1) and (2) are direct; the main challenge is showing
  associativity. For this, we compute the expressions for $(x \cdot y)
  \cdot z$ and $x \cdot (y \cdot z)$.

  By using the equations (T1)-(T4) given above, we can obtain the following simplified expressions:
 
  \begin{align*}
    (x \cdot y) \cdot z & = x + y + z + (x * y) + ((x + y) * z) + ((x * y) * z) + t_3(x,y) + t_3(x + y,z) \tag{T5}\\
    x \cdot (y \cdot z) & = x + y + z + (x * (y + z)) + (y * z) + (x * (y * z)) + t_3(y,z) + t_3(x,y + z) \tag{T6}
  \end{align*}

  Subtracting (T6) - (T5) and using the cocycle condition on $*$ to
  cancel terms, we obtain that:

  $$(x \cdot (y \cdot z)) - ((x \cdot y) \cdot z) = (x * (y * z)) - ((x * y) * z) + (\partial t_3)(x,y,z)$$

  By the lemma calculating $\partial t_3$, we get:

  $$(x \cdot (y \cdot z)) - ((x \cdot y) \cdot z) = (x * (y * z)) - ((x * y) * z) + (y * (z * x))$$

  The Jacobi identity and skew symmetry now make the right side zero,
  completing the proof.
\end{proof}

We can generalize {\em somewhat} from $3$-additive Lie crings to
$3$-additive near-Lie crings that have $3$-skew symmetry:

\begin{lemma}
  The previous three lemmas hold with the weaker hypothesis that $L$
  is a $3$-additive near-Lie cring with $3$-skew symmetry.
\end{lemma}

\subsection{$3$-additive and higher class}

Let's now consider the situation of $3$-additive nilpotent Lie
crings of higher class. To construct the exponential group, here is
the approximate recipe that seems likely to work:

\begin{itemize}
\item Start with the Baker-Campbell-Hausdorff formula for that class.
\item Replace all the Lie bracket operations in that formula by the
  cring operation, and multiply the denominator for any $k$-fold
  product by $2^{k-1}$ (because that is the number of bracket pairings
  used).
\end{itemize}

The first thing we hope for is that in the Baker-Campbell-Hausdorff
formula, the largest power of $2$ dividing the denominator of the
coefficient of any $k$-fold product is at most $2^{k-1}$. This is
true, and is part of a more general statement for primes $p$. The
statement/observation probably originates with Lazard, and the author
learned of the proof from Anton Alekseev via email.

\begin{lemma}
  For a prime $p$ and a natural number $n$, denote by $f(p,n)$ the
  largest $k$ such that, if we truncate the Baker-Campbell-Hausdorff
  formula to terms that involve products of length at most $n$, then
  one or more of the denominators is divisible by $p^k$. Then, we have:

  $$f(p,n) \le \left[\frac{n - 1}{p - 1}\right]$$

  where $[]$ denotes the greatest integer function.
\end{lemma}

(Currently, the proof is copy-and-paste from Alekseev's email, it
needs to be tidied up with adequate explanations)

\begin{proof}
  Assign the $p$-adic valuation $1/(p-1)$ to formal variables $x$ and
  $y$. Then,

  $$v_p(x^n/n!)=n/(p-1) - v_p(n!) \geq 1/(p-1)$$

  where we have used that $v_p(n!) \leq (n-1)/(p-1)$. Then,
  $v_p(e^x-1)\geq 1/(p-1)$. Denote $w=e^xe^y-1$. It is easy to see
  that $v_p(w) \geq 1/(p-1)$. Consider

  $$ch(x,y)=\ln(e^xe^y)=\ln(1+w)$$

  Note that $v_p(w^n/n)=nv_p(w)-v_p(n)\geq n/(p-1) - v_p(n!) \geq
  1/(p-1)$. That is, $v_p(ch(x,y))\geq 1/(p-1)$. For coefficients in
  degree n, we obtain

  $$v_p(\text{coeff in degree n}) \geq 1/(p-1) - n/(p-1)=-(n-1)/(p-1)$$

\end{proof}

The next step is to argue that the formal manipulation needed to prove
that the Baker-Campbell-Hausdorff formula works continues to apply in
this situation. The crucial ingredient is lemma \ref{lemma:distprod},
which guarantees that we can distribute almost everything, with the
only things we cannot distribute being those where the inner terms are
all of degree $1$ (i.e., don't involve products). In other words, when
proving associativity, we are ultimately left with expressions of the
form $t_i(x,y)$, $t_i(x + y,z)$, $t_i(y,z)$, and $t_i(x + y,z)$. The
cocycle condition plus explicit computations of the boundaries of
$t_i$ now do our work for us.

{\em How can this be made formal?}

\subsection*{Crings: the tip of an iceberg}

We can think of cring theory as particularly useful for taking care
of $2$-torsion. A similar approach may be used to take care of
$p$-torsion for odd primes $p$. Here, however, we keep the Lie bracket
as an actual Lie bracket and instead provide new interpretations for
expressions where we have to divide an iterated Lie bracket by a
forbidden prime.

\section{Class two groups and Lie rings: a more detailed understanding}

We begin with the direction going from Lie rings to groups. In the
class two case, things are about the same level of difficulty both
ways. For higher class, it is somewhat easier to start from the Lie
ring side because the addition is commutative and is neatly separated
from the Lie bracket. 

Suppose $L$ is a Lie ring of nilpotency class two. We are interested
in finding a Lie cring, or near-Lie cring structure on $L$ whose
double (respectively, skew) is the Lie bracket. Recall that a Lie
cring structure whose double gives the Lie bracket means a skew
symmetric $2$-cocycle $*$ with the property that $x * y = 0$ for all
$x,y \in L$ that generate a cyclic subgroup, such that $*$ satisfies
the Jacobi identity and $2(x * y) = [x,y]$ for all $x,y \in L$. A
near-Lie cring structure whose skew gives the Lie bracket means a
$2$-cocycle $*$ with the property that $x * y = 0$ for all $x,y \in L$
that generate a cyclic subgroup, such that $*$ satisfies the Jacobi
identity and $(x * y) - (y * x) = [x,y]$ for all $x,y \in L$.

Now, note that there are three broad kinds of possibilities:

\begin{itemize}
\item We obtain a Lie cring structure, or near-Lie cring
  structure, that also has class two, i.e., $x * (y * z) = 0$ and $(x
  * y) * z = 0$ for all $x,y,z$. In this case, we can define a group
  multiplication by:

  $$x \cdot y := x + y + (x * y)$$

  The group thus defined has class two and the commutator map in this
  group coincides with the Lie bracket in the original Lie ring.

  Note that in some cases, we can even manage to get $*$ to give a Lie
  {\em ring} structure, which can be thought of as an added
  bonus. However, nothing in the proof requires a Lie ring structure
  on $*$.
\item We obtain a Lie cring structure that is nilpotent but has
  class more than two. For instance, it might have class three. In the
  class three case, we can define a group multiplication by:

  $$x \cdot y := x + y + (x * y) + (1/3)(x * (x * y)) - (1/3)(y * (x * y))$$

  Clearly, the group we thus obtain has class at most three. But does
  it have class two? Suspect so, but don't see an easy justification yet.
\item We obtain a Lie cring structure that is not nilpotent, and
  does not even satisfy an Engel condition. Thus, the
  Baker-Campbell-Hausdorff formula becomes useless. In these kinds of
  situations, there {\em is} a group lurking somewhere (hopefully of
  class two), but there is no element-to-element bijection between the
  group and the Lie ring.
\end{itemize}

\subsection{When the cring has class two}

We first consider the simplest and most interesting case: when the
cring also has class two. In our initial analysis, we deal with the
whole Lie ring/group/near-Lie cring at once. Then, we introduce the
notion of extensions.

Suppose we denote by $L$ the class two near-Lie cring, with $*$ the
cring multiplication, $[ \ , \ ]$ the Lie bracket obtained as its
skew, and $ \cdot$ the corresponding group multiplication. Explicitly,
we have:

\begin{eqnarray*}
  [x,y] & = & (x * y) - (y * x)\\
  x \cdot y & = & x + y + (x * y)
\end{eqnarray*}

Further, we have the identity $x * (y * z) = (x * y) * z = 0$ for all
$x,y,z \in L$, to indicate class two, as well as the zero identity,
cocycle identity, and the identity $x * y = 0$ whenever $\langle x,y
\rangle$ is cyclic.

We now have {\em two} notions of center and {\em two} notions of
derived series. Specifically:

\begin{enumerate}
\item The center as a {\em group} and as a {\em Lie ring} coincide. We
  denote this as $Z_{\text{Lie}}(L)$.
\item The center as a {\em cring}, while contained in this center,
  could potentially be smaller. We denote this as
  $Z_{\text{cring}}(L)$.
\item The derived subgroup as a group coincides with the derived
  subring as a Lie ring, and we denote this by $L'_{\text{Lie}}$.
\item The derived subcring contains this derived subgroup, but could
  be strictly larger. We denote this by $L'_{\text{cring}}$.
\end{enumerate}

The chain of containments for a class two cring is thus:

$$0 \le L'_{\text{Lie}} \le L'_{\text{cring}} \le Z_{\text{cring}}(L) \le Z_{\text{Lie}}(L) \le L$$

\subsection{Necessary and sufficient conditions}

In the special case that the near-Lie cring is actually a Lie cring, we have that:

$$L'_{\text{Lie}} = 2L'_{\text{cring}}$$

Here, the $2$ means that it comprises doubles (multiplicatively,
squares) of elements in the group on the right side.

In particular, a necessary condition for a class two group or Lie ring
$L$ to admit a Lie cring is that:

$$L'_{\text{Lie}} \le 2Z_{\text{Lie}}(L)$$

The converse is sort of true. Specifically, what's true is that if the
above condition holds, then we can always obtain a {\em near}-Lie
cring, but not necessarily a Lie cring.

\subsection{Class two: Lie cring, Lie ring, and near-Lie ring}

We now build intuition for the case of $2$-groups (note: all the
interesting behavior happens in the $2$-torsion part, so we can
concentrate on $2$-groups). We list key examples:

\begin{enumerate}
\item For order $8$, there are no examples of non-abelian groups
  arising as exponentials of class two Lie crings.
\item For order $16$, the groups $M_{16}$ and $D_8 * \Z_4$ arise as
  exponentials of class two Lie crings. The corresponding additive
  groups are $\Z_8 \times \Z_2$ and $\Z_4 \times V_4$
  respectively. These are the only cases of a $1$-isomorphism between
  a non-abelian group and an abelian group at that order.
\item For order $32$, there are six isomorphism classes non-abelian
  groups arising as exponentials of class two Lie crings. None of the
  Lie crings thus obtained is actually a Lie {\em ring}. There is one
  isomorphism class of non-abelian groups arising as the exponential
  of a near-Lie cring but inexpressible as the exponential of a Lie
  cring. Finally, there is one example of a class two group arising as
  the exponential of a class {\em three} Lie cring but not of any
  class two Lie cring -- a subject we'll study later.
\item For order $64$, there are numerous examples of all sorts,
  including a few cases where the base Lie cring is itself a Lie
  ring. More details to be filled in later.
\end{enumerate}

The case of the unique group of order $32$ that arises as the
exponential of a near-Lie cring but not a Lie cring is
interesting. This group has GAP ID (32,2). It is defined by the
presentation:

$$G := \langle a,b,c \mid a^2 = b^4 = c^4 = 1, ab = ba, ac = ca, cbc^{-1} = ab \rangle$$

The center is elementary abelian of order $8$ and the derived subgroup
has order $2$. The inner automorphism group is a Klein four-group and
the abelianization is isomorphic to $\Z_4 \times \Z_4$. The additive
group of the Lie cring is $\Z_4 \times \Z_4 \times \Z_2$. %The
%existence of the necessary $2$-cocycle seems to be an accident and has
%not been explained in other terms.

\subsection{The full theorem}

The following theorem describes a necessary and sufficient condition
for a group to arise as the exponential of a class two near-Lie cring.

\begin{theorem}
  \begin{enumerate}
  \item {\em Necessary and sufficient: Lie ring as skew of a class two
    near-Lie cring}: Suppose $L$ is a finite Lie ring of nilpotency class
    two. Then, $L$ arises as the skew of a class two near-Lie cring if
    and only if there is a subring $S$ of $L$ such that $[L,L] \le S
    \le Z(L)$, and the following holds for every $x,y \in L$: either
    $[x,y]$ is twice an element of $S$, or $dx \notin S$ and $dy
    \notin S$ where $d$ is the order of $[x,y]$.
  \item {\em Necessary and sufficient: Group as exponential of a class
    two near-Lie cring}: Suppose $G$ is a finite group of nilpotency
    class two. Then, $G$ arises as the exponential of a class two
    near-Lie cring if and only if there is a subgroup $H$ such that
    $[G,G] \le H \le Z(G)$, and the following holds for every $x,y \in
    G$: either $[x,y]$ is the square of an element in $H$, or $x^d
    \notin H$ and $y^d \notin H$ where $d$ is the order of $[x,y]$.
  \item {\em Necessary and sufficient: Lie ring as double of a class
    two Lie cring}: Suppose $L$ is a finite Lie ring of nilpotency class
    two. Then, $L$ arises as the double of a class two Lie cring if
    and only if there is a subring $S$ of $L$ such that $[L,L] \le S
    \le Z(L)$, and the following holds for every $x,y \in L$: either
    $[x,y]$ is twice an element of $S$ {\em and} the 2-torsion part of
    the order of $[x,y]$ is at most $4$, or $dx \notin H$ and $dy
    \notin H$ where $d$ is the order of $[x,y]$.
  \item {\em Necessary and sufficient: Group as exponential of a class
    two Lie cring}: Suppose $G$ is a finite group of nilpotency class
    two. Then, $G$ arises as the exponential of a class two near-Lie
    cring if and only if there is a subgroup $H$ such that $[G,G] \le
    H \le Z(G)$, and the following holds for every $x,y \in G$: either
    $[x,y]$ is the square of an element in $H$ {\and} the $2$-torsion
    part of the order of $[x,y]$ is divisible by $4$, or $x^d \notin
    H$ and $y^d \notin H$ where $d$ is the order of $[x,y]$.
  \item {\em Necessary and sufficient: Lie ring as double of a class
    two Lie ring}: Suppose $L$ is a finite Lie ring of nilpotency class
    two. Then, $L$ arises as the double of a class two Lie cring if
    and only if, for every $x,y \in L$, $[x,y]$ is twice an element of
    $Z(L)$ {\em and} (either the order of $x$ is relatively prime to
    $2$ or the order of $[x,y]$ divides half the order of $x$).
  \item {\em Necessary and sufficient: group as exponential of a class
    two Lie ring}: Suppose $G$ is a group of nilpotency class
    two. Then, $L$ arises as the exponential of a class two Lie ring
    if and only if, for every $x,y \in G$, $[x,y]$ is the square of
    $Z(L)$ {\em and} (either the order of $x$ is relatively prime to
    $2$ or the order of $[x,y]$ divides half the order of $x$).
  \end{enumerate}
\end{theorem}

Note that, for finite Lie rings of $2$-power order, a particular case
of (1) is where $[L,L]$ is contained in $\mho^1(Z(L))$, the set of
squares of elements of $Z(L)$. Similarly, for finite $2$-groups, a
particular case of $2$ is where $[G,G]$, is contained in $\mho^1(Z(G))$.

\subsection{Proof of theorems using iterated cocycle construction}

The heart of the proof of both the theorems lies in constructing a
cocycle that skews to a given alternating bilinear map. {\em This is
the proof involving the iterative construction where we first fill in
the elements of order two, then the elements of order four, wherein at
each stage we are solving a congruence equation that can be
solved. The details haven't been put in the document yet.}

\subsection{When the cring has higher class}

We next consider the case that the cring has higher class. Question:
If we start with a class two group and find a $3$-additive Lie cring
for it that has class three or higher, does doubling that give us a
class {\em two} Lie ring?

The answer is {\em yes}. However, unlike the class two {\em cring}
case, it is no longer necessary that the commutator and Lie bracket
coincide.

It's easier to understand the Lie ring to group version of this
question, which we now do. We restrict attention to Lie crings, rather
than the more general case of near-Lie crings.

Suppose $L$ is a $3$-additive class three Lie cring such that
$\stackrel{2}{L}$ is of class two. This translates to:

$$2((2(x * y)) * z) = 0 \ \forall x,y,z \in L$$

Note that by $3$-additivity, $x * y \in \Delta(L)$, so the inside can
be distributed, and we get an equivalent formulation:

$$4((x * y) * z) = 0 \ \forall \ x,y,z \in L$$

We now consider the group multiplication.

$$x \cdot y = x + y + (x * y) + \frac{1}{3}[(x * (x * y)) - (y * (x * y))]$$

We also get:

$$y \cdot x = y + x + (y * x) + \frac{1}{3}[(y * (y * x)) - (x * (y * x))]$$

Subtracting, we get:

$$(x \cdot y) - (y \cdot x) = [x,y]$$

Also, we have:

$$(x \cdot y) * (-(y \cdot x)) = 2(x * (x * y)) + 2(y * (x * y))$$

Now, the {\em group} commutator is:

$$(x \cdot y) \cdot (y \cdot x)^{-1} = (x \cdot y) \cdot (-(y \cdot x))$$

That simplifies to:

$$[x,y]_{\text{group}} = (x \cdot y) - (y \cdot x) + (x \cdot y) * (-(y \cdot x))$$
 
From the previous calculations, we get:

$$[x,y]_{\text{group}} = [x,y] +  2(x * (x * y)) + 2(y * (x * y))$$

It's now a fairly straightforward calculation to show that:

$$[[x,y]_{\text{group}},z]_{\text{group}} = 0$$

We now briefly sketch the qualitative reason why this proof should
continue to work in higher class. {\em Fill this in later}.

\section{Class three or higher and not $3$-additive}
\subsection{Conjectures on general formulations}

We make the following very general sketchy claim:

\begin{quote}
  Suppose $G$ is a finite $2$-group. Suppose we can find a descending series:

  $$G = G_1 \ge G_2 \ge G_3 \ge \dots \ge G_{n+1} = 1$$

  with the property that $[G_1,G_i] \le \mho^1(G_{i+1})$. We call such
  a series an {\em agemo-central series} for the prime $2$ of length $n$.

  Then, if there is an agemo-central series of length $n$, we should
  be able to find an intermediate structure (an enhanced cring of some
  sort) and a Lie ring on the other side.
\end{quote}

Note that this sketchy claim is true in a very precise and well
understood sense for the case $n = 2$. In fact, in that case, it
simply translates to the assumption that $[G,G] \le \mho^1(Z(G))$,
which, as we saw, is a sufficient condition for the existence of a
near-Lie cring.

\subsection{Tackling non-$3$-additivity in class three}

We've already noted that for the $3$-additive case, everything
proceeds smoothly for higher class, once we modify the
Baker-Campbell-Hausdorff formula appropriately. Unfortunately, if
we're interested in all the possible interesting examples of groups
covered by the general sketchy claim above, then it's not good enough
to restrict to the $3$-additive case. Specifically, there are examples
of groups where we get a Lie cring, or near-Lie cring, of class three,
{\em and that has a corresponding Lie ring, but the Lie cring is not
$3$-additive}.

The smallest such example is to be found with groups of order $64$. In
fact, the smallest examples that {\em violate} $3$-additivity are at
order $64$, whereas the smallest examples that {\em satisfy}
$3$-additivity (but are not cring class two) are to be found only at
order $256$.

To understand this situation, let's look more closely at the
exponential formula for class three:

$$x \cdot y = x + y + (x * y) + t_3(x,y)$$

where, assuming $3$-additivity, $3$-skew-symmetry, and the Jacobi
identity, we find that the following works:

$$t_3(x,y) = \frac{1}{3}(x * (x * y)) - \frac{1}{3}(y * (x * y))$$

In the absence of $3$-additivity (even when the other conditions of
$3$-skew-symmetry and Jacobi are met), {\em this explicit formula for
$t_3$ does not work, even though the formula makes sense}. However, we
are still open to the possibility of another {\em non-formulaic}
$t_3$, i.e., some function $t_3$ that manages to work without being
given by an explicit formula.

Specifically, we want $t_3$ to satisfy the following conditions:

\begin{itemize}
\item The {\em boundary condition}: $\partial t_3$ is the function
  $(x,y,z) \mapsto ((x * y) * z) - (x * (y * z)) + \delta(x + y, x *
  y,z) - \delta(x, y + z, y * z)$. In the special case that the cring
  is outer $3$-additive, both the $\delta$ (distributor) terms vanish
  and we are left with the boundary condition that $\partial t_3$ is
  the function $(x,y,z) \mapsto ((x * y) * z) - (x * (y * z))$.
\item The {\em class three conditions}: $t_3(x,y) * z = z * t_3(x,y) =
  0$ for all $x$, $y$, and $z$; $t_3(t_3(x,y),z) = t_3(x,t_3(y,z)) =
  0$ for all $x$, $y$, and $z$; and $t_3(x*y,z) = 0$ for all $x$, $y$,
  and $z$.
\item The {\em cyclicity-preserving condition}: If $x,y$ generate a
  cyclic subgroup of $L$, then $t_3(x,y) = 0$.
\end{itemize}

We now have the following analogue of
\ref{3additiveclass3expoisgroup}:

\begin{lemma}\label{3additivet3chosenexpoisgroup}
  Suppose $L$ is a class three (near-)Lie cring
  with cring operation $*$ and $t_3$ is a $2$-cochain $L \times L \to
  L$ satisfying the boundary condition, the class three conditions,
  and the faithfulness conditions. Define an operation $\cdot$ as
  follows:

  $$x \cdot y := x + y + (x * y) + t_3(x,y)$$

  The operation $\cdot$ defined above gives a group
  operation on $L$, satisfying the following:

  The identity element is the $0$ of $L$. The restriction to
  cyclic subgroups of $L$ coincides with the addition operation on
  $L$. In particular, the group is $1$-isomorphic to $L$.
\end{lemma}

We define an {\em enhanced class three near-Lie cring} as a class
three near-Lie cring equipped with a choice of $t_3$ that works in the
above sense. Note that any $3$-additive $3$-skew-symmetric near-Lie
cring has a natural enhanced structure using the formulaic $t_3$.

Note that we can in fact drop the Jacobi identity assumption as far as
this theorem is concerned, as long as we make sure that our definition
of {\em class three} includes {\em all products} of length four being
equal to zero, including products of the form $(x * y) * (z * w)$.

\subsection{Enhanced strongly central series for enhanced crings}

Given an enhanced class three near-Lie cring $L$, we can construct a filtration as follows:

$$L = L_1 \ge L_2 \ge L_3 \ge L_4 = 0 = L_n, n \ge 4$$

where all the $L_i$s are subcrings of $L$, and such that:

\begin{itemize}
\item $L_i * L_j$ is contained in $L_{i+j}$ for all $i,j \ge 1$.
\item $t_3(L_i,L_j)$ is contained in $L_{i + j + \min \{ i,j \}}$ for all $i,j \ge 1$.
\end{itemize}

A filtration of this kind is termed an {\em enhanced strongly central
series} for the enhanced cring.

Starting with an enhanced class three near-Lie cring $L$, we can
obtain the {\em enhanced lower central series} which is an example of
an enhanced strongly central series: define $L_1 = L$, $L_2 = L * L$,
and $L_3$ as the subring generated by $L_1 * L_2$, $L_2 * L_1$, and
$t_3(L_1,L_1)$. The class three condition on $*$, as well as the
various class three conditions imposed on $t_3$, now guarantee the
above list of conditions.

It may be more helpful to think of an enhanced class three near-Lie
cring {\em in terms of a specific choice of enhanced strongly central
series}.

We note the following about enhanced strongly central series:

\begin{lemma}
  \begin{enumerate}
  \item $*$ descends to an operation $L_1/L_3 \times L_1/L_3 \to L_2$,
    because $L_3$ is in the center.
  \item If $*$ is outer $3$-additive, then $t_3$ descends to an
  operation $L_1/L_2 \times L_1/L_2 \to L_3$.
  \end{enumerate}
\end{lemma}

\begin{proof}
  (1) is direct from the observation that $L_3$ is in the center for
  $*$, and from the two equivalent definitions of the center.

  For (2), we need to show two things:

  (a) If $z \in L_2$ and $x,y \in L$, then:

  $$t_3(x,y) = t_3(x,y + z)$$

  We show this as follows.

  $$(\partial t_3)(x,y,z) = ((x * y) * z) - (x * (y * z))$$

  Rearranging, we see that:

  $$t_3(x,y + z) - t_3(x,y) = ((x * y) * z) - (x * (y * z)) + t_3(x + y,z) - t_3(y,z)$$

  We argue that since $z \in L_2$, then the right side is zero. For
  this, note that the triple products are $0$ by the class three
  condition on $*$ and the $t_3$ terms are zero by the class three
  conditions on $t_3$.

  Thus, the right side is zero, hence so is the left side, and we get:

  $$t_3(x,y + z) = t_3(x,y)$$

  (b) If $x \in L_2$ and $y,z \in L$, then:

  $$t_3(x + y,z) = t_3(y,z)$$

  We show this as follows.

  $$(\partial t_3)(x,y,z) = (x * y) * z - x * (y * z)$$

  Rearranging, we see that:

  $$t_3(x + y,z) - t_3(y,z) = (x * y) * z - x * (y * z) + t_3(x,y) - t_3(x,y + z)$$

  Since $x \in L_2$, all terms on the right side are zero, hence the
  left side is zero, and we get:

  $$t_3(x + y,z) = t_3(y,z)$$
\end{proof}

\subsection{Enhanced near-Lie crings}

An enhanced near-Lie cring of nilpotency class $c$ is defined as an
abelian group $L$ with a bunch of operations $t_2,t_3,t_4,\dots,t_c: L
\times L \to L$, such that:

\begin{itemize}
\item $t_2$, denoted by an infix operator $*$, gives $L$ a near-Lie
  cring structure. In particular, $\partial t_2 = 0$, $t_2(x,y) = 0$
  when $\langle x,y \rangle$ is cyclic, and $t_2$ satisfies both the
  left and right Jacobi identities.
\item $t_3$ satisfies the boundary condition $\partial t_3(x,y,z) = y
  * (z * x)$.
\item $t_4$, $t_5$, etc. each satisfy a specific boundary condition
  where the boundary is expressed in terms of the lower $t_i$s.
\item We have degree conditions that guarantee that sufficiently
  iterated terms are zero. The basic idea being that something should
  be zero if actual replacement taking the terms as the BCH terms
  gives a sum of products, each of length at least $c + 1$.
\end{itemize}

If this is too constrained, we can relax the near-Lie condition to a
near-alternating condition by dropping the Jacobi identity assumptions
on $t_2$. We then get an {\em enhanced near-alternating cring} of
nilpotency class $c$.

It is helpful to think in terms of a filtration:

$$L = L_1 \ge L_2 \ge L_3 \ge \dots \ge L_c \ge L_{c+1} = 0$$

and reframe the degree conditions in terms of conditions of the form
$t_i(L_j,L_k) \le L_{f(i,j,k)}$ where $f$ is a specified function.

We can also consider the {\em residually nilpotent} case where there
is no finite nilpotency class but all products eventually become zero.
\end{document}
