\documentclass[10pt]{amsart}

%Packages in use
\usepackage{fullpage, hyperref, vipul, enumerate}

%Title details
\title{Diagnostic in-class quiz solutions: due Friday October 11: Gauss-Jordan elimination (originally due Wednesday October 9, but postponed)}
\author{Math 196, Section 57 (Vipul Naik)}
%List of new commands

\begin{document}
\maketitle

\section{Performance review}

28 people took this 6-question quiz. The score distribution was as follows:

\begin{itemize}
\item Score of 3: 3 people
\item Score of 4: 7 people
\item Score of 5: 12 people
\item Score of 6: 6 people
\end{itemize}

The mean score was 4.75.

The question-wise answers and performance review are as follows:

\begin{enumerate}
\item Option (B): 28 people (everybody!)
\item Option (B): 26 people
\item Option (A): 28 people (everybody!)
\item Option (B): 15 people
\item Option (C): 17 people
\item Option (D): 19 people
\end{enumerate}

{\em Note}: This quiz was not administered the last time I taught the
course, so there is no previous performance to compare against.
\section{Solutions}

{\bf PLEASE DO {\em NOT} DISCUSS ANY QUESTIONS}

The quiz covers basics related to Gauss-Jordan elimination (notes
titled {\tt Gauss-Jordan elimination}, corresponding section in the
book Section 1.2). Explicitly, the quiz covers:

\begin{itemize}
\item Setting up linear systems and interpreting the coefficient
  matrix in terms of the setup.
\item Knowledge of the permissible rules for manipulating linear
  systems.
\item Metacognition of the process of Gauss-Jordan elimination and its
  eventual result, the reduced row-echelon form, as well as the
  interpretation in terms of the solution set.
\end{itemize}

The questions are fairly easy questions if you understand the
material. But it's important that you be able to answer them,
otherwise what we study later will not make much sense.

\begin{enumerate}
\item {\em Do not discuss this!}: The row operations that we can
  perform on the augmented matrix of a linear system include adding or
  subtracting another row. However, they do not include multiplying
  another row. In other words, suppose we start with:

  $$\left[\begin{matrix} 1 & 2 & | & 5\\ 2 & 7 & \mid & 6 \\\end{matrix}\right]$$

  What we're not allowed to do is multiply row 2 by row 1 and obtain:

  $$\left[\begin{matrix} 1 & 2 & | & 5\\ 2 & 14 & \mid & 30 \\\end{matrix}\right]$$

  What's the most compelling reason for our not being allowed to
  perform this operation?
  \begin{enumerate}[(A)]
  \item The row operations arise from the corresponding operations on
    equations. For the ``multiplication of rows'' operation to be
    legitimate, it must correspond to multiplication of the
    corresponding equations, and multiplying equations is not a
    legitimate operation.
  \item The row operations arise from the corresponding operations on
    equations. However, the ``multiplication of rows'' operation does
    not correspond to any legitimate operation on equations. Note that
    it does not correspond to multiplying the equations, because that
    is not how multiplication of linear polynomials work (in fact, if
    we multiplied the equations, we would end up with an equation that
    is not linear).
  \end{enumerate}

  {\em Answer}: Option (B)

  {\em Explanation}: In point of fact, both (A) and (B) are true in
  different ways, but (B) is more compelling. Let's first understand
  why (B) is true. We will then turn to why (A) is (somewhat) true.

  Every augmented matrix hides behind it a linear system. Let's call
  the variables $x_1$ and $x_2$. The original linear system is:

  \begin{eqnarray*}
    x_1 + 2x_2 & = & 5\\
    2x_1 + 7x_2 & = & 6\\
  \end{eqnarray*}

  Now, if we multiply the two equations, we get:

  $$(x_1 + 2x_2)(2x_1 + 7x_2) = 30$$

  This simplifies to:

  $$2x_1^2 + 11x_1x_2 + 14x_2^2 = 30$$

  Note that this is quite different from the linear equation that is
  represented by the row obtained by multiplying the two rows. That
  equation is:

  $$2x_1 + 14x_2 = 30$$

  So (B) is the main reason why the operation makes no sense.

  It's worth noting that to a lesser extent, (A) is also an
  issue. Multiplying equations is legitimate: if two equations hold,
  their product also holds. However, replacing one of the equations by
  the product equation could lead to a potential loss of
  information. This loss of information occurs if the other equation
  being multiplied has both sides equal to zero (basically,
  multiplication by zero throws away information). Note that that
  problem does not occur with this linear system.

  {\em Performance review}: All 28 got this.

\item {\em Do not discuss this!}: Consider a model where the
  functional form is linear in the parameters (though not necessarily
  in the inputs). We can use (input, output) pairs to set up a system
  of linear equations in the parameters. Given enough such equations,
  we can determine the values of the parameters.

  What is the relation between the coefficient matrix and the
  parameters and (input, output) pairs?

  \begin{enumerate}[(A)]
  \item The columns of the coefficient matrix correspond to the
    (input, output) pairs and the rows correspond to the parameters.
  \item The rows of the coefficient matrix correspond to the
    (input, output) pairs and the columns correspond to the parameters.
  \end{enumerate}

  {\em Answer}: Option (B)

  {\em Explanation}: Each (input, output) pair gives an
  equation. Since the functional form is linear in the parameters, the
  equation is a linear equation. Each equation corresponds to a row of
  the augmented matrix (the input part affects the left side of the
  equation, and hence the coefficient matrix row, while the output
  part is the constant on the right side of the equation, i.e., the
  augmenting value).

  The parameters are the variables that we are trying to solve
  for. These thus correspond to the columns.

  {\em Performance review}: 26 out of 28 got this. 2 chose (A).

\item {\em Do not discuss this!}: Consider a model where the
  functional form is linear in the parameters (though not necessarily
  in the inputs). We can use (input, output) pairs to set up a system
  of linear equations in the parameters. Given enough such equations,
  we can determine the values of the parameters.

  What is the relation between the inputs, the outputs, the
  coefficient matrix, and the augmenting column?

  \begin{enumerate}[(A)]
  \item The inputs correspond to the coefficient matrix and the
    outputs correspond to the augmenting column. In other words,
    knowing the values of the inputs allows us to write down the
    coefficient matrix. Knowing the values of the outputs allows us to
    write down the augmenting column.
  \item The outputs correspond to the coefficient matrix and the
    inputs correspond to the augment column. In other words, knowing
    the values of the outputs allows us to write down the coefficient
    matrix. Knowing the values of the inputs allows us to write down
    the augmenting column.
  \end{enumerate}

  {\em Answer}: Option (A)

  {\em Explanation}: Read the explanation for the preceding question.

  {\em Performance review}: All 28 people got this.

\item {\em Do not discuss this!}: Consider the following rule to check
  for consistency using the augmented matrix: the system is
  inconsistent if and only if there is a zero row of the coefficient
  matrix with a nonzero value for that row in the augmenting
  column. In what sense does this rule work?

  \begin{enumerate}[(A)]
  \item The rule can be applied to the augmented matrix directly in
    both the {\em if} and the {\em only if} direction.
  \item The rule can be applied to the augmented matrix only in the
    {\em if} direction in general. In the {\em only if} direction, the
    rule can be applied to the augmented matrix {\em after} we have
    reduced the system to a situation where the coefficient matrix is
    in row-echelon form (note: it's not necessary to reach reduced
    row-echelon form).
  \item The rule can be applied to the augmented matrix only in the
    {\em only if} direction in general. In the {\em if} direction, the
    rule can be applied to the augmented matrix {\em after} we have
    reduced the system to a situation where the coefficient matrix is
    in row-echelon form (note: it's not necessary to reach reduced
    row-echelon form).
  \item The rule can be applied in either direction only {\em after}
    we have reduced the system to a situation where the coefficient
    matrix is in row-echelon form (note: it's not necessary to reach
    reduced row-echelon form).
  \end{enumerate}

  {\em Answer}: Option (B)

  {\em Explanation}: The rule obviously applies in the {\em if}
  direction: if there is any row with zeros in the coefficient matrix
  and a nonzero augmenting value, that equation has no
  solutions. Therefore, the system as a whole is inconsistent.

  It does not work in the {\em only if} direction in general. This is
  because the inconsistency may be more subtle: it may arise due to
  equations being inconsistent {\em when viewed together}, rather than
  any individual equation being inconsistent. For instance, consider:

  \begin{eqnarray*}
    x + y & = & 1 \\
    2x + 2y & = & 3 \\
  \end{eqnarray*}

  The augmented matrix is:

  $$\left[\begin{matrix} 1 & 1 & \mid & 1 \\ 2 & 2 & \mid & 3 \\\end{matrix}\right]$$

  Note that there is no zero row for the coefficient matrix. However,
  if we subtract twice the first row from the second row, we obtain:

  $$\left[\begin{matrix} 1 & 1 & \mid & 1 \\ 0 & 0 & \mid & 1 \\\end{matrix}\right]$$

  The coefficient matrix is now in rref, and we can now see that the
  system is inconsistent based on the second row (the coefficient
  matrix is all zeros, and the augmenting entry is nonzero).

  {\em Performance review}: 15 out of 28 got this. 6 chose (D), 3 each
  chose (A) and (C), and 1 left the question blank.

\item {\em Do not discuss this!}: Which of the following is {\em not} a
  possibility for the number of solutions to a system of simultaneous
  linear equations? Please see Options (D) and (E) before answering.

  \begin{enumerate}[(A)]
  \item $0$
  \item $1$
  \item $2$
  \item All of the above, i.e., none of them is a possibility
  \item None of the above, i.e., they are all possibilities
  \end{enumerate}

  {\em Answer}: Option (C)

  {\em Explanation}: $0$ is a possibility that occurs when the system
  is inconsistent. $1$ is a possibility that occurs when all the
  variables are leading variables and the system is consistent.

  For there to be more than one solution, there must be a non-leading
  variable. This variable serves as a parameter that can take
  arbitrary real values. Thus, if there is more than one solution,
  there must be infinitely many solutions.

  {\em Performance review}: 17 out of 28 got this. 9 chose (E), 1 each
  chose (B) and (D).
\item {\em Do not discuss this!}: Which of the following describes the
  situation for a consistent system of simultaneous linear equations?

  \begin{enumerate}[(A)]
  \item The leading variables are the parameters used to describe the
    general solution, and the number of leading variables equals the
    number of nonzero equations in the reduced row-echelon form (here
    nonzero equation makes an equation that does not have a zero row
    in the augmented matrix).
  \item The non-leading variables are the parameters used to describe the
    general solution, and the number of non-leading variables equals the
    number of nonzero equations in the reduced row-echelon form (here
    nonzero equation makes an equation that does not have a zero row
    in the augmented matrix).
  \item The leading variables are the parameters used to describe the
    general solution, and the number of leading variables equals the
    value (number of variables) - (number of nonzero equations in the
    reduced row-echelon form) (here nonzero equation makes an equation
    that does not have a zero row in the augmented matrix).
  \item The non-leading variables are the parameters used to describe
    the general solution, and the number of non-leading variables
    equals the value (number of variables) - (number of nonzero equations
    in the reduced row-echelon form) (here nonzero equation makes an
    equation that does not have a zero row in the augmented matrix).
  \end{enumerate}

  {\em Answer}: Option (D)

  {\em Explanation}: The non-leading variables serve as
  parameters. Also, the number of {\em leading variables} is the
  number of nonzero rows in the reduced row-echelon form. The number
  of non-leading variables is thus the total number of variables minus
  this quantity.

  Note that we are given that the system is consistent, so we do not
  have to worry about the solution set being empty.

  {\em Performance review}: 19 out of 28 got this. 3 each chose (A),
  (B) and (C).
\end{enumerate}

\end{document}
