\documentclass[10pt]{amsart}

%Packages in use
\usepackage{fullpage, hyperref, vipul, enumerate}

%Title details
\title{Diagnostic in-class quiz solutions: originally due Friday November 15, delayed to Wednesday November 20: Linear dependence, bases, and subspaces}
\author{Math 196, Section 57 (Vipul Naik)}
%List of new commands

\begin{document}
\maketitle

\section{Performance review}

25 people took this 3-question quiz. The score distribution was as follows:

\begin{itemize}
\item Score of 0: 1 person
\item Score of 1: 8 people
\item Score of 2: 13 people
\item Score of 3: 3 people
\end{itemize}

The question-wise answers and performance review were as follows:

\begin{enumerate}
\item Option (E): 21 people
\item Option (D): 14 people
\item Option (C): 8 people
\end{enumerate}

\section{Solutions}

{\bf PLEASE DO NOT DISCUSS {\em ANY} QUESTIONS}

The purpose of this quiz is to review some basic ideas from part of
the lecture notes titled {\tt Linear dependence, bases, and
  subspaces}. The corresponding sections of the book are Sections 3.2
and 3.3.

\begin{enumerate}
\item {\em Do not discuss this!}: Suppose $S$ is a finite nonempty set
  of vectors in $\R^n$, and $T$ is a nonempty subset of $S$. What can
  we say about $S$ and $T$?

  \begin{enumerate}[(A)]
  \item $S$ is linearly dependent if and only if $T$ is linearly
    dependent. $S$ is linearly independent if and only if $T$ is
    linearly independent.
  \item If $S$ is linearly dependent, then $T$ is linearly
    dependent. If $S$ is linearly independent, then $T$ is linearly
    independent. However, we cannot deduce anything about the linear
    dependence or independence of $S$ from the linear dependence or
    independence of $T$.
  \item If $T$ is linearly dependent, then $S$ is linearly
    dependent. If $T$ is linearly independent, then $S$ is linearly
    independent. However, we cannot deduce anything about the linear
    dependence or independence of $T$ from the linear dependence or
    independence of $S$.
  \item If $S$ is linearly dependent, then $T$ is linearly
    dependent. If $T$ is linearly independent, then $S$ is linearly
    independent. We cannot make either of the two other deductions.
  \item If $T$ is linearly dependent, then $S$ is linearly
    dependent. If $S$ is linearly independent, then $T$ is linearly
    independent. We cannot make either of the other two deductions.
  \end{enumerate}
    
  {\em Answer}: Option (E)

  {\em Explanation}: Any linear relation between the vectors in $T$ is
  also a linear relation between the vectors in $S$, because all the
  vectors in $T$ are vectors in $S$.

  The corect statement in the reverse direction is actually a
  logically equivalent statement, namely, the {\em
    contrapositive}. All it's saying is that if $S$ is {\em not}
  linearly dependent, then $T$ couldn't have been linearly dependent
  either, because if $T$ had been linearly dependent, then $S$ would
  have been too. Explicitly, given any implication $P \implies Q$, the
  implication (not $Q$) $\implies$ (not $P$), called the {\em
    contrapositive}, also holds. This is just a special case.

  For more details, see the lecture notes or textbook.

  {\em Performance review}: 21 out of 25 got this. 3 chose (B), 1 chose (D).

\item {\em Do not discuss this!}: Suppose $S$ is a finite set of
  vectors in $\R^n$. Consider the three statements: (i) $S$ is
  linearly independent, (ii) $S$ does not contain the zero vector,
  (iii) $S$ does not contain any two vectors that are scalar multiples
  of one another. Which of the following options best describes the
  relationship between these statements?

  \begin{enumerate}[(A)]
  \item (i) is equivalent to (ii), and both imply (iii), but the
    reverse implication does not hold.
  \item (i) is equivalent to (iii), and both imply (ii), but the
    reverse implication does not hold.
  \item (i) is equivalent to (ii) and (iii) combined.
  \item (i) implies both (ii) and (iii), but (ii) and (iii), even if
    combined, do not imply (i).
  \end{enumerate}

  {\em Answer}: Option (D)

  {\em Explanation}: If either (ii) or (iii) is violated, we can
  obtain a linear relation within $S$, making $S$ linearly
  dependent. Thus, the negation of (ii) or (iii) implies the negation
  of (i). Therefore, the contrapositive holds: (i) implies both (ii)
  and (iii).

  However, (ii) and (iii), even if combined, do not imply (i). This is
  because there can be linear relations between the vectors that
  involve more than two vectors at a time. For instance, consider the
  set of vectors $\vec{v}_1 = \left[\begin{matrix} 1 \\ 0
      \\\end{matrix}\right]$, $\vec{v}_2 = \left[\begin{matrix} 0 \\ 1
      \\\end{matrix}\right]$, and $\vec{v}_3 = \left[\begin{matrix} 1
      \\ 1 \\\end{matrix}\right]$. These three vectors are linearly
  dependent, because $\vec{v}_1 + \vec{v}_2 - \vec{v}_3 =
  \vec{0}$. But none of the vectors equals zero, and no two vectors
  are scalar multiples of each other.

  {\em Performance review}: 14 out of 25 got this. 6 chose (C), 5 chose (B).

\item {\em Do not discuss this!}: Suppose $V$ is a linear subspace of
  $\R^n$ for some $n$, and $W$ is a linear subspace of $V$. Assume
  also that $W \ne V$, i.e., $W$ is a {\em proper} subspace of
  $V$. Which of the following correctly describes the relationship
  between bases of $V$ and bases of $W$?

  \begin{enumerate}[(A)]
  \item Given a basis of $V$, we can find a subset of that basis that
    is a basis of $W$. Also, given a basis of $W$, we can find a set
    containing that basis that is a basis of $V$.
  \item Given a basis of $V$, we can find a subset of that basis that
    is a basis of $W$. However, given a basis of $W$, we may not
    necessarily be able to find a set containing that basis that is a
    basis of $V$.
  \item Given a basis of $V$, we may not necessarily be able to find a
    subset of that basis that is a basis of $W$. However, given a
    basis of $W$, we can find a set containing that basis that is a
    basis of $V$.
  \end{enumerate}

  {\em Answer}: Option (C)

  {\em Explanation}: For a counterexample for the part about getting a
  basis for $W$ from a basis for $V$, consider the case that $V =
  \R^2$ has the standard basis ($\vec{e}_1$ and $\vec{e}_2$) and $W$
  is the subspace spanned by $\left[ \begin{matrix} 1 \\ 1
      \\\end{matrix}\right]$. Clearly, there is no subset of the
  standard basis of $V$ that forms a basis for $W$. In fact, none of
  the standard basis vectors of $V$ is in $W$.

  The idea behind showing the other direction is: start with a basis
  of $W$. Then, keep adding vectors one by one, making sure that each
  new vector being added is not in the span of the vectors so far. We
  will eventually obtain a basis of $V$.

  {\em Performance review}: 8 out of 25 got this. 10 chose (A), 7
  chose (B).
\end{enumerate}
\end{document}
