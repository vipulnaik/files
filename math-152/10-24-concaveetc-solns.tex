\documentclass[10pt]{amsart}

%Packages in use
\usepackage{fullpage, hyperref, vipul, enumerate}

%Title details
\title{Class quiz solutions: October 24: Concave, inflections, tangents, cusps, asymptotes}
\author{Math 152, Section 55 (Vipul Naik)}
%List of new commands

\begin{document}
\maketitle

\section{Performance review}

$12$ students took the quiz. The score distribution was as follows:

\begin{itemize}
\item Score of $2$: $2$ people
\item Score of $3$: $3$ people
\item Score of $4$: $5$ people
\item Score of $5$: $2$ people
\end{itemize}

The mean score was $3.6$.

Here are the problem wise answers:

\begin{enumerate}
\item Option (C): $9$ people
\item Option (A): $5$ people
\item Option (C): $6$ people
\item Option (C): $11$ people
\item Option (B): $3$ people
\item Option (A): $6$ people
\item Option (E): $3$ people
\end{enumerate}

\section{Solutions}

\begin{enumerate}

\item Consider the function $f(x) := x^3(x - 1)^4(x - 2)^2$. Which of
  the following {\bf is true}?

  \begin{enumerate}[(A)]
  \item $0$, $1$, and $2$ are all critical points and all of them are
    points of local extrema.
  \item $0$, $1$, and $2$ are all critical points, but only $0$ is a
    point of local extremum.
  \item $0$, $1$, and $2$ are all critical points, but only $1$ and
    $2$ are points of local extrema.
  \item $0$, $1$, and $2$ are all critical points, and none of them is
    a point of local extremum.
  \item $1$ and $2$ are the only critical points.
  \end{enumerate}

  {\em Answer}: Option (C)

  {\em Quick explanation}: $0$, $1$, and $2$ are all roots of $f$ with
  multiplicity greater than one, hence they are also roots of the
  derivative. Moreover, their multiplicity in the derivative is one
  less than their multiplicity in the original function.

  To see which ones are local extrema, just think of the functions
  $x^3$, $(x - 1)^4$, and $(x - 2)^2$ as isolated functions.

  Alternatively, since all these are critical points where the
  function is also zero, we can, instead of using the derivative test,
  directly compute the sign of the function to the left and the right
  of each critical points. For those critical points where there is an
  even power, the sign of the original function is the same on both
  sides close to the point. For thos ecritical points where there is
  an odd power, the sign of the original function flips.

  {\em Full explanation}: Try it yourself! You need to calculate the
  first derivative, see where it is zero, etc.

  {\em Performance review}: $9$ out of $12$ got this correct. $2$
  chose (B), $1$ chose (D).

  {\em Historical note (last year)}: $11$ out of $15$ people got this
  correct, which is a good showing. Other choices were (B) (2), (A)
  (1), and (E) (1).

  However, many people did tedious derivative computations for this
  question. Please try to understand the intuition behind how this
  problem can be solved without computing derivatives.
\item Suppose $f$ and $g$ are continuously differentiable functions on
  $\R$. Suppose $f$ and $g$ are both concave up. Which of the
  following is {\bf always true}?

  \begin{enumerate}[(A)]
  \item $f + g$ is concave up.
  \item $f - g$ is concave up.
  \item $f \cdot g$ is concave up.
  \item $f \circ g$ is concave up.
  \item All of the above.
  \end{enumerate}

  {\em Answer}: Option (A)

  {\em Explanation}: The sum of two increasing functions is
  increasing. Hence, if $f'$ and $g'$ are both increasing, so is the
  sum $f' + g' = (f + g)'$. The other options are false. In fact, for
  any two functions that are concave up, both the differences $f - g$
  and $g - f$ cannot be concave up. As for products, consider the
  example of $f(x) = x^2$ and $g(x) = (x - 1)^2$, which are both
  concave up everywhere, but their product is not. As for composites,
  consider $f(x) = x^2$ and $g(x) = x^2 - 2x$. The composite is $(x^2
  - 2x)^2$, which is not concave up everywhere.

  {\em Performance review}: $5$ out of $12$ got this correct. $3$
  chose (D), $2$ each chose (C) and (E).

{\em Historical note (last year)}: $8$ out of $15$ people got this
  correct. Other choices were (E) (5), (C) (1), and (D) (1).

\item Consider the function $p(x) := x(x-1) \dots (x - n)$, where $n
  \ge 1$ is a positive integer. How many points of inflection does $p$
  have?

  \begin{enumerate}[(A)]
  \item $n - 3$
  \item $n - 2$
  \item $n - 1$
  \item $n$
  \item $n + 1$
  \end{enumerate}

  {\em Answer}: Option (C)

  {\em Explanation}: $p$ has degree $n + 1$, since it is the product
  of $n + 1$ linear polynomials. Thus, $p''$ has degree $n - 1$, and
  hence can have at most $n - 1$ roots. Thus, the number of points of
  inflection of $p$ is at most $n - 1$. If we can locate $n -1$ points
  of inflection, we will be done.

  Note that since $p$ has zeros at $0,1,2,\dots,n$. By the extreme
  value theorem, there exists a local extreme value for $p$ between
  any two consecutive zeros, and this gives a root of $p'$. By degree
  considerations, there must be exactly one root in each interval $(i
  - 1,i)$. There are thus $n$ distinct roots of $p'$, each giving a
  local extreme value of $p$, located in the intervals $(0,1), (1,2),
  \dots, (n - 1,n)$. Call these roots $a_1, a_2, \dots, a_n$. Then
  $a_1 < a_2 < \dots < a_n$. Applying the extreme value theorem again
  on the intervals $[a_{i-1},a_i]$, we see that there is at least one
  local extremum for $p'$, and hence a zero for $p''$, in that
  interval. Since $p''$ has degree $n - 1$, there must be exactly one
  local extremum on each interval. Since local extrema of the
  derivative correspond to points of inflection, we have found $n - 1$
  distinct points of inflection, and we are done.

  {\em Performance review}: $6$ out of $12$ got this correct. $3$ each
  chose (B) and (E).

  {\em Historical note (last year)}: $7$ out of $15$ people got this
  correct. Other choices were (B) (3), (D) (3), (A) (1), and (E) (1).
  Those people who chose (D) typically made the error of doing $(n +
  1) - 1$ instead of $(n + 1) - 2$.
\item Suppose $f$ is a polynomial function of degree $n \ge 2$. What
  can you say about the sense of concavity of the function $f$ for
  {\bf large enough inputs}, i.e., as $x \to +\infty$? (Note that if
  $n \le 1$, $f$ is linear so we do not have concavity in either
  sense).

  \begin{enumerate}[(A)]
  \item $f$ is eventually concave up.
  \item $f$ is eventually concave down.
  \item $f$ is eventually either concave up or concave down, and which
    of these cases occurs depends on the sign of the leading
    coefficient of $f$.
  \item $f$ is eventually either concave up or concave down, and which
    of these cases occurs depends on whether the degree of $f$ is even
    or odd.
  \item $f$ may be concave up, concave down, or neither.
  \end{enumerate}

  {\em Apologies, the language of option (E) was confusing. Although
  the question did say ``eventually'', option (E) should ideally have
  read ``$f$ may be eventually concave up, concave down, or neither''
  rather than forcing you to infer this from the context.}

  {\em Answer}: Option (C).

  {\em Explanation}: Note first that the sign of the leading
  coefficient of $f''$ is the same as the sign of the leading
  coefficient of $f$, because the leading coefficient gets multiplied
  by $n(n-1)$, which is positive for $n \ge 2$. If this sign is
  positive, then $f''(x) \to \infty$ as $x \to \infty$, and hence must
  be eventually positive, forcing $f$ to be eventually concave up. If
  this sign is negative, then $f''(x) \to -\infty$ as $x \to \infty$,
  and hence must be eventually negative, forcing $f$ to be eventually
  concave down.

  {\em Performance review}: $11$ out of $12$ got this correct. $1$
  chose (D).

  {\em Historical note (last year)}: $12$ out of $15$ people got this
  correct. $2$ people chose option (E). This is probably partly
  because of the confusing language, see the apology above.

\item Suppose $f$ is a continuously differentiable function on $[a,b]$
  and $f'$ is continuously differentiable at all points of $[a,b]$
  except an interior point $c$, where it has a vertical cusp. What can
  we say is {\bf definitely true} about the behavior of $f$ at $c$?

  \begin{enumerate}[(A)]
  \item $f$ attains a local extreme value at $c$.
  \item $f$ has a point of inflection at $c$.
  \item $f$ has a critical point at $c$ that does not correspond to a
    local extreme value.
  \item $f$ has a vertical tangent at $c$.
  \item $f$ has a vertical cusp at $c$.
  \end{enumerate}

  {\em Answer}: Option (B)

  {\em Explanation}: A cusp for $f'$ means that $f'$ changes direction
  at $c$ (either from increasing to decreasing or from decreasing to
  increasing). This in turn means that the sense of concavity of $f$
  changes at $c$. Hence, $f$ has a point of inflection at $c$. Note
  that we cannot have the vertical tangent situation because $f'$ is
  continuous and finite at $c$.

  An example of such a function $f$ would be $f(x) := x^{5/3}$. The
  first derivative $f'(x) = (5/3)x^{2/3}$ is everywhere defined and
  has a vertical cusp at $c = 0$. We note that the derivative switches
  from decreasing to increasing at $c = 0$, so the original function
  switches from concave down to concave up.
  
  {\em Performance review}: $3$ out of $12$ got this correct. $6$
  chose (C), $2$ chose (A), $1$ chose (E).

  {\em Historical note (last year)}: $3$ out of $15$ people got this
  correct. Other options were (A) (4), (E) (3), (D) (3), and (C) (2).
  It seems that a lot of people did not make a clear enough
  distinction between the roles of $f$ and $f'$.

  {\em Action point}: This is the kind of question that is hard at
  first but at some stage should feel obvious to you. (Not immediately
  obvious, since it still requires you to read carefully, but the kind
  of thing that would not confuse you).

\item Suppose $f$ and $g$ are continuous functions on $\R$, such that
  $f$ attains a vertical tangent at $a$ and is continuously
  differentiable everywhere else, and $g$ attains a vertical tangent
  at $b$ and is continuously differentiable everywhere else. Further,
  $a \ne b$. What can we say is {\bf definitely true} about $f - g$?

  \begin{enumerate}[(A)]
  \item $f - g$ has vertical tangents at $a$ and $b$.
  \item $f - g$ has a vertical tangent at $a$ and a vertical cusp at $b$.
  \item $f - g$ has a vertical cusp at $a$ and a vertical tangent at $b$.
  \item $f - g$ has no vertical tangents and no vertical cusps.
  \item $f - g$ has either a vertical tangent or a vertical cusp at
    the points $a$ and $b$, but it is not possible to be more specific
    without further information.
  \end{enumerate}

  {\em Answer}: Option (A)

  {\em Explanation}: Note that $\lim_{x \to b} (f - g)'(x) = f'(b) -
  \lim_{x \to b}g'(x)$, which is an infinity of the sign opposite to
  that of $g'$. In particular, we hav a vertical tangent at
  $b$. Similarly, we have a vertical tangent at $a$.

  {\em Performance review}: $6$ out of $12$ got this correct. $5$
  chose (E), $1$ chose (B).

  {\em Historical note (last year)}: $5$ out of $15$ people got this
  correct. Other choices were (E) (5), (C) (3), (B) (1), (D) (1).  The
  main source of confusion here seems to have been that people did not
  realize if $f$ has a vertical tangent at $c$, so does $-f$, because
  the whole picture flips over.
\item Suppose $f$ and $g$ are continuous functions on $\R$, such that
  $f$ is continuously differentiable everywhere and $g$ is
  continuously differentiable everywhere except at $c$, where it has a
  vertical tangent. What can we say is {\bf definitely true} about $f
  \circ g$?

  \begin{enumerate}[(A)]
  \item It has a vertical tangent at $c$.
  \item It has a vertical cusp at $c$.
  \item It has either a vertical tangent or a vertical cusp at $c$.
  \item It has neither a vertical tangent nor a vertical cusp at $c$.
  \item We cannot say anything for certain.
  \end{enumerate}

  {\em Answer}: Option (E).

  {\em Explanation}: Consider $g(x) := x^{1/3}$. This has a vertical
  tangent at $c = 0$. If we choose $f(x) = x$, we get (A). If we
  choose $f(x) = x^2$, we get (B). If we choose $f(x) = x^3$, we get
  neither a vertical tangent nor a vertical cusp. Hence, (E) is the
  only viable option.

  {\em Performance review}: $3$ out of $12$ got this correct. $4$
  chose (C), $2$ each chose (A) and (D), $1$ chose (B).

  {\em Historical note (last year)}: $3$ out of $15$ people got this
  correct. Other choices were (A) (7), (C) (4), and (D) (1). The main
  thing that people had trouble with was thinking of possibilities for
  $f$ that could play the role of converting the vertical tangent
  behavior of the original function $g$ into vertical cusp or
  ``neither'' behavior for the composite function.

  {\em Action point}: This is a devilishly tricky question that I
  don't expect you to get at all in your first try, but that I expect
  that you will remember for the rest of your life (or at least this
  course) after you've seen it.
\end{enumerate}

\end{document}