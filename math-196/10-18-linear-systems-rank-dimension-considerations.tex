\documentclass[10pt]{amsart}

%Packages in use
\usepackage{fullpage, hyperref, vipul, enumerate}

%Title details
\title{Take-home class quiz: due Friday October 18: Linear systems: rank and dimension considerations}
\author{Math 196, Section 57 (Vipul Naik)}
%List of new commands

\begin{document}
\maketitle

Your name (print clearly in capital letters): $\underline{\qquad\qquad\qquad\qquad\qquad\qquad\qquad\qquad\qquad\qquad}$

{\bf PLEASE DO {\em NOT} DISCUSS ANY QUESTIONS EXCEPT THE STARRED OR DOUBLE-STARRED QUESTIONS.}

The questions here consider a wide range of theoretical and practical
settings where linear systems appear, and prompt you to think about
the notion of rank and its relationship with whether we can uniquely
acquire the information that we want. It relates approximately with
the material in the {\tt Linear systems and matrix algebra} notes (the
corresponding section in the book is Section 1.3).

\begin{enumerate}
\item (*) Let $m$ and $n$ be positive integers. It turns out that {\em
  almost all} $m \times n$ matrices over the real numbers have a
  particular rank. What is that rank? (Unfortunately, it is beyond our
  current scope to define ``almost all'').

  \begin{enumerate}[(A)]
  \item $m$ (regardless of whether $m$ or $n$ is bigger)
  \item $n$ (regardless of whether $m$ or $n$ is bigger)
  \item $(m + n)/2$
  \item $\max \{ m,n \}$
  \item $\min \{ m,n \}$
  \end{enumerate}

  \vspace{0.1in}
  Your answer: $\underline{\qquad\qquad\qquad\qquad\qquad\qquad\qquad}$
  \vspace{0.1in}


\item (*) A container has a mix of two known gases that do not react
  with each other. The temperature and pressure of the container are
  known. Assume that $PV = nRT$. The volume of the container is also
  known, and so is the total mass of the gases in the container. Under
  what conditions can we predict the amount (say, in the form of the
  number of moles) of each gas that is present from this information?

  \begin{enumerate}[(A)]
  \item It is possible if both gases have the same molecular mass,
    because in that case, the coefficient matrix of the linear system
    has full rank $2$.
  \item It is possible if both gases have different molecular
    masses, because in that case, the coefficient matrix of the linear
    system has full rank $2$.
  \item It is possible if both gases have the same molecular mass,
    because in that case, the coefficient matrix of the linear system
    has rank $1$.
  \item It is possible if both gases have different molecular
    masses, because in that case, the coefficient matrix of the linear
    system has rank $1$.
  \item It is not possible to deduce the amount of each gas from the
    given information.
  \end{enumerate}

  \vspace{0.1in}
  Your answer: $\underline{\qquad\qquad\qquad\qquad\qquad\qquad\qquad}$
  \vspace{0.1in}

\item (*) A container has a mix of three known gases with no reactions
  between the gases. The temperature and pressure of the container are
  known. Assume that $PV = nRT$. The volume of the container is also
  known, and so is the total mass of the gases in the container. Under
  what conditions can we predict the amount (say, in the form of the
  number of moles) of each gas that is present from this information?

  \begin{enumerate}[(A)]
  \item It is possible if all three gases have the same molecular mass,
    because in that case, the coefficient matrix of the linear system
    has full rank $3$.
  \item It is possible if all three gases have different molecular
    masses, because in that case, the coefficient matrix of the linear
    system has full rank $3$.
  \item It is possible if all three gases have the same molecular mass,
    because in that case, the coefficient matrix of the linear system
    has rank $2$.
  \item It is possible if all three gases have different molecular
    masses, because in that case, the coefficient matrix of the linear
    system has rank $2$.
  \item It is not possible to deduce the amount of each gas from the
    given information.
  \end{enumerate}

  \vspace{0.1in}
  Your answer: $\underline{\qquad\qquad\qquad\qquad\qquad\qquad\qquad}$
  \vspace{0.1in}

The branch of chemistry called quantitative analysis has historically
used stoichiometric methods to determine the proportions of various
chemicals present in a given mix. The idea is to use information about
the amounts needed and produced in various reactions to estimate the
quantities of chemicals present (the possible chemicals are first
identified via ``qualitative analysis'' techniques). We generally find
that these conditions give linear systems, and the coefficient
matrices of these systems have (or can be written in a manner as to
have) small integer entries. 

\item (*) Consider a situation where we have a material that is a mix
  (in fixed proportion) of three known chemicals $X$, $Y$, and
  $Z$. Our goal is to find the amount of $X$, $Y$, and $Z$
  present. Suppose we want to set up a collection of experiments so
  that the coefficient matrix is diagonal, i.e., we are effectively
  solving a diagonal system of equations and can recover the
  quantities of each of $X$, $Y$, and $Z$. Which of the following is
  the best approach?  Assume that we can measure, for each reagent,
  the amount of the reagent that gets used up for the reaction(s) to
  proceed to completion, but cannot isolate or separate the outputs
  from each other.

  \begin{enumerate}[(A)]
  \item Choose a single reagent that reacts with all of $X$, $Y$, and
    $Z$.
  \item Choose a single reagent that reacts with only one of $X$, $Y$,
    and $Z$.
  \item Choose three separate reagents, each of which reacts with {\em
    all} of $X$, $Y$, and $Z$.
  \item Choose three separate reagents, each of which reacts only with
    $X$.
  \item Choose three separate reagents, one of which reacts only with
    $X$, one of which reacts only with $Y$, and one of which reacts
    only with $Z$.
  \end{enumerate}

  \vspace{0.1in}
  Your answer: $\underline{\qquad\qquad\qquad\qquad\qquad\qquad\qquad}$
  \vspace{0.1in}

\item (*) Suppose we are given an aqueous solution with two known
  dissolved substances. There are two different types of
  reactions. One is an acid-base reaction and the other is a redox
  reaction. For both reactions, we can use titrations (separately) to
  deduce the quantity of reagent needed. What type of system should we
  expect to get if only one of the solutes participates in the redox
  reaction but both participate in the acid-base reaction?

  \begin{enumerate}[(A)]
  \item A diagonal system, i.e., the coefficient matrix is a diagonal matrix.
  \item A triangular system, i.e., the coefficient matrix is a
    triangular matrix (whether it is upper or lower triangular depends
    on the order in which we write the rows).
  \item A system of rank one, i.e., the coefficient matrix has rank one.
  \end{enumerate}

  \vspace{0.1in}
  Your answer: $\underline{\qquad\qquad\qquad\qquad\qquad\qquad\qquad}$
  \vspace{0.1in}

\item (*) Suppose we are given an aqueous solution with two known
  dissolved substances. There are two different types of
  reactions. One is an acid-base reaction and the other is a redox
  reaction. For both reactions, we can use titrations (separately) to
  deduce the quantity of reagent needed. Suppose we are given an
  aqueous solution with two known dissolved substances. Suppose both
  solutes participate in both reactions. What should we desire if we
  want to use the data from the two titrations to determine the
  amounts of each of the substances?

  \begin{enumerate}[(A)]
  \item The proportions in which the two substances react should be
    the same for the two reactions.
  \item The proportions in which the two substances react should
    differ for the two reactions.
  \item It does not matter; we will be able to determine the amounts
    of each of the substances in both cases.
  \item It does not matter; we will not be able to determine the
    amounts of each of the substances in either case.
  \end{enumerate}

  \vspace{0.1in}
  Your answer: $\underline{\qquad\qquad\qquad\qquad\qquad\qquad\qquad}$
  \vspace{0.1in}

\item {\em Do not discuss this!}: A consumer price index is obtained
  from a ``goods basket'' by multiplying the price of each good in the
  basket by a fixed weight, and then adding up all the price X weight
  products. The weights are kept fixed, but the prices vary from year
  to year. Thus, the consumer price index value itself fluctuates from
  year to year.
  
  What is a good way of modeling this?

  \begin{enumerate}[(A)]
  \item The prices of the various goods in various years are stored
    in a matrix, the weights used in the index are stored in a
    vector, and the consumer price index values arise as the output vector
    of the matrix-vector product.
  \item The weights used in the index are stored in a matrix, the
    prices of the various goods in various years are stored in a
    vector, and the consumer price index values arise as the output
    vector of the matrix-vector product.
  \item The prices of the various goods in various years are stored in
    a matrix, the consumer price index values are stored as a vector,
    and the weights used in the index arise as the output vector
    of the matrix-vector product.
  \item The weights used in the index are stored in a matrix, the
    consumer price index values are stored in a vector, and the prices
    of the various goods in various years arise as the output vector
    of the matrix-vector product.
  \item The consumer price index values are stored in a matrix, the
    prices of the various goods in various years are stored in a
    vector, and the weights used in the index arise as the output
    vector of the matrix-vector product.
  \end{enumerate}

  \vspace{0.1in}
  Your answer: $\underline{\qquad\qquad\qquad\qquad\qquad\qquad\qquad}$
  \vspace{0.1in}

\item {\em Do not discuss this!}: Amelia wants to choose a healthy
  balanced diet. She has access to $30$ different types of
  foods. There are $400$ different nutrients that she wants a good
  amount of. Each of the foods that Amelia consumes offers a positive
  amount of each nutrient per unit foodstuff. Amelia is interested in
  meeting the daily value requirements for all nutrients. For some
  nutrients, her daily value requirements specify only a minimum. For
  some nutrients, both a minimum and a maximum are specified. Assume
  that the total amount of any nutrient can be obtained by adding up
  the amounts obtained from each of the foodstuffs Amelia
  consumes. Amelia wants to determine how much of each foodstuff she
  should consume. How should she model the situation?

  \begin{enumerate}[(A)]
  \item The matrix with information on the nutritional contents of the
    foodstuffs is a $400 \times 400$ matrix, and the vector of amounts
    of each foodstuff consumed is a $400 \times 1$ column vector.
  \item The matrix with information on the nutritional contents of the
    foodstuffs is a $30 \times 30$ matrix, and the vector of amounts
    of each foodstuff consumed is a $30 \times 1$ column vector.
  \item The matrix with information on the nutritional contents of the
    foodstuffs is a $400 \times 30$ matrix, and the vector of amounts
    of each foodstuff consumed is a $30 \times 1$ column vector.
  \item The matrix with information on the nutritional contents of the
    foodstuffs is a $30 \times 400$ matrix, and the vector of amounts
    of each foodstuff consumed is a $400 \times 1$ column vector.
  \item The matrix with information on the nutritional contents of the
    foodstuffs is a $400 \times 400$ matrix, and the vector of amounts
    of each foodstuff consumed is a $30 \times 1$ column vector.
  \end{enumerate}

  \vspace{0.1in}
  Your answer: $\underline{\qquad\qquad\qquad\qquad\qquad\qquad\qquad}$
  \vspace{0.1in}

\end{enumerate}
\end{document}
