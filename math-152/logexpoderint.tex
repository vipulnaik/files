\documentclass[10pt]{amsart}
\usepackage{fullpage,hyperref,vipul,graphicx}
\title{Logarithm, exponential, derivative, and integral}
\author{Math 152, Section 55 (Vipul Naik)}

\begin{document}
\maketitle

{\bf Corresponding material in the book}: Section 7.2, 7.3, 7.4.

{\bf What students should definitely get}: The definition of logarithm
as an integral, its key properties. The differentiation and
integration formulas for logarithm and exponential, the key ideas
behind combining these with the chain rule and $u$-substitution to
carry out other integrals.

\section*{Executive summary}

\subsection{Logarithm and exponential: basics}

\begin{enumerate}
\item The {\em natural logarithm} is a one-to-one function with domain
  $(0,\infty)$ and range $\R$, and is defined as $\ln(x) := \int_1^x
  (dt/t)$.
\item The natural logarithm is an increasing function that is concave
  down. It satisfies the identities $\ln(1) = 0$, $\ln(ab) = \ln(a) +
  \ln(b)$, $\ln(a^r) = r\ln a$, and $\ln(1/a) = -\ln a$.
\item The limit $\lim_{x \to 0} \ln(x)$ is $-\infty$ and the limit
  $\lim_{x \to \infty} \ln(x)$ is $+\infty$. Note that $\ln$ goes off
  to $+\infty$ at $\infty$ even though its derivative goes to zero as
  $x \to + \infty$.
\item The derivative of $\ln(x)$ is $1/x$ and the derivative
  \underline{of} $\ln(kx)$ is also $1/x$. The derivative of $\ln(x^r)$
  is $r/x$.
\item The antiderivative of $1/x$ is $\ln |x| + C$. What this really
  means is that the antiderivative is $\ln(-x) + C$ when $x$ is
  negative and $\ln(x) + C$ when $x$ is positive. If we consider $1/x$
  on both positive and negative reals, the constant on the negative
  side is unrelated to the constant on the positive side.
\item $e$ is defined as the unique number $x$ such that $\ln(x) =
  1$. $e$ is approximately $2.718$. In particular, it is between $2$ and
  $3$.
\item The inverse of the natural logarithm function is denoted $\exp$,
  and $\exp(x)$ is also written as $e^x$. When $x$ is a rational
  number, $e^x = e^x$ (i.e., the two definitions of exponentiation
  coincide). In particular, $e^1 = e$, $e^0 = 1$, etc.
\item The function $\exp$ equals its own derivative and hence also its
  own antiderivative. Further, the derivative of $x \mapsto e^{mx}$ is
  $me^{mx}$. Similarly, the integral of $e^{mx}$ is $(1/m)e^{mx} +C$.
\item We have $\exp(x + y) = \exp(x)\exp(y)$, $\exp(rx) =
  (\exp(x))^r$, $\exp(0) = 1$, and $\exp(-x) = 1/\exp(x)$. All of
  these follow from the corresponding identities for $\ln$.
\end{enumerate}

Actions...

\begin{enumerate}
\item We can calculate $\ln(x)$ for given $x$ by using the usual
  methods of estimating the values of integrals, applied to the
  function $1/x$. We can also use the known properties of logarithms,
  as well as approximate $\ln$ values for some specific $x$ values, to
  estimate $\ln x$ to a reasonable approximation. For this, it helps
  to remember $\ln 2$, $\ln 3$, and $\ln 5$ or $\ln 10$.
\item Since both $\ln$ and $\exp$ are one-to-one, we can {\em cancel}
  $\ln$ from both sides of an equation and similarly {\em cancel}
  $\exp$. Technically, we cancel $\ln$ by applying $\exp$ to both
  sides, and we cancel $\exp$ by applying $\ln$ to both sides.
\end{enumerate}

\subsection{Integrations involving logarithms and exponents}

Words/actions ...

\begin{enumerate}
\item If the numerator is the derivative of the denominator, the
  integral is the logarithm of the (absolute value of) the
  denominator. In symbols, $\int g'(x)/g(x) \, dx = \ln|g(x)| + C$.
\item More generally, whenever we see an expression of the form
  $g'(x)/g(x)$ inside the integrand, we should consider the
  substitution $u = \ln |g(x)|$. Thus, $\int f(\ln|g(x)|)g'(x)/g(x) \,
  dx = \int f(u) \, du$ where $u = \ln|g(x)|$.
\item $\int f(e^x) e^x \, dx = \int f(u) \, du$ where $u = e^x$.
\item $\int e^x[f(x) + f'(x)] \, dx = e^x f(x) + C$.
\item $\int e^{f(x)} f'(x) \, dx = e^{f(x)} + C$.
\item Trigonometric integrals: $\int \tan x \, dx = -\ln|\cos x| + C$,
  and similar integration formulas for $\cot$, $\sec$ and $\csc$:
  $\int \cot x \, dx = \ln|\sin x| + C$, $\int \sec x = \ln|\sec x +
  \tan x| + C$, and $\int \csc x \, dx = \ln |\csc x - \cot x| + C$.
\end{enumerate}


\section{Logarithms: the adventure begins}

\subsection{Finding an antiderivative of the reciprocal function}

Recall that the process of differentiation never gave us fundamentally
new functions, because the derivatives of all the basic functions that
we knew were expressible in terms of other basic functions, and using
the operations of pointwise combination and composition did not allow
us to break ground into new functions. The situation differs somewhat
for integration. We have seen that we often come across functions for
which we have no clue as to how to find an antiderivative. We now
discuss how to handle one such function.

This function is the function $1/x$, which, for now, we will assume to
be a function on $(0,\infty)$. We want to find an antiderivative for
this function.

The basic results of integration tell us that one way of defining an
antiderivative is by using a definite integral from a fixed value to
$x$, as long as that fixed value is in the domain. For reasons that
are not immediately obvious, we choose the fixed value (the reference
point) as $1$. We thus define the following function:

$$L(x) := \int_1^x \frac{dt}{t}$$

Note that this is the {\em unique} antiderivative which has the
property that its value at $1$ is $0$. By definition, $L'(x) = 1/x$
for all $x$. What further information can we derive about $L$?

\subsection{Using the multiplicative transform}

By the $u$-substitution method, we can readily verify that, for $a,b >
0$:

$$\int_1^a \frac{dt}{t} = \int_b^{ab} \frac{dt}{t}$$

The key thing that is special about $1/x$ is that the multiplicative
factor on the $dt$ part cancels the multiplicative factor on the $t$
part.

This gives us that:

$$L(a) - L(1) = L(ab) - L(b)$$

Since $L(1) = 0$, we obtain that $L$ is a function satisfying the property:

$$L(ab) = L(a) + L(b) \ \forall \ a,b > 0$$

Thus, even though we do not have an explicit description of $L$, we
know that $L$ converts products to sums. In particular, we also see,
for instance, that:

$$L(a^n) = nL(a) \ \forall a > 0, n \in \mathbb{Z}$$

In particular, $L(1/a) = -L(a)$.

We can further see that for any rational number $r$, we have:

$$L(a^r) = rL(a) \ \forall a > 0, r \in \mathbb{Q}$$

In other words, the function $L$ converts products to sums and pulls
the exponent into a multiple. We also know that since $L'(x) > 0$ for
all $x > 0$, $L$ is continuous and increasing. In particular, we see
that $L$ is a one-to-one map on $(0,\infty)$.

What is the range of $L$? Consider $a = 2$. Then, $L(a) = L(2) >
0$. As $n \to \infty$, $L(a^n) = nL(a) \to \infty$, and as $n \to
-\infty$, $L(a^n) = nL(a) \to -\infty$. Since $L$ is increasing, we
can use this to see that $\lim_{x \to \infty} L(x) = \infty$ and
$\lim_{x \to 0} L(x) = -\infty$. Further, by the intermediate value
theorem, we see that the range of $L$ is $\R$. 

The upshot: $L$ is a continuous increasing one-to-one function from
$(0,\infty)$ to $\R$ that sends $1$ to $0$ and converts products to
sums.

\subsection{L for (natural) logarithm}

The function $L$ that is described above is termed the {\em natural
logarithm} function. It is ubiquitous in mathematics, and is denoted
$\ln$. Thus, we have the definition:

$$\ln x := \int_1^x \frac{dt}{t} \ \forall \ x > 0$$

It turns out that this natural logarithm behaves in ways very similar
to logarithms to base $10$. A quick primer for those who didn't live
in prehistoric times: in the olden days, when people had to do
multiplications by hand, they used a tool called {\em logarithm
tables} to do these multiplications. The logarithm tables basically
converted the multiplication problem to an addition problem.

Here is the principle on which the logarithm tables worked. These
tables allowed you to, for a given number $x$, find the approximate
value of $r$ such that $10^r = x$. This value of $r$ is called
$\log_{10}x$. Then, if you had to multiply $x$ and $y$, you first
found $\log_{10}x$ and $\log_{10}y$. It turns out that $\log_{10}(xy)
= \log_{10}(x) + \log_{10}y$, because if $10^r = x$ and $10^s = y$,
then $10^{r + s} = xy$ by properties of exponents. Thus, to find $xy$,
we find $\log_{10}x$ and $\log_{10}(y)$ and add them. Then, there are
antilogarithm tables, that allow us to find the antilogarithm of this
sum that we have computed (or basically, raise $10$ to the power of
that number).

The principle of logarithm tables was later converted to a {\em
mechanical device} called the {\em slide rule}. How many people have
used slide rules? What a slide rule does is use a {\em logarithmic
scale}, i.e., it places numbers on a scale in such a way that the
distance between the positions of two numbers is determined by their
quotient. So, on a logarithmic scale, the distance between $1$ and
$10$ is the same as the distance between $10$ and $100$, and also the
same as the distance between $0.01$ and $0.1$. The distance between
$3$ and $7$ is the same as the distance between $30$ and $70$. (If
you're interested in pictures of slide rules, do a Google image
search. I haven't included any picture here because of potential
copyright considerations).

A slide rule comprises two logarithmic scales (using the same
calibration) but one of them can slide against each other. We can use
the sliding scale to add lengths along the scale, but since the scale
is logarithmic, this ends up multiplying the numbers. You may have
heard about how people with an abacus can often do simple calculations
faster than people with a calculator. It turns out that people with a
slide rule can usually do multiplications faster than people with a
calculator.\footnote{On the other hand, the calculator is a lot more
versatile than the slide rule, and is probably faster for computing
roots, multiplying long sequences of numbers, or combinations of
multiplication and addition.}

Logarithmic scales are used in many measurements. Here are some
examples:

\begin{enumerate}

\item The {\bf Richter magnitude scale} measures the intensity of
  earthquakes. It is calibrated logarithmically to base $10$. An
  earthquake one point higher on the Richter scale is ten times as
  intense. 
\item The {\bf pH scale} in chemistry is a logarithmic scale to
  measure the concentration of the $H^+$ (more precisely, $H_3O^+$)
  ions. It is a negative logarithmic scale to base $10$. An increase
  in the pH value by $1$ corresponds to a decrease in the hydronium
  ion concentration to $1/10$ of its original value.
\item The {\bf decibel scale}, used for sound levels and other level
  measurements, is a logarithmic scale where an increase in $10$
  points along the scale corresponds to a ten-fold increase in
  amplitude. Thus, $20 dB$ is ten times as loud as $10 dB$.

\end{enumerate}

The natural logarithm function can be thought of as creating a
logarithmic scale on the positive reals. But the question we are
concerned with is: what precisely is this scale? How does this compare
with the usual logarithm to base $10$? Our hunch is that there should
be a number $e$ such that $\ln(x)$ is the value $r$ such that $x =
e^r$. What must this number $e$ be?

\section{The back and forth of things: logarithm and exponential}

\subsection{In search of $e$}

If such a number $e$ exists, then it must be the unique number $x$
satisfying $\ln (x) = 1$. Further, since $\ln$ is an increasing
function, we can try locating $e$ between two consecutive integers by
determining $\ln 2$, $\ln 3$, and so on. Actually, we can be more
clever.

We can begin by trying to compute $\ln 2$. We could do this using
upper and lower sums. We could also do it by noting that $\ln x =
\int_1^x dt/t$, and must be located between the antiderivatives of
$x^{-1/2}$ and $x^{-3/2}$. We did this approximation a few weeks ago
and found that $\ln 2$ is located between $0.58$ and $0.83$. Further
approximations using either this method or upper and lower sums for
partitions yields than $\ln 2$ is between $0.69$ and $0.70$. We will
assume $\ln 2 \approx 0.7$ for calculations.

Now that we know $\ln 2$, we do not need to do any more messy work
with upper and lower sums. We know that $2 < 2\sqrt{2} < 3$. We also
have that $\ln(2\sqrt{2}) = (3/2) \ln 2 \approx 1.05$, which is bigger
than $1$. Thus, $\ln 3 > 1$, so $2 < e < 3$. In fact, $2 < e <
2\sqrt{2} \approx 2.83$. We can also calculate, for instance, that the
cuberoot of $2$ is about $1.26$. Thus, $\ln(2.52)$ is approximately
$(4/3)(\ln 2)$ which is approximately $0.93$. Thus, we get that $e$
should be bigger than $2.52$. A similar process of successive
approximations yields that the value of $e$ is approximately
$2.718281828$. You should know $e$ to at least three decimal
places: $2.718$.

Clearly, since $\ln(e) = 1$, we have that $\ln(e^{p/q}) = p/q$ for
integers $p$ and $q$, with $q \ne 0$. It is not clear a priori what we
would mean by the notation $e^r$ for irrational numbers $r$, but
whatever we may mean, it should be the case that $\ln(e^r) = r$. In
other words, the function $x \mapsto e^x$ must be the inverse function
of the one-to-one function $\ln$. The function $x \mapsto e^x$ is also
called the exponentiation function, and sometimes denoted $\exp$. By
the way, the letter $e$ could be thought of as standing for {\em
exponentiation}, but historically it is believed to be named after
Leonhard Euler, a prolific mathematician who studied the properties
both of the number $e$ and the exponentiation function.

\subsection{Logarithms and exponents: some rules}

Here are some of the rules:

\begin{enumerate}
\item $\ln$ is a one-to-one function from $(0,\infty)$ to $\R$ and
  $\exp$ is a one-to-one function from $\R$ to $(0,\infty)$. The two
  functions are inverse functions of each other.
\item $\ln$ converts products to sums and $\exp$ converts sums to
  products. In other words, $\ln(xy) = \ln(x) + \ln(y)$ and $\exp(x +
  y) = \exp(x)\exp(y)$.
\item $\ln(1) = 0$ and $\exp(0) = 1$.
\item $\ln(1/x) = - \ln x$ and $\exp(-x) = 1/\exp(x)$.
\item $\ln(x^r) = r \ln x$ and $\exp(rx) = (\exp(x))^r$.
\item Both $\exp$ and $\ln$ are continuous and increasing functions.
\item $\exp$ has the $x$-axis as a horizontal asymptote as $x \to
  -\infty$, while $\ln$ has the $y$-axis as a vertical asymptote as $x
  \to 0$.
\item $\exp$ is concave up and $\ln$ is concave down.
\end{enumerate}

Here are the graphs:

The logarithm graph, zoomed in for small inputs:

\includegraphics[width=4in]{loggraphforsmallinputs.png}

The logarithm graph, zoomed out:

\includegraphics[width=4in]{loggraphforlargeinputs.png}

The exponential graph:

\includegraphics[width=4in]{expgraph.png}

Here are the logarithm and exponential graphs together, so that we can
see that the graphs are reflections of each other about the $y = x$ line:

\includegraphics[width=2in]{logandexptogether.png}
\subsection{Numerical shennanigans}

In his autobiographical book, Richard Feynmann discusses how he
impressed a bunch of mathematicians by being able to calculate natural
logarithms of many numbers to one or two decimal places. However, what
he did was hardly impressive. It turns out that remembering the
natural logarithms of a few numbers allows us to compute them
approximately for many numbers.

For instance, it is useful to remember that $\ln(2) \approx 0.6931$,
$\ln(3) \approx 1.0986$, $\ln(10) \approx 2.3026$, and $\ln(7) \approx
1.9459$. We can now calculate the logarithm values for most
integers. How? Using the fact that logarithms translate multiplication
to addition. Thus, $\ln(4) = 2\ln(2) \approx 1.3862$, while $\ln(5) =
\ln(10) - \ln(2) \approx 1.6095$. In fact, we can readily calculate
the natural logarithm of any positive integer all of whose prime
factors are among $2$, $3$, $5$, and $7$. What about $\ln(11)$? While
we cannot calculate this precisely, we can calculate $\ln(10)$
and $\ln(12)$ and thus obtain reasonable upper and lower bounds for
$\ln(11)$. Even better, we know that $\ln(120) < 2 \ln(11) <
\ln(125)$, and since we can calculate both $\ln(120)$ and $\ln(125)$,
we get a pretty small range for $\ln(11)$.

In fact, Feynman was able to impress physicists by doing calculations
that essentially relied on only two facts: $\ln(2) \approx 0.7$ and
$\ln(10) \approx 2.3$.

If we are able to quickly calculate natural logarithms, a happy
corollary of that is that we can quickly integrate $dx/x$ on intervals.

\subsection{Domain and range issues}

For domain computations in the past, we used the following basic
guidelines:

\begin{enumerate}
\item Things in the denominator must be nonzero.
\item Things with squareroots or even roots must be nonnegative.
\item Things with squareroots or even roots in the denominator must be
positive.\\
\vspace{0.2in}
We now add two more criteria:
\vspace{0.2in}
\item Things under logarithm must be positive.
\item Things under logarithm of the absolute value must be nonzero.
\end{enumerate}

\subsection{Logarithm of the absolute value}

The natural logarithm function is defined only for positive
reals. However, we can extend it to a function on all reals by taking
the absolute value first, i.e., we look at the function $x \mapsto
\ln(|x|)$. This is an even function and its graph is obtained by
takingthe graph of the logarithm function and adding its mirror image
about the $y$-axis

It turns out that the derivative of $\ln(|x|)$ is $1/x$. In
particular, we see that $\ln(|x|)$ serves as an antiderivative of
$1/x$ for {\em all nonzero $x$}. This is an improvement on $\ln(x)$,
which worked only for positive $x$. However, we should be careful
because the domain of $1/x$ as well as of $\ln|x|$ excludes
zero. Hence, the behavior on the positive and negative side are
totally independent of each other. We shall return to this point in a
later lecure.

\section{Formulas for derivatives and integrals}

\subsection{Derivative and integral formulas for logarithms}

The main formula that we have, which follows from our definition of
natural logarithm, is the following:

$$\frac{d}{dx} (\ln x) = \frac{1}{x}$$

This is the formula for $x > 0$. A more general version, for $x \ne
0$, is:

$$\frac{d}{dx} (\ln |x|) = \frac{1}{x}$$

The corresponding antiderivative formula for $x > 0$ is:

$$\int \frac{dx}{x} = \ln(x) + C$$

In general, the antiderivative formula is:

$$\int \frac{dx}{x} = \ln(|x|) + C$$

However, it should be remembered that this formula is valid only when
we are working with $x$ either in $(0,\infty)$ or in $(-\infty,0)$,
i.e., we cannot use the formula to cross between the interval
$(0,\infty)$ and $(-\infty,0)$. In fact, if we have a function $f: \R
\setminus \{ 0 \} \to \R$ such that $f'(x) = 1/x$, then we can
guarantee that $f(x) - \ln(|x|)$ is constant on $x > 0$ and is
constant on $x < 0$. However, these constants may differ. The behavior
on the $(0,\infty)$ connected component does not in any way constrain
the behavior on the $(-\infty,0)$ connected component.

\subsection{The exponential and its derivative and integral formulas}

Recall that $\exp$ is the inverse of the $\ln$ function. Thus, we can
use the rule for differentiating the inverse function to find $\exp'$. We have:

$$\exp'(x) = \frac{1}{\ln'(\exp x)} = \frac{1}{\frac{1}{\exp(x)}} = \exp(x)$$

Thus, we have the remarkable property that the exponential function,
i.e., the function $x \mapsto e^x$, is its own derivative. Another way
of thinking about this is that the rate of growth of the exponential
function is {\em equal} to its value. Note that this also implies that
the exponential function is infinitely differentiable and all higher
derivatives equal the same function.

We rewrite the above in Liebniz notation:

$$\frac{d}{dx}(e^x) = e^x$$

We also note the corresponding statement for indefinite integration:

$$\int e^x \, dx = e^x + C$$

\subsection{Some corollaries}

Using the above, we obtain the following identities for the logarithm
and exponent:

\begin{eqnarray*}
  \frac{d}{dx} (\ln (kx)) & = & \frac{1}{x}\\
  \frac{d}{dx} (\ln (x^r)) & = & \frac{r}{x}\\
  \frac{d}{dx} (e^{mx}) & = & me^{mx}\\
  \int e^{mx} \, dx & = & \frac{1}{m}e^{mx} + C\\
\end{eqnarray*}

Each of these identities can be derived in two ways: either by using
the properties of logarithms and exponents on the inside and then
differentiating, or by first differentiating and then simplifying. For
instance, for the second identity, we can either simplify $\ln(x^r)$
as $r \ln(x)$ first and then pull the constant $r$ out before
differentiating, or we can use the chain rule to obtain $(1/x^r) \cdot
rx^{r-1}$. It is gratifying to know that the answers we obtain both
ways are the same.

\section{Application to indefinite and definite integration}

\subsection{The $u$-substitution: a textbook example}

We begin with an easy example:

$$\int_{\sqrt{n}}^n \frac{1}{x \ln x} \, dx$$

Here, $n$ is an integer greater than $1$.

Believe it or not, this integral actually came up in some asymptotic
approximations I was doing some time ago to figure out whether some
numbers have large prime divisors! Let us first look at the indefinite
integral. The substitution $u = \ln x$ gives us:

$$\int \frac{1}{x \ln x} = \int \frac{du}{u} = \ln(u) = \ln(\ln x) + C$$

Note that we do not need to put absolute values here because on the
interval of integration, $\ln$ is positive. Now, we can evaluate
between limits:

$$\left[ \ln(\ln x) \right]_{\sqrt{n}}^n = \ln(\ln n) - \ln (\ln \sqrt{n}) = \ln \left[\frac{\ln n}{\ln \sqrt{n}}\right] = \ln \left[\frac{\ln n}{(1/2)\ln n}\right] = \ln 2$$

So, the answer is $\ln 2$, which, as we computed earlier, is
approximate $0.693$. Apparently, this is the rough heuristic argument
for why about $69.3 \%$ of the numbers have a prime divisor greater
than their squareroot.

\subsection{Numerator as derivative of denominator}

The gist of this logarithmic substitution can be captured by the formula:

$$\int \frac{g'(x)}{g(x)} \, dx = \ln|g(x)| + C$$

The proof of this proceeds via setting $u = g(x)$. Thus, the general
idea when using logarithmic substitutions is to try to obtain the
numerator as the derivative of the denominator. For instance, consider
the integral:

$$\int \frac{x}{x^2 + 1} \, dx$$

Here, the derivative of the denominator is $2x$, so we adjust by a
factor of $2$ to obtain:

$$\frac{1}{2} \int \frac{2x}{x^2 + 1} \, dx = \frac{1}{2} \ln(|x^2 + 1|) + C$$

Note that in this case, since $x^2 + 1$ is always positive, the
absolute value can be dropped and we get $\frac{1}{2})(x^2 + 1) +
C$. Further, this antiderivative is valid over all reals.

\subsection{Trigonometric integrals involving logarithms}

Recall so far that we have seen the antiderivatives of $\sin$, $\cos$,
$\sec^2$, $\sec \cdot \tan$, $\csc^2$, and $\csc \cdot \cot$. We also
used these, along with trigonometric identities, to compute
antiderivatives for $\sin^2$, $\cos^2$, $\tan^2$, and $\cot^2$. All
these results were obtained as corollaries of the differentiation
formulas.

We now try to obtain a formula to integrate $\cot$. The key idea is to
note that:

$$\cot x = \frac{\cos x}{\sin x}$$

Since $\cos$ is the derivative of $\sin$, this matches up with the
general pattern that we just discussed, and we obtain that the
antiderivative of $\cot$ is $\ln \circ |\sin|$. In other words:

$$\int \cot x \, dx = \ln|\sin x| + C$$

Note that $\cot$ is undefined at multiples of $\pi$, and so any
integration of this sort is valid only if the entire interval of
integration lies strictly between two consecutive multiples of
$\pi$. It is also instructive to graph the antiderivative of $\cot$.

Here is the graph of $\cot$ and its antiderivative on the interval
$(0,\pi)$, where both are defined:

\includegraphics[width=3in]{cotanditsantiderivative.png}

Similarly, we obtain:

$$\int \tan x \, dx = - \ln|\cos x| + C = \ln|\sec x| + C$$

Here is the picture of $\tan$ and $-\ln|\cos|$ on the interval
$(-\pi/2,\pi/2)$, where both are defined:

\includegraphics[width=3in]{tananditsantiderivative.png}

Note that those two expressions are the same because
$\cos$ and $\sec$ are reciprocals of each other.

Let us look at a somewhat harder integral: the integral of the secant
function:

$$\int \sec x \, dx = \int \frac{\sec x (\sec x + \tan x)}{\sec x + \tan x} \, dx = \int \frac{\sec^2 x + \sec x \tan x}{\tan x + \sec x} \, dx$$

The numerator is the derivative of the denominator, and we obtain:

$$\int \sec x \, dx = \ln|\sec x + \tan x| + C$$

In a similar vein, we obtain that:

$$\int \csc x \, dx = \ln|\csc x - \cot x| + C$$

\subsection{Domain and range issues}

When doing indefinite integration, it is often best to forget about
issues of domain and range and just let the algebraic manipulations
flow. However, to interpret the results at the end, it is important to
look at the domain and range issues. Ideally, the antiderivative
should be defined and should make sense on all intervals where the
function itself is continuous. Further, if there are points where the
function is continuous but the {\em expression obtained for the
antiderivative} is not defined, we should try to obtain the limit at
that point.

\subsection{An application: integrating the cube of the tangent function}

Let us look at an application of the above:

$$\int \tan^3 x \, dx$$

Before we proceed, it is worth remarking how different integration is
from differentiation. For differentiation, there was just the formula
for differentiating $\sin$ and $\cos$, and everything else followed
using the product rule and quotient rule. We still memorized more, but
that was mainly to speed things up, not out of necessity. With
integration, on the other hand, we need to have a whole bag of {\em ad
hoc} tricks that we try one after the other.

Let us look at this integral. The key thing to do here is to break
down $\tan^3 x = \tan x \cdot \tan^2 x$. Next, we use $\tan^2 x =
\sec^2 x - 1$, and we have:

$$\int \tan x \sec^2x \, dx - \int \tan x \, dx$$

The first integral can be quickly calculated using the chain rule or
$u$-substitution, since $\sec^2 x$ is the derivative of $\tan x$. The
second integral comes from our formula, and we get:

$$\frac{\tan^2 x}{2} + \ln|\cos x| + C$$

\subsection{A fancier formula}

Here is a formula that uses the chain rule twice:

$$\int \frac{g'(x)f(\ln|g(x)|)}{g(x)} \, dx = \int f(u) \, du$$

where $u = \ln(|g(x)|)$. For instance:

$$\int \frac{2x(\ln(x^2 + 1))^3}{x^2 + 1} \, dx = \int u^3 \, du$$

where $u = \ln(x^2 + 1)$. This further simplifies to:

$$\frac{1}{4}[\ln(x^2 + 1)]^4 + C$$

\section{More tricks and techniques}

\subsection{Logarithmic differentiation}

Logarithmic derivatives are both a conceptual and a computational
tool. Currently, we focus on the computational aspects. The idea is to
use the same formula that we obtained earlier, but in reverse:

$$\frac{d}{dx} \ln(|g(x)|) = \frac{g'(x)}{g(x)}$$

Rearranging the terms yields:

$$g'(x) = g(x) \frac{d}{dx} \ln(|g(x)|)$$

If $g$ is a product of functions $g_1, g_2, \dots, g_n$, then
$\ln|g(x)| = \ln|g_1(x)| + \dots + \ln|g_n(x)|$, and we get:

$$g'(x) = g(x) \left[\frac{g_1'(x)}{g_1(x)} + \frac{g_2'(x)}{g_2(x)} + \dots + \frac{g_n'(x)}{g_n(x)}\right]$$

Note that this expression is not {\em really} new and did not {\em
really} require logarithms. You can in fact convince yourself that it
is just a reformulation of the product rule. When $g = g_1g_2$, for
instance, this says that:

$$g' = g\left[\frac{g_1'}{g_1} + \frac{g_2'}{g_2}\right]$$

Substituting $g = g_1g_2$, this simplifies to the usual product
rule. However, the logarithmic formulation has some conceptual
advantages.

For instance, suppose $g(x) := x(x-1)(x-2)$. Then, we immediately
obtain that:

$$\frac{g'(x)}{g(x)} = \frac{1}{x} + \frac{1}{x - 1} + \frac{1}{x - 2}$$

Further, of we have $g(x) = g_1(x)^{a_1}g_2(x)^{a_2} \dots
g_n(x)^{a_n}$, we obtain that:

$$\frac{g'(x)}{g(x)} = \frac{a_1g_1'(x)}{g_1(x)} + \frac{a_2g_2'(x)}{g_2(x)} + \dots + \frac{a_ng_n'(x)}{g_n(x)}$$

So, if $g(x) = x^3(x-1)^4(x-2)^5$, we obtain that:

$$\frac{g'(x)}{g(x)} = \frac{3}{x} + \frac{4}{x - 1} + \frac{5}{x - 2}$$

\subsection{Exponentiation tricks}

We have already seen basic integration and differentiation identities
for the exponentiation function. There are some ways of combining
these identities with the chain rule. I note some special cases here.

\begin{enumerate}
\item For any function $f$, the derivative of $f(x)e^x$ is $(f(x) +
  f'(x))e^x$. Thus, the integral of $g(x)e^x$ is $f(x)e^x + C$ where
  $f + f' = g$.
\item (1) is particularly useful when integrating polynomial function
  times $e^x$. This is because we can use linear algebra to find, for
  a given polynomial $g$, the unique polynomial $f$ such that $f + f'
  = g$.
\item The integral $\int f(e^x)e^x \, dx$ is $\int f(u) \, du$ where
  $u = e^x$.
\item The integral $e^{f(x)}f'(x) \, dx$ is $e^{f(x)} + C$.
\end{enumerate}

Let us consider an example to illustrate this. Consider the function:

$$F(x) := (x^2 + 5x + 1)e^x $$

The derivative of this, by the product rule, turn out to be $e^x$
times the sum of $x^2 + 5x + 1$ and its derivative, giving:

$$F'(x) = (x^2 + 7x + 6)e^x$$

Note that this is a new polynomial times $e^x$. An intersting question
would be how we could reverse this procedure, i.e., given $g(x)e^x$
where $g$ is a polynomial, how do we find a polynomial $f$ such that
the derivaitve of $f(x)e^x$ is $g(x)e^x$? By the product rule, we
obtain that:

$$g(x) = f(x) + f'(x)$$

Thus, we need to find the coefficients of $f$. Let us do this in our
concrete case where $g(x) = x^2 + 7x + 6$.

We know that the degree of $f'$ is strictly smaller than the degree of
$f$, so $f + f'$ has the same degree and same leading coefficient as
$f$. In this case, this forces $f$ to be a quadratic polynomial of the
form $x^2 + mx + n$. We then get $f'(x) = 2x + m$, and we obtain that:

$$f(x) + f'(x) = x^2 + (m + 2)x + (m + n)$$

Since we are given that $g(x) = x^2 + 7x + 6$, we can match
coefficients and obtain:

$$m + 2 = 7, \qquad m + n = 6$$

Solving, we get $m = 5$, and $n = 1$, and we get $f(x) = x^2 + 5x +
1$, recovering our original polynomial.

Although the {\em specific procedure involving comparing coefficients}
does require that we are {\em dealing with polynomials}, the general
idea remains true in a broader sense: integrating $\int g(x)e^x$ is
equivalent to finding a function $f$ such that $f + f' = g$. In some
cases, it is easier to think of it as an integration problem, and in
others, it is easier to think of it in terms of the differental
equation $f + f' = g$. The idea that these two apparently different
computations measure the same thing is extremely important.

\end{document}
