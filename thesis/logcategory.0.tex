\documentclass[a4paper]{amsart}

%Packages in use
\usepackage{fullpage, hyperref, vipul}


%Title details
\title{Exponentials and the log category}
%\author{Vipul Naik}
%\thanks{\copyright Vipul Naik, B.Sc. (Hons) Math and C.S., Chennai Mathematical Institute}

%List of new commands
\newcommand{\logcategory}[1]{\textsc{Log-Category}\left(#1\right)}
\newcommand{\aut}[1]{\text{Aut}\left(#1\right)}
\makeindex

\begin{document}
\maketitle
%\tableofcontents

\begin{abstract}
  In this write-up, I jot down some ideas I have about a ``log
  category'', a category such that the exponential maps we usually
  encounter in the theory of Lie algebras actually become
  homomorphisms between objects of the category. 
\end{abstract}

\section{Definition of the log category}

\subsection{The objects and morphisms}

Let $G$ be a group. The log category of $G$ is a category whose
objects are groups enriched with a $G$-action and some other
structure, and the morphisms are maps which preserve that structure..

\begin{definer}[Log category]
  Let $G$ be a group. The log category of $G$, which we denote by
  $\logcategory{G}$ is defined as follows. The objects of
  $\logcategory{G}$ comprise the following data:

  \begin{itemize}

  \item A group $A$

  \item A homomorphism $\phi: G \to \aut{A}$ (viz a $G$-action on $A$)

  \item A reflexive symmetric binary relation $\perp$ on $A$

  \end{itemize}

  Such that the following are satisfied:

  \begin{itemize}

  \item $\perp$ is $G$-invariant, viz:

    $$a \perp b \implies ga \perp gb$$
  \item For any $a$, the set of $b$ such that $a \perp b$ is a
    subgroup of $A$. This set shall be denoted as $a^\perp$.

  \item The set of $a$ such that $a \perp b$ for every $b \in A$, is
    the same as the set of $a$ for which $\phi(g)a = a$ for every $g
    \in G$. This set will be denoted as $A^\perp$.
  \end{itemize}

  We denote the objects as triples $(A,\phi,\perp)$. I will write $ga$
  for $\phi(g)a$ where it is unambiguous, and will use the same symbol
  $\perp$ for all objects.

  If $A$ and $B$ are two objects of $\logcategory{G}$, a morphism from
  $A$ to $B$ is a set-theoretic map $f: A \to B$ such that:

  \begin{itemize}

  \item If $a_1 \perp a_2$, then $f(a_1) \perp f(a_2)$:

    $$f(a_1a_2) = f(a_1)f(a_2)$$

    In particular since $\perp$ is reflexive, we see that $f$ gives a
    homomorphism on any cyclic subgroup of $A$.

  \item $f$ commutes with the $G$-action, viz:

    $$f(ga) = gf(a)$$
  \end{itemize}
\end{definer}

\subsection{Examples of the log category}

Here are some examples of objects in the log category of a group:

\begin{enumerate}

\item Every group is an object in its own log category; the action is
  by conjugation (inner automorphisms) and the relation $\perp$ is
  that of {\em commuting}.

\item Suppose $R$ is a noncommutative ring with unit with the property
  that $R$ is additively generated by its group of units (elements
  with two-sided inverses). Let $G$ be the group of units. Then the
  additive group of $R$ is an object in the log category of $G$. The
  $G$-action on $R$ is by conjugation, and the relation $\perp$ is
  that of commuting multiplicatively.

  The condition of being additively generated by the group of units is
  necessary to ensure the condition that $R^\perp$ is precisely the
  set of $G$-fixed points.

\item Suppose $G$ is a connected Lie group (over $\R$ or $\C$) and
  $\mathfrak{g}$ is its Lie algebra. Then $\g$ is an object
  in the log category of $G$. The action of $G$ on $\g$ is by conjugation
  and the relation $\perp$ is that of the Lie bracket being $0$.

\item Suppose $G$ is the symplectic group over reals, viz the group of
  linear maps which preserve a nondegenerate alternating bilinear form
  on a vector space $V$. Then $V$ is an object in the log category of
  $G$, where the $G$-action on $V$ is the usual action and $\perp$
  is the relation of having inner product zero.

\end{enumerate}

Note that in each case $\perp$ has a somewhat different meaning, and
the advantage of just looking at a symmetric binary relation is that
we can unify all these different meanings into a single context.

We now make the crucial observation which links this to the exponential map:

If $G$ is a (real or complex) Lie group and $\g$ is its Lie algebra,
then the exponential map from $\g$ to $G$ is a morphism in the log
category (where both are given structures as described above).

\subsection{Genuine homomorphisms}

Morphisms in the log category are not genuine group homomorphisms;
however they do have a lot of the structure of group
homomorphisms. Here we record some basic facts about these morphisms:

\begin{itemize}

\item Any morphism in the log category is a homomorphism if restricted
  to a cyclic subgroup. In other words, if one is looking {\em
    locally} at a single element, then the morphism looks like a homomorphism.

\item In particular, any morphism in the log category takes the
  trivial subgroup to the trivial subgroup.

\end{itemize}

There are two other important ways in which morphisms in the log
category give rise to genuine group homomorphisms. For this we
introduce two definitions.

\begin{definer}[$G$-derived subgroup, $G$-Abelianization]
  Given an object $A$ in the log category of $G$, the
  $G$-\definedind{derived subgroup} of $A$ is the kernel of the
  congruence generated by the equivalence relation of being in the
  same $G$-orbit. We shall denote the $G$-derived subgroup as $A'_G$.

  The quotient of $A$ by its $G$-derived subgroup is termed the
  $G$-\definedind{Abelianization}. We shall denote this as $A^{ab}_G$.
\end{definer}

Note that for $A = G$ with the standard structure of action by
conjugation, the $G$-derived subgroup is the same as the usual derived
subgroup, or commutator subgroup. In most of the other cases we
mentioned above, including the Lie algebra case and the case of the
additive group of a ring, the $G$-derived subgroup corresponds to the
usual notion of ``derived ideal''.

A second definition:

\begin{definer}[$G$-center]
  Given an object $A$ in the log category of $G$, the $A$-center of
  $G$, is either of the following equivalent things:

  \begin{itemize}

  \item $A^\perp$, viz the set of $x$ such that $a \perp x$ for all $a
    \in A$.

  \item The set of all $a$ for which $g.a = a$ for all $g \in G$.

  \end{itemize}
  
  The equality of the above two subgroups was one of the axioms for
  defining objects of the log category.
\end{definer}

We now have two genuine group homomorphisms:

\begin{itemize}

\item For $A, B \in \logcategory{G}$, a morphism from $A$ to $B$ in
  the log category of $G$, restricts to a group homomorphism from
  $A^\perp$ to $B^\perp$.
\item For $A,B \in \logcategory{G}$ a morphism from $A$ to $B$ sends
  $A'_G$ inside $B'_G$, and further, it yields a group homomorphism
  from $A^{ab}_G$ to $B^{ab}_G$.

\end{itemize}

\subsection{The matrix ring and general linear group}

The matrix ring $M_n(\R)$ is in the log category of $GL_n(\R)$ by the
action by conjugation; it is a special case of two examples:

\begin{itemize}

\item We can think of $GL_n(\R)$ as the group of units in $M_n(\R)$
  and consider the corresponding action

\item We can view $M_n(\R)$ as the Lie algebra of $GL_n(\R)$ and
  consider the corresponding action

\end{itemize}

With the latter view, we see that the exponential map from $M_n(\R)$
to $GL_n(\R)$ is a morphism in the log category. Let's see how the
general statements made in the last subsection specialize in this
case. Let $G = GL_n(\R)$ and $\g = M_n(\R)$ (one can replace $\R$ by
$\C$ throughout this example):

\begin{itemize}

\item $\g'_G$ is the set of matrices with trace $0$, and the
  Abelianization is the ground field $\R$. The Abelianization map is
  the trace map.

  $G'_G$ is the set of matrices with determinant $1$, and the
  Abelianization is $\R^*$. The Abelianization map is the determinant map.

  The induced map on the Abelianization is simply exponentiation from
  $\R$ to $\R^*$, and the general statement corresponds to the
  well-known fact that the exponential of the trace equals the
  determinant of the exponential.

\item $\g^\perp$ is the set of scalar matrices with real values, and
  $G^\perp = Z(G)$ is the set of scalar matrices with nonzero real
  values.  The induced map from $\g^\perp$ to $G^\perp$ is just the
  usual exponentiation from $\R$ to $\R^*$.

\end{itemize}

\section{Why the log category is important}

\subsection{The goal}

If $G$ is a non-Abelian group, there is absolutely no hope of getting
a surjective group homomorphism from an Abelian group to $G$. This is because
the relation of commuting is preserved on taking homomorphisms.

However, what we would ideally like to do is to think of the group $G$
as being ``covered'' by some Abelian group, in a way that may not
precisely be a group homomorphism, but still has the flavour of a
group homomorphism. This is done very effectively in the theory of Lie
groups: we have the ``exponential'' which maps a Lie algebra to the
Lie group. Even though the exponential is not always surjective, its
image does always generate the Lie group (when the Lie group is
connected).

The log category allows us to ask a related question in a more general
context; we would like maps which behave very much like the
exponential, but may not actually have the topological, analytic or
formal appearance of an exponential.

Further, even for groups where one cannot find any good map from an
{\em Abelian} group, we can still try to ``cover'' our group by some
group which is relatively more well-behaved and tractable.

\subsection{A logarithm}

\begin{definer}[Logarithm for a group]
  Let $G$ be a group. A \definedind{logarithm} for $G$ is an object $A
  \in \logcategory{G}$ along with a morphism $\exp :A \to G$ such that:

  \begin{itemize}

  \item $\exp$ induces a surjective map from $A^\perp$ to $G^\perp = Z(G)$

  \item $\exp$ induces a surjective map from $A^{ab}_G$ to $G^{ab}_G$

  \item The image of $A$ under $\exp$ generates $G$ as a group

  \end{itemize}

\end{definer}

{\em I'm not sure how important the first two conditions are}.

For a connected Lie group, the exponential map from its Lie algebra to
it actually seems to satisfy these conditions, and thus the Lie
algebra is a ``logarithm'' for the Lie group.

This leads us to interesting questions like:

\begin{itemize}

\item What are all the logarithms that a given group possesses?

\item Does every group possess an Abelian logarithm?

\item Are Abelian logarithms in general as well-behaved as Lie
  algebras for connected Lie groups?

\end{itemize}

Although I have not checked this rigourously, it seems that for a
finite group to possess an Abelian logarithm, it must be nilpotent.
If this is true, then one can confine oneself to the study of Abelian
logarithms for $p$-groups. I do not yet have an example of an Abelian
logarithm which is not essentially like a Lie algebra for the
$p$-group, nor am I sure if $p$-groups which do not admit Lie algebras,
could admit Abelian logarithms.

\subsection{The topological group setting}

We can define the log category over a topological group, in which case
we require all our objects to be topological groups and all our
morphisms to be continuous maps. Here, we may be in search of a
logarithm that is not merely Abelian, but also simply connected, or
even better contractible. Thus, the search is for a ``contractible
Abelian logarithm'' or something like that.

For instance, if we consider the topological group $\C^*$ it is
Abelian, so it's nice as an abstract group, but it's bad topologically
because it's not simply connected. Its logarithm $\C$, on the other
hand, is a contractible Abelian group, which is as nice as one can
get.

\subsection{The algebraic group setting}

Instead of defining a log category over an abstract group, one could
define the log category over an {\em algebraic} group. In this case,
the only difference would be that every object in the log category
would also have to be an algebraic group over the same field, and the
maps would have to be regular maps as far as the underlying algebraic
variety structure is concerned.

In this case, the appropriate analogue of ``Abelian logarithm'' would
be a ``linear logarithm'', a logarithm which is a vector space over
the underlying field. Of course, when working over fields, Lie
algebras are vector spaces, so they are examples of linear logarithms.

Thus questions of interest would be like: what are the algebraic groups
that possess linear logarithms?


\printindex

\end{document}
