\documentclass[10pt]{amsart}

%Packages in use
\usepackage{fullpage, hyperref, vipul, enumerate}

%Title details
\title{Take-home class quiz solutions: Wednesday January 9: Parametric stuff}
\author{Math 195, Section 59 (Vipul Naik)}
%List of new commands

\begin{document}
\maketitle

\section{Performance review}

$26$ people took this $11$-question quiz. The score distribution was
as follows:

\begin{itemize}
\item Score of $4$: $1$ person.
\item Score of $6$: $3$ people.
\item Score of $7$: $7$ people.
\item Score of $8$: $1$ person.
\item Score of $9$: $6$ people.
\item Score of $10$: $4$ people.
\item Score of $11$: $4$ people.
\end{itemize}

The question-wise answers and performance review are below.

\begin{enumerate}
\item Option (E): $19$ people.
\item Option (C): $13$ people.
\item Option (C): $25$ people.
\item Option (A): $24$ people.
\item Option (A): $20$ people.
\item Option (D): $25$ people.
\item Option (A): $25$ people.
\item Option (C): $20$ people.
\item Option (B): $11$ people.
\item Option (D): $12$ people.
\item Option (E): $23$ people.
\end{enumerate}

\section{Solutions}

\begin{enumerate}
\item Consider the curve given by the parametric description $x = \cos
  t$, $y = \sin t$, where $t$ varies over the interval $[a,b]$ with $a <
  b$. What is a necessary and sufficient condition on $a$ and $b$ for
  this curve to be the circle $x^2 + y^2 = 1$?

  \begin{enumerate}[(A)]
  \item $b - a =\pi$
  \item $b - a > \pi$
  \item $b - a = 2\pi$
  \item $b - a > 2\pi$
  \item $b - a \ge 2\pi$
  \end{enumerate}

  {\em Answer}: Option (E)

  {\em Explanation}: The curve is traced along the circle starting at
  $(\cos a, \sin a)$ and going around the circle till we reach $b$. In
  order to cover the whole circle, it is necessary that it make at
  least one full angle of $2\pi$. Thus, the condition $b - a \ge 2\pi$.

  Note that the equality case is valid because we are working with the
  {\em closed} interval $[a,b]$. If we were working with the open
  interval $(a,b)$, then strict inequality would be the necessary and
  sufficient condition.

  {\em Performance review}: $19$ out of $26$ got this. $7$ chose $C$,
  which is the correct answer for the curve to {\em just} cover the
  circle but is the wrong choice for a necessary and sufficient
  condition.

  {\em Historical note (last time)}: $11$ people got this correct. $9$ people
  chose (C), $2$ people chose (B), $1$ person each chose
  (A) and (D).

\item (*) Consider the curve given by the parametric description $x =
  \arctan t$ and $y = \arctan t$ for $t \in \R$. Which of the
  following is the best description of this curve?

  \begin{enumerate}[(A)]
  \item It is the graph of the function $\arctan$
  \item It is the line $y = x$
  \item It is a line segment (without endpoints) that is part of the
    line $y = x$
  \item It is a half-line (with endpoint) that is part of the line $y
    = x$
  \item It is a disjoint union of two half-lines that are both part of
    the line $y = x$
  \end{enumerate}

  {\em Answer}: Option (C)

  {\em Explanation}: Eliminating the parameter $t$, we get that $y =
  x$, but with the additional caveat that the value of $x$ (hence also
  $y$) must be in the range of $\arctan$. The range of $\arctan$ is
  the open interval $(-\pi/2,\pi/2)$, thus we get the corresponding
  line segment without endpoints joining the point with coordinates
  $(\pi/2,\pi/2)$ to the point with coordinates $(-\pi/2,-\pi/2)$.

  {\em Performance review}: $13$ out of $26$ got this. $10$ chose (D),
  $2$ chose (B), $1$ chose multiple options.

  {\em Historical note (last time)}: $8$ people got this correct. $9$ people
  chose (B), which would be the right idea {\em except for the issue
  of domain/range restrictions}. $4$ chose (D), $2$ chose (A), $1$
  chose (E).

\item (*) Consider the curve given by the parametric description $x =
  \sin^2t$ and $y = \cos^2t$ for $t \in \R$. Which of the
  following is the best description of this curve?

  \begin{enumerate}[(A)]
  \item It is the arc of the circle $x^2 + y^2 = 1$ comprising the
    first quadrant, i.e., when $x \ge 0$ and $y \ge 0$.
  \item It is the entire circle $x^2 + y^2 = 1$
  \item It is the line segment joining the points $(0,1)$ and $(1,0)$
  \item It is the line $y = 1 - x$
  \item It is a portion of the parabola $y = x^2$
  \end{enumerate}

  {\em Answer}: Option (C)

  {\em Explanation}: Eliminating the parameter, we obtain that $x + y
  = 1$. Further, we much have $x \ge 0$ and $y \ge 0$ since they are
  both squares. Subject to these conditions, any pair $(x,y)$
  works. This is thus the part of the line $x + y = 1$ which lies in
  the first quadrant. This can alternatively be described as the line
  segment joining the points $(0,1)$ and $(1,0)$.

  {\em Performance review}: $25$ out of $26$ got this. $1$ chose (B).

  {\em Historical note (last time)}: $5$ people got this correct. $11$ people
  chose (D), which would be the correct answer {\em except for the
  issue of domain/range restrictions}. $4$ people each chose (A) and
  (B).
\item Identify the parametric description which {\em does not}
  correspond to the set of points $(x,y)$ satisfying $x^3 = y^5$.

  \begin{enumerate}[(A)]
  \item $x = t^3$, $y = t^5$, for $t \in \R$
  \item $x = t^5$, $y = t^3$, for $t \in \R$
  \item $x = t$, $y = t^{3/5}$, for $t \in \R$
  \item $x = t^{5/3}$, $y = t$, for $t \in \R$
  \item All of the above parametric descriptions work
  \end{enumerate}

  {\em Answer}: Option (A)

  {\em Explanation}: The exponents are at the wrong places -- if $x =
  t^3$, then $x^3 = t^9$ and if $y = t^5$, then $y^5 = t^{25}$ --
  these are certainly not equal.

  {\em Performance review}: $24$ out of $26$ got this. $2$ chose (D).

  {\em Historical note (last time)}: $16$ people got this correct. $4$ chose
  (E), $3$ chose (B), $1$ chose (C).

\item (*) Consider the parametric description $x = f(t)$, $y = g(t)$
  where $t$ varies over all of $\R$. What is the necessary and
  sufficient condition for the curve given by this to be the graph of
  a function, i.e., to satisfy the vertical line test?

  \begin{enumerate}[(A)]
  \item For any $t_1$ and $t_2$ satisfying $f(t_1) = f(t_2)$, we must
    have $g(t_1) = g(t_2)$.
  \item For any $t_1$ and $t_2$ satisfying $g(t_1) = g(t_2)$, we must
    have $f(t_1) = f(t_2)$.
  \item Both $f$ and $g$ are one-to-one functions.
  \item For any $t_1$ and $t_2$, we must have $f(t_1) = f(t_2)$.
  \item For any $t_1$ and $t_2$, we must have $g(t_1) = g(t_2)$.
  \end{enumerate}

  {\em Answer}: Option (A)

  {\em Explanation}: We want that for a given $x$-value there is at
  most one $y$-value (the vertical line test). This means that if, at
  two times $t_1$ and $t_2$, the $x$-values $f(t_1)$ and $f(t_2)$ are
  equal to each other, the $y$-values $g(t_1)$ and $g(t_2)$ must also
  be equal to each other. This is option (A).

  {\em Performance review}: $20$ out of $26$ got this. $3$ each chose
  (B) and (C).

  {\em Historical note (last time)}: $10$ people got this correct. $11$ chose
  (C), which is a {\em sufficient} but not a necessary condition. $2$
  chose (B) and $1$ chose (D).

\item Suppose $f$ and $g$ are both twice differentiable functions
  everywhere on $\R$. Which of the following is the correct formula
  for $(f \circ g)''$?

  \begin{enumerate}[(A)]

  \item $(f'' \circ g) \cdot g''$
  \item $(f'' \circ g) \cdot (f' \circ g') \cdot g''$
  \item $(f'' \circ g) \cdot (f' \circ g') \cdot (f \circ g'')$
  \item $(f'' \circ g) \cdot (g')^2 + (f' \circ g) \cdot g''$
  \item $(f' \circ g') \cdot (f \circ g) + (f'' \circ g'')$
  \end{enumerate}

  {\em Answer}: Option (D)

  {\em Explanation}: This question is tricky because it requires the
  application of both the product rule and the chain rule, with the
  latter being used twice. We first note that:

  $$(f \circ g)' = (f' \circ g) \cdot g'$$

  Now, we differentiate both sides:

  $$(f \circ g)'' = [(f' \circ g) \cdot g']'$$

  The expression on the right side that needs to be differentiated is
  a product, so we use the product rule:

  $$(f \circ g)'' = [(f' \circ g)' \cdot g'] + [(f' \circ g) \cdot g'']$$

  Now, the inner composition $f' \circ g$ needs to be
  differentiated. We use the chain rule and obtain that $(f' \circ g)'
  = (f'' \circ g) \cdot g'$. Plugging this back in, we get:

  $$(f \circ g)'' = (f'' \circ g) \cdot (g')^2 + (f' \circ g) \cdot g''$$

  {\em Remark}: What's worth noting here is that in order to
  differentiate composites of functions, you need to use both
  composites {\em and} products (that's the chain rule). And in order
  to differentiate products, you need to use both products {\em and}
  sums (that's the product rule). Thus, in order to differentiate a
  composite twice, we need to use composites, products, {\em and}
  sums.

  {\em Performance review}: $25$ out of $26$ got this. $1$ chose (A).

  {\em Historical note (last time)}: $20$ out of $21$ people got this
  correct. $1$ person left the question blank.

  {\em Historical note}: I put this question in a quiz for Math 152
  back in October 2010, and $14$ of $14$ people who took that
  quiz got it correct.
\item Suppose $x = f(t)$ and $y = g(t)$ where $f$ and $g$ are both
  twice differentiable functions. What is $d^2y/dx^2$ in terms of $f$
  and $g$ and their derivatives evaluated at $t$?

  \begin{enumerate}[(A)]
  \item $(f'(t)g''(t) - g'(t)f''(t))/(f'(t))^3$
  \item $(f'(t)g''(t) - g'(t)f''(t))/(g'(t))^3$
  \item $(g'(t)f''(t) - f'(t)g''(t))/(f'(t))^3$
  \item $(g'(t)f''(t) - f'(t)g''(t))/(g'(t))^3$
  \item None of the above
  \end{enumerate}

  {\em Answer}: Option (A)

  {\em Explanation}: See lecture notes.

  {\em Performance review}: $25$ out of $26$ got this. $1$ chose (B).

  {\em Historical note (last time)}: $20$ out of $21$ people got this
  correct. $1$ person chose (E).
\item Which of the following pair of bounds works for the arc length
  for the portion of the graph of the sine function between $(a,\sin
  a)$ and $(b, \sin b)$ where $a < b$?

  \begin{enumerate}[(A)]
  \item Between $(b - a)/\sqrt{3}$ and $(b - a)/\sqrt{2}$
  \item Between $(b - a)/\sqrt{2}$ and $b - a$
  \item Between $(b - a)$ and $\sqrt{2}(b - a)$
  \item Between $\sqrt{2}(b - a)$ and $\sqrt{3}(b - a)$
  \item Between $\sqrt{3}(b - a)$ and $2(b - a)$
  \end{enumerate}

  {\em Answer}: Option (C)

  {\em Explanation}: The derivative function is $\cos$, so the
  corresponding arc length formula gives:

  $$\int_a^b \sqrt{1 + \cos^2x} \, dx$$

  The integrand is always between $1$ and $\sqrt{2}$, so the integral
  must be between $1 \cdot (b - a)$ and $\sqrt{2} \cdot (b - a)$.

  {\em Performance review}: $20$ out of $26$ got this. $4$ chose (D),
  $1$ each chose (A) and (B).

  {\em Historical note (last time)}: $15$ out of $21$ people got this
  correct. $2$ chose (B), $2$ left the question blank, $1$ each chose
  (A) and (D).

\item (*) Consider the parametric curve $x = e^t$, $y = e^{t^2}$. How does
  $y$ grow in terms of $x$ as $x \to \infty$?

  \begin{enumerate}[(A)]
  \item $y$ grows like a polynomial in $x$.
  \item $y$ grows faster than any polynomial in $x$ but slower than an
    exponential function of $x$.
  \item $y$ grows exponentially in $x$.
  \item $y$ grows super-exponentially in $x$ but slower than a double
    exponential in $x$.
  \item $y$ grows like a double exponential in $x$.
  \end{enumerate}

  {\em Answer}: Option (B)

  {\em Explanation}: Note that a polynomial in $x$ is still
  exponential in $t$, and not in $t^2$, i.e., it is too slow to be
  $y$. Thus $y$ grows faster than any polynomial in $x$. On the other
  hand, an exponential in $x$ is doubly exponential in $t$, which is
  faster in growth than $e^{t^2}$. Thus, option (B).

  {\em Performance review}: $11$ out of $26$ got this. $9$ chose (D),
  $3$ chose (E), $2$ chose (C), $1$ chose (A).

  {\em Historical note (last time)}: $7$ out of $21$
  people got this correct. $4$ each chose (A) and (E), $3$ chose (C),
  $1$ chose (D), and $2$ left the question blank.

\item (*) We say that a curve is {\em algebraic} if it admits a
  parameterization of the form $x = f(t)$, $y = g(t)$, where $f$ and
  $g$ are rational functions and $t$ varies over some subset of the real
  numbers. Which of the following curves is {\em not} algebraic?

  \begin{enumerate}[(A)]
  \item $x = \cos t$, $y = \sin t$, $t \in \R$
  \item $x = \cos t$, $y = \cos(3t)$, $t \in \R$
  \item $x = \cos t$, $y = \cos^2t$, $t \in \R$
  \item $x = \cos t$, $y = \cos(t^2)$, $t \in \R$
  \item None of the above, i.e., they are all algebraic
  \end{enumerate}

  {\em Answer}: Option (D)

  {\em Explanation}: In all the other cases, we can elucidate an
  algebraic relationship between the variables.

  For option (A), we can set both $\cos t$ and $\sin t$ as rational
  functions in $\tan(t/2)$. In fact, the rational functions in
  $\tan(t/2)$ approach works for options (B) and (C) as well, though
  there are simpler approaches in those cases. The approach does not
  work for option (D).

  {\em Performance review}: $12$ out of $26$ got this. $9$ chose (E),
  $3$ chose (A), $1$ each chose (B) and (C).

  {\em Historical note (last time)}: $11$ out of $21$ people got this
  correct. $8$ chose (E), $1$ chose (B), $1$ left the question blank.

\item (+) Suppose $x = f(t)$, $y = g(t)$, $t \in \R$ is a parametric
  description of a curve $\Gamma$ and both $f$ and $g$ are continuous
  on all of $\R$. If both $f$ and $g$ are even, what can we conclude
  about $\Gamma$ and its parameterization?

  \begin{enumerate}[(A)]
  \item $\Gamma$ is symmetric about the $y$-axis
  \item $\Gamma$ is symmetric about the $x$-axis
  \item $\Gamma$ is symmetric about the line $y = x$
  \item $\Gamma$ has half turn symmetry about the origin
  \item The parameterizations of $\Gamma$ for $t \le 0$ and for $t \ge
    0$ both cover all of $\Gamma$, and in directions mutually reverse
    to each other.
  \end{enumerate}

  {\em Answer}: Option (E)

  {\em Explanation}: See lecture notes.

  {\em Performance review}: $23$ out of $26$ got this. $2$ chose (A),
  $1$ chose (C).

  {\em Historical note (last time)}: $5$ out of $21$ people got this
  correct. $7$ chose (A), $4$ chose (D), $2$ each chose (B) and (C),
  $1$ left the question blank.

\end{enumerate}

\end{document}