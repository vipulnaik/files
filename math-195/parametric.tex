\documentclass[10pt]{amsart}
\usepackage{fullpage,hyperref,vipul,graphicx}
\title{Parametric stuff}
\author{Math 195, Section 59 (Vipul Naik)}

\begin{document}
\maketitle

{\bf Corresponding material in the book}: Section 10.1, 10.2. {\em We
are omitting the topic of surface area mentioned at the end of Section
10.2 of the book}.

{\bf What students should definitely get}: Parametric descriptions of
curves, relationship with functional and relational descriptions,
domain of parameters, parametric differentiation, arc length.

{\bf What students should hopefully get}: Distinction between curve
and parameterization, self-intersections, going back and forth between
parametric and relational descriptions, the subtletly behind higher
derivatives.

\section*{Note on lecture notes and book}

The lecture notes are intended to be a reasonably faithful version of
what was planned for the lecture, though the actual contents of the
lecture may deviate somewhat.

I generally avoid doing a lot of worked examples of the kind that are
already present in the book. You are encouraged to read the worked
examples and other discussions in the book to get a different
perspective when you sit down to do the weekly homework.

The book often has some fancy examples (for instance, cycloids,
strophoids, conchoids, cissoids, cochleoids, cardioids, etc.)  that
are fun to read, but you are not expected to have a thorough mastery
of this stuff for the course. The level of difficulty of the tests is
roughly going to be like the routine homeworks, with a few multiple
choice questions of the level of difficulty of the quizzes and one
question at the level of the advanced homeworks.

\section*{Executive summary}

Words ...

\begin{enumerate}
\item A parametric description of a curve is one where both
  coordinates are expressed as functions of of a parameter, typically
  denoted $t$. Parametric descriptions offer an alternative to
  functional and implicit (relational) descriptions of curves. Here,
  $t$ varies over some subset of the real numbers. In symbols, we have
  something like $x = f(t)$, $y = g(t)$, where $t$ varies over some
  subset $D$ of the real numbers.
\item Descriptions where $x$ is a function of $y$ or $y$ is a function
  of $x$ are special cases of parametric descriptions.
\item The same curve may admit multiple parametrizations, and
  different parameterizations may correspond to different speeds and
  different orderings of traversal of the point. The curve itself only
  contains the information of {\em what} points were traversed, not
  the information of the {\em sequence} and {\em pace} in which they
  were traversed.
\item The curve traced by a parameterization depends not only on the
  coordinate functions but also the domain for the parameter. The
  larger the domain, in general, the larger the curve traced. However,
  in some cases, expanding the domain may not make the curve strictly
  larger. This happens in cases where both coordinate functions are
  even or have commensurable periods.
\item A parameterization of a curve may involve self-intersections,
  retracings (e.g., tracing back for even function pairs), or even
  wrapping around itself (for periodic function pairs).
\item Function composition allows us to switch between multiple
  parameterizations.
\item In some cases, it is possible to move back and forth between
  parametric and relational descriptions.
\item Parametric differentiation: if $x = f(t)$ and $y = g(t)$, then
  $dy/dx = (dy/dt)/(dx/dt) = g'(t)/f'(t)$. This can also be used to
  differentiate repeatedly. Note that the derivative is a function of
  $t$ rather than of $(x,y)$, so to find the derivative given the
  point $(x,y)$ we need to go back and determine $t$.
\item Higher derivatives can be computed iteratively using parametric
  differentiation. But note that it is {\em not} true that $d^2y/dx^2
  = (d^2y/dt^2)/(d^2x/dt^2)$. The actual formula/procedure is more
  complicated (see lecture notes).
\item Arc length: The formula for arc length from $t = a$ to $t = b$
  (with $a < b$) is $\int_a^b \sqrt{(dx/dt)^2 + (dy/dt)^2} \, dt$.
\end{enumerate}

Actions ...

\begin{enumerate}
\item {\em Parametric to relational: elimination of parameter}: In
  many cases, it is possible to eliminate a parameter from a
  parametric description. The idea is to use some well known
  identities or manipulation techniques to try to directly relate $x$
  and $y$ by finding some equation between them that is true for all
  $t$. However, this is not the full story. We next need to see if
  there are additional restrictions on $x$ and $y$ deducible from the
  fact that they arose as function of $t$, also keeping in mind the
  domain restrictions on $t$.

  For instance, the parameterization $x = t^2, y = t^4$ for $t \in \R$
  can be rewritten as $y = x^2$, but we need the additional condition
  that $x \ge 0$.

  See more examples in the lecture notes, quizzes, and homeworks.
\item {\em Relational to parametric}: Here, we see a relation between
  $x$ and $y$, and try to choose a parametric description that would
  give rise to the relation. Again, the domain of choice for the
  parameter needs to be chosen wisely.

  See more examples in the lecture notes and quizzes.
\item {\em Parametric differentiation and geometric consequences}: We
  use the formula $(dy/dt)/(dx/dt)$. If $x = f(t)$ and $y = g(t)$,
  then this becomes $g'(t)/f'(t)$. This is valid for all $t$ in the
  interior of the domain of definition where both $f'$ and $g'$ are
  defined and $f' \ne 0$. If $f'(t) = 0$ but $g'(t) \ne 0$, we have a
  vertical tangent situation. If $g'(t) = 0$ but $f'(t) \ne 0$, we
  have a horizontal tangent situation.
\end{enumerate}

\section{An introduction to the parametric approach}

\subsection{Describing curves: the old ways}

In single variable calculus, you saw two main kinds of ways that
curves in the $xy$-coordinate plane can be described:

\begin{itemize}
\item Explicit functional descriptions, where one coordinate is
  written as a function of the other, i.e., $y = f(x)$ or $x = f(y)$
  type form. The $y = f(x)$ type description must satisfy the vertical
  line test.
\item Implicit descriptions or relational descriptions, where an
  expression in both variables is declared to be zero, i.e., $R(x,y) =
  0$ where $R$ is an expression that involves both $x$ and
  $y$. Implicit descriptions need not define functions globally,
  though under suitable conditions, they may still define functions
  locally. You saw implicit descriptions when looking at implicit
  differentiation, and later you saw that the general solution to a
  differential equation often appears naturally first in the form of
  an implicit description, and depending on the circumstances, it may
  or may not be possible to write $y$ {\em explicitly} as a function
  of $x$.
\end{itemize}

\subsection{A new approach: a parameter}

In addition to {\em functional} and {\em relational} descriptions of
curves, there is a third kind of description called a {\em parametric}
description. Here, we find functions $f$ and $g$ and a parameter $t$ and write:

$$x = f(t), y = g(t), t \in D$$

where $D$ is a subset of $\R$. The curve traced by this parametric description is precisely the set of points:

$$\{ (f(t),g(t)) : t \in D \}$$

One way of thinking of this is that $t$ is time, and there is a
particle such that the $x$-coordinate of its location at time $t$ is
$f(t)$ and the $y$-coordinate is $g(t)$. In other words, the location
of the particle at time $t$ is functionally dependent on $t$, with
each coordinate given by its own function. The set $D$ is the set of
times for which this is valid.

\subsection{The relation between functional and parametric descriptions}

Functional descriptions (where one of the coordinates $x$, $y$ is
expressed in terms of the other) are special cases of parametric
descriptions as follows:

\begin{itemize}
\item If $f$ is the identity function (i.e., $f(t) = t$ by definition)
  then the curve is the graph of the function $y = g(x)$ with $x \in
  D$.
\item If $g$ is the identity function (i.e., $g(t) = t$ by definition)
  then the curve is the graph of the function $x = f(y)$ with $y \in
  D$.
\end{itemize} 

\subsection{Parameterization}

Given a curve, a particular parametric description of the curve is
termed a {\em parameterization} of the curve. We say that the
parameterization is {\em continuous} if the function $f$ and $g$ are
continuous (and also, typically, we assume that $D$ is connected) and
we say that the parameterization is {\em smooth} if it is continuous
and the functions $f$ and $g$ are continuously differentiable on the
interior of $D$ (in some definitions, ``smooth'' requires us to assume
that $f$ and $g$ are both infinitely differentiable).

\subsection{Parameterization: a story reconstruction}

A curve, or a subset of the plane, is a static entity. It's just
there, all at once.

A parameterization creates a {\em story} behind the curve. The story
reveals, how, with the passage of time (the parameter $t$) the curve
was gradually constructed.

However, the important point to note here is that we have a lot of
leeway in how to construct this story. We can think of the curve as a
collection of trails left behind by a wild animal -- but it's up to us
to connect the dots and decide just {\em how fast} the wild animal
moved. We even have leeway in determining {\em which direction} the
wild animal moved -- there are two mutually opposite directions of
motion that would leave the same trail. The trails don't give us
conclusive information in this regard.

The upshot is that the same curve can have multiple
parameterizations. These parameterizations differ in the speed with
which they go through stuff. They could also differ in other subtler
ways.

For instance, the curve $y = x^2$ can be represented using the
parameterization $x = t$, $y = t^2$ or using the parameterization $x =
t^3$, $y = t^6$, or using the parameterization $x = \sinh t$, $y =
\sinh^2 t$.

There is an additional leeway we have in some cases: in case the curve
is self-intersecting, we have further leeway in choosing the order of
paths taken by the curves, i.e., the sequence in which various trails
were left.
\subsection{Choice of domain}

The same pair of function $f$ and $g$ can define different curves
depending on the domain of the variable $t$. In general, if $D_1
\subseteq D_2$, then the curve obtained by making $t$ vary over $D_1$
is a subset of the curve obtained by varying $t$ over $D_2$.

However, just because $D_1$ is a {\em proper} subset of $D_2$ does not
necessarily mean that the corresponding curve is a {\em proper}
subset. For instance, the circle $x^2 + y^2 = 1$ can be described by
$x = \cos t$, $y = \sin t$, for $t \in [0,2\pi]$, but allowing $t$ to
vary over the much bigger set of {\em all} real numbers gives the {\em
same} circle. Similarly, the curve given by $x = t^2, y = t^4$ is the
same whether we let $t$ vary over all nonnegative reals or whether we
let $t$ vary over all reals. In both cases, it is the graph of the
function $y = x^2$ restricted to $x \ge 0$.

\subsection{Different letter for the parameter}

In most cases, we use the parameter $t$ and we're secretly thinking of
time. However, there is nothing sacred about $t$, and we could use any
other letter such as $u$ or $\theta$.

\subsection{Switching between multiple parametric descriptions}

Consider a parametric description $x = f(t)$, $y = g(t)$. Suppose we
consider a function $h$. Then we can define a new parametric
description using $t = h(u)$, i.e., we now define $x = f(h(u))$ and $y
= g(h(u))$ with $u$ the new parameter. We do need to worry about
domain issues, but this is best seen in context.

\section{The topology and geometry of parametric descriptions}

\subsection{Various kinds of repetitions}

The same point $(x,y)$ may occur in a given parametric description for
multiple values of $t$. Some possible ways that this repetition could
occur are given below -- note that the list is not exhausitve:

\begin{itemize}
\item {\em Self-intersection at an isolated point}: Here, just a single
  point is repeated, and the behavior around the point is not
  repeated. The curve thus intersects itself. For instance, the curve
  $x = t(t-1)(t-2)$, $y = t(t-1)(t-3)$ gives the same point pair
  $(0,0)$ for $t = 0$ and $t = 1$, but the curve intersects itself
  only at this isolated point.
\item {\em Reverse trace}: Here, the curve traces back along
  itself. This occurs when both functions have a common axis of mirror
  symmetry. For instance, $x = t(1 - t)$ and $y = \sin(\pi t)$ are
  both functions with mirror symmetry about $t = 1/2$. Thus, the path
  traced from $-\infty$ to $1/2$ is retraced in reverse from $1/2$ to
  $\infty$. As an easier example, if both coordinates are defined by
  even functions, then the path traced from $-\infty$ to $0$ is traced
  back in reverse from $0$ to $\infty$.
\item {\em Wrap around itself}: Here, the curve wraps around
  itself. This occurs if both $x$ and $y$ are periodic functions with
  equal or commensurable periods. The canonical example is $x = \cos
  t$, $y = \sin t$.
\end{itemize}

\subsection{Dense fillings and such curves}

Some parameterizations can fill an area pretty densely. One example is
the parameterization $x = \sin t$, $y = \sin(\alpha t)$ where $\alpha$
is an irrational number. Note first that both $x$ and $y$ remain in
$[-1,1]$, so this curve remains in a closed bounded region, namely the
square with vertices with $(x,y)$-coordinates $(1,1)$, $(-1,1)$,
$(-1,-1)$, and $(1,-1)$. It turns out that the curve covers the square
densely as $t \in \R$.

\subsection{Closed intervals, open intervals, disconnections, etc.}

We have an intuitive idea of what it means for a subset of the real
numbers to be {\em connected}. For subsets of the real numbers to be
connected is equivalent to its being an {\em interval}, which may be
open, closed, or infinite at one or both ends.

Typically, when we perform a parameterization $x = f(t)$, $y = g(t)$,
$t \in D$, then $D$ is taken to be an interval. We make some notes:

\begin{itemize}
\item If $D$ is a closed bounded interval, i.e., of the form $[a,b]$
  for $a < b$ both finite, then the curve traced is a curve that
  includes its endpoints, and hence is also closed and
  bounded. Basically, the image of something closed and bounded is
  closed and bounded. This is a result of topology that is also
  responsible for the {\em extreme value theorem} which you have seen
  in single variable calculus.
\item In other cases, the curve traced may or may not be closed and
  bounded. We cannot say anything for sure.
\end{itemize}
\section{Converting back and forth between the various forms}

Here we discuss the back and forth:

Parametric description $\leftrightarrow$ Implicit (relational) description

You will not be expected to actually execute this back and forth except
in some simple cases. The purpose of this discussion is simply to
illustrate this with simple examples.
\subsection{Proceeding from a parametric to a relational description}

The key idea here is to notice how $x$ and $y$ relate to the parameter
$t$, and then try to deduce from this a relationship between $x$ and
$y$ that does not involve $t$. For instance, consider $x = \cos t$ and
$y = \sin^2t$. In this case, we know that:

$$\cos^2 t + \sin^2 t = 1 \ \forall \ t \in \R$$

Thus, we get that $x$ and $y$ satisfy the relation:

$$x^2 + y = 1$$

Note that in this case, we see that $y$ is in fact a function of $x$:

$$y = 1 - x^2$$

{\em However}, and this is the key point, there may well be a loss of
information when we move to the relational description. Specifically,
the parameterization using $t$ imposes the constraints that $-1 \le x
\le 1$ and $0 \le y \le 1$. On the other hand, the relation $x^2 + y =
1$ covers a larger swath of possibilities for $x$ and $y$. Thus, we
need to {\em explicitly} include the restrictions based on the range
of possible values for $x$ and $y$, in this case, the restriction that
$x \in [-1,1]$. Since $y$ is determined based on $x$, the additional
restriction on the value of $y$ is not necessary.

Here are some other examples:

\begin{itemize}
\item $x = \cosh t$, $y = \sinh t$, $t \in \R$. We obtain that $x^2 =
  1 + y^2$, with the additional restriction that $x > 0$ (and no
  restriction on $y$). Alternatively, we can express this as $x =
  \sqrt{1 + y^2}$. If we write it in this form, the additional
  restriction is unnecessary.
\item $x = \cos t$, $y = \cos(2t)$, $t \in \R$. We obtain that $y =
  2x^2 - 1$, with the additional restriction that $x \in [-1,1]$.
\item $x = t^3$, $y = t^7$. We can rewrite this as $x^7 = y^3$, with
  no additional restrictions on $x$ and $y$. Note that the situation
  becomes a little more tricky if we have even exponents instead of
  the odd numbers $3$ and $7$.
\item $x = t^4$, $y = t^2 + 1$, $t \in \R$. We get $x = (y - 1)^2$,
  with the restriction that $y \ge 1$. 
\item $x = t^2$, $y = t^3$, $t \in [1,2]$. We get $x^3 = y^2$, with $y
  \in [1,8]$. 
\end{itemize}

\subsection{Proceeding from a relational to a parametric description}

Converting a relational description to a {\em nice} parametric
description is tricky, and occasionally requires the invention of new
branches of mathematics. For instance, trigonometry is the branch of
mathematics that was developed to provide a nice parametric
description for the circle $x^2 + y^2 = 1$. There are some simple
cases where parametric descriptions are easy to construct, but these
are usually exceptions.

For instance, for $x^2 - y^2 = 1$, $x > 0$, we could take the
parametric description $x = \cosh t$ and $y = \sinh t$ for $t \in
\R$ or the parametric description $x = \sec t$ and $y = \tan t$ for $t
\in (-\pi/2,\pi/2)$.

\section{Calculus with parameters}

\subsection{Implicit differentiation recall}

You're already aware of how to do differentiation and integration, and
apply these to geometric situations (tangents, normals, angle of
intersection, etc.) for curves given explicitly in functional
form. And you've also seen how to do differentiation of curves given
in implicit or relational form (it's called {\em implicit
differentiation}).

We'll spend a few seconds recalling implicit differentiation, since
there are important parallels and similarities between implicit and
parametric differentiation.

Implicit differentiation goes something like this -- we start with a
mixed expression in $x$ and $y$ that describes the curve, e.g.:

$$\sin(x + y) = xy$$

We differentiate both sides:

$$\frac{d(\sin(x + y))}{dx} = \frac{d(xy)}{dx}$$

Now, how would we handle something like $\sin(x + y)$? It is something
in terms of $x + y$, so we use the chain rule on the left side,
thinking of $v = x + y$ as the intermediate function:

$$\frac{d(\sin(x + y))}{d(x + y)} \frac{d(x + y)}{dx} = x \frac{dy}{dx} + y \frac{dx}{dx}$$

This simplifies to:

$$ \cos (x + y) \left[1 + \frac{dy}{dx}\right] = x \frac{dy}{dx} + y$$

Opening up the parentheses, we get:

$$\cos(x + y) + \cos(x + y) \frac{dy}{dx} = x \frac{dy}{dx} + y$$

Now, we move stuff together to one side, to get:

$$(\cos(x + y) - x)\frac{dy}{dx} = y - \cos(x + y)$$

And we now isolate $dy/dx$:

$$\frac{dy}{dx} = \frac{y - \cos(x + y)}{\cos(x + y) - x}$$

Notice first that the right side is a {\em mixed expression in both
$x$ and $y$}, rather than an expression purely in terms of $x$. This
is to be expected since $y$ is not explicitly a function of $x$. 

But something subtler is going on. When we do differentiation with the
English letter ``d'' then that notation makes sense {\em only} if $y$
is (locally) a function of $x$. If it isn't, then the answer above
notwithstanding, the derivative $dy/dx$ does not exist.

Here's another example: the circle of radius $1$ centered at the
origin. This is given by the equation $x^2 + y^2 = 1$. Note that in
this case, $y$ is {\em not} a function of $x$, because for many values
of $x$, there are two values of $y$. For instance, for $x = 0$, we
have $y = 1$ and $y = -1$. For $x = 1/2$, we have $y = \sqrt{3}/2$ and
$y = -\sqrt{3}/2$. So, $y$ is not a function of $x$.

However, {\em locally} $y$ is still a function of $x$, in the
following sense. If you just restrict yourself to the part above the
$x$-axis, then you do get $y$ as a function of $x$. This is the
function $y := \sqrt{1 - x^2}$ for $-1 \le x \le 1$.  If we restrict
ourselves to the part below the $x$-axis, we consider the function $y
:= -\sqrt{1 - x^2}$ for $-1 \le x \le 1$.

Now, how do we calculate $dy/dx$? Well, it depends on whether we are
interested in the part above the $x$-axis or in the part below the
$x$-axis. For the part above the $x$-axis, we have the function
$\sqrt{1 - x^2}$, and we get that the derivative is:

$$\frac{d(\sqrt{1 - x^2})}{dx} = \frac{d(\sqrt{1 - x^2})}{d(1 - x^2)} \frac{d(1 - x^2)}{dx} = \frac{1}{2\sqrt{1 - x^2}} \cdot (2x) = \frac{-x}{\sqrt{1 - x^2}}$$

If we are interested in the lower side, we get $x/\sqrt{1 - x^2}$.

Now, in this case, we have to split into two cases, and do a painful
calculation involving differentiating a square root via the chain rule.

Let's now consider how to use implicit differentiation to do the same example.

We start with the original expression:

$$x^2 + y^2 = 1$$

This is an {\em identity}, which means that it's true for every point
on the curve. When we have an equation that is identically true, it is
legitimate to differentiate both sides and still get an
identity. Differentiating both sides with respect to $x$, we get:

$$\frac{d(x^2)}{dx} + \frac{d(y^2)}{dx} = 0$$

Simplifying and using the chain rule, we get:

$$2x + 2y\frac{dy}{dx} = 0$$

We thus get:

$$\frac{dy}{dx} = \frac{-x}{y}$$

Notice that with this method, we get $-x/y$, which works in {\em both}
cases. When $y = \sqrt{1 - x^2}$, we get $-x/\sqrt{1 - x^2}$, and when
$y = -\sqrt{1 - x^2}$, we get $x/\sqrt{1 - x^2}$. The method that we
used is called {\em implicit differentiation}.

So, with a functional description, we need to split into cases. With
an implicit description, we can give the general answer in one case,
but it involves {\em both} variables $x$ and $y$. With these examples
in mind, let's consider the parametric situation.

\subsection{Parametric differentiation}

For the curve given parametrically by $x = f(t), y = g(t), t \in D$,
where $f$ and $g$ are both differentiable at some point $t_0$ in the
interior of $D$, the derivative $dy/dx$ is given by:

$$\frac{dy}{dx}|_{t = t_0} = \frac{dy/dt}{dx/dt}|_{t = t_0} = \frac{g'(t)}{f'(t)}|_{t = t_0}$$

In short:

$$\frac{dy}{dx} = \frac{g'(t)}{f'(t)}$$

This is an application of the chain rule, or thinking of the
derivative as the relative rate of change. Basically, we are saying
that the relative rate of change of $y$ with respect to $x$ at time
$t_0$ is the rate of change of $y$ with respect to time divided by the
rate of change of $x$ with respect to time. That makes sense.

Also note that the answer that we get is {\em purely} a function of
$t$, and $x$ and $y$ do not appear in the description. This is
important at many levels. The first is that it takes account of
self-intersections. Remember that given a parametric description, the
same point on the curve can be repeated at different times, with the
curve moving in different directions at these different times. Since
the differentiation is with respect to $t$, {\em we can thus get
different derivative values at different times for the same physical
point location}.

Second, if given a point $(x,y)$, we first need to reverse engineer
and find the time $t$ that gave rise to the point $(x,y)$ and {\em
then} plug in the formula to find the derivative value.

Let's apply this to the circle $x^2 + y^2 = 1$ using the
parameterization $x = \cos t$, $y = \sin t$. We get:

$$\frac{dy}{dx} = \frac{\sin't}{\cos't} = \frac{\cos t}{-\sin t} = -\cot t$$

Now, suppose we want to find the derivative value at the point
$(1/2,\sqrt{3}/2)$. We first need to determine that this point arises
from $t = 2n\pi + \pi/3$. Note that since the curve repeats
periodically, the choice of $n$ is irrelevant, so we can choose $t =
\pi/3$. Then, the derivative value is $-\cot(\pi/3) = -1/\sqrt{3}$.

You can compare this to the derivative value obtained using implicit
differentiation -- $-x/y$ -- and see that they are in fact the same.

\subsection{Vertical/horizontal tangents and cusps}

Here $x = f(t)$, $y = g(t)$, and $t \in D$. Even if $f$ and $g$ are
both differentiable functions everywhere on the domain $D$, it is
still possible for there to be vertical/horizontal tangents and
cusps. Below are given some {\em sufficient} conditions for a
horizontal/vertical tangent.

\begin{itemize}
\item {\em Horizontal tangent}: This ocurs when $g'(t) = 0$ and $f'(t)
  \ne 0$. Note that the horizontal tangent itself could be of many
  kinds -- a local extremum type or a point of inflection type, or
  something different.
\item {\em Vertical tangent}: This occurs when $f'(t) = 0$ and $g'(t)
  \ne 0$. Again, there are many possibilities for this.
\end{itemize}

If both $f'(t)$ and $g'(t)$ are zero, then we need to determine which
is more zeroey, and also whether there is a sign change, etc. to
determine whether we have a horizontal/vertical tangent/cusp. We
refrain from the details here.
\subsection{Repeated differentiation}

Note that the second derivative $d^2y/dx^2$ is the derivative of the
first derivative and the roles of $x$ and $y$ are not symmetric. In
particular, the {\em freshman chain rule} does not work for second
derivatives -- we need something more sophisticated. Some of you have
worked out chain rule equivalents for second derivatives, e.g., we
have that $(f \circ g)'' = (f'' \circ g) \cdot (g')^2 + (f' \circ g)
\cdot (g'')$ and also that $(f^{-1})'' = -(f'' \circ f^{-1})/(f' \circ
f^{-1})^3$. You don't need to memorize these, since they can always be
derived from the rules for first derivatives. The key point is that
the {\em naive} chain rules are false. In particular, $d^2y/dx^2$ is
{\em not} the same as $(d^2y/dt^2)/(d^2x/dt^2)$.

So how do we differentiate twice in practice? We first compute the
first derivative in parametric form, and then differentiate that with
respect to $x$, again using parametric differentiation. Specifically,
if $x = f(t)$ and $y = g(t)$, we have:

$$\frac{dy}{dx} = \frac{g'(t)}{f'(t)}$$

Thus, we get:

$$\frac{d^2y}{dx^2} = \frac{\frac{d}{dt}\left(\frac{g'(t)}{f'(t)}\right)}{f'(t)}$$

where the denominator is just $dx/dt$. The same idea can be used to
calculate higher derivatives. Note that we could simplify the above as
a {\em general expression} using the quotient rule for
differentiation, and thus get another version of the same formula:

$$\frac{d^2y}{dx^2} = \frac{f'(t)g''(t) - g'(t)f''(t)}{(f'(t))^3}$$

It's up to you which form you choose to remember. I recommend that for
better conceptual understanding and generalizability, you definitely
remember the first version, since the second version can easily be
derived from it on the spot.

\subsection{Arc length, speed, and unit speed parameterization}

This is new stuff, in the sense that you have probably not seen this
even in the context of functional descriptions. It turns out that the
description is easier to give in parametric form anyway. Basically,
the arc length is given by:

$$\int \sqrt{(dx)^2 + (dy)^2}$$

More concretely, if $x = f(t)$, $y = g(t)$, then the arc length from time $t = a$ to $t = b$ (with $a < b$) is given by:

$$\int_a^b \sqrt{(dx/dt)^2 + (dy/dt)^2} \, dt$$

or equivalently:

$$\int_a^b \sqrt{(f'(t))^2 + (g'(t))^2} \, dt$$

The {\em speed} of a curve is defined as the speed with which it is
traveling with respect to $t$, and is hence simply:

$$\sqrt{(dx/dt)^2 + (dy/dt)^2}$$

In other words, if $x = f(t)$ and $y = g(t)$, the speed as a function
of time $t$ is:

$$\sqrt{(f'(t))^2 + (g'(t))^2}$$

We say that a particular parameterization of a curve is a {\em unit
speed parameterization} if the speed at any time $t$ is $1$. What this
means is that the arc length from $t = a$ to $t = b$ is simply $b -
a$. 

An example of a unit speed parameterization is the parameterization $x
= \cos t$, $y = \sin t$ for a circle. Another example is $x =
t/\sqrt{2}$, $y = t/\sqrt{2}$ for the line $y = x$.

Note that for the particular case where $y$ is a function of $x$, the
arc length from $x = a$ to $x = b$, with $a < b$, becomes:

$$\int_a^b \sqrt{1 + (dy/dx)^2} \, dx$$

\subsection*{Note on the difficulty of arc length computations}

Actually computing the arc length as a number is not easy in most
cases because the integration that we need to do is not possible
within the world of elementary integrations. In fact, whole new
branches of mathematics (such as the theory of elliptic integrals)
were invented to be able to perform some of the integrations that
arose in arc length computations.

Roughly, here are the kinds of curves for which arc lengths can be
computed:

\begin{itemize}
\item Straight lines (yeah, and we didn't even need calculus for those)
\item Circles (again, no need for calculus here -- the radian measure
  directly defines arc length).
\item Parabolas such as $x = y^2$ and $y = x^2$. To compute the arc
  length, we need to do a trigonometric or hyperbolic trigonometric
  substitution. The final answer we get involves something like
  integration $\sec^3 \theta$: For instance, consider $y = x^2$. We get:

  $$\int \sqrt{1 + (2x)^2} \, dx$$

  Put $\theta = \arctan(2x)$, so $x = (1/2)\tan \theta$, and we get:

  $$\frac{1}{2} \int \sqrt{1 + \tan^2\theta} \sec^2 \theta \, d\theta$$

  This simplifies to $(1/2) \int \sec^3\theta \, d\theta$, which can
  be done using integration by parts.

\item The graph of the hyperbolic cosine function (catenary): The
  derivative of $\cosh$ is $\sinh$, so we get:

  $$\int_a^b \sqrt{1 + (\sinh^2x)} \, dx$$

  This becomes:

  $$\int_a^b \cosh x \, dx$$

  giving $[\sinh x]_a^b = \sinh b - \sinh a$.
\end{itemize}

While this list above is by no means exhaustive, most of the
computable examples you will see will be variations on this theme.
\end{document}