\documentclass[10pt]{amsart}

%Packages in use
\usepackage{fullpage, hyperref, vipul, enumerate}

%Title details
\title{Class quiz: November 4: Integration}
\author{Math 152, Section 55 (Vipul Naik)}
%List of new commands

\begin{document}
\maketitle

Your name (print clearly in capital letters): $\underline{\qquad\qquad\qquad\qquad\qquad\qquad\qquad\qquad\qquad\qquad}$

\begin{enumerate}

\item Which of the following is an {\bf antiderivative} of $x\cos x$?

  \begin{enumerate}[(A)]
  \item $x \sin x + \cos x$
  \item $x \sin x - \cos x$
  \item $-x \sin x + \cos x$
  \item $-x \sin x - \cos x$
  \item None of the above
  \end{enumerate}

  \vspace{0.1in}
  Your answer: $\underline{\qquad\qquad\qquad\qquad\qquad\qquad\qquad}$
  \vspace{1in}

\item (*) Suppose $F$ and $G$ are two functions defined on $\R$ and $k$ is
  a natural number such that the $k^{th}$ derivatives of $F$ and $G$
  exist and are equal on all of $\R$. Then, $F - G$ must be a
  polynomial function. What is the {\bf maximum possible degree} of $F
  - G$?  (Note: Assume constant polynomials to have degree zero)

  \begin{enumerate}[(A)]
  \item $k - 2$
  \item $k - 1$
  \item $k$
  \item $k + 1$
  \item There is no bound in terms of $k$.
  \end{enumerate}

  \vspace{0.1in}
  Your answer: $\underline{\qquad\qquad\qquad\qquad\qquad\qquad\qquad}$
  \vspace{1in}

\item (**) Suppose $f$ is a continuous function on $\R$. Clearly, $f$ has
  antiderivatives on $\R$. For all but one of the following
  conditions, it is possible to guarantee, without any further
  information about $f$, that there exists an antiderivative $F$
  satisfying that condition. {\bf Identify the exceptional condition}
  (i.e., the condition that it may not always be possible to satisfy).

  \begin{enumerate}[(A)]
  \item $F(1) = F(0)$.
  \item $F(1) + F(0) = 0$.
  \item $F(1) + F(0) = 1$.
  \item $F(1) = 2F(0)$.
  \item $F(1)F(0) = 0$.
  \end{enumerate}

  \vspace{0.1in}
  Your answer: $\underline{\qquad\qquad\qquad\qquad\qquad\qquad\qquad}$
  \vspace{1in}

\item (**) Suppose $F(x) = \int_0^x \sin^2(t^2) \, dt$ and $G(x) = \int_0^x
  \cos^2(t^2) \, dt$. Which of the following {\bf is true}?

  \begin{enumerate}[(A)]
  \item $F + G$ is the zero function.
  \item $F + G$ is a constant function with nonzero value.
  \item $F(x) + G(x) = x$ for all $x$.
  \item $F(x) + G(x) = x^2$ for all $x$.
  \item $F(x^2) + G(x^2) = x$ for all $x$.
  \end{enumerate}

  \vspace{0.1in}
  Your answer: $\underline{\qquad\qquad\qquad\qquad\qquad\qquad\qquad}$
  \vspace{1in}

\item (**) Suppose $F$ is a function defined on $\R \setminus \{ 0 \}$ such
  that $F'(x) = -1/x^2$ for all $x \in \R \setminus \{ 0 \}$. Which of
  the following pieces of information is/are {\bf sufficient} to determine
  $F$ completely?
  \begin{enumerate}[(A)]
  \item The value of $F$ at any two positive numbers.
  \item The value of $F$ at any two negative numbers.
  \item The value of $F$ at a positive number and a negative number.
  \item Any of the above pieces of information is sufficient, i.e., we
    need to know the value of $F$ at any two numbers.
  \item None of the above pieces of information is sufficient.
  \end{enumerate}

  \vspace{0.1in}
  Your answer: $\underline{\qquad\qquad\qquad\qquad\qquad\qquad\qquad}$
  \vspace{1in}



\end{enumerate}

\end{document}
