\documentclass[10pt]{amsart}

%Packages in use
\usepackage{fullpage, hyperref, vipul, enumerate}

%Title details
\title{Take-home class quiz: due Friday February 1: Limits}
\author{Math 195, Section 59 (Vipul Naik)}
%List of new commands

\begin{document}
\maketitle

Your name (print clearly in capital letters): $\underline{\qquad\qquad\qquad\qquad\qquad\qquad\qquad\qquad\qquad\qquad}$

This is based on the concept of limits you have seen in single
variable calculus. You can view video playlists of the material here:

\url{http://www.youtube.com/playlist?list=PL8483BCA409563C88}

\url{http://www.youtube.com/playlist?list=PLC0bHnWu122lmsGOHv39OSaNwD8MXvlTH}

\url{http://www.youtube.com/playlist?list=PLC0bHnWu122lZYgoCzVmWtnMAmUO6RDXC}

{\bf PLEASE FEEL FREE TO DISCUSS ALL QUESTIONS, BUT PLEASE ONLY ENTER
FINAL ANSWER OPTIONS THAT YOU PERSONALLY CONSIDER MOST LIKELY TO BE
CORRECT -- DO NOT ENGAGE IN GROUPTHINK.}

\begin{enumerate}

\item We call a function $f$ left continuous on an open interval $I$
  if, for all $a \in I$, $\displaystyle \lim_{x \to a^-} f(x) =
  f(a)$. Which of the following is an example of a function that is
  left continuous but not continuous on $(0,1)$? If all are examples,
  please select Option (E).

  \begin{enumerate}[(A)]
  \item $f(x) := \left \lbrace\begin{array}{rl}x, & 0 < x \le 1/2 \\ 2x, & 1/2 < x < 1 \\\end{array}\right.$
  \item $f(x) := \left \lbrace\begin{array}{rl}x, & 0 < x < 1/2 \\ 2x, & 1/2 \le x < 1 \\\end{array}\right.$
  \item $f(x) := \left \lbrace\begin{array}{rl}x, & 0 < x \le 1/2 \\ 2x - (1/2), & 1/2 < x < 1 \\\end{array}\right.$
  \item $f(x) := \left \lbrace\begin{array}{rl}x, & 0 < x < 1/2 \\ 2x - (1/2), & 1/2 \le x < 1 \\\end{array}\right.$
  \item All of the above
  \end{enumerate}

  \vspace{0.1in}
  Your answer: $\underline{\qquad\qquad\qquad\qquad\qquad\qquad\qquad}$
  \vspace{0.1in}
  
\item Suppose $f$ and $g$ are functions $(0,1)$ to $(0,1)$ that are
  both left continuous on $(0,1)$. Which of the following is {\em not}
  guaranteed to be left continuous on $(0,1)$? Please see Option (E)
  before answering.

  \begin{enumerate}[(A)]
  \item $f + g$, i.e., the function $x \mapsto f(x) + g(x)$
  \item $f - g$, i.e., the function $x \mapsto f(x) - g(x)$
  \item $f \cdot g$, i.e., the function $x \mapsto f(x)g(x)$
  \item $f \circ g$, i.e., the function $x \mapsto f(g(x))$
  \item None of the above, i.e., they are all guaranteed to be left
  continuous functions
  \end{enumerate}

  \vspace{0.1in}
  Your answer: $\underline{\qquad\qquad\qquad\qquad\qquad\qquad\qquad}$
  \vspace{0.1in}

\item Which of these is the correct interpretation of $\displaystyle
  \lim_{x \to c} f(x) = L$ in terms of the definition of limit? Please
  see Option (E) before answering.

  \begin{enumerate}[(A)]
  \item For every $\alpha > 0$, there exists $\beta > 0$ such that if
    $0 < |x - c| < \alpha$, then $|f(x) - L| < \beta$.
  \item There exists $\alpha > 0$ such that for every $\beta > 0$, and
    $0 < |x - c| < \alpha$, we have $|f(x) - L| < \beta$.
  \item For every $\alpha > 0$, there exists $\beta > 0$ such that if
    $0 < |x - c| < \beta$, then $|f(x) - L| < \alpha$.
  \item There exists $\alpha > 0$ such that for every $\beta > 0$ and
    $0 < |x - c| < \beta$, we have $|f(x) - L| < \alpha$.
  \item None of the above
  \end{enumerate}

  \vspace{0.1in}
  Your answer: $\underline{\qquad\qquad\qquad\qquad\qquad\qquad\qquad}$
  \vspace{0.5in}

\item Suppose $f:\R \to \R$ is a function. Which of the following says
  that $f$ does not have a limit at any point in $\R$ (i.e., there is
  no point $c \in \R$ for which $\displaystyle \lim_{x \to c} f(x)$
  exists)? If all, please select Option (E).

  \begin{enumerate}[(A)]
  \item For every $c \in \R$, there exists $L \in \R$ such that for
    every $\varepsilon > 0$, there exists $\delta > 0$ such that for all
    $x$ satisfying $0 < |x - c| < \delta$, we have $|f(x) - L| \ge
    \varepsilon$.
  \item There exists $c \in \R$ such that for every $L \in \R$, there
    exists $\varepsilon > 0$ such that for every $\delta > 0$, there exists
    $x$ satisfying $0 < |x - c| < \delta$ and $|f(x) - L| \ge \varepsilon$.
  \item For every $c \in \R$ and every $L \in \R$, there exists
    $\varepsilon > 0$ such that for every $\delta > 0$, there exists $x$
    satisfying $0 < |x - c| < \delta$ and $|f(x) - L| \ge \varepsilon$.
  \item There exists $c \in \R$ and $L \in \R$ such that for
    every $\varepsilon > 0$, there exists $\delta > 0$ such that for all
    $x$ satisfying $0 < |x - c| < \delta$, we have $|f(x) - L| \ge
    \varepsilon$.
  \item All of the above.
  \end{enumerate}

  \vspace{0.1in}
  Your answer: $\underline{\qquad\qquad\qquad\qquad\qquad\qquad\qquad}$
  \vspace{0.5in}

\item In the usual $\varepsilon-\delta$ definition of limit for a given
  limit $\displaystyle \lim_{x \to c} f(x) = L$, if a given value $\delta > 0$ works
  for a given value $\varepsilon > 0$, then which of the following is
  true? Please see Option (E) before answering.

  \begin{enumerate}[(A)]
  \item Every smaller positive value of $\delta$ works for the same
    $\varepsilon$. Also, the given value of $\delta$ works for every
    smaller positive value of $\varepsilon$.
  \item Every smaller positive value of $\delta$ works for the same
    $\varepsilon$. Also, the given value of $\delta$ works for every
    larger value of $\varepsilon$.
  \item Every larger value of $\delta$ works for the same
    $\varepsilon$. Also, the given value of $\delta$ works for every
    smaller positive value of $\varepsilon$.
  \item Every larger value of $\delta$ works for the same
    $\varepsilon$. Also, the given value of $\delta$ works for every
    larger value of $\varepsilon$.
  \item None of the above statements need always be true.
  \end{enumerate}

  \vspace{0.1in}
  Your answer: $\underline{\qquad\qquad\qquad\qquad\qquad\qquad\qquad}$
  \vspace{0.5in}

\item Which of the following is a correct formulation of the statement
  $\displaystyle \lim_{x \to c} f(x) = L$, in a manner that avoids the use of
  $\varepsilon$s and $\delta$s? Please see Option (E) before answering.

  \begin{enumerate}[(A)]
  \item For every open interval centered at $c$, there is an open
    interval centered at $L$ such that the image under $f$ of the open
    interval centered at $c$ (excluding the point $c$ itself) is
    contained in the open interval centered at $L$.
  \item For every open interval centered at $c$, there is an open
    interval centered at $L$ such that the image under $f$ of the open
    interval centered at $c$ (excluding the point $c$ itself) contains
    the open interval centered at $L$.
  \item For every open interval centered at $L$, there is an open
    interval centered at $c$ such that the image under $f$ of the open
    interval centered at $c$ (excluding the point $c$ itself) is
    contained in the open interval centered at $L$.
  \item For every open interval centered at $L$, there is an open
    interval centered at $c$ such that the image under $f$ of the open
    interval centered at $c$ (excluding the point $c$ itself) contains
    the open interval centered at $L$.
  \item None of the above.
  \end{enumerate}

  \vspace{0.1in}
  Your answer: $\underline{\qquad\qquad\qquad\qquad\qquad\qquad\qquad}$
  \vspace{0.5in}

\item Consider the function:

  $$f(x) := \left \lbrace\begin{array}{rl} x, & x \text{ rational }\\1/x, & x \text{ irrational }\\\end{array}\right.$$

  What is the set of all points at which $f$ is continuous? 

  \begin{enumerate}[(A)]
  \item $\{ 0, 1 \}$
  \item $\{ -1,1 \}$
  \item $\{-1,0 \}$
  \item $\{ -1,0,1 \}$
  \item $f$ is continuous everywhere
  \end{enumerate}

  \vspace{0.1in}
  Your answer: $\underline{\qquad\qquad\qquad\qquad\qquad\qquad\qquad}$
  \vspace{0.1in}

\item The graph $y = f(x)$ of a function $f$ defined on all reals has
  a horizontal asymptote $y = c$ as $x$ approaches $+\infty$. Which of
  the following is the correct definition of this?

  \begin{enumerate}[(A)]
  \item For every $a \in \R$, there exists $\delta > 0$ such that for
    all $x$ satisfying $0 < |x - c| < \delta$, we have $f(x) > a$.
  \item For every $a \in \R$, there exists $\varepsilon > 0$ such that
    for all $x$ satisfying $x > a$, we have $|f(x) - c| < \varepsilon$.
  \item For every $\varepsilon > 0$, there exists $a \in \R$ such that
    for all $x$ satisfying $x > a$, we have $|f(x) - c| < \varepsilon$.
  \item For every $\delta > 0$, there exists $a \in \R$ such that for
    all $x$ satisfying $0 < |x - c| < \delta$, we have $f(x) > a$.
  \item For every $\varepsilon > 0$, there exists $\delta > 0$ such
    that for all $x$ satisfying $0 < |x - c| < \delta$, we have $|f(x)
    - c| < \varepsilon$.
  \end{enumerate}

  \vspace{0.1in}
  Your answer: $\underline{\qquad\qquad\qquad\qquad\qquad\qquad\qquad}$
  \vspace{0.1in}

\item Which of the following is the correct definition of
  $\displaystyle \lim_{x \to c^-} f(x) = -\infty$ (in words: the left
  hand limit of $f$ at $c$ is $-\infty$)?

  \begin{enumerate}[(A)]
  \item For every $a \in \R$, there exists $\delta > 0$ such that for
    all $x$ satisfying $0 < |x - c| < \delta$, we have $f(x) > a$.
  \item For every $a \in \R$, there exists $\delta > 0$ such that for
    all $x$ satisfying $0 < x - c < \delta$, we have $f(x) > a$.
  \item For every $a \in \R$, there exists $\delta > 0$ such that for
    all $x$ satisfying $0 < x - c < \delta$, we have $f(x) < a$.
  \item For every $a \in \R$, there exists $\delta > 0$ such that for
    all $x$ satisfying $0 < c - x < \delta$, we have $f(x) > a$.
  \item For every $a \in \R$, there exists $\delta > 0$ such that for
    all $x$ satisfying $0 < c - x < \delta$, we have $f(x) < a$.
  \end{enumerate}

  \vspace{0.1in}
  Your answer: $\underline{\qquad\qquad\qquad\qquad\qquad\qquad\qquad}$
  \vspace{0.1in}

\item Suppose $f$ is a function defined on all of $\R$ and $c \in
  \R$. Which of the following is the correct $\varepsilon-\delta$
  definition for the statement ``$f$ is differentiable at $c$''?
  

  \begin{enumerate}[(A)]
  \item For every $L \in \R$, there exists $\varepsilon > 0$ such that
    for every $\delta > 0$, there exists $x$ such that $0 < |x - c| <
    \delta$ and $|f(x) - f(c) - L(x - c)| \ge |x - c|\varepsilon$.
  \item For every $L \in \R$, there exists $\varepsilon > 0$ such that
    for every $\delta > 0$, there exists $x$ such that $0 < |x - c| <
    \delta$ and $|f(x) - f(c) - L(x - c)| < |x - c|\varepsilon$.
  \item There exists $L \in \R$ such that for every $\varepsilon > 0$,
  there exists $\delta > 0$ such that for every $x$ satisfying $0 < |x
  - c| < \delta$, we have $|f(x) - f(c) - L(x - c)| \ge |x - c|\varepsilon$
  \item There exists $L \in \R$ such that for every $\varepsilon > 0$,
  there exists $\delta > 0$ such that for every $x$ satisfying $0 < |x
  - c| < \delta$, we have $|f(x) - f(c) - L(x - c)| < |x - c|\varepsilon$
  \item There exists $L \in \R$ such that there exists $\varepsilon > 0$
    such that for every $\delta > 0$, there exists $x$ such that $0 <
    |x - c| < \delta$ and $|f(x) - f(c) - L(x - c)| < |x - c|\varepsilon$.
  \end{enumerate}

  \vspace{0.1in}
  Your answer: $\underline{\qquad\qquad\qquad\qquad\qquad\qquad\qquad}$
  \vspace{0.1in}

\item Suppose $f$ is a function defined on all of $\R$ and $c \in
  \R$. Which of the following is the correct $\varepsilon-\delta$
  definition for the statement ``$f$ is not differentiable at $c$''?

  \begin{enumerate}[(A)]
  \item For every $L \in \R$, there exists $\varepsilon > 0$ such that
    for every $\delta > 0$, there exists $x$ such that $0 < |x - c| <
    \delta$ and $|f(x) - f(c) - L(x - c)| \ge |x - c|\varepsilon$.
  \item For every $L \in \R$, there exists $\varepsilon > 0$ such that
    for every $\delta > 0$, there exists $x$ such that $0 < |x - c| <
    \delta$ and $|f(x) - f(c) - L(x - c)| < |x - c|\varepsilon$.
  \item There exists $L \in \R$ such that for every $\varepsilon > 0$,
  there exists $\delta > 0$ such that for every $x$ satisfying $0 < |x
  - c| < \delta$, we have $|f(x) - f(c) - L(x - c)| \ge |x - c|\varepsilon$
  \item There exists $L \in \R$ such that for every $\varepsilon > 0$,
  there exists $\delta > 0$ such that for every $x$ satisfying $0 < |x
  - c| < \delta$, we have $|f(x) - f(c) - L(x - c)| < |x - c|\varepsilon$
  \item There exists $L \in \R$ such that there exists $\varepsilon > 0$
    such that for every $\delta > 0$, there exists $x$ such that $0 <
    |x - c| < \delta$ and $|f(x) - f(c) - L(x - c)| < |x - c|\varepsilon$.
  \end{enumerate}

  \vspace{0.1in}
  Your answer: $\underline{\qquad\qquad\qquad\qquad\qquad\qquad\qquad}$
  \vspace{0.1in}

\item Suppose $f:\R \to \R$ is a function. Identify which of these
  definitions is {\em not} correct for $\displaystyle \lim_{x \to c}
  f(x) = L$, where $c$ and $L$ are both finite real numbers. If all
  are correct, please select Option (E).

  \begin{enumerate}[(A)]
  \item For every $\varepsilon > 0$, there exists $\delta > 0$ such
    that if $x \in (c - \delta, c + \delta) \setminus \{ c \}$, then
    $f(x) \in (L - \varepsilon, L + \varepsilon)$.
  \item For every $\varepsilon_1 > 0$ and $\varepsilon_2 > 0$, there exist
    $\delta_1 > 0$ and $\delta_2 > 0$ such that if $x \in (c -
    \delta_1,c+\delta_2)\setminus \{ c \}$, then $f(x) \in (L -
    \varepsilon_1, L + \varepsilon_2)$.
  \item For every $\varepsilon_1 > 0$ and $\varepsilon_2 > 0$, there exists
    $\delta > 0$ such that if $x \in (c - \delta, c + \delta)
    \setminus \{ c \}$, then $f(x) \in (L - \varepsilon_1, L + \varepsilon_2)$.
  \item For every $\varepsilon > 0$, there exist $\delta_1 > 0$ and
    $\delta_2 > 0$ such that if $x \in (c - \delta_1, c + \delta_2)
    \setminus \{ c \}$, then $f(x) \in (L - \varepsilon, L + \varepsilon)$.
  \item None of these, i.e., all definitions are correct.
  \end{enumerate}

  \vspace{0.1in}
  Your answer: $\underline{\qquad\qquad\qquad\qquad\qquad\qquad\qquad}$
  \vspace{0.1in}

\item In the usual $\varepsilon-\delta$ definition of limit, we find that
  the value $\delta = 0.2$ for $\varepsilon = 0.7$ for a function $f$ at
  $0$, and the value $\delta = 0.5$ works for $\varepsilon = 1.6$ for a
  function $g$ at $0$. What value of $\delta$ {\em definitely} works
  for $\varepsilon = 2.3$ for the function $f + g$ at $0$?

  \begin{enumerate}[(A)]
  \item $0.2$
  \item $0.3$
  \item $0.5$
  \item $0.7$
  \item $0.9$
  \end{enumerate}

  \vspace{0.1in}
  Your answer: $\underline{\qquad\qquad\qquad\qquad\qquad\qquad\qquad}$
  \vspace{0.1in}

\item The sum of limits theorem states that $\displaystyle \lim_{x \to c} [f(x) +
  g(x)] = \displaystyle \lim_{x \to c} f(x) + \displaystyle \lim_{x \to c} g(x)$ {\em if} the
  right side is defined. One of the choices below gives an example
  where the left side of the equality is defined and finite but the right side
  makes no sense. Identify the correct choice.

  \begin{enumerate}[(A)]
  \item $f(x) := 1/x$, $g(x) := -1/(x + 1)$, $c = 0$.
  \item $f(x) := 1/x$, $g(x) := (x - 1)/x$, $c = 0$.
  \item $f(x) := \arcsin x$, $g(x) := \arccos x$, $c = 1/2$.
  \item $f(x) := 1/x$, $g(x) = x$, $c = 0$.
  \item $f(x) := \tan x$, $g(x) := \cot x$, $c = 0$.
  \end{enumerate}

  \vspace{0.1in}
  Your answer: $\underline{\qquad\qquad\qquad\qquad\qquad\qquad\qquad}$
  \vspace{0.1in}

\end{enumerate}
\end{document}