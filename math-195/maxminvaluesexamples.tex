\documentclass[10pt]{amsart}
\usepackage{fullpage,hyperref,vipul,graphicx}
\title{Maximum and minimum values: examples}
\author{Math 195, Section 59 (Vipul Naik)}

\begin{document}
\maketitle

{\bf What students should hopefully get}: The description of critical
points, local extreme values, and absolute extreme values for
additively separable functions and (the more complicated version for)
multiplicatively separable functions. The special nature of extreme
values for quasiconvex and strictly quasiconvex functions and the
notion of extreme points. The nature of extreme values for linear,
quadratic, and homogeneous polynomials. The use of Lagrange
multipliers to find extrema on the boundary.
\section*{Executive summary}

\begin{enumerate}
\item {\em Additively separable, critical points}: For an additively
  separable function $F(x,y) := f(x) + g(y)$, the critical points of
  $F$ are the points whose $x$-coordinate gives a critical point for
  $f$ and $y$-coordinate gives a critical point for $g$.
\item {\em Additively separable, local extreme values}: The local
  maxima occur at points whose $x$-coordinate gives a local maximum
  for $f$ and $y$-coordinates gives a local maximum for $g$. Similarly
  for local minima. If one coordinate gives a local maximum and the
  other coordinate gives a local minimum, we get a saddle point.
\item {\em Additively separable, absolute extreme values}: If the
  domain is a rectangular region, rectangular strip, or the whole
  plane, then the absolute maximum occurs at the point for which each
  coordinate gives the absolute maximum for that coordinate, and
  analogously for absolute minimum. This does {\em not} work for
  non-rectangular regions in general.
\item {\em Multiplicatively separable, critical points}: For a
  multiplicatively separable function $F(x,y) := f(x)g(y)$ with $f$,
  $g$, differentiable, there are four kinds of critical points
  $(x_0,y_0)$: (1) $f'(x_0) = g'(y_0) = 0$, (2) $f(x_0) = f'(x_0) =
  0$, (3) $g(y_0) = g'(y_0) = 0$, (4) $f(x_0) = g(y_0) = 0$.
\item {\em Multiplicatively separable, local extreme values}: At a
  critical point of Type (1), the nature of local extreme value for
  $F$ depends on the signs of $f$ and $g$ {\em and} on the nature of
  local extreme values for each. See the table. Critical points of
  Type (4) alone do not give local extreme values. The situation with
  critical points of Types (2) and (3) is more ambiguous and too
  complicated for discussion.
\item {\em Multiplicatively separable, absolute extreme values}:
  Often, these don't exist, if one function takes arbitrarily large
  magnitude values and the other one takes nonzero values (details
  based on sign). If both functions are everywhere positive, and we
  are on a rectangular region, then the absolute maximum/minimum for
  the product occur at points whose coordinates give respective
  absolute maximum/minimum for $f$ and $g$. (See notes)
\item For a continuous quasiconvex function on a convex domain, the
  maximum must occur at one of the extreme points, in
  particular on the boundary. If the function is strictly quasiconvex,
  the maximum can occur only at a boundary point.
\item For a continuous quasiconvex function on a convex domain, the
  minimum must occur on a convex subset. If the function is strictly
  quasiconvex, it must occur at a unique point.
\item Linear functions are quasiconvex but not strictly so. The
  negative of a linear function is also quasiconvex. The maximum and
  minimum for linear functions on convex domains must occur at extreme
  points.
\item To find maxima/minima on the boundary, we can use the method of
  Lagrange multipliers.
\end{enumerate}

See also: tables, discussion for linear, quadratic, and homogeneous
functions (hard to summarize).

\section{Additively separable functions}

\subsection{Partial derivatives, critical points and Hessian}

Consider a function of the form $F(x,y) := f(x) + g(y)$, i.e., $F$ is
additively separable. Then, $F_x(x,y) = f'(x)$, $F_y(x,y) = g'(y)$,
$F_{xx}(x,y) = f''(x)$, $F_{yy}(x,y) = g''(y)$, and $F_{xy}(x,y) = 0$.

Note that in this case, the Hessian (the determinant used to determine
the nature of extreme value at a critical point) is simply the product
$F_{xx}F_{yy} = f''(x)g''(y)$.

We have the following:

\begin{quote}
  The critical points for $F$ are precisely the points whose
  $x$-coordinate gives a critical point for $f$ and $y$-coordinate
  gives a critical point for $g$. In other words, $(x_0,y_0)$ in the
  domain of $F$ gives a critical point if and only if $x_0$ gives a
  critical point for $f$ and $y_0$ gives a critical point for $g$.
\end{quote}

\subsection{Local extreme values}

We can say the following about local extreme values:

\begin{itemize}
\item The points of local maximum for $F$ are precisely the points
  whose $x$-coordinate gives a local maximum for $f$ and whose
  $y$-coordinate gives a local maximum for $g$. In other words,
  $(x_0,y_0)$ in the domain of $F$ gives a local maximum if and only
  if $x_0$ gives a local maximum for $f$ and $y_0$ gives a local
  maximum for $g$.
\item The points of local minimum for $F$ are precisely the points
  whose $x$-coordinate gives a local minimum for $f$ and whose
  $y$-coordinate gives a local minimum for $g$. In other words,
  $(x_0,y_0)$ in the domain of $F$ gives a local minimum if and only
  if $x_0$ gives a local minimum for $f$ and $y_0$ gives a local
  minimum for $g$.
\item In particular, a critical point gives a saddle point if any of
  these conditions hold: it is not a point of local extremum in one of
  the variables (e.g., a point of inflection type), {\em or} it is a
  point of local extrema of opposite kinds in the two variables. For
  instance, for the function $(x - 1)^2 - (y - 2)^2$, the point
  $(1,2)$ is a point of local minimum for the first coordinate but
  local maximum for the second, so we get a saddle point overall.
\item For an additively separable function, the second derivative test
  simply boils down to checking whether $f''(x)$ and $g''(y)$ have the
  same sign. This is because the {\em interaction term} arising as a
  {\em mixed partial} is {\em absent}.
\end{itemize}

\subsection{Absolute extreme values}

Continuing notation from above, we note that:

\begin{itemize}
\item If the domain of $F$ is rectangular (or the whole plane or a
  rectangular infinite strip) then the absolute maximum value for $F$
  occurs at a point whose $x$-coordinate maximizes $f$ and whose
  $y$-coordinate maximizes $g$.
\item If the domain of $F$ is rectangular (or the whole plane or a
  rectangular infinite strip) then the absolute minimum value for $F$
  occurs at a point whose $x$-coordinate minimizes $f$ and whose
  $y$-coordinate minimizes $g$.
\end{itemize}

These results don't hold for non-rectangular domains because we cannot
carry out separate analysis of the variables. For instance, consider
the function $x + y$ on the circular disk $x^2 + y^2 \le 1$. The
maximum for $x$ occurs at $x = 1$, and the maximum for $y$ occurs at
$y = 1$. However, the point $(1,1)$ lies outside the domain of the
function.

We will deal with non-rectangular regions in more detail a little later.

\subsection{Examples}

Consider the function:

$$F(x,y) := x^2 - 3x + \sin^2y$$

This is the sum of the functions $f(x) := x^2 - 3x$ and $g(y) :=
\sin^2y$. $f$ attains its local and absolute minimum at $x_0 = 3/2$
with value $-9/4$, and it has no local or absolute maximum. $g$
attains its local and absolute minima at multiples of $\pi$ with value
$0$, and its local and absolute maxima at odd multiples of $\pi/2$,
with value $1$.

The upshot is that:

\begin{itemize}
\item $F$ attains its local and absolute minima at points of the form
  $(3/2,n\pi)$, $n$ an integer. This is because $f$ is minimum on the
  $x$-coordinate and $g$ is minimum on the $y$-coordinate.
\item $F$ has saddle points at $(3/2,n\pi + \pi/2)$. This is because
  $f$ is minimum on the $x$-coordinate and $g$ is maximum on the
  $y$-coordinate.
\end{itemize}

\subsection{Key observation: cases where second derivative test doesn't work}

Consider the additively separable function:

$$F(x,y) = (x-1)^3 - (y-2)^2$$

The function has a unique critical point for the point $(1,2)$ in the
domain. If we didn't notice additive separability, and directly tried
to compute the Hessian, we'd get $0$, indicating that the second
derivative test is inconclusive.

We note that this is the sum of the functions $f(x) := (x-1)^3$ and
$g(y) := -(y-2)^2$. Since we are now dealing with functions of {\em one}
variable, we have methods other than the second derivative test (for
instance, the first derivative test or higher derivative tests) to
find out whether a given critical point gives a local extreme value. In
this case we figure that $x = 1$ gives a point of inflection and {\em
not} a local extreme value for $f$, whereas $y = 2$ gives a local
maximum for $g$. Thus, overall, we conclude that $(1,2)$ gives a
saddle point.

In other words, for additively separable functions, we can go beyond
the second derivative test using our knowledge of functions of one
variable, despite our ignorance of analogous results for functions of
two variables.
\section{Multiplicatively separable functions}
\subsection{Partial derivatives, critical points, and Hessian}

Consider a function of the form $F(x,y) := f(x)g(y)$, i.e., $F$ is
multiplicatively separable. Then, $F_x(x,y) = f'(x)g(y)$, $F_y(x,y) =
f(x)g'(y)$, $F_{xx}(x,y) = f''(x)g(y)$, $F_{xy}(x,y) = f'(x)g'(y)$,
and $F_{yy}(x,y) = f(x)g''(y)$. 

The Hessian determinant (used for the second derivative test) at a
point $(x_0,y_0)$ thus becomes:

$$f(x_0)g(y_0)f''(x_0)g''(y_0) - [f'(x_0)g'(y_0)]^2$$

We first try to figure out the necessary and sufficient conditions for
a point $(x_0,y_0)$ to be a critical point for $F$. This happens iff
$f'(x_0)g(y_0) = 0$ and $f(x_0)g'(y_0) = 0$. This could occur for four
different reasons. We provide each reason along with an
interpretation.

\begin{enumerate}
\item $f'(x_0) = g'(y_0) = 0$: This means that $x_0$ is a critical
  point for $f$ and $y_0$ is a critical point for $g$.
\item $f(x_0) = f'(x_0) = 0$: Note that in this case the partial with
  respect to $y$ is $0$ at the point, not because of $y_0$, but
  because of $x_0$. What's happening is that on the line $x = x_0$,
  the function is identically zero, so changes in $g$ do not matter.
\item $g(y_0) = g'(y_0) = 0$: Note that in this case the partial with
  respect to $x$ is $0$ at the point, not because of $x_0$, but
  because of $y_0$. What's happening is that on the line $y = y_0$,
  the function is identically zero, so changes in $f$ do not matter.
\item $f(x_0) = 0$ and $g(y_0) = 0$: In this case, the function is
  identically zero along both the vertical and the horizontal line
  containing $(x_0,y_0)$.
\end{enumerate}

Note that any critical point that is of Type (4) above but not any of
the preceding types must {\em fail} the second derivative test. For a
critical point of Type (2) or (3) above, the second derivative test is
inconclusive because we get $0$ (more is discussed in the next
subsection). For a critical point of Type (1), the second derivative
test is most useful. Note that for such a critical point, the Hessian
determinant simply becomes $f(x_0)g(y_0)f''(x_0)g''(y_0)$, so its sign
depends not only on the signs of the second derivatives of $f$ and $g$
but {\em also} on the signs of the functions $f$ and $g$ themselves.

\subsection{Local extreme values: Type 1 case}

The following table gives conclusions for the nature of local extreme
values of $F(x,y) = f(x)g(y)$ at $(x_0,y_0)$ if $x_0$ gives a local
extreme value for $f$ and $y_0$ gives a local extreme value for $g$.

\begin{tabular}{|l|l|l|l|l|}
  \hline
  $f(x_0)$ sign & $g(y_0)$ sign & $f(x_0)$ (local max/min) & $g(y_0)$ (local max/min) & $F(x_0,y_0)$ (local max/min/saddle)\\
  \hline
  positive & positive & local max & local max & local max\\
  \hline
  positive & positive & local max & local min & saddle point \\
  \hline
  positive & positive & local min & local max & saddle point \\
  \hline
  positive & positive & local min & local min & local min \\
  \hline
  positive & negative & local max & local max & saddle point\\
  \hline
  positive & negative & local max & local min & local min\\
  \hline
  positive & negative & local min & local max & local max\\
  \hline
  positive & negative & local min & local min & saddle point\\
  \hline
  negative & positive & local max & local max & saddle point\\
  \hline
  negative & positive & local max & local min & local max \\
  \hline
  negative & positive & local min & local max & local min \\
  \hline
  negative & positive & local min & local min & saddle point \\
  \hline
  negative & negative & local max & local max & local min \\
  \hline
  negative & negative & local max & local min & saddle point \\
  \hline
  negative & negative & local min & local max & saddle point \\
  \hline
  negative & negative & local min & local min & local max \\
  \hline
\end{tabular}

Note that the conclusion about $F$ depends not merely on whether $f$
and $g$ have local max/min but also on the sign of the local max/min
for $f$. The {\em saddle point cases} arise when $f$ and $g$ are
pulling (multiplicatively) in opposite directions. Here, the function
is a local maximum along one of the $x$- and $y$-directions and a
local minimum along the other.

{\em In cases where the second derivative test is conclusive for both
$f$ and $g$ as functions of one variable}, the above observations can
be cross-checked by looking at the sign of the Hessian, which is
$f(x_0)g(y_0)f''(x_0)g''(y_0)$, and of $F_{xx} = g(y_0)f''(x_0)$ and
$F_{yy} = f(x_0)g''(y_0)$.

We do two of the sixteen examples for illustration:

\begin{itemize}
\item If $f$ has a positive local maximum at $x_0$ and $g$ has a
  negative local maximum at $y_0$, then $F$ has a saddle point: We get
  $f(x_0) > 0$, $g(y_0) < 0$, $f''(x_0) < 0$, $g''(y_0) < 0$. So, the
  Hessian is negative (product of one positive and three
  negatives). Thus, by the second derivative test, $F$ has a saddle
  point.
\item If $f$ and $g$ both have negative local minima, then $F$ has a
  local maximum. Here's how we see this: we get $f(x_0) < 0$, $g(y_0)
  < 0$, $f''(x_0) > 0$ (local minimum), $g''(y_0) > 0$ (local
  minimum), so multiplying all the signs, we see that the Hessian is
  positive. Thus, the function does attain a local extreme
  value. Next, we look at the sign of $F_{xx}(x_0,y_0)$, which is
  $g(y_0)f''(x_0)$. This is negative, since it is the product of a
  negative and a positive number. Thus, $F$ has a local minimum.
\end{itemize}

It's important to keep in mind that the statements in the table {\em
are more general and apply even when the second derivative tests are
inconclusive}. We'll be looking at some examples shortly.

\subsection{Critical points of Types 2 and 3}

We now turn to the situation $F(x,y) = f(x)g(y)$ where there are
points $x_0$ satisfying $f(x_0) = f'(x_0) = 0$. In this case, the
second derivative test is inconclusive because the Hessian determinant
takes the value $0$.

Let's try to examine what's happening near the point. On the line $x =
x_0$, the function is constant at $0$. That explains why $F_y(x_0,y_0)
= 0$ -- the function is not changing along the $y$-direction because
it's the product of $f(x_0)$, which is zero, and a changing number. On
the line $y = y_0$, the function has derivative zero because $f'(x_0)
= 0$, so that is why $F_x(x_0,y_0) = 0$.

Now, the first condition we need to obtain a local minimum is that the
function be a local minimum under slight perturbations in the
$x$-direction. So, we would like that the function $x \mapsto
f(x)g(y_0)$ have a local minimum at $x_0$. If $g(y_0) > 0$, this is
equivalent to wanting $f$ to have a local minimum at $x_0$. If $g(y_0)
< 0$, this is equivalent to wanting $f$ to have a local maximum at
$x_0$. In fact, as long as $g(y_0) \ne 0$, these are necessary and
sufficient conditions to impose. Let's make this explicit in a table:

\begin{tabular}{|l|l|l|l|l|}
  \hline
  $g(y_0)$ sign & $f(x_0)$ (local max/min) at point with $f(x_0) = f'(x_0) = 0$ & $F(x_0,y_0)$ (local max/min) \\\hline
  positive & local max & local max\\\hline
  negative & local max & local min \\\hline
  positive & local min & local min \\\hline
  negative & local min & local max \\\hline
\end{tabular}


\subsection{Absolute extreme values}

Here are some results on the {\em non-existence} of absolute extreme
values:

\begin{itemize}
\item If $f$ takes a positive value anywhere on its domain, and $g$
  takes arbitrarily large positive values, then $F$ takes arbitrarily
  large positive values, and hence has no absolute maximum. 
\item If $f$ takes a negative value anywhere on its domain, and $g$
  takes arbitrarily large positive values, then $F$ takes arbitrarily
  large magnitude negative values, and hence has no absolute
  minimum. 
\item If $f$ takes a positive value anywhere on its domain, and $g$
  takes arbitrarily large magntiude negative values, then $F$ takes
  arbitrarily large magnitude negative values, and hence has no
  absolute minimum.
\item If $f$ takes a negative value anywhere on its domain, and $g$
  takes arbitrarily large magnitude negative values, then $F$ takes
  arbitrarily large positive values, and hence has no absolute
  maximum.
\end{itemize}

To each of the above, an analogous statement holds if we interchange
the roles of $f$ and $g$.

On the other hand, the following is true: if both $f$ and $g$ are
everywhere positive, and the domain is a rectangular region, then the
absolute minimum for $F$ occurs at the point whose $x$-coordinate
gives the absolute minimum for $f$ and whose $y$-coordinate gives the
absolute minimum for $g$.

\subsection{Examples (Type 1 critical points only)}

Consider the function:

$$F(x,y) := (x^2 - x + 2)(3 + \cos y)$$

$F$ is multiplicatively separable and can be written as $f(x)g(y)$
where $f(x) = x^2 - x + 2$ and $g(y) = 3 + \cos y$.

$f$ has a unique critical point with a local and absolute {\em
minimum} at $x = 1/2$, and the value of the minimum is $7/4$.

As for $g$, it attains its local and absolute maxima at multiples of
$2\pi$ (with value $4$) and its local and absolute minima at odd
multiples of $\pi$ (with value $2$).

Note that there are no critical points of types (2), (3), and
(4). This is because $f$ is never $0$ and further, $g$ and $g'$ are
never simultaneously $0$.

We can see from this that:

\begin{itemize}
\item $F$ has local and absolute minimum attained at points of the
  form $(1/2,(2n + 1)\pi)$ with value $7/2$. This is because $f$ has a
  local and absolute {\em positive} minimum at the point $1/2$ and $g$
  has a local and absolute {\em positive} minimum at the point $(2n +
  1)\pi$.
\item $F$ has saddle points at points of the form $(1/2,2n\pi)$ with
  value $7$. This is because $f$ has a local minimum at $1/2$ and $g$
  has a local maximum at $2n\pi$.
\item $F$ has no absolute maximum. To see this, note that $f$ is
  unbounded from above and $g$ takes values in $[2,4]$.
\end{itemize}

Consider a very similar example:

$$F(x,y) := (x^2 - x + 2)\cos y$$

This is similar to the previous example except that the $3 + \cos y$
is replaced by $\cos y$. We take $f(x) = x^2 - x + 2$ and $g(y) = \cos
y$. $f$ has a unique local and absolute minimum at $x_0 = 1/2$ with
value $7/4$. $g$ has local and absolute maxima at even multiples of
$\pi$ with value $1$, and local and absolute minima at odd multiples
of $\pi$ with value $-1$.

From this, we conclude that:

\begin{itemize}
\item $F$ has {\em no} local maxima or minima. To see this, note that
  both the $(1/2,2n\pi)$ and the $(1/2,(2n + 1)\pi)$ cases give saddle
  points. For $(1/2,2n\pi)$, we get minimum and maximum, all positive,
  which gives saddle points. For $(1/2,(2n+1)\pi)$, we get minimum and
  minimum, positive and negative respectively, which again gives
  saddle points.
\item There are no absolute maxima and minima either. To see this,
  note that $f$ is unbounded from above, and $g$ takes values in
  $[-1,1]$, so $F$ can take arbitrarily large positive and negative
  values.
\end{itemize}

\subsection{Example (Type 1 and Type 4 critical points)}

Consider the function:

$$F(x,y) := (x-2)(x-4)(y+1)(y - 5)$$

This is the product of the function $f(x) := (x-2)(x-4)$ and $g(y) :=
(y+1)(y-5)$.

$f$ has a unique critical point at $x_0 = 3$ and $g$ has a unique
critical point at $y_0 = 2$. So $F$ has a unique critical point of
Type 1 (with $f'(x_0) = g'(y_0) = 0$), namely the point $(3,2)$ in the
domain. Since both $f$ and $g$ have {\em negative} local {\em minima}
at the point, we conclude that $F$ has a {\em positive} local {\em
maximum} at the point.

However, there are also other kinds of critical points, specifically
Type 4 critical points where $f(x_0) = g(y_0) = 0$. There are in fact
four such critical points: $(2,-1)$, $(2,5)$, $(4,-1)$, and $(4,5)$.

As already mentioned earlier, the critical points that are only Type 4
critical points {\em cannot} must give saddle points, so we obtain
saddle points at all these four points. The upshot:

\begin{itemize}
\item There is a unique positive local maximum at $(3,2)$.
\item There are four saddle points: $(2,-1)$, $(2,5)$, $(4,-1)$, and
  $(4,5)$.
\item There is {\em no} absolute maximum or minimum. To see this, note
  that $f$ takes both a positive and a negative value, and $g$ takes
  arbitrarily large positive values, so we can arrange the product to
  be a positive or a negative number of arbitrarily large magnitude.
\end{itemize}

\subsection{We don't need no second derivative test}

Consider a function such as:

$$F(x,y) := ((x-1)^4 - 1)((y+3)^6 - 2)$$

If we directly tried to use the second derivative test on this at the
unique critical point $(1,-3)$, we'd get $0$, i.e., the inconclusive
case. However, thinking of the function as multiplicatively separable
allows us to do a little better.

We consider $F$ as a product of the function $f(x) := (x - 1)^4 - 1$
and $g(y) := (y+3)^6 - 2$. Using the first derivative test (or the
higher derivative tests) we can conclude that $f$ has its unique local
and absolute minimum with value $-1$ at $x_0 = 1$. $g$ has its unique
local and absolute minimum with value $-2$ at $y_0 = -3$. Consulting
the table on various combinations, we note that $F$ has a local {\em
maximum} at $(1,-3)$ with value $2$. However, $F$ has no
absolute maxima or minima because $f$ takes both positive and negative
values and $g$ takes arbitrarily large positive values.

{\em Incidentally, this also gives an example of a function with a
unique local maximum and no local minimum but where the local maximum
is not an absolute maximum}.

The saddle points arising as Type 4 critical points in this case are
$(0,\pm 2^{1/6} - 3)$ and $(2, \pm 2^{1/6} - 3)$.

\section{Polynomials}

\subsection{Linear polynomials}

A linear polynomial is a polynomial of the form:

$$f(x,y) := ax + by + c$$

where at least one of the values $a$ and $b$ is nonzero. For a linear
polynomial, there are no critical points and hence no local extreme
values. We may still have boundary extreme values, discussed later
when we talk of maximization on closed bounded subsets.

\subsection{Homogeneous quadratic case}

We begin by looking at a homogeneous quadratic polynomial:

$$f(x,y) := ax^2 + bxy + cy^2$$

where at least one of the coefficients $a$, $b$, and $c$ is nonzero.

First, we calculate the partials:

\begin{eqnarray*}
  f_x(x,y) & = & 2ax + by \\
  f_y(x,y) & = & bx + 2cy \\
  f_{xy}(x,y)&=& b\\
  f_{xx}(x,y)&=& 2a \\
  f_{yy}(x,y)&=& 2c
\end{eqnarray*}

The Hessian determinant in this case becomes the constant $4ac - b^2$,
which is the {\em negative} of the {\em discriminant} of the quadratic
polynomial.

If $4ac - b^2 \ne 0$, then the equations $2ax + by = 0$ and $bx + 2cy
= 0$ are independent linear equations, so their solution set is the
unique point $(0,0)$, so this is the only critical point. We note that:

\begin{itemize}
\item If $4ac - b^2 > 0$, i.e., the discriminant is negative, then
  this is a unique local extreme value for the function with value $0$
  attained at the origin $(0,0)$. Whether it is a maximum or a minimum
  depends on whether $a$ and $c$ are positive or negative. If $a > 0$,
  then the local extreme value of $0$ at the origin is the unique
  minimum for the function. This also turns out to be the absolute
  maximum/minimum.
\item If $4ac - b^2 < 0$, i.e., the discriminant is positive, then the
  function has no local extreme values. In fact, in this case, there
  are two lines through the origin on which the function takes the
  value $0$. The two lines divide the plane into four regions. In two
  of these regions, the function can take negative values of
  arbitrarily large magnitude. In the other two regions, the function
  can take positive values of arbitrarily large magnitude.
\end{itemize}

Finally, if $b^2 = 4ac$, then the function attains its extreme value of $0$
along a single line through the origin, obtained as the line $2ax + by
= 0$. This is a minimum or maximum again depending on the sign of $a$.

These cases are also summarized in the table:

\begin{tabular}{|l|l|l|l|l|}
  \hline
  $4ac - b^2$ sign & $b^2 - 4ac$ sign & Extreme value & Points where it is attained & Nature (max or min?)\\
  \hline
  $> 0$ & $< 0$ & $0$ & $(0,0)$ & min if $a > 0$, max if $a < 0$ \\
  \hline
  $< 0$ & $> 0$ & -- & -- & --\\
  \hline
  $= 0$ & $= 0$ & $0$ & all points on the line $2ax + by = 0$ & min if $a > 0$, max if $a < 0$\\
  \hline
\end{tabular}

What's happening here will become clearer with our general analysis of
homogeneous functions.

\subsection{Homogeneous polynomials and functions}

Suppose $F(x,y)$ is a homogeneous function of degree $d$. Then, we
know by definition that:

$$F(ax,ay) = a^dF(x,y)$$

We note the following:

\begin{itemize}
\item For any line through the origin, either $F$ is identically zero
  along the line, or the gradient of $F$ at any point on the line
  other than the origin is a nonzero vector.
\item In particular, this means that the only possibility for a local
  extreme value is $0$, and this must be attained either only at the
  origin or at a union of lines through the origin.
\item To determine what lines through the origin work, rewrite $F(x,y)
  = x^dg(m)$ where $m = y/x$. Now, find the values of $m$ for which
  this function of one variable attains a local extreme value
  of $0$. The lines $y = mx$ for these values of $m$ are the relevant
  ones. The $y$-axis needs to be checked separately.

  To check whether the origin works, check whether $g$ has a uniform
  sign (excluding points where it is zero).
\end{itemize}

In the light of this, the discussion of quadratics makes extra
sense. Rewriting the quadratic $ax^2 + bxy + cy^2$, we get $x^2(a + bm
+ cm^2)$. The function is $g(m) = cm^2 + bm + a$. If the discriminant
$b^2 - 4ac$ is less than $0$, that means the quadratic has uniform
sign, so the origin is a local extreme value, but there are no lines
on which the function is zero.

If the discriminant is positive, then the quadratic $g$ does not have
uniform sign, and $0$ is not a local extreme value, so there are no
local extreme values for the function $F$ of two variables.

If the discriminant is zero, then the quadratic $g$ has a uniform sign
except at an isolated point where it attains an extreme value of $0$,
so that corresponds to a line on which $F$ attains its local extreme
value of $0$.

\subsection{Non-homogeneous quadratics: general discussion and specific example}

We now turn to the situation of a non-homogeneous quadratic
polynomial. This is of the general form:

$$f(x,y) := ax^2 + bxy + cy^2 + px + qy + l$$

where at least one of the $a$, $b$, and $c$ is nonzero.

Note that the value of $l$ does not affect the points of local
extrema, though it affects their values. The partial derivatives are:

\begin{eqnarray*}
  f_x(x,y) & = & 2ax + by + p \\
  f_y(x,y) & = & bx + 2cy + q \\
  f_{xx}(x,y)&=& 2a\\
  f_{xy}(x,y)&=& b \\
  f_{yy}(x,y)&=&2c\\
\end{eqnarray*}

The Hessian determinant at any point is thus $4ac - b^2$. In other
words, it is a {\em constant independent of the point} and is the
negative of the discriminant of the homogeneous quadratic part.

Thus, we note the following cases:

\begin{itemize}
\item If $4ac - b^2 > 0$, or $b^2 - 4ac < 0$, then we get a unique
  critical point solving the simultaneous system of linear equations
  $f_x(x,y) = 0$ and $f_y(x,y) = 0$ and this gives a local extreme
  value. It is a local minimum if $a > 0$ and a local maximum if $a < 0$.
\item If $4ac - b^2 < 0$ or $b^2 - 4ac > 0$, then we get a unique
  critical point solving the simultaneous system of linear equations
  $f_x(x,y) = 0$ and $f_y(x,y) = 0$, but this gives a saddle point and
  not a local extremum.
\item If $4ac - b^2 = 0$, then, depending on the values $p$ and $q$,
  we either have {\em no} critical points or {\em a line's worth} of
  critical points. In the latter case, the line's worth of critical
  points gives local extrema.
\end{itemize}

\section{Boundary issues}

\subsection{The concept of a quasiconvex function}

Suppose $f$ is a function defined on a domain $D$ in $\R^n$. Suppose
the domain $D$ is a convex subset of $\R^n$, i.e., given any two
points in $D$, the line segment joining them lies completely in $D$.

We say that $f$ is a {\em quasiconvex function} if given any two
points $P$ and $Q$ in $D$, the maximum value of $f$ on the line
segment joining $P$ and $Q$ is attained at one (or possibly both) of
the endpoints $P$ and $Q$. In other words, the value in the interior
of a line segment is less than or equal to the value at one (or
possibly both) of the endpoints.

We say that $f$ is {\em strictly quasiconvex} if the maximum can occur
{\em only} at endpoints, i.e., it is not possible for the maximum
value to also be attained at an interior point.

We note that:

\begin{itemize}
\item All linear functions are quasiconvex but not strictly quasiconvex.
\item All convex functions are quasiconvex and all strictly convex
  functions are strictly quasiconvex. We're not going to go into the
  meaning of convex and strictly convex here (the definition is a
  non-calculus definition and is fairly simple but will take us a
  little off track). For functions of one variable, strictly convex
  simply means {\em concave up} if the function is continuously
  differentiable. For functions of two variables, if the Hessian
  determinant is strictly positive everywhere (except possibly at
  isolated points where it is zero) and the second pure
  partials are positive everywhere, the function is strictly convex.
\item Quadratic functions of two variables with negative discriminant
  of the homogeneous part (i.e., positive Hessian determinant) 
  and with positive coefficients on the square terms are
  strictly convex and hence strictly quasiconvex.
\end{itemize}

There are two reasons quasiconvex functions are significant:

\begin{itemize}
\item The {\em maximum} of a continuous quasiconvex function on a
  closed bounded convex domain {\em must} be attained somewhere on its
  boundary. In fact, we can go further and note that the maximum must
  be attained at an {\em extreme point} of the domain: a point not in
  the interior of any line segment within the domain.

  For a strictly quasiconvex function, the maximum {\em cannot} be
  attained at any point other than an extreme point. For a function
  that's quasiconvex but not strictly so, the maximum may also be
  attained at other points.
\item The {\em minimum} of a continuous quasiconvex function on a
  closed bounded convex domain is attained either at a unique point or
  on a convex subset of the convex domain, i.e., if the minimum occurs
  at two distinct points in the domain, it also occurs at all points
  in the line segment joining them.

  For a strictly quasiconvex function, the minimum {\em must} be
  attained at a {\em unique} point.
\item For a continuous function that is the negative of a quasiconvex
  function, the same observations as above hold but with the roles of
  maximum and minimum interchanged.
\end{itemize}

We now consider some examples of maximization for functions that are
strictly quasiconvex.

\subsection{Linear examples}

Note that {\em for linear functions}, both the function and its
negative are quasiconvex, so both the maximum and the minimum of a
linear function on a closed bounded convex domain must occur at
extreme points.

Consider a function $f(x,y) = 2x - 3y$ on the square region $[-1,1]
\times [-1,1]$.

Since $f$ is a linear function, it is quasiconvex (though not strictly
so). This means that the maximum value, if it occurs, must occur at
one of the extreme points. The extreme points of a square are its four
vertices, i.e., the vertices $(-1,-1)$, $(-1,1)$, $(1,1)$, and
$(1,-1)$. We simply need to evaluate $f$ at all these points and see
which is the largest. We have $f(-1,-1) = 1$, $f(-1,1) = -5$, $f(1,1)
= -1$, $f(1,-1) = 5$. The largest occurs at $(1,-1)$ with value $5$,
so this is the maximum.

In this case, the minimum occurs at $(-1,1)$ with value $-5$.

Now consider another example:

$$f(x,y) := 3x + 4y$$

on the circular disk $x^2 + y^2 \le 1$.

Here, the maximum and minimum both occur on the boundary circle $x^2 +
y^2 = 1$. However, all points on the boundary circle are extreme
points, so the minimum may be attained at any of them -- we cannot
rule any point offhand. The problem thus reduces to maximizing a
function on the circle. There are many ways of doing this, including
Lagrange multipliers (which we'll see shortly) but one approach is to
put $x = \cos \theta$, $y = \sin \theta$, and thus convert the problem
to a maximization/minimization in one variable of the function
$g(\theta) = 3\cos \theta + 4 \sin \theta$.

\subsection{Strictly convex quadratic example}

$$f(x,y) := 2x^2 - 2xy + y^2 - x + 3$$

Suppose we want to calculate the maximum and minimum values of $f$ on
the square region $[0,1] \times [0,1]$.

We compute the first and second partials:

\begin{eqnarray*}
  f_x(x,y) & = & 4x - 2y - 1 \\
  f_y(x,y) & = & 2y - 2x \\
  f_{xx}(x,y)& = & 4\\
  f_{yy}(x,y)& = & 2\\
  f_{xy}(x,y)& = & -2
\end{eqnarray*}

We note that the Hessian determinant is $4$ {\em everywhere} and
$f_{xx} = 4$ is positive everywhere. So this is an example of a
strictly convex, and hence strictly quasiconvex, quadratic
function. As also seen from the discussion of quadratics, we should
get a unique critical point that gives a local extreme value. We first
find the critical point by solving:

\begin{eqnarray*}
  4x - 2y - 1 & = & 0\\
  2y - 2x & = & 0\\
\end{eqnarray*}

Solving, we get $x = 1/2$, $y = 1/2$. This gives the point
$(1/2,1/2)$, which lies within the domain. Evaluating the function at
this point gives $f(1/2,1/2) = 11/4$. This is the unique local minimum
of the function on the whole plane and hence on the domain. In this
case, it also turns out to be the absolute minimum. {\em Note: In
general, there are examples of continuous functions having a unique
local minimum that is not an absolute minimum. But those examples
don't include strictly quasiconvex functions.}

Since the function is strictly convex and hence strictly quasiconvex,
the maximum must occur at one of the four corner points. It remains to
evaluate $f$ at the four boundary points $(0,0)$, $(1,0)$, $(1,1)$,
and $(0,1)$. We get $f(0,0) = 3$, $f(1,0) = 4$, $f(1,1) = 3$, $f(0,1)
= 4$. Of these, we see that the maximum occurs at the points $(1,0)$
and $(0,1)$, with a value of $4$.

\section{Combining max-min and Lagrange}

For quasiconvex functions, we already noted that the maximum must
occur somewhere on the boundary of the domain. For strictly
quasiconvex functions, it can occur {\em only} at boundary points.

Even for functions that aren't quasiconvex, maximization over a
certain domain requires us to find not just local maxima/minima in the
interior but also boundary maxima/minima and then compare them
all. Lagrange multipliers can offer a method to do so.

Let's return to our earlier example:

$$f(x,y) := 3x + 4y$$

on the disk $x^2 + y^2 \le 1$.

Since this is a linear function, both the function and its negative
are quasiconvex, and since the domain of consideration is a convex
domain, the maximum and minimum must both be attained at the boundary
circle $x^2 + y^2 = 1$.

We can now use the method of Lagrange multipliers to carry out the
maximization and minimization relative to the boundary. Here, $g(x,y)
= x^2 + y^2$, so $\nabla g(x,y) = \langle 2x,2y \rangle$. Also,
$\nabla f(x,y) = \langle 3,4 \rangle$. We get:

$$\langle 3,4 \rangle = \lambda \langle 2x, 2y \rangle, \qquad x^2 + y^2 = 1$$

Simplifying, we get:

\begin{eqnarray*}
  3 & = & 2\lambda x\\
  4 & = & 2\lambda y\\
  x^2 + y^2 & = & 1 \\
\end{eqnarray*}

Plug in $x = 3/(2\lambda)$ and $y = 2/\lambda$ into the third equation
and we get:

$$\frac{9}{4\lambda^2} + \frac{4}{\lambda^2} = 1$$

Simplifying, we get:

$$\lambda = \pm 5/2$$

Plugging back, we obtain that the two candidate critical points are
$\langle 3/5,4/5 \rangle$ and $\langle -3/5,-4/5 \rangle$. In this
case, it is easy to see from inspection that $\langle 3/5,4/5 \rangle$
is a point of local maximum with value $5$ and $\langle -3/5,-4/5
\rangle$ is a point of local minimum with value $-5$.

\end{document}
