\documentclass[10pt]{amsart}

%Packages in use
\usepackage{fullpage, hyperref, vipul}

%Title details
\title{Lie pring theory}
\author{Vipul Naik}

%List of new commands
\newcommand{\Skew}{\operatorname{Skew}}

\begin{document}
\maketitle

\section*{Motivational note}

The idea is to generalize for arbitrary primes $p$ what Lie cring
theory does for the prime $p = 2$.

\section{Lie pring: definition}

We define a Lie pring for the prime $p$ as follows: It is a set $L$
equipped with the following:

\begin{itemize}
\item An abelian group structure denoted additively, such that the
  abelian group is uniquely $q$-divisible for all primes $q < p$.
\item A Lie bracket that makes $L$ a Lie ring with addition the above
  abelian group structure.
\item A function $\theta_p: L \times L \to L$ satisfying the following
  three conditions: 

  \begin{itemize}
  \item The {\em boundary condition}, which states that
    $\partial \theta_p$ (in the sense of the boundary of a $2$-cochain)
    is formally the same as $\partial$ of the $p^{th}$ term of the
    Baker-Campbell-Hausdorff formula (note that that formal expression
    has on $p$s in its denominator).
  \item The {\em multiple condition}, which states that$p\theta_p$ is
    formally the same as $p$ times the $p^{th}$ term of the
    Baker-Campbell-Hausdorff formula.
  \item The {\em near-alternating condition}, which states that
    $\theta_p(x,y) = 0$ whenever $x$ and $y$ generate a cyclic
    subgroup of $L$.
  \end{itemize}
\end{itemize}

To avail of the advantages of infix notation for binary operations, we
choose the symbol $\wr$ for $\theta_p$, so $x \wr y$ is shorthand for
$\theta_p(x,y)$.

\subsection{The definition of nilpotency class}

This definition is trickier than we're used to, so some preliminaries.

We define a {\em monomial} as a formal expression in finitely many
variables that involves {\em only} iterating the Lie bracket and the
pring operation $\wr$, i.e., it does not involve any addition or
scalar multiplication, and where all the starting points are among the
variables. To evaluate this monomial at an element of the Lie pring,
we need to specify actual values of all variables.

We define the {\em min-degree} of a monomial inductively as follows:

\begin{itemize}
\item Each of the formal variables has degree $1$.
\item The min-degree of $[A,B]$ is the sum of the min-degrees of $A$
  and of $B$.
\item The min-degree of $A \wr B$ is defined as follows. Let $d_A$ and
  $d_B$ be the degrees of $A$ and $B$. Consider each of the monomials
  in the $p^{th}$ term of the BCH formula. In each of them, $A$ occurs
  $k_i$ number of times and $B$ occurs $p - k_i$ number of times. Take
  $\min_i \{ k_id_A + (p - k_i)d_B \}$. In other words, it is the
  smallest of the degrees of all the monomials.
\end{itemize}

We can now define the min-degree of a general expression as the
minimum of the min-degrees of all monomials used to constitute
it. Note that some additional algebraic simplifications may be used to
push up the min-degree, so the min-degree should really be considered
a lower bound on the min-degree. All our results are of the flavor: if
the min-degree is $\ge k$, then ...

We define a Lie pring $L$ to have class $\le c$ if any expression of
min-degree $\ge c + 1$ takes the value $0$ for all possible choices of
input from $L$.


\section{Cring theory}

\subsection{The case $p = 2$: Lie cring}

A Lie cring is a Lie pring for the prime $p = 2$. Specifically, the
cring multiplication is the operation $\theta_2$, and the Lie bracket
coincides with twice of that. The boundary condition in this case
becomes $\partial \theta_2 = 0$, which is why we get a cocycle. The
multiple condition is $2\theta_2 = [ \ , \ ]$.

Note that for Lie crings, as usually defined, the Lie bracket is not
included as part of the structure, because the Lie bracket is uniquely
recoverable from the cring multiplication or $\theta_2$, i.e., it is
$2\theta_2$. For larger values of $p$, this reverse determination is
not possible.

\section{Tring theory}
\subsection{The case $p = 3$: Lie tring}

A Lie tring is a Lie pring for the prime $p = 3$. We write down the
boundary condition and multiple condition explicitly:

\begin{itemize}
\item The boundary condition states that $(\partial\theta_3)(x,y,z) =
  \frac{-1}{4}[y,[z,x]]$.
\item The multiple condition states that $3\theta_3(x,y) =
  \frac{1}{4}[x,[x,y]] - \frac{1}{4}[y,[x,y]]$.
\end{itemize}

For simplicity, we denote $\theta_3(x,y)$ by $x \wr y$. 

We now try to define a notion of {\em nilpotency class} for
trings. What does it mean for a tring to have a particular class? In
order to make this definition, we need a notion of {\em min-degree of
monomial}, where the monomial involves a composition of both $\wr$ and
$[,]$. We define this degree inductively: 

\begin{itemize}
\item The min-degree of a primitive term/variable is $1$.
\item The min-degree of $[A,B]$ where $A$ and $B$ are both expressions is
  the sum of the degrees of $A$ and $B$.
\item The min-degree of $A \wr B$ where $A$ and $B$ are both expressions
  is $d_A + d_B + \min \{ d_A,d_B\}$.
\end{itemize}

We say that a Lie tring in $L$ has class $k$ if for any monomial of
min-degree $k + 1$, setting values of the term variables to any bunch
of elements in $L$ is $0$.

In particular, a Lie tring has class three iff it satisfies all these
conditions:

\begin{itemize}
\item $[[[x,y],z],w] = 0$ for all $x,y,z,w \in L$ (Note that this
  implies corresponding conditions on all other parenthesizations
  involving purely Lie brackets, via the Jacobi condition and
  alternation).
\item $[(x \wr y),z] = 0$ for all $x,y,z \in L$. This automatically
  implies $[x,y \wr z] = 0$ for all $x,y,z \in L$, again by
  alternation.
\item $(x \wr y) \wr z = x \wr (y \wr z) = 0$ for all $x,y,z \in L$.
\end{itemize}

We can now state the first theorem.

\begin{theorem}
  Suppose $L$ is a Lie tring of class three (with the notation used
  above). We can provide a group structure on $L$ as follows:

  $$x \cdot y = x + y + \frac{1}{2}[x,y] + (x \wr y)$$

  Further, the set identity mapping from $L$ to this group is a
  $1$-isomorphism of groups, i.e., any cyclic subgroup of $L$ goes to
  the corresponding cyclic subgroup in the group.
\end{theorem}

\begin{proof}
  {\em Cyclic subgroup, identity element, inverses}: If $x$ and $y$
  generate a cyclic subgroup, then $[x,y] = 0$ and $x \wr y = 0$ by
  assumption. Thus, $x \cdot y = x + y$. In particular, $x \cdot 0 = x
  + 0 = x$, and $0 \cdot x = 0 + x = x$. Also, $x \cdot (-x) = x +
  (-x) = 0$, so $-x = x^{-1}$.

  {\em Associativity}: We have:

  $$(x \cdot y) \cdot z = \{ x + y + \frac{1}{2}[x,y] + (x \wr y) \}  + z + \frac{1}{2}\left[x + y + \frac{1}{2}[x,y] + (x \wr y),z\right] + (x + y + \frac{1}{2}[x,y] + (x \wr y)) \wr z$$

  and:

  $$x \cdot (y \cdot z) = x + \{ y + z + \frac{1}{2}[y,z] + (y \wr z) \} + \frac{1}{2}\left[x,y + z + \frac{1}{2}[y,z] + (y \wr z)\right] + x \wr (y + z + \frac{1}{2}[y,z] + (y \wr z))$$

  Subtracting these two:

  $$((x \cdot y) \cdot z) - (x \cdot (y \cdot z)) = \frac{1}{2}\{[x,y] + [x+y,z] - [x,y+z] - [y,z]\} + \frac{1}{4}\{[[x,y],z] - [x,[y,z]]\} + \{ (x \wr y) + ((x + y + \frac{1}{2}[x,y] + (x \wr y)) \wr z) - (x \wr (y + z + \frac{1}{2}[y,z] + (y \wr z))) - (y \wr z) \} + \frac{1}{2}\{[x \wr y,z] - [x,y \wr z] \}  $$

  We now argue that:

  $$x \wr (y + z + \frac{1}{2}[y,z] + (y \wr z)) = x \wr (y + z)$$

  The argument here basically uses the boundary condition on the
  triple $x$, $y + z$, and $\frac{1}{2}[y,z] + (y \wr z)$. The
  split-off method works!

  The degree two part:

  $$\frac{1}{2} \{ [x,y] + [x+y,z] - [x,y+z] - [y,z] \}$$

  becomes equal to zero using the additivity of the Lie bracket.

  The degree three part becomes zero using the boundary condition. The
  higher degree parts are automatically zero.
\end{proof}

\end{document}