\section{Isoclinism and homoclinism for groups}\label{sec:isoclinism-and-homoclinism}

The goal of this section is to establish the basic theory of {\em
  isoclinism} and {\em homoclinism} for groups. Informally, a
homoclinism of groups is a homomorphism between the {\em commutator
  structures} of the groups. Informally, two groups are isoclinic if
their commutator maps are equivalent. Isoclinism defines an
equivalence relation on the collection of groups. Under this
equivalence relation, all abelian groups are equivalent to the trivial
group.

The original results that we present later (Section \ref{sec:bcuti}
and \ref{sec:lcuti}) describe bijective correspondences between
certain equivalence classes of groups and certain equivalence classes
of Lie rings. The equivalence classes of groups are based on the
equivalence relation of isoclinism.

Readers already familiar with the definitions of isoclinism and
homoclinism may skip this section and return to it later if needed.
Readers who want the bare minimum necessary for later sections can
read Sections
\ref{sec:isoclinism-definition}-\ref{sec:homoclinism-category}, and
the statements of the theorems in Section
\ref{sec:homoclinism-misc-results}. The proofs of the theorems in
Section \ref{sec:homoclinism-misc-results} can be skipped.


\subsection{Isoclinism of groups: definition}\label{sec:isoclinism-definition}

The concept of isoclinism as introduced here was first defined in 1937
by Philip Hall in \cite{Hall37}. It was used by Philip Hall as an aid
to the classification of finite groups of small prime power
order. Hall's work was later extended by Marshall Hall and Senior, who
published detailed information on the groups of order $2^n, n \le 6$
in \cite{HallSenior}. The basic definition and most of the elementary
facts stated here about isoclinism can be found on Page 93 of Suzuki's
group theory text \cite{SuzukiII}.

For any group $G$, denote by $\operatorname{Inn}(G)$ the inner
automorphism group of $G$, denote by $G'$ the derived subgroup of $G$,
and denote by $Z(G)$ the center of $G$ (this and related notation used
in this document are described in Section \ref{sec:group-notation}). Note that $\operatorname{Inn}(G) \cong G/Z(G)$.

For any group $G$, the commutator map in $G$ descends to a map of sets:

$$\omega_G: \operatorname{Inn}(G) \times \operatorname{Inn}(G) \to G'$$

This map is well-defined because the commutator of two elements
depends only on their cosets modulo the center. Note that this map is
only a {\em set} map at this stage, not a homomorphism. Later, in
Section \ref{sec:exteriorsquare}, we will introduce the concept of the
exterior square of a group, and we will be able to interpret
$\omega_G$ as a homomorphism in that context.

Suppose now that $G_1$ and $G_2$ are groups. The commutator maps in
the groups define the following respective maps:

$$\omega_{G_1}: \operatorname{Inn}(G_1) \times \operatorname{Inn}(G_1) \to G_1'$$

$$\omega_{G_2}: \operatorname{Inn}(G_2) \times \operatorname{Inn}(G_2) \to G_2'$$

An {\em isoclinism} from $G_1$ to $G_2$ is a pair of isomorphisms
$(\zeta,\varphi)$ where $\zeta$ is an isomorphism from
$\operatorname{Inn}(G_1)$ to $\operatorname{Inn}(G_2)$ and $\varphi$ is an
isomorphism from $G_1'$ to $G_2'$, satisfying the condition that:

\begin{equation}\label{eq:homoclinism-point-free}
  \varphi \circ \omega_{G_1} = \omega_{G_2} \circ (\zeta \times \zeta)
\end{equation}

More explicitly, for any $x,y \in \operatorname{Inn}(G_1)$, we require that:

\begin{equation}\label{eq:homoclinism-pointed}
  \varphi(\omega_{G_1}(x,y)) = \omega_{G_2}(\zeta(x),\zeta(y))
\end{equation}

In other words, taking the commutator and then applying the
isomorphism of derived subgroups is equivalent to applying the
isomorphism between inner automorphism groups and then taking the
commutator.

Pictorially, this can be represented as saying that the following
diagram commutes:

$$\begin{array}{ccc}
  \operatorname{Inn}(G_1) \times \operatorname{Inn}(G_1) & \stackrel{\zeta \times \zeta}{\to} & \operatorname{Inn}(G_2) \times \operatorname{Inn}(G_2) \\
  \downarrow^{\omega_{G_1}}  & & \downarrow^{\omega_{G_2}}\\
  G_1' & \stackrel{\varphi}{\to} & G_2'\\
\end{array}$$

Note that both the inner automorphism group and the derived subgroup
are quantitative measurements of the ``non-abelianness'' of the
group. The notion of isoclinism can thus properly be thought of as
saying ``equivalent modulo the subvariety of abelian groups.'' In
particular, a group is abelian if and only if it is isoclinic to the
trivial group.

There is a precise way of formulating this using the more general
notion of isologism, which we describe in the Appendix, Section
\ref{appsec:homologism-theory}.

\subsection{Homoclinism of groups}\label{sec:homoclinism-definition}

The notion of {\em homoclinism of groups} relates to isoclinism of
groups in the same way as homomorphism of groups relates to
isomorphism of groups. We have not been able to confirm the first use
of the term, but a somewhat more general definition called
$n$-homoclinism appears in \cite{Hekster}. We have chosen this
presentation, despite its being non-standard, because it is a
convenient framework for understanding later results. Although the
presentation is non-standard, none of the results in this or the next
few sections are substantively different from results available in the
literature.
 
Suppose $G_1$ and $G_2$ are groups. A {\em homoclinism of groups} from
$G_1$ to $G_2$ is a pair of homomorphisms $(\zeta,\varphi)$ where
$\zeta$ is a homomorphism from $\operatorname{Inn}(G_1)$ to
$\operatorname{Inn}(G_2)$ and $\varphi$ is a homomorphism from $G_1'$
to $G_2'$, satisfying Equation \ref{eq:homoclinism-point-free} (that
can alternatively be stated as \ref{eq:homoclinism-pointed}).

Pictorially, this can be represented as saying that the following
diagram commutes:

$$\begin{array}{ccc}
  \operatorname{Inn}(G_1) \times \operatorname{Inn}(G_1) & \stackrel{\zeta \times \zeta}{\to} & \operatorname{Inn}(G_2) \times \operatorname{Inn}(G_2) \\
  \downarrow^{\omega_{G_1}}  & & \downarrow^{\omega_{G_2}}\\
  G_1' & \stackrel{\varphi}{\to} & G_2'\\
\end{array}$$

Note that this is the same as the diagram for isoclinisms. The only difference is that the horizontal maps are no longer required to be bijective.

\subsection{Composition of homoclinisms}\label{sec:homoclinism-composition}

Suppose $G_1$, $G_2$, and $G_3$ are groups. Suppose
$(\zeta_{12},\varphi_{12})$ is a homoclinism from $G_1$ to $G_2$ and
$(\zeta_{23},\varphi_{23})$ is a homoclinism from $G_2$ to $G_3$. We
then define the {\em composite} of these homoclinisms to be the
following homoclinism from $G_1$ to $G_3$:

$$(\zeta_{23},\varphi_{23}) \circ (\zeta_{12},\varphi_{12}) = (\zeta_{23}\circ \zeta_{12}, \varphi_{23} \circ \varphi_{12})$$

To see that this composite is indeed a homoclinism, we need to check
that both the component maps are homomorphisms, and that the
corresponding diagram commutes. The component maps are homomorphisms
because a composite of homomorphisms is a homomorphism. The fact that
the diagram commutes can be seen from the full diagram below. The left
square commutes because $(\zeta_{12},\varphi_{12})$ is a
homoclinism. The right square commutes because
$(\zeta_{23},\varphi_{23})$ is a homoclinism. Thus, the overall
diagram commutes.

$$\begin{array}{ccccc}
  \operatorname{Inn}(G_1) \times \operatorname{Inn}(G_1) & \stackrel{\zeta_{12} \times \zeta_{12}}{\to} & \operatorname{Inn}(G_2) \times \operatorname{Inn}(G_2) & \stackrel{\zeta_{23} \times \zeta_{23}}{\to} & \operatorname{Inn}(G_3) \times \operatorname{Inn}(G_3)\\
  \downarrow^{\omega_{G_1}}  & & \downarrow^{\omega_{G_2}} & & \downarrow^{\omega_{G_3}}\\
  G_1' & \stackrel{\varphi_{12}}{\to} & G_2' & \stackrel{\varphi_{23}}{\to} & G_3'\\
\end{array}$$

\subsection{Category of groups with homoclinisms}\label{sec:homoclinism-category}

We define a category that will be useful to work with.

\begin{definer}[Category of groups with homoclinisms]
  The {\em category of groups with homoclinisms} is defined as the
  following category:

  \begin{itemize}
  \item The {\em objects} of the category are groups.
  \item The {\em morphisms} of the category are homoclinisms.
  \item Composition of morphisms is composition of homoclinisms.
  \item The identity morphism is the identity homoclinism: it is the
    identity map on both the inner automorphism group and the derived
    subgroup.
  \end{itemize}
\end{definer}

In the category of groups with homoclinisms, the isomorphisms (i.e.,
the invertible morphisms) are precisely the isoclinisms.

\subsection{Homomorphisms and homoclinisms}\label{sec:homomorphisms-and-homoclinisms}

Suppose $G_1$ and $G_2$ are groups and $\theta: G_1 \to G_2$ is a
homomorphism of groups. If $\theta$ satisfies the property that
$\theta(Z(G_1)) \le Z(G_2)$, then $\theta$ induces a homoclinism of
groups. Explicitly the homoclinism induced by $\theta$ is defined as
$(\zeta,\varphi)$ where $\zeta$ and $\varphi$ are as defined below.

\begin{itemize}
\item Since $\theta(Z(G_1)) \le Z(G_2)$, $\theta$ descends to a
  homomorphism from $G_1/Z(G_1) \cong \operatorname{Inn}(G_1)$ to
  $G_2/Z(G_2) \cong \operatorname{Inn}(G_2)$. Denote by $\zeta$ the
  induced homomorphism $\operatorname{Inn}(G_1) \to
  \operatorname{Inn}(G_2)$.
\item The restriction of $\theta$ to $G_1'$ maps inside $G_2'$. Denote
  by $\varphi$ the induced map $G_1' \to G_2'$.
\end{itemize}

It is easy to verify that $(\zeta,\varphi)$ defines a homoclinism.

Note that the condition $\theta(Z(G_1)) \le Z(G_2)$ is necessary in
order to be able to construct $\zeta$.

The following are true:

\begin{itemize}
\item Every {\em surjective} homomorphism $\theta:G_1 \to G_2$
  satisfies the condition that $\theta(Z(G_1)) \le Z(G_2)$. Thus,
  every surjective homomorphism induces a homoclinism.
\item The inclusion of a subgroup $H$ in a group $G$ satisfies the
  condition if and only if $Z(H) \le Z(G)$, or equivalently, $Z(H) = H
  \cap Z(G)$. Thus, these are the subgroups whose inclusions induce
  homoclinisms.
\end{itemize}

\subsection{Miscellaneous results on homoclinisms and words}\label{sec:homoclinism-misc-results}

\begin{lemma}
  Suppose $(\zeta,\varphi)$ is a homoclinism of groups $G_1$ and
  $G_2$, where $\zeta:\operatorname{Inn}(G_1) \to
  \operatorname{Inn}(G_2)$ and $\varphi:G_1' \to G_2'$ are the
  component homomorphisms. Denote by $\theta_1:G_1' \to
  \operatorname{Inn}(G_1)$ the composite of the inclusion of $G_1'$ in
  $G_1$ and the projection from $G_1$ to $G_1/Z(G_1) =
  \operatorname{Inn}(G_1)$. Similarly define $\theta_2:G_2' \to
  \operatorname{Inn}(G_2)$. Then, we have:
  
  $$\zeta \circ \theta_1 = \theta_2 \circ \varphi$$

  or equivalently, for any $w \in G_1'$:

  $$\zeta(\theta_1(w)) = \theta_2(\varphi(w))$$
\end{lemma}

\begin{proof}
  To show the equality of the two expressions, it suffices to show
  equality on a generating set for $G_1'$. By definition, the set of
  commutators of elements in $G_1$ is a generating set for
  $G_1'$. Thus, it suffices to show that:

  $$\zeta(\theta_1([u,v])) = \theta_2(\varphi([u,v])) \ \forall \ u,v \in G_1$$

  This is equivalent to showing that:

  $$\zeta(\theta_1(\omega_{G_1}(x,y))) = \theta_2(\varphi(\omega_{G_1}(x,y))) \ \forall \ x,y \in \operatorname{Inn}(G_1)$$

  Let us examine the left and right sides separately. 

  {\em The left side}: The expression $\theta_1(\omega_{G_1}(x,y))$
  first computes the commutator of lifts of $x$ and $y$ in $G_1$, then
  projects to $G_1/Z(G_1)$. This is equivalent to directly computing
  the commutator in $G_1/Z(G_1)$, so $\theta_1(\omega_{G_1}(x,y)) =
  [x,y]$. Thus, the left side becomes $\zeta([x,y])$.

  {\em The right side}: By the definition of homoclinism,
  $\varphi(\omega_{G_1}(x,y)) = \omega_{G_2}(\zeta(x),\zeta(y))$. The
  right side now becomes
  $\theta_2(\omega_{G_2}(\zeta(x),\zeta(y)))$. In other words, we are
  taking the lifts of $\zeta(x)$ and $\zeta(y)$ in $G_2$, then
  computing the commutator, then projecting to $G_2/Z(G_2)$. This is
  equivalent to directly computing the commutator in $G_2/Z(G_2)$, so
  the right side simplifies to $[\zeta(x),\zeta(y)]$. Since $\zeta$ is
  a homomorphism, this is equal to $\zeta([x,y])$, and hence agrees
  with the left side.
\end{proof}

We state two important theorems. Both theorems reference the concept
of a {\em word map}. The concept is defined and some of the properties
of word maps are described in the Appendix, Section
\ref{appsec:word-maps} and the subsequent sections. However, we do not
use any nontrivial facts about word maps, so it is not necessary to
read that section to understand the theorems that follow.

\begin{theorem}\label{thm:iterated-commutator-descends-to-inn}
  Suppose $w(g_1,g_2,\dots,g_n)$ is a word in $n$ letters with the
  property that $w$ evaluates to the identity element in {\em any}
  abelian group. This is equivalent to saying that $w$, viewed as an
  element of the free group on $g_1,g_2,\dots,g_n$, is in the derived
  subgroup. Then, for any group $G$, the word map $w:G^n \to G$ obtained
  by evaluating $w$ descends to a map:

  $$\chi_{w,G}: (\operatorname{Inn}(G))^n  \to G'$$

  Any word $w$ that is an iterated commutator (with any bracketing)
  satisfies this condition.
\end{theorem}

\begin{proof}
  Denote by $\nu:G \to \operatorname{Inn}(G)$ the quotient map.

  $w$ can be written in the form (note that the product is in general
  noncommutative):

  $$w(g_1,g_2,\dots,g_n) = \prod_{i=1}^m[u_i(g_1,g_2,\dots,g_n),v_i(g_1,g_2,\dots,g_n)]$$

  where $u_i,v_i, 1 \le i \le m$ are words. Suppose $y_i \in G$ are
  elements for which $\nu(y_i) = x_i$. Then:

  $$w(y_1,y_2,\dots,y_n) := \prod_{i=1}^m[u_i(y_1,y_2,\dots,y_n),v_i(y_1,y_2,\dots,y_n)]$$

  We have that:

  $$\nu(u_i(y_1,y_2,\dots,y_n)) = u_i(x_1,x_2,\dots,x_n), \qquad \nu(v_i(y_1,y_2,\dots,y_n)) = v_i(x_1,x_2,\dots,x_n)$$

  Thus, we obtain that:

  $$[u_i(y_1,y_2,\dots,y_n),v_i(y_1,y_2,\dots,y_n)] = \omega_G(u_i(x_1,x_2,\dots,x_n),v_i(x_1,x_2,\dots,x_n))$$

  In particular, the expression
  $[u_i(y_1,y_2,\dots,y_n),v_i(y_1,y_2,\dots,y_n)]$ depends only on
  $x_1$, $x_2$, $\dots$, $x_n$ and not on the choice of lifts $y_i$. Thus, the
  product $w(y_1,y_2,\dots,y_n)$ also depends only on the values of
  $x_i$, and we obtain the function:

  $$\chi_{w,G}(x_1,x_2,\dots,x_n) = \prod_{i=1}^m \omega_G(u_i(x_1,x_2,\dots,x_n),v_i(x_1,x_2,\dots,x_n))$$
\end{proof}


\begin{theorem}\label{thm:iterated-commutator-commutes-homoclinisms}
  Suppose $(\zeta,\varphi)$ is a homoclinism of groups $G_1$ and
  $G_2$, where $\zeta:\operatorname{Inn}(G_1) \to
  \operatorname{Inn}(G_2)$ and $\varphi:G_1' \to G_2'$ are the
  component homomorphisms. Then for any word $w(g_1,g_2,\dots,g_n)$
  that is trivial in every abelian group (as described above), we have:

  $$\chi_{w,G_2}(\zeta(x_1),\zeta(x_2),\dots,\zeta(x_n)) = \varphi(\chi_{w,G_1}(x_1,x_2,\dots,x_n))$$

  for all $x_1,x_2,\dots,x_n \in \operatorname{Inn}(G)$.

  Any word $w$ that is an iterated commutator (with any order of
  bracketing) satisfies this condition, and the theorem applies to
  such word maps.
\end{theorem}

\begin{proof}
  Denote by $\nu_1:G_1 \to \operatorname{Inn}(G_1)$ and $\nu_2:G_2 \to
  \operatorname{Inn}(G_2)$ the canonical quotient maps.

  We use the same notation and steps in the proof of the preceding
  theorem, replacing $G$ by $G_1$. We obtain:

  $$w(g_1,g_2,\dots,g_n) =  \prod_{i=1}^m[u_i(g_1,g_2,\dots,g_n),v_i(g_1,g_2,\dots,g_n)]$$

  where $u_i,v_i, 1 \le i \le m$ are words. Suppose $y_i \in G_1$ are
  elements for which $\nu_1(y_i) = x_i$. As demonstrated in the proof of
  the preceding theorem:

  \begin{equation*}
    \chi_{w,G_1}(x_1,x_2,\dots,x_n) = \prod_{i=1}^m \omega_{G_1}(u_i(x_1,x_2,\dots,x_n),v_i(x_1,x_2,\dots,x_n)) \tag{$\dagger$}
  \end{equation*}

  Suppose $z_i \in G_2$ are elements for which $\nu_2(z_i) =
  \zeta(x_i)$. Similar reasoning to the above yields that:

  \begin{small}
  \begin{equation*}
    \chi_{w,G_2}(\zeta(x_1),\zeta(x_2),\dots,\zeta(x_n)) = \prod_{i=1}^m \omega_{G_2}(u_i(\zeta(x_1),\zeta(x_2),\dots,\zeta(x_n)),v_i(\zeta(x_1),\zeta(x_2),\dots,\zeta(x_n))) \tag{$\dagger\dagger$}
  \end{equation*}
  \end{small}
  Apply $\varphi$ to both sides of $(\dagger)$, use the defining
  property of homoclinisms, and compare with $(\dagger\dagger)$ to
  obtain the result.
\end{proof}

\subsection{Isoclinic groups: how similar are they?}

We say that groups $G_1$ and $G_2$ are {\em isoclinic groups} if there
exists an isoclinism from $G_1$ to $G_2$. The relation of being
isoclinic is an equivalence relation. Briefly:

\begin{itemize}
\item The relation of being isoclinic is {\em reflexive} because we
  can choose both the isomorphisms to be the respective identity
  maps. Explicitly, for any group $G$,
  $(\operatorname{id}_{\operatorname{Inn}(G)},\operatorname{id}_{G'})$
  defines an isoclinism from $G$ to itself.
\item The relation of being isoclinic is {\em symmetric} because we
  can take the inverse isomorphisms to both the
  isomorphisms. Explicitly, if $(\zeta,\varphi)$ describes the
  isoclinism from $G_1$ to $G_2$, then $(\zeta^{-1},\varphi^{-1})$
  describes the isoclinism from $G_2$ to $G_1$.
\item The relation of being isoclinic is {\em transitive} because we
  can compose both kinds of isomorphisms separately. Explicitly, if
  $(\zeta_{12},\varphi_{12})$ describes the isoclinism from $G_1$ to
  $G_2$ and $(\zeta_{23},\varphi_{23})$ describes the isomorphism from
  $G_2$ to $G_3$, then $(\zeta_{23} \circ \zeta_{12}, \varphi_{23}
  \circ \varphi_{12})$ describes the isoclinism from $G_1$ to $G_3$.
\end{itemize}

Here is an alternative way of seeing that being isoclinic is an
equivalence relation: isoclinisms are precisely the isomorphisms in
the category of groups with homoclinisms, and being isomorphic in any
category is an equivalence relation.

We first list some very obvious similarities between isoclinic groups.

\begin{itemize}
\item They have isomorphic derived subgroups: This is direct from the
  definition, which includes an isomorphism between the derived
  subgroups.
\item They have isomorphic inner automorphism groups: This is direct
  from the definition, which includes an isomorphism between the inner
  automorphism groups.
\item They have precisely the same non-abelian composition factors (if
  the composition factors do exist): Since the center is abelian, all
  the non-abelian composition factors occur inside the inner
  automorphism group for both, which we know to be isomorphic.
\item If one is nilpotent, so is the other, and they have the same
  nilpotency class (with the exception of class zero getting conflated
  with class one): The nilpotency class is one more than the
  nilpotency class of the inner automorphism group.
\item If one is solvable, so is the other, and they have the same
  derived length (with the exception of length zero getting conflated
  with length one): The derived length is one more than the derived
  length of the derived subgroup.
\end{itemize}

We move to the first straightforward but somewhat non-obvious fact:
isoclinic finite groups have the same {\em proportions} of conjugacy
class sizes. The statement of the theorem is below. The proof can be
found in the Appendix, Section \ref{appsec:isoclinism-extra-proofs}.

\begin{theorem}\label{isoclinic-same-proportions-conjugacy-class-sizes}
  Suppose $G_1$ and $G_2$ are isoclinic finite groups. Suppose $c$ is
  a positive integer. Let $m_1$ be the number of conjugacy classes in
  $G_1$ of size $c$ (so that the {\em total} number of elements in
  such conjugacy classes is $m_1c$). Let $m_2$ be the number of
  conjugacy classes in $G_2$ of size $c$ (so that the {\em total}
  number of elements in such conjugacy classes is $m_2c$). Then, $m_1$
  is nonzero if and only if $m_2$ is nonzero, and if so, $m_1/m_2 =
  |G_1|/|G_2|$.

  In particular, if $G_1$ and $G_2$ additionally have the same order,
  then they have precisely the same multiset of conjugacy class sizes.
\end{theorem}

The next theorem is a similar result for the degrees of irreducible
representations. The proof of this is also in the Appendix, Section
\ref{appsec:isoclinism-extra-proofs}.

\begin{theorem}\label{isoclinic-same-proportions-irrep-degrees}
  Suppose $G_1$ and $G_2$ are isoclinic finite groups. Suppose $d$ is
  a positive integer. Let $m_1$ denote the number of equivalence
  classes of irreducible representations of $G_1$ over $\mathbb{C}$
  that have degree $d$. Let $m_2$ denote the number of equivalence
  classes of irreducible representations of $G_2$ over $\mathbb{C}$
  that have degree $d$. Then, $m_1$ is nonzero if and only if $m_2$ is
  nonzero, and if so, $m_1/m_2 = |G_1|/|G_2|$.

  In particular, if $G_1$ and $G_2$ additionally have the same order,
  then they have precisely the same multiset of degrees of irreducible
  representations.
\end{theorem}

\begin{theorem}
  \begin{enumerate}
  \item Suppose $G_1$ and $G_2$ are isoclinic finite groups. Then, the
    ratio of the number of conjugacy classes in $G_1$ to the number of
    conjugacy classes in $G_2$ is $|G_1|/|G_2|$. In particular, if
    $G_1$ and $G_2$ also have the same order, they have the same
    number of conjugacy classes.
  \item Suppose $G_1$ and $G_2$ are isoclinic finite groups. Then, the
    centers of their respective group algebras over $\mathbb{C}$ are
    both algebras that are direct products of copies of
    $\mathbb{C}$. The ratio of the number of copies used for $G_1$ and
    for $G_2$ is $|G_1|/|G_2|$. In particular, if $G_1$ and $G_2$ also
    have the same order, then the centers of their group algebras are
    isomorphic.
  \end{enumerate}
\end{theorem}

\begin{proof}
  These follow quite directly from either of the preceding
  theorems. More specifically, the proof for part (1) can be deduced
  from either Theorem
  \ref{isoclinic-same-proportions-conjugacy-class-sizes} or Theorem
  \ref{isoclinic-same-proportions-irrep-degrees}. Note that we can use
  the latter because the number of conjugacy classes equals the number
  of irreducible representations.

  For (2), note that the center of the group algebra is a direct
  product of as many copies of $\mathbb{C}$ as the number of conjugacy
  classes. We can use the conjugacy class element sums as a
  basis. Alternatively, we can use the centers of the irreducible
  constituents in a direct sum decomposition into two-sided ideals as
  a basis. Thus, (2) follows directly from (1).
\end{proof}

\subsection{Isoclinism defines a correspondence between some subgroups}\label{sec:isoclinism-correspondence-some-subgroups}

Suppose $G_1$ and $G_2$ are isoclinic groups with an isoclinism
$(\zeta,\varphi): G_1 \to G_2$ where $\zeta:\operatorname{Inn}(G_1)
\to \operatorname{Inn}(G_2)$ and $\varphi:G_1' \to G_2'$ are the
component isomorphisms. Then, $\zeta$ gives a correspondence:

\begin{center}
Subgroups of $G_1$ that contain $Z(G_1)$ $\leftrightarrow$ Subgroups
of $G_2$ that contain $Z(G_2)$
\end{center}
This correspondence does not preserve the isomorphism type of the
subgroup, but it preserves some related structure. Explicitly, the
following hold whenever a subgroup $H_1$ of $G_1$ containing $Z(G_1)$
corresponds with a subgroup $H_2$ of $G_2$ containing $Z(G_2)$:

\begin{itemize}
\item $H_1/Z(G_1)$ is isomorphic to $H_2/Z(G_2)$.
\item $H_1$ and $H_2$ are isoclinic.
\item $H_1$ is normal in $G_1$ if and only if $H_2$ is normal in
  $G_2$, and if so, then $G_1/H_1$ is isomorphic to $G_2/H_2$.
\end{itemize}

We have a similar correspondence given by $\varphi$:

\begin{center}
  Subgroups of $G_1$ that are contained in $G_1'$ $\leftrightarrow$
  Subgroups of $G_2$ that are contained in $G_2'$
\end{center}

This correspondence preserves a number of structural
features. Explicitly, the following hold if a subgroup $H_1$ of $G_1'$
is in correspondence with a subgroup $H_2$ of $G_2'$:

\begin{itemize}
\item $H_1$ is isomorphic to $H_2$
\item $H_1$ is normal in $G_1'$ if and only if $H_2$ is normal in
  $G_2'$, and if so, then $G_1'/H_1$ is isomorphic to $G_2'/H_2$.
\item $H_1$ is normal in $G_1$ if and only if $H_2$ is normal in
  $G_2$, and if so, then $G_1/H_1$ is isoclinic to $G_2/H_2$.
\end{itemize}

The two correspondences discussed above may partially overlap, and
they agree with each other wherever they overlap. Explicitly, if $H_1$
is a subgroup of $G_1$ that satisfies {\em both the conditions} (it
contains $Z(G_1)$ and is contained in $G_1'$), then the subgroup $H_2$
obtained by both correspondences is identical.

\subsection{Characteristic subgroups, quotient groups, and subquotients determined by the group up to isoclinism}\label{sec:isoclinism-char-sub-quot}

The vast majority of characteristic subgroups that we see defined
(particularly for $p$-groups) are either contained in the derived
subgroup or contain the center. The exceptions are those such as the
socle and Frattini subgroup, which are smaller than the center and
larger than the derived subgroup respectively.

Based on the correspondences discussed in the preceding section, we
can deduce the following regarding important subgroups, quotients, and
subquotients of a group $G$ that are determined up to isomorphism by
knowing $G$ up to isoclinism:

\begin{itemize}
\item All lower central series member subgroups $\gamma_c(G), c \ge
  2$. Note that $\gamma_1(G) = G$ needs to be excluded. Further, the
  isomorphism types of successive quotients between lower central
  series members of the form $\gamma_i(G)/\gamma_j(G)$ with $j \ge i
  \ge 2$ are also determined by the knowledge of $G$ up to
  isoclinism. Note that the quotient groups $G/\gamma_c(G)$ are in
  general determined only up to isoclinism and not up to isomorphism.

\item All derived series member subgroups $G^{(i)}$, $i \ge 1$. Note
  that we need to exclude $G^{(0)} = G$. Further, the isomorphism
  types of quotients between derived series members of the form
  $G^{(i)}/G^{(j)}$ with $j \ge i \ge 1$ are also determined by the
  knowledge of $G$ up to isoclinism. Note that the quotient groups
  $G/G^{(i)}$ are determined only up to isoclinism and not up to
  isomorphism.
\item Quotients $G/Z^c(G)$ for all upper central series member
  subgroups $Z^c(G)$, $c \ge 1$. We need to exclude $c = 0$ which
  would give $G/Z^0(G) = G$. Further, the isomorphism types of
  subquotients of the form $Z^i(G)/Z^j(G)$ where $i \ge j \ge 1$ are
  also determined up to isomorphism by the knowledge of $G$ up to
  isoclinism. Note that the subgroups $Z^i(G)$ themselves are
  determined only up to isoclinism and not up to isomorphism.
\end{itemize}

\subsection{Correspondence between abelian subgroups}\label{sec:isoclinism-abelian-subgroups}

Suppose $G_1$ and $G_2$ are isoclinic groups. The following are true:

\begin{itemize}
\item The isoclinism establishes a correspondence between abelian
  subgroups of $G_1$ containing $Z(G_1)$ and abelian subgroups of
  $G_2$ containing $Z(G_2)$. Note that the abelian subgroups that are
  in correspondence are not necessarily isomorphic to each other. In
  fact, unless $Z(G_1)$ and $Z(G_2)$ have the same order, the abelian
  subgroups in correspondence need not even have the same order as
  each other.
\item The isoclinism establishes a correspondence between the abelian
  subgroups of $G_1$ that are self-centralizing and the abelian
  subgroups of $G_2$ that are self-centralizing. A self-centralizing
  abelian subgroup is an abelian subgroup that equals its own
  centralizer, or equivalently, it is a subgroup that is maximal among
  abelian subgroups of the group.
\item In the case that $G_1$ and $G_2$ are both finite, the isoclinism
  establishes a correspondence between abelian subgroups of maximum
  order in $G_1$ and abelian subgroups of maximum order in
  $G_2$.
\item Each of the correspondences above preserves normality.
\item If $G_1$ and $G_2$ are both finite, then each of the
  correspondences above preserves the index of the subgroups.
\end{itemize}

In particular, this means that isoclinic finite $p$-groups of the same
order have the same value for the maximum order of abelian subgroup,
the same value for the maximum order of abelian normal subgroup, and
the same values for the orders of self-centralizing abelian normal
subgroups.

\subsection{Constructing isoclinic groups}

Here are some ways of constructing groups isoclinic to a given group:

\begin{itemize}
\item Take a direct product with an abelian group.
\item Find a subgroup whose product with the center is the whole
  group. In symbols, if $H$ is a subgroup of $G$ and $HZ(G) = G$
  (where $Z(G)$ denotes the center of $G$), then $H$ is isoclinic to
  $G$. Note that for finite groups, this is the only way to find
  isoclinic subgroups to the whole group: a subgroup is isoclinic to
  the whole group if and only if its product with the center of the
  whole group is the whole group.
\end{itemize}

\subsection{Hall's purpose in introducing isoclinism}

Although this is not directly relevant, it might be helpful for
historical motivation to understand why Philip Hall introduced the
concept of isoclinism. At the time that Hall wrote his paper
\cite{Hall37}, very few systematic lists of finite $p$-groups of small
order were available. Existing classifications tended to be {\em ad
  hoc} and use a bunch of invariants. In hindsight, many of these
invariants were invariants up to isoclinism. As we saw in the
preceding section, this is true for information about conjugacy
classes and irreducible representations, and many important attributes
related to characteristic subgroups and their quotient
groups. However, since they were purely numerical invariants rather
than invariants capturing structural information, they were too weak
to meaningfully distinguish groups once the orders got large. Below
are some invariants that are ``good enough'' to uniquely determine
groups up to isoclinism for small orders, but fail at larger
orders. The second column gives the smallest $n$ for which there exist
groups of order $2^n$ that have the same value of the invariant but
are not isoclinic to each other.

\vspace{0.3in}

\begin{small}
\begin{table}[htbp]
\caption{The smallest $n$ for which a given isoclinism-invariant fails to classify groups of order $2^n$ up to isoclinism}\label{T1}
\begin{tabular}{|l|l|l|}
  \hline
  Isoclinism-invariant (fixed order) & Smallest $n$ where it fails to classify \\\hline
  Derived length & 4 \\\hline
  Nilpotency class & 5 \\\hline
  Conjugacy class sizes & 5 \\\hline
  Degrees of irreducible representations & 5 \\\hline
  Inner automorphism group & 6 \\\hline
  Derived subgroup & 5 \\\hline
  Inner automorphism group, derived subgroup & 6 \\\hline
\end{tabular}
\end{table}
\end{small}

\vspace{0.3in}

As the orders get bigger, numerical invariants becomes progressively
more inadequate in describing the structure. They are also not helpful
to computing the algebraic structure of the group.

As indicated in the table above, even knowledge of the inner
automorphism group and the derived subgroup up to isomorphism does not
determine the group uniquely up to isoclinism, with the smallest
counterexamples occurring for order $2^6$. The commutator {\em map} is
crucial to describing the group structure up to isoclinism.

Hall sought to introduce a systematic procedure that could be used to
generate all the $p$-groups of a particular order based on smaller
groups, and group them together in ways that made it easy to compute
and remember important invariants (such as their nilpotency class,
number of conjugacy classes, etc.)

The use of isoclinism allows for a recursive procedure to go from
order $p^{n-1}$ to $p^n$. In broad strokes, the idea is as follows:

\begin{itemize}
\item Assume we have classified all the groups of order up to
  $p^{n-1}$, and we need to classify groups of order $p^n$.

\item First, we need to identify the equivalence classes up to
  isoclinism for groups of order $p^n$. This involves identifying
  candidate pairs of inner automorphism group and derived subgroup
  with a candidate for the commutator map. Note that the concept of
  ``candidate for the commutator map'' is somewhat problematic without
  reference to the ambient group, but we will see later that it can be
  made precise using the concept of exterior squares. Hall did not
  have this formalism at his disposal, but used a similar idea in a
  more {\em ad hoc} fashion in his classification efforts.

\item For each such equivalence class up to isoclinism, identify all
  the groups of order $p^n$ up to isomorphism in that equivalence
  class up to isoclinism.

\end{itemize}

Our purpose differs somewhat from Hall's, but is broadly in the same
spirit. Instead of classifying groups, we are interested in
identifying some regular aspects of their behavior.

For a detailed classification that builds on Hall's ideas, see
\cite{HallSenior}, which classifies groups of order $2^n$ for $n \le
6$. The classification of groups of order $2^n$ for $n \ge 7$ was done
using somewhat different methods. Specifically, the focus shifted from
using isoclinism (which is based on the central series) to using the
exponent-$p$ central series, and computing immediate descendants based
on the exponent-$p$ central series. This is more amenable to
computation because we are working with central extensions where the
base group is elementary abelian. Algorithms in this genre are termed
{\em nilpotent quotient algorithms}. See \cite{Order128}
(classification for order $2^7 = 128$), \cite{Order256}
(classification for order $2^8 = 256$), and \cite{enumeratingpgroups}
(general description of the classification strategy) for more details.

\subsection{Stem groups for a given equivalence class under isoclinism}\label{sec:stem-group}

Every equivalence class of groups under isoclinism contains one or
more {\em stem groups}. A group $G$ is a stem group if $Z(G) \le
G'$. All stem groups for a given equivalence class under
isoclinism have the same order, and the order of any isoclinic group is
a multiple of this order.

Hall stated this fact, with a sketch of a proof, in his 1937 paper
introducing isoclinism. We will provide a proof of the statement in
Section \ref{sec:stem-group-existence} using modern language.

Here are some examples of stem groups:

\begin{itemize}
\item For the class of abelian groups, the unique stem group
  is the trivial group. 

\item For groups of class two with inner automorphism group a Klein
  four-group and derived subgroup of order two, there are two
  possibilities for the stem group: the dihedral group of order eight
  and the quaternion group of order eight.

\end{itemize}

Unlike what one might naively expect, it is not true that all groups
in the equivalence class under isoclinism contain a stem group as a
subquotient. For instance, the group $M_{16} = M_4(2)$ given as
$\langle a,x \mid a^8 = x^2 = 1, xax = a^5 \rangle$ is a non-abelian
group of order $16$.\footnote{The group has ID (16,6) in the
  SmallGroups library available for GAP and Magma.} This group is
isoclinic to the dihedral group of order eight and the quaternion
group of order eight, which are the only stem groups in that
equivalence class up to isoclinism. However, $M_{16}$ does not have
any subgroup, quotient, or subquotient isomorphic to either of these
groups. In fact, every proper subquotient of $M_{16}$ is abelian.

\subsection{Some low order classification information}

In this section, we provide a quick summary of the classification of
groups of order $2^n$ and groups of order $p^n$ (for odd $p$) for
small $n$, based on isoclinism. A detailed exposition can be found in
\cite{HallSenior} and also in some online sources included in the
appendix. is also possible to explore these groups using a
computational algebra package such as GAP or Magma. More information
about exploring group information in GAP is available in the appendix.

The most salient information is provided below.

{\bf For groups of order $2^n$}: Note that the last column is the
number of equivalence classes up to isoclinism of the preceding
column. It can be computed by subtracting from the value of the
preceding column the value in the row above for the preceding column.

\begin{small}
\begin{table}[htbp]
\caption{Number of equivalence classes up to isoclinism for groups of
  order $2^n$}\label{T2}
%\rowcolors{1}{green}{pink}
\begin{tabular}{|l|l|l|l|l|}
  \hline
  $n$ & $2^n$ & Number of groups & Number up to isoclinism & ``New'' equivalence classes\\
  \hline
  0 & 1 & 1 & 1 & 1\\\hline
  1 & 2 & 1 & 1 & 0\\\hline
  2 & 4 & 2 & 1 & 0\\\hline
  3 & 8 & 5 & 2 & 1\\\hline
  4 &16 &14 & 3 & 1\\\hline
  5 &32 &51 & 8 & 5\\\hline
  6 &64&267 & 27& 19\\\hline
  7&128&2328&115& 88\\\hline
\end{tabular}
\end{table}
\end{small}
{\bf For groups of order $p^n$, $p \ge 3$}: The details depend on $p$,
but the classification up to $p^4$ is independent of $p$, so we
construct the table for $n$ up to $4$:

\begin{small}
\begin{table}[htbp]
\caption{Number of equivalence classes up to isoclinism for groups of
  order $p^n$}\label{T3}
\begin{tabular}{|l|l|l|l|l|}
  \hline
  $n$ & $p^n$ & Number of groups & Number up to isoclinism & ``New'' equivalence classes\\
  \hline
  0 & 1 & 1 & 1 & 1\\\hline
  1 & 2 & 1 & 1 & 0\\\hline
  2 & 4 & 2 & 1 & 0\\\hline
  3 & 8 & 5 & 2 & 1\\\hline
  4 &16 &15 & 3 & 1\\\hline
\end{tabular}
\end{table}
\end{small}
%\newpage

\section{Isoclinism and homoclinism for Lie rings}\label{sec:isoclinism-and-homoclinism-lie}

The goal of this section is to establish the basic theory of {\em
  isoclinism} and {\em homoclinism} for {\em Lie rings}. The theory is
analogous to the theory for groups developed in the preceding section
(Section \ref{sec:isoclinism-and-homoclinism}. 

Informally, a homoclinism of Lie rings is a homomorphism between the
{\em Lie bracket structures} of the Lie rings. Informally, two Lie
rings are isoclinic if their Lie bracket maps are
equivalent. Isoclinism defines an equivalence relation on the
collection of Lie rings. Under this equivalence relation, all abelian
Lie rings are equivalent to the trivial group.

The original results that we present later (Section \ref{sec:bcuti}
and \ref{sec:lcuti}) describe bijective correspondences between
certain equivalence classes of groups and certain equivalence classes
of Lie rings. The equivalence classes of Lie rings are based on the
equivalence relation of isoclinism of Lie rings.

\subsection{Definitions of homoclinism and isoclinism}\label{sec:isoclinism-definition-lie}

The notion of isoclinism of Lie algebras seems to have been introduced
by Moneyhun in \cite{Moneyhun}. We introduce a corresponding notion of
homoclinism to parallel the notion for groups. The historical origin
of the notion of homoclinism for Lie rings is unclear, but it appears
for instance in the paper \cite{Moghaddametal} published in 2011.

For simplicity, we restrict attention to the case of Lie {\em rings},
which are Lie algebras over the ring of integers. All our definitions
and theorems here have very natural analogues in Lie algebras over
other commutative unital rings. Note that if two Lie algebras over a
commutative unital ring are isoclinic as Lie algebras over that ring,
they are also isoclinic as Lie rings. In the Appendix, Section
\ref{appsec:Lie}, we describe the theory of Lie algebras over
arbitrary commutative unital rings, and how the general theory of Lie
algebras relates to the theory of Lie rings.

For a Lie ring $L$, denote by $\operatorname{Inn}(L)$ the inner
derivation Lie ring of $L$, denote by $L'$ the derived subring of $L$,
and denote by $Z(L)$ the center of $L$. $\operatorname{Inn}(L)$ is
canonically isomorphic to the quotient ring $L/Z(L)$. (For other
notation related to Lie rings that we use in this document, see
Section \ref{sec:lie-ring-notation}).

The {\em Lie bracket} map in $L$ descends to a map:

$$\omega_L: \operatorname{Inn}(L) \times \operatorname{Inn}(L) \to L'$$

Note that $\omega_L$ is $\mathbb{Z}$-bilinear, but the additional
structure on it (that is forced from its arising as a Lie bracket) is
hard to describe explicitly. In Section \ref{sec:exteriorsquare-lie},
we will describe a structure called the exterior square of a Lie ring
and reframe the condition on $\omega_L$ as being a bilinear map that
induces a homomorphism from the exterior square. This situation is
similar to the situation for groups we discussed earlier, but somewhat
easier to describe because of the underlying additive group structure.

Suppose $L_1$ and $L_2$ are Lie rings. The Lie brackets of $L_1$ and
$L_2$ respectively induce $\Z$-bilinear maps:

$$\omega_{L_1}: \operatorname{Inn}(L_1) \times \operatorname{Inn}(L_1) \to L_1'$$

$$\omega_{L_2}: \operatorname{Inn}(L_2) \times \operatorname{Inn}(L_2) \to L_2'$$

A {\em homoclinism} from $L_1$ to $L_2$ is a pair of homomorphisms
$(\zeta,\varphi)$ where $\zeta$ is a homomorphism from
$\operatorname{Inn}(L_1)$ to $\operatorname{Inn}(L_2)$ and $\varphi$ is an
homomorphism from $L_1'$ to $L_2'$, satisfying the condition that:

$$\varphi \circ \omega_{L_1} = \omega_{L_2} \circ (\zeta \times \zeta)$$

More explicitly, for any $x,y \in \operatorname{Inn}(L_1)$, we require that:

$$\varphi(\omega_{L_1}(x,y)) = \omega_{L_2}(\zeta(x),\zeta(y))$$

Pictorially, the following diagram commutes:

$$\begin{array}{ccc}
  \operatorname{Inn}(L_1) \times \operatorname{Inn}(L_1) & \stackrel{\zeta \times \zeta}{\to} & \operatorname{Inn}(L_2) \times \operatorname{Inn}(L_2) \\
  \downarrow^{\omega_{L_1}}  & & \downarrow^{\omega_{L_2}}\\
  L_1' & \stackrel{\varphi}{\to} & L_2'\\
\end{array}$$

The homoclinism $(\zeta,\varphi)$ from $L_1$ to $L_2$ is termed an
{\em isoclinism} if both $\zeta$ and $\varphi$ are isomorphisms of Lie
rings.

We compose homoclinisms of Lie rings by separately composing the
homomorphisms on the inner derivation Lie rings and on the derived
subrings. Explicitly, suppose $(\zeta_{12},\varphi_{12})$ is a
homoclinism from $L_1$ to $L_2$ and suppose
$(\zeta_{23},\varphi_{23})$ is a homoclinism from $L_2$ to
$L_3$. Then, the composite $(\zeta_{23},\varphi_{23}) \circ
(\zeta_{12}, \varphi_{12})$ is $(\zeta_{23} \circ \zeta_{12},
\varphi_{23} \circ \varphi_{12})$. The proof that this works follows
from the commutativity of this diagram.

$$\begin{array}{ccccc}
  \operatorname{Inn}(L_1) \times \operatorname{Inn}(L_1) & \stackrel{\zeta_{12} \times \zeta_{12}}{\to} & \operatorname{Inn}(L_2) \times \operatorname{Inn}(L_2) & \stackrel{\zeta_{23} \times \zeta_{23}}{\to} & \operatorname{Inn}(L_3) \times \operatorname{Inn}(L_3)\\
  \downarrow^{\omega_{G_1}}  & & \downarrow^{\omega_{G_2}} & & \downarrow^{\omega_{G_3}}\\
  L_1' & \stackrel{\varphi_{12}}{\to} & L_2' & \stackrel{\varphi_{23}}{\to} & L_3'\\
\end{array}$$

As was the case with groups, we can define a category where the
morphisms are homoclinisms.

\begin{definer}[Category of Lie rings with homoclinisms]
  The {\em category of Lie rings with homoclinisms} is defined as the
  following category:

  \begin{itemize}
  \item The {\em objects} of the category are Lie rings.
  \item The {\em morphisms} of the category are homoclinisms of Lie rings.
  \item Composition of morphisms is composition of homoclinisms.
  \item The identity morphism is the identity homoclinism: it is the
    identity map both on the inner derivation Lie ring and on the
    derived subring.
  \end{itemize}
\end{definer}

In the category of Lie rings with homoclinisms, the isomorphisms
(i.e., the invertible morphisms) are precisely the isoclinisms.

\subsection{Homomorphisms and homoclinisms}

Suppose $L_1$ and $L_2$ are Lie rings and $\theta: L_1 \to L_2$ is a
homomorphism of Lie rings. If $\theta$ satisfies the property that
$\theta(Z(L_1)) \le Z(L_2)$, then $\theta$ induces a homoclinism of
Lie rings. Explicitly the homoclinism induced by $\theta$ is defined as
$(\zeta,\varphi)$ where $\zeta$ and $\varphi$ are as defined below.

\begin{itemize}
\item Since $\theta(Z(L_1)) \le Z(L_2)$, $\theta$ descends to a
  homomorphism from $L_1/Z(L_1) \cong \operatorname{Inn}(L_1)$ to
  $L_2/Z(L_2) \cong \operatorname{Inn}(L_2)$. Denote by $\zeta$ the
  induced homomorphism $\operatorname{Inn}(L_1) \to
  \operatorname{Inn}(L_2)$.
\item The restriction of $\theta$ to $L_1'$ maps inside $L_2'$. Denote
  by $\varphi$ the induced map $L_1' \to L_2'$.
\end{itemize}

It is easy to verify that $(\zeta,\varphi)$ defines a homoclinism.

Note that the condition $\theta(Z(L_1)) \le Z(L_2)$ is necessary in
order to be able to construct $\zeta$.

The following are true:

\begin{itemize}
\item Every {\em surjective} homomorphism $\theta:L_1 \to L_2$
  satisfies the condition that $\theta(Z(L_1)) \le Z(L_2)$. Thus,
  every surjective homomorphism induces a homoclinism.
\item The inclusion of a Lie subring $M$ in a Lie ring $L$ satisfies
  the condition if and only if $Z(M) \le Z(L)$, or equivalently, $Z(M)
  = M \cap Z(L)$. Thus, these are the subrings whose inclusions
  induce homoclinisms.
\end{itemize}

\subsection{Miscellaneous results on homoclinisms and words}\label{sec:homoclinism-misc-results-lie}

\begin{lemma}
  Suppose $(\zeta,\varphi)$ is a homoclinism of Lie rings $L_1$ and
  $L_2$, where $\zeta:\operatorname{Inn}(L_1) \to
  \operatorname{Inn}(L_2)$ and $\varphi:L_1' \to L_2'$ are the
  component homomorphisms. Denote by $\theta_1:L_1' \to
  \operatorname{Inn}(L_1)$ the composite of the inclusion of $L_1'$ in
  $L_1$ and the projection from $L_1$ to $L_1/Z(L_1) =
  \operatorname{Inn}(L_1)$. Similarly define $\theta_2:L_2' \to
  \operatorname{Inn}(L_2)$. Then, we have:
  
  $$\zeta \circ \theta_1 = \theta_2 \circ \varphi$$

  or equivalently, for any $w \in L_1'$:

  $$\zeta(\theta_1(w)) = \theta_2(\varphi(w))$$
\end{lemma}

\begin{proof}
  To show the equality of the two expressions, it suffices to show
  equality on a generating set for $L_1'$. By definition, the set of
  Lie brackets of elements in $L_1$ is a generating set for
  $L_1'$. Thus, it suffices to show that:

  $$\zeta(\theta_1([u,v])) = \theta_2(\varphi([u,v])) \ \forall \ u,v \in L_1$$

  This is equivalent to showing that:

  $$\zeta(\theta_1(\omega_{L_1}(x,y))) = \theta_2(\varphi(\omega_{L_1}(x,y))) \ \forall \ x,y \in \operatorname{Inn}(L_1)$$

  Let us examine the left and right sides separately. 

  {\em The left side}: The expression $\theta_1(\omega_{L_1}(x,y))$
  first computes the Lie bracket of lifts of $x$ and $y$ in $L_1$, then
  projects to $L_1/Z(L_1)$. This is equivalent to directly computing
  the Lie bracket in $L_1/Z(L_1)$, so $\theta_1(\omega_{L_1}(x,y)) =
  [x,y]$. Thus, the left side becomes $\zeta([x,y])$.

  {\em The right side}: By the definition of homoclinism,
  $\varphi(\omega_{L_1}(x,y)) = \omega_{L_2}(\zeta(x),\zeta(y))$. The
  right side now becomes
  $\theta_2(\omega_{L_2}(\zeta(x),\zeta(y)))$. In other words, we are
  taking the lifts of $\zeta(x)$ and $\zeta(y)$ in $L_2$, then
  computing the Lie bracket, then projecting to $L_2/Z(L_2)$. This is
  equivalent to directly computing the Lie bracket in $L_2/Z(L_2)$, so
  the right side simplifies to $[\zeta(x),\zeta(y)]$. Since $\zeta$ is
  a homomorphism, this is equal to $\zeta([x,y])$, and hence agrees
  with the left side.
\end{proof}

We state two important theorems.

\begin{theorem}\label{thm:iterated-bracket-descends-to-inn}
  Suppose $w(g_1,g_2,\dots,g_n)$ is a word in $n$ letters with the
  property that $w$ evaluates to the zero element in {\em any}
  abelian Lie ring. This is equivalent to saying that $w$, viewed as an
  element of the free Lie ring on $g_1,g_2,\dots,g_n$, is in the derived
  subring. Then, for any Lie ring $L$, the word map $w:L^n \to L$ obtained
  by evaluating $w$ descends to a map:

  $$\chi_{w,L}: (\operatorname{Inn}(L))^n  \to L'$$

  Any word $w$ that is an iterated Lie bracket (with any bracketing)
  satisfies this condition.
\end{theorem}

\begin{proof}
  Denote by $\nu:L \to \operatorname{Inn}(L)$ the quotient map.

  $w$ can be written in the form:

  $$w(g_1,g_2,\dots,g_n) = \sum_{i=1}^m [u_i(g_1,g_2,\dots,g_n),v_i(g_1,g_2,\dots,g_n)]$$

  where $u_i,v_i, 1 \le i \le m$ are words. Suppose $y_i \in L$ are
  elements for which $\nu(y_i) = x_i$. Then:

  $$w(y_1,y_2,\dots,y_n) = \sum_{i=1}^m [u_i(y_1,y_2,\dots,y_n),v_i(y_1,y_2,\dots,y_n)]$$

  We have that:

  $$\nu(u_i(y_1,y_2,\dots,y_n)) = u_i(x_1,x_2,\dots,x_n), \qquad \nu(v_i(y_1,y_2,\dots,y_n)) = v_i(x_1,x_2,\dots,x_n)$$

  Thus, we obtain that:

  $$[u_i(y_1,y_2,\dots,y_n),v_i(y_1,y_2,\dots,y_n)] = \omega_L(u_i(x_1,x_2,\dots,x_n),v_i(x_1,x_2,\dots,x_n))$$

  In particular, the expression
  $[u_i(y_1,y_2,\dots,y_n),v_i(y_1,y_2,\dots,y_n)]$ depends only on
  $x_1$, $x_2$, $\dots$, $x_n$ and not on the choice of lifts $y_i$. Thus, the
  sum $w(y_1,y_2,\dots,y_n)$ also depends only on the values of
  $x_i$, and we obtain the function:

  $$\chi_{w,L}(x_1,x_2,\dots,x_n) = \sum_{i=1}^m \omega_L(u_i(x_1,x_2,\dots,x_n),v_i(x_1,x_2,\dots,x_n))$$
\end{proof}

\begin{theorem}\label{thm:iterated-bracket-commutes-homoclinisms}
  Suppose $(\zeta,\varphi)$ is a homoclinism of Lie rings $L_1$ and
  $L_2$, where $\zeta:\operatorname{Inn}(L_1) \to
  \operatorname{Inn}(L_2)$ and $\varphi:L_1' \to L_2'$ are the
  component homomorphisms. Then for any word $w(g_1,g_2,\dots,g_n)$
  that is trivial in every abelian Lie ring (as described above), we have:

  $$\chi_{w,L_2}(\zeta(x_1),\zeta(x_2),\dots,\zeta(x_n)) = \varphi(\chi_{w,L_1}(x_1,x_2,\dots,x_n))$$

  for all $x_1,x_2,\dots,x_n \in \operatorname{Inn}(L)$.

  Any word $w$ that is an iterated Lie bracket (with any order of
  bracketing) satisfies this condition, and the theorem applies to
  such word maps.
\end{theorem}

\begin{proof}
  Denote by $\nu_1:L_1 \to \operatorname{Inn}(L_1)$ and $\nu_2:L_2 \to
  \operatorname{Inn}(L_2)$ the canonical quotient maps.

  We use the same notation and steps in the proof of the preceding
  theorem, replacing $L$ by $L_1$. We obtain:

  $$w(g_1,g_2,\dots,g_n) := \sum_{i=1}^m [u_i(g_1,g_2,\dots,g_n),v_i(g_1,g_2,\dots,g_n)]$$

  where $u_i,v_i, 1 \le i \le m$ are words. Suppose $y_i \in L_1$ are
  elements for which $\nu_1(y_i) = x_i$. As demonstrated in the proof of
  the preceding theorem:

  \begin{equation*}
    \chi_{w,L_1}(x_1,x_2,\dots,x_n) = \sum_{i=1}^m \omega_{L_1}(u_i(x_1,x_2,\dots,x_n),v_i(x_1,x_2,\dots,x_n)) \tag{$\dagger$}
  \end{equation*}

  Suppose $z_i \in L_2$ are elements for which $\nu_2(z_i) =
  \zeta(x_i)$. Similar reasoning to the above yields that:

  \begin{small}
  \begin{equation*}
    \chi_{w,L_2}(\zeta(x_1),\zeta(x_2),\dots,\zeta(x_n)) = \sum_{i=1}^m \omega_{L_2}(u_i(\zeta(x_1),\zeta(x_2),\dots,\zeta(x_n)),v_i(\zeta(x_1),\zeta(x_2),\dots,\zeta(x_n))) \tag{$\dagger\dagger$}
  \end{equation*}
  \end{small}

  Apply $\varphi$ to both sides of $(\dagger)$, use the defining
  property of homoclinisms, and compare with $(\dagger\dagger)$ to
  obtain the result.
\end{proof}

\subsection{Isoclinic Lie rings: how similar are they?}

We say that $L_1$ and $L_2$ are isoclinic Lie rings if there is an
isoclinism of Lie rings from $L_1$ to $L_2$. The relation of being
isoclinic is an equivalence relation. Briefly:

\begin{itemize}
\item The relation of being isoclinic is {\em reflexive} because we
  can choose both the isomorphisms to be the identity
  maps. Explicitly, for any Lie ring $L$,
  $(\operatorname{id}_{\operatorname{Inn}(L)},\operatorname{id}_{L'})$
  defines an isoclinism from $L$ to itself.
\item The relation of being isoclinic is {\em symmetric} because we
  can take the inverse isomorphisms to both the
  isomorphisms. Explicitly, if $(\zeta,\varphi)$ describes the
  isoclinism from $L_1$ to $L_2$, then $(\zeta^{-1},\varphi^{-1})$
  describes the isoclinism from $L_2$ to $L_1$.
\item The relation of being isoclinic is {\em transitive} because we
  can compose both kinds of isomorphisms separately. Explicitly, if
  $(\zeta_{12},\varphi_{12})$ describes the isoclinism from $L_1$ to
  $L_2$ and $(\zeta_{23},\varphi_{23})$ describes the isomorphism from
  $L_2$ to $L_3$, then $(\zeta_{23} \circ \zeta_{12}, \varphi_{23}
  \circ \varphi_{12})$ describes the isoclinism from $L_1$ to $L_3$.
\end{itemize}

Here is an alternative way of seeing that being isoclinic is an
equivalence relation: isoclinisms are precisely the isomorphisms in
the category of Lie rings with homoclinisms, and being isomorphic in any
category is an equivalence relation.

We first list some very obvious similarities between isoclinic Lie rings.

\begin{itemize}
\item They have isomorphic derived subrings: This is direct from the
  definition, which includes an isomorphism of the derived subrings.
\item They have isomorphic inner derivation Lie rings: This is direct
  from the definition, which includes an isomorphism between the inner
  derivation Lie rings.
\item They have precisely the same non-abelian composition factors (if
  the composition factors do exist): Since the center is abelian, all
  the non-abelian composition factors occur inside the inner
  derivation Lie ring for both, which we know to be isomorphic.
\item If one is nilpotent, so is the other, and they have the same
  nilpotency class (with the exception of class zero getting conflated
  with class one): The nilpotency class is one more than the
  nilpotency class of the inner derivation Lie ring.
\item If one is solvable, so is the other, and they have the same
  derived length (with the exception of length zero getting conflated
  with length one): The derived length is one more than the derived
  length of the derived subring.
\end{itemize}

Note that conjugacy class sizes and degrees of irreducible
representations do not make direct sense for Lie rings. But there are
important analogues of these statements that apply in a number of
cases. In Section \ref{sec:kirillov-orbit-method}, we discuss the
Kirillov orbit method, which relates the irreducible representations
of a group with its Lazard Lie ring.

\subsection{Isoclinism defines a correspondence between some subrings}\label{sec:isoclinism-correspondence-some-subrings}

Suppose $L_1$ and $L_2$ are isoclinic Lie rings with an isoclinism
$(\zeta,\varphi): L_1 \to L_2$ where $\zeta:\operatorname{Inn}(L_1)
\to \operatorname{Inn}(L_2)$ and $\varphi:L_1' \to L_2'$ are the
component isomorphisms. Then, $\zeta$ gives a correspondence:

Lie subrings of $L_1$ that contain $Z(L_1)$ $\leftrightarrow$
Lie subrings of $L_2$ that contain $Z(L_2)$

This correspondence does not preserve the isomorphism type of the
subring, but it preserves some related structure. Explicitly, the
following hold whenever a subring $M_1$ of $L_1$ corresponds with a
subring $M_2$ of $L_2$:

\begin{itemize}
\item $M_1/Z(L_1)$ is isomorphic to $M_2/Z(L_2)$.
\item $M_1$ and $M_2$ are isoclinic.
\item $M_1$ is an ideal in $L_1$ if and only if $M_2$ is an ideal in
  $L_2$, and if so, then $L_1/M_1$ is isomorphic to $L_2/M_2$.
\end{itemize}

We have a similar correspondence given by $\varphi$:

Lie subrings of $L_1$ that are contained in $L_1'$ $\leftrightarrow$
Lie subrings of $L_2$ that are contained in $L_2'$

This correspondence preserves a number of structural
features. Explicitly, the following hold whenever a subring $M_1$ of
$L_1$ is in correspondence with a subring $M_2$ of $L_2$:

\begin{itemize}
\item $M_1$ is isomorphic to $M_2$
\item $M_1$ is an ideal in $L_1'$ if and only if $M_2$ is an ideal in
  $L_2'$, and if so, then $L_1'/M_1$ is isomorphic to $L_2'/M_2$.
\item $M_1$ is an ideal in $L_1$ if and only if $M_2$ is an ideal in
  $L_2$, and if so, then $L_1/M_1$ is isoclinic to $L_2/M_2$.
\end{itemize}

The two correspondences discussed above overlap somewhat, and they
agree with each other wherever they overlap. Explicitly, if $M_1$ is a
subring of $L_1$ that contains $Z(L_1)$ and is contained in $L_1'$,
then the $M_2$ obtained by both correspondences is identical.

\subsection{Characteristic subrings, quotient rings, and subquotients determined by the Lie ring up to isoclinism}

We can deduce the following regarding important characteristic
subrings, quotients, and subquotients of a Lie ring $L$ that are
determined up to isomorphism by knowing $L$ up to isoclinism:

\begin{itemize}
\item All lower central series member subrings $\gamma_c(L), c \ge
  2$. Note that $\gamma_1(L) = L$ needs to be excluded. Further, the
  isomorphism types of successive quotients between lower central
  series members of the form $\gamma_i(L)/\gamma_j(L)$ with $j \ge i
  \ge 2$ are also determined by the knowledge of $L$ up to
  isoclinism. Note that the quotient rings $L/\gamma_c(L)$ are in
  general determined only up to isoclinism and not up to isomorphism.

\item All derived series member subrings $L^{(i)}$, $i \ge 1$. Note
  that we need to exclude $L^{(0)} = L$. Further, the isomorphism
  types of quotients between derived series members of the form
  $L^{(i)}/L^{(j)}$ with $j \ge i \ge 1$ are also determined by the
  knowledge of $L$ up to isoclinism. Note that the quotient Lie rings
  $L/L^{(i)}$ are determined only up to isoclinism and not up to
  isomorphism.
\item Quotients $L/Z^c(L)$ for all upper central series member
  subrings $Z^c(L)$, $c \ge 1$. We need to exclude $c = 0$ which
  would give $L/Z^0(L) = L$. Further, the isomorphism types of
  subquotients of the form $Z^i(L)/Z^j(L)$ where $i \ge j \ge 1$ are
  also determined up to isomorphism by the knowledge of $L$ up to
  isoclinism. Note that the subrings $Z^i(L)$ themselves are
  determined only up to isoclinism and not up to isomorphism.
\end{itemize}

\subsection{Correspondence between abelian subrings}\label{sec:isoclinism-abelian-subrings}

Suppose $L_1$ and $L_2$ are isoclinic Lie rings. The following are true:

\begin{itemize}
\item The isoclinism establishes a correspondence between abelian
  subrings of $L_1$ containing $Z(L_1)$ and abelian subrings of $L_2$
  containing $Z(L_2)$. Note that the abelian subrings that are in
  correspondence are not necessarily isomorphic to each other. In
  fact, unless $Z(L_1)$ and $Z(L_2)$ have the same order, the abelian
  subrings in correspondence need not even have the same order as
  each other.
\item The isoclinism establishes a correspondence between the abelian
  subrings of $L_1$ that are self-centralizing and the abelian
  subrings of $L_2$ that are self-centralizing. A self-centralizing
  abelian subring is an abelian subring that equals its own
  centralizer, or equivalently, it is maximal among abelian subrings
  of the Lie ring.
\item In the case that $L_1$ and $L_2$ are both finite, the isoclinism
  establishes a correspondence between abelian subrings of maximum
  order in $L_1$ and abelian subrings of maximum order in
  $L_2$.
\end{itemize}

If $L_1$ and $L_2$ are both finite, then each of the correspondences
above preserves the index of the subrings. In particular, this means
that isoclinic finite Lie rings of the same order have the same value
for the maximum order of abelian subring, the same value for the
maximum order of abelian ideal, and the same value for the orders of
self-centralizing abelian subrings.

\subsection{Constructing isoclinic Lie rings}

Here are some ways of constructing Lie rings isoclinic to a given Lie ring:

\begin{itemize}
\item Take a direct product with an abelian Lie ring.
\item Find a Lie subring whose sum with the center is the whole Lie
  ring. In symbols, if $M$ is a subring of $L$ and $M + Z(L) = L$
  (where $Z(L)$ denotes the center of $L$), then $M$ is isoclinic to
  $L$. Note that for finite Lie rings, this is the only way to find
  isoclinic subrings to the whole Lie ring: a subring is isoclinic to
  the whole Lie ring if and only if its sum with the center of the
  whole ring is the whole ring.
\end{itemize}
