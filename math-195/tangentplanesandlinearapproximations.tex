\documentclass[10pt]{amsart}
\usepackage{fullpage,hyperref,vipul,graphicx}
\title{Tangent planes and linear approximations}
\author{Math 195, Section 59 (Vipul Naik)}

\begin{document}
\maketitle

{\bf Corresponding material in the book}: Section 14.4. {\em Note: We
are, for now, omitting the topic of differentials, which is the second
half of this section (Pages 931--934). We may return to it later in
the course, if we get time after completing the rest of Chapters 14
and 15.}

{\bf What students should definitely get}: Finding the tangent plane
at a point, the concept of best linear approximation.

\section*{Executive summary}

Words ...

\begin{enumerate}
\item For a $d$-dimensional subset of $\R^n$, it (occasionally) makes
  sense to talk of the tangent space and the normal space at a
  point. The tangent space is a linear/affine $d$-dimensional space
  and the normal space is a linear/affine $(n - d)$-dimensional
  space. Both pass through the point and are mutually orthogonal.
\item For a function $z = f(x,y)$, the tangent plane to the graph of
  this function (a surface in $\R^3$) at the point
  $(x_0,y_0,f(x_0,y_0))$ {\em such that $f$ is differentiable at the
    point $(x_0,y_0)$} is the plane:

  $$z - f(x_0,y_0) = f_x(x_0,y_0)(x - x_0) + f_y(x_0,y_0)(y - y_0)$$

  The corresponding linear function we get is:

  $$L(x,y) = f(x_0,y_0) + f_x(x_0,y_0)(x - x_0) + f_y(x_0,y_0)(y - y_0)$$

  This provides a linear approximation to the function near the point
  where we are computing the tangent plane.
\item It may be the case that a function $f$ of two variables is not
  differentiable at a point in its domain but the partial derivatives
  exist. In this case, although the {\em above formula makes sense as
    a formula}, the plane it gives {\em is not the tangent plane} --
  in fact, {\em no tangent plane exists}. Similarly, {\em no
    linearization exists}, and the linear function given by the above
  formula {\em is not a close approximation to the function near the
    point}.
\end{enumerate}
\section{Approximation theory: recall of one variable}

\subsection{Taylor polynomials}

For a function $f$ of one variable, we can define some Taylor
polynomials of $f$. If $f$ is $n$ times differentiable around a point
$x = c$, we can define the $n^{th}$ Taylor polynomial of $f$ about $c$
as the polynomial:

$$P_n(f,c)(x) := \sum_{k=0}^n \frac{f^{(k)}(c)}{k!}(x - c)^k$$

If $f$ is $(n + 1)$ times differentiable, it turns out that $f -
P_n(f,c)$ is a function with a zero of order at least $n + 1$ at
$c$. This means that we can approximate $f$ by a {\em polynomial
function} of degree at most $n$ such that the error term is very
zeroey (order at least $n + 1$). Note that the Taylor polynomial has
degree at most $n$, but the degree may be less than $n$ if the
$n^{th}$ derivative takes the value $0$ at $c$.

\subsection{Special cases: degrees zero and one}

The zeroth degree Taylor polynomial is a constant function with the
same value as $f(c)$. This is a very crude description of the function
around the point and ignores any change in the value.

The first degree Taylor polynomial is the function whose graph gives
the tangent line, i.e., the line:

$$y = f(c) + f'(c)(x - c)$$

Note that this is the tangent line in point-slope form.

The first degree Taylor polynomial, or the tangent line, captures the
rate of change of the function {\em at} the point. However, it fails
to capture second derivative and higher derivative behavior, i.e., how
this rate of change itself is changing.

{\em Geometrically}, the tangent line to a point on a curve is the
``best linear approximation'' to the curve locally around the point,
i.e., the line that comes closest to describing the curve near the
point. Note that it is {\em not} true that the tangent line intersects
the curve at a unique point, or that no other line has this property.

\section{The multivariable situation}

\subsection{Geometric notion of tangent space and normal space}

Given a subset of dimension $d$ in $n$-dimensional space, we can try
talking of a tangent space at a point on the subset to the
subset. This attempt at talk succeeds only when that subset has some
particularly nice properties. Anyway, the point is that this tangent
space looks like a {\em flat} $d$-dimensional space, i.e., a line, or
a plane, or a higher-dimensional analogue thereof.

For instance, the ``tangent'' to a curve (which is a one-dimensional
construct) is a line (the linear one-dimensional
construct). Similarly, the tangent to a surface in $\R^3$ is a plane
(a linear two-dimensional construct). To take the example of a sphere,
think of a sphere resting on a floor. The floor is the tangent plane
to the sphere through the point of contact.

The ``normal space'' at a point to a $d$-dimensional subset in
$n$-dimensional space is a $(n - d)$-dimensional linear space through
the same point, such that the two spaces intersect orthogonally, i.e.,
every direction in one space is orthogonal to every direction in the
other space. Thus, for instance:

\begin{itemize}
\item For a curve in $\R^2$, the tangent space and normal space are
  both one-dimensional.
\item For a curve in $\R^3$, the tangent space is one-dimensional and
  the normal space is two-dimensional.
\item For a surface in $\R^3$, the tangent space is two-dimensional
  and the normal space is one-dimensional.
\end{itemize}

Note that for geometric subsets, we can only make sense of tangent and
normal spaces, rather than specific tangent and normal
vectors. However, given a {\em parametric} description of a curve, we
can make sense of the tangent {\em vector}, with the length and
direction of the vector determined by the speed and direction of
motion along the curve as per the parameterization.

\subsection{Non-existent tangent planes}

There are many possible reasons why the tangent plane to a surface at
a point on the surface may not exist. First, the surface may be broken
at that point, i.e., it may not locally look like a plane in a small
neighborhood of the point. In such a case, it might be an abuse of
language to call it a surface.

Second, the surface may have a lot of variation around the point --
too many hills and valleys to make sense of a meaningful tangent
plane. This is a surface analogue of the function $x\sin(1/x)$ which,
if extended to the value $0$ at $0$, becomes continuous but not
differentiable at $0$.

Third, the surface may be sharp and pointy at the point. Think, for
instance, of the curved surface of a right circular cone. At most
points on this curved surface, the tangent plane exists. However, at
the vertex of the point, the tangent plane does not exist. This is a
surface analogue of a function that has one-sided derivatives that are
not equal, i.e., it takes a sharp turn.

\subsection{How should the tangent plane relate geometrically to the surface?}

In the one-dimensional situation, we recall that {\em generally
speaking}, the tangent line {\em locally} lies entirely to one side of
the curve, i.e., the curve rests against the tangent line. In fact, it
is the only line that does not {\em cut through} the curve.

But this is not always true. Two notable kinds of exceptions are:

\begin{itemize}
\item {\em Points of inflection}: Here, the tangent line {\em cuts
  through} the curve, i.e., the curve and the tangent line cross each
  other. An example is $y = x^3$ at $(0,0)$. The tangent line is the
  $x$-axis, and it crosses the curve.
\item Points where the curve keeps crossing above and below the
  tangent line arbitrarily close to the point of tangency: For this to
  occur, the second derivative must change sign infinitely often close
  to the point. Examples include functions such as $x^2\sin(1/x)$
  (with the value $0$ at $0$) about the origin $(0,0)$. The tangent
  line is the $x$-axis, and it intersects the curve at points
  arbitrarily close to $(0,0)$, hence fails to be {\em on one side} of
  the curve.
\end{itemize}

Does something similar happen for planes? Yes. {\em Generically}, we
expect that {\em locally}, the tangent plane lies to one side of the
surface, and this can be a reasonable characterization of the tangent
plane. That's the reason why for a sphere ``resting'' on a floor, the
floor is the tangent plane to the sphere at the point of contact.

However, there is an analogue of point of inflection, where the
tangent plane {\em cuts through} the surface at the point. This type
of two-dimensional analogue of point of inflection is termed a {\em
  saddle point}. We will deal with saddle points when we cover the
topic of maxima and minima for functions of many variables later in
the course.
\subsection{Relation between tangents for curve on surface}

If a curve is in/on a surface in $\R^3$, then the tangent {\em line}
to the curve at a point on the curve lies in the tangent {\em plane}
to the surface at the point. Similarly, the normal plane to the curve
at a point contains the normal line to the surface at that point.

In particular, if we have two different curves on a surface
intersecting at a point, and the tangent lines to these curves at the
point do not coincide, then these lines together determine the tangent
plane to the surface, if it exists: it is the unique plane containing
both the tangent lines.

If, on the other hand, we find a situation where there are three
curves intersecting at a point, all in a surface, and the tangent
lines at the point to these three curves do not lie in the same plane,
then the tangent plane at the point does not exist.

\subsection{Partial derivatives of functions of many variables}

Consider a function $z = f(x,y)$, a function of two variables. The
graph of this function is a surface in $\R^3$. Recall that the partial
derivatives can be interpreted as slopes of tangent lines as follows:

\begin{itemize}
\item $f_x(x_0,y_0)$: Consider the plane $y = y_0$. This is a plane
  parallel to the $xz$-plane, and the intersection of the surface with
  this plane can be thought of as the graph of the function $z =
  f(x,y_0)$ of one variable. The partial derivative is the slope of
  the tangent line in this plane to the graph at the point
  $(x_0,y_0,f(x_0,y_0))$. A free vector along this tangent line,
  viewed in $\R^3$, is $\langle 1,0,f_x(x_0,y_0) \rangle$.
\item $f_y(x_0,y_0)$: Here, we fix the plane $x = x_0$, parallel to
  the $yz$-plane, and a similar interpretation follows. A free vector
  along this tangent line, viewed in $\R^3$, is $\langle
  0,1,f_y(x_0,y_0)\rangle$.
\end{itemize}

Thus, computing the partial derivatives allows us to compute tangent
vectors to two curves in the surface that's the graph of this
function. Since we know the basepoint, we can also compute parametric
descriptions of the corresponding tangent lines.

If both the partial derivative $f_x$ and $f_y$ are continuous, then we
can make sense of the notion of tangent plane, to which we now turn.

These two vectors are both parallel to the tangent plane, so we can
take their cross product and find a normal vector. A quick computation
of the cross product shows that $\langle f_x(x_0,y_0), f_y(x_0,y_0),
-1 \rangle$ is a normal vector. Thus, the tangent plane passes through
the point $(x_0,y_0,f(x_0,y_0))$ and has normal vector $\langle
f_x(x_0,y_0),f_y(x_0,y_0),-1 \rangle$. Working out the scalar equation
from the vector equation, we get:

$$(x - x_0)f_x(x_0,y_0) + (y - y_0)f_y(x_0,y_0) + (z - f(x_0,y_0))(-1) = 0$$

Rearranging, we get:

$$z - f(x_0,y_0) = f_x(x_0,y_0)(x - x_0) + f_y(x_0,y_0)(y - y_0)$$

We can rewrite this as:

$$z = f(x_0,y_0) + f_x(x_0,y_0)(x - x_0) + f_y(x_0,y_0)(y - y_0)$$

This is the equation of the tangent plane, and is the most convenient
form for applications.

\subsection{Tangent plane as good linear approximation}

Just as the tangent line is a good linear approximation to the graph
of a function in one variable, the tangent plane is a good linear
approximation to the graph of a function in two variables. Roughly, it
can be thought of as a first-order approximation, so that any ``error
term'' will be zeroey of order two or higher. In particular, for
points close to the point at which we are computing the tangent plane,
the function value arising from the linear approximation is pretty
close to the actual function value.

More concretely, with the above setup, we have the plane:

$$z = f(x_0,y_0) + f_x(x_0,y_0)(x - x_0) + f_y(x_0,y_0)(y - y_0)$$

The corresponding linear function is the right side of this equation:

$$L(x,y) =  f(x_0,y_0) + f_x(x_0,y_0)(x - x_0) + f_y(x_0,y_0)(y - y_0)$$

\subsection{Continuity of partial derivatives necessary}

Recall the earlier example we studied of a function that is separately
continuous in each variable but is not jointly continuous. The
function is $f(x,y) = (xy)/(x^2 + y^2)$ except at the origin, and
$f(0,0) = 0$. This can also be described as $(1/2)\sin(2\theta)$ with
respect to polar coordinates. At the origin, the function is
separately continuous in each variable, but not jointly continuous.

We can further note that in fact, since the function is a constant
zero function along both the axes, all its partial derivatives exist
and equal zero at the origin. However, the function is not jointly
continuous, and hence, we should not expect it to have anything like a
tangent plane. In fact, it does not. Although we can blindly apply the
above formula to obtain the equation of a plane, this is not a
``tangent plane'' to the surface that is the graph of the
function. (See Page 930 of the book).

\subsection{Tangent plane and total derivative}

Recall an earlier conundrum:

Separate continuity:Joint continuity::Partial derivatives:?

In other words, what is the ``joint'' equivalent of partial
derivatives? Unfortunately, the proper way of thinking of this joint
equivalent requires the use of linear algebra, and we will therefore
not be able to cover it. I will simply state the following result:

\begin{quote}
  If all the first-order partial derivatives of a function exist
  around a point, and each first-order partial derivative is jointly
  continuous at the point, then the function is differentiable at the
  point. However, the converse is not true.
\end{quote}

In other words:

Partials exist around the point, are continuous $\implies$ Function is
differentiable (total derivative exists) $\implies$ Partials exist
{\em at} the point

\end{document}