\documentclass[10pt]{amsart}

%Packages in use
\usepackage{fullpage, hyperref, vipul, enumerate}

%Title details
\title{Class quiz: February 17: Limit, order of zero, LH rule}
\author{Math 153, Section 55 (Vipul Naik)}
%List of new commands

\begin{document}
\maketitle

Your name (print clearly in capital letters): $\underline{\qquad\qquad\qquad\qquad\qquad\qquad\qquad\qquad\qquad\qquad}$

\begin{enumerate}

\item Suppose $g:\R \to \R$ is a continuous function such that
  $\lim_{x \to 0} g(x)/x = A$ for some constant $A \ne 0$. What is
  $\lim_{x \to 0} g(g(x))/x$? 

  \begin{enumerate}[(A)]
  \item $0$
  \item $A$
  \item $A^2$
  \item $g(A)$
  \item $g(A)/A$
  \end{enumerate}

  \vspace{0.1in}
  Your answer: $\underline{\qquad\qquad\qquad\qquad\qquad\qquad\qquad}$
  \vspace{0.15in}

\item Suppose $g:\R \to \R$ is a continuous function such that
  $\lim_{x \to 0} g(x)/x^2 = A$ for some constant $A \ne 0$. What is
  $\lim_{x \to 0} g(g(x))/x^4$? 

  \begin{enumerate}[(A)]
  \item $A$
  \item $A^2$
  \item $A^3$
  \item $A^2g(A)$
  \item $g(A)/A^2$
  \end{enumerate}

  \vspace{0.1in}
  Your answer: $\underline{\qquad\qquad\qquad\qquad\qquad\qquad\qquad}$
  \vspace{0.15in}

  For the remaining questions, keep in mind that the {\em order of a
  zero} for a function $f$ at a point $c$ in its domain (where it's
  continuous) such that $f(c) = 0$ is defined as the lub of the set $\{
  \beta \ge 0 \mid \lim_{x \to c} |f(x)|/|x - c|^\beta = 0 \}$.

  If $f$ is an infinitely differentiable function at $c$, then the
  order, if finite, must be a positive integer. If the order is a
  positive integer $r$, then the first $r - 1$ derivatives of $f$ at
  $c$ equal zero and the $r^{th}$ derivative at $c$ is nonzero
  (assuming $f$ to be infinitely differentiable).

  For convenience, we take $c = 0$ in the next three questions, i.e.,
  all limits are being taken as $x \to 0$.

\item If $f$ has a zero of order $2$ and $g$ has a zero of order $3$
  at the point $0$, what is the order of zero for the pointwise sum $f
  + g$ at zero?  (Example: $f(x) = \sin^2x$ and $g(x) = \ln(1 +
  x^3)$).

  \begin{enumerate}[(A)]
  \item $1$
  \item $2$
  \item $3$
  \item $5$
  \item $6$
  \end{enumerate}

  \vspace{0.1in}
  Your answer: $\underline{\qquad\qquad\qquad\qquad\qquad\qquad\qquad}$
  \vspace{0.15in}

\item If $f$ has a zero of order $2$ and $g$ has a zero of order $3$
  at the point $0$, what is the order of zero for the pointwise
  product $fg$ at zero?  (Example: $f(x) = \sin^2x$ and $g(x) = \ln(1
  + x^3)$).

  \begin{enumerate}[(A)]
  \item $1$
  \item $2$
  \item $3$
  \item $5$
  \item $6$
  \end{enumerate}

  \vspace{0.1in}
  Your answer: $\underline{\qquad\qquad\qquad\qquad\qquad\qquad\qquad}$
  \vspace{0.15in}

\item If $f$ has a zero of order $2$ and $g$ has a zero of order $3$
  at the point $0$, what is the order of zero for the composite
  function $f \circ g$ at zero?  (Example: $f(x) = \sin^2x$ and $g(x) =
  \ln(1 + x^3)$).

  \begin{enumerate}[(A)]
  \item $1$
  \item $2$
  \item $3$
  \item $5$
  \item $6$
  \end{enumerate}

  \vspace{0.1in}
  Your answer: $\underline{\qquad\qquad\qquad\qquad\qquad\qquad\qquad}$
  \vspace{0.15in}

\item The L'Hopital rule can be related with order of zero in the
  following manner: Every time the rule is applied to a $(\to 0)/(\to
  0)$ form, the order of zero of the numerator and denominator go {\em
    down by one}. Repeated application hopefully yields a situation
  where either the numerator or the denominator has a nonzero value.

  Assume that we start with a limit $\lim_{x \to c} f(x)/g(x)$ where
  both $f$ and $g$ are infinitely differentiable at $c$, and further,
  that $f(c) = g(c) = 0$. If the order of zero of $f$ is $d_f$ and the
  order of zero of $g$ is $d_g$, which of the
  following is true?

  \begin{enumerate}[(A)]
  \item If $d_f = d_g$, then we need to apply the LH rule $d_f$ times
    and we will then get a nonzero numerator and nonzero denominator,
    that we can evaluate to get the limit. If $d_f < d_g$, then we
    apply the LH rule $d_f$ times to get a nonzero numerator and zero
    denominator, so the limit is undefined. If $d_g < d_f$, then we
    apply the LH rule $d_g$ times to get a zero numerator and nonzero
    denominator, so the limit is zero.
  \item If $d_f = d_g$, then we need to apply the LH rule $d_f$ times
    and we will then get a nonzero numerator and nonzero denominator,
    that we can evaluate to get the limit. If $d_f < d_g$, then we
    apply the LH rule $d_f$ times to get a zero numerator and nonzero
    denominator, so the limit is undefined. If $d_g < d_f$, then we
    apply the LH rule $d_g$ times to get a nonzero numerator and zero
    denominator, so the limit is zero.
  \item If $d_f = d_g$, then we need to apply the LH rule $d_f$ times
    and we will then get a nonzero numerator and nonzero denominator,
    that we can evaluate to get the limit. If $d_f < d_g$, then we
    apply the LH rule $d_f$ times to get a nonzero numerator and zero
    denominator, so the limit is zero. If $d_g < d_f$, then we
    apply the LH rule $d_g$ times to get a zero numerator and nonzero
    denominator, so the limit is undefined.
  \item If $d_f = d_g$, then we need to apply the LH rule $d_f$ times
    and we will then get a nonzero numerator and nonzero denominator,
    that we can evaluate to get the limit. If $d_f < d_g$, then we
    apply the LH rule $d_f$ times to get a zero numerator and nonzero
    denominator, so the limit is zero. If $d_g < d_f$, then we
    apply the LH rule $d_g$ times to get a nonzero numerator and zero
    denominator, so the limit is undefined.
  \item In all cases, we perform the LH rule $\min \{ d_f, d_g \}$
    times and obtain a nonzero numerator and nonzero denominator.
  \end{enumerate}

  \vspace{0.1in}
  Your answer: $\underline{\qquad\qquad\qquad\qquad\qquad\qquad\qquad}$
  \vspace{0.15in}

\end{enumerate}
\end{document}