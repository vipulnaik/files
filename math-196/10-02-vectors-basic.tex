\documentclass[10pt]{amsart}

%Packages in use
\usepackage{fullpage, hyperref, vipul, enumerate}

%Title details
\title{Diagnostic in-class quiz: due Wednesday October 2: Vectors (basic stuff)}
\author{Math 196, Section 57 (Vipul Naik)}
%List of new commands

\begin{document}
\maketitle

Your name (print clearly in capital letters): $\underline{\qquad\qquad\qquad\qquad\qquad\qquad\qquad\qquad\qquad\qquad}$

{\bf PLEASE DO {\em NOT} DISCUSS ANY QUESTIONS.}

Many of you are familiar with vectors, either from Math 195 or some
exposure to vectors in high school (or perhaps both).  This quiz is to
help gauge your level of understanding coming in. We will not get to
start using the ideas in their full depth until a few weeks later.

For the benefit of those who haven't seen vectors at all, the
definitions are briefly provided.

There are many ways of describing the vector in $\R^n$ with
coordinates $a_1$, $a_2$, $\dots$, $a_n$. You may have seen the vector
described using angled braces as $\langle a_1,a_2,\dots,a_n
\rangle$. In this linear algebra course, we will customarily write the
vector as a {\em column} vector, i.e., the coordinates will be written
in a vertical column. For instance, the vector $\langle 2,3,7 \rangle$
will be written as the column vector $\left[\begin{matrix} 2 \\ 3 \\ 7
    \\\end{matrix}\right]$.

Two vectors in $\R^n$ can be added with each other (note that both
vectors need to be in the {\em same} $\R^n$ in order to be added). The
addition is coordinate-wise:

$$\left[\begin{matrix} v_1 \\ v_2 \\ \cdot \\ \cdot \\ \cdot \\ v_n \\\end{matrix}\right] + \left[\begin{matrix} w_1 \\ w_2 \\ \cdot \\ \cdot \\ \cdot \\ w_n \\\end{matrix}\right] = \left[\begin{matrix} v_1 + w_1 \\ v_2 + w_2 \\ \cdot \\ \cdot \\ \cdot \\ v_n + w_n \\\end{matrix}\right]$$

Also, given any real number $\lambda$ (called a {\em scalar} to
distinguish from a vector) and a vector $\vec{v} = \left[\begin{matrix} v_1 \\ v_2 \\ \cdot \\ \cdot \\ \cdot \\ v_n \\\end{matrix}\right]$, we can define:

$$\lambda \vec{v} = \lambda\left[\begin{matrix} v_1 \\ v_2 \\ \cdot \\ \cdot \\ \cdot \\ v_n \\\end{matrix}\right] := \left[\begin{matrix} \lambda v_1 \\ \lambda v_2 \\ \cdot \\ \cdot \\ \cdot \\ \lambda v_n \\\end{matrix}\right]$$

We can identify the set of $n$-dimensional vectors with the set of
points in $\R^n$. The vector $\vec{v} = \left[\begin{matrix}
    v_1 \\ v_2 \\ \cdot \\ \cdot \\ \cdot \\ v_n
    \\\end{matrix}\right]$ in this case corresponds to the point with
coordinates $(v_1,v_2,\dots,v_n)$.

\begin{enumerate}
\item {\em Do not discuss this!}: For a $n$-dimensional vector
  $\vec{v}$, the {\em set of scalar multiples} of $\vec{v}$ is the set
  of vectors that can be expressed in the form $\lambda \vec{v}$,
  $\lambda \in \R$. Assume that $\vec{v}$ is a nonzero vector. What
  can we say geometrically about the set of points in $\R^n$ that
  correspond to the scalar multiples of $\vec{v}$?

  \begin{enumerate}[(A)]
  \item It is a straight line in $\R^n$ that passes through the
    origin.
  \item It is a straight line in $\R^n$. However, it need not pass
    through the origin.
  \item It is a straight half-line in $\R^n$ with the endpoint at the
    origin.
  \item It is a straight half-line in $\R^n$, but the endpoint need
    not be at the origin.
  \item It is a line segment in $\R^n$.
  \end{enumerate}

  \vspace{0.1in}
  Your answer: $\underline{\qquad\qquad\qquad\qquad\qquad\qquad\qquad}$
  \vspace{0.1in}

\item {\em Do not discuss this!}: Given two $n$-dimensional vectors
  $\vec{v}$ and $\vec{w}$, the {\em set of linear combinations} of
  $\vec{v}$ and $\vec{w}$ is the set of all vectors that can be
  written in the form $\lambda \vec{v} + \mu \vec{w}$ where
  $\lambda,\mu \in \R$ (note that $\lambda$ and $\mu$ can take
  arbitrary real values, and are allowed to be equal to each
  other). In other words, you can take scalar multiples, and you can
  then add these scalar multiples.

  The set of linear combinations of $\vec{v}$ and $\vec{w}$ is
  sometimes also called the {\em span} of $\vec{v}$ and $\vec{w}$.

  What is the span of the vectors $\left[\begin{matrix} 1 \\ 0
      \\\end{matrix}\right]$ and $\left[\begin{matrix} 0 \\ 1
      \\\end{matrix}\right]$ in $\R^2$?

  \begin{enumerate}[(A)]
  \item The zero vector only, because that is the only vector that can
    be expressed both as a multiple of $\left[\begin{matrix} 1 \\ 0
        \\\end{matrix}\right]$ and as a multiple of
    $\left[\begin{matrix} 0 \\ 1 \\\end{matrix}\right]$.
  \item The set of vectors that can be expressed as a scalar multiple
    either of $\left[\begin{matrix} 1 \\ 0 \\\end{matrix}\right]$ or
    of $\left[\begin{matrix} 0 \\ 1 \\\end{matrix}\right]$.
  \item The set of vectors that can be expressed as a scalar multiple
    of at least one of these three vectors: $\left[\begin{matrix} 1
        \\ 0 \\\end{matrix}\right]$, $\left[\begin{matrix} 0 \\ 1
        \\\end{matrix}\right]$, and $\left[\begin{matrix} 1 \\ 1
        \\\end{matrix}\right]$.
  \item All vectors in the first quadrant of $\R^2$, including the
    bounding half-lines. In other words, the set of vectors of the
    form $\left[\begin{matrix} x \\ y \\\end{matrix}\right]$ where $x
    \ge 0$ and $y \ge 0$.
  \item All vectors in $\R^2$.
  \end{enumerate}

  \vspace{0.1in}
  Your answer: $\underline{\qquad\qquad\qquad\qquad\qquad\qquad\qquad}$
  \vspace{0.1in}

\item {\em Do not discuss this!}: Consider the transformation from
  $\R^2$ to $\R^2$ that interchanges the coordinates of a
  vector. Explicitly, the transformation is given as:

  $$\left[\begin{matrix} x \\ y \\\end{matrix}\right] \mapsto \left[\begin{matrix} y \\ x \\\end{matrix}\right]$$

  Which of the following describes the transformation geometrically,
  with $\R^2$ viewed as the $xy$-plane?
  \begin{enumerate}[(A)]
  \item It is a reflection about the $x$-axis in $\R^2$, i.e., the
    axis for the first coordinate.
  \item It is a reflection about the $y$-axis in $\R^2$, i.e., the
    axis for the second coordinate.
  \item It is a reflection about the line $y = x$ in $\R^2$, i.e., the
    line of vectors where both coordinates are equal.
  \item It is a reflection about the line $y = -x$ in $\R^2$, i.e.,
    the line of vectors where the coordinates are negatives of each
    other.
  \end{enumerate}

  \vspace{0.1in}
  Your answer: $\underline{\qquad\qquad\qquad\qquad\qquad\qquad\qquad}$
  \vspace{0.1in}

\end{enumerate}
\end{document}
