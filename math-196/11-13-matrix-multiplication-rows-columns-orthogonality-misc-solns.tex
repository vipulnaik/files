\documentclass[10pt]{amsart}

%Packages in use
\usepackage{fullpage, hyperref, vipul, enumerate}

%Title details
\title{Take-home class quiz solutions: due Wednesday November 13: Matrix multiplication: rows, columns, orthogonality, and other miscellanea}
\author{Math 196, Section 57 (Vipul Naik)}
%List of new commands

\begin{document}
\maketitle

\section{Performance review}

25 people took this 13-question quiz. The score distribution was as follows:

\begin{itemize}
\item Score of 4: 2 people
\item Score of 5: 1 person
\item Score of 6: 1 person
\item Score of 7: 3 people
\item Score of 8: 3 people
\item Score of 9: 5 people
\item Score of 10: 4 people
\item Score of 11: 4 people
\item Score of 12: 2 people
\end{itemize}

The mean score was 8.68.

The question-wise answers and performance review are as follows:

\begin{enumerate}
\item Option (D): 24 people
\item Option (C): 22 people
\item Option (D): 22 people
\item Option (E): 19 people
\item Option (A): 17 people
\item Option (B): 23 people
\item Option (E): 14 people
\item Option (E): 20 people
\item Option (C): 7 people
\item Option (A): 12 people
\item Option (B): 7 people
\item Option (E): 14 people
\item Option (B): 16 people
\end{enumerate}

\section{Solutions}

{\bf PLEASE FEEL FREE TO DISCUSS {\em ALL} QUESTIONS.}

The purpose of this quiz is two-fold. First, many of the ideas related
to matrix multiplication are at the stage where a bit of review will
help prevent their fading out. Drawing from the best research on {\em
  spaced repetition} (see for instance
\url{http://en.wikipedia.org/wiki/Spaced_repetition}) we will try to
recall some of the stuff. But with a twist, because we consider it
from a somewhat different angle.

Second, the new angle will also turn out to be useful for later
material.

For Questions 1-5: Given a $n$-dimensional vector $\langle
a_1,a_2,\dots,a_n \rangle \in \R^n$, the vector can be interpreted as
a $n \times 1$ matrix (a column vector). This is the default
interpretation. But there are also two other interpretations: as a $1
\times n$ matrix (a row vector) and as a diagonal $n \times n$ matrix.

Also note that for Questions 1-5, all the three ways of representing
vectors coincide with each other for $n = 1$, so the questions are
uninteresting for $n = 1$ because all answer options are
equivalent. You may therefore assume that $n > 1$ for these questions,
though obviously the correct answers are correct for $n = 1$ as well.

\begin{enumerate}
\item Suppose I want to add two vectors $\vec{a} = \langle a_1, a_2,\dots,a_n
  \rangle$ and $\vec{b} = \langle b_1,b_2,\dots,b_n \rangle$ to obtain the
  output vector $\langle a_1 + b_1, a_2 + b_2, \dots, a_n + b_n
  \rangle$ using matrix addition. What format (row vector, column
  vector, or diagonal matrix) should I use? Please see Option (D)
  before answering and select the option that best describes your view.

  \begin{enumerate}[(A)]
  \item Represent both $\vec{a}$ and $\vec{b}$ as row vectors and
    interpret the sum as a row vector.
  \item Represent both $\vec{a}$ and $\vec{b}$ as column vectors and
    interpret the sum as a column vector.
  \item Represent both $\vec{a}$ and $\vec{b}$ as diagonal matrices and
    interpret the sum as a diagonal matrix.
  \item We can use any of the above.
  \end{enumerate}

  {\em Answer}: Option (D)

  {\em Explanation}: Regardless of whether we represent the vectors as
  row vectors, column vectors, or diagonal matrices, matrix addition
  is executed coordinate-wise, as desired.

  {\em Performance review}: 24 out of 25 people got this. 1 chose (B).

\item Suppose I want to perform coordinate-wise multiplication on two
  vectors. Explicitly, I have two vectors $\vec{a} = \langle a_1,
  a_2,\dots,a_n \rangle$ and $\vec{b} = \langle b_1,b_2,\dots,b_n
  \rangle$ and I want to obtain the output vector $\langle a_1b_1,
  a_2b_2, \dots, a_nb_n \rangle$ using matrix multiplication (with the
  matrix for $\vec{a}$ written on the left and the matrix for
  $\vec{b}$ written on the right). What format (row vector, column
  vector, or diagonal matrix) should I use?  Please see Option (D)
  before answering and select the option that best describes your
  view.

  \begin{enumerate}[(A)]
  \item Represent both $\vec{a}$ and $\vec{b}$ as row vectors and
    interpret the matrix product as a row vector.
  \item Represent both $\vec{a}$ and $\vec{b}$ as column vectors and
    interpret the matrix product as a column vector.
  \item Represent both $\vec{a}$ and $\vec{b}$ as diagonal matrices and
    interpret the matrix product as a diagonal matrix.
  \item We can use any of the above.
  \end{enumerate}

  {\em Answer}: Option (C)

  {\em Explanation}: The multiplication of diagonal matrices is
  coordinate-wise along the diagonal. For instance, when $n = 2$, we
  get:

  $$\left[\begin{matrix} a_1 & 0 \\ 0 & a_2 \\\end{matrix}\right] \left[\begin{matrix} b_1 & 0 \\ 0 & b_2 \\\end{matrix}\right] = \left[\begin{matrix} a_1b_1 & 0 \\ 0 & a_2b_2 \\\end{matrix}\right]$$

  In general:

  $$\left[\begin{matrix} a_1 & 0 & \dots & 0 \\ 0 & a_2 & \dots & 0 \\ \cdot & \cdot & \cdot & \cdot \\ 0 & \dots & 0 & a_n \\\end{matrix}\right]\left[\begin{matrix} b_1 & 0 & \dots & 0 \\ 0 & b_2 & \dots & 0 \\ \cdot & \cdot & \cdot & \cdot \\ 0 & \dots & 0 & b_n \\\end{matrix}\right] = \left[\begin{matrix} a_1b_1 & 0 & \dots & 0 \\ 0 & a_2b_2 & \dots & 0 \\ \cdot & \cdot & \cdot & \cdot \\ 0 & \dots & 0 & a_nb_n \\\end{matrix}\right]$$

  Note that for Options (A) and (B), it does not even make sense to
  try computing the product.

  {\em Option (A)}: Here, $\vec{a}$ and $\vec{b}$ are both represented
  by $n$-dimensional row vectors, i.e., $1 \times n$ matrices, so we cannot
  multiply them because the number of columns of the first matrix does
  not equal the number of rows of the second matrix.

  {\em Option (B)}: Here, $\vec{a}$ and $\vec{b}$ are both represented
  by $n$-dimensional column vectors, i.e., $n \times 1$ matrices, so
  we cannot multiply them because the number of columns of the first
  matrix does not equal the number of rows of the second matrix.

  {\em Performance review}: 22 out of 25 got this. 3 chose (D).

\item Suppose I am given two vectors $\vec{a} = \langle a_1,
  a_2,\dots,a_n \rangle$ and $\vec{b} = \langle b_1,b_2,\dots,b_n
  \rangle$ and I want to obtain a $1 \times 1$ matrix with entry
  $\sum_{i=1}^n a_ib_i$ using matrix multiplication (with the matrix
  for $\vec{a}$ written on the left and the matrix for $\vec{b}$
  written on the right). What format (row vector, column vector, or
  diagonal matrix) should I use?
  \begin{enumerate}[(A)]
  \item Represent both $\vec{a}$ and $\vec{b}$ as row vectors.
  \item Represent both $\vec{a}$ and $\vec{b}$ as column vectors.
  \item Represent both $\vec{a}$ and $\vec{b}$ as diagonal matrices.
  \item Represent $\vec{a}$ as a row vector and $\vec{b}$ as a column
    vector.
  \item Represent $\vec{a}$ as a column vector and $\vec{b}$ as a row vector.
  \end{enumerate}

  {\em Answer}: Option (D)

  {\em Explanation}: The dimensions match: $\vec{a}$ is represented by
  a $1 \times n$ matrix and $\vec{b}$ is represented by a $n \times 1$
  matrix. The product definition also matches. Note that this
  particular form of product is the {\em dot product} of the vectors.

  Note that Option (C) would give a diagonal matrix with the products
  $a_ib_i$ along the diagonal, but would not add them up. Options (A)
  and (B) do not make sense, for the same reason as discussed in the
  answer to Question 2. Option (E) would give a product that is a $n
  \times n$ matrix whose $(ij)^{th}$ entry is the product
  $a_ib_j$. The matrix described by Option (E) is termed the {\em
    Hadamard product}.

  {\em Performance review}: 22 out of 25 got this.  2 chose (E), 1
  left the question blank.
\item Suppose I am given three vectors $\vec{a} = \langle
  a_1,a_2,\dots,a_n \rangle$, $\vec{b} = \langle b_1,b_2,\dots,b_n
  \rangle$, and $\vec{c} = \langle c_1,c_2,\dots,c_n \rangle$. I want
  to obtain a $1 \times 1$ matrix with entry $\sum_{i=1}^n
  (a_ib_ic_i)$ using matrix multiplication (with the matrix for
  $\vec{a}$ written on the left, the matrix for $\vec{b}$ written in
  the middle, and the matrix for $\vec{c}$ written on the right). What
  format should I use?

  \begin{enumerate}[(A)]
  \item $\vec{a}$ as a row vector, $\vec{b}$ as a column vector,
    $\vec{c}$ as a diagonal matrix.
  \item $\vec{a}$ as a column vector, $\vec{b}$ as a row vector,
    $\vec{c}$ as a diagonal matrix.
  \item $\vec{a}$ as a diagonal matrix, $\vec{b}$ as a row vector,
    $\vec{c}$ as a column vector.
  \item $\vec{a}$ as a column vector, $\vec{b}$ as a diagonal matrix,
    $\vec{c}$ as a row vector.
  \item $\vec{a}$ as a row vector, $\vec{b}$ as a diagonal matrix,
    $\vec{c}$ as a column vector.
  \end{enumerate}

  {\em Answer}: Option (E)

  {\em Explanation}: First, note that the dimensions match. $\vec{a}$
  is represented by a $1 \times n$ matrix, $\vec{b}$ by a diagonal $n
  \times n$ matrix, and $\vec{c}$ by a $n \times 1$ matrix. The
  multiplication makes sense, and the product is a $1 \times 1$
  matrix.
 
  Second, note that the matrix we get is the correct one. The product
  of the diagonal matrix for $\vec{b}$ and the column vector for
  $\vec{c}$ gives a column vector whose $i^{th}$ coordinate is
  $b_ic_i$. The product of the row vector for $\vec{a}$ with this
  column vector gives $\sum_{i=1}^n (a_ib_ic_i)$ as desired.

  Here is how it looks in the case $n = 2$:

  $$\left[\begin{matrix} a_1 & a_2 \\\end{matrix}\right] \left[\begin{matrix} b_1 & 0 \\ 0 & b_2 \\\end{matrix}\right] \left[\begin{matrix} c_1 \\ c_2 \\\end{matrix}\right]$$

  If we begin our simplification by multiplying the second and third matrix, we obtain:

  $$\left[\begin{matrix} a_1 & a_2 \\\end{matrix}\right] \left[\begin{matrix} b_1c_1 \\ b_2c_2 \\\end{matrix}\right]$$

  We now do the remaining multiplication and obtain the desired $1 \times 1$ matrix:

  $$\left[\begin{matrix} a_1b_1c_1 + a_2b_2c_2 \\\end{matrix}\right]$$

  Alternatively, we could simplify the original product by multiplying the first two matrices to begin with, and obtain:

  $$\left[\begin{matrix} a_1b_1 & a_2b_2 \\\end{matrix}\right]\left[\begin{matrix} c_1 \\ c_2 \\\end{matrix}\right]$$

  Multiplying them out gives the desired result. {\em Note}: It's not
  surprising that both ways of simplifying the product of three
  matrices give the same result. That follows from associativity of
  matrix multiplication. We did it both ways for a sanity check.

  {\em Performance review}: 19 out of 25 got this. 2 each chose (A)
  and (C). 1 each chose (B) and (D).

  \vspace{0.5in}

  The next few questions rely on the concept of orthogonality ({\em
    orthogonal} is a synonym for {\em perpendicular} or {\em at right
    angles}). We say that two vectors (of the same dimension) are
  orthogonal if their dot product is zero. By this definition, the
  zero vector of a given dimension is orthogonal to every vector of
  that dimension. Note that it does not make sense to talk of
  orthogonality for vectors with different dimensions, i.e., with
  different numbers of coordinates.

\item Suppose $A$ is a $n \times m$ matrix. We can think of solving
  the system $A\vec{x} = \vec{0}$ (where $\vec{x}$ is a $m \times 1$
  column vector of unknowns) as trying to find all the vectors
  orthogonal to all the vectors in a given set of vectors. What set of
  vectors is that?

  \begin{enumerate}[(A)]
  \item The set of row vectors of $A$, i.e., the rows of $A$, viewed
    as $m$-dimensional vectors.
  \item The set of column vectors of $A$, i.e., the columns of $A$,
    viewed as $n$-dimensional vectors.
  \end{enumerate}

  {\em Answer}: Option (A)

  {\em Explanation}: $A\vec{x}$ is a $n \times 1$ matrix (i.e., a
  column vector with $n$ coordinates) and its $i^{th}$ entry is the
  dot product of the $i^{th}$ row of $A$ and the vector $\vec{x}$,
  both of which are $m$-dimensional vectors. This is zero if and only
  if the $i^{th}$ row of $A$ and the vector $\vec{x}$ are orthogonal
  to each other. In order to have $A\vec{x} = \vec{0}$, we need
  $\vec{x}$ to be orthogonal to all the rows of $A$.

  {\em Performance review}: 17 out of 25 got this. 8 chose (B).

\item Suppose $A$ is a $p \times q$ matrix and $B$ is a $q \times r$
  matrix where $p$, $q$, and $r$ are positive integers. The matrix
  product $AB$ is a $p \times r$ matrix. What orthogonality condition
  corresponds to the condition that the matrix product $AB$ is a zero
  matrix (i.e., all its entries are zero)?

  \begin{enumerate}[(A)]
  \item Every row of $A$ is orthogonal to every row of $B$.
  \item Every row of $A$ is orthogonal to every column of $B$.
  \item Every column of $A$ is orthogonal to every row of $B$.
  \item Every column of $A$ is orthogonal to every column of $B$.
  \end{enumerate}

  {\em Answer}: Option (B)

  {\em Explanation}: The $ik^{th}$ entry of $AB$ can be viewed as the
  dot product of the $i^{th}$ row of $A$ and the $k^{th}$ column of
  $B$. In particular, this entry is zero if and only if the $i^{th}$
  row of $A$ is orthogonal to the $k^{th}$ column of $B$. In order for
  $AB$ to be the zero matrix, we need this to hold for {\em every} row
  of $A$ and {\em every} column of $B$, giving the answer option.

  {\em Performance review}: 23 out of 25 got this. 2 chose (C).
\item Suppose $A$ is an invertible $n \times n$ square matrix. Which
  of the following correctly characterizes the $n \times n$ matrix
  $A^{-1}$ using orthogonality? Recall that $AA^{-1}$ and $A^{-1}A$
  are both equal to the $n \times n$ identity matrix.

  \begin{enumerate}[(A)]
  \item For every $i$ in $\{ 1,2,\dots,n\}$, the $i^{th}$ row of $A$ is
    orthogonal to the $i^{th}$ row of $A^{-1}$. The dot product of the
    $i^{th}$ row of $A$ and the $j^{th}$ row of $A^{-1}$ for distinct
    $i,j$ in $\{1,2,\dots,n\}$ equals $1$.
  \item For every $i$ in $\{ 1,2,\dots,n\}$, the $i^{th}$ column of $A$ is
    orthogonal to the $i^{th}$ column of $A^{-1}$. The dot product of the
    $i^{th}$ row of $A$ and the $j^{th}$ column of $A^{-1}$ for distinct
    $i,j$ in $\{1,2,\dots,n\}$ equals $1$.
 \item For every distinct $i,j$ in $\{1,2,\dots,n\}$, the $i^{th}$ row
   of $A$ is orthogonal to the $j^{th}$ row of $A^{-1}$. The dot
   product of the $i^{th}$ row of $A$ with the $i^{th}$ row of
   $A^{-1}$ equals $1$.
   \item For every $i$ in $\{ 1,2,\dots,n\}$, the $i^{th}$ row of $A$ is
    orthogonal to the $i^{th}$ column of $A^{-1}$. The dot product of the
    $i^{th}$ row of $A$ and the $j^{th}$ column of $A^{-1}$ for distinct
    $i,j$ in $\{1,2,\dots,n\}$ equals $1$.
  \item For every distinct $i,j$ in $\{1,2,\dots,n\}$, the $i^{th}$
    row of $A$ is orthogonal to the $j^{th}$ column of $A^{-1}$. The
    dot product of the $i^{th}$ row of $A$ and the $i^{th}$ column of
    $A^{-1}$ equals $1$.
  \end{enumerate}

  {\em Answer}: Option (E)

  {\em Explanation}: The product $AA^{-1}$ is the identity matrix. The
  entries of this matrix can be described as follows:

  \begin{itemize}
  \item For distinct $i,j$ in $\{ 1,2,\dots,n \}$, the $(ij)^{th}$
    entry of $AA^{-1}$ is $0$. This translates to saying that the
    $i^{th}$ row of $A$ is orthogonal to the $j^{th}$ column of
    $A^{-1}$.
  \item For any $i$ in $\{ 1,2,\dots, n\}$, the $(ii)^{th}$ entry of
    $AA^{-1}$ is $1$. This translates to saying that the dot product of
    the $i^{th}$ row of $A$ and the $i^{th}$ column of $A^{-1}$ is
    equal to $1$.
  \end{itemize}

  These correspond to Option (E).

  Note that it is {\em also} true that $A^{-1}A$ is the identity
  matrix. We can use this to obtain an alternative characterization of
  $A^{-1}$. This condition will use the rows of $A^{-1}$ and the
  columns of $A$. Explicitly:

  \begin{itemize}
  \item For distinct $i,j$ in $\{ 1,2,\dots,n \}$, the $(ij)^{th}$
    entry of $A^{-1}A$ is $0$. This translates to saying that the
    $i^{th}$ row of $A^{-1}$ is orthogonal to the $j^{th}$ column of
    $A$.
  \item For any $i$ in $\{ 1,2,\dots, n\}$, the $(ii)^{th}$ entry of
    $A^{-1}A$ is $1$. This translates to saying that the dot product of
    the $i^{th}$ row of $A^{-1}$ and the $i^{th}$ column of $A$ is
    equal to $1$.
  \end{itemize}

  {\em Performance review}: 14 out of 25 got this. 6 chose (C), 3
  chose (D), 1 each chose (A) and (B).

  \vspace{0.3in}

  The remaining questions review your skills at abstract behavior
  prediction.

\item Suppose $n$ is a positive integer greater than $1$. Which of the
  following is always true for two invertible $n \times n$ matrices
  $A$ and $B$?

  \begin{enumerate}[(A)]
  \item $A + B$ is invertible, and $(A + B)^{-1} = A^{-1} + B^{-1}$
  \item $A + B$ is invertible, and $(A + B)^{-1} = B^{-1} + A^{-1}$
  \item $A + B$ is invertible, though neither of the formulas of the
    preceding two options is correct
  \item $AB$ is invertible, and $(AB)^{-1} = A^{-1}B^{-1}$
  \item $AB$ is invertible, and $(AB)^{-1} = B^{-1}A^{-1}$
  \end{enumerate}

  {\em Answer}: Option (E)

  {\em Explanation}: $A$ and $B$ being invertible does not imply that
  $A + B$ is invertible. For instance, $A$ may be the identity matrix
  and $B$ may be its negative.

  On the other hand, $AB$ is invertible and $(AB)^{-1} =
  B^{-1}A^{-1}$. This is because when we invert (i.e., go backward) we
  must do so in reverse.

  {\em Performance review}: 20 out of 25 got this. 3 chose (D), 1 each
  chose (A) and (B).

  {\em Historical note (last time, appeared in a midterm)}: $21$ out of
  $30$ people got this. $5$ chose (C), $4$ chose (D).

\item Suppose $n$ is a positive integer greater than $1$. For a
  nilpotent $n \times n$ matrix $C$, define the {\em nilpotency} of
  $C$ as the smallest positive integer $r$ such that $C^r = 0$. Note
  that the nilpotency is not defined for a non-nilpotent matrix. Given
  two $n \times n$ matrices $A$ and $B$, what is the relation between
  the nilpotencies of $AB$ and $BA$?

  \begin{enumerate}[(A)]
  \item $AB$ is nilpotent if and only if $BA$ is nilpotent, and if
    so, their nilpotencies must be equal.
  \item $AB$ is nilpotent if and only if $BA$ is nilpotent, and if so,
    their nilpotencies must differ by $1$.
  \item $AB$ is nilpotent if and only if $BA$ is nilpotent, and if so,
    their nilpotencies must either be equal or differ by $1$.
  \item It is possible for $AB$ to be nilpotent and $BA$ to be
    non-nilpotent; however, {\em if} both are nilpotent, then their
    nilpotencies must be equal.
  \item It is possible for $AB$ to be nilpotent and $BA$ to be
    non-nilpotent, however {\em if} both are nilpotent, then their
    nilpotencies must differ by $1$.
  \end{enumerate}

  {\em Answer}: Option (C)

  {\em Explanation}: We have seen examples where the nilpotencies are
  not equal. For instance:

  $$A = \left[\begin{matrix} 0 & 1 \\ 0 & 0 \\\end{matrix}\right], B = \left[ \begin{matrix} 1 & 0 \\ 0 & 0 \\\end{matrix}\right]$$

  $AB = 0$ but $BA$ is not zero.

  On the other hand, we also have examples where the nilpotencies are
  equal. For instance:

  $$A = \left[\begin{matrix} 1 & 0 \\ 0 & 0 \\\end{matrix}\right], B = \left[ \begin{matrix} 0 & 0 \\ 0 & 1 \\\end{matrix}\right]$$

  Note, however, that $(AB)^r = 0$ implies $(BA)^{r+1} = 0$ and
  $(BA)^s = 0$ implies $(AB)^{s+1} = 0$. Thus, the nilpotencies can
  differ by at most one, since each nilpotency is bounded by $1$ more
  than the other.
 
  {\em Performance review}: 7 out of 25 got this. 9 chose (D), 4 chose
  (E), 2 chose (A), 1 chose (B), and 2 left the question blank.

  {\em Historical note (last time, appeared in a midterm)}: $5$ out of
  $30$ got this. $12$ chose (E), $11$ chose (D), and $2$ chose (B).

\item What is the smallest $n$ for which there exist examples of
  invertible $n \times n$ matrices $A$ and $B$ such that $A \ne B$ but
  $A^2 = B^2$?

  \begin{enumerate}[(A)]
  \item $1$
  \item $2$
  \item $3$
  \item $4$
  \item This is not possible for any $n$.
  \end{enumerate}

  {\em Answer}: Option (A)

  {\em Explananation}: We can take $A = [-1]$ and $B = [1]$.

  {\em Performance review}: 12 out of 25 got this. 10 chose (B), 2
  chose (E), 1 chose (C).

\item What is the smallest $n$ for which there exist examples of
  invertible $n \times n$ matrices $A$ and $B$ such that $A \ne B$ but
  $A^3 = B^3$?

  \begin{enumerate}[(A)]
  \item $1$
  \item $2$
  \item $3$
  \item $4$
  \item This is not possible for any $n$.
  \end{enumerate}

  {\em Answer}: Option (B)

  {\em Explanation}: Note first that $n = 1$ does not work, because if
  two real numbers have the same cube, they must be equal (this is
  because cubing is a one-one function).

  However, $n = 2$ works. Explicitly, we can take $A$ as a rotation
  matrix by $2\pi/3$ and $B$ as the identity matrix. Both $A^3$ and
  $B^3$ equal the identity matrix. Explicitly:

  $$A = \left[\begin{matrix} -1/2 & -\sqrt{3}/2 \\ \sqrt{3}/2 & -1/2 \\\end{matrix}\right], B = \left[\begin{matrix} 1 & 0 \\ 0 & 1 \\\end{matrix}\right]$$

  If you prefer dealing only with matrices with integer entries, consider the following matrix. This is harder to think of, however:

  $$A = \left[\begin{matrix} -1 & -1 \\ 1 & 0 \\\end{matrix}\right], B = \left[\begin{matrix} 1 & 0 \\0 & 1 \\\end{matrix}\right]$$

  {\em Performance review}: 7 out of 25 got this. 13 chose (C), 4
  chose (E), 1 chose (D).
\item What is the smallest $n$ for which there exist examples of
  invertible $n \times n$ matrices $A$ and $B$ such that $A \ne B$ but
  $A^2 = B^2$ and $A^3 = B^3$?

  \begin{enumerate}[(A)]
  \item $1$
  \item $2$
  \item $3$
  \item $4$
  \item This is not possible for any $n$.
  \end{enumerate}

  {\em Answer}: Option (E)

  {\em Explanation}: Suppose $A^2 = B^2$ with $A$ and $B$ both
  invertible. Then, $A^{-2} = B^{-2}$ as well. We are also given $A^3
  = B^3$. Multiplying the two equations, we get $A = B$. Thus, the
  specification required is not possible.

  {\em Performance review}: 14 out of 25 got this. 7 chose (B), 3
  chose (C), 1 chose (D).
\item What is the smallest $n$ for which there exist examples of (not
  necessarily invertible) $n \times n$ matrices $A$ and $B$ such that
  $A \ne B$ but $A^2 = B^2$ and $A^3 = B^3$?

  \begin{enumerate}[(A)]
  \item $1$
  \item $2$
  \item $3$
  \item $4$
  \item This is not possible for any $n$.
  \end{enumerate}

  {\em Answer}: Option (B)

  {\em Explanation}: Note that the condition $A^3 = B^3$ would imply
  $A = B$ if $n = 1$. Therefore, the smallest possible case is $n =
  2$. We will furnish an example in this case. In our example, we take:

  $$A = \left[\begin{matrix} 0 & 1 \\ 0 & 0 \\\end{matrix} \right], B = \left[\begin{matrix} 0 & 0 \\ 0 & 0 \\\end{matrix}\right]$$

  Note how and why this example works: $A^2 = B^2 = 0$, so all higher
  powers of $A$ and of $B$ are equal to $0$.

  {\em Performance review}: 16 out of 25 got this. 5 chose (E), 3
  chose (C), 1 chose (A).
\end{enumerate}
\end{document}
