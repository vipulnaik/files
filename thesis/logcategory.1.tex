\documentclass[a4paper]{amsart}

%Packages in use
\usepackage{fullpage, hyperref, vipul}


%Title details
\title{Exponentials and the log category}
\author{Vipul Naik}
\thanks{\copyright Vipul Naik, Ph.D. student, University of Chicago}

%List of new commands
\newcommand{\logcategory}[1]{\textsc{Log-Category}\left(#1\right)}
\newcommand{\enrichedlogcategory}[2]{\textsc{Enriched-Log-Category}\left(#1 \, ; \, #2\right)}
\newcommand{\aut}[1]{\text{Aut}\left(#1\right)}
\newcommand{\inn}[1]{\text{Inn}\left(#1\right)}
\makeindex

\begin{document}
\maketitle
%\tableofcontents

\begin{abstract}
  In this write-up, I jot down some ideas I have about a ``log
  category'', a category such that the exponential maps we usually
  encounter in the theory of Lie algebras actually become
  homomorphisms between objects of the category. 
\end{abstract}

\section{Definition of the log category}

\subsection{The objects and morphisms}

Let $G$ be a group. The log category of $G$ is a category whose
objects are groups enriched with a $G$-action and some other
structure, and the morphisms are maps which preserve that structure.

\begin{definer}[Log category]
  Let $G$ be a group. The log category of $G$, which we denote by
  $\logcategory{G}$ is defined as follows. The objects of
  $\logcategory{G}$ comprise the following data:

  \begin{enumerate}

  \item A group $A$

  \item A homomorphism $\phi: G \to \aut{A}$ (viz a $G$-action on $A$)

  \item A reflexive symmetric binary relation $\perp$ on $A$

  \end{enumerate}

  Such that the following are satisfied:

  \begin{enumerate}

  \item $\perp$ is $G$-invariant, viz:

    $$a \perp b \implies \phi(g)a \perp \phi(g)b$$
  \item For any $a$, the set of $b$ such that $a \perp b$ is a
    subgroup of $A$. This set shall be denoted as $a^\perp$.

  \item The set of $a$ such that $a \perp b$ for every $b \in A$, is
    the same as the set of $a$ for which $\phi(g)a = a$ for every $g
    \in G$. This set will be denoted as $A^\perp$.

  \item The kernel of the action of $G$ on $A$ is precisely the center
    $Z(G)$. Thus, the map $\phi:G \to \aut{A}$ factors through the
    quotient map $G \to G/Z(G) = \inn{G}$, and the map from $\inn{G}$
    to $\aut{A}$ is injective.

  \end{enumerate}

  We denote the objects as triples $(A,\phi,\perp)$. I will write $ga$
  for $\phi(g)a$ where it is unambiguous, and will use the same symbol
  $\perp$ for all objects.

  If $A$ and $B$ are two objects of $\logcategory{G}$, a morphism from
  $A$ to $B$ is a set-theoretic map $f: A \to B$ such that:

  \begin{enumerate}

  \item If $a_1 \perp a_2$, then $f(a_1) \perp f(a_2)$ and also:

    $$f(a_1a_2) = f(a_1)f(a_2)$$

    In particular since $\perp$ is reflexive, we see that $f$ gives a
    homomorphism on any cyclic subgroup of $A$.

  \item $f$ commutes with the $G$-action, viz:

    $$f(ga) = gf(a)$$
  \end{enumerate}
\end{definer}

\subsection{Examples of the log category}

Here are some examples of objects in the log category of a group:

\begin{enumerate}

\item Every group is an object in its own log category; the action is
  by conjugation (inner automorphisms) and the relation $\perp$ is
  that of {\em commuting}.

\item Suppose $R$ is a noncommutative ring with unit with the property
  that $R$ is additively generated by its group of units (elements
  with two-sided inverses). Let $G$ be the group of units. Then the
  additive group of $R$ is an object in the log category of $G$. The
  $G$-action on $R$ is by conjugation, and the relation $\perp$ is
  that of commuting multiplicatively.

  The condition of being additively generated by the group of units is
  sufficient to ensure the condition that $R^\perp$ is precisely the
  set of $G$-fixed points, and also to ensure that any element in the
  center of $G$ acts trivially on $R$.

\item Suppose $G$ is a connected Lie group (over $\R$ or $\C$) and
  $\mathfrak{g}$ is its Lie algebra. Then $\g$ is an object
  in the log category of $G$. The action of $G$ on $\g$ is by conjugation
  and the relation $\perp$ is that of the Lie bracket being $0$.

\end{enumerate}

Note that in each case $\perp$ has a somewhat different meaning, and
the advantage of just looking at a symmetric binary relation is that
we can unify all these different meanings into a single context.

We now make the crucial observation which links this to the exponential map:

If $G$ is a (real or complex) Lie group and $\g$ is its Lie algebra,
then the exponential map from $\g$ to $G$ is a morphism in the log
category (where both are given structures as described above).

\subsection{Genuine homomorphisms}

Morphisms in the log category are not genuine group homomorphisms;
however they do have a lot of the structure of group
homomorphisms. Here we record some basic facts about these morphisms:

\begin{itemize}

\item Any morphism in the log category is a homomorphism if restricted
  to a cyclic subgroup. In other words, if one is looking {\em
    locally} at a single element, then the morphism looks like a homomorphism.

\item In particular, any morphism in the log category takes the
  trivial subgroup to the trivial subgroup.

\end{itemize}

There are two other important ways in which morphisms in the log
category give rise to genuine group homomorphisms. For this we
introduce two definitions.

\begin{definer}[$G$-derived subgroup, $G$-Abelianization]
  Given an object $A$ in the log category of $G$, the
  $G$-\definedind{derived subgroup} of $A$ is the kernel of the
  congruence generated by the equivalence relation of being in the
  same $G$-orbit. We shall denote the $G$-derived subgroup as $A'_G$.

  The quotient of $A$ by its $G$-derived subgroup is termed the
  $G$-\definedind{Abelianization}. We shall denote this as $A^{ab}_G$.
\end{definer}

Note that for $A = G$ with the standard structure of action by
conjugation, the $G$-derived subgroup is the same as the usual derived
subgroup, or commutator subgroup. In most of the other cases we
mentioned above, including the Lie algebra case and the case of the
additive group of a ring, the $G$-derived subgroup corresponds to the
usual notion of ``derived ideal''.

A second definition:

\begin{definer}[$G$-center]
  Given an object $A$ in the log category of $G$, the $A$-center of
  $G$, is either of the following equivalent things:

  \begin{itemize}

  \item $A^\perp$, viz the set of $x$ such that $a \perp x$ for all $a
    \in A$.

  \item The set of all $a$ for which $g.a = a$ for all $g \in G$.

  \end{itemize}
  
  The equality of the above two subgroups was one of the axioms for
  defining objects of the log category.
\end{definer}

We now have two genuine group homomorphisms:

\begin{itemize}

\item For $A, B \in \logcategory{G}$, a morphism from $A$ to $B$ in
  the log category of $G$, restricts to a group homomorphism from
  $A^\perp$ to $B^\perp$.
\item For $A,B \in \logcategory{G}$ a morphism from $A$ to $B$ sends
  $A'_G$ inside $B'_G$, and further, it yields a group homomorphism
  from $A^{ab}_G$ to $B^{ab}_G$.

\end{itemize}

\subsection{The matrix ring and general linear group}

The matrix ring $M_n(\R)$ is in the log category of $GL_n(\R)$ by the
action by conjugation; it is a special case of two examples:

\begin{itemize}

\item We can think of $GL_n(\R)$ as the group of units in $M_n(\R)$
  and consider the corresponding action

\item We can view $M_n(\R)$ as the Lie algebra of $GL_n(\R)$ and
  consider the corresponding action

\end{itemize}

With the latter view, we see that the exponential map from $M_n(\R)$
to $GL_n(\R)$ is a morphism in the log category. Let's see how the
general statements made in the last subsection specialize in this
case. Let $G = GL_n(\R)$ and $\g = M_n(\R)$ (one can replace $\R$ by
$\C$ throughout this example):

\begin{itemize}

\item $\g'_G$ is the set of matrices with trace $0$, and the
  Abelianization is the ground field $\R$. The Abelianization map is
  the trace map.

  $G'_G$ is the set of matrices with determinant $1$, and the
  Abelianization is $\R^*$. The Abelianization map is the determinant map.

  The induced map on the Abelianization is simply exponentiation from
  $\R$ to $\R^*$, and the general statement corresponds to the
  well-known fact that the exponential of the trace equals the
  determinant of the exponential.

\item $\g^\perp$ is the set of scalar matrices with real values, and
  $G^\perp = Z(G)$ is the set of scalar matrices with nonzero real
  values.  The induced map from $\g^\perp$ to $G^\perp$ is just the
  usual exponentiation from $\R$ to $\R^*$.

\end{itemize}

\section{Funny things about the log category}

\subsection{Log categories of some groups}

The real ``controlling'' axiom of the log category is the statement
that for $A$ in the log category of $G$, we need $A^\perp$ to be the
set of $G$-fixed points of $A$. With this controlling axiom in mind,
we see the following:

\begin{quote}
  The log category of the trivial group (and more generally, of any
  Abelian group) is simply the category of all groups, with usual
  group homomorphisms. The fact that we get usual group homomorphisms
  follows from the fact that the $G$-center is forced to be the whole
  group.
\end{quote}

I will not be interested in all the objects of the log category, but
rather at those objects which are in some sense close to the group
itself. However it is possible that studying the log category at large
might also be a useful variant of representation theory.

\subsection{Isomorphisms in the log category}

Before proceeding further, I want to note that the log category is not
like most ``algebraic'' categories, in the following senses:

\begin{itemize}

\item One can have bijective maps in the log category that are not
  isomorphisms. In category-theoretic language, the forgetful functor
  to sets is not conservative.

\item One can have isomorphisms between objects of the log category
  where the underlying groups are not isomorphic as abstract
  groups. In fact we shall see that in a number of cases (namely, the
  unipotent cases) the exponential map from a Lie algebra to its Lie group
  is an ``isomorphism'' in the sense of the log category.

\end{itemize}

I'll introduce some terminology here.

\begin{definer}[Terminology for morphisms in the log category]
  \begin{itemize}

  \item A morphism $f:A \to B$ of objects in the log category is
    termed \adefinedproperty{morphism}{faithful} if $a \perp a' \iff
    f(a) \perp f(a')$.

  \item If $A$ is a group with a $G$-action, and $\perp$ and $\perp'$
    are two different reflexive symmetric binary relations turning $A$
    into objects of the log-category of $G$, we say that $\perp$ is
    \definedind{finer} than $\perp'$ if $a \perp b \implies a \perp'
    b$. Another way of putting this is that $\perp$ is finer than
    $\perp'$ iff the identity map from $(A,\perp)$ to $(A,\perp')$ is
    a morphism in the log category.

  \end{itemize}
\end{definer}

\section{Why the log category is important}

\subsection{The goal}

If $G$ is a non-Abelian group, there is absolutely no hope of getting
a surjective group homomorphism from an Abelian group to $G$. This is because
the relation of commuting is preserved on taking homomorphisms.

However, what we would ideally like to do is to think of the group $G$
as being ``covered'' by some Abelian group, in a way that may not
precisely be a group homomorphism, but still has the flavour of a
group homomorphism. This is done very effectively in the theory of Lie
groups: we have the ``exponential'' which maps a Lie algebra to the
Lie group. Even though the exponential is not always surjective, its
image does always generate the Lie group (when the Lie group is
connected).

The log category allows us to ask a related question in a more general
context; we would like maps which behave very much like the
exponential, but may not actually have the topological, analytic or
formal appearance of an exponential.

Further, even for groups where one cannot find any good map from an
{\em Abelian} group, we can still try to ``cover'' our group by some
group which is relatively more well-behaved and tractable.

\subsection{A logarithm}

\begin{definer}[Logarithm for a group]
  Let $G$ be a group. A \definedind{logarithm} for $G$ is an object $A
  \in \logcategory{G}$ along with a morphism $\exp :A \to G$ such that:

  \begin{itemize}

  \item $\exp$ induces a surjective map from $A^\perp$ to $G^\perp = Z(G)$

  \item $\exp$ induces a surjective map from $A^{ab}_G$ to $G^{ab}_G$

  \item The image of $A$ under $\exp$ generates $G$ as a group

  \end{itemize}

  I will use the term \sdefinedproperty{logarithm}{full} when $\exp$
  is actually a surjective map (something which happens in most of the
  finite cases we are dealing).
\end{definer}

{\em I'm not sure how important the first two conditions are}.

For the trivial group, every object of the log category is naturally a
full logarithm.

\begin{definer}[Faithful and reversible logarithms]
  \begin{itemize}

  \item A logarithm $\exp:A \to G$ for a group $G$ is termed
    \adefinedproperty{logarithm}{faithful} if $\exp$ is a faithful
    morphism, if $a \perp b \iff \exp(a) \perp \exp(b)$.

  \item A logarithm $\exp:A \to G$ for a group $G$ is termed
    \adefinedproperty{logarithm}{injective} if $\exp$ is injective.

  \item A logarithm $\exp:A \to G$ for a group $G$ is termed
    \adefinedproperty{logarithm}{reversible} if $\exp$ is an
    isomorphism in the log-category; equivalently, $\exp$ is faithful,
    full, and injective.

  \end{itemize}
\end{definer}

Let's review some cases from the real/complex world:

\begin{itemize}

\item $M_n(\R)$ is a logarithm for $GL_n^+(\R)$, but it is not full,
  for instance, the matrix:

  $$\matrixtwobytwo{-1}{0}{0}{-1}$$

  cannot be written as the exponential of a real matrix.

  It is also not faithful and it is not injective.

\item $M_n(\C)$ is a full logarithm for $GL_n(\C)$. In other words,
  every matrix in $GL_n(\C)$ can be written as the exponential of some
  matrix. However, it is not faithful: there are
  examples of matrices which do not commute, but whose exponentials
  commute. It is also not injective.

\item The strictly upper triangular matrices over $\R$ form a
  reversible logarithm for the multiplicative group of upper
  triangular unipotent matrices. To see this, we observe that there
  are power series formulae for $X$ in terms of $\exp X$ and vice
  versa, and those formulae guarantee both bijectivity and the fact
  that $X$ and $Y$ commute iff $\exp X$ and $\exp Y$.

\item In fact, the above holds if we replace $\R$ by a finite field
  (or by any field?) because the formulae are actual polynomials
  rather than power series (since $X$ and $\exp X - 1$ are both
  nilpotent).
\end{itemize}
\subsection{The topological group setting}

We can define the log category over a topological group, in which case
we require all our objects to be topological groups and all our
morphisms to be continuous maps. Here, we may be in search of a
logarithm that is not merely Abelian, but also simply connected, or
even better contractible. Thus, the search is for a ``contractible
Abelian logarithm'' or something like that.

For instance, if we consider the topological group $\C^*$ it is
Abelian, so it's nice as an abstract group, but it's bad topologically
because it's not simply connected. Its logarithm $\C$, on the other
hand, is a contractible Abelian group, which is as nice as one can
get.

\subsection{The algebraic group setting}

Instead of defining a log category over an abstract group, one could
define the log category over an {\em algebraic} group. In this case,
the only difference would be that every object in the log category
would also have to be an algebraic group over the same field, and the
maps would have to be regular maps as far as the underlying algebraic
variety structure is concerned.

In this case, the appropriate analogue of ``Abelian logarithm'' would
be a ``linear logarithm'', a logarithm which is a vector space over
the underlying field. Of course, when working over fields, Lie
algebras are vector spaces, so they are examples of linear logarithms.

Thus questions of interest would be like: what are the algebraic groups
that possess linear logarithms?

\section{Generation by normal subgroups}

\subsection{The base case}

Here I consider a question motivated by Theorem 2 of Professor
Glauberman's paper:

\begin{quote}
  If $G$ is generated by normal subgroups $N_1, N_2, \ldots, N_r$ and
  each of them possesses a full logarithm, can one use them to obtain a
  logarithm for $G$?
\end{quote}

This is a tricky question in general, and we shall answer it first in
some specific cases.

\begin{theorem}[Group generated by normal subgroups]
  Suppose $G$ is generated by a collection of normal subgroups $N_1,
  N_2, \ldots, N_r$. Then $G$ has a full logarithm, which as an
  abstract group is $N_1 \oplus N_2 \oplus \ldots N_r$. In particular
  when each $N_i$ is Abelian, then $G$ has a full Abelian logarithm.
\end{theorem}

\begin{proof}
  Let $M = N_1 \oplus N_2 \oplus \ldots \oplus N_r$ as a group.  We
  first make $M$ into an object in the log-category of $G$. $G$ acts
  on each $N_i$ by conjugation; hence the $G$-action on $M$ is by the
  coordinate-wise action on each $N_i$.

  Next, we define $\perp$ on $M$ as follows. Two tuples
  $(a_1,a_2,\ldots,a_r)$ and $(b_1,b_2,\ldots,b_r)$ are $\perp$ to
  each other if each $a_i$ commutes with each $b_j$.

  Now $\perp$ is reflexive and symmetric, and for any $a$, $a^\perp$
  is in fact a direct sum of subgroups of each $N_i$ so it is a
  subgroup. Also $\perp$ commutes with the $G$-action.

  Further $M^\perp$ is the direct sum of $N_i \cap Z(G)$ for
  each $i$, which is precisely the same as the set of fixed points of
  the $G$-action. 

  Finally, note that any element of $Z(G)$ acts trivially on each
  $N_i$ and hence on the direct sum. Conversely, if an element of $G$
  acts trivially on each $N_i$, it must be in $Z(G)$.

  Thus, $M$ has the structure of an object of the log category.

  We now describe the exponential map from $M$ to $G$. Define this as:

  $$\exp:M \to G$$

  $$\exp(a_1,a_2,\ldots,a_r) = a_1a_2\ldots a_r$$

  The map clearly commutes with the $G$-action. Also if $a \perp b$
  then each $a_i \perp b_j$ so in particular the product of the $a_i$s
  commutes with the product of the $b_j$s. The main condition to check
  is that if $a \perp b$, then $\exp(ab) = \exp(a)\exp(b)$. This
  follows from the fact that:

  $$\exp(a)\exp(b) = a_1a_2\ldots a_rb_1b_2\ldots b_r = a_1b_1a_2b_2 \ldots a_rb_r = \exp(ab)$$

  Surjectivity of $\exp$ follows from the fact that since $N_1, N_2,
  \ldots, N_r$ are normal and generate $G$, every element of $G$ can
  be written as a product of the form $a_1a_2\ldots a_r$ where each
  $a_i \in N_i$.
\end{proof}

Note that any group generated by finitely many normal Abelian
subgroups is nilpotent; in fact it is nilpotent of nilpotence class at
most $r$ where $r$ is the number of normal Abelian subgroups. 

More generally a group generated by finitely many normal {\em
  nilpotent} subgroups of nilpotence classes $c_1, c_2, \ldots c_r$
has nilpotence class at most $c_1 + c_2 + \ldots + c_r$. The ``full
logarithm'' that we have constructed above, on the other hand, has
nilpotence class only $\max \{ c_i \}$, a significant reduction in
complexity. 

Note that the logarithms we have constructed in these cases are far
from faithful and far from injective; they could be viewed as highly
wasteful.
\subsection{Inducting on this case}

Ideally, we would like something of the form: if $M$ is a full
logarithm for $G$, and $A$ is a full logarithm for $M$, then $A$ is a
full logarithm for $G$. The hope is that if $G$ is generated by a
family of normal subgroups, and we have logarithms for each of {\em
  those}, then we can use them to construct a logarithm of $G$. More
specifically, the question is:

\begin{quote}
  Suppose $G$ is generated by subgroups $N_i$ with full logarithms
  $\exp:M_i \to N_i$. If $M$ be the direct sum of the $M_i$s, can we
  give $M$ the structure of an object in the log category of $G$? And
  under what conditions is the analogue of the map defined in the last
  subsection, viz $\exp:M \to G$, a morphism in the log category?
\end{quote}

The first problem we face is that it is not clear how $G$ acts on the
$M_i$s. Certainly $G$ acts on the $N_i$s, but there is {\em a priori}
no reason to believe that there is any natural way of pulling back the
action to the $M_i$s. In fact, even in the special case where we have
a {\em reversible} logarithm (so that the $M_i$ is identified with
$N_i$ as a set) we cannot conclude that the set-theoretic action of an
element of $G$ on $M_i$ by the pullback, is actually a group action. 

The solution relies on putting further structure on the log category.

\subsection{The enriched log category}

\begin{definer}[Enriched log category]
  Let $G$ be a group and $K$ a subgroup of $\aut{G}$, such that $K$
  contains $\inn{G}$. The $K$-enriched log category of $G$, denoted
  $\enrichedlogcategory{G}{K}$, is defined as follows. The objects
  comprise the following data:

  \begin{enumerate}

  \item A group $A$

  \item A homomorphism $\phi: K \to \aut{A}$ (viz a $K$-action on $A$)

  \item A reflexive symmetric binary relation $\perp$ on $A$

  \end{enumerate}

  Such that the following are satisfied:

  \begin{enumerate}

  \item $\perp$ is $K$-invariant, viz:

    $$a \perp b \implies \phi(g)a \perp \phi(g)b$$
  \item For any $a$, the set of $b$ such that $a \perp b$ is a
    subgroup of $A$. This set shall be denoted as $a^\perp$.

  \item The set of $a$ such that $a \perp b$ for every $b \in A$, is
    the same as the set of $a$ for which $g.a = a$ for every $g
    \in \inn{G}$. This set will be denoted as $A^\perp$.

  \end{enumerate}

  We denote the objects as triples $(A,\phi,\perp)$. For $g \in G$, I
  will write $ga$ for $\phi(c_g)a$ where $c_g$ denotes the element of
  $\aut{G}$ defined as conjugation by $g$.

  If $A$ and $B$ are two objects, a morphism from $A$ to $B$ is a
  set-theoretic map $f: A \to B$ such that:

  \begin{enumerate}

  \item If $a_1 \perp a_2$, then $f(a_1) \perp f(a_2)$:

    $$f(a_1a_2) = f(a_1)f(a_2)$$

    In particular since $\perp$ is reflexive, we see that $f$ gives a
    homomorphism on any cyclic subgroup of $A$.

  \item $f$ commutes with the $K$-action, viz:

    $$f(ka) = kf(a)$$

    where $k \in K$.
  \end{enumerate}
\end{definer}

$K$-enrichment is essentially enlarging the group of automorphisms we
have to act on each object, while preserving the good properties we
had before. The best possible enrichment we could have is if $K = \aut{G}$;
the worst gives the ordinary log category.

\begin{definer}[Characteristic log category]
  The \definedind{characteristic log category} of a group $G$ is
  defined as $\enrichedlogcategory{G}{\aut{G}}$, viz., the log
  category enriched with all automorphisms of $G$.
\end{definer}

\subsection{Enriched logarithm}

We can now consider the problem of finding logarithms in an enriched
log category. The definitions remain the same, except that we now
impose the condition that the logarithm group live in the enriched log
category. Some important notions:

\begin{definer}[Enriched reversible logarithm]
  Let $G$ be a group and $K$ a subgroup of $\aut{G}$, which contains
  $\inn{G}$. Then a $K$-enriched reversible logarithm for $G$ is a
  group $A \in \enrichedlogcategory{G}{K}$ with a map $\exp:A \to G$
  which is faithful, full and injective. Thus, $\exp$ is an
  isomorphism in $\enrichedlogcategory{G}{K}$.
\end{definer}

Note that given an (unenriched) reversible logarithm for a group,
there is only one possible $K$-enrichment for it, because the
$K$-action on $G$ dictates element-wise the $K$-action on $A$.

Thus, given a reversible logarithm $A$ for $G$, and any subgroup $K$
of $\aut{G}$ such that $K$ contains $\inn{G}$, we can ask whether $A$
admits a $K$-enrichment.

\subsection{Normal subgroups and enriched log category}

Let's first state the result:

\begin{theorem}[Group generated by normal subgroups with enriched logarithms]
  Suppose $G$ is generated by normal subgroups $N_1, N_2, \ldots,
  N_r$. Suppose $M_i$ is a reversible logarithm for $N_i$ enriched
  under the $G$-action on $M_i$ (in other words, it is enriched by the
  image of $G$ in $\aut{N_i}$ via the action by conjugation). Then,
  the group $M = M_1 \oplus M_2 \oplus \ldots \oplus M_r$ is a ({\em
    not necessarily reversible}) logarithm for $G$.
\end{theorem}

\begin{proof}
  The proof follows from the previous case, once we note that the
  natural map $M \to N$ which sends each $M_i$ to the corresponding
  $N_i$, is an isomorphism in the log category of $G$. (I might fill
  in the proof details some time later).
\end{proof}

We see immediately some limitations:

\begin{itemize}

\item We cannot say unconditionally that the existence of a reversible
  logarithm on each normal subgroup guarantees the existence of a
  logarithm on the whole group.

\item Even under the hypotheses of enrichment, we do not get a {\em
    reversible} logarithm for the whole group. In fact, the logarithm
  we get in general is neither faithful nor injective. Hence, we
  cannot keep inducting.

\end{itemize}

Nonetheless, it is a start. For instance, consider:

\begin{itemize}

\item The dihedral group of order eight (acting on a set of four
  elements) does not admit a Lie algebra. However, it does admit an
  Abelian logarithm, because it is generated by two normal Abelian
  subgroups. The logarithm is an Abelian group of order $16$, and is
  neither faithful nor injective.

\item Lazard proved that any $p$-group of class at most $p-1$, has a
  ``Lie algebra'' which, in our language, is an Abelian reversible
  logarithm for the group. The above result then tells us that if $G$
  is a $p$-group generated by normal subgroups each of nilpotence
  class at most $p-1$, then $G$ admits an Abelian logarithm (which,
  again, is far from reversible).

\end{itemize}

\section{Another look at finite groups}

$p$-groups can be classified according to their level of complexity;
for instance:

\begin{itemize}

\item $p$-groups which admit Lie algebras: Lazard proved that this
  contains all $p$-groups of nilpotence class at most $p-1$; in fact,
  it contains all $p$-groups where the subgroup generated by any three
  elements has nilpotence class at most $p-1$ (there {\em are} other
  $p$-groups which admit Lie algebras).

\item $p$-groups which admit reversible Abelian logarithms: This seems
  a weaker condition than the former; however, I do not know of any
  examples of $p$-groups admitting reversible Abelian logarithms, which do
  not admit Lie algebras in the sense of Lazard.

\item $p$-groups which admit Abelian logarithms: This class is
  significantly bigger; we have shown that if a group is generated by
  normal subgroups each of which admits a {\em reversible} Abelian
  logarithm, then the group admits an Abelian logarithm.

\end{itemize}

It seems highly likely that not every nilpotent group admits an
Abelian logarithm. There are two interesting questions we can ask in
general:

\begin{itemize}

\item Given a $p$-group, what is the smallest nilpotence class we can
  have for a reversible logarithm for the group?

\item Given a $p$-group, what is the smallest nilpotence class we can
  have for a logarithm for the group?

\end{itemize}

\subsection{Iterated logarithms}

The next interesting question is about the existence of iterated
logarithms.


\section{Miscellanea}
\subsection{Pulling back logarithms}

Suppose $H$ and $G$ are groups, and $\alpha:G \to H$ is a
homomorphism. Then given any group $A \in \logcategory{H}$, we can
give $A$ a $G$-action, by defining $g.a = \alpha(g)a$. However, this
does not necessarily make $A$ an object of $\logcategory{G}$, because:

\begin{itemize}
\item It is not necessary that the inverse image of the center of $H$,
  be the center of $G$. If this does not happen, then the condition
  that the kernel of the map $G \to \aut{A}$ be $\inn{G}$ is violated.
\item It is not necessary that the set of $G$-fixed points is
  precisely the set $A^\perp$. This would be guaranteed if the map
  $\alpha$ were surjective.
\end{itemize}

%something to do with isoclinies?

\printindex

\end{document}
