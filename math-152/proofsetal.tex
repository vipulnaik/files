\documentclass{amsart}
\usepackage{fullpage,hyperref,vipul}
\title{Proofs and other ideas of mathematics}
\author{Math 152, Section 55 (Vipul Naik)}

\begin{document}
\maketitle

\section{General ideas related to proofs}

\subsection{The idea of models and proof}

The main idea behind the concept of proof is to establish something
clearly, give a fool-proof, error-free explanation of {\em why}
something is true. That means that all cases must be covered, every
step of the argument should be justified, and there shouldn't be any
hidden assumptions that aren't true. We'll go over some of these in
detail, but the first question you might be wondering is: {\em why
proof}?

So the one thing you have to remember about mathematics is that the
world of mathematics is a world of its own creation. It's true that
mathematical ideas and formalisms are applied a lot to real-world
settings, but often the mathematical ideas go far beyond what we can
determine or verify through real-world observation. In other words,
for a lot of the things we want to do in the mathematical world, it is
hard to be sure of them simply by looking around at the real
world. So, in mathematics, it is important to develop a way of being
sure of things that depends purely on internal reasoning. In fact,
this concept of internal reasoning, or {\em reasoning within the
framework of rules of the system}, is what defines mathematics.

If you think about mathematics as a system of rules-based internal
reasoning, your most significant introduction to mathematics isn't
when you learn numbers, it is when you learn how to play games, and
how to deploy strategies within games.

And it so happens that most systems of rules that we deal with involve
the concepts of numbers, many of them involve some geometric ideas,
and some of them need the tools of algebra and trigonometry and
calculus. But you shouldn't think of mathematics as being the same as
algebra or trigonometry or calculus. These are just tools. The main
feature of mathematics is that in mathematics, there is a strong place
for {\em internal reasoning} -- reasoning within the system to try to
determine what is true and what is false.

And, if you look at major developments in a number of academic
disciplines, you see that mathematics is seeping into most of
them. And I don't just mean that they are getting more quantitative,
though that's part of the story. Those who've seen Newton's laws and
classical mechanics know that there's a specific model (which, it
turns out, isn't exactly how the real world operates) and we can
predict how things will behave in that model. Nowadays, many papers in
the social sciences also start by creating some artificial model that
has a reasonable resemblance to reality, and then try to derive
formally what happens in that model. And the main difference between
mathematicians and those in other sciences is that for people in the
other sciences, they need to justify that their model has some kind of
resemblance with, or explanatory power about, the real world. But
mathematicians aren't subject to that constraint.

So think of mathematical rigor as something that allows mathematicians
to explore things where intuition, or real-world checks and balances,
are hard to find.

\subsection{Proof by example isn't; cover your bases, consider all cases}

So one of the things people often do in the real world is when they
want to know if something is true they take some example and check
it. And you see the media and politicians do that kind of thing
everyday. So whenever somebody wants to prove that some thing works,
they'll find one person to give a testimonial for it.

But in mathematics, we don't consider a few isolated examples to be
proof. And the reason is simple: different cases behave differently,
so the examples we choose are probably not representative. That's true
in the real world, but it is often even more true in mathematics,
where things aren't constrained to be realistic.

So mathematicians try to {\em cover all cases} in proofs. What does
this mean? If you want to prove a statement for all real numbers, it
isn't enough to prove it for all rational numbers. After all, there
are real numbers that aren't rational. So you need to prove the
statement for all rational numbers {\em and} all irrational numbers.

Now, {\em how} we choose to break down the problem into cases is up to
us. For some problems, the natural way of breaking up the problem may
be to first consider rational numbers and then consider irrational
numbers. Sometimes, it may be helpful to first consider positive
numbers and then consider negative numbers. If you have to prove a
statement for all numbers in a finite set, the ultimate break-up would
be to check it separately for every element of the finite set.

The thing you should remember is that {\em if you are breaking things
up into cases}, you should {\em remember to cover all cases}. And one
way to remember this is to think of mathematics as just about the
smartest adversary you can find in the battlefield. If you don't cover
every possible line of attack on your adversary, your adversary will
hide in the one place you forgot to cover.

\subsection{Conditional implication}

In mathematics, we often consider statements of the form:

``If $A$, then $B$''

Now, these kinds of statements can sometimes be confusing, so let's
try to understand what exactly this means. This roughly means that,
assuming that $A$ is given to be true, $B$ is true. For instance, ``if
I don't oversleep, I will attend the calculus lecture on
Friday''. That is a conditional statement.

There are a lot of subtleties about conditional implications that we
need to understand. The first is that ``If $A$, then $B$'' only means
that $A$ is {\em sufficient} for $B$. It doesn't mean that $A$ is
necessary for $B$. There may be other ways that $B$ could become true,
even if $A$ were false. For instance, you may say ``If I have enough
money, I'll eat lunch''. But you may be able to eat lunch even though
you don't have enough money -- by going to one of U of C's Free Food
events.

So ``if $A$, then $B$'' means that if, somehow, one could guarantee
$A$ to be true, $B$ would follow -- but there may be other ways to
guarantee $B$. In particular, if you prove a statement ``if $A$ then
$B$'' and then you separately prove that $A$ is true, then you would
have proved that $B$ is true.

\subsection{Rough work and fair work}

In many situations where we need to do a proof, there are two parts to
doing the proof. The first is the exploratory phase, or the discovery
phase, where we need to find a strategy that works for the proof. For
instance, in the case of $\epsilon-\delta$ proofs, the exploratory
phase involved coming up with a winning strategy for the prover or the
skeptic as the case may be. In this exploratory phase, we may do some
rough calculations, make some wild guesses, check out our intuition on
examples, etc. The exploratory phase may involve {\em working
backwards}, {\em splitting into cases}, etc. At the end of exploratory
phase, we have an overall proof strategy.

At the end of the exploratory phase, we hopefully have a clear proof
strategy. The next phase is that of clearly expressing the strategy
and showing that it works. When writing this final stategy and the
proof, you do not need to cover everything you went through in the
exploratory phase. Stick only to that which is more relevant to the
final proof strategy. Also, state the strategy right upfront and
proceed, to the extent possible, starting from what you know and
proceeding towards what you need to show.

\subsection{Opposite statement}

Another concept that I should mention, and that you've had a bit of
past experience with, is the {\em opposite} of a statement. This is
related to the question: {\em how do I prove that $A$ is not true}?
Well, in order to prove that, you first need a clear formulation of
what it means for $A$ to {\em not} be true. This new statement is
sometimes called the {\em negation} or {\em opposite} of $A$.

Now, some of you may have seen some Boolean algebra or logic, so you
might have some idea of the formal process of negating a statement,
but even if you haven't, most of the rules are intuitive provided you
pause to think and don't just try to rush. Keep your cool, and it's
not hard. I'll just mention some important ideas:

\begin{enumerate}

\item Negation turns {\em and} to {\em or}, and {\em or} to {\em
  and}. For instance, the negation of the statement $x = 1$ or $x = 2$
  is the statement $x \ne 1$ {\em and} $x \ne 2$.

\item Negation on a $\forall$ quantifier gives a $\exists$ quantifier
  and negation on a $\exists$ quantifier gives a $\forall$
  quantifier. For instance, the negation of the statement $\forall x
  \in \R, f(x^2) = f(x)^2$ is the statement $\exists x \in \R, f(x^2)
  \ne f(x)$. This came up when we looked at $\epsilon-\delta$ proofs.

\end{enumerate}

\subsection{Proof by contradiction}

One of the useful proof techniques is proof by contradiction. This
comes up sometimes, and I'll talk more about it when it does, but the
way it works is like this: suppose you are trying to prove $A$. So the
first thing you may try to do is prove $A$ straightforward, but that
may seem tricky. So what you do is this. You assume that the opposite
of $A$ is true. So you write down the opposite of $A$, and start with
that as given. And then, from that, you derive some statement that is
plainly {\em not} true. Since the conclusion isn't true, the statement
you started by assuming, namely, the opposite of $A$, couldn't have
been true either. And since the opposite of $A$ is false, $A$ itself
must be true.

Some of you may have seen the proof that $\sqrt{2}$ is
irrational. That proof is a classic example of proof by contradiction.

\section{Specific issues}

The material in the previous section is very general and I think most
of you would lap it up pretty easily. Most of you seem to have a
reasonable understanding of these ideas, but there are some more
specific issues that you may have with expressing your proofs. Below
are listed some of the specific issues that students in past years
have had in the first two advanced homeworks.

\subsection{Making your strategy and specific claims clear upfront}

This issue has occurred in the past with some of the $\epsilon-\delta$
proofs. If you're the prover, then the stategy involves finding an
expression for $\delta$ that works in terms of $\epsilon$. If you're
the skeptic, the strategy involves finding an $\epsilon$ for which no
$\delta$ works, and then being able to choose a value of $x$ in $(c -
\delta, c + \delta) \setminus \{ c\}$.

In the exploratory phase, you try to figure out a strategy that
works. Then, in the actual proof phase, you show that the strategy works.

{\em When writing up the final proof, please do not show the
exploratory phase}. Please write the final winning strategy
upfront. Then, proceed to translate the general statement about the
existence or non-existence of limit into a specific claim based on
your strategy. Then, do some algebraic manipulation or case-by-case
reasoning to prove that your strategy works.

Some examples:

\begin{itemize}
\item For the homework problem $\lim_{x \to 2} x^2 = 4$, state right
  at the beginning that the winning strategy is $\delta = \min \{1,
  \epsilon/5\}$. Then, state the specific claim: if $0 < |x - 2| <
  \min \{ 1, \epsilon/5 \}$, then $|x^2 - 4| < \epsilon$. Now, prove
  the specific claim.

  Some people do some algebraic manipulation to discover the $\delta$
  that works. Others are comfortable using the general formula that
  works for the quadratic. Whichever thing you choose to do, please
  remember that the less of this exploratory work you show, the
  clearer your proof is. This is because exploratory work, as a
  general rule, is mesy, with conditionals much more complicated,
  steps going forward and backward, etc. So please skip this and write
  your winning strategy clearly.

\item Consider problems where, for instance, we need to select a
  $\delta$ value for a given $\epsilon$ value, and the function is
  defined differently on rationals and irrationals. Here, we need to
  find a $\delta_1$ that works for rationals, a $\delta_2$ that works
  for irrationals, and then take $\delta = \min \{ \delta_1,
  \delta_2\}$.

  You should write down the strategy for choosing $\delta$ right on
  top, {\em make the specific claim}, and split into cases {\em to
  prove the specific claim}.

  Some of you split into cases first, proved things in each case, and
  gave the overall winning strategy at the end. {\em This is probably
  the way that you discover things in the exploratory phase, but it's
  not the prettiest way of presenting a final proof.}

\end{itemize}

{\em Caveat}: There are situations where it is advantageous to show
your exploratory phase. For instance, if you were a teacher and were
guiding students through a learning process, this exploratory phase
might be helpful. If you were trying to break ground with a similar
new problem, it might help to revisit the exploratory phase.

However, you should think of showing your exploratory phase as filming
the process of the manufacture of sausage, and the fair work proof
phase as the phase of enjoying the final sausage.

\subsection{Doing the general case clearly}

This problem arose with some advanced homework solutions in Homework
1, and a subsequent clarification was made. However, it's worth
reiterating here.

In the exploratory phase, we may use some specific numerical examples
to check if something is true. Then, we discover that the actual steps
work in somewhat greater generality, and we need that greater
generality in order to do the whole proof.

When writing down the final proof, jump directly to proving that the
actual steps work in somewhat greater generality.

The example from the first advanced homework was: ``Show that the
  function $f(x) := \lbrace \begin{array}{l} 1, \ x \text{ rational }
  \\ 0, \ x \text{ irrational }\end{array}$ is periodic but has no
  period.''

One possible discovery approach is as follows:

\begin{enumerate}
\item We notice that the number $1$ works in the sense that $f (x +
  1) = f(x)$ for all $x \in \R$. We prove this by splitting into the
  cases where $x$ is rational and $x$ is irrational.
\item After finishing that proof, we notice that, in fact, the proof
  depended only on the fact that rational + rational = rational and
  irrational + rational = irrational. Crucially, the only thing we
  were using about $1$ was that it is rational.
\item We thus conclude that any rational $h > 0$ works in place of
  $1$.
\item Since there are arbitrarily small positive rational numbers, we
  concluded that there is no period.
\end{enumerate}

In the final write-up of the proof, we remove steps (1) and (2) and
directly proceed with the claim of step (3), with the proof of that
claim basically mimicking our original proof of (1).

\subsection{Meta-strategies}

Some of the advanced problems involve constructing a strategy for one
game using strategies for other games {\em as black boxes}. For
instance, in problems 1 and 5 of advanced homework 2, you are asked
to come up with winning strategies for the prover for $|f|$, $\max \{
f, g\}$, and $\min \{ f, g \}$, assuming that there exist winning
strategies for $f$ and $g$.

Here, you assume that the winning strategies for $f$ and $g$ are given
to you on a platter, but you have to treat them as black boxes. In
other words, you assume some statement of the form:

``For every $\epsilon > 0$, there exists a $\delta_1 > 0$ such that if
$0 < |x - c| < \delta_1$, then $|f(x) - L| < \epsilon$.''

and:

``For every $\epsilon > 0$, there exists a $\delta_2 > 0$ such that if
$0 < |x - c| < \delta_2$, then $|g(x) - L| < \epsilon$.''

Our ``winning strategy'' for $H := \max \{ f,g \}$, is to choose, for a
given $\epsilon > 0$, $\delta = \min \{ \delta_1, \delta_2\}$, i.e.,
the minimum of the $\delta$s that work for $f$ and $g$.

We then make the specific claim: ``If $0 < |x - c| < \min \{ \delta_1,
\delta_2 \}$, then $|H(x) - L| < \epsilon$.''

After this, we prove the specific claim by splitting into cases for
$x$, based on whether $H(x) = f(x)$ or $H(x) = g(x)$.

Meta-strategies are tricky to understand at first, because the
strategies that we are using as black boxes are {\em unknown knowns}
-- we can use them, but have to treat them as black boxes.

\subsection{Fixed but arbitrary}

Another note about the $\epsilon-\delta$ proofs. In all these proofs,
$\epsilon$ is ``fixed but arbitrary.'' What this basically means is
that $\epsilon$ is fixed, but it is fixed by the skeptic, so we (as
the provers) have no control over the choice so we should be prepared
for the worst.

\end{document}