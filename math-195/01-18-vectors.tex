\documentclass[10pt]{amsart}

%Packages in use
\usepackage{fullpage, hyperref, vipul, enumerate}

%Title details
\title{Class quiz: Friday January 18: Vectors}
\author{Math 195, Section 59 (Vipul Naik)}
%List of new commands

\begin{document}
\maketitle

Your name (print clearly in capital letters): $\underline{\qquad\qquad\qquad\qquad\qquad\qquad\qquad\qquad\qquad\qquad}$

\vspace{0.1in}

{\bf YOU ARE ALLOWED TO DISCUSS ONLY QUESTIONS THAT BEGIN WITH A (*)
  OR (**). PLEASE ATTEMPT ALL OTHER QUESTIONS BY YOURSELF. EVEN FOR
  THE QUESTIONS YOU DISCUSS, PLEASE FINALLY ENTER ONLY THE ANSWER
  OPTION YOU ARE PERSONALLY MOST CONVINCED ABOUT -- DON'T ENGAGE IN
  GROUPTHINK.}

\begin{enumerate}

\item Suppose $S$ is a collection of {\em nonzero} vectors in $\R^3$
  with the property that the dot product of any two distinct elements
  of $S$ is zero. What is the maximum possible size of $S$? {\em Last
    time: $14/23$ correct}

  \begin{enumerate}[(A)]
  \item $1$
  \item $2$
  \item $3$
  \item $4$
  \item There is no finite bound on the size of $S$
  \end{enumerate}

  \vspace{0.1in}
  Your answer: $\underline{\qquad\qquad\qquad\qquad\qquad\qquad\qquad}$
  \vspace{0.6in}

\item Suppose $S$ is a collection of {\em nonzero} vectors in $\R^3$
  such that the cross product of any two distinct elements of $S$ is
  the zero vector. What is the maximum possible size of $S$? {\em Last
    time: $17/23$ correct}

  \begin{enumerate}[(A)]
  \item $1$
  \item $2$
  \item $3$
  \item $4$
  \item There is no finite bound on the size of $S$
  \end{enumerate}

  \vspace{0.1in}
  Your answer: $\underline{\qquad\qquad\qquad\qquad\qquad\qquad\qquad}$
  \vspace{0.6in}

\item (**) Suppose $a$ and $b$ are vectors in $\R^3$. Which of the
  following is/are true? {\em Last time: $6/23$ correct}

  \begin{enumerate}[(A)]
  \item If both $a$ and $b$ are nonzero vectors, then $a \times b$ is
    a nonzero vector.
  \item If $a \times b$ is a nonzero vector, then $a \cdot (a \times
    b)$ is a nonzero real number.
  \item If $a \times b$ is a nonzero vector, then $a \times (a \times
    b)$ is a nonzero vector.
  \item All of the above
  \item None of the above
  \end{enumerate}

  \vspace{0.1in}
  Your answer: $\underline{\qquad\qquad\qquad\qquad\qquad\qquad\qquad}$
  \vspace{0.6in}

\item (*) Suppose $a$, $b$, $c$, and $d$ are vectors in $\R^3$, with
  $a \times b \ne 0$ and $c \times d \ne 0$. What does $(a \times b)
  \times (c \times d) = 0$ mean? {\em Last time: $9/23$ correct}

  \begin{enumerate}[(A)]
  \item Both the vectors $a$ and $b$ are perpendicular to both the
    vectors $c$ and $d$.
  \item $a$ and $b$ are perpendicular to each other and $c$ and $d$
    are perpendicular to each other.
  \item $a$ and $c$ are perpendicular to each other and $b$ and $d$
    are perpendicular to each other.
  \item The plane spanned by $a$ and $b$ is perpendicular to the plane
    spanned by $c$ and $d$.
  \item $a$, $b$, $c$, and $d$ are all coplanar.
  \end{enumerate}

  \vspace{0.1in}
  Your answer: $\underline{\qquad\qquad\qquad\qquad\qquad\qquad\qquad}$
  \vspace{0.6in}

\item (**) The {\em correlation} between two vectors in $\R^n$ is
  defined as the quotient of the dot product of the vectors by the
  product of their lengths. Suppose $a$, $b$, and $c$ are vectors in
  $\R^n$ such that the correlation between vectors $a$ and $b$ is a
  number $x$ and the correlation between $b$ and $c$ is a number $y$,
  and suppose $x,y$ are both positive. What is the maximum possible
  value of the correlation between $a$ and $c$ given this information?
  {\em Hint: Geometrically if $\theta_{ab}$ is the angle between $a$
  and $b$, $\theta_{bc}$ is the angle between $b$ and $c$, and
  $\theta_{ac}$ is the angle between $a$ and $c$, then $|\theta_{ab} -
  \theta_{bc}| \le \theta_{ac} \le \theta_{ab} + \theta_{bc}$.}  {\em
  Last time: $5/23$ correct}

  \begin{enumerate}[(A)]
  \item $xy$
  \item $\max \{ 1, xy \}$
  \item $\min \{ 1, xy \}$
  \item $xy + \sqrt{(1 - x^2)(1 - y^2)}$
  \item $xy - \sqrt{(1 - x^2)(1 - y^2)}$
  \end{enumerate}

  \vspace{0.1in}
  Your answer: $\underline{\qquad\qquad\qquad\qquad\qquad\qquad\qquad}$
  \vspace{0.6in}

\item If the correlation between nonzero vector $v$ and nonzero vector
  $w$ in $\R^n$ is $c$, then we say that the {\em proportion} of
  vector $w$ {\em explained by} vector $v$ is $c^2$. If $v_1, v_2,
  \dots, v_k$ are all pairwise orthogonal nonzero vectors, and $c_i$
  is the correlation between $v_i$ and $w$, then $c_1^2 + c_2^2 +
  \dots + c_k^2 \le 1$, with equality occurring if and only if $k =
  n$. (This is all a result of the Pythagorean theorem). If $k < n$,
  then $1 - (c_1^2 + c_2^2 + \dots + c_k^2)$ is the {\em unexplained
    proportion} of $w$.

  Suppose $w$ is the {\em variation of beauty} vector, $v_1$ is the
  {\em variation of genes} vector, and $v_2$ is the {\em variance of
  make-up} vector. Assume that $v_1$ and $v_2$ are orthogonal (i.e.,
  there is no correlation between genes and make-up choice). If the
  correlation between $v_1$ and $w$ is $0.6$ and the correlation
  between $v_2$ and $w$ is $0.3$, what proportion of the variation of
  beauty remains unexplained (i.e., is not explained by either genes
  or make-up)? {\em Last time: $17/23$ correct}

  \begin{enumerate}[(A)]
  \item $0.1$
  \item $0.19$
  \item $0.55$
  \item $0.74$
  \item $1$
  \end{enumerate}

  \vspace{0.1in}
  Your answer: $\underline{\qquad\qquad\qquad\qquad\qquad\qquad\qquad}$
  \vspace{0.6in}

\end{enumerate}
\end{document}