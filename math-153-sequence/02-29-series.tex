\documentclass[10pt]{amsart}

%Packages in use
\usepackage{fullpage, hyperref, vipul, enumerate}

%Title details
\title{Class quiz: February 29: Series}
\author{Math 153, Section 55 (Vipul Naik)}
%List of new commands

\begin{document}
\maketitle

Your name (print clearly in capital letters): $\underline{\qquad\qquad\qquad\qquad\qquad\qquad\qquad\qquad\qquad\qquad}$

\begin{enumerate}

\item Suppose $p$ is a polynomial that take positive values on all
  nonnegative integers. Consider the summation $\sum_{k=1}^\infty
  \frac{(k^2 + 1)^{2/3}}{p(k)}$. Under what conditions does the
  summation converge? Note that the degree of $p$ must be a
  nonnegative integer.

  \begin{enumerate}[(A)]
  \item The summation converges if and only if the degree of $p$ is
    {\em at least} one
  \item The summation converges if and only if the degree of $p$ is
    {\em at least} two
  \item The summation converges if and only if the degree of $p$ is
    {\em at least} three
  \item The summation converges if and only if the degree of $p$ is
    {\em at most} two
  \item The summation converges if and only if the degree of $p$ is
    {\em at most} one
  \end{enumerate}

  \vspace{0.1in}
  Your answer: $\underline{\qquad\qquad\qquad\qquad\qquad\qquad\qquad}$
  \vspace{0.15in}
\item Suppose $p$ is a polynomial that take positive values on all
  nonnegative integers. Consider the summation $\sum_{k=1}^\infty
  \frac{(-1)^k(k^2 + 1)^{2/3}}{p(k)}$. Under what conditions does the
  summation converge? Note that the degree of $p$ must be a
  nonnegative integer.

  \begin{enumerate}[(A)]
  \item The summation converges if and only if the degree of $p$ is
    {\em at least} one
  \item The summation converges if and only if the degree of $p$ is
    {\em at least} two
  \item The summation converges if and only if the degree of $p$ is
    {\em at least} three
  \item The summation converges if and only if the degree of $p$ is
    {\em at most} two
  \item The summation converges if and only if the degree of $p$ is
    {\em at most} one
  \end{enumerate}

  \vspace{0.1in}
  Your answer: $\underline{\qquad\qquad\qquad\qquad\qquad\qquad\qquad}$
  \vspace{0.15in}

\item (*) Which of the following series converges? Assume for all
  series that the starting point of summation is large enough that the
  terms are well defined. {\em Last year: $11/25$ correct}

  \begin{enumerate}[(A)]
  \item $\sum 1/(k \ln (\ln k))$
  \item $\sum 1/(k \ln k)$
  \item $\sum 1/(k (\ln (\ln k))^2)$
  \item $\sum 1/(k (\ln k)(\ln (\ln k)))$
  \item $\sum 1/(k (\ln k)(\ln (\ln k))^2)$
  \end{enumerate}

  \vspace{0.1in}
  Your answer: $\underline{\qquad\qquad\qquad\qquad\qquad\qquad\qquad}$
  \vspace{0.15in}

\item Which of the following series converges? {\em Last year: $23/25$
  correct}

  \begin{enumerate}[(A)]
  \item $\sum \frac{k + \sin k}{k^2 + 1}$
  \item $\sum \frac{k + \cos k}{k^3 + 1}$
  \item $\sum \frac{k^2 - \sin k}{k + 1}$
  \item $\sum \frac{k^3 + \cos k}{k^2 + 1}$
  \item $\sum \frac{k}{\sin(k^3 + 1)}$
  \end{enumerate}

  \vspace{0.1in}
  Your answer: $\underline{\qquad\qquad\qquad\qquad\qquad\qquad\qquad}$
  \vspace{0.15in}

\item Consider the series $\sum_{k=0}^\infty \frac{1}{2^{2^k}}$. What
  can we say about the sum of this series? {\em Last year: $14/26$
  correct}

  \begin{enumerate}[(A)]
  \item It is finite and strictly between $0$ and $1$.
  \item It is finite and equal to $1$.
  \item It is finite and strictly between $1$ and $2$.
  \item It is finite and equal to $2$.
  \item It is infinite.
  \end{enumerate}

  \vspace{0.1in}
  Your answer: $\underline{\qquad\qquad\qquad\qquad\qquad\qquad\qquad}$
  \vspace{0.15in}

\item For one of the following functions $f$ on $(0,\infty)$, the
  integral $\int_0^\infty f(x) \, dx$ converges but $\int_0^\infty
  |f(x)| \, dx$ does not converge. What is that function $f$? (Note
  that this is similar to, but not quite the same as, the absolute
  versus conditional convergence notion for series).

  \begin{enumerate}[(A)]
  \item $f(x) = \sin x$
  \item $f(x) = \sin(\sin x)$
  \item $f(x) = (\sin \sqrt{x})/\sqrt{x}$
  \item $f(x) = (\sin x)/x$
  \item $f(x) = (\sin^3x)/x^3$
  \end{enumerate}

  \vspace{0.1in}
  Your answer: $\underline{\qquad\qquad\qquad\qquad\qquad\qquad\qquad}$
  \vspace{0.15in}

\item (**) Consider the function $F(x,p) := \sum_{n=1}^\infty
  \frac{x^n}{n^p}$ with $x$ and $p$ both real numbers. For what values
  of $x$ and what values of $p$ does this summation converge? {\em
  Last year: $7/26$ correct}
  \begin{enumerate}[(A)]
  \item For $|x| < 1$, it converges for all $p \in \R$. For $|x| \ge
    1$, it does not converge for any $p$.
  \item For $|x| \le 1$, it converges for all $p \in \R$. For $|x| >
    1$, it does not converge for any $p$.
  \item For $|x| < 1$, it converges for all $p \in \R$. For $|x| > 1$,
    it does not converge for any $p$. For $|x| = 1$, it converges if
    and only if $p > 1$.
  \item For $|x| < 1$, it converges for all $p \in \R$. For $|x| > 1$,
    it does not converge for any $p$. For $x = 1$, it converges
    if and only if $p > 0$. For $x = -1$, it converges if and only if
    $p > 1$.
  \item For $|x| < 1$, it converges for all $p \in \R$. For $|x| > 1$,
    it does not converge for any $p \in \R$. For $x = 1$, it converges
    if and only if $p > 1$. For $x = -1$, it converges if and only if
    $p > 0$.
  \end{enumerate}

  \vspace{0.1in}
  Your answer: $\underline{\qquad\qquad\qquad\qquad\qquad\qquad\qquad}$
  \vspace{0.15in}

  There is a result of calculus which states that, under suitable
  conditions, if $f_1, f_2, \dots, f_n, \dots$ are all functions, and
  we define $f(x) := \sum_{n=1}^\infty f_n(x)$, then $f'(x) =
  \sum_{n=1}^\infty f_n'(x)$. In other words, under suitable
  assumptions, we can differentiate a sum of countably many functions
  by differentiating each of them and adding up the derivatives.

  We will not be going into what those assumptions are, but will
  consider some applications where you are explicitly told that these
  assumptions are satisfied.

\item (**) Consider the summation $\zeta(p) := \sum_{n=1}^\infty
  \frac{1}{n^p}$ for $p > 1$. Assume that the required assumptions are
  valid for this summation, so that $\zeta'(p)$ is the sum of the
  derivatives of each of the terms (summands) with respect to
  $p$. What is the correct expression for $\zeta'(p)$? {\em Last year:
  $8/26$ correct}
  
  \begin{enumerate}[(A)]
  \item $\sum_{n=1}^\infty \frac{-p}{n^{p+1}}$
  \item $\sum_{n=1}^\infty \frac{-1}{(p+1)n^{p+1}}$
  \item $\sum_{n=1}^\infty \frac{p}{n^{p-1}}$
  \item $\sum_{n=1}^\infty \frac{-\ln n}{n^p}$
  \item $\sum_{n=1}^\infty \frac{-\ln n}{n^{p+1}}$
  \end{enumerate}

  \vspace{0.1in}
  Your answer: $\underline{\qquad\qquad\qquad\qquad\qquad\qquad\qquad}$
  \vspace{0.15in}

\item Recall that we defined $F(x,p) := \sum_{n=1}^\infty
  \frac{x^n}{n^p}$ with $x$ and $p$ both real numbers. Assume that,
  for a particular fixed value of $p$, the summation satisfies the
  conditions as a function of $x$ for $|x| < 1$. What is its
  derivative with respect to $x$, keeping $p$ constant? {\em Last
  year: $14/26$ correct}

  \begin{enumerate}[(A)]
  \item $\sum_{n=1}^\infty \frac{x^{n+1}}{n^{p+1}}$
  \item $\sum_{n=1}^\infty \frac{x^{n+1}}{n^{p-1}}$
  \item $\sum_{n=1}^\infty \frac{x^{n-1}}{n^{p+1}}$
  \item $\sum_{n=1}^\infty \frac{x^{n-1}\ln n}{n^{p+1}}$
  \item $\sum_{n=1}^\infty \frac{x^{n-1}}{n^{p-1}}$
  \end{enumerate}

  \vspace{0.1in}
  Your answer: $\underline{\qquad\qquad\qquad\qquad\qquad\qquad\qquad}$
  \vspace{0.15in}

\item The series $\sum_{n=1}^\infty \frac{1}{n}$ diverges. Since it is
  a series of positive terms, this means that the partial sums get
  arbitrarily large. What is the approximate smallest value of $N$
  such that $\sum_{n=1}^N \frac{1}{n} > 100$? {\em Last year: $14/26$
  correct}
  \begin{enumerate}[(A)]
  \item Between $90$ and $110$
  \item Between $2000$ and $3000$
  \item Between $10^{40}$ and $10^{50}$
  \item Between $10^{90}$ and $10^{110}$
  \item Between $10^{220}$ and $10^{250}$
  \end{enumerate}

  \vspace{0.1in}
  Your answer: $\underline{\qquad\qquad\qquad\qquad\qquad\qquad\qquad}$
  \vspace{0.15in}

\end{enumerate}

\end{document}
