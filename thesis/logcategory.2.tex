\documentclass[a4paper]{amsart}

%Packages in use
\usepackage{fullpage, hyperref, vipul, diagrams}


%Title details
\title{Exponentials and the log category}
\author{Vipul Naik}
\thanks{\copyright Vipul Naik, Ph.D. student, University of Chicago}

%List of new commands
\newcommand{\logcategory}[1]{\textsc{Log-Category}\left(#1\right)}
\newcommand{\enrichedlogcategory}[2]{\textsc{Enriched-Log-Category}\left(#1 \, ; \, #2\right)}
\newcommand{\characteristiclogcategory}[1]{\textsc{Characteristic-Log-Category}\left(#1 \right)}
\newcommand{\aut}[1]{\text{Aut}\left(#1\right)}
\newcommand{\inn}[1]{\text{Inn}\left(#1\right)}
\makeindex

\begin{document}
\maketitle
%\tableofcontents

\begin{abstract}
  In this write-up, I jot down some ideas I have about a ``log
  category'', a category such that the exponential maps we usually
  encounter in the theory of Lie algebras actually become
  homomorphisms between objects of the category. 
\end{abstract}

\section{Definition of the log category}

Let $G$ be a group. The log category of $G$ is a category whose
objects are groups enriched with a $G$-action and some other
structure, and the morphisms are maps which preserve that structure.

\begin{definer}[Log category]
  Let $G$ be a group. The log category of $G$, which we denote by
  $\logcategory{G}$ is defined as follows. The objects of
  $\logcategory{G}$ comprise the following data:

  \begin{enumerate}

  \item A group $A$

  \item A homomorphism $\phi: G \to \aut{A}$ (viz a $G$-action on $A$)

  \item A reflexive symmetric binary relation $\perp$ on $A$

  \end{enumerate}

  Such that the following are satisfied:

  \begin{enumerate}

  \item $\perp$ is $G$-invariant, viz:

    $$a \perp b \implies \phi(g)a \perp \phi(g)b$$
  \item For any $a$, the set of $b$ such that $a \perp b$ is a
    subgroup of $A$. This set shall be denoted as $a^\perp$.

  \item The set of $a$ such that $a \perp b$ for every $b \in A$, is
    the same as the set of $a$ for which $\phi(g)a = a$ for every $g
    \in G$. This set will be denoted as $A^\perp$.

  \item The kernel of the action of $G$ on $A$ is precisely the center
    $Z(G)$. Thus, the map $\phi:G \to \aut{A}$ factors through the
    quotient map $G \to G/Z(G) = \inn{G}$, and the map from $\inn{G}$
    to $\aut{A}$ is injective.

  \end{enumerate}

  We denote the objects as triples $(A,\phi,\perp)$. I will write $ga$
  for $\phi(g)a$ where it is unambiguous, and will use the same symbol
  $\perp$ for all objects.

  If $A$ and $B$ are two objects of $\logcategory{G}$, a morphism from
  $A$ to $B$ is a set-theoretic map $f: A \to B$ such that:

  \begin{enumerate}

  \item If $a_1 \perp a_2$, then $f(a_1) \perp f(a_2)$ and also:

    $$f(a_1a_2) = f(a_1)f(�_2)$$

    In particular since $\perp$ is reflexive, we see that $f$ gives a
    homomorphism on any cyclic subgroup of $A$.

  \item $f$ commutes with the $G$-action, viz:

    $$f(ga) = gf(a)$$
  \end{enumerate}
\end{definer}


The log category of a group behaves very differently from most
algebraic categories, in a number of ways. In the coming sections, we
shall try to see what information about a group is encoded in its log
category.

\section{Objects in the log category}

\subsection{Group-types in the log category}

Given any group $G$, $G$ is naturally a member of $\logcategory{G}$,
where:

\begin{itemize}

\item The action of $G$ on itself is by conjugation (inner
  automorphisms).

\item The relation $\perp$ is that of {\em commuting}.

\end{itemize}

More generally, an object $A$ in the log category of $G$ is said to be
of \definedind{group-type}, if the relation $\perp$ in $A$ is that of
commuting. Note that for objects of group type, the fact that
$H^\perp$ is a subgroup for any subgroup $H$ of $A$ is automatic. It
is also automatic that the action of $g \in G$ preserves $\perp$.

\subsection{Ring-types in the log category}

Suppose $A$ is a ring whose multiplicative group contains $G$, and
with the property that $A$ is generated, as a $Z(A)$-module, by the
image of $G$. Then, $A$ can be viewed as an object of
$\logcategory{G}$, as follows:

\begin{itemize}

\item The group structure on $A$ is that of addition

\item The relation $\perp$ is that of commuting {\em multiplicatively}

\item The action by $G$ is via conjugation

\end{itemize}

The fact that $A$ is generated by $G$ as a $Z(A)$-module is precisely
what we need to ensure that $A^\perp$ is the set of $G$-fixed points,
and that the kernel of the action of $G$ on $A$ is precisely the
center of $G$.

An object $A$ which arises in the above fashion is said to be of
\definedind{ring-type}.

Examples of the above situation:

\begin{itemize}

\item Let $R$ be any commutative ring. Then $R[G]$ (the group algebra
  of $G$ over $R$) is an object of ring-type in the log category of $G$.

\item Suppose $R$ is a ring with the property that $R$ is {\em
    additively generated by its invertibles} (or $R$ is a
  $\field$-algebra which is generated as a $\field$-vector space by
  its invertibles). Then $R$ is in the log category of its group of
  invertible elements. Thus, $M_n(\R)$ is in the log category of
  $GL_n(\R)$.
\end{itemize}

The two examples are at far extremes: in the first example, $R[G]$ is
exponential in the size of $G$; in the second example, the group is
almost the whole ring.

\subsection{Ring-without-identity types in the log category}

The statements made in the previous section go through even for rings
without identity. A particularly interesting case is that of algebra
groups. Suppose $N$ is a nilpotent ring (without identity) -- in
particular, any element of $N$ is nilpotent. Then we can associate to
$N$ a group, called its \definedind{algebra group}, which for each $x
\in N$ is the formal element $1 + x$, and where the multiplication is
defined as $(1 + x)(1 + y) = 1 + (x + y + xy)$. Nilpotence is required
to be able to invert $(1 + x)$.

In the above language, any nilpotent ring (without identity) is in the
log category of the corresponding algebra group.

\subsection{Lie ring-types in the log category}

An object of \definedind{Lie ring-type} in the log category is an object
obtained from a Lie ring $L$, as follows:

\begin{itemize}

\item The group structure is the additive structure of $L$

\item The relation $\perp$ is the relation of the Lie bracket being
  zero.

\item The $G$-action on $L$ gives Lie ring automorphisms of $L$ (and
  not just Abelian group automorphisms).
\end{itemize}

Note that any object of ring-type or ring-without-identity-type is
{\em also} of Lie ring-type, because to any ring we have an associated
Lie ring and the relation $\perp$ of commuting in the ring becomes the
relation of the Lie bracket being zero, in the associated Lie
ring. However, many Lie rings do not have associated rings.

Given any connected real Lie group, its ``Lie algebra'' is an object
of Lie ring-type in its log category, where the action of the group on
the Lie algebra is the adjoint representation.

\subsection{Biadditive types in the log category}

\begin{definer}[Object of biadditive type]
  An object of \definedind{biadditive type} in the log category of a
  group $G$, is an Abelian group $A$ in the log category of $G$,
  equipped with a biadditive map $b:A \times A \to B$, to another
  Abelian group $B$, such that for $a, a' \in A$, $a \perp a' \iff
  b(a,a') = 0$.
\end{definer}

Any object of Lie ring-type (and hence also any object of ring-type or
ring-without-identity type) is of biadditive type; the biadditive map
in that case is the Lie bracket. Objects of biadditive type possess
structure over and above the mere structure of an Abelian group in the
log category, because there is a biadditive function that measures
``how far'' $a$ and $a'$ are from being $\perp$.

\section{Morphisms in the log category}

The log-category is different from the usual algebraic categories we
deal with, in two important ways:

\begin{itemize}

\item Objects could be isomorphic in the log category, even if they
  are not structurally the same.

\item A bijective homomorphism in the log category need not be an
  isomorphism.

\end{itemize}

\subsection{Notions of center and Abelianization}

\begin{definer}[$G$-center]
  Let $A \in \logcategory{G}$. Then the $G$-center of $A$, denoted
  $A^\perp$, is defined as the set of elements in $A$ which are fixed
  under the $G$-action. Equivalently, it is the set of elements in $A$
  which are $\perp$ to every element in $A$.
\end{definer}

\begin{definer}[$G$-derived subgroup, $G$-Abelianization]
  Let $A \in \logcategory{G}$. The $G$-\definedind{derived subgroup}
  of $A$, denoted $A'_G$, is defined as the quotient of $A$ by the
  congruence generated by the equivalence relation of being in the
  same $G$-orbit. The $G$-\definedind{Abelianization} of $A$, denoted
  $A^{ab}_G$, is defined as the quotient of $A$ by $A'_G$.
\end{definer}

When $G$ is viewed as an object in its own log category, we get the
usual notions of center, derived subgroup and Abelianization. The good
thing about the center and Abelianization is:

\begin{theorem}
  If $f:A \to B$ is a morphism in $\logcategory{G}$, then $f$ induces
  a group homomorphism from $A^\perp$ to $B^\perp$ (this is obvious
  from the description of $A^\perp$ as the set of $G$-fixed points of
  $A$).
\end{theorem}

Although it is not necessary from the definitions that $f$ sends
$A'_G$ inside $B'_G$, we shall see that this is what happens in a
number of situations.
\subsection{Notions of faithful, injective, full and almost full}

Here are some definitions:

\begin{definer}[Morphism properties]
  A morphism $f:A \to B$ of objects in $\logcategory{G}$ is termed:

  \begin{itemize}

  \item \adefinedproperty{morphism}{faithful} if $f(x) \perp f(y) \iff
    x \perp y$

  \item \adefinedproperty{morphism}{injective} if the associated
    set-theoretic map is injective.

  \item \adefinedproperty{morphism}{central} if the inverse image of
    the identity element of $B$ contains only elements in the center
    of $A$.

  \item \adefinedproperty{morphism}{almost full} if the image of $A$
    under $f$ generates $B$ as a group, and if the induced maps on the
    $G$-center is surjective.

  \item \adefinedproperty{morphism}{full} if it is surjective.
  \end{itemize}

\end{definer}

A map in the log category is an isomorphism iff it is faithful, full
and injective.
\subsection{Logarithms}

We can now define the notion of logarithm of a group:

\begin{definer}[Logarithm of a group]
  A \definedind{logarithm} of a group $G$ is an object $A \in
  \logcategory{G}$ along with an almost full morphism $\exp:A \to G$ in
  $\logcategory{G}$. The logarithm is termed:

  \begin{itemize}

  \item \adefinedproperty{logarithm}{faithful} if $\exp$ is a faithful
    morphism.

  \item \adefinedproperty{logarithm}{full} if $\exp$ is a full
    morphism, viz., $\exp$ is surjective.

  \item \adefinedproperty{logarithm}{reversible} if $\exp$ is
    faithful, full and injective. In other words, $\exp$ is an
    isomorphism in the log category.

  \item \adefinedproperty{logarithm}{central} if $\exp$ is a central
    morphism.
  \end{itemize}
\end{definer}

Note that any injective morphism is central. Also, any faithful almost full
morphism is central.

\subsection{What does isomorphism mean in a log category?}

If $A$ and $B$ are isomorphic as objects in $\logcategory{G}$, that
does {\em not} imply that $A$ and $B$ are isomorphic as abstract
groups.  However, we can still say that $A$ and $B$ are ``reasonably''
similar. For instance:

\begin{itemize}

\item $A$ and $B$ have the same cardinality.

\item If both $A$ and $B$ are finite groups, they have the same order
  statistics. In other words, the number of elements of any given
  order in $A$, equals the number of elements of that order in $B$.
 
\end{itemize}

More constraints depend on the nature of the group $G$; as we shall
see, when $G$ is Abelian, isomorphism in the log category is the usual
notion of isomorphism.

\subsection{Automorphisms in the log category}

The automorphisms of $A$ as an object in the log category of $G$ could
be very different from its automorphisms as a group. First, not all
automorphisms of $A$ as a group need be automorphisms in the log
category; conversely, not every automorphism of $A$ as an object in
$\logcategory{G}$ need be an automorphism of $A$ as an abstract group.

The automorphism group of $G$ as an object in its own log category
contains the usual $\aut{G}$, but also contains other maps which look
like automorphisms when restricted to ``Abelian'' subgroups but which
are not globally automorphisms. Such maps are termed
``quasi-automorphisms''.

\subsection{Direct products}

Given a group $G$, and two objects, $A, B \in \logcategory{G}$, we can
define the \definedind{direct product} of $A$ and $B$, denoted $A
\times B$, as follows:

\begin{itemize}

\item As a group, it is the group $A \times B$.

\item The $G$-action is coordinate-wise.

\item The relation $\perp$ is also applied coordinate-wise. In other
  words, $(a,b) \perp (a',b') \iff a \perp a' \text{ and } b \perp
  b'$.

\end{itemize}

The direct product {\em is} a category-theoretic product, in the sense
that given any $C \in \logcategory{G}$ with morphism $C \to A$ and $C
\to B$, there is a unique morphism $C \to A \times B$ which, composed
with the projection maps, gives the required morphisms.

\section{The questions we want to study}

Starting out with a group $G$, we want to study the question: how does
$\logcategory{G}$ look? We will look at two aspects of this question:

\begin{itemize}

\item What are the possible logarithms for $G$? In particular, what
  are the possible reversible logarithms of $G$? Are there any
  reversible logarithms of ring-type, or Lie ring-type, or biadditive
  type?

\item What does $\logcategory{G}$ look like, in the large?

\end{itemize}

\subsection{For any Abelian group}

Let $G$ be an Abelian group. Then, for any $A \in \logcategory{G}$,
the $G$-action on $A$ must be trivial. Thus, by the hypotheses, we
conclude that $\perp$ is satisfied between any two elements of $A$,
and hence homomorphisms between objects of $\logcategory{G}$ are
genuine group homomorphisms. Conversely, it is easy to see that given
any group $A$ and $G$ acting trivially on $A$, we get an object of
$\logcategory{G}$.

In summary:

\begin{quote}
  The log category of an Abelian group is the usual category of groups
  with group homomorphisms.
\end{quote}

From these, the following are easy to deduce:

\begin{itemize}

\item A logarithm for an Abelian group is the same thing as a group
  along with a quotient map to that Abelian group.

\item The only possible reversible logarithm for the Abelian group is
  an isomorphic Abelian group.

\end{itemize}

We know that any finite Abelian group occurs as the additive group of
a commutative ring, so we can view the group itself as an object of
ring-type.

\subsection{Existence of a central full Abelian logarithm}

We prove our first result:

\begin{theorem}[Existence of a faithful full Abelian logarithm]
  Suppose $G$ is a finite group, and possesses a central full Abelian
  logarithm $A$. Then $G$ is nilpotent.
\end{theorem}

\begin{proof}
  It suffices to show that if $g,h \in G$ have orders $p^\alpha$ and
  $q^\beta$ where $p \ne q$ are primes, then $g$ commutes with $h$.
  We shall prove this by showing that they are both powers of another
  element of the group.

  Since $A$ is a full Abelian logarithm for $G$, there exist $a,b \in
  A$ such that $\exp(a) = g$ and $\exp(b) = h$. Let us try to compute
  the order of $a + b$. Since $\exp(p^\alpha a) = e$, and
  $\exp$ is central, we see that $p^\alpha a \perp p^\alpha b$ and we
  thus get:

  $$(\exp(a + b))^{p^\alpha} = \exp(p^\alpha(a + b)) = \exp(p^\alpha a + p^\alpha b) = \exp (p^\alpha b) = h^{p^{\alpha}}$$

  Similarly:

  $$(\exp(a + b))^{q^\beta} = g^{q^\beta}$$

  A little group theory tells us that $\exp(a + b)$ has order exactly
  $p^\alpha q^\beta$, and that both $g$ and $h$ are powers of it. This
  completes the proof.
\end{proof}

I am not sure if the proof can be altered to drop the assumptions of
central and full. At any rate, this shows that a non-nilpotent group
cannot possess a {\em reversible} Abelian logarithm.

\subsection{Do all nilpotent groups have Abelian logarithms?}

It seems a hard question to judge whether a nilpotent group has an
Abelian logarithm. However, we can very easily show that there are
nilpotent groups which do not have {\em reversible} Abelian
logarithms. For some such groups, the order statistics don't match up
with an Abelian group, for others, we can do a hand-check. Here are
some examples:

\begin{itemize}

\item The eight-element quaternion group has no reversible Abelian
  logarithm. It has six elements of order $4$, and no Abelian group of
  order $8$ has that many elements of order $4$.

\item The eight-element dihedral group has no reversible Abelian
  logarithm. This again can be done via a hand-check. The only
  ``candidate'' is the group $\Z/4\Z \times \Z/2\Z$, and there's no
  appropriate map from it.

\end{itemize}

Both these groups, however, possess full logarithms which are twice
their size. The full Abelian logarithms are neither faithful nor
injective, and they are {\em not} Lie algebras. However, there are
interesting Lie algebra structures lurking underneath
them. Specifically, in both cases, we can find an object of Lie
ring-type and having the same size ($8$) such that the full Abelian
logarithm has a full morphism to that Lie ring-type object as
well. Thus, although the group cannot directly be matched up with an
object of Lie ring-type, there is such an object and they both have a
common ``double''. The picture is like this:

\begin{diagram}
  & & \text{Abelian group} & \text{ of size 16}& \\
  & \ldTo^{\exp} & & \rdTo & \\
  \text{Original group} & \text{ of size 8}& & \text{Lie algebra} & \text{ of size 8}
\end{diagram}

A mere Abelian logarithm is in general not good enough; we would
ideally like the Abelian logarithm to be of ring-type or Lie
ring-type. Thus, the questions we want to explore are:

\begin{enumerate}

\item Which nilpotent groups possess reversible Abelian logarithms of
  Lie ring-type? A reversible Abelian logarithm of Lie-type is a ``Lie
  algebra'' with the give group as its Lie group.

\item Which nilpotent groups possess reversible Abelian logarithms?

\item Which nilpotent groups possess full Abelian logarithms?

\end{enumerate}

\section{Enriched and characteristic log categories}

\subsection{Log category enriched with more automorphisms}

\begin{definer}[Enriched log category]
  Let $G$ be a group and $K$ a subgroup of $\aut{G}$, such that $K$
  contains $\inn{G}$. The $K$-enriched log category of $G$, denoted
  $\enrichedlogcategory{G}{K}$, is defined as follows. The objects
  comprise the following data:

  \begin{enumerate}

  \item A group $A$

  \item A homomorphism $\phi: K \to \aut{A}$ (viz., a $K$-action on $A$)

  \item A reflexive symmetric binary relation $\perp$ on $A$

  \end{enumerate}

  Such that the following are satisfied:

  \begin{enumerate}

  \item $\perp$ is $K$-invariant, viz:

    $$a \perp b \implies \phi(g)a \perp \phi(g)b$$
  \item For any $a$, the set of $b$ such that $a \perp b$ is a
    subgroup of $A$. This set shall be denoted as $a^\perp$.

  \item The set of $a$ such that $a \perp b$ for every $b \in A$, is
    the same as the set of $a$ for which $g.a = a$ for every $g
    \in \inn{G}$. This set will be denoted as $A^\perp$.

  \end{enumerate}

  We denote the objects as triples $(A,\phi,\perp)$. For $g \in G$, I
  will write $ga$ for $\phi(c_g)a$ where $c_g$ denotes the element of
  $\aut{G}$ defined as conjugation by $g$.

  If $A$ and $B$ are two objects, a morphism from $A$ to $B$ is a
  set-theoretic map $f: A \to B$ such that:

  \begin{enumerate}

  \item If $a_1 \perp a_2$, then $f(a_1) \perp f(a_2)$:

    $$f(a_1a_2) = f(a_1)f(a_2)$$

    In particular since $\perp$ is reflexive, we see that $f$ gives a
    homomorphism on any cyclic subgroup of $A$.

  \item $f$ commutes with the $K$-action, viz:

    $$f(ka) = kf(a)$$

    where $k \in K$.
  \end{enumerate}
\end{definer}

$K$-enrichment is essentially enlarging the group of automorphisms we
have to act on each object, while preserving the good properties we
had before. The best possible enrichment we could have is if $K = \aut{G}$;
the worst gives the ordinary log category.

\begin{definer}[Characteristic log category]
  The \definedind{characteristic log category} of a group $G$ is
  defined as $\enrichedlogcategory{G}{\aut{G}}$, viz., the log
  category enriched with all automorphisms of $G$. We'll denote the
  characteristic log category of $G$ by
  $\characteristiclogcategory{G}$.
\end{definer}

Note that $G$ is naturally an object in the characteristic log category,
where $\aut{G}$ acts on $G$ in the usual way.

\subsection{Additional structure in the characteristic log category}

We can talk of objects of group-type, ring-type, and Lie ring-type in
the characteristic log category of $G$. The definition is the same,
except that we now require that {\em all} the automorphisms of $G$ act
as automorphisms {\em preserving the additional structure}. Thus, an
object $A$ of ring-type in $\characteristiclogcategory{G}$ must
satisfy the property that any $\sigma \in \aut{G}$ acts on $A$
as a {\em ring} automorphism.

\subsection{Enriching and forgetting}

Suppose $\inn{G} \le K_1 \le K_2 \le \aut{G}$. Then, there is a
forgetful functor from $\enrichedlogcategory{G}{K_2}$ to
$\enrichedlogcategory{G}{K_1}$, which only remembers the $K_1$-action
on $G$. In particular, when $K_2 = \aut{G}$ and $K_1 = \inn{G}$, we
get a forgetful functor from the characteristic log category to the
log category.

A natural reverse question: given an object of the log category, what
are the ways in which it can be given the structure of an object in
the characteristic log category? In other words, how many ways are
there of extending the action we already have for $\inn{G}$, to an
action of $\aut{G}$? In general, there could be many ways, but if we
are given constraints (like certain maps must continue to remain
homomorphisms even after enrichment), these constraints may help
determine the additional structure.

\subsection{Characteristic logarithm}

\begin{definer}[Reversible characteristic logarithm]
  Let $G$ be a group. A \sdefinedproperty{logarithm}{reversible
    characteristic} for $G$ is a group $A \in
  \characteristiclogcategory{G}$ with a morphism $\exp:A \to G$
  which is an isomorphism in the characteristic log category.
\end{definer}

Any reversible characteristic logarithm, gives, via the forgetful
functor to the log category, a reversible logarithm. Conversely, given
a reversible logarithm, there exists at most one way of making it into
a reversible characteristic logarithm. The necessary and sufficient
condition is that the isomorphism must map all group automorphisms of $G$,
to group automorphisms of $A$.

\subsection{The Lazard correspondence}

\begin{theorem}[Lazard correspondence]
  Suppose $G$ is a $p$-group of nilpotence class at most $p-1$ (more
  generally, $G$ is a $p$-group where the subgroup generated by any
  three elements has nilpotence class at most $p-1$). Then $G$ admits a
  characteristic reversible logarithm of Lie ring-type.
\end{theorem}

\subsection{Fixed-point subgroups}

In general, it is {\em not} necessary that under an isomorphism in the
log category, subgroups go to subgroups. However, if $A \cong B$ as
elements of $\logcategory{G}$, then the subgroup of fixed points of $g
\in G$ in $A$, gets mapped to the subgroup of fixed points under $g$,
in $B$.

A more general version of this is given below.

\begin{lemma}[Fixed-point subgroups of automorphisms]
  Suppose $A \cong B$ as objects of
  $\enrichedlogcategory{G}{K}$. Then, for any subgroup $L$ of $K$, the
  set of fixed points in $A$ under $L$, gets mapped isomorphically to
  the set of fixed points in $B$ under $L$.
\end{lemma}

\section{Extending logarithms from normal subgroups}

\subsection{The aim}

The aim is as follows:

\begin{quote}
  Suppose we are given reversible logarithms for a family of subgroups
  of a group. Use them to construct a logarithm for the whole group.
\end{quote}

We shall achieve this, partially:

\begin{theorem}[Full logarithm in terms of logarithms for normal subgroups]\label{fulllogs}
  Suppose $G$ is generated by normal subgroups $N_1, N_2, \ldots,
  N_r$. Suppose $M_i$ is a reversible logarithm for $N_i$ enriched
  under the $G$-action on $M_i$ (in other words, it is enriched by the
  image of $G$ in $\aut{N_i}$ via the action by conjugation). Then,
  the group $M = M_1 \oplus M_2 \oplus \ldots \oplus M_r$ is a full ({\em
    not necessarily reversible}) logarithm for $G$.
\end{theorem}

In particular, this means that if $M_i$ is a {\em characteristic}
reversible logarithm for $N_i$, then $M$ is a full logarithm for $G$.

\begin{proof}[Description of $M$ as an object in $\logcategory{G}$]
  Let $\exp_i$ denote the exponential map from $M_i$ to $N_i$.
  
  To describe $M$ as an object in $\logcategory{G}$, we need to
  describe the $G$-action on $M$, and the relation $\perp$ on $M$:

  \begin{itemize}

  \item The $G$-action on $M$ is defined by the $G$-action on each $M_i$.
    Such an action exists because we assumed that each $M_i$ was $G$-enriched.

  \item The relation $\perp$ on $M$ is defined as follows:

    $$(a_1,a_2, \ldots, a_r) \perp (b_1,b_2, \ldots, b_r) \iff \exp(a_i)\exp(b_j) = \exp(b_j)\exp(a_i) \ \forall \ i,j \in \oneton{r}$$

  \end{itemize}

  We first need to show that under this relation, $a^\perp$ is a
  subgroup for any $a \in M$. To see this, observe that if $a \perp
  b$, then $\exp(b_1)$ commutes with $\exp(a_i)$ for every $i \in
  \oneton{r}$. But this is equivalent to saying that $b_1$ is a fixed
  point under the action by conjugation of $\exp(a_i)$ for all $i$
  (here we are using the fact that $M_1$ is a $G$-enriched logarithm
  for $N_1$). Thus, the set of possibilities for $b_1$ is precisely the
  subgroup of $M_1$ which is fixed under the action of each
  $\exp(a_i)$.  (we are implicitly using here the lemma for
  fixed-point subgroups of automorphisms).

  $a^\perp$ can be described as the direct sum of subgroups of $M_i$,
  each subgroup being the set of fixed points in $M_i$ under the
  action of all the $\exp(a_j)$s.

  To verify that this gives $M$ the structure of an object of
  $\logcategory{G}$, note that:

  \begin{itemize}

  \item Any element in the center of $G$ acts trivially by conjugation
    on each $N_i$, and hence, acts trivially by conjugation on each
    $M_i$ as well. 

  \item The only elements of $M$ that are fixed by {\em all} elements
    of $G$ are those whose $i^{th}$ coordinate is in the inverse image
    of $N_i \cap Z(G)$, for each $i$. Clearly, these are also the
    elements in $M^\perp$.
    
  \end{itemize}
\end{proof}

\begin{proof}[The exponential map]
  The map $\exp:M \to G$ is now defined as:

  $$\exp(a_1,a_2,\ldots,a_r) = \exp_1(a_1)\exp_2(a_2) \ldots \exp_r(a_r)$$

  {\em Proof of being a morphism}: The $G$-action is certainly
  preserved. Clearly, when $a,b \in M$ satisfy $a \perp b$, then
  $\exp(a)$ and $\exp(b)$ commute. Moreover, $\exp(ab) =
  \exp(a)\exp(b)$. The proof of both these facts relies on our being
  able to permute the $\exp_i(a_i)$ past the $\exp_j(b_j)$.

  {\em Proof of surjectivity}: First note that $\exp_i:M_i \to N_i$ is
  bijective. Also, the $N_i$ generate $G$, and hence, since the
  $N_i$s are {\em normal}, we have:

  $$G = N_1N_2\ldots N_r$$
\end{proof}

This result is not satisfactory from the inductive point of view, because:

\begin{itemize}

\item We require the logarithms we start with to be reversible, but
  the logarithm we get at the end is far from reversible.

\item We require the logarithms on the normal subgroups to be {\em
    enriched} by the action on the whole group, but the logarithm we
  obtain at the end is far from enriched. Thus, even if we start out
  with characteristic reversible logarithms, the logarithm we get on
  the whole group is unlikely to be either characteristic or reversible.

\item The reversible logarithms on the subgroups may have some
  additional structure. For instance, they may be of ring-type, or Lie
  ring-type. However, the logarithm we get for the whole group may not
  have any of the additional structure.

\end{itemize}

On the plus side, we do have the following:

\begin{corollary}
  If $G$ has a family of normal subgroups $N_i$ such that each $N_i$
  possesses a characteristic reversible Abelian logarithm, then $G$
  possesses a full Abelian logarithm (which need not be either
  characteristic or reversible).
\end{corollary}

\subsection{Relation with theorem 2 of Professor Glauberman's paper}

Here is the statement of Theorem 2 of Professor Glauberman's note:

\begin{theorem}\label{glaub2}
  Suppose $S$ is a finite $p$-group generated by a set $\S$ of normal
  subgroups $N$ of $S$ having nilpotence class at most $p-1$. Let
  $\mathfrak U$ be the set-theoretic union of the elements of $\S$.
  For each $N$ in $\S$, define $+$ on $N$ by Lazard's definition. For
  each $u,\,v$ in $\mathfrak U$, define $[u,\,v]$ as in Theorem A.
  
  Let $E=End(\S)$ be the set of all mappings $\phi$ from $\mathfrak U$
  to $\mathfrak U$ such that, for each $N$ in $\S,$
  \begin{center}
    $\phi$ maps $N$ into $N$ and induces an endomorphism of $N$ under
    $+$.
  \end{center}
  Define addition and multiplication on $E$ by
  $$(\phi+\phi')(x)=\phi(x)+\phi'(x)\quad\text{and}\quad(\phi\phi')(x)=\phi(\phi'(x)).$$
  Then $E$ forms an associative ring, and hence also a Lie ring under
  the definition $$\left[\phi,\phi'\right]=\phi\phi'-\phi'\phi.$$ For
  each $v$ in $\mathfrak U$, define a mapping $ad\ v$ on $\mathfrak U$
  by
  $$(ad\ v)(u)=[u,\,v].$$
  Then
  \begin{enumerate}
  \item[(i)] $ad\ v$ lies in $E$ for each $v$ in $\mathfrak U$,
  \item[(ii)] for each $N$ in $\S$ and each $v,\,w$ in $N$,
    $ad(v+w)=ad\ v+ad\ w,$
  \item[(iii)] for $v,\,w$ in $\mathfrak U$,
    $$[ad\ v, ad\ w]=ad\ [w,\,v]=-ad\ [v,\,w], \text{ and }$$
    $$ad\ v=ad\ w\text{ iff } v\equiv w\ (mod\ Z(S)),$$
  \item[(iv)] the additive subgroup $L(\S)$ of $E$ spanned by
    the mappings $ad\ v$ for $v$ in $\mathfrak U$ is a Lie subring of
    $E,$ and
  \item[(v)] for $L(\S)$ as in (d), each element of $\phi$ of
    $L(\S)$ satisfies
    $$\phi([u,\,v])=\left[\phi(u),\,v\right]+\left[u,\phi(v)\right], \text{ for every $u,\,v$ in $\mathfrak U.$}$$
  \end{enumerate}
\end{theorem}

The $S$ of this theorem plays the role of the group $G$ that we worked
with earlier, and the family $\S$ of normal subgroups is analogous to
$N_1, N_2, \ldots, N_r$ that we worked with.

What the above theorem does is to associate to the group $G$ a Lie
ring $E$, such that a large subset of $E$ (namely the union of the
normal subgroups) can be identified with a large subset of $G$;
however, there is no map either way from $E$ to $G$. The result has a
neat interpretation in terms of the log category:

\begin{enumerate}

\item $E$ has the structure of an object in $\logcategory{G}$. Namely,
  we can make $g \in G$ act on each $N \in \S$ in the usual way, and
  consider the generated action on $E$.

\item On the other hand, we can follow the prescription of the
  previous theorem, to construct a group $M$ which is a full logarithm
  for $G$. It turns out that we get this picture:

  \begin{diagram}
    & & M & & \\
    & \ldTo & & \rdTo & \\
    G & & & & E
  \end{diagram}

  where all three objects are viewed as living inside
  $\logcategory{G}$.  $E$ is an object of Lie ring-type, whereas $G$
  is an object of group-type.  $M$ does not come with any additional
  structure. The map on the right is the map discussed in the previous
  subsection; the map on the left is the natural projection from a direct sum
  to the subgroup generated.
\end{enumerate}

\section{Some particular cases}

In this section, we look at two examples in detail: the dihedral group
of size $8$ (acting on a $4$-element set) and the quaternion group,
which also has size $8$. Neither of these groups admits a reversible
Abelian logarithm, but both of these can be generated by normal Abelian subgroups.

\subsection{The quaternion group}

Let us begin with a group $Q$ of order eight called the quaternion
group. The quaternion group has eight elements:

$$\pm 1, \pm i, \pm j, \pm k$$

There are three normal cyclic subgroups with $4$ elements: the
subgroups generated by $i$, $j$ and $k$ respectively. Any two of these 
generates $Q$. Suppose $N_1$ and $N_2$ are the cyclic subgroups generated
by $i$ and $j$ respectively.

Now view $N_1$ and $N_2$ as characteristic logarithms over themselves,
and applying theorem \ref{fulllogs}, we see that the group $M = N_1 \oplus
N_2 = \Z/4\Z \oplus \Z/4\Z$, is a full logarithm for $Q$. Elements of $M$
look like $(i^\alpha,j^\beta)$ and the image of such an element in $Q$
is simply $i^\alpha j^\beta$. 

Note that in $M$, we do {\em not} identify the pair $(-1,1)$ with the pair $(1,-1)$,
even though their images in $Q$ are the same.

The group $E$ mentioned in Professor Glauberman's theorem (theorem
\ref{glaub2}) is a group obtained by taking $N_1 \oplus N_2$, and then
quotienting out by the common subgroup they both have: namely the
subgroup $\{ \pm 1 \}$. Thus $E$ is just like $Q$, except that we now
have $i + j = j + i$ instead of $ij = (-1)ji$.

The Lie bracket relations on $E$ are: $1$ and $-1$ are in the center; $[i,j]
= -1$,, $[j,i] = -1$.

The action of $Q$ on $E$ is described as follows: the action of $\pm
i$ fixes $\pm 1$ and $\pm i$, reverses the sign on $\pm j$, and
interchanges $i + j$ with $i - j$. A similar effect is achieved by the
action of $\pm j$, and the action of $\pm k$ is the composite of the two
effects.

\subsection{The dihedral group}

Consider now the dihedral group of eight elements. We denote the group
by $D$. $D$ has a normal cyclic subgroup or order $4$, and a normal
Abelian subgroup of order $4$ isomorphic to the Klein four-group. Let $r$
denote an element of order $4$, and $s$ denote an element of order $2$.
Our normal subgroups are:

\begin{eqnarray*}
  N_1 & = & \{ 1, r, r^2, r^3 \}\\
  N_2 & = & \{ 1, r^2, s, r^2s \}
\end{eqnarray*}

$r$ and $s$ do {\em not} commute: we have $rs = sr^3$.

The group $M = N_1 \oplus N_2$ is a group of order 16, abstractly
isomorphic to $\Z/4\Z \oplus \Z/2\Z \oplus \Z/2\Z$. The group $E$ is a
group of order $8$, obtained by quotienting out $M$ by identification
of the $r^2$ from $N_2$ with the $r^2$ from $N_1$. $E$ is just like
$D$, except that $rs = sr$ now.

The lie bracket relations on $E$ are: $r^2$ and $1$ are in the center
($1$ is in fact the $0$ of the Lie algebra), and $[r,s] = r^2$. 

The action of $D$ on $E$ (making $E$ into an object in the log
category of $D$) is as follows: the action of $r$ fixes $r$ and sends
$s$ to $r^2s$. Similarly, the action of $s$ fixes $s$ and sends $r$ to
$r^3$.

(this is confusing notation, because of the mix of additive and
multiplicative ideas).

\section{Logarithms for non-nilpotent groups}

\subsection{Symmetric groups}

We shall study the problem of whether logarithms exist for
non-nilpotent groups. While logarithms for nilpotent groups are
modelled on the exponential map from strictly upper triangular
matrices to upper triangular unipotent matrices, the problem of
logarithms for non-nilpotent groups is modelled on the more general
exponential map. We begin our study by looking at some groups which
are as far from nilpotent groups as possible: the symmetric groups.

Let's look at the symmetric group $S_3$. Consider the generating set
for $S_3$ comprising the $2$-cycles $s_3 = (1 2)$, $s_1 = (2 3)$ and
$s_2 = (1 3)$. $S_3$ acts faithfully on this set by conjugation.

Let $M$ be a group of order $8$, given as $M = \Z/2\Z \times \Z/2\Z
\times \Z/2\Z$. Let $t_1, t_2, t_3$ be the generators of the three
direct factors, and define a $S_3$-action on $M$ by the corresponding
action on the elements $s_1, s_2, s_3$. Define the relation $\perp$ on
$M$ as follows: $t_i^\perp$ is the two-element subgroup generated by
$s_i$, and $x^\perp$ is the trivial subgroup for any non-identity
element {\em not} among the $t_i$s.

The $S_3$-center of $M$ is trivial, and the $S_3$-derived subgroup of $M$
comprises the identity element, and the elements $t_1 + t_2$, $t_2 + t_3$,
and $t_3 + t_1$.

The map $\exp:M \to S_3$ is now defined as follows:

$$\exp(t_i) = s_i,  \qquad \exp(x) = e \ \forall \ x \ne t_1,t_2,t_3$$

This is an almost full morphism; it sends the center to the center,
and maps $M$ to a subset that generates $S_3$. Hence, $M$ is an
Abelian logarithm for $S_3$. Note, however, that $M$ is {\em not}
full, and it is very far from central.

\subsection{Additional structure on the logarithms}

We saw in the previous subsection that existence of generating sets of
a certain kind corresponds to existence of almost full
logarithms. This can better be seen by looking at the
associated root systems.

Let's now state the result precisely:

\begin{theorem}[Group generated by a union of conjugacy classes]
  Suppose $G$ is a finite group, generated by a union of conjugacy
  classes $C$, with elements $c_1, c_2, \ldots, c_d$ of orders $r_1,
  r_2, \ldots, r_d$. Then, consider the group:

  $$M = \Z/r_1\Z \times \Z/r_2\Z \times \ldots \times \Z/r_d\Z$$

  where $t_i$ is a generator of the $i^{th}$ direct factor. Then $M$
  can be given the structure of an object in $\logcategory{G}$, and
  there is a map $\exp:M \to G$, which makes $M$ a logarithm for $G$.
\end{theorem}

\begin{proof}
  We use the action of $G$ on $C$ by conjugation to get an action of
  $G$ on $M$, by identifying the $t_i$ with the corresponding $c_i$.

  Now give the structure $\perp$ on $M$ as follows. Let:

  $$a = \sum_i a_it_i, b = \sum_j b_jt_j$$

  We say that $a \perp b$ if whenever $i$ is such that $a_i \ne 0$ but
  $b_i = 0$, and $j$ is such that $a_j = 0$ but $b_j \ne 0$, then
  $c_ic_j = c_jc_i$. It is a routine check that this gives $M$ the
  structure of an object in the log category of $G$.

  Now consider the map $\exp:M \to G$ given by:

  $$\exp(a) = c_1^{a_1}c_2^{a_2} \ldots c_d^{a_d}$$

  {\em if} $c_i^{a_i}c_j^{a_j} = c_j^{a_j}c_i^{a_i}$ for any $i \ne
  j$. Otherwise:

  $$\exp(a) = e$$

  This gives $M$ the structure of a logarithm for $G$.
\end{proof}

Although the specific context here is different from the theorem on a
group generated by normal subgroups, the result is of a very similar
flavour, and a natural next step is to consider a generalization:

\begin{theorem}[Group generated by a union of conjugacy classes of subgroups]
  Suppose $G$ is generated by a family of subgroups $A_1, A_2, \ldots,
  A_d$, such that for any $A_i$, all the conjugates of $A_i$ are also
  in the family. Suppose $M_i$ is a characteristic reversible
  logarithm for $A_i$. Then the group:

  $$M = M_1 \oplus M_2 \oplus \ldots \oplus M_r$$

  is a logarithm for $G$, in a natural manner. The logarithm is full
  if all the $A_i$ are normal subgroups of $G$.
\end{theorem}

Before proceeding to the proof, we remark that %fillin %ref
the previous theorem is a special case of this where all the subgroups
are cyclic subgroups, and the theorem encountered earlier in this text
is a special case where all the subgroups are normal subgroups.

\begin{proof}
  
\end{proof}

\subsection{Particular corollary for root systems}

Although I have not formulated this precisely, I believe that we can
do approximately the following for any root system:

\begin{itemize}

\item Pick a root system $R$.

\item Take an elementary Abelian group $M$ of exponent $2$, obtained by taking
  as many copies of $\Z/2\Z$, as the number of roots in the root system

\item For nonzero elements $a, b \in M$, say $a \perp b$ if any root
  in $a$ which does {\em not} occur in $b$, is orthogonal to any root
  in $b$ which does {\em not} occur in $a$.

\item The Weyl group $W$ associated to $R$ acts naturally on the root
  system $R$ by permutation, and hence also on $M$. This action makes
  $M$ an object of $\logcategory{W}$.

\item Define a map $\exp:M \to W$ as follows. $\exp(a) = e$ if $a$
  contains any two non-orthogonal roots; otherwise, $\exp(a)$ is the
  product of generators corresponding to the roots in $a$ (these
  generators commute, since the roots are pairwise orthogonal). Then,
  $\exp$ is a morphism in $\logcategory{W}$, and makes $M$ a logarithm
  for $W$.

\end{itemize}

If the above is to be believed, the Weyl groups of root systems have
Abelian logarithms, in fact, they have elementary Abelian logarithms.
Similar constructions can be performed for other groups generated 

\printindex

\end{document}