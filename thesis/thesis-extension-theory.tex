%\chapter{Extension theory}

%\newpage

\section{Short exact sequences and central group extensions}\label{sec:ses-group}

Central extensions are fundamental to how we think about nilpotent
groups: nilpotent groups can be thought of as groups that can be
obtained by iteratively taking central extensions, starting with an
abelian group. The notion of central extension is also closely related
to the notion of isoclinism, although we will defer the explicit
connection for later. Central extensions are instrumental to
formulating and proving the generalizations that we develop in
Sections \ref{sec:bcuti} and \ref{sec:lcuti}.

In this section, we develop the rudimentary vocabulary of central
extensions.

\subsection{Definition of short exact sequence and group extension}

A {\em short exact sequence of groups} is an exact sequence of groups
with homomorphisms as follows:

$$1 \to A \to E \to G \to 1$$

In words, the homomorphism from $A$ to $E$ is injective, the
homomorphism from $E$ to $G$ is surjective, and the image of $A$ in
$E$ equals the kernel of the homomorphism from $E$ to $G$. The
standard abuse of notation identifies $A$ with its image in $E$ (so
$A$ is viewed as a normal subgroup of $E$) and $G$ with the quotient
group $E/A$.

We may also frame this as follows: $E$ is a {\em group extension} with
normal subgroup $A$ and quotient group $G$.

A {\em morphism of short exact sequences} from a short exact sequence:

$$1 \to A_1 \to E_1 \to G_1 \to 1$$

to another short exact sequence:

$$1 \to A_2 \to E_2 \to G_2 \to 1$$

is defined as a triple of homomorphisms $A_1 \to A_2$, $E_1 \to E_2$,
$G_1 \to G_2$, such that the following diagram commutes:

$$\begin{array}{rrrrr}
  1 & \to A_1 & \to E_1 & \to G_1 & \to 1 \\
  \downarrow & \downarrow & \downarrow & \downarrow & \downarrow\\
  1 & \to A_2 & \to E_2 & \to G_2 & \to 1 \\
\end{array}$$

Note that the arrows at the extremes do not convey useful information,
so the above is equivalent to asserting that the following diagram
commutes:

$$\begin{array}{rrrrr}
  1 & \to A_1 & \to E_1 & \to G_1 & \to 1 \\
  & \downarrow & \downarrow & \downarrow &\\
  1 & \to A_2 & \to E_2 & \to G_2 & \to 1 \\
\end{array}$$

We can compose two morphisms of short exact sequences in the obvious
way. In the diagram below, this corresponds to composing the vertical
morphisms:

$$\begin{array}{rrrrr}
  1 & \to A_1 & \to E_1 & \to G_1 & \to 1 \\
  & \downarrow & \downarrow & \downarrow & \\
  1 & \to A_2 & \to E_2 & \to G_2 & \to 1 \\
  & \downarrow & \downarrow & \downarrow & \\
  1 & \to A_3 & \to E_3 & \to G_3 & \to 1 \\
\end{array}$$

We can thus define a {\em category of short exact sequences}. An
{\em isomorphism of short exact sequences} is defined as an
isomorphism in this category. Explicitly, it is defined as a morphism
of short exact sequences where all the component homomorphisms are
isomorphisms.

\subsection{Group extensions with fixed base and quotient; congruence and pseudo-congruence}

We often study group extensions of the form:

$$1 \to A \to E \to G \to 1$$

where $A$ and $G$ are both fixed in advance and different
possibilities for $E$ are considered. Two group extensions:

$$1 \to A \to E_1 \to G \to 1$$

and

$$1 \to A \to E_2 \to G \to 1$$

are said to be {\em congruent} if there is an isomorphism $\varphi:
E_1 \to E_2$ such that the triple comprising the identity map $A \to
A$, the map $\varphi:E_1 \to E_2$, and the identity map $G \to G$ give
an isomorphism of short exact sequences. In other words, we can get an
isomorphism from $E_1$ to $E_2$ that induces the identity maps both on
the subgroup $A$ and the quotient group $G$.

Congruence defines an equivalence relation on the set of all
group extensions with normal subgroup $A$ and quotient group $G$. The
equivalence classes for this equivalence relation are termed {\em
  congruence classes}. 

Two group extensions:

$$1 \to A \to E_1 \to G \to 1$$

and

$$1 \to A \to E_2 \to G \to 1$$

are said to be {\em pseudo-congruent} if there is an isomorphism
between the short exact sequences. The isomorphism need not induce the
identity map on $A$ and need not induce the identity map on $G$.

Pseudo-congruence defines an equivalence relation on the set of all
group extensions with normal subgroup $A$ and quotient group $G$. The
equivalence classes for this equivalence relation are termed {\em
  pseudo-congruence classes}.

Pseudo-congruence is a coarser equivalence relation than congruence
because it allows for ``re-labeling'' on both the subgroup side and
the quotient group side.

\subsection{Abelian normal subgroups}

In the case that $A$ is an abelian group, the short exact sequence:

$$1 \to A \to E \to G \to 1$$

may also be written as

$$0 \to A \to E \to G \to 1$$

This is because when working with abelian groups, we denote the
trivial group by $0$ instead of $1$.

\subsection{Central extensions and stem extensions}\label{sec:central-and-stem-extension}

In this document, we use the term {\em central subgroup} for a
subgroup that is contained inside the center.

A {\em central extension} refers to a group extension where the
subgroup is central. Explicitly, consider a short exact sequence of
the following form, where $A$ is abelian:

$$0 \to A \to E \to G \to 1$$

We say that $E$ is a {\em central extension} with central subgroup $A$
and quotient group $G$ if the image of $A$ in $E$ is a central
subgroup of $E$. If we engage in the usual abuse of notation that
conflates $A$ with its image in $E$, we could shorten this to saying
that $A$ is a central subgroup of $E$.

We will often say ``$E$ is a central extension of $G$'' as shorthand
for ``there exists an abelian group $A$ such that $E$ is a central
extension with central subgroup $A$ and quotient group $G$.''

A {\em stem extension} is a central extension where the central
subgroup is also contained in the derived subgroup of the extension
group. Explicitly, consider a short exact sequence of the following
form, where $A$ is abelian:

$$0 \to A \to E \to G \to 1$$

We say that $E$ is a {\em stem extension} with central subgroup $A$
and quotient group $G$ if the image of $A$ in $E$ is contained in $E'
\cap Z(E)$.

\subsection{The use of cohomology groups to classify central extensions}\label{sec:second-cohomology-group-classify-extensions}

Suppose $A$ is an abelian group and $G$ is a group. The group
$H^2(G;A)$ (also denoted $H^2(G,A)$), called the {\em second
  cohomology group for trivial group action} of $G$ on $A$, is a group
whose elements correspond to the congruence classes of central
extensions with central subgroup $A$ and quotient group $G$. Here, by
``congruence class'' we mean equivalence class under the equivalence
relation of being congruent group extensions. The group structure on
$H^2(G;A)$ is not {\em prima facie} obvious. We will describe it in
detail in Section \ref{sec:cohomology-explicit}.

Further, there is a natural action of $\operatorname{Aut}(G) \times
\operatorname{Aut}(A)$ on $H^2(G;A)$ and the orbits of $H^2(G;A)$
under this natural action correspond precisely to the
pseudo-congruence classes of central extensions with central subgroup
$A$ and quotient group $G$.

Note that there is a more general definition of the second cohomology
group that works for non-central extensions where the action of the
quotient group $G$ on the abelian normal subgroup $A$ is
specified. Throughout this document, however, when referring to the
second cohomology group, we mean the second cohomology group for
trivial group action.

Basic background about the second cohomology group can be found in
\cite{Baer38}, \cite{Karpilovsky}, \cite{DummitFoote}, or in any
standard reference on group cohomology.

\subsection{Homomorphism of central extensions}\label{sec:homomorphism-central-extensions}

In the discussion so far, we have fixed both the normal subgroup $A$
and the quotient group $G$ and considered possibilities for the group
extension. We now consider the case where the quotient group $G$ is
fixed. We are interested in all central extensions with quotient group
$G$. The theory undergirding these should be hidden within the group
structure of $G$. Our goal is to make that theory more
explicit. Unfortunately, this is a long task, and we therefore only
include a first step here. We will pick up from where we leave here in
Section \ref{sec:exteriorsquare}.

We begin by defining the concept of {\em homomorphism of central
  extensions}. Consider two central extensions, both of which have $G$
as the quotient group:

$$0 \to A_1 \to E_1 \to G \to 1$$

and

$$0 \to A_2 \to E_2 \to G \to 1$$

A homomorphism of central extensions from the first central extension
to the second is a pair of homomorphisms $A_1 \to A_2$, $E_1 \to E_2$,
that, together with the identity map $G \to G$, give a homomorphism of
short exact sequences.

We can consider the {\em category of central extensions} of $G$:

\begin{itemize}
\item The {\em objects} of this category are the central extensions
  with quotient group $G$.
\item The {\em morphisms} of this category are homomorphisms of
  central extensions of $G$, as defined above.
\item Composition of morphisms is defined as the usual composition of
  homomorphisms of short exact sequences.
\end{itemize}

An object in the category of central extensions can be completely
described up to isomorphism in this category simply by specifying its
right map. Explicitly, consider two central extensions:

$$0 \to A_1 \to E \stackrel{\nu}{\to} G \to 1$$

and

$$0 \to A_2 \to E \stackrel{\nu}{\to} G \to 1$$

where the map $\nu$ is the same in both cases. In that case, the
central extensions are isomorphic in the category. Explicitly, this is
because if we consider the partial commutative diagram:

$$\begin{array}{rrrrr}
  0 & \to A_1 & \to E & \stackrel{\nu}{\to} G & \to 1 \\
  &  & \downarrow^{\text{id}} & \downarrow^{\text{id}} &\\
  0 & \to A_2 & \to E & \stackrel{\nu}{\to} G & \to 1 \\
\end{array}$$

there is a unique choice of isomorphism $A_1 \to A_2$ so that the
diagram as a whole commutes:

$$\begin{array}{rrrrr}
  0 & \to A_1 & \to E & \stackrel{\nu}{\to} G & \to 1 \\
  & \downarrow & \downarrow^{\text{id}} & \downarrow^{\text{id}} &\\
  0 & \to A_2 & \to E & \stackrel{\nu}{\to} G & \to 1 \\
\end{array}$$

Further, specifying a homomorphism from one object:

$$0 \to A_1 \to E_1 \stackrel{\nu_1}{\to} G \to 1$$

to another:

$$0 \to A_2 \to E_2 \stackrel{\nu_2}{\to} G \to 1$$

is equivalent to simply specifying the homomorphism $E_1 \to E_2$,
because the homomorphism $A_1 \to A_2$ is uniquely determined by
it. Explicitly, this is because in the commutative diagram:

$$\begin{array}{rrrrr}
  0 & \to A_1 & \to E_1 & \stackrel{\nu_1}{\to} G & \to 1 \\
  &  & \downarrow^{\theta} & \downarrow^{\text{id}} &\\
  0 & \to A_2 & \to E_2 & \stackrel{\nu_2}{\to} G & \to 1 \\
\end{array}$$

there is a unique morphism $A_1 \to A_2$ that completes the
commutative diagram.

Further, the set of permissible homomorphisms
$E_1 \to E_2$ is precisely the set of homomorphisms $\theta: E_1 \to
E_2$ such that $\nu_2 \circ \theta = \nu_1$.

The {\em category of central extensions} of $G$ thus has the following
alternative description. Note that strictly speaking, this is a
different category, but the preceding remarks establish that there is
a canonical {\em equivalence of categories} between the categories.

\begin{itemize}
\item The {\em objects} of the category are ``central extensions'' of
  $G$ in the sense of being pairs $(E,\nu)$ where $\nu:E \to G$ is a
  surjective group homomorphism and the kernel of $\nu$ is central in $G$.
\item Given two objects $(E_1,\nu_1)$ and $(E_2,\nu_2)$ in the
  category, a {\em morphism} between them is a homomorphism $\theta:
  E_1 \to E_2$ such that $\nu_2 \circ \theta = \nu_1$.
\end{itemize}

The equivalence of categories is given by the obvious forgetful
functor from the short exact sequence category to this new category,
that sends a short exact sequence $0 \to A \to E \stackrel{\nu}{\to} G
\to 1$ to the quotient map $E \stackrel{\nu}{\to} G$. The functor is
clearly essentially surjective (in fact, it is surjective on
objects). The preceding remarks establish that the functor is full and
faithful, and therefore an equivalence of categories. From this point
onward, we we talk of the ``category of central extensions of $G$'' we
will refer to the latter category.

We might hope that this category has an initial object, which could
then serve as the ``source'' classifying central extensions of
$G$. However, such an initial object does not always exist. We will
show in Section \ref{sec:freeinitialobject} that there {\em do} exist
objects in this category that admit homomorphisms to every other
object in the category. These are not in general initial objects
because the homomorphisms admitted are not unique.

%\newpage

\section{Short exact sequences and central extensions of Lie rings}\label{sec:ses-lie}

In this section, we develop the theory of short exact sequences of Lie
rings parallel to the development in the preceding section (Section
\ref{sec:ses-group}) of the theory for groups. The sections are
almost completely analogous and readers who have thoroughly understood
the preceding section can safely skip this section. The reasons behind
developing the theory are also analogous to those offered for groups.

The underlying theory for the cohomology group is different in
substantive ways for groups and Lie rings. However, these differences
do not show up at the level of abstraction at which we are dealing
with groups and Lie rings in this and the preceding section.

\subsection{Definition of short exact sequence and Lie ring extension}

A {\em short exact sequence of Lie rings} is an exact sequence of Lie
rings with homomorphisms as follows:

$$0 \to A \to N \to L \to 0$$

In words, the homomorphism from $A$ to $N$ is injective, the
homomorphism from $N$ to $L$ is surjective, and the image of $A$ in
$N$ equals the kernel of the homomorphism from $N$ to $L$. The
standard abuse of notation identifies $A$ with its image in $N$ (so
$A$ is viewed as an ideal of $N$) and $L$ with the quotient
Lie ring $N/A$.

We may also frame this as follows: $N$ is a {\em Lie ring extension}
with (base) ideal $A$ and quotient Lie ring $L$.

A {\em morphism of short exact sequences} from a short exact sequence:

$$0 \to A_1 \to N_1 \to L_1 \to 0$$

to another short exact sequence:

$$0 \to A_2 \to N_2 \to L_2 \to 0$$

is defined as a triple of homomorphisms $A_1 \to A_2$, $N_1 \to N_2$,
$L_1 \to L_2$, such that the following diagram commutes:

$$\begin{array}{rrrrr}
  0 & \to A_1 & \to N_1 & \to L_1 & \to 0 \\
  \downarrow & \downarrow & \downarrow & \downarrow & \\
  0 & \to A_2 & \to N_2 & \to L_2 & \to 0 \\
\end{array}$$

Note that the arrows at the extremes do not convey useful information,
so this is equivalent to saying that the following diagram commutes:

$$\begin{array}{rrrrr}
  0 & \to A_1 & \to N_1 & \to L_1 & \to 0 \\
  & \downarrow & \downarrow & \downarrow &\\
  0 & \to A_2 & \to N_2 & \to L_2 & \to 0 \\
\end{array}$$

We can compose two morphisms of short exact sequences in the obvious
way. In the diagram below, this corresponds to composing the vertical
morphisms:

$$\begin{array}{rrrrr}
  0 & \to A_1 & \to N_1 & \to L_1 & \to 0 \\
  & \downarrow & \downarrow & \downarrow & \\
  0 & \to A_2 & \to N_2 & \to L_2 & \to 0 \\
  & \downarrow & \downarrow & \downarrow & \\
  0 & \to A_3 & \to N_3 & \to L_3 & \to 0 \\
\end{array}$$

We can thus define a {\em category of short exact sequences}. An
{\em isomorphism of short exact sequences} is defined as an
isomorphism in this category. Explicitly, it is defined as a morphism
of short exact sequences where all the component homomorphisms are
isomorphisms.

\subsection{Lie ring extensions with fixed base and quotient; congruence and pseudo-congruence}

We often study Lie ring extensions of the form:

$$0 \to A \to N \to L \to 0$$

where $A$ and $L$ are both fixed in advance and different
possibilities for $N$ are considered. Two Lie ring extensions:

$$0 \to A \to N_1 \to L \to 0$$

and

$$0 \to A \to N_2 \to L \to 0$$

are said to be {\em congruent} if there is an isomorphism $\varphi:
N_1 \to N_2$ such that the triple comprising the identity map $A \to
A$, the map $\varphi:E_1 \to E_2$, and the identity map $L \to L$ give
an isomorphism of short exact sequences. In other words, we can get an
isomorphism from $N_1$ to $N_2$ that induces the identity maps both on
the ideal $A$ and the quotient Lie ring $L$.

Congruence defines an equivalence relation on the set of all Lie ring
extensions with ideal $A$ and quotient Lie ring $L$. The
equivalence classes for this equivalence relation are termed {\em
  congruence classes}.

Two Lie ring extensions:

$$0 \to A \to N_1 \to L \to 0$$

and

$$0 \to A \to N_2 \to L \to 0$$

are said to be {\em pseudo-congruent} if there is an isomorphism
between the short exact sequences. The isomorphism need not induce the
identity map on $A$ and need not induce the identity map on $L$.

Pseudo-congruence defines an equivalence relation on the set of all
Lie ring extensions with ideal $A$ and quotient Lie ring $L$. The
equivalence classes for this equivalence relation are termed {\em
  pseudo-congruence classes}.

Pseudo-congruence is a coarser equivalence relation than congruence
because it allows for ``re-labeling'' on both the ideal side and the
quotient Lie ring side.

\subsection{Central extensions and stem extensions}\label{sec:central-and-stem-extension-lie}

A {\em central extension} refers to a Lie ring extension where the
ideal is central. Explicitly, consider a short exact sequence of
the following form, where $A$ is an abelian Lie ring:

$$0 \to A \to N \to L \to 0$$

We say that $N$ is a {\em central extension} with central subring $A$
and quotient Lie ring $L$ if the image of $A$ in $N$ is a central
subring of $N$. If we engage in the usual abuse of notation that
conflates $A$ with its image in $N$, we could shorten this to saying
that $A$ is a central subring (or equivalently, central ideal) of $N$.

We will often say ``$N$ is a central extension of $L$'' as shorthand
for ``there exists an abelian Lie ring $A$ such that $N$ is a central
extension with central subring $A$ and quotient Lie ring $L$.''

A {\em stem extension} is a central extension where the central
subring is also contained in the derived subring of the extension
Lie ring. Explicitly, consider a short exact sequence of the following
form, where $A$ is abelian:

$$0 \to A \to N \to L \to 0$$

We say that $N$ is a {\em stem extension} with central subring $A$
and quotient Lie ring $L$ if the image of $A$ in $N$ is contained in $N'
\cap Z(N)$.

\subsection{The use of cohomology groups to classify central extensions}\label{sec:second-cohomology-lie-ring-classify-extensions}

Suppose $A$ is an abelian Lie ring and $L$ is a Lie ring. The group
$H^2_{\text{Lie}}(L;A)$, called the {\em second cohomology group for
  trivial Lie ring action} of $L$ on $A$, is a group whose elements
correspond to the congruence classes of central extensions with
central subring $A$ and quotient Lie ring $L$. Here, by ``congruence
class'' we mean equivalence class under the equivalence relation of
being congruent Lie ring extensions. The group structure on
$H^2_{\text{Lie}}(L;A)$ is not {\em prima facie} obvious. For a detailed discussion of the group structure, please refer to Weibel's book \cite{Weibel}.
$H^2_{\text{Lie}}(L;A)$ is simply denoted $H^2(L;A)$ in cases where
there is no potential for ambiguity with the cohomology group
describing group extensions.

Further, there is a natural action of $\operatorname{Aut}(L) \times
\operatorname{Aut}(A)$ on $H^2(L;A)$ and the orbits of
$H^2_{\text{Lie}}(L;A)$ under this natural action correspond precisely
to the pseudo-congruence classes of central extensions with central
subring $A$ and quotient Lie ring $L$.

Note that there is some potential for abuse of notation here, namely,
we often view $A$ both as an abelian {\em group} and as an abelian
{\em Lie ring}. From a pedantic perspective, it would be preferable to
use the $\exp$ and $\log$ functors to transition between the abelian
group and abelian Lie ring via the abelian Lie correspondence, as
described in Section \ref{sec:abelian-lie-correspondence}. However,
doing so would complicate our notation considerably, so we avoid it in
this section. Later, when applying the results here to the Baer
correspondence up to isoclinism as described in Section
\ref{sec:bcuti}, we will be more careful.

\subsection{Homomorphism of central extensions}\label{sec:homomorphism-central-extensions-lie}

In the discussion so far, we have fixed both the central subring $A$
and the quotient Lie ring $L$ and considered possibilities for the
extension Lie ring. We now consider the case where the quotient Lie
ring $L$ is fixed. We are interested in all central extensions with
quotient Lie ring $L$. The theory undergirding these should be hidden
within the internal structure of $L$ as a Lie ring. Our goal is to
make that theory more explicit.

We begin by defining the concept of {\em homomorphism of central
  extensions}. Consider two central extensions, both of which have $L$
as the quotient Lie ring:

$$0 \to A_1 \to N_1 \to L \to 0$$

and

$$0 \to A_2 \to N_2 \to L \to 0$$

A homomorphism of central extensions from the first central extension
to the second is a pair of homomorphisms $A_1 \to A_2$, $N_1 \to N_2$,
that, together with the identity map $L \to L$, give a homomorphism of
short exact sequences.

We can consider the {\em category of central extensions} of $L$:

\begin{itemize}
\item The {\em objects} of this category are the central extensions
  with quotient Lie ring $L$.
\item The {\em morphisms} of this category are homomorphisms of
  central extensions of $L$, as defined above.
\item Composition of morphisms is defined as the usual composition of
  homomorphisms of short exact sequences.
\end{itemize}

An object in the category of central extensions can be completely
described up to isomorphism in this category simply by specifying its
right map. Explicitly, consider two central extensions:

$$0 \to A_1 \to N \stackrel{\nu}{\to} L \to 0$$

and

$$0 \to A_2 \to N \stackrel{\nu}{\to} L \to 0$$

where the map $\nu$ is the same in both cases. In that case, the
central extensions are isomorphic in the category. Explicitly, this is
because if we consider the partial commutative diagram:

$$\begin{array}{rrrrr}
  0 & \to A_1 & \to N & \stackrel{\nu}{\to} L & \to 0 \\
  &  & \downarrow^{\text{id}} & \downarrow^{\text{id}} &\\
  0 & \to A_2 & \to N & \stackrel{\nu}{\to} L & \to 0 \\
\end{array}$$

there is a unique choice of isomorphism $A_1 \to A_2$ so that the
diagram as a whole commutes:

$$\begin{array}{rrrrr}
  0 & \to A_1 & \to N & \stackrel{\nu}{\to} L & \to 0 \\
  & \downarrow & \downarrow^{\text{id}} & \downarrow^{\text{id}} &\\
  0 & \to A_2 & \to N & \stackrel{\nu}{\to} L & \to 0 \\
\end{array}$$

Further, specifying a homomorphism from one object:

$$0 \to A_1 \to N_1 \stackrel{\nu_1}{\to} L \to 0$$

to another:

$$0 \to A_2 \to N_2 \stackrel{\nu_2}{\to} L \to 0$$

is equivalent to simply specifying the homomorphism $N_1 \to N_2$,
because the homomorphism $A_1 \to A_2$ is uniquely determined by
it. Explicitly, this is because in the commutative diagram:

$$\begin{array}{rrrrr}
  0 & \to A_1 & \to N_1 & \stackrel{\nu_1}{\to} L & \to 0 \\
  &  & \downarrow^{\theta} & \downarrow^{\text{id}} &\\
  0 & \to A_2 & \to N_2 & \stackrel{\nu_2}{\to} L & \to 0 \\
\end{array}$$

there is a unique morphism $A_1 \to A_2$ that completes the
commutative diagram.

Further, the set of permissible homomorphisms
$N_1 \to N_2$ is precisely the set of homomorphisms $\theta: N_1 \to
N_2$ such that $\nu_2 \circ \theta = \nu_1$.

The {\em category of central extensions} of $L$ thus has the following
alternative description. Note that strictly speaking, this is a
different category, but the preceding remarks establish that there is
a canonical {\em equivalence of categories} between the categories:

\begin{itemize}
\item The {\em objects} of the category are ``central extensions'' of
  $L$ in the sense of being pairs $(N,\nu)$ where $\nu:N \to L$ is a
  surjective Lie ring homomorphism and the kernel of $\nu$ is central
  in $L$.
\item Given two objects $(N_1,\nu_1)$ and $(N_2,\nu_2)$ in the
  category, a {\em morphism} between them is a homomorphism $\theta:
  N_1 \to N_2$ such that $\nu_2 \circ \theta = \nu_1$.
\end{itemize}

The equivalence of categories is given by the obvious forgetful
functor from the short exact sequence category to this new category,
that sends a short exact sequence $0 \to A \to N \stackrel{\nu}{\to} L
\to 0$ to the quotient map $N \stackrel{\nu}{\to} L$. The functor is
clearly essentially surjective (in fact, it is surjective on
objects). The preceding remarks establish that the functor is full and
faithful, and therefore an equivalence of categories. From this point
onward, we we talk of the ``category of central extensions of $L$'' we
will refer to the latter category.

We might hope that this category has an initial object, which could
then serve as the ``source'' classifying central extensions of
$L$. However, such an initial object does not always exist. We will
show in Section \ref{sec:freeinitialobject-lie}, there {\em do} exist
objects in this category that admit homomorphisms to every other
object in the category. These are not in general initial objects
because the homomorphisms admitted are not unique.

%\newpage

\section{Explicit description of second cohomology group using the bar resolution}\label{sec:cohomology-explicit}

In Section \ref{sec:second-cohomology-group-classify-extensions}, we
stated that the second cohomology group for trivial group action
$H^2(G;A)$ is a group whose elements correspond with congruence
classes of central extensions with central subgroup $A$ and quotient
group $G$. However, we did not specify the group structure at the
time. In this section, we explicitly construct $H^2(G;A)$ as a group.

Interested readers can learn more from \cite{Baer38},
\cite{Karpilovsky}, \cite{DummitFoote}, or any standard reference on
group cohomology.


%% In the Appendix, we provide a general description of all homology and
%% cohomology groups. We also discuss the corresponding description for
%% Lie rings, which is somewhat more complicated. None of this material
%% is critical to our main proofs.

\subsection{Explicit description of second cohomology group using cocycles and coboundaries}

Suppose $G$ is a group and $A$ is an abelian group. A {\em 2-cochain
  for trivial group action} of $G$ on $A$ is defined as a set map $f:
G \times G \to A$. With pointwise addition of functions, the set of
$2$-cochains acquires an abelian group structure. We denote this group
as $C^2(G;A)$.

A 2-cochain $f:G \times G \to A$ is termed a {\em 2-cocycle for
  trivial group action} if it satisfies the following condition:

$$f(g_1,g_2) + f(g_1g_2,g_3) = f(g_1,g_2g_3) + f(g_2g_3) \ \forall \ g_1,g_2,g_3 \in G$$

The 2-cocycles form a subgroup of $C^2(G;A)$. This subgroup is denoted
$Z^2(G;A)$.

A 2-cochain $f:G \times G \to A$ is termed a {\em 2-coboundary for
  trivial group action} if there exists a set map $\varphi:G \to A$
such that:

$$f(g_1,g_2) = \varphi(g_1) + \varphi(g_2) - \varphi(g_1g_2)\ \forall \ g_1,g_2 \in G$$

Every 2-coboundary is a 2-cocycle, and the 2-coboundaries form a
subgroup of the group of 2-cocycles. We denote this subgroup as
$B^2(G;A)$. The group $H^2(G;A)$, called the {\em second cohomology
  group for trivial group action} of $G$ on $A$, is defined as the
quotient group $Z^2(G;A)/B^2(G;A)$ (note that both are subgroups of
the abelian group $C^2(G;A)$, hence $B^2(G;A)$ is normal in
$Z^2(G;A)$). The elements of $H^2(G;A)$, i.e., the {\em cosets} of
$B^2(G;A)$ in $Z^2(G;A)$, are termed {\em cohomology classes}. Given
two elements of $Z^2(G;A)$ that are in the same cohomology class, we
will say that they are {\em cohomologous} to each other.

We will reconcile this definition
with the earlier definition from Section \ref{sec:second-cohomology-group-classify-extensions} in Section \ref{sec:explicit-cocycle-description-of-extension}.

\subsection{Functoriality and automorphism group action}

Each of $C^2$, $Z^2$, $B^2$, and $H^2$, viewed in terms of $G$ and
$A$, is {\em contravariant} in the first argument $G$ and {\em
  covariant} in the second argument $A$. Explicitly:

\begin{itemize}
\item If $\theta:G_1 \to G_2$ is a homomorphism, then $\theta$ induces
  a homomorphism $C^2(G_2;A) \to C^2(G_1;A)$ by composition: given a
  map $f:G_2 \times G_2 \to A$ that is an element of $C^2(G_2;A)$, its
  image in $C^2(G_1;A)$ is the map $(x,y) \mapsto
  f(\theta(x),\theta(y))$. This homomorphism restricts to
  homomorphisms $Z^2(G_2;A) \to Z^2(G_1;A)$ and $B^2(G_2;A) \to
  B^2(G_1;A)$. Thus, it also induces a homomorphism $H^2(G_2;A) \to
  H^2(G_1;A)$. All these induced homomorphisms define contravariant
  functors.
\item If $\alpha:A_1 \to A_2$ is a homomorphism, then $\alpha$ induces
  a homomorphism $C^2(G;A_1) \to C^2(G;A_2)$ by composition: $f
  \mapsto \alpha \circ f$. This homomorphism restricts to
  homomorphisms $Z^2(G;A_1) \to Z^2(G;A_2)$ and $B^2(G;A_1) \to
  B^2(G;A_2)$. Thus, it also induces a homomorphism $H^2(G;A_1) \to
  H^2(G;A_2)$. All these induced homomorphisms define covariant
  functors.
\end{itemize}

Based on this functoriality, we obtain a natural action of
$\operatorname{Aut}(G) \times \operatorname{Aut}(A)$ on each of the
groups $C^2(G;A)$, $Z^2(G;A)$, $B^2(G;A)$, and $H^2(G;A)$. We compose
on both sides. Note that, due to {\em contravariance} in the
$G$-argument, we need to use the inverse of the element on the
$\operatorname{Aut}(G)$ side to keep the action a left
action. Explicitly, $(\varphi,\alpha) \circ f$ is defined as:

$$(x,y) \mapsto \alpha(f(\varphi^{-1}(x),\varphi^{-1}(y)))$$

\subsection{Identifying cohomology classes with congruence classes of central extensions}\label{sec:explicit-cocycle-description-of-extension}

We will now describe a bijection:

Elements of the second cohomology group $H^2(G;A)$ $\leftrightarrow$
Congruence classes of central extensions with central subgroup $A$ and
quotient group $G$

We will describe the bijection in the reverse direction. Explicitly,
given a central extension group $E$, we will describe how to use $E$
to obtain a cohomology class.

We have the short exact sequence:

$$0 \to A \stackrel{\iota}{\to} E \stackrel{\nu}{\to} G \to 1$$

Pick any set map $s:G \to E$ that is a one-sided inverse to the
surjective homomorphism $\nu: E \to G$ (such a set map is called a
{\em section} of the extension). We can think of $s$ as specifying the
coset representatives in $E$ for each element of $G$.

Now, define the following 2-cochain $f: G \times G \to A$: for
$g_1,g_2 \in G$, consider the element of $E$ given by
$s(g_1g_2)(s(g_1)s(g_2))^{-1}$. This element of $E$ maps to the
identity element of $G$, hence is in the image of $A$. Define
$f(g_1,g_2)$ to be its inverse image under $\iota$ in $A$.

Explicitly:

$$s(g_1g_2) = \iota(f(g_1,g_2))s(g_1)s(g_2) \ \forall \ g_1,g_2 \in G$$

We can think of $f$ as measuring the extent to which $s$ fails to be a
homomorphism. Since $\nu \circ s$ is the identity map, $s$ is a
homomorphism {\em modulo} $A$. The ``error term'' for $s$ therefore
lies in $A$, and this is how we get a 2-cochain $f: G \times G \to A$.

We can now verify the following:

\begin{itemize}
\item The function $f$ constructed for any section $s:G \to E$ is a
  2-cocycle, i.e., an element of $Z^2(G;A)$. This follows from
  associativity of group multiplication. Explicitly, if we expand
  $s(g_1g_2g_3)$ using the two different ways of associating the
  expression, and compare, we get the result. Note that we need to use
  the centrality of $\iota(A)$ in $E$ to commute elements.
\item The set of all possible functions $f$ that we can get by
  choosing different sections $s :G \to E$ for a {\em single}
  extension $E$ correspond to a single cohomology class, i.e., a
  single coset of $B^2(G;A)$ in $Z^2(G;A)$, and therefore, a single
  element of $H^2(G;A)$.  
\item Two central extensions with central subgroup $A$ and quotient
  group $G$ are congruent if and only if they give the same element of
  $H^2(G;A)$.
\end{itemize}

We have now completely described one direction of the
correspondence. The construction in the other direction is similar: we
need to explicitly construct a group extension based on a cohomology
class. We will omit the details, but they can be found in
\cite{DummitFoote} or in any of the standard references on cohomology.

It also follows from the above that the orbits of $H^2(G;A)$ under the
$\operatorname{Aut}(G) \times \operatorname{Aut}(A)$ action correspond
with the pseudo-congruence classes of extensions.

\subsection{Short exact sequence of coboundaries, cocycles, and cohomology classes}\label{sec:ses-coboundary-cocycle-cohomology}

We have a natural short exact sequence:

$$0 \to B^2(G;A) \to Z^2(G;A) \to H^2(G;A) \to 0$$

This short exact sequence does not always split. For instance,
consider the case where $G = \mathbb{Z}/2\mathbb{Z}$ and $A =
\mathbb{Z}$. In this case, $C^2(G;A)$ is, as an abelian group,
isomorphic to $A^{|G| \times |G|}$, which is $\mathbb{Z}^4$. In
particular, it is a finitely generated free abelian group. Thus, both
$B^2(G;A)$ and $Z^2(G;A)$, which are subgroups of $C^2(G;A)$, are also
free abelian groups. However, $H^2(G;A) \cong \mathbb{Z}/2\mathbb{Z}$, %%TONOTDO: Insert a reference
which is not a free abelian group. If the short exact sequence {\em
  did} split, then $H^2(G;A)$ would have been free abelian. Therefore,
the short exact sequence does not split.

In Section \ref{sec:baer-correspondence-cocycle-level}, we will see
that in the special case that $G$ and $A$ are both $2$-powered abelian
groups, this short exact sequence splits {\em canonically}.

%\newpage

\section{Exterior square, Schur multiplier, and homoclinism}\label{sec:exteriorsquare-and-homoclinism}

In Sections \ref{sec:ses-group} and \ref{sec:ses-lie}, we introduced,
for groups and for Lie rings, the notion of central extension. In this
section, our goal is to understand, for a group $G$, the category of
central extensions of $G$ (introduced in Section
\ref{sec:homomorphism-central-extensions-lie}) in terms of the
isomorphism type of $G$.

We will attack this question by looking at one key attribute of the
extension: how the commutator map behaves. Roughly speaking, the
behavior of the commutator map classifies the central extension {\em
  up to isoclinism}, and if we want to study the collection of central
extensions focusing {\em only} on this attribute, we consider the {\em
  category of central extensions with homoclinisms}, a variant of the
category of central extensions with homomorphisms. We will establish
key features of this category, including the fact that there is at
most one morphism between any two objects, and the existence of
initial objects. There are two structures in particular that store a
lot of the information related to the central extensions of $G$. These
are the exterior square $G \wedge G$ (which serves as a source object
for the derived subgroups in all central extensions) and the Schur
multiplier $M(G)$ (which is the kernel of the canonical map $G \wedge
G \to [G,G]$).

We tangentially mention in this section the well-known fact that the
Schur multiplier $M(G)$ has an alternative description as the second
homology group $H_2(G;\mathbb{Z})$. We do not provide a proof here,
since developing the underlying machinery of homology would take us
too far afield. However, the techniques used to establish this are
closely related to the explicit description of the second cohomology
group in Section \ref{sec:cohomology-explicit}. For reference, see the
exact sequence appearing as (2.8) in Loday and Brown's 1987 paper
\cite{BrownLoday}.

\subsection{Exterior square}\label{sec:exteriorsquare}

The exterior square of a group was originally considered (though not
with that name) in the paper \cite{Miller52} by Clair Miller in
1952. It was later defined as a special case of a more general concept
called the {\em exterior product of groups} in \cite{BrownLoday}. More
information about the exterior square and related constructions can be
found in \cite{McDermottThesis} and \cite{Ellis87}.

The definition that we provide here for the exterior square is the
``abstract'' definition. We will provide a concrete definition (based
on generators and relations) in Section
\ref{sec:exteriorsquare-explicit}. The equivalence of the two
approaches is discussed in Miller's original paper, and we include
some further discussion of the equivalence in Section
\ref{sec:exteriorsquare-reconciliation}. As we demonstrate in this
section, however, the initial theory is best established using the
abstract definition.

Suppose $G$ is a group. The {\em exterior square} of $G$, denoted by
$G \wedge G$, is defined as follows. Let $\mathcal{F}$ be the free
group on the set $G \times G$. 

For any central extension:

$$0 \to A \to E \to G \to 1$$

there is a set map:

$$\omega_{E,G}: G \times G \to [E,E]$$

given by:

$$\omega_{E,G}(x,y) = [\tilde{x},\tilde{y}]$$

where $\tilde{x}$ denotes any element of $E$ that maps to $x$ and
$\tilde{y}$ denotes any element of $E$ that maps to $y$. Note that the
map is well defined (i.e., it does not depend on the choice of the
lifts $\tilde{x}$ and $\tilde{y}$) because the extension is a central
extension.

Note that the set map $\omega_{E,G}$ described here differs from
$\omega_E$ in the following important respect: $\omega_{E,G}$ is a map
from $G \times G$, whereas $\omega_E$ is a map from $E/Z(E) \times
E/Z(E)$. However, it is obvious that $\omega_{E,G}$ {\em factors
  through} $\omega_E$.

$\mathcal{F}$ is the free group on $G \times G$, so $\omega_{E,G}$ gives rise to a group homomorphism:

$$\hat{\omega}_{E,G}:\mathcal{F} \to [E,E]$$

Note also that this homomorphism is {\em surjective}, because by
definition, $[E,E]$ is the subgroup of $E$ generated by the image of the
set map $\omega_{E,G}$.

Define $\mathcal{R}$ as the intersection of the kernels of all such
homomorphisms $\hat{\omega}_{E,G}$ where $E$ varies over all central
extension groups with quotient group $G$. Note that even though the
collection of all such homomorphisms is too large to be a set, the
collection of possible kernels is a set, so the intersection is well
defined. In other words, $\mathcal{R}$ is the set of all products of
formal pairs of elements and their inverses such that the
corresponding commutator words become trivial in every central
extension of $G$. We define the exterior square $G \wedge G$ as the
quotient group $\mathcal{F}/\mathcal{R}$. The image of $(x,y)$ in the
group is denoted $x \wedge y$.

It is clear from the definition that, for any central
extension $E$ with short exact sequence:

$$0 \to A \to E \to G \to 1$$

there exists a unique natural homomorphism $\Omega_{E,G}$ from $G \wedge G$ to
$[E,E]$ satisfying the condition that for any $x,y \in G$ we have:

$$\Omega_{E,G}(x \wedge y) = [\tilde{x},\tilde{y}]$$

where $\tilde{x}$ and $\tilde{y}$ are elements of $E$ that map to $x$
and $y$ respectively. Note also that $\Omega_{E,G}$ is {\em
  surjective}.

As a special case of the above, there is a natural homomorphism:

$$G \wedge G \to [G,G]$$

given on a generating set by:

$$x \wedge y \mapsto [x,y]$$

The kernel of this homomorphism is called the {\em Schur multiplier}
of $G$ and is denoted $M(G)$. We can easily deduce that $M(G)$ is a
central subgroup of $G \wedge G$. We thus have a short exact sequence:

$$0 \to M(G) \to G \wedge G \to [G,G] \to 1$$

There are numerous other definitions of the Schur multiplier. The most
common textbook definition is that $M(G) = H_2(G;\mathbb{Z})$, i.e.,
it is the second homology group for trivial group action with
coefficients in the integers. \cite{Karpilovsky} has a detailed
description of techniques to compute the Schur multiplier for finite
groups. \cite{BrownLoday} and \cite{McDermottThesis} provide
background on why the two definitions of Schur multiplier agree. In
particular, see the exact sequence appearing as (2.8) in Loday and
Brown's 1987 paper \cite{BrownLoday}.

\subsection{The existence of a single central extension that realizes the exterior square}\label{sec:grandcentralproduct}

Consider a group $G$. A natural question is whether there exists a
central extension group $E$ with quotient group $G$ with the property
that the natural homomorphism:

$$\Omega_{E,G}: G \wedge G \to [E,E]$$

is an isomorphism.

The answer to this question is {\em yes}. We provide one construction
below. We will provide another construction in
Section \ref{sec:freeinitialobject}.

Recall the earlier description of $G \wedge G$ as a quotient
$\mathcal{F}/\mathcal{R}$. The normal subgroup $\mathcal{R}$ was
defined as the intersection of all possible normal subgroups arising
as kernels of the natural homomorphisms $\mathcal{F} \to [E,E]$ for a
central extension group $E$. For each possible normal subgroup $N_i, i
\in I$ of $\mathcal{F}$ that arises this way, let $E_i$ denote a
corresponding central extension of $G$.

Define $E_0$ to be the pullback (also called the {\em fiber product}
or the {\em subdirect product}) corresponding to all the quotient maps
$E_i \to G$. We can verify that the natural mapping:

$$\mathcal{F} \to [E_0,E_0]$$

has kernel precisely $\mathcal{R}$, and hence, the mapping:

$$G \wedge G \to [E_0,E_0]$$

is an isomorphism.

\subsection{Homoclinism of central extensions}\label{sec:homoclinism-central-extensions}

Suppose $G$ is a group. We define a certain category for which we are
interested in computing the initial object. We will call this category
the {\em category of central extensions of $G$ with homoclinisms}. Explicitly,
the objects of the category are short exact sequences of the form:

$$0 \to A \to E \to G \to 1$$

where the image of $A$ is central in $E$.

The morphisms in the category, which we call {\em homoclinisms of
  central extensions}, are defined as follows. For two objects:

$$0 \to A_1 \to E_1 \to G \to 1$$

and

$$0 \to A_2 \to E_2 \to G \to 1$$

a morphism from the first to the second is a group homomorphism
$\varphi: E_1' \to E_2'$ between the derived subgroups $E_1' =
[E_1,E_1]$ and $E_2' = [E_2,E_2]$ such that the following holds. Let
$\omega_1: G \times G \to E_1'$ denote the map arising from the
commutator map in $E_1$ and let $\omega_2: G \times G \to E_2'$ denote
the corresponding map in $E_2$. We require that $\varphi \circ
\omega_1 = \omega_2$ as set maps.

The above condition can be reframed in terms of group homomorphisms if
we use the exterior square: let $\Omega_1: G \wedge G \to E_1'$,
$\Omega_2: G \wedge G \to E_2'$ denote the natural homomorphisms
described in Section \ref{sec:exteriorsquare}. The condition we need
is that $\varphi \circ \Omega_1 = \Omega_2$.

\subsection{Relation between the category of central extensions and the category of central extensions with homoclinisms}

Any {\em homomorphism} of central extensions induces a {\em
  homoclinism} of central extensions. Explicitly, consider two central
extensions:

$$0 \to A_1 \to E_1 \stackrel{\nu_1}{\to} G \to 1$$

and

$$0 \to A_2 \to E_2 \stackrel{\nu_2}{\to} G \to 1$$

As discussed in Section \ref{sec:homomorphism-central-extensions}, the
central extensions are completely described by the pairs $(E_1,\nu_1)$
and $(E_2,\nu_2)$ respectively. A homomorphism of central extensions
can be specified as a homomorphism $\theta:E_1 \to E_2$ satisfying the
condition that $\nu_2 \circ \theta = \nu_1$.

Any such homomorphism of central extensions induces a {\em
  homoclinism} of central extensions. Explicitly, for a homomorphism
$\theta: E_1 \to E_2$ satisfying $\nu_2 \circ \theta = \nu_1$, define
$\varphi$ as the homomorphism $E_1' \to E_2'$ obtained by restricting
$\theta$ to $E_1'$. We claim that $\varphi$ defines a homoclinism of
the central extensions. We now prove that this construction works.

\begin{lemma}\label{lemma:homomorphism-restriction-homoclinism}
  Suppose $(E_1,\nu_1)$ and $(E_2,\nu_2)$ are central extensions of a
  group $G$, and $\theta:E_1 \to E_2$ is a homomorphism of central
  extensions, i.e., $\nu_2 \circ \theta = \nu_1$. Denote by $\omega_1:
  G \times G \to E_1'$ and $\omega_2: G \times G \to E_2'$ the maps
  induced by the commutator maps in $E_1$ and $E_2$ respectively. Let
  $\varphi:E_1' \to E_2'$ be the homomorphism obtained by restricting
  $\theta$ to the derived subgroup $E_1'$. Then, $\varphi$ is a
  homoclinism of central extensions, i.e., $\varphi \circ \omega_1 =
  \omega_2$.
\end{lemma}

\begin{proof}
  Let $u$, $v$ be arbitrary elements of $G$ (possibly equal, possibly
  distinct). Our goal is to show that:

  $$\varphi(\omega_1(u,v)) = \omega_2(u,v)$$

  Let $x, y \in E_1$ be elements such that $\nu_1(x) = u$ and
  $\nu_1(y) = v$. Then, by definition, $\omega_1(u,v) =
  [x,y]$. Simplify the left side:

  $$\varphi(\omega_1(u,v)) = \varphi([x,y]) = \theta([x,y]) = [\theta(x),\theta(y)] = \omega_2(\nu_2(\theta(x)),\nu_2(\theta(y)))$$

  Now, use that $\nu_2 \circ \theta = \nu_1$ and simplify further to:

  $$\omega_2(\nu_1(x),\nu_1(y)) = \omega_2(u,v)$$

  which is the right side.
\end{proof}

\subsection{Uniqueness of homoclinism if it exists}

We will show that if a homoclinism exists between two central
extensions, it must be unique.

\begin{lemma}\label{lemma:uniqueness-of-homoclinism}
  Consider two short exact sequences that give central extensions of a group $G$:

  $$0 \to A_1 \to E_1 \to G \to 1$$

  $$0 \to A_2 \to E_2 \to G \to 1$$

  Denote by $\omega_1:G \times G \to E_1'$ and $\omega_2:G \times G
  \to E_2'$ the commutator maps.

  Suppose there exists homoclinisms $\varphi,\theta$ from the first
  central extension to the second. Explicitly, $\varphi:E_1' \to E_2'$
  and $\theta:E_1' \to E_2'$ are homomorphisms such that $\varphi
  \circ \omega_1 = \omega_2$ and $\theta \circ \omega_1 =
  \omega_2$. Then, $\varphi = \theta$.
\end{lemma}

\begin{proof}
  Denote by $\nu_1$ the quotient map $E_1 \to G$ and by $\nu_2$ the
  quotient map $E_2 \to G$.

  It will suffice to show that $\varphi$ and $\theta$ agree with each
  other on the set of all commutators, which is a generating set for
  $E_1'$. Consider a commutator $[x,y]$ with $x,y \in E_1$. Let $u =
  \nu_1(x)$ and $v = \nu_1(y)$. 

  By definition, $[x,y] = \omega_1(u,v)$. Thus, $\varphi([x,y]) =
  \varphi(\omega_1(u,v)) = \omega_2(u,v)$. Similarly, $\theta([x,y]) =
  \theta(\omega_1(u,v)) = \omega_2(u,v)$. We thus obtain that
  $\varphi([x,y]) = \theta([x,y])$, completing the proof.
\end{proof}

Thus, if a homoclinism exists, it is unique. However, a homoclinism
need not exist. The obstruction occurs if there are relations within
the derived subgroup $E_1'$ such that the corresponding relations are
not valid in the derived subgroup $E_2'$.

\subsection{Existence and description of initial objects in the category of central extensions up to homoclinisms}

In category theory, an object $X$ in a category $\mathcal{C}$ is
termed an initial object if for every object $Y \in \mathcal{C}$,
there is a unique morphism from $X$ to $Y$, sometimes called the {\em
  initial morphism}. It can easily be seen using ``abstract
nonsense''\footnote{``Abstract nonsense'' is a non-derogatory term
  used to refer to proof methods that rely on formalistic ideas,
  typically from category theory, rather than on the specifics of the
  structures being studied. Statements proved using abstrcat nonsense
  are often very general.} that if $X_1$ and $X_2$ are both initial
  objects in a category $\mathcal{C}$, then there exists a unique
  isomorphism between $X_1$ and $X_2$. In other words, the initial
  object in a category is uniquely determined up to (unique)
  isomorphism.

We are interested in identifying the initial objects in the category
of extensions of $G$ with homoclinisms discussed in Section
\ref{sec:homoclinism-central-extensions}.

\begin{lemma}[Existence and description of initial objects]\label{lemma:initobj}
  For a group $G$, consider the category of central extensions of $G$
  with homoclinisms. The following are true for this category.
  \begin{enumerate}
  \item There exists a central extension $E_0$ of $G$ for which the
    natural homomorphism $\Omega_0: G \wedge G \to E_0'$ is an isomorphism.
  \item Any central extension $E_0$ of $G$ for which the natural
    homomorphism $\Omega_0:G \wedge G \to E_0'$ is an isomorphism is
    an initial object of the category.
  \item If a central extension $E_1$ of $G$ is an initial object of
    the category, the corresponding homomorphism $\Omega_1:G \wedge G
    \to E_1'$ is an isomorphism.
  \item Combining all the above: the category of central extensions of
    $G$ with homoclinisms admits initial objects, and a central
    extension $E \to G$ is an initial object for the category if and
    only if the commutator map homomorphism $\Omega_{E,G}: G \wedge G
    \to [E,E]$ is an isomorphism.
  \end{enumerate}
\end{lemma}

\begin{proof}
  \begin{itemize}
  \item Proof of (1): In Section \ref{sec:grandcentralproduct}, we
    constructed a central extension group $E_0$ for which the natural
    map $\Omega_0: G \wedge G \to E_0'$ is an isomorphism.

  \item Proof of (2): Suppose $E_0$ is a central extension for which
    the commutator map $\Omega_0:G \wedge G \to E_0'$ is an
    isomorphism. For any central extension $E_2$, there is a natural
    homomorphism $\Omega_2: G \wedge G \to E_2'$. Composing this with
    the inverse of $\Omega_0$, we obtain a homomorphism
    $\varphi:E_0' \to E_2'$ that defines a homoclinism of the
    extensions. Moreover, by Lemma
    \ref{lemma:uniqueness-of-homoclinism}, this is the {\em unique}
    homoclinism of the extensions.

  \item Proof of (3): We already know of the existence of a central
    extension $E_0$ for which $\Omega_0:G \wedge G \to E_0'$ is
    an isomorphism by (1). We also know that it is an initial object
    by (2). By the uniqueness of initial objects up to isomorphism,
    $E_0$ and $E_1$ are isomorphic in the category of central
    extensions of $G$ with homoclinisms. Thus, there exists an
    isomorphism $\varphi_1:E_0' \to E_1'$ such that $\varphi_1 \circ
    \Omega_0 = \Omega_1$. Since both $\varphi_1$ and $\Omega_0$ are
    isomorphisms, $\Omega_1$ is also an isomorphism.

  \item Proof of (4): This follows by combining (1), (2), and (3).
  \end{itemize}
\end{proof}

\subsection{An alternate construction of the initial object}\label{sec:freeinitialobject}

Here is an alternative way of constructing a central extension $E_1$ for which the natural map:

$$\Omega_1:G \wedge G \to E_1$$

is an isomorphism.

Write $G$ as the quotient of a free group $F$ by a normal subgroup $R$
of $F$. Let $\nu:F \to G$ denote the quotient map. The kernel of $\nu$
is $R$.

The group $E_1$ that we are interested in is $F/[F,R]$. This group
$E_1$ is a central extension of $G$ in a natural fashion. Denote by
$\overline{\nu}$ the corresponding quotient map $E_1 \to G$. Consider
the commutator map $\omega_1:G \times G \to E_1$ and denote by
$\Omega_1$ the corresponding group homomorphism:

$$\Omega_1: G \wedge G \to [E_1,E_1]$$

Consider any extension:

$$0 \to A \to E_2 \stackrel{\mu}{\to} G \to 1$$

with the natural commutator map $\omega_2:G \times G \to E_2$ and the
corresponding commutator map homomorphism:

$$\Omega_2:G \wedge G \to [E_2,E_2]$$

Our goal is to show that there there exists a unique homomorphism
$\varphi:[E_1,E_1] \to [E_2,E_2]$ such that $\varphi \circ
\omega_1 = \omega_2$, or equivalently, $\varphi \circ \Omega_1 =
\Omega_2$.

The map $\nu:F \to G$ lifts to a map $\psi:F \to E_2$ because
$F$ is a free group (note that the lift is not necessarily
unique). Explicitly, this means that $\mu \circ \psi = \nu$.

We know that $\nu(R)$ is trivial, so $\mu(\psi(R))$ is trivial. Thus,
$\psi(R)$ lands inside the kernel of $\mu$, which is the image of $A$
in $E_2$. Thus, $\psi(R)$ is a central subgroup of $E_2$. Therefore,
$\psi([F,R]) = [\psi(F),\psi(R)]$ is trivial.

Thus, $\psi$ descends to a homomorphism $\theta:E_1 \to E_2$, where
$E_1 = F/[F,R]$ as defined above, with the property that $\mu \circ
\theta = \overline{\nu}$. The condition $\mu \circ \theta =
\overline{\nu}$ can be interpreted as saying that $\theta$ is a
homomorphism from the central extensions $(E_1,\overline{\nu})$ to the
central extension $(E_2,\mu)$. Denote by $\varphi:E_1' \to E_2'$ the
restriction of $\theta$ to $E_1'$. Thus, by Lemma
\ref{lemma:homomorphism-restriction-homoclinism}, $\varphi$ defines a
homoclinism of the central extensions. Lemma
\ref{lemma:uniqueness-of-homoclinism} now establishes that the
homoclinism is unique. Finally, Lemma \ref{lemma:initobj} establishes
from this that $(E_1,\overline{\nu})$ defines an initial object in the
category of central extensions of $G$ with homoclinisms.

{\em Clarification regarding uniqueness}: In the discussion above, the
lift $\psi: F \to E_2$ of $\nu:F \to G$ is not unique, because it
involves picking {\em arbitrary} coset representatives of $G$ in $E_2$
for the freely generating set of $F$. The homomorphism $\theta:E_1 \to
E_2$ also need not be unique. However, the map $\varphi:E_1' \to
E_2'$, obtained by restricting $\theta$ to $E_1'$, {\em is} unique.

\subsubsection{Canonical choice of $E_1$}

The description of $E_1$ above is unique once we fix the description
of $G$ of the form $F/R$ where $F$ is a free group. Specifying a
description of this form is equivalent to specifying a generating set
for $G$, and thus, the description relies on a choice of generating
set for $G$.

It is possible to make a {\em canonical} choice of $E_1$ by making a
canonical choice of generating set for $G$, namely, the entire
underlying set of $G$. In this case, the free group $F$ is the free
group on the underlying set of $G$, and the normal subgroup $R$ is
generated by the multiplication table of $G$, viewed as relations
within $F$ (explicitly, for any product relation of the form $gh = k$
in $G$, we introduce the relation $ghk^{-1}$ in $F$).

There is an alternative description of the group $E_1$ that
demonstrates its canonical nature: $E_1$ is the freest possible group
admitting $G$ as a quotient by a central subgroup.\footnote{This might
  tempt one to think that $(E_1,\overline{\nu})$ is an initial object
  in the category of central extensions of $G$ with {\em
    homomorphisms}, but the non-uniqueness of homomorphisms involved,
  along with some other considerations, makes this false.}

\subsection{Functoriality of exterior square and Schur multiplier}\label{sec:functoriality}

The exterior square and Schur multiplier are both {\em
  functorial}. Explicitly, for any homomorphism $\varphi:G \to H$ of
groups, there are homomorphisms $\varphi \wedge \varphi: G \wedge G
\to H \wedge H$ and $M(\varphi): M(G) \to M(H)$ and the associations
are functorial. This means that for homomorphisms $\varphi:G \to H$
and $\theta: H \to K$, $(\theta \circ \varphi) \wedge (\theta \circ
\varphi) = (\theta \wedge \theta) \circ (\varphi \wedge \varphi)$ and
also $M(\theta \circ \varphi) = M(\theta) \circ M(\varphi)$.

The proofs of both assertions are straightforward, but we are not
including them here because we do not use the functoriality of the
Schur multiplier. See a more detailed discussion of functoriality in
the Appendix, Section \ref{appsec:functor}.

\subsection{Homoclinisms and words for central extensions}\label{sec:homoclinisms-words-central-extensions}

We state and prove some results that are similar in spirit to the
results in Section \ref{sec:homoclinism-misc-results}.

\begin{lemma}\label{lemma:iterated-commutator-descends-extension-version}
  Suppose $G$ is a group and $w(g_1,g_2,\dots,g_n)$ is a word in $n$
  letters with the property that $w$ evaluates to the identity element
  in every abelian group. The following are true.

  \begin{enumerate}
  \item For every central extension $E$ of $G$, $w$ can be used to
    define a set map $\chi_{w,E}: G^n \to [E,E]$.
  \item For any homoclinism between central extensions $E_1$ and
    $E_2$, with the central extension specified via a homomorphism
    $\varphi:[E_1,E_1] \to [E_2,E_2]$, we have that:

    $$\varphi \circ \chi_{w,E_1} = \chi_{w,E_2}$$
  \end{enumerate}
\end{lemma}

%%TONOTDO: Possibly insert more detail here

\begin{proof}
  {\em Proof of (1)}: This is similar to the proof of Theorem
  \ref{thm:iterated-commutator-descends-to-inn}. Alternatively, we can
  deduce it from the {\em result} of Theorem
  \ref{thm:iterated-commutator-descends-to-inn} by noting that the
  map factors as follows:

  $$G^n \to (E/Z(E))^n \to [E,E]$$

  {\em Proof of (2)}: This is similar to the proof of Theorem
  \ref{thm:iterated-commutator-commutes-homoclinisms}. Alternatively,
  we can deduce it from the {\em result} of Theorem
  \ref{thm:iterated-commutator-commutes-homoclinisms} by factoring
  through $E/Z(E)$.
\end{proof}

We can now prove the theorem.

\begin{theorem}\label{thm:iterated-commutator-map-to-exteriorsquare}
  Suppose $G$ is a group and $w(g_1,g_2,\dots,g_n)$ is a word in $n$
  letters with the property that $w$ evaluates to the identity element
  in every abelian group. Then, there exists a set map $X_w:G^n \to G
  \wedge G$ with the property that for any central extension $E$ of
  $G$, $\Omega_{E,G} \circ X_w = \chi_{w,E}$.
\end{theorem}

\begin{proof}
  Apply Part (1) of Lemma
  \ref{lemma:iterated-commutator-descends-extension-version} to the
  case where the extension $E_0$ is an initial object in the category
  of central extensions of $G$, so that the map $\Omega_{E_0,G}: G
  \wedge G \to [E_0,E_0]$ is an isomorphism. Define $X_w =
  \Omega_{E_0,G}^{-1} \circ \chi_{w,E_0}$. For any central extension
  $E$ of $G$, there is a homoclinism from the extension $E_0$ to the
  extension $E$ defined via the homomorphism $\varphi: [E_0,E_0] \to
  [E,E]$. By Part (2) of Lemma
  \ref{lemma:iterated-commutator-descends-extension-version}, we have:

  $$\varphi \circ \chi_{w,E_0} = \chi_{w,E}$$

  We can rewrite $\chi_{w,E_0}$ as $\Omega_{E,G} \circ X_w$, and obtain:

  $$\varphi \circ (\Omega_{E,G} \circ X_w) = \chi_{w,E}$$

  Using associativity of composition, we obtain:

  $$(\varphi \circ \Omega_{E,G}) \circ X_w = \chi_{w,E}$$

  $\Omega$ itself commutes with homoclinisms, so we obtain:

  $$\Omega_{E_0,G} \circ X_w = \chi_{w,E}$$
\end{proof}
%\newpage

\section{Exterior square, Schur multiplier, and homoclinism for Lie rings}\label{sec:exteriorsquare-and-homoclinism-lie}

A large part of this section repeats for Lie rings what the previous
section did for groups. The main exception is the content in Sections
\ref{sec:free-lie-ring-on-abelian-group} and Section
\ref{sec:exteriorsquare-abelian-lie}. The material presented in
Section \ref{sec:free-lie-ring-on-abelian-group} has no natural group
analogue, while the results Section
\ref{sec:exteriorsquare-abelian-lie} have group analogues that are
harder to prove, and are deferred to Sections
\ref{sec:exteriorsquare-abelian-group-intro} and
\ref{sec:exteriorsquare-abelian-group-proofs}.

For background on the homology and cohomology theory of Lie rings, see
Weibel's homological algebra textbook \cite{Weibel}.

\subsection{Exterior square}\label{sec:exteriorsquare-lie}

The concept of exterior square of a Lie ring appears to have first
been explicitly discussed in the literature in the paper
\cite{EllisLie} by Graham Ellis.

The definition that we provide here for the exterior square is the
``abstract'' definition. We will provide a concrete definition (based
on generators and relations) in Section
\ref{sec:exteriorsquare-explicit-lie}. The equivalence of the two
approaches follows from the work in \cite{EllisLie}. Background theory
on the homology of Lie rings is discussed in
\cite{SchurmultiplierandLazard} and the references therein.

Suppose $L$ is a Lie ring. The {\em exterior square} of $L$, denoted by
$L \wedge L$, is defined as follows. Let $\mathcal{F}$ be the free
Lie ring on the set $L \times L$. 

For any central extension:

$$0 \to A \to N \to L \to 0$$

there is a set map (in fact, a $\mathbb{Z}$-bilinear map):

$$\omega_{N,L}: L \times L \to [N,N]$$

given by:

$$\omega_{N,L}(x,y) = [\tilde{x},\tilde{y}]$$

where $\tilde{x}$ denotes any element of $N$ that maps to $x$ and
$\tilde{y}$ denotes any element of $N$ that maps to $y$. Note that the
map is well defined (i.e., it does not depend on the choice of the
lifts $\tilde{x}$ and $\tilde{y}$) because the extension is a central
extension.

$\mathcal{F}$ is the free Lie ring on $L \times L$, so $\omega_{N,L}$
gives rise to a Lie ring homomorphism:

$$\hat{\omega}_{N,L}:\mathcal{F} \to [N,N]$$

Note also that this homomorphism is {\em surjective}, because by
definition, $[N,N]$ is the subring of $N$ generated by the image of the
set map $\omega_{N,L}$.

Define $\mathcal{R}$ as the intersection of the kernels of all such
homomorphisms $\hat{\omega}_{N,L}$ where $N$ varies over all central
extension Lie rings with quotient ring $L$. Note that even though the
collection of all such homomorphisms is too large to be a set, the
collection of possible kernels is a set, so the intersection is well
defined. In other words, $\mathcal{R}$ is the set of all
$\mathbb{Z}$-linear combinations of formal pairs such that the
corresponding sums of Lie brackets would become trivial in every
central extension of $L$. We define the exterior square $L \wedge L$
as the quotient Lie ring $\mathcal{F}/\mathcal{R}$. The image of $(x,y)$
in the Lie ring is denoted $x \wedge y$.

It is clear from the definition that, for any central
extension $N$ with short exact sequence:

$$0 \to A \to N \to L \to 0$$

there exists a unique natural homomorphism $\Omega_N$ from $L \wedge L$ to
$[N,N]$ satisfying the condition that for any $x,y \in L$ we have:

$$\Omega_N(x \wedge y) = [\tilde{x},\tilde{y}]$$

where $\tilde{x}$ and $\tilde{y}$ are elements of $N$ that map to $x$
and $y$ respectively. Note also that $\Omega_N$ is {\em surjective}.

As a special case of the above, there is a natural homomorphism:

$$L \wedge L \to [L,L]$$

given on a generating set by:

$$x \wedge y \mapsto [x,y]$$

The kernel of this homomorphism is called the {\em Schur multiplier}
of $L$ and is denoted $M(L)$. We can easily deduce that $M(L)$ is a
central subring of $L \wedge L$. We thus have a short exact sequence:

$$0 \to M(L) \to L \wedge L \to [L,L] \to 0$$

\subsection{Free Lie ring on an abelian group}\label{sec:free-lie-ring-on-abelian-group}

Suppose $G$ is an abelian group. The {\em free Lie ring} on $G$ is
defined as the initial object in the category of Lie rings $L$ with
group homomorphisms from $G$ to them.

\begin{lemma}
  The free Lie ring on $G$ exists and is a $\N$-graded Lie ring where
  the degree $1$ homogeneous component is isomorphic to $G$.
\end{lemma}

\begin{proof}
  The free Lie ring on $G$ is the quotient of the free ring on $G$ by
  the ideal generated by all the Lie identities. The free ring on $G$
  is given as the infinite direct sum:

  $$\bigoplus_{i=1}^\infty \bigotimes^iG$$

  The ideal that we need to factor out by is a homogeneous ideal
  because all the Lie identities are homogeneous identities. Thus, the
  free Lie ring is naturally a $\N$-graded Lie ring.
\end{proof}

Denote by $\mathcal{L}$ the free Lie ring on $G$. Then, for any
positive integer $c$, we can define the free class $c$ Lie ring on $G$
as the quotient ring $\mathcal{L}/\gamma_{c+1}(\mathcal{L})$. Note that this is
also a $\N$-graded Lie ring, but it is zero except in the first $c$
homogeneous components. In particular, this is a quotient of:

$$\bigoplus_{i=1}^c \bigotimes^i G$$

\begin{lemma}
  Suppose $G$ is an abelian group. The degree $2$ homogeneous
  component of the free Lie ring on $G$ is isomorphic to the ring $G
  \wedge_\Z G$. Equivalently, the free class two Lie ring on $G$ has
  additive group $G \oplus (G \wedge_\Z G)$ with Lie bracket:

  $$[(x,u),(y,v)] = [0,x \wedge_\Z y]$$
\end{lemma}

\begin{proof}
  Note that of the Lie ring identities, the Jacobi identity becomes
  redundant because of the class two condition. The only condition on
  the Lie bracket is that it is alternating. Taking the quotient of $G
  \oplus (G \otimes G)$ by this relation gives $G \oplus (G \wedge_\Z
  G)$.
\end{proof}

\subsection{Relation between exterior square of a Lie ring and exterior square in the abelian group sense}\label{sec:exteriorsquare-abelian-lie}

Recall that the additive group of $L$ has an exterior square {\em as
  an abelian group}. Denote this as $L \wedge_\Z L$.

We have a canonical abelian group homomorphism:

$$L \wedge_\Z L \to L \wedge L$$

The homomorphism is constructed as follows. For every central
extension $N$ of $L$, the map $\Omega_N:L \times L \to [N,N]$ is
$\mathbb{Z}$-bilinear because the Lie bracket map itself is
$\mathbb{Z}$-bilinear.

Thus, the natural map $L \times L \to L \wedge L$ given by $(x,y)
\mapsto x \wedge y$ is $\mathbb{Z}$-bilinear. Hence, it induces an
abelian group homomorphism $L \wedge_\Z L \to L \wedge L$. This is the
homomorphism we seek.

The canonical homomorphism $L \wedge_\Z L \to L \wedge L$ is
surjective: It is obvious that $L \wedge L$ is generated as a Lie ring
by the image of $L \wedge_\Z L$. The main thing to verify is that, in
fact, the image of $L \wedge_\Z L$ is closed under the Lie
bracket. The following identity demonstrates this:

$$[(m_1 \wedge n_1),(m_2 \wedge n_2)] = -[n_1,m_1] \wedge [m_2,n_2]$$

We now turn to a proof of an important result. Note that this is
subsumed by the explicit presentation of the exterior square in
Section \ref{sec:exteriorsquare-explicit-lie}, but we give an explicit
proof here for convenience.

\begin{lemma}\label{lemma:exteriorsquare-of-abelian-group}
  Suppose $L$ is an abelian Lie ring. Then, the canonical surjective
  homomorphism $L \wedge_\Z L \to L \wedge L$ is an isomorphism. This
\end{lemma}

\begin{proof}
  We have already established surjectivity, so to demonstrate
  injectivity, it suffices to construct a central extension $N$ of $L$
  for which the composite map $L \wedge_\Z L \to [N,N]$ is an
  isomorphism. Taking $N$ to be the free class two Lie ring on the
  additive group of $L$ works, based on the discussion in Section
  \ref{sec:free-lie-ring-on-abelian-group}.
\end{proof}

\subsection{The existence of a single central extension that realizes the exterior square}\label{sec:grandcentralproduct-lie}

Consider a Lie ring $L$. A natural question is whether there exists a
central extension Lie ring $N$ with quotient Lie ring $L$ with the
property that the natural homomorphism:

$$\Omega_N: L \wedge L \to [N,N]$$

is an isomorphism.

The answer to this question is {\em yes}. We provide one construction
below. We will provide another construction in
Section \ref{sec:freeinitialobject-lie}.

Recall the earlier description of $L \wedge L$ as a quotient
$\mathcal{F}/\mathcal{R}$. The ideal $\mathcal{R}$ was
defined as the intersection of all possible ideals arising
as kernels of the natural homomorphisms $\mathcal{F} \to [N,N]$ for a
central extension Lie ring $N$. For each possible ideal $J_i, i
\in I$ of $\mathcal{F}$ that arises this way, let $N_i$ denote a
corresponding central extension of $L$.

Define $N_0$ to be the pullback (also called the {\em fiber product}
or the {\em subdirect product}) corresponding to all the quotient maps
$N_i \to L$. We can verify that the natural mapping:

$$\mathcal{F} \to [N_0,N_0]$$

has kernel precisely $\mathcal{R}$, and hence, the mapping:

$$L \wedge L \to [N_0,N_0]$$

is an isomorphism.

\subsection{Homoclinism of central extensions}\label{sec:homoclinism-central-extensions-lie}

Suppose $L$ is a Lie ring. We define a certain category for which we are
interested in computing the initial object. We will call this category
the {\em category of central extensions of $L$ with homoclinisms}. Explicitly,
the objects of the category are short exact sequences of the form:

$$0 \to A \to N \to L \to 0$$

where the image of $A$ is central in $N$.

The morphisms in the category, which we call {\em homoclinisms of
  central extensions}, are defined as follows. For two objects:

$$0 \to A_1 \to N_1 \to L \to 0$$

and

$$0 \to A_2 \to N_2 \to L \to 0$$

a morphism from the first to the second is a Lie ring homomorphism
$\varphi: N_1' \to N_2'$ such that the following holds. Let $\omega_1:
L \times L \to N_1'$ denote the map arising from the Lie bracket map in
$N_1$ and let $\omega_2: L \times L \to N_2'$ denote the corresponding
map in $N_2$. We require that $\varphi \circ \omega_1 = \omega_2$ as set maps.

The above condition can be reframed in terms of Lie ring homomorphisms if
we use the exterior square: let $\Omega_1: L \wedge L \to N_1'$,
$\Omega_2: L \wedge L \to N_2'$ denote the natural homomorphisms
described in Section \ref{sec:exteriorsquare-lie}. The condition we need
is that $\varphi \circ \Omega_1 = \Omega_2$.

\subsection{Relation between the category of central extensions and the category of central extensions with homoclinisms}

Any {\em homomorphism} of central extensions induces a {\em
  homoclinism} of central extensions. Explicitly, consider two central
extensions:

$$0 \to A_1 \to N_1 \stackrel{\nu_1}{\to} L \to 0$$

and

$$0 \to A_2 \to N_2 \stackrel{\nu_2}{\to} L \to 0$$

As discussed in Section \ref{sec:homomorphism-central-extensions-lie}, the
central extensions are completely described by the pairs $(N_1,\nu_1)$
and $(N_2,\nu_2)$ respectively. A homomorphism of central extensions
can be specified as a homomorphism $\theta:N_1 \to N_2$ satisfying the
condition that $\nu_2 \circ \theta = \nu_1$.

Any such homomorphism of central extensions induces a {\em
  homoclinism} of central extensions. Explicitly, for a homomorphism
$\theta: N_1 \to N_2$ satisfying $\nu_2 \circ \theta = \nu_1$, define
$\varphi$ as the homomorphism $N_1' \to N_2'$ obtained by restricting
$\theta$ to $N_1'$. We claim that $\varphi$ defines a homoclinism of
the central extensions. We now prove that this construction works.

\begin{lemma}\label{lemma:homomorphism-restriction-homoclinism-lie}
  Suppose $(N_1,\nu_1)$ and $(N_2,\nu_2)$ are central extensions of a
  Lie ring $L$, and $\theta:N_1 \to N_2$ is a homomorphism of central
  extensions, i.e., $\nu_2 \circ \theta = \nu_1$. Denote by $\omega_1:
  L \times L \to N_1'$ and $\omega_2: L \times L \to N_2'$ the maps
  induced by the commutator maps in $N_1$ and $N_2$ respectively. Let
  $\varphi:N_1' \to N_2'$ be the homomorphism obtained by restricting
  $\theta$ to the derived subring $N_1'$. Then, $\varphi$ is a
  homoclinism of central extensions, i.e., $\varphi \circ \omega_1 =
  \omega_2$.
\end{lemma}

\begin{proof}
  Let $u$, $v$ be arbitrary elements of $L$ (possibly equal, possibly
  distinct). Our goal is to show that:

  $$\varphi(\omega_1(u,v)) = \omega_2(u,v)$$

  Let $x, y \in N_1$ be elements such that $\nu_1(x) = u$ and
  $\nu_2(y) = v$. Then, by definition, $\omega_1(u,v) =
  [x,y]$. Simplify the left side:

  $$\varphi(\omega_1(u,v)) = \varphi([x,y]) = \theta([x,y]) = [\theta(x),\theta(y)] = \omega_2(\nu_2(\theta(x)),\nu_2(\theta(y)))$$

  Now, use that $\nu_2 \circ \theta = \nu_1$ and simplify further to:

  $$\omega_2(\nu_1(x),\nu_1(y)) = \omega_2(u,v)$$

  which is the right side.
\end{proof}

\subsection{Uniqueness of homoclinism if it exists}

We will show that if a homoclinism exists between two central
extensions, it must be unique.

\begin{lemma}\label{lemma:uniqueness-of-homoclinism-lie}
  Consider two short exact sequences that give central extensions of a Lie ring $L$:

  $$0 \to A_1 \to N_1 \to L \to 0$$

  $$0 \to A_2 \to N_2 \to L \to 0$$

  Denote by $\omega_1:L \times L \to N_1'$ and $\omega_2:L \times L
  \to N_2'$ the Lie bracket maps.

  Suppose there exists homoclinisms $\varphi,\theta$ from the first
  central extension to the second. Explicitly, $\varphi:N_1' \to N_2'$
  and $\theta:N_1' \to N_2'$ are homomorphisms such that $\varphi
  \circ \omega_1 = \omega_2$ and $\theta \circ \omega_1 =
  \omega_2$. Then, $\varphi = \theta$.
\end{lemma}

\begin{proof}
  Denote by $\nu_1$ the quotient map $N_1 \to L$ and by $\nu_2$ the
  quotient map $N_2 \to L$.

  It will suffice to show that $\varphi$ and $\theta$ agree with each
  other on the set of all Lie brackets, which is a generating set for
  $N_1'$. Consider a Lie bracket $[x,y]$ with $x,y \in N_1$. Let $u =
  \nu_1(x)$ and $v = \nu_1(y)$. 

  By definition, $[x,y] = \omega_1(u,v)$. Thus, $\varphi([x,y]) =
  \varphi(\omega_1(u,v)) = \omega_2(u,v)$. Similarly, $\theta([x,y]) =
  \theta(\omega_1(u,v)) = \omega_2(u,v)$. We thus obtain that
  $\varphi([x,y]) = \theta([x,y])$, completing the proof.
\end{proof}

Thus, if a homoclinism exists, it is unique. However, a homoclinism
need not exist. The obstruction occurs if there are relations within
the derived subring $N_1'$ such that the corresponding relations are
not valid in the derived subring $N_2'$.

\subsection{Existence and description of initial objects in the category of central extensions up to homoclinisms}

We are interested in identifying the initial objects in the category
of extensions of $L$ with homoclinisms discussed in Section
\ref{sec:homoclinism-central-extensions-lie}.

\begin{lemma}[Existence and description of initial objects]\label{lemma:initobj-lie}
  For a Lie ring $L$, consider the category of central extensions of $L$
  with homoclinisms. The following are true for this category.
  \begin{enumerate}
  \item There exists a central extension $N_0$ of $L$ for which the
    natural homomorphism $\Omega_0: L \wedge L \to N_0'$ is an isomorphism.
  \item Any central extension $N_0$ of $L$ for which the natural
    homomorphism $\Omega_0:L \wedge L \to N_0'$ is an isomorphism is
    an initial object of the category.
  \item If a central extension $N_1$ of $L$ is an initial object of
    the category, the corresponding homomorphism $\Omega_1:L \wedge L
    \to N_1'$ is an isomorphism.
  \item Combining all the above: the category of central extensions of
    $L$ with homoclinisms admits initial objects, and a central
    extension $N \to L$ is an initial object for the category if and
    only if the Lie bracket map homomorphism $\Omega_N: L \wedge L \to
    [N,N]$ is an isomorphism.
  \end{enumerate}
\end{lemma}

\begin{proof}
  \begin{itemize}
  \item Proof of (1): In Section \ref{sec:grandcentralproduct-lie}, we
    constructed a central extension Lie ring $N_0$ for which the natural
    map $\Omega_0: L \wedge L \to N_0'$ is an isomorphism.

  \item Proof of (2): Suppose $N_0$ is a central extension for which
    the Lie bracket map $\Omega_0:L \wedge L \to N_0'$ is an
    isomorphism. For any central extension $N_2$, there is a natural
    homomorphism $\Omega_2: L \wedge L \to N_2'$. Composing this with
    the inverse of $\Omega_0$, we obtain a homomorphism
    $\varphi:N_0' \to N_2'$ that defines a homoclinism of the
    extensions. Moreover, by Lemma
    \ref{lemma:uniqueness-of-homoclinism-lie}, this is the {\em unique}
    homoclinism of the extensions.

  \item Proof of (3): We already know of the existence of a central
    extension $N_0$ for which $\Omega_0:L \wedge L \to N_0'$ is
    an isomorphism by (1). We also know that it is an initial object
    by (2). By the uniqueness of initial objects up to isomorphism,
    $N_0$ and $N_1$ are isomorphic in the category of central
    extensions of $L$ with homoclinisms. Thus, there exists an
    isomorphism $\varphi_1:N_0' \to N_1'$ such that $\varphi_1 \circ
    \Omega_0 = \Omega_1$. Since both $\varphi_1$ and $\Omega_0$ are
    isomorphisms, $\Omega_1$ is also an isomorphism.

  \item Proof of (4): This follows by combining (1), (2), and (3).
  \end{itemize}
\end{proof}

\subsection{An alternate construction of the initial object}\label{sec:freeinitialobject-lie}

Here is an alternative way of constructing a central extension $N_1$ for which the natural map:

$$\Omega_1:L \wedge L \to N_1$$

is an isomorphism.

Write $L$ as the quotient of a free Lie ring $F$ by an ideal $R$ of
$F$. Let $\nu:F \to L$ denote the quotient map. The kernel of $\nu$ is
$R$.

The Lie ring $N_1$ that we are interested in is $F/[F,R]$. This Lie ring
$N_1$ is a central extension of $L$ in a natural fashion. Denote by
$\overline{\nu}$ the corresponding quotient map $N_1 \to L$. Consider
the Lie bracket map $\omega_1:L \times L \to N_1$ and denote by
$\Omega_1$ the corresponding Lie ring homomorphism:

$$\Omega_1: L \wedge L \to [N_1,N_1]$$

Consider any extension:

$$0 \to A \to N_2 \stackrel{\mu}{\to} L \to 0$$

with the natural Lie bracket map $\omega_2:L \times L \to N_2$ and the
corresponding Lie bracket map homomorphism:

$$\Omega_2:L \wedge L \to [N_2,N_2]$$

Our goal is to show that there there exists a unique homomorphism
$\varphi:[N_1,N_1] \to [N_2,N_2]$ such that $\varphi \circ
\omega_1 = \omega_2$, or equivalently, $\varphi \circ \Omega_1 =
\Omega_2$.

The map $\nu:F \to L$ lifts to a map $\psi:F \to N_2$ because $F$
is a free Lie ring (note that the lift is not necessarily
unique). Explicitly, this means that $\mu \circ \psi = \nu$.

We know that $\nu(R)$ is trivial, so $\mu(\psi(R))$ is trivial. Thus,
$\psi(R)$ lands inside the kernel of $\mu$, which is the image of $A$
in $N_2$. Thus, $\psi(R)$ is a central subring of $N_2$. Therefore,
$\psi([F,R]) = [\psi(F),\psi(R)]$ is trivial.

Thus, $\psi$ descends to a homomorphism $\theta:N_1 \to N_2$, where
$N_1 = F/[F,R]$ as defined above, with the property that $\mu \circ
\theta = \overline{\nu}$. The condition $\mu \circ \theta =
\overline{\nu}$ can be interpreted as saying that $\theta$ is a
homomorphism from the central extensions $(N_1,\overline{\nu})$ to the
central extension $(N_2,\mu)$. Denote by $\varphi:N_1' \to N_2'$ the
restriction of $\theta$ to $N_1'$. Thus, by Lemma
\ref{lemma:homomorphism-restriction-homoclinism-lie}, $\varphi$
defines a homoclinism of the central extensions. Lemma
\ref{lemma:uniqueness-of-homoclinism-lie} now establishes that the
homoclinism is unique. Finally, Lemma \ref{lemma:initobj-lie}
establishes that ($N_1,\nu_1)$ defines an initial object in the
category of central extensions of $L$ with homoclinisms.

{\em Clarification regarding uniqueness}: In the discussion above, the
lift $\psi: F \to N_2$ of $\nu:F \to L$ is not unique, because it
involves picking {\em arbitrary} coset representatives of $L$ in $N_2$
for the freely generating set of $F$. The homomorphism $\theta:N_1 \to
N_2$ also need not be unique. However, the map $\varphi:N_1' \to
N_2'$, obtained by restricting $\theta$ to $N_1'$, {\em is} unique.

\subsection{Functoriality of exterior square and Schur multiplier}\label{sec:functoriality-lie}

The exterior square and Schur multiplier are both {\em
  functorial}. Explicitly, for any homomorphism $\varphi:L \to H$ of
Lie rings, there are homomorphisms $\varphi \wedge \varphi: L \wedge L
\to H \wedge H$ and $M(\varphi): M(L) \to M(H)$ and the associations
are functorial. This means that for homomorphisms $\varphi:L \to H$
and $\theta: H \to K$, $(\theta \circ \varphi) \wedge (\theta \circ
\varphi) = (\theta \wedge \theta) \circ (\varphi \wedge \varphi)$ and
also $M(\theta \circ \varphi) = M(\theta) \circ M(\varphi)$.

The proofs of both assertions are straightforward, but we are not
including them here because we do not use the functoriality of the
Schur multiplier. See a more detailed discussion of functoriality in
the Appendix, Section \ref{appsec:functor}.

\subsection{Homoclinisms and words for central extensions}\label{sec:homoclinisms-words-central-extensions-lie}

We state and prove some results that are similar in spirit to the
results in Section \ref{sec:homoclinism-misc-results-lie}.

\begin{lemma}\label{lemma:iterated-bracket-descends-extension-version}
  Suppose $L$ is a Lie ring and $w(g_1,g_2,\dots,g_n)$ is a Lie ring
  word in $n$ letters with the property that $w$ evaluates to the zero
  element in every abelian Lie ring. The following are true.

  \begin{enumerate}
  \item For every central extension $N$ of $L$, $w$ can be used to
    define a set map $\chi_{w,N}: L^n \to [N,N]$.
  \item For any homoclinism between central extensions $N_1$ and
    $N_2$, with the central extension specified via a homomorphism
    $\varphi:[N_1,N_1] \to [N_2,N_2]$, we have that:

    $$\varphi \circ \chi_{w,N_1} = \chi_{w,N_2}$$
  \end{enumerate}
\end{lemma}

%%TONOTDO: Possibly insert more detail here

\begin{proof}
  {\em Proof of (1)}: This is similar to the proof of Theorem
  \ref{thm:iterated-bracket-descends-to-inn}. Alternatively, we can
  deduce it from the {\em result} of Theorem
  \ref{thm:iterated-bracket-descends-to-inn} by noting that the
  map factors as follows:

  $$L^n \to (N/Z(N))^n \to [N,N]$$

  {\em Proof of (2)}: This is similar to the proof of Theorem
  \ref{thm:iterated-bracket-commutes-homoclinisms}. Alternatively,
  we can deduce it from the {\em result} of Theorem
  \ref{thm:iterated-bracket-commutes-homoclinisms} by factoring
  through $N/Z(N)$.
\end{proof}

We can now prove the theorem.

\begin{theorem}\label{thm:iterated-bracket-map-to-exteriorsquare}
  Suppose $L$ is a Lie ring and $w(g_1,g_2,\dots,g_n)$ is a word in $n$
  letters with the property that $w$ evaluates to the identity element
  in every abelian Lie ring. Then, there exists a set map $X_w:L^n \to
  L \wedge L$ with the property that for any central extension $N$ of
  $L$, $\Omega_{N,L} \circ X_w = \chi_{w,N}$.
\end{theorem}

\begin{proof}
  Apply Part (1) of Lemma
  \ref{lemma:iterated-bracket-descends-extension-version} to the
  case where the extension $N_0$ is an initial object in the category
  of central extensions of $L$, so that the map $\Omega_{N_0,L}: L
  \wedge L \to [N_0,N_0]$ is an isomorphism. Define $X_w =
  \Omega_{N_0,L}^{-1} \circ \chi_{w,E_0}$. For any central extension
  $N$ of $L$, there is a homoclinism from the extension $N_0$ to the
  extension $N$ defined via the homomorphism $\varphi: [N_0,N_0] \to
  [N,N]$. By Part (2) of Lemma
  \ref{lemma:iterated-bracket-descends-extension-version}, we have:

  $$\varphi \circ \chi_{w,N_0} = \chi_{w,N}$$

  We can rewrite $\chi_{w,N_0}$ as $\Omega_{N,L} \circ X_w$, and obtain:

  $$\varphi \circ (\Omega_{N,L} \circ X_w) = \chi_{w,N}$$

  Using associativity of composition, we obtain:

  $$(\varphi \circ \Omega_{N,L}) \circ X_w = \chi_{w,N}$$

  $\Omega$ itself commutes with homoclinisms, so we obtain:

  $$\Omega_{N_0,L} \circ X_w = \chi_{w,N}$$
\end{proof}

%\newpage

\section{Exterior square, Schur multiplier, and the second cohomology group}\label{sec:schur-multiplier-and-second-cohomology}

In Section \ref{sec:exteriorsquare-and-homoclinism}, we studied the
category of central extensions of a group $G$ using the groups $G
\wedge G$ (the exterior square of $G$) and $M(G)$ (the Schur
multiplier of $G$). Instead of studying the original category, we
considered the category of central extensions with {\em homoclinisms}
as the morphisms.

Our goal now is to consider, for any group $G$ and abelian group $A$,
the relation between the group $H^2(G;A)$ (described in Section
\ref{sec:second-cohomology-group-classify-extensions}) and the
isomorphism types of $G$ and $A$. More specifically, we will consider
the equivalence relation of being isoclinic on $H^2(G;A)$. The
equivalence classes turn out to be the fibers of a surjective map from
$H^2(G;A)$ that is the right map of an important short exact
sequence. We will construct the maps explicitly.

One crucial and nontrivial fact stated in this section will not be
proved here, namely, that the sequence is right exact, and more
specifically, that the short exact sequence under consideration is the
same as the universal coefficient theorem short exact sequence. For a
detailed description as well as proofs of these facts, see
\cite{BeylIsoclinisms}, Theorem 1.8, and in
\cite{EckmannHiltonStammbach}, Theorem 2.2.

\subsection{Homomorphism from the Schur multiplier to the kernel of the extension}\label{sec:homschurkernel}

We will now describe a very important homomorphism. For any central extension of the form:

$$0 \to A \to E \to G \to 1$$

there is a natural homomorphism from the Schur multiplier of $G$ to $A$, i.e., a homomorphism:

$$\beta: M(G) \to A$$

We now proceed to describe this homomorphism.

As discussed in Section \ref{sec:exteriorsquare}, there is a natural
homomorphism:

$$\Omega: G \wedge G \to [E,E]$$

Compose this with the inclusion of $[E,E]$ in $E$ to get a map $G
\wedge G \to E$. We obtain:

$$\begin{array}{ccccccccc}
0 & \to & M(G) & \to & G \wedge G & \to & [G,G] & \to & 1\\
 &&               && \downarrow     &&\downarrow&& \\
0 & \to & A & \to & E & \to & G & \to & 1\\
\end{array}$$

It is immediate that this diagram commutes.

By general diagram-chasing, we can construct a unique map $M(G) \to A$
such that the diagram continues to be commutative, and that is the
homomorphism $\beta$ that we want:

$$\begin{array}{ccccccccc}
0 & \to & M(G) & \to & G \wedge G & \to & [G,G] & \to & 1\\
 &&   \downarrow^{\beta}  &&  \downarrow     && \downarrow&& \\
0 & \to & A &\to & E & \to & G & \to & 1\\
\end{array}$$

\subsection{Classification of extensions up to isoclinism}\label{sec:extensionsuptoisoclinism}

Given a group $G$ and an abelian group $A$, we say that the central
extensions $E_1$ and $E_2$ with short exact sequences:

$$0 \to A \to E_1 \to G \to 1$$

and

$$0 \to A \to E_2 \to G \to 1$$

are {\em isoclinic (fixing both $G$ and $A$)} if there exists an
isomorphism of groups $\varphi:E_1' \to E_2'$ satisfying {\em both}
the following conditions:

\begin{itemize}
\item Isoclinic as extensions of $G$: If $\Omega_1:G \wedge G \to
  E_1'$ and $\Omega_2:G \wedge G \to E_2'$ are the commutator map
  homomorphisms, then $\varphi \circ \Omega_1 = \Omega_2$.
\item Suppose $B$ is the inverse image in $A$ of $[E_1,E_1]$. Then,
  $B$ is also the inverse image in $A$ of $[E_2,E_2]$. Moreover,
  composing $\varphi$ with the inclusion of $B$ in $[E_1,E_1]$ gives
  the inclusion of $B$ in $[E_2,E_2]$.
\end{itemize}

\subsection{Relating the classification of extensions up to isoclinism with the homomorphism from the Schur multiplier}\label{sec:beta-map}

In Section \ref{sec:homschurkernel}, we noted that for any central extension:

$$0 \to A \to E \to G \to 1$$

we have a natural homomorphism $\beta: M(G) \to A$.

The homomorphism is uniquely determined by the choice of extension up
to congruence, so we get a {\em set} map:

$$H^2(G;A) \to \operatorname{Hom}(M(G),A)$$

In Section \ref{sec:ses-uct}, we will see that this set map is a {\em
  group homomorphism}. We alluded to the group structure on $H^2(G;A)$
in Section \ref{sec:second-cohomology-group-classify-extensions} and
described it in detail in Section \ref{sec:cohomology-explicit}.

As we will see in Section \ref{sec:ses-uct}, this group homomorphism
is surjective and is the right map in an important short exact
sequence. For now, however, we note that {\em this group homomorphism
  classifies extensions up to isoclinism}. Explicitly, two extensions
$E_1,E_2$ are isoclinic in the sense of Section
\ref{sec:extensionsuptoisoclinism} if and only if they induce the same
homomorphism $M(G) \to A$. We now proceed to explain why. Note that
one direction, namely the direction that isoclinic extensions define
the same homomorphism from $M(G)$ to $A$, is obvious from the
definition. The other direction requires some work.

Consider the two short exact sequences below:

$$\begin{array}{ccccccccc}
0 & \to & M(G) & \to & G \wedge G & \to & [G,G] & \to & 1\\
 &&   \downarrow^{\beta}  &&  \downarrow     && \downarrow^{\text{id}} && \\
0 & \to & A &\to & E & \to & G & \to & 1\\
\end{array}$$

Suppose $B$ is the subgroup of $A$ that arises as the image of the
homomorphism $\beta: M(G) \to A$ and $\beta':M(G) \to B$ is the map
obtained by restricting the co-domain. We then have the following two
short exact sequences, where all the downward maps are surjective:

$$\begin{array}{ccccccccc}
0 & \to & M(G) & \to & G \wedge G & \to & [G,G] & \to & 1\\
 &&   \downarrow^{\beta'}  &&  \downarrow^{\Omega_{E,G}}     && \downarrow^{\text{id}} && \\
0 & \to & B &\to & [E,E] & \to & [G,G] & \to & 1\\
\end{array}$$

Note that the second row sequence is exact because all the downward
maps are surjective.\footnote{Some proof details involving diagram
  chasing are being omitted for brevity.}

It is easy to see that if $E_1$ and $E_2$ are two central extensions
of $G$ that give the same map $\beta$, then we can obtain an
isomorphism $[E_1,E_1] \to [E_2,E_2]$ such that in the diagram below,
the composite of the downward maps in the middle column is
$\Omega_{E_2,G}$, and the lower downward arrow in the middle column is
an isomorphism.

$$\begin{array}{ccccccccc}
0 & \to & M(G) & \to & G \wedge G & \to & [G,G] & \to & 1\\
 &&   \downarrow^{\beta'}  &&  \downarrow^{\Omega_{E,G}}     && \downarrow^{\text{id}} && \\
0 & \to & B &\to & [E_1,E_1] & \to & [G,G] & \to & 1\\
 &&   \downarrow^{\operatorname{id}}  && \downarrow && \downarrow^{\text{id}} && \\
0 & \to & B &\to & [E_2,E_2] & \to & [G,G] & \to & 1\\
\end{array}$$

\subsection{The universal coefficient theorem short exact sequence}\label{sec:ses-uct}

As before, let $G$ be a group and let $A$ be an abelian group. Our
goal is to understand all central extension groups $E$, i.e., short
exact sequences of the following form where the image of $A$ in $E$ is
in the center of $E$:

$$0 \to A \to E \to G \to 1$$

As discussed earlier, the set of all congruence classes of extensions
is classified by the group $H^2(G;A)$ (the second cohomology group for
trivial group action). We now proceed to describe a related short exact
sequence. The short exact sequence is discussed in
\cite{BeylIsoclinisms}, Theorem 1.8, and in
\cite{EckmannHiltonStammbach}, Theorem 2.2. The short exact sequence
is as follows:

\begin{equation}\label{eq:ses-uct}
  0 \to \operatorname{Ext}^1_{\mathbb{Z}}(G^{\operatorname{ab}},A) \to H^2(G;A) \to \operatorname{Hom}(M(G),A) \to 0
\end{equation}

\subsubsection{Interpretation of the left map of the sequence}\label{sec:ses-uct-left-map}

The map:

$$\operatorname{Ext}^1_{\mathbb{Z}}(G^{\operatorname{ab}},A) \to H^2(G;A)$$

takes an abelian group extension with normal subgroup $A$ and quotient
group $G^{\operatorname{ab}}$, and gives an extension with $G$ on top
of $A$ that can loosely be described as follows: the restriction to
the derived subgroup $[G,G]$ splits and the quotient sits as per the
element of
$\operatorname{Ext}^1_{\mathbb{Z}}(G^{\operatorname{ab}},A)$. Explicitly,
it can be thought of as a composite of two maps:

$$\operatorname{Ext}^1_{\mathbb{Z}}(G^{\operatorname{ab}},A) \to H^2(G^{\operatorname{ab}};A) \to H^2(G;A)$$

where the first map treats an abelian group extension simply as a
group extension, and the second map uses the contravariance of $H^2$
in its first argument.

\subsubsection{Interpretation of the right map of the sequence}\label{sec:ses-uct-right-map}

The right map of the sequence:

$$H^2(G;A) \to \operatorname{Hom}(M(G),A)$$

sends an extension group to the corresponding map $\beta$ described in
Section \ref{sec:homschurkernel}. In Section \ref{sec:beta-map}, we
showed that the map $H^2(G;A) \to \operatorname{Hom}(M(G),A)$
classifies extensions up to isoclinism.

\subsubsection{What the existence of the short exact sequence tells us}

Consider again the short exact sequence:

\begin{equation*}
  0 \to \operatorname{Ext}^1_{\mathbb{Z}}(G^{\operatorname{ab}},A) \to H^2(G;A) \to \operatorname{Hom}(M(G),A) \to 0
\end{equation*}

We now consider the three aspects of {\em exactness}:

\begin{itemize}
\item Left exactness, i.e., the injectivity of the map
  $\operatorname{Ext}^1_{\mathbb{Z}}(G^{\operatorname{ab}},A) \to
  H^2(G;A)$. This is the assertion that the only abelian group
  extension for $G^{\operatorname{ab}}$ on top of $A$ that maps to $G
  \times A$ is the trivial extension. This is immediate from the
  definition.
\item Middle exactness, i.e., the image of the map
  $\operatorname{Ext}^1_{\mathbb{Z}}(G^{\operatorname{ab}},A) \to
  H^2(G;A)$ is precisely the same as the kernel of the map $ H^2(G;A)
  \to \operatorname{Hom}(M(G),A)$. This is easy to see from the
  definition.
\item Right exactness, i.e., the surjectivity of the map $H^2(G;A) \to
  \operatorname{Hom}(M(G),A)$. This says that {\em every homomorphism}
  from $M(G)$ to $A$ arises from an extension with central subgroup
  $A$ and quotient group $G$. In other words, the set of extension
  types up to isoclinism can be identified with the group
  $\operatorname{Hom}(M(G),A)$. {\em This is the most important and
    least obvious of the three exactness statements}. Many of our
  later constructive results will rely crucially on right exactness.
\end{itemize}

\subsubsection{How it is a special case of the dual universal coefficient theorem}

The general version of the dual universal coefficient theorem for
group cohomology is as follows:

$$0 \to \operatorname{Ext}^1_{\mathbb{Z}}(H_{k-1}(G;\mathbb{Z}),A) \to H^k(G;A) \to \operatorname{Hom}(H_k(G;\mathbb{Z}),A) \to 0$$

If we set $k = 2$ and use the fact that $M(G)$ is canonically
isomorphic to $H_2(G;\mathbb{Z})$, and also that
$G^{\operatorname{ab}}$ is canonically isomorphic to
$H_1(G;\mathbb{Z})$, we get the short exact sequence we have been
discussing.

\subsubsection{The splitting of the short exact sequence}\label{sec:ses-uct-non-canonical-splitting}

The dual universal coefficient theorem for group cohomology, in addition
to providing the short exact sequence above, also states that the
short exact sequence always splits, but the splitting is not in
general canonical. Explicitly, the universal coefficient theorem
states that:

$$H^k(G;A) \cong \operatorname{Ext}^1_{\mathbb{Z}}(H_{k-1}(G;\mathbb{Z}),A) \oplus \operatorname{Hom}(H_k(G;\mathbb{Z}),A)$$

In the special case of interest to us, we obtain:

$$H^2(G;A) \cong \operatorname{Ext}^1_{\mathbb{Z}}(G^{\operatorname{ab}},A) \oplus \operatorname{Hom}(M(G),A)$$

The direct sum decomposition is {\em not} in general canonical. %% In
%% fact, there are examples where there is {\em no}
%% $\operatorname{Aut}(G) \times \operatorname{Aut}(A)$-invariant
%% splitting. {\em TONOTDO: Insert link to Appendix section}

In Section \ref{sec:ses-uct-lie-abelian-canonical-splitting}, we will
identify some special circumstances where the short exact sequence
splits canonically.

\subsection{An alternate characterization of initial objects, and the existence of Schur covering groups}

Recall that, by Lemma \ref{lemma:initobj}, a central extension:

$$0 \to A \to E \to G \to 1$$

is an initial object in the category of central extensions of $G$ with
homoclinisms if the natural homomorphism:

$$\Omega_{E,G}: G \wedge G \to [E,E]$$

is an isomorphism. We now provide an alternative characterization.

\begin{lemma}\label{lemma:initial-beta-injective}
  Consider a group $G$ and a central extension:

  $$0 \to A \to E \to G \to 1$$

  The central extension is an initial object in the category of
  central extensions of $G$ with homoclinisms if and only if the
  corresponding homomorphism $\beta:M(G) \to A$ (described in Section
  \ref{sec:homschurkernel}) is injective.
\end{lemma}

\begin{proof}
  Let $B$ be the image in $A$ of $\beta$ and let $\beta'$ be the
  restriction of $\beta$ to co-domain $B$, so $\beta'$ is a surjective
  homomorphism from $M(G)$ to $B$. Note also that $\beta$ is injective
  if and only if $\beta'$ is an isomorphism.

  As discussed in Section \ref{sec:beta-map}, we have the following
  morphism of short exact sequences, where all the downward maps are
  surjective:

  $$\begin{array}{ccccccccc}
    0 & \to & M(G) & \to & G \wedge G & \to & [G,G] & \to & 1\\
    &&   \downarrow^{\beta'}  &&  \downarrow^{\Omega_{E,G}}     && \downarrow^{\text{id}} && \\
    0 & \to & B &\to & [E,E] & \to & [G,G] & \to & 1\\
  \end{array}$$
  
  Since the right-most downward map is the identity map, we see (from
  some elementary diagram chasing) that $\beta'$ is an isomorphism if
  and only if $\Omega_{E,G}$ is an isomorphism. 
\end{proof}

Recall the definition of stem extension from Section
\ref{sec:central-and-stem-extension}. We provide an alternative
characterization of such extensions:

\begin{lemma}\label{lemma:stem-beta-surjective}
  A central extension $0 \to A \to E \to G \to 1$ is a stem extension
  if and only if the corresponding map $\beta: M(G) \to A$ is
  surjective.
\end{lemma}

\begin{proof}
  Let $B$ be the image in $A$ of $\beta$ and let $\beta'$ be the
  restriction of $\beta$ to co-domain $B$, so $\beta'$ is a surjective
  homomorphism from $M(G)$ to $B$. Note also that $\beta$ is
  surjective if and only if $B = A$. Note also that $A \le Z(E)$ by
  the assumption that the extension is central, so the challenge is to
  show that $A \le [E,E]$ if and only if $B = A$.

  As described in Section \ref{sec:beta-map}, we have the following
  morphism of two short exact sequences, with all the downward maps
  surjective:

  $$\begin{array}{ccccccccc}
    0 & \to & M(G) & \to & G \wedge G & \to & [G,G] & \to & 1\\
    &&   \downarrow^{\beta'}  &&  \downarrow^{\Omega_{E,G}}     && \downarrow^{\text{id}} && \\
    0 & \to & B &\to & [E,E] & \to & [G,G] & \to & 1\\
  \end{array}$$

  Explicitly, $B$ is the kernel of the homomorphism from $[E,E]$ to
  $[G,G]$. The homomorphism from $[E,E]$ to $[G,G]$ is obtained by
  restricting to $[E,E]$ the homomorphism from $E$ to $G$. 

  Thus, $B = A \cap [E,E]$. It follows that $A \le [E,E]$ if and only
  if $B = A$, completing the proof.
\end{proof}

We are now prepared for a definition.

\begin{definer}[Schur covering group]
  We define a {\em Schur covering group} of $G$ as a group
  extension $E$ of $G$ with short exact sequence:

  $$0 \to A \to E \to G \to 1$$
  
  satisfying the condition that it is a central extension and the
  corresponding map $\beta: M(G) \to A$ (defined in Section
  \ref{sec:homschurkernel}) is an isomorphism. Equivalently, the
  extension must satisfy {\em both} these conditions:
  
  \begin{itemize}
  \item The extension is a stem extension, i.e., the image of $A$ in $E$
    is contained in $Z(E) \cap [E,E]$.
  \item The natural homomorphism $\Omega_{E,G}: G \wedge G \to [E,E]$ is an
    isomorphism.
  \end{itemize}
\end{definer}
  
The equivalence of the two versions of the definition follows from the
two preceding lemmas (Lemmas \ref{lemma:initial-beta-injective} and
\ref{lemma:stem-beta-surjective}).

The existence of Schur covering groups is not {\em a priori} clear,
but can be deduced from the short exact sequence of the preceding
section. 

\begin{theorem}\label{thm:schur-covering-groups-exist}
  For any group $G$, Schur covering groups of $G$ exist.
\end{theorem}

\begin{proof}
  For any abelian group $A$, we have the short exact sequence
  described in Section \ref{sec:ses-uct}:

  \begin{equation*}
    0 \to \operatorname{Ext}^1_{\mathbb{Z}}(G^{\operatorname{ab}},A) \to H^2(G;A) \to \operatorname{Hom}(M(G),A) \to 0
  \end{equation*}
  
  Now, set $A = M(G)$:

  \begin{equation*}
    0 \to \operatorname{Ext}^1_{\mathbb{Z}}(G^{\operatorname{ab}},M(G)) \to H^2(G;M(G)) \to \operatorname{Hom}(M(G),M(G)) \to 0
  \end{equation*}
  
  Consider the element $\operatorname{Id}_{M(G)} \in
  \operatorname{Hom}(M(G),M(G))$. By surjectivity (i.e., right
  exactness), there exists at least one element of $H^2(G;M(G))$ that
  maps to this. Note that the inverse image is in fact a coset in
  $H^2(G;M(G))$ of the image of
  $\operatorname{Ext}^1_{\mathbb{Z}}(G^{\operatorname{ab}},M(G))$. Each
  element in this inverse image corresponds to a Schur covering
  group. If
  $\operatorname{Ext}^1_{\mathbb{Z}}(G^{\operatorname{ab}},M(G))$ is
  nontrivial, the Schur covering group need not be unique.
\end{proof}

\subsection{Realizability of surjective homomorphisms from the exterior square}

Suppose $G$ and $D$ are groups and $\alpha:G \wedge G \to D$ and
$\delta:D \to [G,G]$ are surjective homomorphisms such that $\delta
\circ \alpha:G \wedge G \to [G,G]$ is the canonical map sending $x
\wedge y$ to $[x,y]$. We would like to know whether there is a central
extension:

$$0 \to A \to E \to G \to 1$$

with the property that there is an isomorphism $\theta:E' \to D$ such
that if we consider the homomorphism:

$$\Omega_{E,G}: G \wedge G \to E'$$

then $\theta \circ \Omega_{E,G} = \alpha$. The answer to this question is
{\em yes}. In fact, we can even choose $E$ to be a {\em stem}
extension of $G$. We outline the construction below.

Recall that we have the following short exact sequence, describing $G
\wedge G$ as a central extension of $[G,G]$:

$$0 \to M(G) \to G \wedge G \to [G,G] \to 1$$

Denote by $A$ the image of $M(G)$ under the set map $\alpha:G \wedge G
\to D$ and by $\beta:M(G) \to A$ the restricted map. We therefore
have the following commutative diagram:

$$\begin{array}{ccccccccc}
  0 & \to & M(G) & \to & G \wedge G & \to & [G,G] & \to & 1\\
  &&   \downarrow^{\beta}  &&  \downarrow^{\alpha}     && \downarrow^{\text{id}} && \\
  0 & \to & A &\to & D & \stackrel{\delta}{\to} & [G,G] & \to & 1\\
\end{array}$$

Now, consider the short exact sequence described in Section
\ref{sec:ses-uct}:

$$0 \to \operatorname{Ext}^1_{\mathbb{Z}}(G^{\operatorname{ab}},A) \to H^2(G;A) \to \operatorname{Hom}(M(G),A) \to 0$$

The right map is surjective, so there exists an extension group $E$
(corresponding to an element of $H^2(G,A)$) such that the map $\beta$
corresponding to $E$ (as described in Sections
\ref{sec:homschurkernel} and \ref{sec:beta-map}) is the map $\beta$
that we specified. Also, for reasons discussed in Section
\ref{sec:beta-map}, we can find an isomorphism $\theta:[E,E] \to D$
such that $\theta \circ \Omega_{E,G} = \alpha$.

\subsection{The existence of stem groups}\label{sec:stem-group-existence}

In Section \ref{sec:stem-group}, we defined a group $G$ as a {\em stem
  group} if $Z(G) \le G'$. Now that we have defined the concept of
stem extension, we can provide an alternate definition of stem group:
$G$ is a stem group if the short exact sequence:

$$0 \to Z(G) \to G \to G/Z(G) \to 1$$

makes $G$ a stem extension.

We now turn to the proof of a statement made in Section
\ref{sec:stem-group} without proof.

\begin{theorem}\label{thm:stem-group-existence}
  Suppose $G$ is a group. Then, the following are true:

  \begin{enumerate}
  \item There exists a stem group $K$ that is isoclinic to $G$.
  \item In case $G$ is finite, all stem groups isoclinic to $G$ are
    finite and have the same order as each other.
  \end{enumerate}
\end{theorem}

\begin{proof}
  {\em Proof of (1)}: Consider the short exact sequence:

  $$0 \to Z(G) \to G \to G/Z(G) \to 1$$

  This short exact sequence allows us to think of $G$ as a central
  extension with central subgroup $Z(G)$ and quotient group
  $G/Z(G)$. We apply the construction in Section
  \ref{sec:homschurkernel} (further discussed in Section
  \ref{sec:beta-map}) to obtain the natural map $\beta:M(G/Z(G)) \to
  Z(G)$. Let $B$ be the image of $\beta$. By the explicit
  construction, note that the image of $\beta$ is actually inside
  $Z(G) \cap G'$. Let $\beta':M(G/Z(G)) \to B$ be the map obtained by
  restricting the co-domain of $\beta$ to the image of $\beta$.

  Consider now the short exact sequence of Section \ref{sec:ses-uct}
  for central subgroup $B$ and quotient group $G/Z(G)$. The short exact sequence is:

  $$0 \to \operatorname{Ext}^1_{\mathbb{Z}}((G/Z(G))^{\operatorname{ab}},B) \to H^2(G/Z(G);B) \to \operatorname{Hom}(M(G/Z(G)),B) \to 0$$

  In particular, the map:

  $$H^2(G/Z(G);B) \to \operatorname{Hom}(M(G/Z(G)),B)$$

  is surjective. This means that we can find a (possibly non-unique
  and non-canonical) central extension $K$ with short exact sequence:

  $$0 \to B \to K \to G/Z(G) \to 1$$

  such that the map $\beta_K$ corresponding to this extension (per
  Section \ref{sec:homschurkernel}) is the same as $\beta'$. The following
  are now easy to verify: %{\em TONOTDO: Fill in details}

  \begin{itemize}
  \item The image of $B$ in $K$ is the center of $K$.
  \item The image of $B$ in $K$ is contained in the derived subgroup $K'$.
  \item $G$ and $K$ are isoclinic.
  \end{itemize}

  {\em Proof of (2)}: It is easy to verify that any stem group
  isoclinic to $G$ can be constructed in the above fashion, i.e., it
  is a central extension group $K$ with quotient group $G/Z(G)$ and
  central subgroup $B$ such that the map $\beta_K: M(G/Z(G)) \to B$ is
  equal to $\beta'$. In particular, if $G$ is finite, then both
  $G/Z(G)$ and $B$ are finite, so that $K$ is finite. Further, the
  order of $K$ is $|B||G/Z(G)|$, so all stem groups isoclinic to $G$
  have the same order.

  Finally, we wish to show that the order of all stem groups is less
  than or equal to the order of $G$. For this, note that $B$ is a
  subgroup of $Z(G) \cap G'$, so that $|B| \le |Z(G)|$. Thus, $|K| =
  |B||G/Z(G)| \le |Z(G)||G/Z(G)| = |G|$.
\end{proof}
  
Note that although stem groups exist, there may be no canonical choice
of stem group. The problem is the absence of a canonical splitting of
the short exact sequence, described in Section
\ref{sec:ses-uct-non-canonical-splitting}. If a canonical splitting
did exist, we could use that splitting to obtain a canonical choice of
extension.

\subsection{The Stallings exact sequence}

The {\em Stallings exact sequence} was defined by Stallings in
\cite{Stallings} and explored further by Eckmann, Hilton, and
Stammbach in \cite{EckmannHiltonStammbach} for arbitrary group
extensions.

Start with a short exact sequence of groups (note that $A$ is not
necessarily abelian, but we use this notation to stay consistent with
the other sections):

$$1 \to A \to E \to G \to 1$$

Then, the Stallings exact sequence is as follows:

$$M(E) \stackrel{\alpha}{\to} M(G) \stackrel{\beta}{\to} A/[E,A] \stackrel{\sigma}{\to} E^{\operatorname{ab}} \stackrel{\tau}{\to} G^{\operatorname{ab}}$$

The maps are described as follows:

\begin{itemize}
\item The homomorphism $\alpha:M(E) \to M(G)$ arises from the
  functoriality of the Schur multiplier, discussed in Section
  \ref{sec:functoriality}.
\item The homomorphism $\beta:M(G) \to A/[E,A]$ arises from the
  corresponding map in the central extension case (discussed below)
  once we replace the original short exact sequence by the short exact
  sequence $1 \to A/[E,A] \to E/[E,A] \to G \to 1$.
\item The homomorphism $\sigma:A/[E,A] \to E^{\operatorname{ab}} =
  E/[E,E]$ arises directly from the natural inclusion $A \to E$ under
  which $[E,A]$ is mapped inside $[E,E]$.
\item The homomorphism $\tau:E^{\operatorname{ab}} \to
  G^{\operatorname{ab}}$ arises from the quotient map $E \to G$ under
  which $[E,E]$ gets mapped inside $[G,G]$.
\end{itemize}

In the central extension case, the Stallings exact sequence simplifies to:

$$M(E) \stackrel{\alpha}{\to} M(G) \stackrel{\beta}{\to} A \stackrel{\sigma}{\to} E^{\operatorname{ab}} \stackrel{\tau}{\to} G^{\operatorname{ab}}$$

The maps are described as follows:

\begin{itemize}
\item The homomorphism $\alpha:M(E) \to M(G)$ arises from the
  functoriality of the Schur multiplier.
\item The homomorphism $\beta:M(G) \to A$ is the same as that
  described in Section \ref{sec:homschurkernel}.
\item The homomorphism $\sigma:A \to E^{\operatorname{ab}} = E/[E,E]$
  arises directly from the natural inclusion $A \to E$ under which
  $[E,A]$ is mapped inside $[E,E]$.
\item The homomorphism $\tau:E^{\operatorname{ab}} \to
  G^{\operatorname{ab}}$ arises from the quotient map $E \to G$ under
  which $[E,E]$ gets mapped inside $[G,G]$.
\end{itemize}

\subsection{Hopf's formula: two proofs}\label{sec:hopf-formula-proofs}

Hopf's formula states that if a group $G$ can be expressed in the form
$F/R$ where $F$ is a free group and $R$ is a normal subgroup of $F$, then:

\begin{equation}\label{eq:hopf-formula}
  M(G) \cong (R \cap [F,F])/([F,R])
\end{equation}

We provide two related proofs. The first proof relies on the
observation in Section \ref{sec:freeinitialobject}
that if we consider $E_1 = F/[F,R]$ as a central extension of $G$ in a
natural fashion, then this central extension is an initial object in
the category of group extensions of $G$ with homoclinisms. The
exterior square $G \wedge G$ is therefore isomorphic to $E_1' =
[F,F]/[F,R]$:

\begin{equation}\label{eq:exteriorsquare-hopf-formula}
  G \wedge G \cong [F,F]/[F,R]
\end{equation}

The Schur multiplier $M(G)$ is isomorphic to the kernel of the natural
homomorphism $E_1' \to G'$, which is the subgroup $(R \cap
[F,F])/[F,R]$ inside $[F,F]/[F,R]$.

Alternately, Hopf's formula can be deduced from the Stallings exact
sequence applied to the short exact sequence:

$$1 \to R \to F \to G \to 1$$

combined with the information that $M(F)$ is trivial. Explicitly,
the Stallings exact sequence is:

$$M(F) \stackrel{\alpha}{\to} M(G) \stackrel{\beta}{\to} R/[F,R] \stackrel{\sigma}{\to} F/[F,F] \stackrel{\tau}{\to} G/[G,G]$$

Since $M(F)$ is trivial, we obtain from exactness that $M(G)$ is
isomorphic to the kernel of the map $\sigma$. From this, we obtain
Hopf's formula.

\subsection{Hopf's formula: class one more version}\label{sec:hopf-formula-class-one-more}

The following is a slight variant of Hopf's formula for nilpotent
groups. Its equivalence with the version of the preceding section
(Section \ref{sec:hopf-formula-proofs}) is clear. It is
computationally somewhat more useful.

Suppose $G$ is a nilpotent group of nilpotency class $c$. Suppose $G$
can be expressed in the form $F/R$ where $F$ is a free nilpotent group
of class $c + 1$ and $R$ is a normal subgroup of $F$. Then:

\begin{equation}\label{eq:hopf-formula-class-one-more}
  M(G) \cong (R \cap [F,F])/([F,R])
\end{equation}

Also:

\begin{equation}\label{eq:exteriorsquare-hopf-formula-class-one-more}
  G \wedge G \cong [F,F]/[F,R]
\end{equation}

Free nilpotent groups are described later in more detail in Section
\ref{sec:free-nilpotent-groups-homology}.

\subsection{Exterior square of an abelian group}\label{sec:exteriorsquare-abelian-group-intro}

Suppose $G$ is an abelian group. Denote by $G \wedge_\Z G$ the
exterior square of $G$ as an abelian group. We will show here that $G
\wedge_\Z G$ is canonically isomorphic to $G \wedge G$.

For any central extension:

$$0 \to A \to E \to G \to 1$$

we know that $E$ is a group of nilpotency class at most two. Further,
the commutator map:

$$\omega_{E,G}: G \times G \to [E,E]$$

is $\mathbb{Z}$-bilinear. This observation is specific to $G$ being
abelian (see Lemma \ref{lemma:iterated-commutator-is-multilinear} in
the Appendix).

Based on this, we obtain that the map below is $\Z$-bilinear:

$$G \times G \to G \wedge G$$

Thus, it induces a map:

$$G \wedge_\Z G \to G \wedge G$$

Moreover, since $G \wedge G$ is generated by the elements $x \wedge
y$, $x,y \in G$, the map above is surjective.

The part that is {\em not} immediately obvious is that the map above
is injective. In other words, it is {\em prima facie} plausible that
there are additional relations, beyond bilinearity, that are always
satisfied in central extensions of $G$. The explicit description of
the exterior square based on generators and relations in Section
\ref{sec:exteriorsquare-explicit} will settle this point in a
straightforward manner: for an abelian group $G$, it will turn out
that the above map is an isomorphism (see Section
\ref{sec:exterior-tensor-abelian-reconciliation} for a summary of the
conclusions).

\subsection{Relationship with K-theory}

For any associative unital ring $A$, we can define a short exact
sequence:

$$0 \to K_2(A) \to \operatorname{St}(A) \to E(A) \to 1$$

Here, $\operatorname{St}(A)$ denotes the Steinberg group of $A$ and
$E(A)$ denotes the group of elementary matrices over $A$ (note that
all these groups are the direct limits of the corresponding groups for
$n \times n$ matrices under the obvious inclusion mappings). It is
known that $K_2(A) = M(E(A))$ is the Schur multiplier of $E(A)$. It
follows by unwinding the definitions that $\operatorname{St}(A)$ is
the exterior square of $E(A)$. However, we have not been able to
locate the statement (namely, that $\operatorname{St}(A)$ is the
exterior square of $E(A)$) anywhere explicitly in the K-theory
literature. This may well be because people who work in that area are
unaware of the terminology related to the exterior square.

For more information about the Steinberg group and $K_2(A)$, see
\cite{MilnorKTheory}.
%\newpage

\section{Exterior square, Schur multiplier, and the second cohomology group: Lie ring version}\label{sec:schur-multiplier-and-second-cohomology-lie}

This section covers the Lie ring analogue of the material in Section
\ref{sec:schur-multiplier-and-second-cohomology-lie}. The motivation
is largely the same. The analogous material to Section
\ref{sec:exteriorsquare-abelian-group-intro} was already covered in
Section \ref{sec:exteriorsquare-abelian-lie}, and is therefore not
repeated here.

\subsection{Homomorphism from the Schur multiplier to the kernel of the extension}\label{sec:homschurkernel-lie}

We will now describe a very important homomorphism. For any central extension of the form:

$$0 \to A \to N \to L \to 0$$

there is a natural homomorphism from the Schur multiplier of $L$ to $A$, i.e., a homomorphism:

$$\beta: M(L) \to A$$

We now proceed to describe this homomorphism.

As discussed in Section \ref{sec:exteriorsquare-lie}, there is a natural
homomorphism:

$$\Omega: L \wedge L \to [N,N]$$

Compose this with the inclusion of $[N,N]$ in $N$ to get a map $L
\wedge L \to N$. We obtain:

$$\begin{array}{ccccccccc}
0 & \to & M(L) & \to & L \wedge L & \to & [L,L] & \to & 0\\
&&               && \downarrow     &&\downarrow&&\\
0 & \to & A & \to & N & \to & L & \to & 0\\
\end{array}$$

It is immediate that this diagram commutes.

By a special case of the snake lemma, we can construct a unique map
$M(L) \to A$ such that the diagram continues to be commutative, and
that is the homomorphism $\beta$ that we want:

$$\begin{array}{ccccccccc}
0 & \to & M(L) & \to & L \wedge L & \to & [L,L] & \to & 0\\
&&   \downarrow^{\beta}  &&  \downarrow     && \downarrow&& \\
0 & \to & A &\to & N & \to & L & \to & 0\\
\end{array}$$

\subsection{Classification of extensions up to isoclinism}\label{sec:extensionsuptoisoclinism-lie}

Given a Lie ring $L$ and an abelian Lie ring $A$, we say that the central
extensions $N_1$ and $N_2$ with short exact sequences:

$$0 \to A \to N_1 \to L \to 0$$

and

$$0 \to A \to N_2 \to L \to 0$$

are {\em isoclinic (fixing both $L$ and $A$)} if there exists an
isomorphism of Lie rings $\varphi:N_1' \to N_2'$ satisfying {\em both}
the following conditions:

\begin{itemize}
\item Isoclinic as extensions of $L$: If $\Omega_1:L \wedge L \to
  N_1'$ and $\Omega_2:L \wedge L \to N_2'$ are the Lie bracket map
  homomorphisms, then $\varphi \circ \Omega_1 = \Omega_2$.
\item Suppose $B$ is the inverse image in $A$ of $[N_1,N_1]$. Then,
  $B$ is also the inverse image in $A$ of $[N_2,N_2]$. Moreover,
  composing $\varphi$ with the inclusion of $B$ in $[N_1,N_1]$ gives
  the inclusion of $B$ in $[N_2,N_2]$.
\end{itemize}

\subsection{Relating the classification of extensions up to isoclinism with the homomorphism from the Schur multiplier}\label{sec:beta-map-lie}

In Section \ref{sec:homschurkernel-lie}, we noted that for any central extension:

$$0 \to A \to N \to L \to 0$$

we have a natural homomorphism $\beta: M(L) \to A$.

The homomorphism is uniquely determined by the choice of extension up
to congruence, so we get a {\em set} map:

$$H^2(L;A) \to \operatorname{Hom}(M(L),A)$$

In Section \ref{sec:ses-uct-lie}, we will see that this set map is a {\em
  Lie ring homomorphism}. We alluded to the Lie ring structure on
$H^2_{\text{Lie}}(L;A)$ in Section
\ref{sec:second-cohomology-lie-ring-classify-extensions} and a
detailed description is in the Appendix.

As we will see in Section \ref{sec:ses-uct-lie}, this Lie ring homomorphism
is surjective and is the right map in an important short exact
sequence. For now, however, we note that this Lie ring homomorphism
classifies extensions up to isoclinism. Explicitly, two extensions
$N_1,N_2$ are isoclinic in the sense of the preceding section if and
only if they induce the same homomorphism $M(L) \to A$. We now proceed
to explain why.

Consider the two short exact sequences below:

$$\begin{array}{ccccccccc}
0 & \to & M(L) & \to & L \wedge L & \to & [L,L] & \to & 0\\
 &&   \downarrow^{\beta}  &&  \downarrow     && \downarrow&&\\
0 & \to & A &\to & N & \to & L & \to & 0\\
\end{array}$$

Suppose $B$ is the subring of $A$ that arises as the image of the
homomorphism $\beta: M(L) \to A$ and $\beta':M(L) \to B$ is the map
obtained by restricting the co-domain. We then have the following two
short exact sequences, where all the downward maps are surjective:

$$\begin{array}{ccccccccc}
0 & \to & M(L) & \to & L \wedge L & \to & [L,L] & \to & 0\\
 &&   \downarrow^{\beta'}  &&  \downarrow^{\Omega_{N,L}}     && \downarrow^{\text{id}}&& \\
0 & \to & B &\to & [N,N] & \to & [L,L] & \to & 0\\
\end{array}$$

Note that the second row sequence is exact because all the downward
maps are surjective.\footnote{Some proof details involving diagram
  chasing are being omitted for brevity.}

It is easy to see that if $N_1$ and $N_2$ are two central Lie ring
extensions of $L$ that give the same map $\beta$, then we can obtain
an isomorphism $[N_1,N_1] \to [N_2,N_2]$ such that in the diagram
below, the composite of the downward maps in the middle column is
$\Omega_{N_2,L}$, and the lower downward arrow in the middle column is
an isomorphism.

$$\begin{array}{ccccccccc}
0 & \to & M(L) & \to & L \wedge L & \to & [L,L] & \to & 0\\
 &&   \downarrow^{\beta'}  &&  \downarrow^{\Omega_{N,L}}     && \downarrow^{\text{id}}&& \\
0 & \to & B &\to & [N_1,N_1] & \to & [L,L] & \to & 0\\
 &&   \downarrow^{\operatorname{id}}  && \downarrow && \downarrow^{\text{id}} && \\
0 & \to & B &\to & [N_2,N_2] & \to & [L,L] & \to & 0\\
\end{array}$$

\subsection{The universal coefficient theorem short exact sequence}\label{sec:ses-uct-lie}

As before, let $L$ be a Lie ring and let $A$ be an abelian Lie ring. Our
goal is to understand all central extension Lie rings $N$, i.e., short
exact sequences of the following form where the image of $A$ in $N$ is
in the center of $N$:

$$0 \to A \to N \to L \to 0$$

As discussed earlier, the set of all congruence classes of extensions
is classified by the group $H^2_{\text{Lie}}(L;A)$ (the second
cohomology group for trivial Lie ring action). We now proceed to
describe a related short exact sequence, analogous to the short exact
sequence discussed in Section \ref{sec:ses-uct} for groups. The short exact
sequence is as follows:

\begin{equation}\label{eq:ses-uct-lie}
  0 \to \operatorname{Ext}^1_{\mathbb{Z}}(L^{\operatorname{ab}},A) \to H^2(L;A) \to \operatorname{Hom}(M(L),A) \to 0
\end{equation}

Note that 
\subsubsection{Interpretation of the left map of the sequence}\label{sec:ses-uct-lie-left-map}

The map:

$$\operatorname{Ext}^1_{\mathbb{Z}}(L^{\operatorname{ab}},A) \to H^2(L;A)$$

takes an abelian Lie ring extension with ideal $A$ and quotient Lie
ring $L^{\operatorname{ab}}$, and gives an extension with $L$ on top
of $A$ that can loosely be described as follows: the restriction to
the derived subring $[L,L]$ splits and the quotient sits as per the
element of
$\operatorname{Ext}^1_{\mathbb{Z}}(L^{\operatorname{ab}},A)$. Explicitly,
it can be thought of as a composite of two maps:

$$\operatorname{Ext}^1_{\mathbb{Z}}(L^{\operatorname{ab}},A) \to H^2(L^{\operatorname{ab}};A) \to H^2(L;A)$$

where the first map treats an abelian Lie ring extension simply as a
Lie ring extension, and the second map uses the contravariance of $H^2$
in its first argument.

\subsubsection{Interpretation of the right map of the sequence}\label{sec:ses-uct-lie-right-map}

The right map of the sequence:

$$H^2(L;A) \to \operatorname{Hom}(M(L),A)$$

sends an extension Lie ring to the corresponding map $\beta$ described in
Section \ref{sec:homschurkernel-lie}. In Section \ref{sec:beta-map-lie}, we
showed that the map $H^2(L;A) \to \operatorname{Hom}(M(L),A)$
classifies extensions up to isoclinism.

\subsubsection{What the existence of the short exact sequence tells us}

Consider again the short exact sequence:

\begin{equation*}
  0 \to \operatorname{Ext}^1_{\mathbb{Z}}(L^{\operatorname{ab}},A) \to H^2(L;A) \to \operatorname{Hom}(M(L),A) \to 0
\end{equation*}

We now consider the three aspects of {\em exactness}:

\begin{itemize}
\item Left exactness, i.e., the injectivity of the map
  $\operatorname{Ext}^1_{\mathbb{Z}}(L^{\operatorname{ab}},A) \to
  H^2(L;A)$. This is the assertion that the only abelian Lie ring
  extension for $L^{\operatorname{ab}}$ on top of $A$ that maps to $L
  \times A$ is the trivial extension. This is immediate from the
  definition.
\item Middle exactness, i.e., the image of the map
  $\operatorname{Ext}^1_{\mathbb{Z}}(L^{\operatorname{ab}},A) \to
  H^2(L;A)$ is precisely the same as the kernel of the map $ H^2(L;A)
  \to \operatorname{Hom}(M(L),A)$. This is easy to see from the
  definition.
\item Right exactness, i.e., the surjectivity of the map $H^2(L;A) \to
  \operatorname{Hom}(M(L),A)$. This says that {\em every homomorphism}
  from $M(L)$ to $A$ arises from an extension with central subring
  $A$ and quotient Lie ring $L$. In other words, the set of extension
  types up to isoclinism can be identified with the group
  $\operatorname{Hom}(M(L),A)$. {\em This is the most important and
    least obvious of the three exactness statements}. Many of our
  later constructive results will rely crucially on right exactness.
\end{itemize}

\subsubsection{How it is a special case of the dual universal coefficient theorem}

The general version of the dual universal coefficient theorem for
Lie ring cohomology is as follows:

$$0 \to \operatorname{Ext}^1_{\mathbb{Z}}(H_{k-1}(L;\mathbb{Z}),A) \to H^k(L;A) \to \operatorname{Hom}(H_k(L;\mathbb{Z}),A) \to 0$$

If we set $p = 2$ and use the fact that $M(L)$ is canonically
isomorphic to $H_2(L;\mathbb{Z})$, and also that
$L^{\operatorname{ab}}$ is canonically isomorphic to
$H_1(L;\mathbb{Z})$, we get the short exact sequence we have been
discussing.

\subsubsection{The splitting of the short exact sequence}

The dual universal coefficient theorem for Lie ring cohomology, in addition
to providing the short exact sequence above, also states that the
short exact sequence always splits, but the splitting is not in
general canonical. Explicitly, the universal coefficient theorem
states that:

$$H^k(L;A) \cong \operatorname{Ext}^1_{\mathbb{Z}}(H_{k-1}(L;\mathbb{Z}),A) \oplus \operatorname{Hom}(H_k(L;\mathbb{Z}),A)$$

In the special case of interest to us, we obtain:

$$H^2(L;A) \cong \operatorname{Ext}^1_{\mathbb{Z}}(L^{\operatorname{ab}},A) \oplus \operatorname{Hom}(M(L),A)$$

The direct sum decomposition is {\em not} in general canonical. %% In
%% fact, there are examples where there is {\em no}
%% $\operatorname{Aut}(L) \times \operatorname{Aut}(A)$-invariant
%% splitting. {\em TONOTDO: Insert link to Appendix section}

In Section \ref{sec:ses-uct-lie-abelian-canonical-splitting}, we will note that the
short exact sequence splits canonically in the case that $L$ itself is
an abelian Lie ring.

\subsection{An alternate characterization of initial objects, and the existence of Schur covering Lie rings}

Recall that, by Lemma \ref{lemma:initobj-lie}, a central extension:

$$0 \to A \to N \to L \to 0$$

is an initial object in the category of central extensions of $L$ with
homoclinisms if the natural homomorphism:

$$\Omega_{N,L}: L \wedge L \to [N,N]$$

is an isomorphism. We now provide an alternative characterization.

\begin{lemma}\label{lemma:initial-beta-injective-lie}
  Consider a Lie ring $L$ and a central extension:

  $$0 \to A \to N \to L \to 0$$

  The central extension is an initial object in the category of
  central extensions of $L$ with homoclinisms if and only if the
  corresponding homomorphism $\beta:M(L) \to A$ (described in Section
  \ref{sec:homschurkernel}) is injective.
\end{lemma}

\begin{proof}
  Let $B$ be the image in $A$ of $\beta$ and let $\beta'$ be the
  restriction of $\beta$ to co-domain $B$, so $\beta'$ is a surjective
  homomorphism from $M(L)$ to $B$. Note also that $\beta$ is injective
  if and only if $\beta'$ is an isomorphism.

  As discussed in Section \ref{sec:beta-map}, we have the following
  morphism of short exact sequences, where all the downward maps are
  surjective:

  $$\begin{array}{ccccccccc}
    0 & \to & M(L) & \to & L \wedge L & \to & [L,L] & \to & 0\\
    &&   \downarrow^{\beta'}  &&  \downarrow^{\Omega_{N,L}}     && \downarrow^{\text{id}} && \\
    0 & \to & B &\to & [N,N] & \to & [L,L] & \to & 0\\
  \end{array}$$
  
  Since the right-most downward map is the identity map, we see (from
  some elementary diagram chasing) that $\beta'$ is an isomorphism if
  and only if $\Omega_{N,L}$ is an isomorphism. 
\end{proof}

Recall the definition of stem extension from Section
\ref{sec:central-and-stem-extension-lie}. We provide an alternative
characterization of such extensions:

\begin{lemma}\label{lemma:stem-beta-surjective-lie}
  A central extension $0 \to A \to N \to L \to 0$ is a stem extension
  if and only if the corresponding map $\beta: M(L) \to A$ is
  surjective.
\end{lemma}

\begin{proof}
  Let $B$ be the image in $A$ of $\beta$ and let $\beta'$ be the
  restriction of $\beta$ to co-domain $B$, so $\beta'$ is a surjective
  homomorphism from $M(L)$ to $B$. Note also that $\beta$ is
  surjective if and only if $B = A$. Note also that $A \le Z(N)$ by
  the assumption that the extension is central, so the challenge is to
  show that $A \le [N,N]$ if and only if $B = A$.

  As described in Section \ref{sec:beta-map-lie}, we have the following
  morphism of two short exact sequences, with all the downward maps
  surjective:

  $$\begin{array}{ccccccccc}
    0 & \to & M(L) & \to & L \wedge L & \to & [L,L] & \to & 0\\
    &&   \downarrow^{\beta'}  &&  \downarrow^{\Omega_{N,L}}     && \downarrow^{\text{id}} && \\
    0 & \to & B &\to & [N,N] & \to & [L,L] & \to & 0\\
  \end{array}$$

  Explicitly, $B$ is the kernel of the homomorphism from $[N,N]$ to
  $[L,L]$. The homomorphism from $[N,N]$ to $[L,L]$ is obtained by
  restricting to $[N,N]$ the homomorphism from $N$ to $L$. 

  Thus, $B = A \cap [N,N]$. It follows that $A \le [N,N]$ if and only
  if $B = A$, completing the proof.
\end{proof}

We are now prepared for a definition.

\begin{definer}[Schur covering Lie ring]
  We define a {\em Schur covering Lie ring} of $L$ as a Lie ring
  extension $N$ of $L$ with short exact sequence:

  $$0 \to A \to N \to L \to 0$$
  
  satisfying the condition that it is a central extension and the
  corresponding map $\beta: M(L) \to A$ (defined in Section
  \ref{sec:homschurkernel-lie}) is an isomorphism. Equivalently, the
  extension must satisfy {\em both} these conditions:
  
  \begin{itemize}
  \item The extension is a stem extension, i.e., the image of $A$ in $N$
    is contained in $Z(N) \cap [N,N]$.
  \item The natural homomorphism $\Omega_{N,L}: L \wedge L \to [N,N]$ is an
    isomorphism.
  \end{itemize}
\end{definer}
  
The equivalence of the two versions of the definition follows from the
two preceding lemmas (Lemmas \ref{lemma:initial-beta-injective-lie} and
\ref{lemma:stem-beta-surjective-lie}).

The existence of Schur covering Lie rings is not {\em a priori} clear,
but can be deduced from the short exact sequence of the preceding
section. 

\begin{theorem}\label{thm:schur-covering-lie-rings-exist}
  For any Lie ring $L$, Schur covering Lie rings of $L$ exist.
\end{theorem}

\begin{proof}
  For any abelian Lie ring $A$, we have the short exact sequence
  described in Section \ref{sec:ses-uct}:

  \begin{equation*}
    0 \to \operatorname{Ext}^1_{\mathbb{Z}}(L^{\operatorname{ab}},A) \to H^2(L;A) \to \operatorname{Hom}(M(L),A) \to 0
  \end{equation*}
  
  Now, set $A = M(L)$:

  \begin{equation*}
    0 \to \operatorname{Ext}^1_{\mathbb{Z}}(L^{\operatorname{ab}},M(L)) \to H^2(L;M(L)) \to \operatorname{Hom}(M(L),M(L)) \to 0
  \end{equation*}
  
  Consider the element $\operatorname{Id}_{M(L)} \in
  \operatorname{Hom}(M(L),M(L))$. By surjectivity (i.e., right
  exactness), there exists at least one element of $H^2(L;M(L))$ that
  maps to this. Note that the inverse image is in fact a coset in
  $H^2(L;M(L))$ of the image of
  $\operatorname{Ext}^1_{\mathbb{Z}}(L^{\operatorname{ab}},M(L))$. Each
  element in this inverse image corresponds to a Schur covering
  Lie ring. If
  $\operatorname{Ext}^1_{\mathbb{Z}}(L^{\operatorname{ab}},M(L))$ is
  nontrivial, the Schur covering Lie ring need not be unique.
\end{proof}

\subsection{Realizability of surjective homomorphisms from the exterior square}

Suppose $L$ and $D$ are Lie rings and $\alpha:L \wedge L \to D$ and
$\delta:D \to [L,L]$ are surjective homomorphisms such that $\delta
\circ \alpha:L \wedge L \to [L,L]$ is the canonical map sending $x
\wedge y$ to $[x,y]$. We would like to know whether there is a central
extension:

$$0 \to A \to N \to L \to 0$$

with the property that there is an isomorphism $\theta:N' \to D$ such
that if we consider the homomorphism:

$$\Omega_{N,L}: L \wedge L \to N'$$

then $\theta \circ \Omega_{N,L} = \alpha$. The answer to this question is
{\em yes}. In fact, we can even choose $N$ to be a {\em stem}
extension of $L$. We outline the construction below.

Recall that we have the following short exact sequence, describing $L
\wedge L$ as a central extension of $[L,L]$:

$$0 \to M(L) \to L \wedge L \to [L,L] \to 0$$

Denote by $A$ the image of $M(L)$ under the set map $\alpha:L \wedge L
\to D$ and by $\beta:M(L) \to A$ the restricted map. We therefore
have the following commutative diagram:

$$\begin{array}{ccccccccc}
  0 & \to & M(L) & \to & L \wedge L & \to & [L,L] & \to & 0\\
  &&   \downarrow^{\beta}  &&  \downarrow^{\alpha}     && \downarrow^{\text{id}} && \\
  0 & \to & A &\to & D & \stackrel{\delta}{\to} & [L,L] & \to & 0\\
\end{array}$$

Now, consider the short exact sequence described in Section
\ref{sec:ses-uct-lie}:

$$0 \to \operatorname{Ext}^1_{\mathbb{Z}}(L^{\operatorname{ab}},A) \to H^2(L;A) \to \operatorname{Hom}(M(L),A) \to 0$$

The right map is surjective, so there exists an extension Lie ring $N$
(corresponding to an element of $H^2(L,A)$) such that the map $\beta$
corresponding to $N$ (as described in Sections
\ref{sec:homschurkernel-lie} and \ref{sec:beta-map-lie}) is the map $\beta$
that we specified. Also, for reasons discussed in Section
\ref{sec:beta-map-lie}, we can find an isomorphism $\theta:[N,N] \to D$
such that $\theta \circ \Omega_{N,L} = \alpha$.

\subsection{The existence of stem Lie rings}\label{sec:stem-lie-ring-existence}

Define a Lie ring $L$ as a {\em stem Lie ring} if $Z(L) \le L'$. Now
that we have defined the concept of stem extension, we can provide an
alternate definition of stem Lie ring: $L$ is a stem Lie ring if the
short exact sequence:

$$0 \to Z(L) \to L \to L/Z(L) \to 0$$

makes $L$ a stem extension.

\begin{theorem}\label{thm:stem-lie-ring-existence}
  Suppose $L$ is a Lie ring. There exists a stem Lie ring $K$ that is
  isoclinic to $L$.
\end{theorem}

\begin{proof}
  Consider the short exact sequence:

  $$0 \to Z(L) \to L \to L/Z(L) \to 0$$

  This short exact sequence allows us to think of $L$ as a central
  extension with central subring $Z(L)$ and quotient Lie ring
  $L/Z(L)$. We apply the construction in Section \ref{sec:beta-map-lie} to
  obtain the natural map $\beta:M(L/Z(L)) \to Z(L)$. Let $B$ be the
  image of $\beta$. By the explicit construction, note that the image of
  $\beta$ is actually inside $Z(L) \cap L'$. Let $\beta':M(L/Z(L)) \to
  B$ be the map obtained by restricting the co-domain of $\beta$ to
  the image of $\beta$.

  Consider now the short exact sequence of Section \ref{sec:ses-uct-lie}
  for central subring $B$ and quotient Lie ring $L/Z(L)$. The short exact sequence is:

  $$0 \to \operatorname{Ext}^1_{\mathbb{Z}}((L/Z(L))^{\operatorname{ab}},B) \to H^2(L/Z(L);B) \to \operatorname{Hom}(M(L/Z(L)),B) \to 0$$

  In particular, the map:

  $$H^2(L/Z(L);B) \to \operatorname{Hom}(M(L/Z(L)),B)$$

  is surjective. This means that we can find a (possibly non-unique
  and non-canonical) central extension $K$ with short exact sequence:

  $$0 \to B \to K \to L/Z(L) \to 0$$

  such that the map $\beta_K$ corresponding to this extension (per
  Section \ref{sec:beta-map}) is the same as $\beta'$. The following
  are now easy to verify: %{\em TONOTDO: Fill in details}

  \begin{itemize}
  \item The image of $B$ in $K$ is the center of $K$.
  \item The image of $B$ in $K$ is contained in the derived subring $K'$.
  \item $L$ and $K$ are isoclinic.
  \end{itemize}
\end{proof}
  
Note that although stem Lie rings exist, there may be no canonical choice
of stem Lie ring. The problem is the absence of a canonical splitting of
the short exact sequence. If a canonical splitting did exist, we could
use that splitting to obtain a canonical choice of extension.

\subsection{The Stallings exact sequence}

The {\em Stallings exact sequence} was defined by Stallings in
\cite{Stallings} and explored further by Eckmann, Hilton, and
Stammbach in \cite{EckmannHiltonStammbach} for arbitrary Lie ring
extensions.

Start with a short exact sequence of Lie rings (note that $A$ is not
necessarily abelian, but we use this notation to stay consistent with
the other sections):

$$1 \to A \to N \to L \to 0$$

Then, the Stallings exact sequence is as follows:

$$M(N) \stackrel{\alpha}{\to} M(L) \stackrel{\beta}{\to} A/[N,A] \stackrel{\sigma}{\to} N^{\operatorname{ab}} \stackrel{\tau}{\to} L^{\operatorname{ab}}$$

The maps are described as follows:

\begin{itemize}
\item The homomorphism $\alpha:M(N) \to M(L)$ arises from the
  functoriality of the Schur multiplier, discussed in Section
  \ref{sec:functoriality}.
\item The homomorphism $\beta:M(L) \to A/[N,A]$ arises from the
  corresponding map in the central extension case (discussed below)
  once we replace the original short exact sequence by the short exact
  sequence $1 \to A/[N,A] \to N/[N,A] \to L \to 0$.
\item The homomorphism $\sigma:A/[N,A] \to N^{\operatorname{ab}} =
  N/[N,N]$ arises directly from the natural inclusion $A \to N$ under
  which $[N,A]$ is mapped inside $[N,N]$.
\item The homomorphism $\tau:N^{\operatorname{ab}} \to
  L^{\operatorname{ab}}$ arises from the quotient map $N \to L$ under
  which $[N,N]$ gets mapped inside $[L,L]$.
\end{itemize}

In the central extension case, the Stallings exact sequence simplifies to:

$$M(N) \stackrel{\alpha}{\to} M(L) \stackrel{\beta}{\to} A \stackrel{\sigma}{\to} N^{\operatorname{ab}} \stackrel{\tau}{\to} L^{\operatorname{ab}}$$

The maps are described as follows:

\begin{itemize}
\item The homomorphism $\alpha:M(N) \to M(L)$ arises from the
  functoriality of the Schur multiplier.
\item The homomorphism $\beta:M(L) \to A$ is the same as that
  described in Section \ref{sec:homschurkernel}.
\item The homomorphism $\sigma:A \to N^{\operatorname{ab}} = N/[N,N]$
  arises directly from the natural inclusion $A \to N$ under which
  $[N,A]$ is mapped inside $[N,N]$.
\item The homomorphism $\tau:N^{\operatorname{ab}} \to
  L^{\operatorname{ab}}$ arises from the quotient map $N \to L$ under
  which $[N,N]$ gets mapped inside $[L,L]$.
\end{itemize}

\subsection{Hopf's formula for Lie rings: two proofs}\label{sec:hopf-formula-proofs-lie}

Hopf's formula for Lie rings states that if a Lie ring $L$ can be
expressed in the form $F/R$ where $F$ is a free Lie ring and $R$ is an
ideal of $F$, then:

\begin{equation}\label{eq:hopf-formula-lie}
  M(L) \cong (R \cap [F,F])/([F,R])
\end{equation}

We provide two related proofs. The first proof relies on the
observation in Section \ref{sec:freeinitialobject-lie}
that if we consider $N_1 = F/[F,R]$ as a central extension of $L$ in a
natural fashion, then this central extension is an initial object in
the category of Lie ring extensions of $L$ with homoclinisms. The
exterior square $L \wedge L$ is therefore isomorphic to $N_1' =
[F,F]/[F,R]$. The Schur multiplier $M(L)$ is isomorphic to the kernel
of the natural homomorphism $N_1' \to L'$, which is the subring $(R
\cap [F,F])/[F,R]$ inside $[F,F]/[F,R]$.

Alternately, Hopf's formula can be deduced from the Stallings exact
sequence applied to the short exact sequence:

$$1 \to R \to F \to L \to 0$$

combined with the information that $M(F)$ is trivial. Explicitly,
the Stallings exact sequence is:

$$M(F) \stackrel{\alpha}{\to} M(L) \stackrel{\beta}{\to} R/[F,R] \stackrel{\sigma}{\to} F/[F,F] \stackrel{\tau}{\to} L/[L,L]$$

Since $M(F)$ is trivial, we obtain from exactness that $M(L)$ is
isomorphic to the kernel of the map $\sigma$. From this, we obtain
Hopf's formula.

%\newpage

\section{Exterior and tensor product for groups: explicit descriptions}\label{sec:exterior-and-tensor-product}

The treatment of tensor products and exterior products found here is
similar to that found in \cite{BrownLoday}, \cite{McDermottThesis},
and \cite{Ellis87}.

In keeping with the literature on the topic, we use the convention of
groups acting on the left. In particular, when talking of a
conjugation action, we refer to the action $(g,x) \mapsto {}^gx = gxg^{-1}$.

The material included in this section can be skipped. Its relevance is
primarily that it shows, as a special case, that the exterior square
of an abelian group {\em as a group} is the same as its exterior
square {\em as an abelian group}. We alluded to this, without proof,
in Section \ref{sec:exteriorsquare-abelian-group-intro}. However, we
also provide an alternative proof in Section
\ref{sec:exteriorsquare-abelian-group-proofs}.\footnote{The proofs are
  not really different once we write down all the details.}

\subsection{Compatible pair of actions}

Suppose $G$ and $H$ are groups and $\alpha: G \to
\operatorname{Aut}(H)$ and $\beta: H \to \operatorname{Aut}(G)$ are
group homomorphisms. We say that $(\alpha,\beta)$ form a {\em
  compatible pair of actions} if {\em both} the following conditions
hold:

\begin{eqnarray*}
  \beta(\alpha(g_1)(h))(g_2) & = & {}^{g_1}(\beta(h)({}^{g_1^{-1}}(g_2)))) \ \forall \ g_1,g_2 \in G, h \in H\\
  \alpha(\beta(h_1)(g))(h_2) & = & {}^{h_1}(\alpha(g)({}^{h_1^{-1}}(h_2))) \ \forall \ h_1,h_2 \in H, g \in G\\
\end{eqnarray*}

If we use $\cdot$ to denote the action of each group on itself by
conjugation {\em and} both the actions $\alpha$ and $\beta$, the above
can be written as:

\begin{eqnarray*}
  (g_1 \cdot h) \cdot g_2 & = & g_1 \cdot (h \cdot (g_1^{-1} \cdot g_2)) \ \forall \ g_1,g_2 \in G, h \in H\\
  (h_1 \cdot g) \cdot h_2 & = & h_1 \cdot (g \cdot (h_1^{-1} \cdot h_2)) \ \forall \ h_1,h_2 \in H, g \in G\\
\end{eqnarray*}

The following is an alternate description of the axioms that is
sometimes easier to work with:

\begin{eqnarray*}
  {}^{g_1}(\beta(h)g_2) & = & \beta(\alpha(g_1)h)({}^{g_1}(g_2)) \ \forall \ g_1,g_2 \in G, h \in H\\ 
  {}^{h_1}(\alpha(g)h_2) & = & \alpha(\beta(h_1)g)({}^{h_1}(h_2)) \ \forall \ g \in G, h_1, h_2 \in H\\
\end{eqnarray*}

With the $\cdot$ notation, this becomes:

\begin{eqnarray*}
  g_1 \cdot (h \cdot g_2) & = & (g_1 \cdot h) \cdot (g_1 \cdot g_2) \ \forall \ g_1,g_2 \in G, h \in H\\
  h_1 \cdot (g \cdot h_2) & = & (h_1 \cdot g) \cdot (h_1 \cdot h_2) \ \forall g \in G, h_1,h_2 \in H\\
\end{eqnarray*}

The $g_2$ of the first identity here equals the element $g_1^{-1}
\cdot g_2$ of the preceding formulation. The $h_2$ of the second
identity equals the element $h_1^{-1} \cdot h_2$ of the
preceding formulation.

The following are examples of compatible pairs of actions:

\begin{itemize}
\item The trivial pair of actions is compatible. By ``trivial pair of
  actions'' we mean that both the homomorphisms $\alpha:G \to
  \operatorname{Aut}(H)$ and $\beta:H \to \operatorname{Aut}(G)$ are
  trivial homomorphisms.
\item For a group $G$, setting $G = H$ and taking both actions to be
  the action of a group on itself by conjugation gives a compatible
  pair of actions.
\item This generalizes both the preceding examples: if $G$ and $H$ can
  be embedded as subgroups inside a group $Q$ such that $G$ and $H$
  normalize each other in $Q$, then the actions of $G$ and $H$ on each
  other by conjugation are compatible. Note that this generalizes the
  trivial pair of actions because we can set $Q = G \times H$. It
  generalizes the action of a group on itself by conjugation because,
  if $G = H$, we can set $Q = G = H$.
\end{itemize}

\subsection{Tensor product for a compatible pair of actions}\label{sec:tensorproduct-explicit}

Suppose $G$ and $H$ are groups and $\alpha:G \to
\operatorname{Aut}(H)$, $\beta:H \to \operatorname{Aut}(G)$ form a
compatible pair of actions. For simplicity of notation, we will use
$\cdot$ to denote the action of each group on itself by conjugation
{\em and} both the actions $\alpha$ and $\beta$.

The {\em tensor product} of $G$ and $H$ for this compatible pair of
actions, denoted $G \otimes H$, is the quotient of the free group on
the set $\{ g \otimes h \mid g \in G, h \in H \}$ by the following
relations:

\begin{eqnarray*}
  (g_1g_2) \otimes h = ((g_1 \cdot g_2) \otimes (g_1 \cdot h))(g_1 \otimes h) \ \forall \ g_1,g_2 \in G, h \in H\\
  g \otimes (h_1h_2) = (g \otimes h_1)((h_1 \cdot g) \otimes (h_1 \cdot h_2)) \ \forall \ g \in G, h_1,h_2 \in H\\
\end{eqnarray*}

\subsection{Exterior product of normal subgroups of a group}

Suppose $G$ and $H$ are subgroups of a group $Q$ such that $G$ and $H$
both normalize each other. Then, the actions of $G$ and $H$ on each
other by conjugation form a compatible pair of actions. Note that we
can assume without loss of generality that $G$ and $H$ are both normal
in $Q$, because if not, then $Q$ can be replaced by the subgroup $GH$
of $Q$ and the rest of the construction is unaffected.

We define the {\em exterior product} $G \wedge H$ as the quotient of
the tensor product $G \otimes H$ by the normal subgroup generated by
the set $\{ x \otimes x \mid x \in G \cap H \}$. The image of $g
\otimes h$ is denoted $g \wedge h$.

\subsection{Tensor square and exterior square of a group}\label{sec:exteriorsquare-explicit}

Let $G$ be a group. The {\em tensor square} of $G$, denoted $G \otimes
G$ or $\bigotimes^2G$, is defined as the tensor product of $G$ with
itself for the compatible pair of actions where both actions equal the
action of $G$ on itself by conjugation.

The {\em exterior square} of $G$, denoted $G \wedge G$ or
$\bigwedge^2G$, is defined as the exterior product of $G$ and $G$
where both copies of $G$ are viewed as normal subgroups inside
$G$. Explicitly, $G = H = Q$ in the notation used in the preceding
subsection.

Alternately the exterior square of $G$ can be defined as the quotient
of the tensor square of $G$ by the normal subgroup generated by the
subset $\{ g \otimes g \mid g \in G \}$.

\subsection{Reconciling the definitions of exterior square}\label{sec:exteriorsquare-reconciliation}

Clair Miller, in her 1952 paper \cite{Miller52} introducing the
concept of the exterior square, proved the equivalence of the two
descriptions of exterior square.\footnote{This is a somewhat
  hard-to-verify statement, since Miller's paper uses different
  terminology and language from what we use.} A later paper
\cite{Ellis93} by Graham J. Ellis, published in 1993, discussed the
matter and related questions in considerably greater detail.

\subsection{The special case of abelian groups}\label{sec:exterior-tensor-abelian-reconciliation}

There are pre-existing concepts of tensor product, tensor square, and
exterior square for abelian groups. These coincide with our general
definitions above when both definitions make sense. Explicitly, the
following are true and can be readily verified from the definitions
above. We will $\otimes_\Z$ and $\wedge_\Z$ to denote tensor and
exterior product operations {\em as abelian groups}.

\begin{itemize}
\item The tensor product $G \otimes H$ for the trivial pair of actions
  of $G$ and $H$ on each other is an abelian group that is canonically
  isomorphic to the tensor product of abelian groups
  $G^{\operatorname{ab}} \otimes_\Z H^{\operatorname{ab}}$.
\item In particular, the tensor square $G \otimes G$ for an abelian
  group $G$ agrees with its tensor square {\em as an abelian group},
  i.e., $G \otimes_\Z G \cong G \otimes G$.
\item The exterior square $G \wedge G$ for an abelian group agrees
  with its exterior square {\em as an abelian group}, i.e., $G
  \wedge_\Z G \cong G \wedge G$.
\end{itemize}

%\newpage

\section{Exterior and tensor product for Lie rings: explicit descriptions}\label{sec:exterior-and-tensor-product-lie}

This section does for Lie rings what the preceding section (Section
\ref{sec:exterior-and-tensor-product}) did for groups.

Our treatment here closely follows the 1989 paper \cite{EllisLie} by
Graham Ellis. Proofs of unproved assertions here can be found in the
paper.

The section can be skipped without any loss of continuity.

\subsection{Compatible pair of actions of Lie rings}

Suppose $M$ and $N$ are Lie rings. Suppose $\alpha:M \to
\operatorname{Der}(N)$ and $\beta:N \to \operatorname{Der}(M)$ are
homomorphism of Lie rings, where
$\operatorname{Der}(M)$ and $\operatorname{Der}(N)$ denote the Lie
ring of derivations of $M$ and of $N$ respectively. We say that
$\alpha,\beta$ form a {\em compatible pair of actions} if the
following two conditions hold:

\begin{eqnarray*}
  \alpha(\beta(n_1)m)(n_2) & = & [n_2,\alpha(m)n_1] \ \forall \ m \in M, n_1,n_2 \in N\\
  \beta(\alpha(m_1)n)(m_2) & = & [m_2,\beta(n)m_1] \ \forall \ m_1,m_2 \in M, n \in N\\
\end{eqnarray*}

The above expressions are easier to write down if we use $\cdot$ to denote the actions. In this case, the above become:

\begin{eqnarray*}
  (n_1 \cdot m) \cdot n_2 & = & [n_2,m \cdot n_1] \ \forall \ m \in M, n_1,n_2 \in N\\
  (m_1 \cdot n) \cdot m_2 & = & [m_2, n \cdot m_1] \ \forall \ m_1,m_2 \in M, n \in N \\
\end{eqnarray*}

The following are true:

\begin{itemize}
\item For any Lie ring $L$, the adjoint action of $L$ on itself forms
  a compatible pair of actions with itself.
\item For any Lie rings $M$ and $N$, the trivial Lie ring actions of
  $M$ and $N$ on each other form a compatible pair of actions.
\item The following generalizes the preceding two examples: if $M$ and
  $N$ can be embedded as ideals inside a Lie ring $Q$, then the
  adjoint actions of $M$ and $N$ on each other form a compatible pair
  of actions. Note that it suffices to assume that $M$ and $N$ are Lie
  subrings that idealize each other, but there is no loss of generality
  since we can replace $Q$ by the subring $M + N$.
\end{itemize}

\subsection{Tensor product of Lie rings}

Suppose $M$ and $N$ are Lie rings and $\alpha:M \to
\operatorname{Der}(N)$ and $\beta:N \to \operatorname{Der}(M)$ is a
compatible pair of actions of Lie rings. We define the {\em tensor
  product} $M \otimes N$ for this pair of actions as follows. It is
the quotient of the free Lie ring on formal symbols of the form $m
\otimes n$ ($m \in M, n \in N$) by the following relations:

\begin{enumerate}
\item Additive in $M$: $(m_1 + m_2) \otimes n = (m_1 \otimes n) + (m_2
  \otimes n) \ \forall \ m_1,m_2 \in M, n \in N$. Note that if we are
  dealing with Lie algebras instead of Lie rings, we will replace
  additivity by ''linearity'' in $M$ with respect to the ground ring.
\item Additive in $N$: $m \otimes (n_1 + n_2) = (m \otimes n_1) + (m \otimes n_2) \ \forall \ m \in M, n_1,n_2 \in N$. Note that if we are dealing with Lie algebras instead of Lie rings, we will replace additivity by ''linearity'' in $N$ with respect to the ground ring.
\item Expanding a tensor product involving one Lie bracket: 
  \begin{itemize}
  \item $[m_1,m_2] \otimes n = m_1 \otimes \alpha(m_2)(n) - m_2 \otimes
    \alpha(m_1)(n) \ \forall \ m_1,m_2 \in M, n \in N$
  \item $m \otimes [n_1,n_2] = \beta(n_2)m \otimes n_1 - \beta(n_1)(m)
    \otimes n_2 \ \forall m \in M, n_1,n_2 \in N$
  \end{itemize}
  If both the actions are rewritten using $\cdot$, this simplifies to:
  
  \begin{itemize}
  \item $[m_1,m_2] \otimes n = m_1 \otimes (m_2 \cdot n) - m_2 \otimes (m_1 \cdot n) \ \forall \ m_1,m_2 \in M, n \in N$
  \item $m \otimes [n_1,n_2] = (n_2 \cdot m) \otimes n_1 - (n_1 \cdot m) \otimes n_2 \ \forall m \in M, n_1,n_2 \in N$
  \end{itemize}
\item Expanding a Lie bracket of two pure tensors:

  $$[(m_1 \otimes n_1),(m_2 \otimes n_2)] = -(\beta(n_1)(m_1)) \otimes (\alpha(m_2)(n_2))$$
\item If both the actions are rewritten using $\cdot$, this becomes:

  $$[(m_1 \otimes n_1),(m_2 \otimes n_2)] = -(n_1 \cdot m_1) \otimes (m_2 \cdot n_2)$$
\end{enumerate}

\subsection{Exterior product of Lie rings}\label{sec:exteriorsquare-explicit-lie}

Suppose $M,N$ are (possibly equal, possibly distinct) ideals in a Lie
ring $Q$. Note that in fact it suffices to assume that they idealize
each other, but there is no loss of generality in assuming that they
are both ideals because we could replace $Q$ by $M + N$. 

Define a compatible pair of actions of Lie rings of $M$ and $N$ on
each other via the adjoint action on each other, i.e., the action that
each induces on the other by restricting the inner derivation given by
the adjoint action in the whole Lie ring. The exterior product of $M$ and
$N$ is then defined as the quotient of the tensor product of Lie rings
$M \otimes N$ for this compatible pair of actions by the ideal
generated by elements of the form $x \otimes x, x \in M \cap N$.

\subsection{Tensor square and exterior square of a Lie ring}

Let $L$ be a Lie ring. The {\em tensor square} of $L$, denoted $L \otimes
L$ or $\bigotimes^2L$, is defined as the tensor product of $L$ with
itself for the compatible pair of actions where both actions equal the
adjoint action of $L$ on itself.

The {\em exterior square} of $L$, denoted $L \wedge L$ or
$\bigwedge^2L$, is defined as the exterior product of $L$ and $L$
where both copies of $L$ are viewed as ideals inside $L$. Explicitly,
$M = N = Q = L$ in the notation used in the preceding subsection.

Alternately the exterior square of $L$ can be defined as the quotient
of the tensor square of $L$ by the ideal generated by the
subset $\{ x \otimes x \mid x \in L \}$.

%\newpage

\section{Free nilpotent groups: basic facts about their homology groups}\label{sec:free-nilpotent-groups-homology}

\subsection{Free nilpotent group: definition}

The free nilpotent group of class $c$ on a set $S$ can be defined as
the free algebra on $S$ in the variety of groups of nilpotency class
at most $c$. Below is an explicit definition in terms of the free
group.

\begin{definer}[Free nilpotent group]
  Suppose $S$ is a set and $c$ is a positive integer. The {\em free
    nilpotent group} of class $c$ on the set $S$ is defined as the
  quotient group $F(S)/\gamma_{c+1}(F(S))$ where $F(S)$ is the free
  group on $S$. Equivalently, this group, along with the set map to it
  from $S$, is the initial object in the category of groups of
  nilpotency class at most $c$ with set maps to them from $S$.

  The functor sending a set to the free nilpotent group of class $c$
  is left adjoint to the forgetful functor from nilpotent groups of
  class $c$ to sets.\footnote{This means that given a set $S$ and a
    group $G$ of nilpotency class at most $c$, there is a canonical
    bijection between the set of {\em set maps} from $S$ to $G$ and
    the set of {\em group homomorphisms} from $F(S)$ to $G$. For more
    on adjoint functors, see the Appendix, Section
    \ref{appsec:adjoint}.}
\end{definer}

\subsection{Homology of free nilpotent groups}

Suppose $G$ is the free nilpotent group of class $c$ on a generating
set $S$. $G$ can be naturally identified with $F/\gamma_{c+1}(F)$
where $F$ is the {\em free group} of class $c$ (i.e., $F$ is a free
algebra in the variety of groups). We wish to compute the homology of
$G$.

Setting $R = \gamma_{c+1}(F)$ and working out the details as discussed
in Section \ref{sec:hopf-formula-proofs}, we obtain:

\begin{itemize}
\item The group $[F,R]$ equals $[F,\gamma_{c+1}(F)] = \gamma_{c+2}(F)$.
\item The group $E = F/[F,R]$, with the natural quotient map $E \to
  G$, is an initial object in the category of central extensions of
  $G$ with homoclinisms. Note that $E$ is a free nilpotent group of
  class $c + 1$ on the same generating set $S$.
\item The exterior square $G \wedge G$ is canonically isomorphic to
  $[E,E]$, or equivalently, to $[F,F]/[F,R] =
  \gamma_2(F)/\gamma_{c+2}(F)$.
\item The Schur multiplier $M(G)$ is canonically isomorphic to the
  quotient group $(R \cap [F,F])/[F,R] =
  \gamma_{c+1}(F)/\gamma_{c+2}(F)$.
\item The canonical short exact sequence:

  $$0 \to M(G) \to G \wedge G \to [G,G] \to 1$$

  is isomorphic to the short exact sequence:
  
  $$0 \to \gamma_{c+1}(F)/\gamma_{c+2}(F) \to \gamma_2(F)/\gamma_{c+2}(F) \to \gamma_2(F)/\gamma_{c+1}(F) \to 1$$
\end{itemize}

%\newpage

\section{Free nilpotent Lie rings: basic facts about their homology groups}\label{sec:free-nilpotent-lie-homology}

\subsection{Free nilpotent Lie ring: definition}

The free nilpotent Lie ring of class $c$ on a set $S$ can be defined
as the free algebra on $S$ in the variety of Lie rings of nilpotency
Lie rings at most $c$. Below is an explicit definition in terms of the
free group.

\begin{definer}[Free nilpotent Lie ring]
  Suppose $S$ is a set and $c$ is a positive integer. The {\em free
    nilpotent } of class $c$ on the set $S$ is defined as the
  quotient Lie ring $F(S)/\gamma_{c+1}(F(S))$ where $F(S)$ is the free
  Lie ring on $S$. Equivalently, this Lie ring, along with the set map to it
  from $S$, is the initial object in the category of Lie rings of
  nilpotency class at most $c$ with set maps to them from $S$.

  The functor sending a set to the free nilpotent Lie ring of class $c$
  is left adjoint to the forgetful functor from nilpotent Lie rings of
  class $c$ to sets.\footnote{This means that given a set $S$ and a
    Lie ring $L$ of nilpotency class at most $c$, there is a canonical
    bijection between the set of {\em set maps} from $S$ to $L$ and
    the set of {\em Lie ring homomorphisms} from $F(S)$ to $L$.}
\end{definer}

\subsection{Homology of free nilpotent Lie rings}

Suppose $L$ is the free nilpotent Lie ring of class $c$ on a generating
set $S$. $L$ can be naturally identified with $F/\gamma_{c+1}(F)$
where $F$ is the {\em free Lie ring} of class $c$ (i.e., $F$ is a free
algebra in the variety of Lie rings). We wish to compute the homology of
$L$.

Setting $R = \gamma_{c+1}(F)$ and working out the details as discussed
in Section \ref{sec:hopf-formula-proofs-lie}, we obtain:

\begin{itemize}
\item The Lie ring $[F,R]$ equals $[F,\gamma_{c+1}(F)] = \gamma_{c+2}(F)$.
\item The Lie ring $N = F/[F,R]$, with the natural quotient map $N \to
  L$, is an initial object in the category of central extensions of
  $L$ with homoclinisms. Note that $N$ is a free nilpotent Lie ring of
  class $c + 1$ on the same generating set $S$.
\item The exterior square $L \wedge L$ is canonically isomorphic to
  $[N,N]$, or equivalently, to $[F,F]/[F,R] =
  \gamma_2(F)/\gamma_{c+2}(F)$.
\item The Schur multiplier $M(L)$ is canonically isomorphic to the
  quotient Lie ring $(R \cap [F,F])/[F,R] =
  \gamma_{c+1}(F)/\gamma_{c+2}(F)$.
\item The canonical short exact sequence:

  $$0 \to M(L) \to L \wedge L \to [L,L] \to 0$$

  is isomorphic to the short exact sequence:
  
  $$0 \to \gamma_{c+1}(F)/\gamma_{c+2}(F) \to \gamma_2(F)/\gamma_{c+2}(F) \to \gamma_2(F)/\gamma_{c+1}(F) \to 0$$

\end{itemize}
