\documentclass[10pt]{amsart}

%Packages in use
\usepackage{fullpage, hyperref, vipul}

%Title details
\title{Division by primes: chain map formulation}
\author{Vipul Naik}

%List of new commands
\newcommand{\Skew}{\operatorname{Skew}}

\makeindex

\begin{document}
\maketitle

This article discusses a general formulation of the ``division by
primes'' problem that I'm working on with John Wiltshire-Gordon. I
begin by giving the general formulation, then proceeding to the
specific cases of interest.

\section{The chain map formulation}

\subsection{Cochain complex: definition}
Suppose $C^*$ is a cochain complex of abelian groups (or more
generally $R$-modules for a commutative unital ring $R$), starting at
$0$, with $\partial_i$ the boundary map from $C^i$ to $C^{i+1}$. Each
$\partial_i$ is a group (respectively, $R$-module) homomorphism, and
$\partial_i\partial_{i-1} = 0$ for all $i$. The cochain complex thus
looks as follows:

$$C^0 \stackrel{\partial_0}{\to} C^1 \stackrel{\partial_1}{\to} C^2\stackrel{\partial_2}{\to} \dots$$

We denote the coboundary group $\operatorname{Im} \partial_{i-1}$ as
$B^i$, and the cocycle group $\operatorname{Ker} \partial_i$ as
$Z^i$. By the definition of cochain complex, $B^i$ is a subgroup of
$Z^i$. Each coset of $B^i$ in $Z^i$ is termed a {\em cohomology class}
and the group $Z^i/B^i$ is denoted $H^i$. If $H^i$ is the zero group
(or zero module), then we say that that cochain complex is exact at $i$.

A {\em chain map} (of degree $0$) between two cochain complexes $C^*$
and $D^*$ is a bunch $f$ of homomorphisms $f_i: C^i \to D^i$ such that
$\partial^{(D)}_if_i = f_{i+1}\partial^{(C)}_i$, where the
superscripts indicate the cochain complex in which we are taking the
boundary maps.

\subsection{Multiplication by ring elements is a self-chain-map}

For any cochain complex $C^*$ of abelian groups, and any integer $n$,
multiplication by $n$ is a self-chain-map of $C^*$. Here,
``multiplication by $n$'' is the map that gives multiplication by $n$
on each of the cochain groups $C^i$. It commutes with the boundary map
because the boundary map is a homomorphism.

More generally, for a cochain complex of $R$-modules, multiplication
by any ring element gives a ring homomorphism.

Multiplication by $n$ is an automorphism of a cochain complex of
abelian groups iff each abelian group is uniquely $n$-divisible, or
equivalently, uniquely $p$-divisible for every prime $p$ dividing
$n$. In this case, the ring of coefficients of the cochain complex can
be extended naturally from $\Z$ to $\Z[1/n]$. It is injective in each
level if there is no $n$-torsion, or equivalently, there is no
$p$-torsion for any prime $p$ dividing $n$. It is surjective if each
abelian group is $n$-divisible (not necessarily uniquely).

For most of our purposes, we will restrict attention to the case of
multiplication by $p$ maps. In other words, we are interested in a
cochain complex $C^*$, a prime $p$, and the self-chain-map induced on
$C^*$ by multiplication by $p$.

\subsection{The question of interest}

We are interested in pulling back elements along the following
commutative square:

\begin{eqnarray*}
  C^i & \stackrel{\partial_i}{\to} &C^{i+1}\\
  \downarrow^{p} & & \downarrow^{p}\\
  C^i & \stackrel{\partial_i}{\to} &C^{i+1}
\end{eqnarray*}

Specifically, we are interested in the following: suppose $u \in C^i$
and $v \in C^{i+1}$ are such that $\partial_i(a) = pv$, can we find an
element $t \in C^i$ such that $pt = a$ and $\partial_i(t) = v$? What
are the additional conditions that we need to impose on $u$ and $v$?
In some cases, we want $t$ to satisfy some additional conditions as
well, and hence need more conditions on $u$ and $v$ as well.

\subsection{The uniquely $p$-divisible case}

We first show that the element $t$ exists and is unique in the case
that both $C^i$ and $C^{i+1}$ are uniquely $p$-divisible:

\begin{proof}
  There is a unique $t \in C^i$ such that $pt = a$. Taking
  $\partial_i$ on both sides, we get $p\partial_i(t) =
  \partial_i(a)$. Since $\partial_i(a) = pv$, we get $p\partial_i(t) =
  pv$. Since $C^{i+1}$ is uniquely $p$-divisible, $\partial_i(t) = v$.
\end{proof}

\subsection{Trying to solve the problem: local and global}

We are trying to fill in the left corner value in the following square:
\begin{eqnarray*}
  t (?) & \stackrel{\partial_i}{\mapsto} & v\\
  \downarrow^{p} & & \downarrow^{p}\\
  u & \stackrel{\partial_i}{\mapsto} & \partial_i(a) = pv
\end{eqnarray*}

Obviously, we need the following two additional conditions: (i) $u$ is
in $pC^i$, i.e., $u$ is divisible by $p$ in $C^i$, and (ii) $v$ is in
$B^{i+1}$, i.e., $v$ is an $(i + 1)$-coboundary.

If both conditions (i) and (ii) are satisfied, then the set of
solutions to $px = a$ is a coset of the $p$-torsion subgroup of $C^i$
and the set of solutions to $\partial_i(x) = v$ is a coset of
$Z^i$. The question now is whether these cosets intersect. If they
{\em do} intersect, then their intersection is a coset of the
intersection of the corresponding subgroups, i.e., a coset of the
intersection of $Z^i$ and the $p$-torsion.

Suppose $u$ is guaranteed to be in a subgroup $U$ of $C^i$, $v$ is
guaranteed to be in a subgroup $V$ of $C^{i+1}$, and we want $t$ to be
in a subgroup $T$ of $C^i$. Then, the following conditions guarantee
the existence of $t$:

\begin{itemize}
\item $U \subseteq pT$, i.e., for every $a \in A$, there exists $t \in
  T$ such that $pt = a$.
\item $V \subseteq \partial_i(T)$, i.e., for every $v \in V$, there
  exists $t \in T$ such that $\partial_i(t) = v$
\item $T$ is generated by $p^{-1}U \cap T$ and $\partial_i^{-1}V \cap
  T$.
\end{itemize}

The uniqueness result that we hope for is uniqueness up to
$i$-coboundaries, i.e., we want that $p^{-1}U \cap \partial_i^{-1}V
\cap T \subseteq B^i$.
\subsection{Nothing special about multiplication by $p$}

In all the discussion so far, there has been nothing special about
multiplication by $p$. All we need is maps $f_i:C_i \to C_i$ and
$f_{i+1}:C_{i+1} \to C_{i+1}$ such that the diagram commutes. In fact,
we don't even need the diagram to commute globally, we just need it to
commute on the subgroups of interest. Each of these generalizations
has some significance.

\section{Complexes of interest}

We are interested in the following two complexes:

\begin{itemize}
\item The {\em bar complex} for an action of a (possibly non-abelian)
  group $G$ on an abelian group $A$. The $n^{th}$ cochain group,
  denoted $C^n(G,A)$ is the group of $A$-valued functions on
  $G^n$. The boundary maps are defined the usual way. The bar complex
  arises naturally from the bar resolution. In most cases, we are
  interested in the bar complex for a trivial action.
\item An analogous notion for Lie rings.
\end{itemize}

Note that these complex constructions are functorial -- contravariant
in the first argument (i.e., $G$) and covariant in the second argument
(i.e., $A$).

\subsection{$k$-locally zero, $k$-local cocycles and $k$-local coboundaries}

We say that $f \in C^i(G,A)$ is $k$-locally zero if the restriction of
$f$ to any subgroup generated by at most $k$ elements is the zero map.

We say that $f \in C^i(G,A)$ is $k$-local cocycle if, for any subgroup
$H$ of $G$ with a generating set of size at most $k$, the restriction
of $f$ to $H$ is an $i$-cocycle. Note that if $k \ge i$, being a
$k$-local $i$-cocycle is equivalent to being an $i$-cocycle. In
general, it is weaker.

Similarly, we say that $f \in C^i(G,A)$ is a $k$-local coboundary if,
for any subgroup $H$ of $G$ with a generating set of size at most $k$,
the restriction of $f$ to $H$ is an $i$-coboundary. Unlike the case of
cocycles, an $i$-local $i$-coboundary need not be a coboundary (at
least, it is not obvious).

We can generalize these notions from $k$-local to $\mathcal{S}$-local
where $\mathcal{S}$ is any collection of subgroups of $G$. We impose
the condition on restrictions to all subgroups that are elements of
$\mathcal{S}$. For $k$-local, $\mathcal{S}$ is the collection of
subgroups generated by at most $k$ elements.

\subsection{The Lie ring setup}

A Lie ring is a Lie algebra over $\Z$. Thus, all the things we say
about Lie rings are things that work in greater generality for Lie
algebras, and in fact are usually discussed for Lie algebras. Note
that we will be talking of Lie algebras where the base ring is a
commutative unital ring. For now, we stick to the base ring $\Z$, and
thus talk of Lie rings.

Supplse $L$ is a Lie ring and $M$ is a $L$-module, i.e., an abelian
group with a Lie ring homomorphism from $L$ to $\operatorname{End}(M)$
where the latter's Lie ring structure is the usual one arising from an
associative ring.

Then, the group $C^1(L,M)$ is the group of all additive
homomorphisms from $L$ to $M$ and $C^2(L,M)$ is the group of all
additive homomorphisms from $L \times L$ to $M$.

The boundary map:

$$\partial_1: C^1(L,M) \to C^2(L,M)$$

is given as follows:

$$\partial_1(f) = (x,y) \mapsto f(x) \cdot y - f(y) \cdot x - f([x,y])$$

The kernel of this action (i.e., the $1$-cocycles) is the set of
additive maps $f:L \to M$ such that:

$$f([x,y]) = f(x) \cdot y - f(y) \cdot x$$

The $1$-coboundaries are trivial, so the group of $1$-cocycles is
precisely the first cohomology group.

In the special case that $L = M$ and the action is the adjoint action,
the $1$-cocycles are precisely the derivations.

A related situation is as follows. $E$ is a Lie ring with an abelian
Lie ideal $M$ and an isomorphism from $E/M$ to $L$. In this case, $L$
acts on $M$ by restricting the adjoint action. The action is defined
precisely because $M$ is abelian: any two elements in the same coset
of $M$ in $E$ have the same action on $M$.

We can generalize the situation even further: $E$ is a Lie ring with
Lie ideals $M$ and $N$ such that $M + N = E$ and $N/(M \cap N)$ is
abelian. Then, setting $L = E/N$, we can define an action of $L$ on
$M$ by restricting the adjoint action. Note that this generalizes both
previous cases: the whole Lie ring case is covered by $M = E$, $N =
0$, and the abelian ideal case is covered by $N = E$.

In this general situation, the action can be thought of as Lie
brackets in something bigger, so the $1$-cocycles can be thought of as
(generalized versions of) derivations.

Note that in the further case that $M$ is central in $E$, the action
is trivial, and the condition for being a $1$-cocycle becomes:

$$f(x) \cdot y = f(y) \cdot x$$

\subsection{$i = 1$ case: interpretation}



\end{document}
