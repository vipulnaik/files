\documentclass[10pt]{amsart}

%Packages in use
\usepackage{fullpage, hyperref, vipul, enumerate}

%Title details
\title{Class quiz solutions: October 7: Limit theorems}
\author{Math 152, Section 55 (Vipul Naik)}
%List of new commands

\begin{document}
\maketitle

\section{Performance review}

$12$ students took this quiz. The score distribution was as follows:

\begin{itemize}
\item Score of $2$: $3$ people
\item Score of $3$: $5$ people
\item Score of $4$: $4$ people
\end{itemize}

The mean score was $3.08$. Here are the problem-wise answers and
scores:

\begin{enumerate}
\item Option (A): $6$ people
\item Option (C): $4$ people
\item Option (D): $10$ people
\item Option (D): $9$ people
\item Option (B): $8$ people
\end{enumerate}

More details below.

\section{Solutions}

\begin{enumerate}

\item (**) Which of the following statements is {\bf always true}?

  \begin{enumerate}[(A)]

  \item The range of a continuous nonconstant function on a closed
    bounded interval (i.e., an interval of the form $[a,b]$) is a
    closed bounded interval (i.e., an interval of the form $[m,M]$).
  \item The range of a continuous nonconstant function on an open
    bounded interval (i.e., an interval of the form $(a,b)$) is an
    open bounded interval (i.e., an interval of the form $(m,M)$).
  \item The range of a continuous nonconstant function on a closed
    interval that may be bounded or unbounded (i.e., an interval of
    the form $[a,b]$, $[a,\infty)$, $(-\infty,a]$, or
    $(-\infty,\infty)$) is also a closed interval that may be bounded
    or unbounded.
  \item The range of a continuous nonconstant function on an open
    interval that may be bounded or unbounded (i.e., an interval of
    the form $(a,b)$,$(a,\infty)$, $(-\infty,a)$, or
    $(-\infty,\infty)$), is also an open interval that may be bounded
    or unbounded.
  \item None of the above.
  \end{enumerate}

  {\em Answer}: Option (A)

  {\em Explanation}: This is a combination of the extreme-value
  theorem and the intermediate-value theorem. By the extreme-value
  theorem, the continuous function attains a minimum value $m$ and a
  maximum value $M$. By the intermediate-value theorem, it attains
  every value between $m$ and $M$. Further, it can attain no other
  values because $m$ is after all the minimum and $M$ the maximum.

  {\em The other choices}:

  Option (B): Think of a function that increases first and then
  decreases. For instance, the function $f(x) := \sqrt{1 - x^2}$ on
  $(-1,1)$ has range $(0,1]$, which is not open. Or, the function
  $\sin x$ on the interval $(0,2\pi)$ has range $[-1,1]$.

  Option (C): We can get counterexamples for unbounded intervals. For
  instance, consider the function $f(x) := 1/x$ on $[1,\infty)$. The
  range of this function is $(0,1]$, which is not closed. The idea is
  that we make the function approach but not reach a finite value as
  $x \to \infty$ (we'll talk more about this when we deal with
  asymptotes).

  Option (D): The same counterexample as for option (B) works.

  {\em Performance review}: $6$ out of $12$ got this correct. $3$
  chose (C), $2$ chose (D), $1$ chose (E).

  {\em Historical note (last year)}: $2$ out of $11$ people got this
  correct. (C) was the most frequently chosen incorrect answer.

  {\em Action point}: Please review the statement of the extreme value
  theorem, as well as understand why all the other examples are
  incorrect.

\item (**) Suppose $g:\R \to \R$ is a continuous function such that
  $\lim_{x \to 0} g(x)/x = A$ for some constant $A \ne 0$. What is
  $\lim_{x \to 0} g(g(x))/x$?

  \begin{enumerate}[(A)]
  \item $0$
  \item $A$
  \item $A^2$
  \item $g(A)$
  \item $g(A)/A$
  \end{enumerate}

  {\em Answer}: Option (C)

  {\em Explanation}: We have $\lim_{x \to 0} g(x) = \lim_{x \to 0}
  (g(x)/x) \lim_{x \to 0} x = A \cdot 0 = 0$.

  Also, we have:

  $$\lim_{x \to 0} \frac{g(g(x))}{x} = \lim_{x \to 0} \frac{g(g(x))}{g(x)} \lim_{x \to 0} \frac{g(x)}{x}$$

  The second limit is $A$. For the first limit, note that as $x \to
  0$, we also have $g(x) \to 0$, so the first limit can be rewritten
  as $\lim_{y \to 0} g(y)/y$, which is also equal to $A$. Hence, the
  overall limit is the product $A^2$.

  {\em Performance review}: $4$ out of $12$ go this correct. $3$ each
  chose (A) and (E), $2$ chose (D).

  {\em Historical note (last year)}: $1$ out of $12$ people got this
  correct. $5$ people chose (D), $2$ people each chose (B) and (E),
  $1$ person chose (A), and $1$ person left the question blank.


\item Suppose $I = (a,b)$ is an open interval. A function $f:I \to \R$
  is termed {\em piecewise continuous} if there eixst points $a_0 <
  a_1 < a_2 < \dots < a_n$ (dependent on $f$) with $a = a_0$ and $a_n
  = b$, such that $f$ is continuous on each interval $(a_i,
  a_{i+1})$. In other words, $f$ is continuous at every point in
  $(a,b)$ except possibly the $a_i$s.

  Suppose $f$ and $g$ are piecewise continuous functions on the same
  interval $I$ (with possibly different sets of $a_i$s). Which of the
  following is/are guaranteed to be piecewise continuous on $I$?

  \begin{enumerate}[(A)]
  \item $f + g$, i.e., the function $x \mapsto f(x) + g(x)$
  \item $f - g$, i.e., the function $x \mapsto f(x) - g(x)$
  \item $f \cdot g$, i.e., the function $x \mapsto f(x)g(x)$
  \item All of the above
  \item None of the above
  \end{enumerate}

  {\em Answer}: Option (D)

  {\em Explanation}: We take the points where $f$ is possibly
  discontinuous and the points where $g$ is possible discontinuous,
  and we take the union of these sets of points. We get a new finite
  set of points. Note that everywhere except these points, both $f$
  and $g$ are continuous, hence $f + g$, $f - g$, and $f \cdot g$ are
  all continuous.

  A numerical illustration might help here. (Note, however, that there
  is nothing special about the numbers). Suppose $a = 1$ and $b =
  2$. Let's say that $f$ is continuous on $(1,1.5)$ and $(1.5,2)$, so
  it is possibly discontinuous at $1.5$. Suppose $g$ is continuous on
  $(1,\sqrt{2})$, $(\sqrt{2},\sqrt{3})$ and $(\sqrt{3},2)$, so the
  points where it may be discontinuous are $\sqrt{2}$ and $\sqrt{3}$.

  We now take the union of the points of discontinuity of $f$ and
  $g$. We get the points $1.5$, $\sqrt{2}$, and $\sqrt{3}$. Recall
  that $\sqrt{2} \approx 1.414 < 1.5$ while $\sqrt{3} \approx 1.732 >
  1.5$, so rearranging in increasing order, we get $1 < \sqrt{2} < 1.5
  < \sqrt{3} < 2$. We can now see that $f + g$, $f - g$ and $f \cdot
  g$ are all continuous on the intervals $(1,\sqrt{2})$,
  $(\sqrt{2},1.5)$, $(1.5,\sqrt{3})$ and $(\sqrt{3},2)$.

  {\em Memory lane}: This is the idea of {\em breaking up the domains
  for two piecewise definned functions in the same manner} so as to be
  able to add, subtract, and multiply them. You have seen a problem
  with this theme in Homework 1, Problem 8 (Exercise 1.7.14 of the
  book, Page 46). There, goal was to compute $f + g$, $f - g$, and $f
  \cdot g$ with $f$ and $g$ given piecewise with different domain
  breakdowns. Here, our goal is to pontificate about continuity, but
  the idea is the same.

  {\em Future teaser}: This idea of partitioning an interval into
  sub-intervals by choosing some points keeps coming up. Further, the
  idea of combining two partitions of the same interval into a finer
  partition that refines both of them will also come up. Specifically,
  both these ideas turn up when we try to define the integral of a
  continuous (or piecewise continuous) function on an interval.

  {\em Performance review}: $10$ out of $12$ got this correct. $2$
  chose (E).

  {\em Historical note (last year)}: $9$ out of $11$ people got this
  correct. $1$ person chose (C) and $1$ person chose (E).

  {\em Action point}: Whether or not you got this correct, make sure
  that you {\em now} understand the logic behind it. This idea is
  extremely important in the future.

\item Suppose $f$ and $g$ are everywhere defined and $\lim_{x \to 0}
  f(x) = 0$. Which of these pieces of information is {\bf not
  sufficient} to conclude that $\lim_{x \to 0} f(x)g(x) = 0$?

  \begin{enumerate}[(A)]
  \item $\lim_{x \to 0} g(x) = 0$.
  \item $\lim_{x \to 0} g(x)$ is a constant not equal to zero.
  \item There exists $\delta > 0$ and $B > 0$ such that for $0 < |x| <
    \delta$, $|g(x)| < B$.
  \item $\lim_{x \to 0} g(x) = \infty$, i.e., for every $N > 0$ there
    exists $\delta > 0$ such that if $0 < |x| < \delta$, then $g(x) >
    N$.
  \item None of the above, i.e., they are all sufficient to conclude
    that $\lim_{x \to 0} f(x)g(x) = 0$.
  \end{enumerate}

  {\em Answer}: Option (D)

  {\em Explanation}: If $f(x) \to 0$ and $g(x) \to \infty$, then the
  limit of $f(x)g(x)$ is indeterminate. It may be $0$, finite,
  infinite, or oscillatory. For instance, if $f(x) = x^2$ and $g(x) =
  1/x^2$, then the limit of $f(x)g(x)$ is $1$. Thus, we cannot conclude
  that the limit of the product is $0$.

  {\em Memory lane}: Routine Problem 5 on Homework 2 (Exercise 2.3.3
  of the book, Page 79) explores a similar theme. The new ingredient
  here is that, in cases where $f(x)$ goes to zero and $g(x)$ does not
  have a limit but is still bounded, we {\em can} say that the product
  goes to zero.

  {\em The other choices}:

  Option (A) is sufficient because the limit of the sums is the sum of
  the limits.

  Option (B) is sufficient for the same reason.

  Option (C) is a little trickier to justify. Here, what we are saying
  is that $\lim_{x \to 0} f(x) = 0$ and, for $x$ close enough to $0$,
  $g$ is bounded, though it need not have a limit. The bound here is
  $B$. In particular, what this is saying is that if $0 < |x| <
  \delta$, then $g(x)$ is between $-B$ and $B$.

  Thus, we can see that:

  $$-Bf(x) \le f(x)g(x) \le Bf(x) \ \forall \ 0 < |x| < \delta$$

  We now note that both $-Bf(x)$ and $Bf(x)$ tend to $0$ as $x \to
  0$. Hence, by the pinching theorem, $f(x)g(x) \to 0$.

  {\em Examples for $g$}: One example of such a function $g$ is the
  Dirichlet function, i.e., $g(x)$ is $1$ if $x$ is rational and $0$
  if $x$ is irrational. Clearly, the Dirichlet function is bounded
  near $0$ (in fact, it is universally bounded). If we set $f(x) : =
  x$, then $f(x)g(x) = \lbrace\begin{array}{rl} x,& x \text{
  rational}\\0, & x \text{ irrational}\\\end{array}$, and the limit of
  this function at $0$ is $0$. Incidentally, Advanced Problem 3 of
  Homework 2 (Exercise 2.2.54 of the book, Page 72) asked you to give
  an explicit $\epsilon-\delta$ proof of this fact.

  Another example for $g$ is the $\sin(1/x)$ function. This function
  oscillates between $-1$ and $1$, hence does not converge to a limit
  as $x \to 0$. However, it is bounded. Thus, if we have $f(x) :=
  x$, the function $x\sin(1/x)$ must converge to $0$ as $x$ goes to
  $0$. This function appears in Advanced Problem 4 of Homework 3
  (Exercise 3.6.67 of the book, Page 146).

  {\em Performance review}: $10$ out of $12$ got this correct. $1$
  each chose (A), (C), and (E).

  {\em Historical note (last year)}: $8$ out of $11$ people got this
  correct. $1$ person each chose (A), (B), and (C).

  {\em Action point}: You should understand this, but don't have to
  worry too much about it for now. We will cover these issues in more
  detail later.
\item $f$ and $g$ are functions defined for all real values. $c$ is a
  real number. Which of these statements is {\bf {\em not} necessarily
  true}? 

  \begin{enumerate}[(A)]
  \item If $\lim_{x \to c^-} f(x) = L$ and $\lim_{x \to c^-} g(x) =
    M$, then $\lim_{x \to c^-} (f(x) + g(x))$ exists and is equal to
    $L + M$.
  \item If $\lim_{x \to c^-} g(x) = L$ and $\lim_{x \to L^-} f(x) =
    M$, then $\lim_{x \to c^-} f(g(x)) = M$.
  \item If there exists an open interval containing $c$ on which $f$
    is continuous and there exists an open interval containing $c$
    on which $g$ is continuous, then there exists an open interval
    containing $c$ on which $f + g$ is continuous.
  \item If there exists an open interval containing $c$ on which $f$
    is continuous and there exists an open interval containing $c$ on
    which $g$ is continuous, then there exists an open interval
    containing $c$ on which the product $f \cdot g$ (i.e., the
    function $x \mapsto f(x)g(x)$) is continuous.
  \item None of the above, i.e., they are all necessarily true.
  \end{enumerate}

  {\em Answer}: Option (B)

  {\em Explanation}: This is the cliched fact that composition results
  do not hold for one-sided limits. The main reason is that when we
  compose, we need the inner function of the composition to approach
  the limit {\em from the correct side} in order for the result to go
  through. Thus, in this case, for instance, the result would be true
  if the function $g$ were strictly increasing on the immediate left
  of $c$.

  {\em Memory lane}: We already saw this fact in the October 1 quiz on
  limits, Problem 2. Please review the solution to that (where we've
  also given an explicit example).

  {\em The other choices}:

  Option (A) is the sum theorem for one-sided limits: the limit of the
  sum is the sum of the limits.

  For options (C) and (D), note that if $f$ is continuous on one open
  interval containing $c$ and $g$ is continuous on another open
  interval containing $c$, then {\em both} $f$ and $g$ are continuous
  on the {\em intersection} of the two open intervals containing $c$
  (which is also an open interval containing $c$). Thus, $f + g$ is
  also continuous on this intersection. This is analogous to the trick
  we often use of picking $\delta = \min \{ \delta_1, \delta_2 \}$ in
  $\epsilon-\delta$ proofs for piecewise functions.

  {\em Performance review}: $8$ out of $12$ got this correct. $2$
  chose (C) and $2$ chose (E).

  {\em Historical note (last year)}: $9$ out of $11$ people got this
  correct. $1$ person each chose (A) and (E).


\end{enumerate}

\end{document}