\documentclass[10pt]{amsart}
\usepackage{fullpage,hyperref,vipul}
\title{Volume computations using integrals}
\author{Math 152, Section 55 (Vipul Naik)}

\begin{document}
\maketitle

{\bf Corresponding material in the book}: Section 6.2, 6.3.

{\bf Difficulty level}: Hard (degree of hardness depends on your
visuo-spatial skills and prior exposure to these ideas).

{\bf What students should definitely get}: The basic constructive
ideas for volume: cylinders with constant and varying cross section,
surfaces of revolution (disks and washers). The formula and mechanics
for the shell method.

{\em Note: I haven't included pictures, since these are hard to
draw. I suggest that you look at pictures in the book, which are
pretty well done, and of course, pay attention in class.}

\section*{Executive summary}

Words ...

\begin{enumerate}
\item The cross section method for computing volume is an analogue of
  the two-dimensional area computation method: our slices are replaced
  by cross sections by planes parallel to a fixed plane, and the line
  of integration is a line perpendicular to the
  planes. One-dimensional slices are replaced by two-dimensional cross
  sections.
\item Suppose $\Omega$ is a region in the plane. We can construct a
  right cylinder with base $\Omega$ and height $h$. This is obtained
  by translating $\Omega$ in a direction perpendicular to its plane by
  a length of $h$. The cross section of this right cylinder along any
  plane parallel to the original plane looks like $\Omega$ if that
  plane is within range. The volume is the product of the area of
  $\Omega$ and the height $h$. This is also called the right cylinder
  with constant cross section $\Omega$.
\item We can also construct an oblique cylinder. Here, the direction
  of translation is not perpendicular to the original plane. The total
  volume is the product of the area of $\Omega$ and the height
  perpendicular to $\Omega$. Oblique cylinders are to right cylinders
  what parallelograms are to rectangles.
\item More generally, the volume of a solid can be computed using the
  cross section method. Here, we choose a direction. We measure areas
  of cross sections along planes perpendicular to that direction, and
  integrate these areas along that direction.
\item This general approach has another special case that is perhaps
  as important as right cylinders. These are the {\em cones} (there
  are right cones and oblique cones). A cone is obtained by taking a
  region in a plane and connecting all points in it to a point outside
  the plane. It is a right cone if that point is directly above the
  center of the region. The volume of a cone is $1/3$ times the
  product of the base area and the height, i.e., the perpendicular
  distance from the outside point to the plane. In particular, a cone
  has one-third the volume of a cylinder of the same base and height.
\item A solid of revolution is a solid obtained by revolving a region
  in a plane about a line (called the axis of revolution). The volume
  of a solid of revolution can be computed by choosing the axis as the
  axis of integration and using the planes of cross section as planes
  perpendicular to it. These cross sections are either circular disks
  or annuli.
\item The {\em disk method} is a special case of the above, where the
  region between revolved is supported on the axis of revolution. For
  instance, consider the region between the $x$-axis, the graph of a
  function $f$, and the lines $x = a$ and $x = b$. The volume of the
  corresponding solid of revolution is $\pi \int_a^b [f(x)]^2 \,
  dx$. This is because the radius of the cross section disk at $x=
  x_0$ is $|f(x_0)|$.
\item The {\em washer method} is the more general case where the
  region need not adhere to the axis of revolution. For instance,
  consider two nonnegative functions $f,g$ and suppose $0 \le g \le
  f$. Consider the region bounded by the graphs of these two functions
  and the lines $x = a$ and $x = b$. The volume of the corresponding
  solid of revolution is $\pi \int_a^b ([f(x)]^2 - [g(x)]^2) \,
  dx$. Note that in the more general case where the functions cross
  each other, we may need to split into sub-intervals so that we can
  apply the washer method on each sub-interval.
\item The shell method works for situations where we revolve about the
  $y$-axis the region made between the graph of a function and the
  $x$-axis. The formula here is $2\pi \int_a^b xf(x) \, dx$ for $f$
  nonnegative and $0 < a < b$. If $f$ could be positive or negative,
  we use $2 \pi \int_a^b x|f(x)| \, dx$.. More generally, if we are
  looking at the region between the graphs of $f$ and $g$ (vertically)
  with $g \le f$, we get $2\pi \int_a^b x[f(x) - g(x)] \, dx$. If we
  don't know which one is bigger where, we use $2\pi \int_a^b x|f(x) -
  g(x)| \, dx$.
\end{enumerate}

Actions ...

\begin{enumerate}
\item To compute the volume using cross sections, we first need to set
  things up so that we know the cross section areas as a function of
  the position of the plane. For this, it is usually necessary to use
  either coordinate geometry or basic trigonometry, or a combination.
\item A solid occurs as a solid of revolution if it has complete
  rotational symmetry about some axis. In that case, that axis is the
  axis of revolution and the original region that we need is obtained
  by taking a cross section in any plane containing the axis of
  revolution and looking at the part of that cross section that is on
  one side of the axis of revolution.
\item For solids of revolution, be particularly wary if the original
  figure being revolved has parts on both sides of the axis of
  revolution. If it is symmetric about the axis of revolution, delete
  one side.
\item Be careful about the situations where you have to be
  sign-sensitive and the situations where you do not. In the disk
  method sensitivity to signs is not important. In the washer method
  and shell method, it is.
\item The farther the shape being revolved is from the axis, the
  greater the volume of the solid of revolution.
\item The average value point of view is sometimes useful for
  understanding such situations.
\end{enumerate}

\section{Motivation: from area to volume}

\subsection{What are we trying to do?}

Our purpose right now is to find formulas for the volumes of various
three-dimensional figures. This is a little like our attempts at
finding areas of regions, which we successfully did, at least for some
regions. Wait, what?

The process that we are going through is something whose broad
outlines should be familiar to you. Think back, for instance, to how
we dealt with differentiation. We first computed formulas for the
derivatives of a few functions. Then, we considered all the ways that
new functions can be created from old functions. Finally, we found
formulas that tackled each way of creating a new function from an old
function. Combined with the knowledge of how to differentiate the
basic functions, this allowed us to differentiate any function given
to us using a simple set of rules.

Similarly, when trying to figure out general strategies for finding
limits, we started out by computing a few basic limits, and looking at
rules for computing limits of functions created from simpler functions.

We did a similar thing for integration: we learned rules for finding
antiderivatives for some basic functions, and then we learned various
processes of combination (something we're not quite done with
yet). The overall strategy is:

\begin{enumerate}
\item Find out how to deal with basic situations.
\item Identify the typical ways that basic situations are combined to
  create more complicated situations.
\item For each such process of combining basic situations into more
  complicated situations, identify a way of reducing the problem for
  the complicated situation in terms of the basic situations.
\end{enumerate}

We shall consider how to deal with volumes. Our main difficulty in
calculating volumes is with step (2) -- we don't have an understanding
of the systemic processes whereby new three-dimensional figures can be
created. Once we do, we can try to find a volume formula for each such
process, and use these formulas to calculate the areas of a number of
figures.

\subsection{A recapitulation of how we handled area computations}

We have so far dealt with two kinds of area computations. The first is
computing areas {\em against the $x$-axis}. Here, we are measuring the
area bounded between a curve and the $x$-axis, or the area between two
curves and two vertical lines.

Let us reflect more carefully on how we can characterize these
situations geometrically. In all these situations, the region $\Omega$
that we have has the property that the intersection of $\Omega$ with
any vertical line is either empty or a line segment. Regions of this
kind are sometimes called Type I regions. For Type I regions, the
general formula for the unsigned area is:

$$\int \text{(Length of the line segment as a function of $x$)} \, dx$$

This process can be thought of as {\em vertical slicing}. We are
dividing the area that we want to measure into vertical slices, and
then integrating the length along the perpendicular axis (which is
horizontal).

The other procedure that we saw for integration is integration against
the $y$-axis. This kind of integration works for regions $\Omega$
which have the property: the intersection of $\Omega$ with any
horizontal line is either empty or a line segment. Regions of this
type are sometimes called Type II regions. The formula for the area of
a Type II region is

$$ \int \text{(Length of the line segment as a function of $y$)} \, dy$$

This process can be thought of as {\em horizontal slicing}. We are
dividing the area that we want to measure into horizontal slices, and
the integrating the length along the perpendicular axis (which is
vertical).

Thus, we have seen two processes of breaking up an area into slices:
vertical slicing (where we integrate the lengths along a horizontal
axis) and horizontal slicing (where we integrate the lengths along a
vertical axis).

Notice that both these procedures are variants of the same basic
procedure: choose two mutually perpendicular directions, such that all
lines in one direction have intersection with the region that is
either empty or a line segment. Then, integrate the length of the line
segment along the perpendicular direction.

Note also that the extreme case of both these occurs in
rectangles. Here, whether we use horizontal or vertical slicing, we
are integrating a constant function.

\subsection{How does this general idea carry over to three dimensions?}

$1 + 1 = 2$, but $1 + 1 \ne 3$. So, the idea of choosing two mutually
perpendicular directions, one for the slices and the other as the
direction of integration, does not work directly for computing
volumes. However, it {\em is} true that $2 + 1 = 3$. This suggests a
slightly different strategy to measure the volume of a
three-dimensional region $\Omega$: choose a plane $\pi$ and a line
$\ell$ perpendicular to $\pi$. Now, measure the areas of the
intersection of $\Omega$ with regions perpendicular to $\pi$, and
integrate this area along $\ell$.

In other words, the {\em slices} are two-dimensional and parallel to
each other, and the direction of the line along which we integrate is
perpendicular to those planes.\footnote{The reason why we are forced
to use $2 + 1 = 3$ rather than $1 + 2 = 3$ is because the only kind of
integration that we have explicitly dealt with is integration in one
variable, i.e., along a line.}

In the forthcoming section, we look at some systemic processes for
creating three-dimensional structures and for slicing them suitably.

\subsection*{Sidenote: Distinction between a disk and a circle}

Henceforth, when I refer to a {\em circle}, I refer to the {\em
boundary}, i.e., the set of points whose distance from the center
equals the radius. When I want to talk of the circle along with the
interior region, I will use the term {\em circular disk}, or, more
briefly, {\em disk}. When I want to simply look at the interior and
exclude the boundary, I will use the term {\em interior of the disk}
or {\em open disk}.

I will, however, switch between a circle and its disk easily, hence
when I talk about the center, radius, or diameter of a disk, I am
referring to those notions for its boundary circle.

\section{Creating three-dimensional structures}

\subsection{The general concept of a right cylinder}

What you may have been told is a {\em cylinder} is more appropriately
termed a {\em right circular cylinder}. The adjectival qualifier {\em
circular} indicates that the base is a circle (more precisely, the
boundary is a circle and the base is a circular disk). The term {\em
right cylinder}, in general, means something like a right circular
cylinder except that the base need not be circular.

Basically, we take a region $\Omega$ in the plane with boundary
$\Lambda$ and then translate $\Omega$ along a direction perpendicular
to the plane for a fixed length. That fixed length is called the {\em
height} of the right cylinder. This gives the (solid) right cylinder
with cross section $\Omega$. The curved surface of the cylinder is the
boundary of this, which is obtained by translating $\Lambda$ in a
direction perpendicular to the plane of $\Omega$. The two {\em caps}
are the two copies of $\Omega$ located at the two ends.

The term {\em cross section} here refers to the fact that if we take
any plane parallel to the plane of $\Omega$, its intersection with the
right cylinder is a copy of $\Omega$ if the plane is located in the
relevant region; otherwise it is empty. 

You may have heard the term {\em cross section} arising in different
contexts. It basically means the intersection with a given plane. For
instance, in biology, when studying things ranging from tree trunks to
micro-organisms and cells, we take cross-sections in various
directions.

The right cylinder has a constant cross section. In this sense, it is
similar to a rectangle in two dimensions, which has constant cross
sections.

The volume of a right cylinder is given by:

\begin{equation*}
  \text{Volume of right cylinder} = \text{Area of cross section} \times \text{Height}
\end{equation*}

Some particular cases of interest:

\begin{itemize}
\item When the base cross section is a circular disk, we get a {\em
  right circular cylinder}.
\item When the base cross section is a polygon, we get what is often
  called a {\em prism}. In particular, when the base is a rectangle,
  we get a rectangular prism.
\end{itemize}

\subsection{Oblique cylinders}

A slight variant on right cylinder is oblique cylinder. Oblique
cylinders are to parallelograms what right cylinders are to
rectangles. Here is the construction of an oblique cylinder.

Start with a region $\Omega$ in a plane $\pi$. Now, choose a direction
in space that is not parallel to the plane $\pi$. Translate $\Omega$
by a length $l$ along this direction. The region traced this way is
termed an oblique cylinder.

The volume of an oblique cylinder is given by:

\begin{equation*}
  \text{Volume of oblique cylinder} = \text{Area of cross section} \times \text{Height perpendicular to cross section}
\end{equation*}

Equivalently:

\begin{equation*}
  \text{Volume of oblique cylinder} = \text{Area of cross section} \times \text{Length $l$} \times \sin \theta
\end{equation*}

where $\theta$ is the angle between the plane $\pi$ and the direction
of translation. In particular, when $\theta = \pi/2$, we get a right
cylinder.

If we consider cross sections of oblique cylinder parallel to $\pi$,
each of these cross sections looks like $\Omega$. However, unlike the
right cylinder case, the {\em location} of the $\Omega$ in the cross
section plane keeps changing.

\subsection{More oblique than oblique}

In fact, it is possible to get eve nmore oblique than oblique -- we
translate a shape in a plane along a direction other than the plane,
but we keep changing the direction. Thus, each cross section still has
the same shape, but its location changes rather unpredictable. We'll
see some such situations in a homework/quiz/test.

\subsection{Variable cross sections}

We next consider a situation where the cross sections are
variable. This is no longer a right cylinder, but we can use the idea
mentioned a little while ago -- integrating the area
function. Earlier, we multiplied a constant with the height over which
that constant was valid. Now, we integrate a variable function over an
interval. Remember, integration is like multiplication where the thing
you're trying to multiply keeps changing. The important thing is that
the area of each cross section should be something we know how to
measure. The general formula is:

$$\text{Volume} = \int \text{(Area of cross section perpendicular to $x$)} \, dx$$

\subsection{Cones}

One case of particular importance, where it is useful to remember a
general approach as well as the specific answer, is that of the {\em
cone}. A cone is defined as follows. Suppose $\Omega$ is a region in a
plane $\pi$ and $P$ is a pont not in $\pi$. The cone corresponding to
$\Omega$ and $P$ is the union of all the line segments joining $P$ to
points in $\Omega$.

Some examples of cones are:

\begin{enumerate}
\item A {\em tetrahedron} is a cone where the base is a triangular region.
\item A {\em right circular cone} is a cone where the base is a
  circular disk.
\item A {\em pyramid} is a cone where the base is some polygon.
\end{enumerate}

When we set up the cross section integration for the cone, we see that
the shape of any cross section parallel to $\pi$ is the same as that
of $\Omega$, but the size is different. We can use similar triangles
to determine the size. If we define:

$$\alpha = \frac{\text{Distance from $P$ to cross section}}{\text{Distance from $P$ to $\pi$}}$$

Then the linear measurements for the cross section are $\alpha$ times
the corresponding linear measurements for $\Omega$. Since area is
two-dimensional, the area of the cross section is $\alpha^2$ times the
area of $\Omega$. We now get that the overall volume is:

$$\int_0^h (x/h)^2 \text{Ar}(\Omega) \, dx$$

Plugging in $x = \alpha h$, we get:

$$\int_0^1 \alpha^2 \text{Ar}(\Omega) h \, d\alpha$$

We pull out the constants, and get:

$$\text{Ar}(\Omega) h \int_0^1 \alpha^2 \, d\alpha$$

The integral now gives $1/3$, and we thus get:

\begin{equation*}
  \text{Volume of cone} = \frac{1}{3} \text{Area of base region} \times \text{Height}
\end{equation*}

{\em Now} you understand why you have that $1/3$ in the formula for
the volume of a cone: $(1/3) (\pi r^2)(h)$.

But not completely. Why $1/3$? Well, let's think back to the
two-dimensional analogue of this. What's a two-dimensional analogue of
a cone? It's just a triangular region. The analogue of the
two-dimensional base is a one-dimensional line segment. And we remember that:

\begin{equation*}
  \text{Area of triangle} = \frac{1}{2} \text{Length of base line segment} \times \text{Height}
\end{equation*}

So why do we get $1/2$ in the two-dimensional case and $1/3$ in the
three-dimensional case? Well, you might guess that we basically get
$1/n$ in the $n$-dimensional case. And then you go back and look at
the proof, and see that it essentially works this way:

$$\int_0^1 \alpha^{n - 1}\, d\alpha = [\alpha^n/n]_0^1 = 1/n$$

\section{Solids of revolution: the disk and washer method}

\subsection{Definition of solid of revolution}

There is another procedure for constructing three-dimensional
figures. Three-dimensional figures constructed this way are called
{\em solids of revolution}. This is obtained as follows: we start with
a region $\Omega$ and a line $\ell$. Next, we rotate $\Omega$ about
the line $\ell$ in three dimensions. The region obtained in this way
is termed the {\em solid of revolution} of $\Omega$.

For simplicity, we will assume that $\Omega$ lies completely to one
side of $\ell$. We study such surfaces in two steps. First, we study
the special case where one boundary of $\Omega$ is along $\ell$. After
that, we study the case where all of $\Omega$ could lie on one side of
$\ell$. The method for the first case is termed the {\em disk method}
and the method for the second case is termed the {\em washer method}.

{\em Aside}: The surface of a solid of revolution includes two capping
disks. The remaining part of this surface is the curved surface, and
this is often called a {\em surface of revolution}. Surfaces of
revolution turn out to be very important in a variety of natural
processes.

\subsection{Disk method}

Consider the area bounded by the graph of the function $y = f(x)$ and
the $x$-axis between $x = a$ and $x = b$ (with $a < b$). Assume, for
now, that the graph of $f$ lies completely on the positive side of the
$x$-axis. So, the picture looks something like Figure 6.2.8 (left) of
the book. Revolving this about the $x$-axis gives a solid of
revolution as shown in fiure 6.2.8 (right) of the book.

We now consider how to apply the method of parallel cross sections to
this volume computation. We consider the axis as the $x$-axis and the
cross sections are thus in the $yz$-plane. In particular, we see by
our construction that all the cross-sections are disks and the disk
for a cross section at $x = x_0$ has radius $f(x_0)$ and area $\pi
(f(x_0))^2$. The area is thus:

$$\int_a^b \pi (f(x))^2 \, dx$$

We can pull the $\pi$ out of the integral if we want. This is the
general formula for calculating the area.

It turns out that the formula is also valid for a function that
crosses the $x$-axis. In this case, the parts abover the $x$-axis and
the parts below the $x$-axis are out of phase by $\pi$ as we revolve
them. However, the overall analysis remains the same, with the radius
being $|f(x_0)|$ instead of $f(x_0)$. Since we are squaring it anyway,
the final answer remains the same.

Here are some particular cases of solids of revolution whose volume can be computed using the disk method:

\begin{enumerate}
\item The right circular cylinder with radius $r$ and height $h$ can
  be realized as the solid of revolution for the region between the
  $x$-axis and the graph of a constant function with value $r$
  (bounded by vertical lines) over an interval of length $h$. The
  region being rotated is thus a rectangle with dimensions $r$, $h$,
  and $h$ is the fixed side.
\item The right circular cone with radius $r$ and height $h$ can be
  realized as the solid of revolution for the region between the
  $x$-axis and the graph of the function $y = rx/h$ on the interval
  $[0,h]$ (bounded by a vertical line at $x = h$). The region being
  rotated is thus a right triangle with legs $r$ and $h$ and $h$ is
  the fixed side.
\end{enumerate}

We can verify that we get the same answer as usual when we apply the
disk method.

\subsection{Solids of revolution: the washer method}

What if the region being rotated is completely on one side of the axis
of rotation? For instance, imagine a disk far away from the $x$-axis
being revolved about the $x$-axis. The corresponding solid is
sometimes called a {\em filled torus} or {\em solid torus} (the
boundary of this, which is a surface obtained by revolving the
boundary circle, is usually simply called a {\em torus}).

The washer method is a method that allows us to compute the areas of
such solids. Again, the idea is to use parallel cross sections. In
this case, the cross sections are not disks, but regions called {\em
annuli}. Given a point $P$ and two concentric circles centered at $P$
(in the same plane) with radii $r < R$, the annulus for these two
radii is the set of points in the bigger disk that are not there in
the interior of the smaller disk. Thus, it is the region between the
circles of radii $r$ and $R$, along with the two boundary circles.

The area of such an annulus is given by $\pi(R^2 - r^2)$.

The upshot of this is that the volume of the solid of revolution
obtained by revolving the region between $y = g(x)$ and $y = f(x)$,
with $0 \le g(x) \le f(x)$, on $[a,b]$, is:

$$\int_a^b \pi[(f(x))^2 - (g(x))^2 ] \, dx$$

If the two functions cross each other, then if we are interested in
the unsigned volume, we need to split into intervals based on which
one is bigger where, calculate the volumes of the solids of revolution
corresponding to each interval, and add up. In other words, we need to
compute:

$$\int_a^b \pi|(f(x))^2 - (g(x))^2| \, dx$$

\subsection{Solids of revolution: the tale of the receding axis}

The first thing worth noticing about the volumes of solids of
revolution is that the volume is {\em not determined} by the area of
th region being rotated. It {\em also depends} on the choice of
axis. As a general rule, the farther the axis from the region being
rotated, the bigger the volume.

To understand this, consider the question: given a fixed number $h >
0$, what can we say about the area of the annulus of thickness $h$,
i.e., where the outer radius is $h$ more than the inner radius? For
fixed $h$, this number increases as we increase the two radii. This is
because the area is $\pi[(r + h)^2 - r^2] = \pi(2r + h)h$. The $2r +
h$ term increases as $r$ increases.

To give you some intuition about this, here is something that might
strike you as visually counterintuitive: the area of the annulus with
inner radius $4$ and outer radius $5$ equals the area of the disk of
radius $3$ (since $5^2 - 4^2 = 3^2$) even though the former has a much
smaller thickness. The smaller thickness is compensated for (roughly)
by the larger circumference.

The calculations that we did for the annulus show that as we move our
axis farther and farther from $\Omega$, the solid of revolution
becomes larger and larger in volume. Remember: to calculate the volume
of the solid of revolution, we create slices perpendicular to the axis
of revolution, but we are not integrating the length of these slices;
we are integrating the differences of {\em squares} of the endpoints
of the slices. And this difference of squares increases as both
numbers get bigger, even when the actual difference between them is
constant.

\subsection{Solids of revolution: when the axis straddles the region}

So far, we have considered a situation where the region being revolved
is completely on one side of the axis of revolution.

In the case that the region being revolved is partly on one side and
partly on the other side of the solid of revolution, we must keep the
following things in mind:

\begin{enumerate}
\item If the region has {\em mirror symmetry} about the axis of
  revolution, then we can simply delete the half of the region on one
  side and consider the solid of revolution for the other half.
\item Otherwise, in general, we must {\em fold} the region being
  rotated along the axis, i.e., reflect all the stuff on one side to
  the other, while keeping the stuff on the other side unchanged. Note
  that in the case of mirror symmetry, the reflected material overlaps
  with the original material. In some cases, such as the graph of a
  function about the $x$-axis, the reflected portion has no area of
  overlap with the stuff already there. In yet other cases, part of
  the reflected region overlaps, and the rest doesn't.
\end{enumerate}

\section{The shell method}

\subsection{Formula}

The shell method applies to situations where we revolve about the
$y$-axis the region made between a graph $y = f(x)$ and the
$x$-axis. As before, we work with a nonnegative continuous function
$f$ on a closed interval $[a,b]$ with $0 < a < b$. Consider the region
bounded by the graph of $f$, the $x$-axis, and the vertical lines $x =
a$ and $x = b$. Now, consider the solid of revolution obtained by
revolving this region about the $y$-axis. The volume of this solid of
revolution is given by the formula

$$\int_a^b 2\pi x f(x) \, dx = 2\pi \int_a^b xf(x) \, dx$$

In the case that the function is not nonnegative throughout, we can
use the more general formula:

$$\int_a^b 2\pi x |f(x)| \, dx$$

The best way of doing this is to partition the interval according to
the sign of $f$.
\subsection{Slight generalization of this formula}

Consider now a slightly more general situation: we are looking at the
region between the graphs of the functions $f$ and $g$ between $x = a$
and $x =b$. We consider the solid of revolution obtained by revolving
this region about the $y$-axis. If $g \le f$ on $[a,b]$, then the
volume is given by:

$$\int_a^b 2\pi x[f(x) - g(x)] \, dx = 2\pi \int_a^b x[f(x) - g(x)] \, dx$$

(Note: We don't need any conditions on the nonnegativity of $f$ and
$g$ here).

If $f$ and $g$ cross each other, we can use the general formula:

$$\int_a^b 2\pi x|f(x) - g(x)| \, dx$$

This is best handled by partitioning the interval according to where
$f$ is greater and where $g$ is greater.

\section{Average value point of view}

\subsection{Overview}

For the various approaches we have seen so far for volume computation,
there is an {\em average value point of view}. This can be thought of
as a process whereby we compare our actual imperfect solid to a more
perfect solid which is more uniform, and where the volume is given as
a simple product. Let's illustrate this by beginning with our
interpretation of volume as the integral of a variable cross section
area.

Here, our {\em ideal} figure is a right cylinder, where the cross
section area does not change for the cross sections (more generally,
this is also true for oblique cylinders). In these ideal figures, the
volume is the product of the constant cross section area and the
height.

The volume in general can be thought of as the product of the {\em
average} cross section area and the height. Here, the {\em average}
cross section area is {\em defined} the way we calculate the average
value for a function: we integrate it over the entire interval, and
then divide by the length of the interval. In other words, the average
cross section area is defined so that a right cylinder with that cross
section and the height of our current figure has the same volume.

How does the average value point of view help? Computationally, it
doesn't, but it gives us some intuition as to what kind of answers to
expect. This is because, looking at the figure, we have some ideas
about the average value: it must be somewhere between the minimum and
the maximum value, for instance. This provides a reality check on the
computations that we do.

\subsection{Average value for shell method}

Here, the ideal function is a constant function $f$ on $[a,b]$ with
constant value $C$. Revolving it about the $y$-axis yields a
cylindrical shell with inner radius $a$, outer radius $b$, and height
$C$. The volume is $\pi C(b^2 - a^2) = \pi C(b + a)(b - a)$. The value
$\pi C (b + a) = 2\pi C (b + a)/2$ is the curved surface area of the
cylinder whose radius is $(b + a)/2$, which is the cylinder whose
radius is halfway between the inner and outer radius. We thus see that:

$$\text{Volume of cylindrical shell} = \text{Curved surface area of mid-value cylinder} \times \text{Difference of outer and inner radii}$$

This is the ideal situation. In the real situation, we define the {\em
average curved surface area} as:

$$\text{Average curved surface area} = \frac{\text{Volume of solid of revolution}}{\text{Difference of upper and lower limits}}$$

In our notation, the denominator is $b - a$. Thus, we obtain that the
volume of the solid of revolution is $b - a$ times the {\em average
curved surface area}. As before, this is not really computationally
useful, but it might give us some intuition.

\subsection{Average value for disk method: different notions of average!}

The average value point can also be used to understand the disk method.

Recall that the volume of a solid of revolution obtained by revolving
about the $x$-axis the region between the $x$-axis and the graph of
$f$ from $x = a$ to $x = b$ is given by $\pi \int_a^b (f(x))^2 \,
dx$. Recall that we proved this formula by taking cross sections
perpendicular to the $x$-axis. The area of a cross section at the
value $x$ is $\pi(f(x))^2$, because the cross section is a disk of
radius $|f(x)|$.

Under the average value point of view, we are interested in the
average value of this cross section area. There's a little subtlety in
this.

To find the {\em area} between the graph of $f$ and the $x$-axis from
$x = a$ to $x = b$, we perform a simple integration $\int_a^b f(x) \,
dx$ (or $\int_a^b |f(x)| \, dx$). On the other hand, to find the
volume of the solid of revolution, we perform the integration
$\int_a^b (f(x))^2 \, dx$.

In other words, when finding the volume of the solid of revolution, we
give a lot more weight to larger radii -- because the radius is being
squared. Remember the discussion from last time where we saw that an
annulus with inner and outer radii $4$ and $5$ has the same area as
the disk of radius $3$. This is because the square of a number grows
much faster than the number itself.

Our averaging process is also correspondingly biased. When we are
calculating the average value in the ordinary sense, we do $\int_a^b
f(x) \, dx /(b - a)$. However, when calculating the average of the
areas of the disks, we are doing $\pi \int_a^b (f(x))^2 \, dx/(b -
a)$. The latter average value is {\em usually not the same} as the
area of the disk whose radius is the average radius. Rather, it is
usually larger, because taking the squares assigns greater weight to
the bigger radii.\footnote{It turns out that the two averages are
equal only for a constant function. The inequality being alluded to
here indirectly is known as the arithmeti mean-quadratic mean (AM-QM)
inequality or the arithmetic mean-root mean square (AM-RMS)
inequality.}

\section{Stock-taking}

We have now seen some formulas and general approaches that use the
ideas of integration to compute areas and volumes. Later in the course
and/or in later life, you will encounter formulas to do a lot of the
other things you've always wanted to do, such as formulas for arc
lengths and surface areas. We are not getting into those formulas
right now for two reasons: (i) they require more conceptual apparatus
to understand, (ii) the kind of expressions that you typically get to
integrate are expressions that you do not know how to deal with.

This brings us to one of the things that differentiates (pun!)
differentiation from integration. Differentiation was based on a set
of rules that we could apply blindly, because for every way of
combining and composing existing functions, we had a corresponding way
of breaking down the differentiation problem. With integration,
however, we are in more wild territory, since there are no easy
hard-and-fast rules and a lot depends on creativity and spotting
persuasive patterns. This makes integration more fascinating, but it
also means that ever so often, we come across a situation from the
real world that boils down to computing an integral, and we don't
really have an idea how to go about it.

Nonetheless, reducing a geometric problem of volume computation into a
purely arithmetic/algebraic problem of evaluating a definite integral
should be seen as a major step forward. Even if we have no clue about
what an antiderivative might be, we can still use the upper sum/lower
sum method to approximate this integral.

\section{More computational intuition}

\subsection{Stretching, shrinking, and scaling}

We can use the ``solid of revolution'' idea to compute the volume of a
sphere. A solid sphere of radius $r$ is obtained by revolving a
semicircular region of radius $r$ about its diameter. The volume
formula is thus:

$$\pi \int_{-r}^r (r^2 - x^2) \, dx$$

This gives the familiar formula $(4\pi/3)r^3$.

Note that a sphere is also the solid of revolution of a circular disk
about its diameter. As noted earlier, since a circular disk has mirror
symmetry about its diameter, so we can delete one of the semicircular
pieces and still get the same solid of revolution.

Now, let's think about what happens if, instead of revolving a
circular (or semicircular) disk, we revolve the region enclosed by an
ellipse about its major or minor axis. An ellipse oriented along the
axes and centered at the origin is a curve given by the equation:

$$\frac{x^2}{a^2} + \frac{y^2}{b^2} = 1$$

with $a,b$ positive.

If $a > b$ then the $x$-axis is the major axis and the $y$-axis is the
minor axis. Note that both the $x$-axis and $y$-axis are axes of
mirror symmetry; however, unlike the case of the circle, it is no
longer true that every line through the origin is an axis of mirror
symmetry.

Now, we could do the calculations pretty easily to compute the volume
of the solid of revolution about either axis, but let's give an
intuitive explanation that allows us to get at the answer. If we start
with a circle centered at the origin and of radius $b$ and stretch it
by a factor of $a/b$ in the $x$-direction, we get an ellipse. Clearly,
the {\em area} of the ellipse is therefore $a/b$ times the area of the
circle, hence it is $\pi a b$. What about the volume? We note that the
axis along which we integrate gets stretched by a factor of $a/b$. A
little thought now tells us that the answer will be $(4\pi/3)ab^2$.

More generally, we see that:

\begin{enumerate}
\item If the region being revolved is stretched by a factor of
  $\lambda$ {\em along} the axis of revolution, the volume is
  multiplied by a factor of $\lambda$.
\item If the region being revolved is stretch by a factor of $\mu$
  {\em along} the axis of revolution, the volume is multiplied by a
  factor of $\mu^2$. The square happens because when we revolve, the
  area contribution in each slice is proportional to the square of the
  radius or difference of squares of radius.
\end{enumerate}

Thus, if the same ellipse were revolved about its minor axis, we'd get
a volume of $(4\pi/3)a^2b$.

\subsection{Brief mention: Pappus' theorem}

Pappus' theorem is in a later section of the chapter that we're not
including in this course, but it's a theorem worth taking a look at
and understanding at least temporarily. For Exercise 6.3.44 (featuring
in Homework 8), Pappus' theorem gives an alternative solution approach
that is much shorter than the disk and shell methods that we will use
to solve the problem. It also tells us what the answer will be -- in
this case $2\pi^2a^3$. The reason it is easier is because for the case
of the circle, we know exactly where the center (centroid) is.

\end{document}