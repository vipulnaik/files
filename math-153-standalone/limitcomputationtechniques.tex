\documentclass{amsart}
\usepackage{fullpage,hyperref,vipul}
\title{Limit computation: formulas and techniques, including  L'H\^{o}pital's rule}
\author{Math 153, Section 55 (Vipul Naik)}

\begin{document}
\maketitle

{\bf Corresponding material in the book}: Sections 11.4, 11.5, 11.6.

{\bf What students should already know}: Basic limit computations.

{\bf What students should definitely get}: A good jolt that brings
back into active memory all the past knowledge of limits. The
statement of L'H\^{o}pital's rule for $0/0$ form and $\infty/\infty$
form, how to apply it, and the important constraints and caveats.

{\bf What students should hopefully get}: The notion of order of a
zero, how the LH rule can be effectively combined with stripping and
substitution, and the intuition for when a limit is going to be zero,
infinity, or finite and nonzero.

\section*{Executive summary}

Words ...

\begin{enumerate}
\item L'H\^{o}pital's rule for $0/0$ form: Consider $\lim_{x \to c}
  f(x)/g(x)$ where $\lim_{x \to c} f(x) = \lim_{x \to c} g(x) = 0$. If
  both $f$ and $g$ are differentiable around $c$, then this limit is
  equal to $\lim_{x \to c} f'(x)/g'(x)$.
\item Analogous statements hold for one-sided limits and in the case
  $c = \pm \infty$.
\item The $0/0$ form LH rule {\em cannot} be applied if the numerator
  does not limit to zero or if the denominator does not limit to zero.
\item L'H\^{o}pital's rule for $\infty/\infty$ form: Consider $\lim_{x
  \to c} f(x)/g(x)$ where $\lim_{x \to c} f(x) = \pm \infty$ and
  $\lim_{x \to c} g(x) = \pm \infty$. If both $f$ and $g$ are
  differentiable around $c$, then this limit is equal to $\lim_{x \to
  c} f'(x)/g'(x)$.
\item Analogous statements hold for one-sided limits and for $c = \pm
  \infty$.
\item For a function $f$ having a zero at $c$, the {\em order} of the
  zero at $c$ is the least upper bound of $\beta$ such that $\lim_{x
  \to c} |f(x)|/|(x - c)|^{\beta} = 0$. At this least upper bound, the
  limit is usually finite and nonzero. For larger values of $\beta$,
  the limit is undefined.
\item Given a quotient $f/g$ for which we need to calculate the limit
  at $c$, the limit is zero if the order of the zero for $f$ is
  greater than the order for $g$, and undefined if the order of the
  zero for $f$ is less than the order for $g$. When the orders are the
  same, the limit could potentially be a finite nonzero number.
\end{enumerate}

Actions ...

\begin{enumerate}
\item For polynomial functions and other continuous functions, we can
  calculate the limit at a point by evaluating at the point. For
  rational functions, we can cancel common factors between the
  numerator and the denominator till one of them becomes nonzero at
  the point.
\item There is a bunch of basic limits that translate to saying that
  for the following functions $f$: $f(0) = 0$ and $f'(0) = 1$, which
  is equivalent to saying that $\lim_{x \to 0} f(x)/x = 1$. These
  functions are $\sin x$, $x \mapsto \ln(1 + x)$, $x \mapsto e^x - 1$,
  $x \mapsto \tan x$, $x \mapsto \arcsin x$, and $x \mapsto \arctan
  x$.
\item For all the functions $f$ of the above kind (that we call {\em
  strippable}), the following is true: in any multiplicative
  situation, if the input to the function goes to zero in the limit,
  the function can be stripped off.
\item Two other basic limits are $\lim_{x \to 0} (1 - \cos x)/x^2 =
  1/2$ and $\lim_{x \to 0} (x - \sin x)/x^3 = 1/6$. These can be
  obtained using the LH rule.
\item Typically, to compute limits, we can combine the LH rule,
  stripping, and removing multiplicative components that we can
  calculate directly.
\item Applying the LH rule $0/0$ form pushes the orders of both the
  numerator and the denominator down by one.
\item It is also useful to remember that logarithmic functions are
  dominated by polynomial functions, which in turn are dominated by
  exponential functions. These facts can be seen in various ways,
  including the LH rule.
\end{enumerate}

\section{Limits: some of the key ideas}

\subsection{The basic process for real limits at finite values}

\begin{enumerate}
\item For continuous functions, evaluate at the point.
\item If evaluation gives a form that is determinately undefined (such
  as a quotient where the numerator approaches a finite nonzero number
  and the denominator approaches zero) then declare that the limit
  does not exist as a finite number. We may still be interested in
  whether the limit exists as $+\infty$ or $-\infty$, and we can
  follow the rules to find out if that is the case.
\item If evaluation gives an {\em indeterminate form}, manipulate the
  expression to get an {\em equivalent function} that can be evaluated
  at the point. For quotients, the process typically involves the
  cancellation of some common factors between the numerator and the
  denominator.
\end{enumerate}

\subsection{Some important basic cases}

\begin{enumerate}
\item For the limit of a rational function $p(x)/q(x)$ at $x = a$, try
  evaluating at $a$. If evaluation gives a $0/0$ form, that means that
  $x - a$ divides both $p(x)$ and $q(x)$. Factorize and cancel this
  common factor. Now evaluate again. If we still get a $0/0$ form,
  there is another $x -a$ common to both the numerator and the
  denominator, and it can be canceled again. The process continues
  till either the numerator or the denominator has no factor of $x -
  a$. The limit is (a) finite and nonzero if $x - a$ divides neither
  numerator nor denominator, (b) zero if $x - a$ divides the numerator
  but not the denominator, and (c) undefined if $x - a$ divides the
  denominator but not the numerator.
\item For limits involving trigonometric functions, we can use
  trigonometric identities in case the expression is of a $0/0$
  form. Remember to {\em first evaluate and check}.
\item For limits involving a mix of trigonometric functions and
  polynomial functions, we use the standard limit $\lim_{x \to 0}
  (\sin x)/x = 1$ and its many corollaries, such as $\lim_{x \to
  0} (\tan x)/x = 1$ and $\lim_{x \to 0} (1 - \cos x)/x^2 = 1/2$.
\end{enumerate}

\subsection{Some key important limits}

Some key nontrivial limits that we need to remember are:

\begin{eqnarray*}
  \lim_{x \to 0} \frac{\sin x}{x} & = & 1\\
  \lim_{x \to 0} \frac{1 - \cos x}{x^2} & = & \frac{1}{2}\\
  \lim_{x \to 0} \frac{\tan x}{x} & = & 1 \\
  \lim_{x \to 0} \frac{e^x - 1}{x} & = & 1 \\
  \lim_{x \to 0} \frac{\ln(1 + x)}{x} & = & 1\\
  \lim_{x \to 0} \frac{\arcsin x}{x} & = & 1 \\
  \lim_{x \to 0} \frac{\arctan x}{x} & = & 1
\end{eqnarray*}

Each of these limits has an interpretation as a derivative: it simply
says that the numerator functions all have value $0$ and derivative
$1$ at $x = 0$. We will see this in more detail later.

\subsection{Seeing these limits in composites}

Consider the limit:

$$\lim_{x \to 0} \frac{\ln(1 + \sin x)}{x}$$

Note that the numerator is obtained by {\em composing} the $\sin$
function and the $t \mapsto \ln(1 + t)$ function. This composition of
functions translates, in computational terms, to a {\em product} of
limits. In a process similar to the chain rule, we write the limit as:

$$\lim_{x \to 0} \frac{\ln(1 + \sin x)}{\sin x} \frac{\sin x}{x}$$

We now split the limit:

$$\lim_{x \to 0} \frac{\ln(1 + \sin x)}{\sin x} \lim_{x \to 0} \frac{\sin x}{x}$$

Now, the second limit is $1$. The first limit is {\em also} $1$,
because as $x \to 0$, we also have $\sin x \to 0$, and we can thus
apply the limit formula.

This compositional breaking up is sometimes accompanied by a few
constants here and there. For instance:

$$\lim_{x \to 0} \frac{\sin(2\ln(1 + x^2))}{e^{x^2} - 1}$$

This is too tedious to write in full, but the numerator is, in the
limit, multiplicatively the same as $2\ln(1 + x^2)$ (i.e., we can
strip the $\sin$ off, which is twice of $\ln(1 + x^2)$, which is
multiplicatively the same as $x^2$. The denominator is
multiplicatively the same as $x^2$. So, the overall quotient is $2$.

Similarly, consider:

$$\lim_{x \to 0} \frac{\sin(ax)}{\sin(bx)}$$

where $a,b \in \R$ and $b \ne 0$. In this case, the numerator is
multiplicatively the same as $ax$, and the denominator is
multiplicatively the same as $bx$, so the quotient is $a/b$. On the
other hand:

$$\lim_{x \to 0} \frac{\cos(ax)}{\cos(bx)}$$

succumbs to a direct evaluation: at $0$, both the numerator and the
denominator equal $1$, so the quotient equals $1$.

\subsection{Heavier and lighter denominators}

Consider the limit:

$$\lim_{x \to 0} \frac{\sin x}{x^2}$$

Here, the numerator is roughly like $x$ (multiplicatively), so the
limit becomes:

$$\lim_{x \to 0} \frac{x}{x^2} = \lim_{x \to 0} \frac{1}{x}$$

which is undefined. On the other hand:

$$\lim_{x \to 0} \frac{\sin x}{\sqrt{x}} = \lim_{x \to 0} \frac{x}{\sqrt{x}} = \lim_{x \to 0} \sqrt{x} = 0$$

Basically, we can {\em strip off} things which multiplicatively go to
$1$ and apply the usual rules of limits for rational functions to
whatever is left (more on this in the next section).

\subsection{Sum and product breakups}

Also consider:

$$\lim_{x \to 0} \frac{\sin(\pi x) \sin(ex)}{x^2}$$

You can convince yourself that the limit in this case is $\pi e$. This
essentially arises by breaking up the limit as the product of two
limits. Similarly:

$$\lim_{x \to 0} \frac{\sin(\sqrt{2}x) + \tan(\sqrt{3}x)}{x}$$

is $\sqrt{2} + \sqrt{3}$ -- again obtained by breaking up.

\section{Limits at infinity}

We have previously reviewed the limits of rational functions at
infinity, so we will not cover them again. Rather, we concentrate on a
general approach here: to calculate $\lim_{x \to \infty} f(x)$, put $t
= 1/x$ and calculate $\lim_{t \to 0^+} f(1/t)$.

\subsection{The original identities transformed to limits at infinity}

\begin{eqnarray*}
  \lim_{x \to \infty} x\sin(1/x) & = & 1\\
  \lim_{x \to \infty} x^2(1 - \cos(1/x)) & = & \frac{1}{2}\\
  \lim_{x \to \infty} x\tan(1/x) & = & 1 \\
  \lim_{x \to \infty} x(e^{1/x} - 1) & = & 1 \\
  \lim_{x \to \infty} x\ln(1 + (1/x)) & = & 1
\end{eqnarray*}

The last identity is particularly important, and is also often written
in exponential form:

$$\lim_{x \to \infty} \left[1 + \frac{1}{x}\right]^x = e$$

In general, whenever we have a limit of the form:

$$\lim_{x \to \infty} \left[1 + \frac{1}{f(x)}\right]^{g(x)}$$

where, as $x \to \infty$, we also have $f(x) \to \infty$, then this
just becomes:

$$\lim_{x \to \infty} \exp\left[\frac{g(x)}{f(x)}\right]$$

Try to convince yourself of this.

\section{Stripping in full glory}

This is an intuitive way of thinking about and calculating {\em some}
limits that incorporates the principles of chaining discussed above,
without rewriting all the intermediate steps in gory details. We had a
sneak peek of this approach in Math 152, but now we go the whole hog.

\subsection{Simple examples}

To motivate stripping, let us look at a fancy example:

$$\lim_{x \to 0} \frac{\sin(\tan(\sin x))}{x}$$

This is a composite of three functions, so if we want to chain it, we
will chain it as follows:

$$\lim_{x \to 0} \frac{\sin(\tan(\sin x))}{\tan(\sin x)}\frac{\tan(\sin x)}{\sin x}\frac{\sin x}{x}$$

We now split the limit as a product, and we get:

$$\lim_{x \to 0} \frac{\sin(\tan(\sin x))}{\tan(\sin x)}\lim_{x \to 0} \frac{\tan(\sin x)}{\sin x}\lim_{x \to 0} \frac{\sin x}{x}$$

Now, we argue that each of the inner limits is $1$. The final limit is
clearly $1$. The middle limit is $1$ because the inner function $\sin
x$ goes to $0$. The left most limit is $1$ because the inner function
$\tan(\sin x)$ goes to $0$. Thus, the product is $1 \times 1 \times 1$
which is $1$.

If you are convinced, you can further convince yourself that the same
principle applies to a much more convoluted composite:

$$\lim_{x \to 0} \frac{\sin(\sin(\tan(\sin(\tan(\tan x)))))}{x}$$

However, {\em writing that thing out takes loads of time}. Wouldn't it
be nice if we could just strip off those $\sin$s and $\tan$s? In fact,
we can do that.

The key stripping rule is this: {\em in a multiplicative situation}
(i.e. there is no addition or subtraction happening), if we see
something like $\sin(f(x))$ or $\tan(f(x))$, and $f(x) \to 0$ in the
relevant limit, then we can strip off the $\sin$ or $\tan$. In this
sense, both $\sin$ and $\tan$ are {\em strippable} functions. A
function $g$ is strippable if $\lim_{x \to 0} g(x)/x = 1$.

The reason we can strip off the $\sin$ from $\sin(f(x))$ is that we
can multiply and divide by $f(x)$, just as we did in the above
examples.

Stripping can be viewed as a special case of the l'Hopital rule as
well, but it's a much quicker shortcut in the cases where it works.

Thus, in the above examples, we could just have stripped off the
$\sin$s and $\tan$s all the way through.

Here's another example:

$$\lim_{x \to 0} \frac{\sin(2 \tan (3x))}{x}$$

As $x \to 0$, $3x \to 0$, so $2 \tan 3x \to 0$. Thus, we can strip off
the outer $\sin$. We can then strip off the inner $\tan$ as well,
since its input $3x$ goes to $0$. We are thus left with:

$$\lim_{x \to 0} \frac{2(3x)}{x}$$

Cancel the $x$ and get a $6$. We could also do this problem by
chaining or the l'Hopital rule, but stripping is quicker and perhaps
more intuitive.

Here's yet another example:

$$\lim_{x \to 0} \frac{\sin (x \sin (\sin x))}{x^2}$$

As $x \to 0$, $x \sin(\sin x) \to 0$, so we can strip off the
outermost $\sin$ and get:

$$\lim_{x \to 0} \frac{x \sin(\sin x)}{x^2}$$

We cancel a factor of $x$ and get:

$$\lim_{x \to 0} \frac{\sin(\sin x)}{x}$$

Two quick $\sin$ strips and we get $x/x$, which becomes $1$.

Yet another example:

$$\lim_{x \to 0} \frac{\sin(ax)\tan(bx)}{x}$$

where $a$ and $b$ are constants. Since this is a multiplicative
situation, and $ax \to 0$ and $bx \to 0$, we can strip the $\sin$ and
$\tan$, and get:

$$\lim_{x \to 0} \frac{(ax)(bx)}{x}$$

This limit becomes $0$, because there is a $x^2$ in the numerator and
a $x$ in the denominator, and cancellation of one factor still leaves
a $x$ in the numerator.

Here is yet another example:

$$\lim_{x \to 0} \frac{\sin^2(ax)}{\sin^2(bx)}$$

where $a,b$ are nonzero constants. We can pull the square out of the
whole expression, strip the $\sin$s in both numerator and denominator,
and end up with $a^2/b^2$.

Here's another example:

$$\lim_{x \to 0} \frac{\arcsin(2\sin^2x)}{x \arctan x}$$

Repeated stripping reveals that the answer is $2$. Note that $\arcsin$
and $\arctan$ are also strippable because $\lim_{x \to 0} (\arcsin
x)/x = 1$ and $\lim_{x \to 0} (\arctan x)/x = 1$.

\subsection{A more complicated example}

Here's another example:

$$\lim_{x \to 0} \frac{\ln(1 + \sin(3x\sin(2\sin x)))}{x \tan(e^x - 1)}$$

We note that as $x \to 0$, we have $3x \sin(2 \sin x) \to 0$, so
$\sin$ of that goes to $0$, so we can strip off $\ln (1 + \_)$ and get:

$$\lim_{x \to 0} \frac{\sin(3x\sin(2\sin x))}{x \tan(e^x - 1)}$$

We can now strip off the outer $\sin$, then take $3$ outside of the
limit, to get:

$$3 \lim_{x \to 0} \frac{x\sin(2\sin x)}{x \tan(e^x - 1)}$$

The $x$ and $x$ cancel, and we get:

$$3 \lim_{x \to 0} \frac{\sin(2\sin x)}{\tan(e^x - 1)}$$

Now, the $\sin$ gets stripped off from the numerator, and the $\tan$
from the denominator, because the inputs to both functions go to $0$. So we are left with:

$$6 \lim_{x \to 0} \frac{\sin x}{e^x - 1}$$

Finally, the $\sin$ and the $\exp - 1$ can be stripped off and we just
get the answer as $6$.

Note that $\sin$, $\tan$, $\ln(1 + \_)$, and $\exp(\_) - 1$ are all
strippable functions, because for each of these, the limit of the
function divided by its input goes to $1$ as the input goes to $0$.

\subsection{Strip with discretion}

Consider:

$$\lim_{x \to 0} \frac{\sin(x\sin(\pi \cos x))}{x^2}$$

Here, the outer $\sin$ can be stripped off, because its input goes to
$0$, so we get:

$$\lim_{x \to 0} \frac{x \sin(\pi \cos x)}{x^2}$$

Cancel the $x$ and we get:

$$\lim_{x \to 0} \frac{\sin(\pi \cos x)}{x}$$

However, it is {\em not} legitimate now to strip off the $\sin$ at
this stage, because the input to the sign, namely $\pi \cos x$, does
not tend to $0$ as $x \to 0$. There are other right ways of doing this
problem, which we will discuss later, but stripping is not one of them.

\subsection{When you can't strip}

The kind of situations where we are not allowed to strip are where the
expression $\sin(f(x))$ is not just multiplied but is being added to
or subtracted from something else. For instance, in order to calculate
the limit:

$$\lim_{x \to 0} \frac{x - \sin x}{x^3}$$

We are {\em not} allowed to strip.

Roughly speaking, what goes wrong if we just strip is something like
this: $x$ and $\sin x$ are {\em multiplicatively} the same, but they
are {\em additively} different, near $0$. When taking the difference,
we are picking out on a very thin, very small additive difference that
is invisible multiplicatively.

\subsection*{What about $1 - \cos(f(x))$?}

If we see something like $1 - \cos(f(x))$ with $f(x) \to 0$ in the
limit, then we know that it ``looks like'' $f(x)^2/2$ and we can
replace it by $f(x)^2/2$. This can be thought of as a sophisticated
version of stripping. For instance:

$$\lim_{x \to 0} \frac{(1 - \cos(5x^2))}{x^4}$$

Here $f(x) = 5x^2$, so the numerator is like $(5x^2)^2/2$, and the
limit just becomes $25/2$.

\section{Understanding indeterminate form $0/0$, towards L'H\^{o}pital's rule}

The expression $0/0$ is undefined. It does not make sense to divide
$0$ by $0$, because any real number times $0$ is $0$. However, $(\to
0)/(\to 0)$ is an {\em indeterminate form}. In other words, if we're
trying to compute the limit:

$$\lim_{x \to c} \frac{f(x)}{g(x)}$$

such that $\lim_{x \to c} f(x) = 0$ and $\lim_{x \to c} g(x) = 0$,
then we cannot {\em prima facie} say anything for sure about the limit
of the quotient. Let's try to understand the source of the uncertainty
a little more carefully.

\subsection{The rate of approach to zero}

The key thing to understand is that even though two quantities may
approach $0$, they may approach it at very different rates. For
instance, consider $f(x) = 4x$ and $g(x) = 7x$. As $x$ approaches $0$,
both $f(x)$ and $g(x)$ approach $0$. However, for small $x$, $f(x)$
comes a little closer to $0$ than $g(x)$. The rate of approach for $f$
is $4$ and the rate of approach of $g$ is $7$. The quotient thus
approaches $4/7$.

Essentially, if $\lim_{x \to c} f(x) = 0$, then the derivative $f'(c)$
describes the rate at which $f$ approaches $0$ at $c$. The smaller the
value of $f'(c)$, the faster $f$ approaches $0$. Thus, the derivative
measures the rate of approach to zero. To determine the limit of the
quotient of two functions, we need to compare their rates of approach
to zero. This is the intuitive basis for L'H\^{o}pital's rule.

\subsection{A primitive form of the rule}

This primitive form says that if $f$ and $g$ are continuous at $c$ and
$f(c) = g(c) = 0$, and if $f'(c)$ and $g'(c)$ are both defined, with $g'(c)
\ne 0$, then we have:

$$\lim_{x \to c} \frac{f(x)}{g(x)} = \frac{f'(c)}{g'(c)}$$

Here is how we can think of this intuitively. Near the point $(c,0)$,
the graph of $f$ looks very close to the straight line passing through
$(c,0)$ with slope $f'(c)$ and the graph of $g$ looks very close to
the straight line passing through $(c,0)$ with slope $g'(c)$. The
quotient of the function values is, in the limit, the same as the
quotient of the slopes of these lines.

\subsection{A more sophisticated form of the rule}

The primitive version above works if $g$ has nonzero derivative at
$c$. A slight modification of it works if $g$ has zero derivative at
$c$. Namely, if $\lim_{x \to c} f(x) = \lim_{x \to c} g(x) = 0$ and $f$
and $g$ are both differentiable around $c$ with $g$ not identically
zero around $c$, then:

$$\lim_{x \to c} \frac{f(x)}{g(x)} = \lim_{x \to c}\frac{f'(x)}{g'(x)}$$

The advantage of this is that this works even if $g'(c)$ is zero or
undefined, i.e., it works even at critical points of $c$. After one
application of this, there are three possibilities:

\begin{enumerate}
\item The new limit can be directly computed because the denominator
  approaches a nonzero number.
\item The denominator approaches zero and the numerator approaches a
  nonzero number. The limit in this case is undefined.
\item Both the numerator and denominator approach $0$. We thus {\em
  again} have the indeterminate form $(\to 0)/(\to 0)$. We can now use
  the usual suite of techniques, which includes applying
  L'H\^{o}pital's rule yet again.
\end{enumerate}

There are actually more possibilities, because the numerator and/or
denominator may approach $\infty$. But we ignore those possibilities
for now just to keep the narrative simple.
\subsection{Caveat for applying the rule}

The rule as stated here should be applied {\em only} when both the
numerator and the denominator approach $0$. (We shall later see that
it also applies when both approach $\infty$). This is important
because applying the rule in other situations yields incorrect
answers. Simply speaking, it makes sense to compare the {\em rates of
approach to zero} only when they both {\em are actually heading to
zero}.

\subsection{A formal proof of the rule}

We prove here the primitive version of the rule, which assumes that
$f$ and $g$ are continuous and differentiable at $c$ with $f(c) = g(c)
= 0$. We consider:

$$\lim_{x \to c} \frac{f(x)}{g(x)}$$

Using $f(c) = g(c) = 0$, we can rewrite this as:

$$\lim_{x \to c} \frac{f(x) - f(c)}{g(x) - g(c)}$$

Dividing both numerator and denominator by $x - c$, we obtain:

$$\lim_{x \to c} \frac{\frac{f(x) - f(c)}{x - c}}{\frac{g(x) - g(c)}{x - c}}$$

The limit of the quotients is the quotient of the limits, and we obtain:

$$\frac{f'(c)}{g'(c)}$$

The same idea can be extended to the more general case, but the
details get more messy so we avoid this.

\subsection{The rule applies to one-sided limits and limits for $c = \infty$}

L'H\^{o}pital's rule also applies for one-sided limits (i.e., the left
hand limit and the right hand limit) for finite values of $c$ and for
the limit at $c = \infty$ or at $c = -\infty$. When applying to
one-sided limits, we take the corresponding one-sided limit of the
derivative. Thus, if $\lim_{x \to c^-} f(x) = \lim_{x \to c^-} g(x) =
0$, and if $f$ and $g$ are differentiable on the immediate left of
$c$, we have:

$$\lim_{x \to c^-} \frac{f(x)}{g(x)} = \lim_{x \to c^-} \frac{f'(x)}{g'(x)}$$

Similarly, if $\lim_{x \to \infty} f(x) = \lim_{x \to \infty} g(x) =
0$, and if $f$ and $g$ are differentiable for large inputs, we have:

$$\lim_{x \to \infty} \frac{f(x)}{g(x)} = \lim_{x \to \infty} \frac{f'(x)}{g'(x)}$$
\section{Applying L'H\^{o}pital's rule}

\subsection{Simple applications}

Consider the limit:

$$\lim_{x \to 0^+} \frac{\sin x}{\sqrt{x}}$$

This is a $0/0$ form. Our earlier way of doing this limit was simply
to strip off the $\sin$ in front of the $x$, notice that we get
$x/\sqrt{x} = \sqrt{x}$, which goes to $0$. Now, we try the LH rule,
and we get:

$$\lim_{x \to 0^+} \frac{\cos x}{1/(2\sqrt{x})}$$

A good shorthand for the application of this rule, used in the book,
is to write it using $\stackrel{*}{=}$, as follows:

$$\lim_{x \to 0^+} \frac{\sin x}{\sqrt{x}} \stackrel{*}{=} \lim_{x \to 0^+} \frac{\cos x}{1/(2\sqrt{x})} = \lim_{x \to 0^+} 2 \sqrt{x} \cos x$$

We now evaluate and obtain $0$.

Notice that in the earlier procedure (that did not use the LH rule),
we did not obtain any factor of $2$ -- we just obtained the limit of
$\cos x$. In both cases, the final answer is $0$. It turns out that
with the LH rule, when the answer is $0$ or $\infty$, using the LH
rule often gives a lot of superfluous constants that ultimately don't
matter, as compared to the primitive stripping techniques. On the
other hand, in those cases where the limit is nonzero, all those
constants {\em do} end up mattering.

\subsection{Another example}

Consider the example:

$$\lim_{x \to 0} \frac{x - \sin x}{x^3}$$

The first thing we try is evaluate. We notice that the numerator and
denominator both approach zero. We also notice something else: there
is no simple way to get started on the problem. For instance, if we
try to split across the $-$ sign, we get $1/x^2 - (\sin x)/x^2$, both
of which are undefined in the limit. Also, since the numerator is an
additive mix of the algebraic and the trigonometric, we cannot
directly use trigonometric identities.

So we use the LH rule. We obtain:

$$\lim_{x \to 0} \frac{x - \sin x}{x^3} \stackrel{*}{=} \lim_{x \to 0} \frac{1 - \cos x}{3x^2}$$

At this stage, we have two choices. The limit $\lim_{x \to 0} (1 -
\cos x)/x^2$ is a standard trigonometric limit that we know to be
$1/2$, and we can use this to get the answer of $1/6$. If, however, we
have forgotten the answer and the procedure for computing this limit,
we can use the LH rule again, noting that it is again the $(\to
0)/(\to 0)$ form. We get:

$$\lim_{x \to 0} \frac{x - \sin x}{x^3} \stackrel{*}{=} \lim_{x \to 0} \frac{1 - \cos x}{3x^2} = \lim_{x \to 0} \frac{\sin x}{6x}$$

Again, we can choose from two alternatives: use the well-known fact
that $\lim_{x \to 0} (\sin x)/x = 1$, or conveniently forget it and
hence {\em again} use the LH rule. We get:

$$\lim_{x \to 0} \frac{x - \sin x}{x^3} \stackrel{*}{=} \lim_{x \to 0} \frac{1 - \cos x}{3x^2} = \lim_{x \to 0} \frac{\sin x}{6x} = \lim_{x \to 0} \frac{\cos x}{6}$$

Now, we no longer have a $0/0$ form, so we {\em must} evaluate, and we
obtain the answer $1/6$.

The point here is that the LH rule can be used in conjunction with
other rules. If using the LH rule, you can choose to keep
differentiating till you get something that is not of the $0/0$
form. But you can also choose to fork at any stage where you know how
to compute the limit by other means.

\subsection{Combining the LH rule with stripping}

Stripping is the procedure where we throw off unnecessarily confusing
layers. For instance, if the previous question had been:

$$\lim_{x \to 0} \frac{\sin(x - \sin x)}{\ln(1 + x^3)}$$

We could {\em strip off} the $\sin$ in the numerator and the $\ln(1 +
\_)$ in the denominator to get:

$$\lim_{x \to 0} \frac{x - \sin x}{x^3}$$

Now, {\em this} limit can be evaluated using the LH rule. Note that
even the original limit can be evaluated using the LH rule, but the
derivatives get extremely messy, and these messy derivatives don't
contribute anything to the discussion.

\subsection{Combining the LH rule with substitution}

Consider:

$$\lim_{x \to 0} \frac{x - \arcsin x}{x^3}$$

We can do this using the LH rule, but the derivatives of $\arcsin$
tend to get messy. It is smarter here to put $\theta = \arcsin x$, and
obtain:

$$\lim_{\theta \to 0} \frac{\sin \theta - \theta}{\sin^3\theta}$$

Using stripping, this can be transformed to:

$$\lim_{\theta \to 0} \frac{\sin \theta - \theta}{\theta^3}$$

which, as we have seen above, is $-1/6$.

\subsection{The order of a function near a zero}

The LH rule and prior discussions give rise to a notion of the order
of a function near a zero of the function. Suppose that, as $x \to c$,
we have $f(x) \to 0$. The {\em order} of $f$ at $c$ is the least
upper bound of the set of $\beta$ for which $\lim_{x \to c}
f(x)/(x-c)^\beta = 0$.

For instance, for the function $x$ at $c = 0$, the order is $1$,
because $\lim_{x \to 0} x/x^\beta = 0$ for all $\beta < 1$, whereas
$\lim_{x \to 0} x/x^\beta$ is undefined for all $\beta > 1$. The
number $1$ is a threshold value at which the limit goes from being $0$
to being undefined.

Here now are some observations:

\begin{enumerate}
\item For any function $f$ with the property that $f(c) = 0$ and
  $f'(c)$ is finite and nonzero, $f$ has order $1$ at $c$. In other
  words, $\lim_{x \to c} |f(x)|/|x-c|^\beta = 0$ for $\beta < 1$ and
  $\lim_{x \to c} |f(x)|/|x-c|^\beta$ is undefined for $\beta > 1$. At
  $\beta = 1$, the limit is $f'(c)$, which is a finite and nonzero
  number.

  Examples for $c = 0$ are the functions $\sin x$, $\ln(1 + x)$, $\tan
  x$, $e^x - 1$, and $\arcsin x$.
\item More generally, if $f(c) = 0$ and the smallest order of
  derivative of $f$ that is nonzero at $c$ is $k$ (i.e., the $k^{th}$
  derivative of $f$ is nonzero at $c$ and all lower order derivatives
  are zero), then the order of $f$ at $c$ is $k$. In other words,
  $\lim_{x \to c} f(x)/(x-c)^\beta = 0$ for all $\beta < k$ and
  $\lim_{x \to c} f(x)/(x-c)^\beta$ is undefined for all $\beta >
  k$. At $\beta = k$, the limit is finite and nonzero, and is in fact
  equal to $f^{(k)}(c)/k!$, as we can readily see by a repeated
  application of the LH rule.

  Here are some examples of functions that have order $2$ at $c = 0$:
  $x^2$, $1 - \cos x$, $e^x - 1 - x$, $\ln(1 + x^2)$.

  Here are some examples of functions that have order $3$ at $c = 0$:
  $x - \sin x$, $x - \tan x$, $x^3$.
\item The function $f(x) := x^r$ for some $r > 0$ has order $r$ at
  $0$. $r$ need not be an integer.
\item Multiplication adds orders and function composition multiplies orders. 
\end{enumerate}

We can now interpret the LH rule as follows:

\begin{enumerate}
\item Each application of the LH rule pushes down the order of both
  the numerator and the denominator by $1$.
\item When the numerator has higher order than the denominator, then
  the quotient approaches zero. In the case where both orders are
  positive integers, repeated application of the LH rule will get us
  to a situation where the denominator becomes nonzero (because the
  order of the zero in the denominator becomes zero) while the
  numerator is still zero (because the order of the zero in the
  denominator is still positive) -- yes, you read that correct.
\item When the numerator and the denominator have the same order, the
  quotient {\em could} approach something finite and nonzero. In most
  cases, repeated application of the LH rule gets us down to a
  quotient of two nonzero quantities.
\item When the denominator has the higher order, the quotient has an
  undefined limit (the one-sided limits are usually $\pm \infty$). In
  the case where both orders are positive integers, repeated
  application of the LH rule will get us to a situation where the
  numerator becomes nonzero while the denominator is still zero
  (because the order of the zero is still positive).
\end{enumerate}

It should be possible to determine these limits at $0$ simply by
inspection: $(\cos x)/x$, $(\sin x)/x^2$, $(\sin^2 x)/x$, $(1 - \cos
x)/x^3$, $(x - \sin x)/x^2$.

One way of thinking of this is in terms of a race for domination. The
order roughly measures the extent to which something is zero.

\section{The LH rule for $\infty/\infty$}
This version of the LH rule states that if both the numerator and the
denominator of a fraction go to $\pm \infty$ (with the two infinities
not necessarily being the same) then the limit of the quotient equals
the limit of the quotient of the derivatives. In symbols, if $f(x) \to
\pm \infty$ and $g(x) \to \pm \infty$ as $x \to c$ and $f$ and $g$ are
both differentiable around $c$ with $g$ not identically zero near $c$,
we have:

$$\lim_{x \to c} \frac{f(x)}{g(x)} = \lim_{x \to c} \frac{f'(x)}{g'(x)}$$

This version of the LH rule should {\em not} be mixed with the earlier
version. In particular, if the numerator approaches $0$ and the
denominator approaches $+\infty$, the LH rule does not
apply. Similarly, if the numerator approaches $-\infty$ and the
denominator approaches $0$, the LH rule does not apply. (In both these
cases, the limits are directly obvious: in the first case, the limit
is $0$, and in the second, the limit is undefined).

\subsection{Example of $e^x/p(x)$}

$e^x$ races out polynomial functions. Specifically, for any nonzero
polynomial function $p(x)$, $\lim_{x \to \infty} \frac{e^x}{p(x)} =
\pm \infty$, with the sign of infinity depending upon the leading
coefficient of $p$. The LH rule gives an easy proof. For a polynomial
of degree $k$, applying the LH rule $k$ times gives
$\frac{e^x}{\text{constant}}$ where the constant in the denominator
has the same sign as the leading coefficient of $p$ (in fact, it is
$k!$ times the leading coefficient of $p$). Since $e^x \to \infty$ as
$x \to \infty$, we see tha tthis limit is $\infty$.

\subsection{Obtaining these from other indeterminate forms}

Suppose we have an indeterminate form other than $0/0$ or
$\infty/\infty$. We cannot directly use the LH rule. However, we can
often do some manipulation to bring it into the $0/0$ or
$\infty/\infty$ form where the LH rule can be applied. In particular:

\begin{enumerate}
\item A form of $(\to 0)(\to \infty)$ can be rewritten as $(\to
  0)/(\to 0)$ or as $(\to \infty)/(\to \infty)$, by taking the
  reciprocal of one of the factors. For instance, $\lim_{x \to \infty}
  xe^{-x}$ can be written as $\lim_{x \to \infty} x/e^x$.
\item A form of $\to \infty - \to \infty$ can often be converted to a
  $\to 0/\to 0$ form. The trick is usually to take a common
  denominator and simplify. For instance, $(1/x) - (1/\sin x)$ can be
  rewritten as $(\sin x - x)/x^2$.
\end{enumerate}

\end{document}