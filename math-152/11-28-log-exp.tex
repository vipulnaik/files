\documentclass[10pt]{amsart}

%Packages in use
\usepackage{fullpage, hyperref, vipul, enumerate}

%Title details
\title{Class quiz: November 28: Logarithm and exponential}
\author{Math 152, Section 55 (Vipul Naik)}
%List of new commands

\begin{document}
\maketitle

Your name (print clearly in capital letters): $\underline{\qquad\qquad\qquad\qquad\qquad\qquad\qquad\qquad\qquad\qquad}$

Note: I didn't administer this quiz last year, so I don't have data on
the level of difficulty of the questions. Thus, the starring of
questions is based on guesswork.

\begin{enumerate}
\item Consider the function $f(x) := \exp(5 \ln x)$ defined for $x \in
  (0,\infty)$. How does $f(x)$ grow as a function of $x$?

  \begin{enumerate}[(A)]
  \item As a linear function
  \item As a polynomial function but faster than a linear function
  \item Faster than a polynomial function but slower than an
    exponential function
  \item As an exponential function, i.e., $x \mapsto \exp(kx)$ for some $k > 0$
  \item Faster than an exponential function
  \end{enumerate}

  \vspace{0.1in}
  Your answer: $\underline{\qquad\qquad\qquad\qquad\qquad\qquad\qquad}$
  \vspace{0.6in}

\item Consider the function $f(x) := \ln (5 \exp x)$ for $x \in
  (0,\infty)$. How does $f(x)$ grow as a function of $x$?

  \begin{enumerate}[(A)]
  \item As a linear function
  \item As a polynomial function but faster than a linear function
  \item Faster than a polynomial function but slower than an
    exponential function
  \item As an exponential function, i.e., $x \mapsto \exp(kx)$ for some $k > 0$
  \item Faster than an exponential function
  \end{enumerate}

  \vspace{0.1in}
  Your answer: $\underline{\qquad\qquad\qquad\qquad\qquad\qquad\qquad}$
  \vspace{0.6in}

\item Consider the function $f(x) := \ln((\exp x)^5)$ defined for $x \in
  (0,\infty)$. How does $f(x)$ grow as a function of $x$?

  \begin{enumerate}[(A)]
  \item As a linear function
  \item As a polynomial function but faster than a linear function
  \item Faster than a polynomial function but slower than an
    exponential function
  \item As an exponential function, i.e., $x \mapsto \exp(kx)$ for some $k > 0$
  \item Faster than an exponential function
  \end{enumerate}

  \vspace{0.1in}
  Your answer: $\underline{\qquad\qquad\qquad\qquad\qquad\qquad\qquad}$
  \vspace{0.6in}

  {\bf PLEASE TURN OVER FOR REMAINING QUESTIONS}

\newpage
\item (*) Consider the function $f(x) := \exp((\ln x)^5)$ defined for
  $x \in (0,\infty)$. How does $f(x)$ grow as a function of $x$?

  \begin{enumerate}[(A)]
  \item As a linear function
  \item As a polynomial function but faster than a linear function
  \item Faster than a polynomial function but slower than an
    exponential function
  \item As an exponential function, i.e., $x \mapsto \exp(kx)$ for some $k > 0$
  \item Faster than an exponential function
  \end{enumerate}

  \vspace{0.1in}
  Your answer: $\underline{\qquad\qquad\qquad\qquad\qquad\qquad\qquad}$
  \vspace{0.6in}

\item (*) {\em Consumption smoothing}: A certain measure of happiness
  is found to be a logarithmic function of consumption, i.e., the
  happiness level $H$ of a person is found to be of the form $H = a +
  b \ln C$ where $C$ is the person's current consumption level, and
  $a$ and $b$ are positive constants independent of the consumption
  level.

  The person has a certain total consumption $C_{tot}$ to be split
  within two years, year 1 and year 2, i.e., $C_{tot} = C_1 +
  C_2$. Thus, the person's happiness level in year 1 is $H_1 = a + b
  \ln C_1$ and the person's happiness level in year 2 is $H_2 = a + b
  \ln C_2$. How would the person choose to split consumption between
  the two years to maximize average happiness across the years?

  \begin{enumerate}[(A)]
  \item All the consumption in either one year
  \item Equal amount of consumption in the two years
  \item Consume twice as much in one year as in the other year
  \item Consumption in the two years is in the ratio $a:b$
  \item It does not matter because any choice of split of consumption
    level between the two years produces the same average happiness
  \end{enumerate}

  \vspace{0.1in}
  Your answer: $\underline{\qquad\qquad\qquad\qquad\qquad\qquad\qquad}$
  \vspace{0.6in}

\item (*) {\em Income inequality and subjective well being}:
  Subjective well being {\em across} individuals is found to be
  logarithmically related to income. Every doubling of income is found
  to increase an individuals' measured subjective well being by $0.3$
  points on a certain scale. {\em Holding total income across two
  individuals constant}, how should that income be divided between the
  two individuals to maximize their average subjective well being?

  \begin{enumerate}[(A)]
  \item All the income goes to one person
  \item Both earn the exact same income
  \item One person earns twice as much as the other
  \item One person earns $0.3$ times as much as the other
  \item It does not matter because the average subjective well being
  is independent of the distribution of income.
  \end{enumerate}

  \vspace{0.1in}
  Your answer: $\underline{\qquad\qquad\qquad\qquad\qquad\qquad\qquad}$
  \vspace{0.6in}

\end{enumerate}
\end{document}