\documentclass[10pt]{amsart}

%Packages in use
\usepackage{fullpage, hyperref, vipul, amssymb}

%Title details
\title{Lazard correspondence: key proof}
\author{Vipul Naik}

%List of new commands
\newcommand{\Skew}{\operatorname{Skew}}
\newcommand{\ad}{\operatorname{ad}}
\begin{document}
\maketitle

Suppose $G$ is a group of (global) nilpotency class $c$ such that $G$
is powered over all primes less than or equal to $c$. Then, $G$ has a
Lazard Lie ring. Let's call the Lazard Lie ring $L$. Further, we will
denote by $e^x$ the element of $G$ corresponding to an element $x \in
L$, and by $\log g$ the element of $L$ corresponding to an element $g
\in G$ under this correspondence. To avoid the proliferation of
unnecessary letters, we will denote the Lazard Lie ring of $G$ as
$\log G$.

Our ultimate goal is to show that the Lazard correspondence between
$\log G$ and $G$ induces a canonical Lazard correspondence between the
exterior square $\log G \wedge \log G$ and the exterior square $G
\wedge G$, in such a manner that the following diagram commutes:

\begin{eqnarray*}
  \log G \wedge \log G & \rightarrow & \log G\\
  \downarrow & & \downarrow\\
  G \wedge G & \rightarrow & G \\
\end{eqnarray*}

where the map $L \wedge L \to L$ is the Lie bracket homomorphism, the
map $G \wedge G \to G$ is the commutator map homomorphism, and the
downward maps are the respective Lazard correspondences.

\subsection{Preliminary lemmas}

We begin with some lemmas.

\begin{lemma}
  Let $\pi$ be a set of primes and $G$ be a nilpotent group. Then, $G$
  is $\pi$-divisible if and only if its abelianization is $\pi$-divisible.
\end{lemma}

\begin{proof}
  The full proof will be included in the main document, but here's a
  quick and complete sketch.
\end{proof}

We can refine this result a bit:

\begin{lemma}
  Suppose $\pi$ is a set of primes and $K$ is a nilpotent group. If
  $H$ is a normal subgroup of $K$ that is contained inside $K'$, then
  $K$ is $\pi$-divisible if and only if $K/H$ is $\pi$-divisible.
\end{lemma}

\begin{proof}
  Both $K$ and $K/H$ have the same abelianization, so this follows from
  the preceding lemma.

  The conclusion that $H$ is also $\pi$-powered follows directly from
  both $K$ and $K/H$ being $\pi$-powered.
\end{proof}

\begin{lemma}
  Suppose $G$ is a nilpotent group and $K$ is a stem extension of $G$
  with quotient map $\varphi:K \to G$ having kernel a central subgroup
  $H$ of $K$. Then, $K$ is also nilpotent of class either equal to or
  one more than the nilpotency class of $G$. Further, if $\pi$ is a
  set of primes, $G$ is $\pi$-powered if and only if $K$ is
  $\pi$-powered.
\end{lemma}

\begin{proof}
  Recall that the definition of stem extension requires that the
  kernel $H$ of $\varphi$ should be contained in $K' \cap Z(K)$. Also,
  if $G$ is nilpotent, so is $K$ (with class either equal to or more
  than the class of $G$). Thus, the preceding lemma applies, and we get
  the result.
\end{proof}

We thus get the result:

\begin{lemma}\label{schurcoverpipower}
  Suppose $G$ is a $\pi$-powered nilpotent group of class $c$. Suppose
  $K$ is a Schur covering group for $G$ with $\varphi:K \to G$ the
  quotient map. Let $M$ be the kernel of $\varphi$. The following are
  true.

  \begin{enumerate}
  \item $K$ is a $\pi$-powered nilpotent group of class either $c$ or
    $c + 1$.
  \item The Schur multiplier of $G$, which can be identified
    with the kernel $M$ of $\varphi$, is also $\pi$-powered.
  \item The non-abelian exterior square $G \wedge G$ of $G$, which is
    isomorphic to $[K,K]$, is $\pi$-powered.
  \end{enumerate}
\end{lemma}

\begin{proof}
  Parts (1) and (2) follows from the preceding lemma, and the observation that
  any Schur covering group is by definition a stem extension.

  Part (3) follows from Part (1) and Lemma 7 of Section 2.5 in the
  original document. Specifically, the group $G \wedge G \cong [K,K]$,
  is the derived subgroup of the $\pi$-powered nilpotent group.
\end{proof}

Note that although the Schur covering group need not be unique up to
isomorphism, the exterior square and the Schur multiplier are uniquely
determined.

\subsection{Similar results on the Lie ring side}

A set of results exactly analogous to those in the preceding section
can also be obtained for Lie rings.

\subsection{The rationally powered case}

For this section, we restrict attention to the case where $G$ is a
rationally powered nilpotent group. The advantage of this is that all
groups involved will be Malcev Lie groups and hence Lazard Lie
groups. The complications arising in other situations will be
discussed in the next section.

Suppose $G$ is a rationally powered nilpotent group and $L$ is its
Lazard Lie ring. $L$ is a nilpotent $\Q$-Lie algebra. Notationally, $L
= \log G$ and $G = \exp L$.

Suppose $K$ is a Schur covering group of $G$ with covering map
$\varphi:K \to G$. By lemma \ref{schurcoverpipower}, $K$ is also a
rationally powered nilpotent group. Hence, it has its own Lazard Lie
ring $\log K$, which is a $\Q$-Lie algebra, and we get a $\Q$-Lie
algebra homomorphism $\log \varphi: \log K \to \log G = L$.

On the other hand, suppose $N$ is a Schur covering Lie ring of $L$
with covering map $\psi:N \to L$. By the Lie ring analogue of lemma
\ref{schurcoverpipower}, $N$ is also a nilpotent $\Q$-Lie algebra, so
it has its own Lazard Lie group $\exp(N)$, and we get a mapping $\exp
\psi: \exp(N) \to \exp(L) = G$.

What we would have ideally hoped for is that $N = \log K$. Alas, this
is not guaranteed, even in the rationally powered case, because there
could be multiple Schur covering groups and multiple Schur covering
Lie rings, and we may not have picked the correspondent
ones. Nonetheless, we would still like to show that the $[N,N]$ is
isomorphic to $\log[K,K] = [\log K,\log K]$, or equivalently, that $[K,K]$ is isomorphic to $\exp[N,N] = [\exp N,\exp N]$.

Let's first state what we already have:

\begin{itemize}
\item $[N,N]$ is isomorphic to $L \wedge L$ under a canonical
  isomorphism of Lie rings where the Lie bracket homomorphism $L
  \wedge L \to L$ corresponds to $\psi|_{[N,N]}: [N,N] \to L$.
\item $[K,K]$ is isomorphic to $G \wedge G$ under a canonical
  isomorphism of groups where the group homomorphism $G \wedge G \to
  G$ corresponds to $\varphi_{[K,K]}: [K,K] \to G$.
\end{itemize}

\begin{lemma}
  We have a canonical isomorphism between $L \wedge L \cong [N,N]$ and
  $[\log K,\log K] = \log [K,K]$ as Lie rings. Equivalently, we have a
  canonical isomorphism between $\exp [N,N] = [\exp N, \exp N]$ and
  $[K,K]$. These canonical isomorphisms give us that $L \wedge L$ and
  $G \wedge G$ are Lazard correspondent, with the following
  commutative diagram:

  \begin{eqnarray*}
    \log G \wedge \log G & \rightarrow & \log G\\
    \downarrow & & \downarrow\\
    G \wedge G & \rightarrow & G \\
  \end{eqnarray*}

\end{lemma}

\begin{proof}
  The proof idea is very similar to the idea generally used for
  universal properties.

  First, note that $\log K$ is a central extension for quotient $L =
  \log G$ via the map $\log \varphi$. Thus, we get a Lie ring
  homomorphism corresponding to the Lie bracket:

  $$L \wedge L \to [\log K,\log K] = \log[K,K] = \log(G \wedge G)$$

  Second, note that $\exp N$ is a central extension for quotient $G =
  \exp L$ via the map $\exp \psi$. Thus, we get a group homomorphism
  corresponding to the group commutator map:

  $$G \wedge G \to [\exp N,\exp N] = \exp[N,N] = \exp(L \wedge L)$$

  Taking logs on everything, we get:

  $$\log(G \wedge G) \to [N,N] = L \wedge L$$

  We would like to assert that these maps compose both ways to the
  identity, hence are inverses. That would be enough. {\em TODO: Do
  diagram chasing to show that the composite is the identity}.
\end{proof}

\subsection{Extending from the rationally powered to the general case}

We have established the result in the rationally powered group
case. We now wish to extend it. The main problem arises in situations
where $G$ has class $p - 1$ for some prime $p$, and is not $p$-powered
but is powered over all smaller primes. In these situations, we cannot
directly mimic the above argument of using the Lazard correspondence
for the Schur covering group and Lie ring, because the Schur covering
group or Lie ring have class $p$, and we cannot do the Lazard
correspondence on the Schur covering group.

What we would like to argue is that even though we cannot do the
Lazard correspondence directly on the Schur covering groups, we can
get around this problem. The idea is simple.

\begin{lemma}
  Suppose $c$ is a natural number and $\pi$ is the set of all primes
  less than or equal to $c$. Suppose $L$ is a free $\pi$-powered class
  $c$ Lie ring (on any number of generators) and $G$ is its
  corresponding Lazard Lie group. Then:

  \begin{enumerate}
  \item $G$ is also a free $\pi$-powered class $c$ group with a
    freely generating set given by the elements corresponding to a
    freely generating set of $L$.
  \item We have that $\exp(\Q L)$ is canonically identified with
    $\sqrt{G}$, the minimal rationally powered nilpotent group
    containing $G$.
  \end{enumerate}
\end{lemma}

\begin{proof}
  The results and notation are all borrowed from Khukhro's treatment
  of the malcev and Lazard correspondences.
\end{proof}

We can now turn to the key idea of the proof. Here's an omnibus
lemma.

\begin{lemma}
  \begin{enumerate}
  \item $\mathbb(Q)(L \wedge L)$ is naturally identified with
    $\mathbb{Q}L \wedge \mathbb{Q}L$.
  \item $\sqrt{G \wedge G}$ is naturally identified with $\sqrt{G}
    \wedge \sqrt{G}$.
  \item $\mathbb{Q}(L \wedge L) $ and $\sqrt{G \wedge G}$ are Lazard
    correspondent.
  \item The Lazard correspondence of the previous step maps $L \wedge
    L$ to $G \wedge G$. Hence, $L \wedge L$ and $G \wedge G$ are
    Lazard correspondent.
  \end{enumerate}
\end{lemma}

\begin{proof}
  Parts (1) and (2) are direct from the definitions. Part (3) follows
  from Parts (1) and (2), plus the result of the preceding section. 

  Part (4) is the tricky result. What this is saying, essentially, is
  that if we start off with terms that do not have any ``bad''
  denominators, then the things that we arrive at after doing the
  correspondence will also not have any bad denominators. The main
  reason this works is because the correspondence uses the formula for
  the group commutator in terms of the Lie bracket for class $c + 1$
  (see section 3.4 of the old document), as well as the
  Baker-Campbell-Hausdorff formula for class $c$. Neither of these is
  problematic. {\em TODO: Add more detail here}.
\end{proof}

The above establishes the result in the free $\pi$-powered case. For
the non-free case, we use the following.

\begin{lemma}
  \begin{enumerate}
  \item If $L$ is a Lie ring and $I$ is an ideal in $L$, then $L/I
    \wedge L/I$ can be identified with $(L \wedge L)/(I \wedge L)$.
  \item If $G$ is a group and $H$ is a normal subgroup in $G$, then
    $G/H \wedge G/H$ can be identified with $(G \wedge G)/(H \wedge G)$.
  \item Suppose $L$ is a free $\pi$-powered class $c$ nilpotent Lie
    ring and $G$ is its Lazard Lie group. Suppose $I$ is an ideal in
    $L$ and $H = \exp(I)$ is the corresponding normal subgroup of
    $G$. Then, under the Lazard correspondence used above between $L
    \wedge L$ and $G \wedge G$, $I \wedge L$ is correspondent with $H
    \wedge G$.
  \item Continuing with the setup of Part (3), we get that $L/I \wedge
    L/I$ is Lazard correspondent with $G/H \wedge G/H$.
  \end{enumerate}
\end{lemma}

This lemma essentially does the trick, because any $\pi$-powered class
$c$ group can be expressed as a quotient of a free $\pi$-powered class
$c$ group, and we can thus use the above lemma to get the Lazard
correspondence.

\end{document}
