\documentclass[10pt]{amsart}

%Packages in use
\usepackage{fullpage, hyperref, vipul, enumerate}

%Title details
\title{Class quiz solutions: February 17: Limit, order of zero, LH rule}
\author{Math 153, Section 55 (Vipul Naik)}
%List of new commands

\begin{document}
\maketitle

\section{Performance review}

The score distribution was as follows:

\begin{itemize}
\item Score of $0$: $1$ person
\item Score of $1$: $1$ person
\item Score of $2$: $1$ person
\item Score of $3$: $1$ person
\item Score of $4$: $2$ people
\item Score of $5$: $3$ people
\item Score of $6$: $2$ people
\end{itemize}

The mean score was $3.73$. The question wise answers and performance
review are below:

\begin{enumerate}
\item Option (C): $6$ people
\item Option (C): $5$ people
\item Option (B): $6$ people
\item Option (D): $8$ people
\item Option (E): $9$ people
\item Option (A): $7$ people
\end{enumerate}

\section{Solutions}

\begin{enumerate}

\item Suppose $g:\R \to \R$ is a continuous function such that
  $\lim_{x \to 0} g(x)/x = A$ for some constant $A \ne 0$. What is
  $\lim_{x \to 0} g(g(x))/x$? 

  \begin{enumerate}[(A)]
  \item $0$
  \item $A$
  \item $A^2$
  \item $g(A)$
  \item $g(A)/A$
  \end{enumerate}

  {\em Answer}: Option (C)

  {\em Explanation}: We have $\lim_{x \to 0} g(x) = \lim_{x \to 0}
  (g(x)/x) \lim_{x \to 0} x = A \cdot 0 = 0$.

  Also, we have:

  $$\lim_{x \to 0} \frac{g(g(x))}{x} = \lim_{x \to 0} \frac{g(g(x))}{g(x)} \lim_{x \to 0} \frac{g(x)}{x}$$

  The second limit is $A$. For the first limit, note that as $x \to
  0$, we also have $g(x) \to 0$, so the first limit can be rewritten
  as $\lim_{y \to 0} g(y)/y$, which is also equal to $A$. Hence, the
  overall limit is the product $A^2$.

  {\em Performance review}: $6$ out of $11$ got this. $3$ chose (B)
  and $2$ chose (D).

  {\em Historical note (last quarter)}: $4$ out of $12$ go this
  correct. $3$ each chose (A) and (E), $2$ chose (D).

  {\em Historical note (last year)}: $1$ out of $12$ people got this
  correct. $5$ people chose (D), $2$ people each chose (B) and (E),
  $1$ person chose (A), and $1$ person left the question blank.


\item Suppose $g:\R \to \R$ is a continuous function such that
  $\lim_{x \to 0} g(x)/x^2 = A$ for some constant $A \ne 0$. What is
  $\lim_{x \to 0} g(g(x))/x^4$? 

  \begin{enumerate}[(A)]
  \item $A$
  \item $A^2$
  \item $A^3$
  \item $A^2g(A)$
  \item $g(A)/A^2$
  \end{enumerate}

  {\em Answer}: Option (C)

  {\em Explanation}: First, note that since $g(x)/x^2 \to A$ as $x \to
  0$, we must have $g(x) \to 0$ as $x \to 0$. In particular, $g(0) =
  0$.

  Now, consider:

  $$\lim_{x \to 0} \frac{g(g(x))}{x^4} = \lim_{x \to 0} \frac{g(g(x))}{(g(x))^2} \cdot \frac{(g(x))^2}{x^4}$$

  Splitting the limit, we get:

  $$\lim_{x \to 0} \frac{g(g(x))}{(g(x))^2} \lim_{x \to 0} \left(\frac{g(x)}{x^2}\right)^2$$

  Setting $u = g(x)$ for the first limit, and using the fact that as
  $x \to 0$, $u \to 0$ we see that the first limit is $A$. For the
  second limit, pulling the square out yields that the second limit is
  $A^2$. The overall limit is thus $A \cdot A^2 = A^3$.

  We can also use an actual example to solve this problem. For
  instance, consider the extreme case where $g(x) = Ax^2$ (identically). In
  this case, $g(g(x)) = A(Ax^2)^2 = A^3x^4$. Thus, $g(g(x))/x^4 =
  A^3$, and the limit is thus $A^3$.

  Even more generally, if $\lim_{x \to 0} g(x)/x^n = A$, then $\lim_{x
  \to 0} g(g(x))/x^{n^2} = A^{n + 1}$.

  {\em Performance review}: $5$ out of $11$ got this. $3$ chose (D),
  $2$ chose (E) $1$ chose (B).

  {\em Historical note (last quarter)}: $3$ out of $12$ got this correct. $7$
  chose (D), $1$ chose (B), $1$ chose (E).

  {\em Historical note (last year)}: $4$ out of $16$ people got this
  correct. $5$ people chose (A), $4$ people chose (B), $2$ people
  chose (D), and $1$ person chose (E).

  For the remaining questions, keep in mind that the {\em order of a
  zero} for a function $f$ at a point $c$ in its domain (where it's
  continuous) such that $f(c) = 0$ is defined as the lub of the set $\{
  \beta \ge 0 \mid \lim_{x \to c} |f(x)|/|x - c|^\beta = 0 \}$.

  If $f$ is an infinitely differentiable function at $c$, then the
  order, if finite, must be a positive integer. If the order is a
  positive integer $r$, then the first $r - 1$ derivatives of $f$ at
  $c$ equal zero and the $r^{th}$ derivative at $c$ is nonzero
  (assuming $f$ to be infinitely differentiable).

  For convenience, we take $c = 0$ in the next three questions, i.e.,
  all limits are being taken as $x \to 0$.

\item If $f$ has a zero of order $2$ and $g$ has a zero of order $3$
  at the point $0$, what is the order of zero for the pointwise sum $f
  + g$ at zero?  (Example: $f(x) = \sin^2x$ and $g(x) = \ln(1 +
  x^3)$).

  \begin{enumerate}[(A)]
  \item $1$
  \item $2$
  \item $3$
  \item $5$
  \item $6$
  \end{enumerate}

  {\em Answer}: Option (B)

  {\em Explanation}: The general rule is that when $f$ and $g$ both
  have zeros of {\em different} orders at a point, the order of zero
  of their sum is the minimum of the orders of zeros for the individual
  functions. We can interpret this result in terms of the limit
  definitions, or in terms of what's the first iterated derivative to
  take a nonzero value.

  {\em Performance review}: $6$ out of $11$ got this. $4$ chose (C),
  $1$ chose (A).
\item If $f$ has a zero of order $2$ and $g$ has a zero of order $3$
  at the point $0$, what is the order of zero for the pointwise
  product $fg$ at zero?  (Example: $f(x) = \sin^2x$ and $g(x) = \ln(1
  + x^3)$).

  \begin{enumerate}[(A)]
  \item $1$
  \item $2$
  \item $3$
  \item $5$
  \item $6$
  \end{enumerate}

  {\em Answer}: Option (D)

  {\em Explanation}: The order of zero for a product of two function
  is the sum of the orders of zeros for the two functions. This can be
  seen by thinking of the limit definition of order of zero.

  {\em Performance review}: $8$ out of $11$ got this. $2$ chose (E),
  $1$ chose (C).
\item If $f$ has a zero of order $2$ and $g$ has a zero of order $3$
  at the point $0$, what is the order of zero for the composite
  function $f \circ g$ at zero?  (Example: $f(x) = \sin^2x$ and $g(x) =
  \ln(1 + x^3)$).

  \begin{enumerate}[(A)]
  \item $1$
  \item $2$
  \item $3$
  \item $5$
  \item $6$
  \end{enumerate}

  {\em Answer}: Option (E)

  {\em Explanation}: Roughly speaking, the order of zero of the
  composite function is the product of the orders of zeros. This is
  valid when $c = 0$, i.e., we are taking the order of zero at
  zero. Otherwise, the statement needs to be modified somewhat.

  {\em Performance review}: $9$ out of $11$ got this. $1$ each chose
  (C) and (D).
\item The L'Hopital rule can be related with order of zero in the
  following manner: Every time the rule is applied to a $(\to 0)/(\to
  0)$ form, the order of zero of the numerator and denominator go {\em
    down by one}. Repeated application hopefully yields a situation
  where either the numerator or the denominator has a nonzero value.

  Assume that we start with a limit $\lim_{x \to c} f(x)/g(x)$ where
  both $f$ and $g$ are infinitely differentiable at $c$, and further,
  that $f(c) = g(c) = 0$. If the order of zero of $f$ is $d_f$ and the
  order of zero of $g$ is $d_g$, which of the
  following is true?

  \begin{enumerate}[(A)]
  \item If $d_f = d_g$, then we need to apply the LH rule $d_f$ times
    and we will then get a nonzero numerator and nonzero denominator,
    that we can evaluate to get the limit. If $d_f < d_g$, then we
    apply the LH rule $d_f$ times to get a nonzero numerator and zero
    denominator, so the limit is undefined. If $d_g < d_f$, then we
    apply the LH rule $d_g$ times to get a zero numerator and nonzero
    denominator, so the limit is zero.
  \item If $d_f = d_g$, then we need to apply the LH rule $d_f$ times
    and we will then get a nonzero numerator and nonzero denominator,
    that we can evaluate to get the limit. If $d_f < d_g$, then we
    apply the LH rule $d_f$ times to get a zero numerator and nonzero
    denominator, so the limit is undefined. If $d_g < d_f$, then we
    apply the LH rule $d_g$ times to get a nonzero numerator and zero
    denominator, so the limit is zero.
  \item If $d_f = d_g$, then we need to apply the LH rule $d_f$ times
    and we will then get a nonzero numerator and nonzero denominator,
    that we can evaluate to get the limit. If $d_f < d_g$, then we
    apply the LH rule $d_f$ times to get a nonzero numerator and zero
    denominator, so the limit is zero. If $d_g < d_f$, then we
    apply the LH rule $d_g$ times to get a zero numerator and nonzero
    denominator, so the limit is undefined.
  \item If $d_f = d_g$, then we need to apply the LH rule $d_f$ times
    and we will then get a nonzero numerator and nonzero denominator,
    that we can evaluate to get the limit. If $d_f < d_g$, then we
    apply the LH rule $d_f$ times to get a zero numerator and nonzero
    denominator, so the limit is zero. If $d_g < d_f$, then we
    apply the LH rule $d_g$ times to get a nonzero numerator and zero
    denominator, so the limit is undefined.
  \item In all cases, we perform the LH rule $\min \{ d_f, d_g \}$
    times and obtain a nonzero numerator and nonzero denominator.
  \end{enumerate}

  {\em Answer}: Option (A)

  {\em Explanation}: Each time the LH rule is applied, the order of
  zero on the numerator goes down by one and the order of zero on the
  denominator goes down by one. Thus, we need to perform the LH rule
  $\min \{ d_f, d_g \}$ times to reach a situation where either the
  numerator or the denominator gets a zero order of zero, which means
  (from the information we have) that it evaluates to something
  nonzero.

  If $d_f = d_g$, then applying the LH rule $d_f$ times yields a
  situation where both the numerator and denominator become nonzero.

  If $d_f < d_g$ then we need to apply the LH rule $d_f$ times. The
  denominator in this case still has a zero of order $d_g - d_f$,
  hence evaluates to zero. The numerator has a zero of order zero,
  i.e., it evaluates to something nonzero. The (nonzero)/(zero) form means that the limit is undefined.

  If $d_g < d_f$, then we need to apply the LH rule $d_g$ times. The
  numerator in this case still has a zero of order $d_f - d_g$, so is
  zero, whereas the denominator is nonzero. The (zero)/(nonzero) form
  means that the limit is zero.

  {\em Performance review}: $7$ out of $11$ got this. $3$ chose (D),
  $1$ chose (B).
\end{enumerate}
\end{document}