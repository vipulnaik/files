\documentclass[10pt]{amsart}

%Packages in use
\usepackage{fullpage, hyperref, vipul, enumerate}

%Title details
\title{Class quiz: October 19: Increase/decrease and maxima/minima}
\author{Math 152, Section 55 (Vipul Naik)}
%List of new commands

\begin{document}
\maketitle

Your name (print clearly in capital letters): $\underline{\qquad\qquad\qquad\qquad\qquad\qquad\qquad\qquad\qquad\qquad}$

You are expected to take about one minute per question.

\begin{enumerate}

\item (*) Suppose $f$ is a function defined on a closed interval
  $[a,c]$. Suppose that the left-hand derivative of $f$ at $c$ exists
  and equals $\ell$. Which of the following implications is {\bf true
  in general}? {\em Last year: $8/15$ correct}

  \begin{enumerate}[(A)]
  \item If $f(x) < f(c)$ for all $a \le x < c$, then $\ell < 0$.
  \item If $f(x) \le f(c)$ for all $a \le x < c$, then $\ell \le 0$.
  \item If $f(x) < f(c)$ for all $a \le x < c$, then $\ell > 0$.
  \item If $f(x) \le f(c)$ for all $a \le x < c$, then $\ell \ge 0$.
  \item None of the above is true in general.
  \end{enumerate}

  \vspace{0.1in}
  Your answer: $\underline{\qquad\qquad\qquad\qquad\qquad\qquad\qquad}$
  \vspace{1.5in}

\item (**) Suppose $f$ and $g$ are increasing functions from $\R$ to
  $\R$. Which of the following functions is {\em not} guaranteed to be
  an increasing function from $\R$ to $\R$? {\em Last year: $1/15$
  correct}

  \begin{enumerate}[(A)]

  \item $f + g$
  \item $f \cdot g$
  \item $f \circ g$
  \item All of the above, i.e., none of them is guaranteed to be increasing.
  \item None of the above, i.e., they are all guaranteed to be increasing.
  \end{enumerate}

  \vspace{0.1in}
  Your answer: $\underline{\qquad\qquad\qquad\qquad\qquad\qquad\qquad}$
  \vspace{1.5in}


  {\bf PLEASE TURN OVER FOR THE THIRD AND FOURTH QUESTION.}

\newpage
\item (**) Suppose $f$ is a continuous function defined on an open interval
  $(a,b)$ and $c$ is a point in $(a,b)$. Which of the following
  implications is {\bf true}? {\em Last year: $5/15$ correct}

  \begin{enumerate}[(A)]

  \item If $c$ is a point of local minimum for $f$, then there is a
    value $\delta > 0$ and an open interval $(c - \delta, c + \delta)
    \subseteq (a,b)$ such that $f$ is non-increasing on $(c -
    \delta,c)$ and non-decreasing on $(c,c+\delta)$.
  \item If there is a value $\delta > 0$ and an open interval $(c -
    \delta, c + \delta) \subseteq (a,b)$ such that $f$ is
    non-increasing on $(c - \delta,c)$ and non-decreasing on
    $(c,c+\delta)$, then $c$ is a point of local minimum for $f$.
  \item If $c$ is a point of local minimum for $f$, then there is a
    value $\delta > 0$ and an open interval $(c - \delta, c + \delta)
    \subseteq (a,b)$ such that $f$ is non-decreasing on $(c -
    \delta,c)$ and non-increasing on $(c,c+\delta)$.
  \item If there is a value $\delta > 0$ and an open interval $(c -
    \delta, c + \delta) \subseteq (a,b)$ such that $f$ is
    non-decreasing on $(c - \delta,c)$ and non-increasing on
    $(c,c+\delta)$, then $c$ is a point of local minimum for $f$.
  \item All of the above are true.
  \end{enumerate}

  \vspace{0.1in}
  Your answer: $\underline{\qquad\qquad\qquad\qquad\qquad\qquad\qquad}$
  \vspace{1.5in}

\item (**) Suppose $f$ is a continuously differentiable function on $\R$
  and $f'$ is a periodic function with period $h$. (Recall that
  periodic derivative implies that the original function is a sum of
  ...). Suppose $S$ is the set of points of local maximum for $f$, and
  $T$ is the set of local maximum values. Which of the following is
  {\bf true in general} about the sets $S$ and $T$? {\em Last year:
  $3/15$ correct}

  \begin{enumerate}[(A)]
  \item The set $S$ is invariant under translation by $h$ (i.e., $x
    \in S$ if and only if $x + h \in S$) and all the values in the set
    $T$ are in the image of the set $[0,h]$ under $f$.
  \item The set $S$ is invariant under translation by $h$ (i.e., $x
    \in S$ if and only if $x + h \in S$) but all the values in the set
    $T$ need not be in the image of the set $[0,h]$ under $f$.
  \item Both the sets $S$ and $T$ are invariant under translation by $h$.
  \item Both the sets $S$ and $T$ are finite.
  \item Both the sets $S$ and $T$ are infinite.
  \end{enumerate}

  \vspace{0.1in}
  Your answer: $\underline{\qquad\qquad\qquad\qquad\qquad\qquad\qquad}$
  \vspace{1.5in}
\end{enumerate}

\end{document}