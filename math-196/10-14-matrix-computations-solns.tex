\documentclass[10pt]{amsart}

%Packages in use
\usepackage{fullpage, hyperref, vipul, enumerate}

%Title details
\title{Take-home class quiz solutions: due Monday October 14: Matrix computations}
\author{Math 196, Section 57 (Vipul Naik)}
%List of new commands

\begin{document}
\maketitle

{\bf PLEASE DO {\em NOT} DISCUSS ANY QUESTIONS EXCEPT THE STARRED OR DOUBLE-STARRED QUESTIONS.}

\section{Performance review}

28 people took this $5$-question quiz. The score distribution was as
follows:

\begin{itemize}
\item Score of $1$: $1$ person
\item Score of $2$: $3$ people
\item Score of $3$: $6$ people
\item Score of $4$: $11$ people
\item Score of $5$: $7$ people
\end{itemize}

The question-wise answers and performance review are below:

\begin{enumerate}
\item Option (C): 27 people
\item Option (B): 25 people
\item Option (B): 13 people
\item Option (D): 18 people
\item Option (B): 21 people
\end{enumerate}

\section{Solutions}

This quiz has a few questions on the mechanics of the computational
execution of Gauss-Jordan elimination, and it has one question on
setting up a linear system.

Suppose $f$ is a function on the positive integers that takes positive
integer values. Suppose $n$ is a parameter related to the input size
of an algorithm. We say that the running time of an algorithm
(respectively, the space requirement of the algorithm) is:

\begin{itemize}
\item $O(f(n))$ if, for large enough $n$, it can be bounded from above
  by a positive constant times $f(n)$.
\item $\Omega(f(n))$ if, for large enough $n$, it can be bounded from
  below by a positive constant times $f(n)$.
\item $\Theta(f(n))$ if it is both $O(f(n))$ and $\Omega(f(n))$.
\end{itemize}

You can read more at:

\url{http://en.wikipedia.org/wiki/Big_O_notation}

\begin{enumerate}
\item (*) If you treat each arithmetic operation (addition,
  subtraction, multiplication, division) of numbers as taking constant
  time, and all entry rewrites and changes as again taking constant
  time per entry, what would be the best description of the worst-case
  running time of the algorithm to convert a $n \times n$ matrix to
  reduced row-echelon form? (Note that this complexity is termed {\em
    arithmetic complexity} and can be distinguished from the {\em bit
    complexity} of the algorithm, which could be considerably
  higher).

  \begin{enumerate}[(A)]
  \item $\Theta(n)$
  \item $\Theta(n^2)$
  \item $\Theta(n^3)$
  \item $\Theta(n^4)$
  \item $\Theta(n^5)$
  \end{enumerate}

  {\em Answer}: Option (C)

  {\em Explanation}: We have $\Theta(n^2)$ row operations, and the row
  operations all take $\Theta(n)$ time. Overall, we get $\Theta(n^3)$
  as the arithmetic complexity.

  More can be found in the lecture notes on Gauss-Jordan
  elimination. You can also learn more about the arithmetic complexity
  of Gaussian elimination by looking it up online.

  {\em Performance review}: 27 out of 28 got this. 1 chose (B).

  {\em Historical note (last time)}: $24$ out of $27$ got tihs. $1$ each chose
  (A), (B), and (E).

\item (*) If you treat each arithmetic operation (addition,
  subtraction, multiplication, division) of numbers as taking constant
  space, and all matrix entries as taking constant space, what would
  be the best description of the worst-case space requirement of the
  algorithm to convert a $n \times n$ matrix to reduced row-echelon
  form? Assume that space is reusable, i.e., it is possible to rewrite
  over existing space used.

  \begin{enumerate}[(A)]
  \item $\Theta(n)$
  \item $\Theta(n^2)$
  \item $\Theta(n^3)$
  \item $\Theta(n^4)$
  \item $\Theta(n^5)$
  \end{enumerate}

  {\em Answer}: Option (B)

  {\em Explanation}: We can reuse the matrix space to keep rewriting
  over existing entries. We need some additional workspace for working
  memory, but this is quite small relative to the size of the matrix,
  and does not affect the order estimate.

  More in the lecture notes on Gauss-Jordan elimination.

  {\em Performance review}: 25 out of 28 got this. 2 chose (D), 1
  chose (C).

  {\em Historical note (last time)}: $26$ out of $27$ got this. $1$ chose (A).

\item (*) Suppose the coefficient matrix of a linear system with $n$
  variables and $n$ equations is known in advance, and we can spend as
  much time processing it as we desire in advance (this time will not
  count towards the running time of the algorithm). In other words, we
  can use Gauss-Jordan elimination to row-reduce the coefficient
  matrix in advance. However, we do not have the output column with us
  in advance. What is the worst-case running time of the part of the
  algorithm that runs after the output column is known?

  \begin{enumerate}[(A)]
  \item $\Theta(n)$
  \item $\Theta(n^2)$
  \item $\Theta(n^3)$
  \item $\Theta(n^4)$
  \item $\Theta(n^5)$
  \end{enumerate}

  {\em Answer}: Option (B)

  {\em Explanation}: If we store the sequence of row operations used
  to convert the coefficient matrix to reduced row-echelon form (there
  are $\Theta(n^2)$ such operations) we simply need to apply these
  operations to the output column, then read off the solutions. The
  arithmetic time complexity of this is $\Theta(n^2)$.

  {\em Performance review}: 13 out of 28 got this. 11 chose (A), 3
  chose (C), 1 chose (D).

  {\em Historical note (last time)}: $17$ out of $27$ got this. $5$ chose (A),
  $4$ chose (C), $1$ chose (E).

\item Which of the following matrices does {\em not} have the identity
  matrix as its reduced row-echelon form?

  \begin{enumerate}[(A)]
  \item $$\left[\begin{matrix} 2 & 0 & 0 \\ 0 & 5 & 0 \\ 0 & 0 & -1 \\\end{matrix}\right]$$
  \item $$\left[\begin{matrix} 1 & 2 & 3 \\ 0 & 3 & 5 \\ 0 & 0 & 7 \\\end{matrix} \right]$$
  \item $$\left[\begin{matrix} 4 & 0 & 0 \\ 3 & 1 & 0 \\ 0 & 5 & -6 \\\end{matrix}\right]$$
  \item $$\left[\begin{matrix} 1 & 2 & -3 \\ 4 & -3 & -1 \\ -2 & 1 & 1 \\\end{matrix}\right]$$
  \item $$\left[\begin{matrix} 1 & 2 & 3 \\ 2 & 4 & 7 \\ 3 & 7 & 11 \\\end{matrix}\right]$$
  \end{enumerate}

  {\em Answer}: Option (D)

  {\em Explanation}: We can work to convert to rref to verify this,
  but one easy way of seeing that this matrix does not have full rank
  is to note that each row sum is $0$, which indicates that
  $\left[\begin{matrix} 1 \\ 1 \\ 1 \\\end{matrix}\right]$ and
  $\left[\begin{matrix} 0 \\ 0 \\ 0 \\\end{matrix}\right]$ are both
  solutions to the system of linear equations with this as the
  coefficient matrix and outputs all zeros. In other words, the
  solution space is not zero-dimensional, and hence, the matrix does
  not have full rank.

  As for the other options:

  \begin{itemize}
  \item Option (A): This is a diagonal matrix with all entries
    nonzero. Thus, its rref is the identity matrix. It is easy to see
    this: just divide each row by its diagonal entry.
  \item Option (B): This is upper-triangular. We can first convert the
    diagonal entries to $1$ by dividing. Then, we can do row
    subtractions and clear out everything above the diagonal.
  \item Option (C): This is lower-triangular, and works for a reason
    similar to Option (B).
  \item Option (E): This actually needs to be worked out using row
    reduction.
  \end{itemize}

  {\em Performance review}: 18 out of 28 got this. 4 chose (E), 2 each
  chose (B) and (C), 2 wrote incorrect free-form responses.

  {\em Historical note (last time)}: $24$ out of $27$ got this. $1$ each chose
  (B), (C), and (E).

\item A number of different consumer price indices have been
  constructed. All of them use the market prices for an existing
  collection of commodities (though not all of them use every
  commodity in the collection) and take a different ``weighted''
  linear combination of those. For instance, one price index might be
  3 times (the price per ton of wheat on the Chicago wheat market) + 4
  times (the price of 1 gallon of unleaded gasoline at a particular
  gas station) + 17 times (the price of Burt's chapstick). Another
  price index might use 30 times (the price of Transcend's 32 GB flash
  drive) + 14 times (the price of 1 gallon of gasoline at a particular
  gas station).

  What is a good way of modeling these?

  \begin{enumerate}[(A)]
  \item The prices of the various goods are stored in a matrix, the
    different weightings used in various indices are stored in a
    vector, and the consumer price indices arise as the output vector
    of the matrix-vector product.
  \item The different weightings used in various indices are stored in
    a matrix, the prices of the various goods are stored in a vector,
    and the consumer price indices arise as the output vector of the
    matrix-vector product.
  \item The prices of the various goods are stored in a matrix, the
    consumer price indices are stored as a vector, and the weightings
    used in the indices arise as the output vector of the
    matrix-vector product.
  \item The different weightings used in various indices are stored in
    a matrix, the consumer price indices are stored in a vector, and
    the prices of the various goods arise as the output vector of the
    matrix-vector product.
  \item The consumer price indices are stored in a matrix, the prices
    of the various goods are stored in a vector, and the weightings
    used in the indices arise as the output vector of the
    matrix-vector product.
  \end{enumerate}

  {\em Answer}: Option (B)

  {\em Explanation}: Each row represents the weightings used for a
  particular index. The input column is the input vector of
  prices. The output column is the vector of the various index values.

  {\em Performance review}: 21 out of 28 got this. 7 chose (A).

  {\em Historical note (last time)}: $22$ out of $27$ got this. $4$ chose (A),
  $1$ chose (C).
\end{enumerate}
\end{document}
