\documentclass[10pt]{amsart}
\usepackage{fullpage,hyperref,vipul,graphicx}
\title{Functions of several variables: definition, examples}
\author{Math 195, Section 59 (Vipul Naik)}

\begin{document}
\maketitle

{\bf Corresponding material in the book}: Section 14.1.

{\bf What students should definitely get}: Definition of function of
two variables, concept of graph of such a function, level curves,
function of three variables, level surfaces, determining the domain of
a function, functions of $n$ variables.

\section*{Executive summary}

Words ...

\begin{enumerate}
\item A function of $n$ variables is a function on a subset of
  $\R^n$. We can think of it in three ways: as a function with $n$
  real inputs, as a function with input a point in (a subset of)
  $\R^n$, and as a function with $n$-dimensional vector inputs. We
  often write the inputs with numerical subscripts, so a function $f$
  of $n$ inputs is written as $f(x_1,x_2,\dots,x_n)$.
\item In the case $n = 2$, we often write the inputs as $x,y$ so we
  write $f(x,y)$. This may be concretely described as an expression in
  terms of $x$ and $y$.
\item The graph of a function $f(x,y)$ of the two variables $x$ and
  $y$ is the surface $z = f(x,y)$. The $xy$-plane plays the role of
  the independent variable plane and the $z$-axis is the dependent
  variable axis. Any such graph satisfies the ``vertical'' line test
  where vertical means parallel to the $z$-axis.
\item The level curves of a function $f(x,y)$ are curves satisfying
  $f(x,y) = z_0$ for some fixed $z_0$. These are curves in the
  $xy$-plane.
\item The level surfaces of a function $f(x,y,z)$ of three variables
  are the surfaces satisfying $f(x,y,z) = c$ for some fixed $c$.
\item Domain convention: If nothing else is specified, the domain of a
  function in $n$ variables given by an expression is defined as the
  largest subset of $\R^n$ on which that expression makes sense.
\item We can also define vector-valued functions of many variables,
  e.g., a function from a subset of $\R^m$ to a subset of $\R^n$.
\item We can do various pointwise combination operations on functions
  of many variables, similar to what we do for functions of one variable
  (both the scalar and vector cases).
\item To compose functions, we need that the number of outputs of the
  inner/right function equals the number of inputs of the outer/left
  function.
\end{enumerate}

Actions ...

\begin{enumerate}
\item To find the domain, we first apply the usual conditions on
  denominators, things under square roots, and inputs to logarithms
  and inverse trigonometric functions. For functions of two variables,
  each such condition usually gives a region of $\R^2$ bounded by a
  line or curve.
\item After getting a bunch of conditions that need to be satisfied,
  we try to find the common solution set for all of these. This
  involves intersecting the regions in $\R^2$ obtained previously.
\end{enumerate}

\section{Functions of two variables}

\subsection{Definition}

A {\em function of two variables} is a function whose domain is a
subset of the plane $\R^2$ and whose range is a subset of $\R$. If we
denote the domain set by $D$, then a function $f$ is a rule that assigns
to every point $(x,y) \in D$ a real number $f(x,y) \in \R$.

The typical way of describing a function in two variables is by means
of an expression that may include both variables. For instance $f(x,y)
:= \frac{xy}{\sqrt{1 + x^2 + y^2}}$ is a function of the two variables
$x$ and $y$. The usual remarks from functions of one variable apply:

\begin{itemize}
\item The key fact about a function is that {\em equal inputs give
  equal outputs}. This remains true for functions of two variables:
  the value of $f(x_0,y_0)$ depends only on $(x_0,y_0)$.
\item We typically define a function using an algebraic
  expression. When the expression is given without a specified domain,
  we take the domain to be the largest possible subset of $\R^2$ where
  the expression makes sense.
\item The same function can be given by multiple different-looking
  expressions (which turn out to be the same upon algebraic
  simplification). Also, the same expression can give different
  functions if the domains are taken to be different.
\end{itemize}

Instead of a function of two variables, we can also consider a
function of three variables, four variables, or more. For instance, a
function of three variables is defined as a function whose domain is a
subset of $\R^3$ and whose range is a subset of $\R$.

Some of the visualization techniques we have start breaking down for
more than two input variables, because we're constrained to live in a
three-dimensional world, and $2 + 1 = 3$. To overcome these
limitations, we need to resort to techniques like color, texture, and
pattern to visualize such functions.

\subsection{Graphs of functions of two variables}

The convention we'll follow here is that, by default, the dependent
variable (the output of the function) will be denoted $z$ and the
independent variables will be noted $x$ and $y$, so we have $z =
f(x,y)$.

The {\em graph} of such a function is thus the {\em surface} given by
the equation $z = f(x,y)$. Comparing with the way we usually think of
functions in one variable as $y = f(x)$, we have the following
analogies:

\begin{itemize}
\item The $xy$-plane plays the role that the $x$-axis played in the
  past, as the space of possible {\em inputs} (the actual domain may
  be a subset rather than the whole plane).
\item The $z$-axis plays the role that the $y$-axis played earlier, as
  the set of all possible {\em outputs}.
\item To read the function value at a point in the $xy$-plane, we note
  the corresponding value of the $z$-coordinate so that we hit a point
  in the surface.
\item Graphs of functions in this sense satisfy the {\em vertical line
  test} where ``vertical'' here means parallel to the $z$-axis.
\end{itemize}

\subsection{Two inputs, one output, much repetition, level curves}

In the midst of one-variable calculus, we introduced a notion of {\em
one-to-one function}. A one-to-one function is a function with the
property that no two different input values give the same output,
i.e., {\em equal outputs must have arisen from equal inputs}.

When we're dealing with functions where the number of input variables
is greater than the number of outputs, however, being one-to-one is
highly unlikely (in fact, impossible for continuous functions defined
on any big enough subset). Rather, for any given output, it is highly
likely that the number of inputs isn't just more than one, it is
itself a curve.

Let's think about this. To find all the input pairs $(x,y)$ which map
to a fixed value $z = z_0$, we need to solve:

$$f(x,y) = z_0$$

In other words, we need to find the $(x,y)$-coordinate of the
intersection of the surface $z = f(x,y)$ and the plane $z = z_0$.

This is an intersection of two surfaces, and is expected to be a
curve. Thus, the general collection of points which map to a given
$z$-value is a {\em curve} in the $xy$-plane (the actual intersection
of surfaces will be the same curve translated to the $z = z_0$ plane).

These curves are called the {\em level curves} of the function. We
could choose to depict all the level curves in the $xy$-plane. {\em
The picture of the level curves, along with the labels of the levels
by the $z$-value}, together describe the function completely.

Note that for $z$-values which are outside the range, the
corresponding level curve is empty. For $z$-values in the boundary of
the range, the level curve may not be a curve at all but may be a
bunch of points.

For instance, consider the function:

$$f(x,y) := x^2 + y^2$$

The graph of this is the surface $z = x^2 + y^2$. It lies in the $z
\ge 0$ region. For a fixed value $z_0 > 0$, the corresponding level
curve is a circle centered at the origin and with radius
$\sqrt{z_0}$. For $z_0 = 0$, the corresponding level ``curve'' is the
single point $(0,0)$, and for $z_0 < 0$, the corresponding level curve
is empty.

\subsection{Color, a new dimension}

Level curves suggest a new, fascinating approach to depict functions
with two inputs and one output using {\em pictures in the plane}. The
key idea is to use a dimension such as color or texture to code for
the function value. Specifically, we first map the output values to a
color spectrum. For instance, if the range of a function $f$ is $[0,1]$,
we may map $[0,1]$ linearly to the wavelengths for the visible color
spectrum, setting $0$ for the violet color and $1$ for the red color.

Next, for any point $(x,y)$, we ``color'' that point with the color
value associated with $f(x,y)$. In particular, this means that the
``single color curves'' are precisely the ``level curves'' for the
function. The more reddish the color, the larger the function value at
the point. The more violetish the color, the smaller the function
value at that point.

In other words, instead of choosing a physical or height axis for the
$z$-axis, we choose a {\em color axis}.

\subsection{Finding the domain of a function}

As indicated earlier, the largest possible domain of a function of two
variables is the set of all possible input pairs on which the function
makes sense. In particular, this means things like:

\begin{itemize}
\item Any denominator should be nonzero
\item Any thing under a square root should be nonnegative
\item Any thing under a square root sign {\em in the denominator}
  should be positive
\item Any thing inside a logarithm should be positive
\end{itemize}

and more similar stuff.

What each of these conditions do typically is remove subsets from the
plane, and what's left is the domain. For instance, consider the function:

$$f(x,y) := \frac{x + y}{y^2 - 4x^2}$$

Here, the denominator factors as $(y + 2x)(y - 2x) = 0$, giving the
lines $y = -2x$ and $y = 2x$. These are the two forbidden lines --
everything else is the domain. The domain is thus the {\em complement
of the union of these lines in the plane}.

Similarly, consider:

$$f(x,y) := \frac{x + \ln(x - y)}{\sqrt{x^2 + y}}$$

The domain here must satisfy both the conditions $x > y$ and $x^2 + y
> 0$, so it is the intersection of the subsets satisfying these two
conditions. The condition $x > y$ is the half-plane region to the
left/down of the $y = x$ line. The condition $x^2 + y > 0$ is the
region above the parabola $y = -x^2$. The intersection of these
regions gives a particular subset of the plane.

See more examples of this sort in the book.

\section{Functions of more than two variables}

\subsection{Functions of three variables: depiction}

Most of the ideas mentioned above have analogues for functions of
three variables. The key difference is that there is no fourth
dimension in our three-dimensionally constrained visualization
capacities to make the graph of the function. Nonetheless, we can
still study the behavior in a similar manner.

First, we can talk of {\em level surfaces} (in place of level
curves). For a function $f(x,y,z)$ of three variables, the level
surface corresponding to a function value $c$ is the set of all input
points which get mapped to $c$ under $f$. This typically looks like a
surface -- we have one equation in three variables, so the solution
space is expected to be two-dimensional. In degenerate cases, it may
be a curve or even a point or bunch of points.

Second, in order to depict such functions, we can again use the {\em
color axis} trick -- determine the function value by the choice of
color at the point. This allows us to {\em depict functions of three
variables in three-dimensional space}. Of course, we cannot see all of
the function at the same time, because we can only see the outer
regions of the domain (they block the sight to the inner regions). But
with sophisticated slicing and visualization tools, we can get to see
all parts of the function.

\subsection{Functions of $n$ variables}

A function of $n$ variables is a function from a subset of $\R^n$ to
$\R$. In other words, the input of this function is a point in $\R^n$
(possibly restricted to a subset) and the output is a real number.

There are three ways of thinking of a function of $n$ variables:

\begin{enumerate}
\item As a function of $n$ real variables $x_1, x_2, \dots, x_n$.
\item As a function of a single point variable $(x_1,x_2,\dots,x_n)$.
\item As a function of a single vector variable $\mathbf{x} = \langle
  x_1, x_2, \dots, x_n \rangle$.
\end{enumerate}

All these perspectives turn out to be useful.

\subsection{Multiple inputs -- what's the big deal?}

We earlier dealt with functions that take one real input and give more
than one real output -- we called these {\em vector-valued
functions}. We found that vector-valued functions are no big deal --
to handle a vector-valued function, we simply look at each of the
outputs separately.

The story is more complicated for multiple inputs. It is {\em not}
possible to look at each input separately, because the inputs interact
with each other in msyterious ways and it is hard to disentangle them.

Here's an analogy that may help you understand this. If you write a
press release, it is easy to send it to a hundred different news
wires, put it up on your website, and print copies to put on loads of
bulletin boards. It may require more effort than simply putting it up
on your website, but the complexity of the task does not increase.

On the other hand, if, in order to write the press release, you need
to coordinate the activities of a hundred people, bring them together
in a meeting, and have them interchange ideas to come up with the
wording, that's a fundamentally more challenging task.

That's because multiple {\em inputs} are a lot harder than multiple
{\em outputs}.

\section{Operations on functions of many variables}

\subsection{Pointwise combination}

It is possible to do pointwise addition, subtraction, and
multiplication on real-valued functions of more than one variable. For
instance, if $f$ and $g$ are functions of two variables, then $f + g$
is also a function of two variables, defined as:

$$(f + g)(x,y) = f(x,y) + g(x,y)$$

Similarly, we can define the pointwise difference, pointwise product,
and pointwise quotient.

The usual comments on domain apply: the domain for any pointwise sum,
difference, or product is the intersection of the domains of the
functions. With pointwise quotient, we also need to exclude points
where the denominator function takes the value zero.

\subsection{Vector-valued functions: multiple inputs and multiple outputs}

Before moving to the next topic, i.e., composition, we consider the
concept of a {\em vector-valued function of multiple variables}. Here,
we simultaneously have multiple inputs (i.e., the domain is a subset
of $\R^m$) and multiple outputs (i.e., the range is a subset of
$\R^n$).

For vector-valued functions of multiple variables, we can do the same
types of pointwise combinations as for vector-valued functions of one
variable: pointwise addition, subtraction, scalar-vector
multiplication, dot product, and (when the output vector sare
three-dimensional) cross product.

\subsection{Composition}

If $f$ and $g$ are both real-valued functions of two variables, it
does not make direct sense to talk of $f \circ g$. The problem is that
the output of $g$ is a single real number, and this cannot be fed as
the input to $f$. Please remember: {\em with pointwise combination,
like combines with like. With composition, on the other hand, what
matters is that the output of one should feed into the input of the
other}.

Here is the kind of composition we can do: If $g$ is a vector-valued
function with $m$ inputs and $n$ outputs (i.e., the output is a
$n$-dimensional vector), and $f$ is a vector-valued function with $n$
inputs and $p$ outputs, then $f \circ g$ makes sense as a
vector-valued function with $m$ inputs and $p$ outputs. If this
reminds you of matrix multiplication, that is for good reason!
Unfortunately, the details are beyond the current scope.

\section{Real world examples}

\subsection{Quantity demanded: in the microeconomic realm}

We consider an example from microeconomics to illustrate the notion of
functions of many variables. The book has a number of other examples
that you can also review.

According to standard microeconomic theory, the ``quantity demanded''
for a good by a household is determined by six kinds of things:

\begin{itemize}
\item The unit price of the good
\item The unit prices of substitute goods
\item The unit prices of complementary goods
\item The income/wealth of the household
\item The tastes and preferences of the household
\item Expectations regarding future prices
\end{itemize}

Each of these things in turn may involve multiple sub-items. The unit
price of the good is a single real number, but the unit prices of
substitute and complementary goods could be as many real numbers as
the number of such substitute and complementary goods. For instance,
if we are looking at the quantity demanded for whole wheat bread, we
may identify substitute goods such as pita bread and baguettes, and
complementary goods such as butter and jam. Thus, we already have five
input variables: the price of whole wheat bread, the price of pita
bread, the price of baguettes, the price of butter, and the price of
jam.

The income of the household is a single variable, but there may be
other income-related variables of interest. The tastes and preferences
may be modeled using many measurements, such as a scale for the taste
for novelty, a scale for the preference for ``healthy'' food, a scale
for the taste of sweetness, etc. Expectations regarding future prices
may again be modeled using many real inputs.

Thus, we see that the quantity demanded for a good is a function of a
potentially large number of input variables, even in an ideal
microeconomic world of rational decision making. The kinds of
questions that we are interested in, and that will motivate further
mathematical analysis, include:

\begin{itemize}
\item When these ``determinants of demand'' are perturbed slightly,
  does the quantity demanded also respond by changing only slightly,
  or does it experience sudden, massive shifts? The ``changing only
  slightly'' is the key motivation behind the concept of continuity.
\item If we fix all the determinants of demand except one, how does
  the quantity demanded vary with changes in that one parameter? Does
  it increase or decrease? Can we differentiate this function? In
  economics, keeping everything else fixed is termed {\em ceteris
  paribus}. The related ideas in multivariable calculus are {\em
  separate continuity in the variables} and {\em partial derivatives}.
\end{itemize}

\subsection{Feels like temperature: of wind and water}

Page 892, Section 15.1, Example 2 in the book talks of the {\em wind
chill index} and the ``feels like'' temperature. Basically, when it is
cold and windy, it appears colder than it actually is, and weather
services have constructed tables that compute the ``feels like''
temperature as a function of the actual temperature and the wind
speed.

There is also a {\em temperature-humidity index} or humidex, which
measures the perceived air temperature as a function of actual
temperature and humidity. In general, the more the humidity, the
higher the perceived temperature. See Exercise 2, Section 15.1, Page
902 and also the discussion at the beginning of Section 15.3, Page 914
of the book.

These examples are not of direct relevance for us, but they may be fun
to read about (and you can learn more on the Internet) since they'll
give you a better perspective on interpreting weather forecasts.

\subsection{Functions on physical locations/geographic areas}

We consider another related concept where the book offers some
examples: functions whose inputs are physical locations, and which
output quantities that can be measured at those locations. For
instance, we can think of temperature as a function defined at every
point in physical space. It can thus be modeled as a function from a
subset of $\R^3$ to $\R$. If we are interested in measuring the
``surface temperature'' on the earth, then this can be modeled as a
function from the subset of $\R^3$ comprising the surface of the
earth, to $\R$.

The surface of the earth is an interesting domain upon which to define
a function. Although it lives in a three-dimensional world, the
surface itself is a two-dimensional construct. This means that it is
possible to parameterize the surface of the earth using a {\em
two-parameter formulation} (something we will explore in more detail
later). For those of you who remember high school geography, one way
of parameterizing the surface of the earth using two coordinates is
via the {\em latitude} and {\em longitude}. Once we choose this kind
of parameterization, then a function whose domain is the surface of
the earth can be viewed as a function of two variables (the latitude
and longitude coordinates) and hence can be viewed as a function on a
subset of $\R^2$ (with some caveats).

With this in mind, here is some terminology:

\begin{itemize}
\item If the function we are considering is temperature, then the
  level curves (i.e., the curves corresponding to fixed temperature
  values) are called {\em isotherms} or {\em isothermal curves}.
\item If the function we are considering is atmospheric pressure, then
  the level curves are called {\em isobars}.
\end{itemize}

\subsection{Cobb-Douglas production function}

This is an important idea but we will postpone discussion of this
until later, once we have seen partial derivatives.
\end{document}
