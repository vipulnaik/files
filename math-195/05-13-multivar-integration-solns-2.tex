\documentclass[10pt]{amsart}

%Packages in use
\usepackage{fullpage, hyperref, vipul, enumerate}

%Title details
\title{Class quiz solutions: May 13: Multi-variable integration (Qs 1-5)}
\author{Math 195, Section 59 (Vipul Naik)}
%List of new commands

\begin{document}
\maketitle

\section{Performance review}

$19$ people took this quiz. The score distribution was as follows:

\begin{itemize}
\item Score of $0$: $1$ person
\item Score of $1$: $3$ people
\item Score of $2$: $3$ people
\item Score of $3$: $6$ people
\item Score of $4$: $6$ people
\end{itemize}

The question wise answers and performance are as follows:

\begin{enumerate}
\item Option (C): $15$ people
\item Option (A): $13$ people
\item Option (C): $13$ people
\item Option (D): $7$ people
\item Option (A): $3$ people. {\em Please review this solution!}
\end{enumerate}

\section{Solutions}

Suppose $F$ is a function of two real variables, say $x$
and $t$, so $F(x,t)$ is a real number for $x$ and $t$ restricted to
suitable open intervals in the real number. Suppose, further, that $F$
is jointly continuous (whatever that means) in $x$ and $t$.

Define $f(t) := \int_0^\infty F(x,t) \, dx$. Here, while doing the
integration, $t$ is treated as a constant. $x$, the variable of
integration, is being integrated on $[0,\infty)$.
  
Suppose further that $f$ is defined and continuous for $t$ in
$(0,\infty)$.

In the next few questions, you are asked to compute the function $f$
explicitly given the function $F$, for $t \in (0,\infty)$.

\begin{enumerate}

\item $F(x,t) := e^{-tx}$. Find $f$.

  \begin{enumerate}[(A)]
  \item $f(t) = e^{-t}/t$
  \item $f(t) = e^t/t$
  \item $f(t) = 1/t$
  \item $f(t) = -1/t$
  \item $f(t) = -t$
  \end{enumerate}

  {\em Answer}: Option (C)

  {\em Explanation}: The integral becomes
  $[-e^{-tx}/t]_0^\infty$. Plugging in at $\infty$ gives $0$ and
  plugging in at $0$ gives $-1/t$. Since the value at $0$ is being
  subtracted, we eventually get $1/t$.

  Note that the answer must be positive for the simple reason that we
  are integrating a positive function from left to right across an
  interval.

  {\em Performance review}: $15$ out of $19$ people got this
  correct. $3$ people chose (D) and $1$ person chose (B)q.

  {\em Historical note}: This question appeared in a 153 quiz. At the
  time, $17$ out of $25$ people got this correct. $4$ chose (A), $3$
  chose (D), and $1$ chose (E).

\item $F(x,t) := 1/(t^2 + x^2)$. Find $f$.

  \begin{enumerate}[(A)]
  \item $f(t) = \pi/(2t)$
  \item $f(t) = \pi/t$
  \item $f(t) = 2\pi/t$
  \item $f(t) = \pi t$
  \item $f(t) = 2\pi t$
  \end{enumerate}

  {\em Answer}: Option (A)

  {\em Explanation}: We get $[(1/t)\arctan(x/t)]_0^\infty$. The
  evaluation at $\infty$ gives $\pi/(2t)$ and the evaluation at $0$
  gives $0$. Subtracting, we get $\pi/(2t)$.

  {\em Performance review}: $13$ out of $19$ people got this
  correct. $2$ people each chose (B), (C), and (D).

  {\em Historical note}: This question appeared in a 153 quiz. At the
  time, $17$ out of $25$ got this correct. $5$ chose (B), $2$ chose
  (D), $1$ chose (C).

\item $F(x,t) := 1/(t^2 + x^2)^2$. Find $f$.

  \begin{enumerate}[(A)]
  \item $f(t) = \pi/t^3$
  \item $f(t) = \pi/(2t^3)$
  \item $f(t) = \pi/(4t^3)$
  \item $f(t) = \pi/(8t^3)$
  \item $f(t) = 3\pi/(8t^3)$
  \end{enumerate}

  {\em Answer}: Option (C)

  {\em Explanation}: Put in $\theta = \arctan(x/t)$. Substitute, and
  we get $(1/t^3) \int_0^{\pi/2} \cos^2\theta \,
  d\theta$. Integrating, we get $[\theta/2t^3 +
  \sin(2\theta)/4t^3]_0^{\pi/2}$. The trigonometric part vanishes
  between limits, and we are left with $\pi/(4t^3)$

  {\em Performance review}: $13$ out of $19$ people got this
  correct. $3$ people chose (D) and $1$ each chose (A), (B), and (E).

  {\em Historical note}: This question appeared in a 153 quiz. At the
  time, $15$ out of $25$ people got this correct. $5$ chose (B), $2$
  chose (A), $1$ each chose (D) and (E), $1$ left the question blank.

\item $F(x,t) = \exp(-(tx)^2)$. Use that $\int_0^\infty \exp(-x^2) \, dx=
  \sqrt{\pi}/2$. Find $f$.

  \begin{enumerate}[(A)]
  \item $f(t) = t^2\sqrt{\pi}/2$
  \item $f(t) = t\sqrt{\pi}/2$
  \item $f(t) = \sqrt{\pi}/2$
  \item $f(t) = \sqrt{\pi}/(2t)$
  \item $f(t) = \sqrt{\pi}/(2t^2)$
  \end{enumerate}

  {\em Answer}: Option (D)

  {\em Explanation}: Put $u = tx$, get a $1/t$ on the outside, giving
  $(1/t) \int_0^\infty \exp(-u^2) \, du$.

  {\em Performance review}: $7$ out of $19$ people got this
  correct. $7$ people chose (E), $2$ each chose (A) and (B), and $1$
  chose (C).

  {\em Historical note}: This question appeared in a 153 quiz. At the
  time, $12$ out of $25$ people got the question correct. $6$ chose
  (E), $3$ chose (C), $2$ each chose (A) and (B).

\item In the same general setup as above (but with none of these
  specific $F$s), which of the following is a {\em sufficient}
  condition for $f$ to be an increasing function of $t$?

  \begin{enumerate}[(A)]
  \item $t \mapsto F(x_0,t)$ is an increasing function of $t$ for
    every choice of $x_0 \ge 0$.
  \item $x \mapsto F(x,t_0)$ is an increasing function of $x$ for
    every choice of $t_0 \in (0,\infty)$.
  \item $t \mapsto F(x_0,t)$ is a decreasing function of $t$ for
    every choice of $x_0 \ge 0$.
  \item $x \mapsto F(x,t_0)$ is a decreasing function of $x$ for
  every choice of $t_0 \in (0,\infty)$.
  \item None of the above.
  \end{enumerate}

  {\em Answer}: Option (A)

  {\em Explanation}: If $F$ is increasing in $t$ for every value of
  $x_0$, then that means that as $t$ gets bigger, the function $F$
  being integrated gets bigger everywhere in $x$, i.e., if $t_1 <
  t_2$, then $F(t_1,x_0) < F(t_2,x_0)$ for every $x_0 \ge 0$. The
  integral for the larger value $t_2$ must therefore also be
  bigger. (We looked at this stuff in Section 5.8 of the book).

  {\em Performance review}: $3$ out of $19$ people got this
  correct. $9$ chose (B), $4$ chose (E), $2$ chose (C), $1$ chose (D).

  {\em Historical note}: This question appeared in a 153 quiz. At the
  time, $4$ out of $25$ got the question correct. $10$ chose (B), $5$
  chose (E), $3$ chose (C), $2$ chose (D), and $1$ left the question
  blank.

  (A) was the ``obvious'' choice -- people may have tried to seek more
  subtletly in the question than it had.


\end{enumerate}

\end{document}