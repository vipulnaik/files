\documentclass[10pt]{amsart}

%Packages in use
\usepackage{fullpage, hyperref, vipul, enumerate}

%Title details
\title{Class quiz solutions: May 23: Max-min values: two-variable version}
\author{Math 195, Section 59 (Vipul Naik)}
%List of new commands

\begin{document}
\maketitle

\section{Performance review}

$17$ people took the quiz. The score distribution was as follows:

\begin{itemize}
\item Score of $0$: $1$ person
\item Score of $1$: $3$ people
\item Score of $2$: $8$ people
\item Score of $3$: $3$ people
\item Score of $4$: $1$ person
\item Score of $5$: $1$ person
\end{itemize}

The question wise answers and performance review were as follows:

\begin{enumerate}
\item Option (B): $12$ people
\item Option (B): $9$ people
\item Option (C): $5$ people
\item Option (C): $8$ people. {\em Please review this after Wednesday's lecture!}
\item Option (D): $3$ people {\em Please review this after Wednesday's lecture!}
\end{enumerate}

\section{Solutions}

\begin{enumerate}
\item Suppose $F(x,y) := f(x) + g(y)$, i.e., $F$ is additively
  separable. Suppose $f$ and $g$ are differentiable functions of one
  variable, defined for all real numbers. What can we say about the
  critical points of $F$ in its domain $\R^2$?

  \begin{enumerate}[(A)]
  \item $F$ has a critical point at $(x_0,y_0)$ iff $x_0$ is a
    critical point for $f$ {\em or} $y_0$ is a critical point for $g$.
  \item $F$ has a critical point at $(x_0,y_0)$ iff $x_0$ is a
    critical point for $f$ {\em and} $y_0$ is a critical point for
    $g$.
  \item $F$ has a critical point at $(x_0,y_0)$ iff $x_0 + y_0$ is a
    critical point for $f + g$, i.e., the function $x \mapsto f(x) +
    g(x)$.
  \item $F$ has a critical point at $(x_0,y_0)$ iff $x_0y_0$ is a
    critical point for $fg$, i.e., the function $x \mapsto f(x)g(x)$.
  \item None of the above.
  \end{enumerate}
  
  {\em Answer}: Option (B)

  {\em Explanation}: $F_x(x_0,y_0) = f'(x_0)$ and $F_y(x_0,y_0) =
  g'(y_0)$. In order for both of these to be $0$, we must have
  $f'(x_0) = g'(y_0) = 0$. Thus, $x_0$ is a critical point for $f$ and
  $y_0$ is a critical point for $g$.

  {\em Performance review}: $12$ out of $17$ people got this
  correct. $3$ chose (C) and $2$ chose (A).
\item Suppose $F(x,y) := f(x)g(y)$ is a multiplicatively separable
  function. Suppose $f$ and $g$ are both differentiable functions of
  one variable defined for all real inputs. Consider a point
  $(x_0,y_0)$ in the domain of $F$, which is $\R^2$. Which of the
  following is true?

  \begin{enumerate}[(A)]
  \item $F$ has a critical point at $(x_0,y_0)$ if and only if $x_0$
    is a critical point for $f$ and $y_0$ is a critical point for $g$.
  \item If $x_0$ is a critical point for $f$ and $y_0$ is a critical
    point for $g$, then $(x_0,y_0)$ is a critical point for
    $F$. However, the converse is not necessarily true, i.e.,
    $(x_0,y_0)$ may be a critical point for $F$ even without $x_0$
    being a critical point for $f$ and $y_0$ being a critical point
    for $g$.
  \item If $(x_0,y_0)$ is a critical point for $F$, then $x_0$ must be
    a critical point for $f$ and $y_0$ must be a critical point for
    $g$. However, the converse is not necessarily true.
  \item $(x_0,y_0)$ is a critical point for $F$ if and only if {\em at
    least} one of these is true: $x_0$ is a critical point for $f$ and
    $y_0$ is a critical point for $g$.
  \item None of the above.
  \end{enumerate}

  {\em Answer}: Option (B)

  {\em Explanation}: We have $F_x(x_0,y_0) = f'(x_0)g(y_0)$ and
  $F_y(x_0,y_0) = f(x_0)g'(y_0)$. We see that if $f'(x_0) = 0$ and
  $g'(y_0) = 0$, then $F$ has a critical point at
  $(x_0,y_0)$. However, $F$ could also have a critical point for other
  reasons: for instance, $f(x_0) = f'(x_0) = 0$ (and other cases).

  {\em Performance review}: $9$ out of $17$ people got this
  correct. $5$ chose (A), $2$ chose (D), $1$ chose (C).
\item Consider a homogeneous polynomial $ax^2 + bxy + cy^2$ of degree
  two in two variables $x$ and $y$. Assume that at least one of the
  numbers $a$, $b$, and $c$ is nonzero. What can we say about the local
  extreme values of this polynomial on $\R^2$?

  \begin{enumerate}[(A)]
  \item If $b^2 - 4ac < 0$, then the function has no local extreme
    values and its value is unbounded from both above and below. If
    $b^2 - 4ac = 0$, the function has local extreme value $0$ and this
    is attained on a line through the origin. If $b^2 - 4ac > 0$, the
    function has local extreme value $0$ and this is attained only at
    the origin.
  \item If $b^2 - 4ac < 0$, then the function has no local extreme
    values and its value is unbounded from both above and below. If
    $b^2 - 4ac > 0$, the function has local extreme value $0$ and this
    is attained on a line through the origin. If $b^2 - 4ac = 0$, the
    function has local extreme value $0$ and this is attained only at
    the origin.
   \item If $b^2 - 4ac > 0$, then the function has no local extreme
    values and its value is unbounded from both above and below. If
    $b^2 - 4ac = 0$, the function has local extreme value $0$ and this
    is attained on a line through the origin. If $b^2 - 4ac < 0$, the
    function has local extreme value $0$ and this is attained only at
    the origin.
  \item If $b^2 - 4ac > 0$, then the function has no local extreme
    values and its value is unbounded from both above and below. If
    $b^2 - 4ac < 0$, the function has local extreme value $0$ and this
    is attained on a line through the origin. If $b^2 - 4ac = 0$, the
    function has local extreme value $0$ and this is attained only at
    the origin.
  \item If $b^2 - 4ac = 0$, then the function has no local extreme
    values and its value is unbounded from both above and below. If
    $b^2 - 4ac < 0$, the function has local extreme value $0$ and this
    is attained on a line through the origin. If $b^2 - 4ac > 0$, the
    function has local extreme value $0$ and this is attained only at
    the origin.
  \end{enumerate}

  {\em Answer}: Option (C)

  {\em Explanation}: See discussion of homogeneous quadratic in the
  relevant lecture notes.

  {\em Performance review}: $5$ out of $17$ people got this
  correct. $4$ chose (A), $4$ chose (B), $2$ chose (D), $2$ chose (E).

  A subset of $\R^n$ is termed {\em convex} if the line segment
  joining any two points in the subset is completely within the
  subset. A function $f$ of two variables defined on a closed convex
  domain is termed {\em quasiconvex} if given any two points $P$ and
  $Q$ in the domain, the maximum of $f$ restricted to the line segment
  joining $P$ and $Q$ is attained at one (possibly both) of the
  endpoints $P$ or $Q$.

  There are many examples of quasiconvex functions, including linear
  functions (which are quasiconvex but not strictly quasiconvex)
  and all convex functions.


\item What can we say about the maximum of a continuous quasiconvex
  function defined on the circular disk $x^2 + y^2 \le 1$?

  \begin{enumerate}[(A)]
  \item It must be attained at the center of the disk, i.e., the
    origin $(0,0)$.
  \item It must be attained somewhere in the interior of the disk, but
    we cannot be more specific with the given information.
  \item It must be attained somewhere on the boundary circle $x^2 +
    y^2 = 1$. However, we cannot be more specific than that with the
    given information.
  \item It must be attained at one of the four points $(1,0)$,
    $(0,1)$, $(-1,0)$, and $(0,-1)$. 
  \item It could be attained at any point. We cannot be specific at all.
  \end{enumerate}

  {\em Answer}: Option (C)

  {\em Explanation}: For any point in the interior of the circular
  disk, the point can be put on a chord with endpoints on the boundary
  circle. The maximum of the function restricted to this chord occurs
  at one of the boundary points, hence, the value at any point in the
  interior is equaled or exceeded by some point in the boundary. Thus,
  the maximum is attained at some point in the boundary.

  However, no point in the boundary can be put in the interior of a
  line segment joining two other points, i.e., each point in the
  boundary is extreme. So we cannot narrow things down further.

  {\em Performance review}: $8$ out of $17$ people got this
  correct. $5$ chose (B), $3$ chose (E), $1$ chose (D).

\item What can we say about the maximum of a continuous quasiconvex
  function defined on the square region $|x| + |y|\le 1$? This is the
  region bounded by the square with vertices $(1,0)$, $(0,1)$,
  $(-1,0)$, and $(0,-1)$.

  \begin{enumerate}[(A)]
  \item It must be attained at the center of the square, i.e., the
    origin $(0,0)$.
  \item It must be attained somewhere in the interior of the square, but
    we cannot be more specific with the given information.
  \item It must be attained somewhere on the boundary square $|x| +
    |y| \le 1$. However, we cannot be more specific than that with the
    given information.
  \item It must be attained at one of the four points $(1,0)$,
    $(0,1)$, $(-1,0)$, and $(0,-1)$. 
  \item It could be attained at any point. We cannot be specific at all.
  \end{enumerate}

  {\em Answer}: Option (D)

  {\em Explanation}: We can first push to the boundary the same way as
  for the circular disk. Now, unlike the circular disk, we note that
  any point in the boundary that is to one of the vertices is on a
  line segment joining two adjacent vertices, so we can push out to
  the vertices.

  {\em Performance review}: $3$ out of $17$ people got this
  correct. $8$ chose (C), $3$ chose (B), $2$ chose (A), $1$ chose (E).
\end{enumerate}

\end{document}