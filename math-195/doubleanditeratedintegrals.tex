\documentclass[10pt]{amsart}
\usepackage{fullpage,hyperref,vipul,graphicx}
\title{Double integrals and iterated integrals}
\author{Math 195, Section 59 (Vipul Naik)}

\begin{document}
\maketitle

{\bf Corresponding material in the book}: Section 15.2, 15.3. {\em
Note: We are omitting the question types from the book that require
three-dimensional visualization, i.e., those that require sketching
figures in three dimensions to compute volumes}.

{\bf What students should definitely get}: The procedure for computing
double integrals over rectangles using iterated integrals, the
procedure for computing double integrals over other regions using
iterated integrals, the idea of Fubini's theorem and its use in
interchanging the order of integration. Use of symmetry and
inequality-based bounding/estimation techniques.

{\bf What students should hopefully get}: Relation between single and
double integrals, dealing with piecewise cases, breaking up domain
into smaller pieces when direct integration over entire domain is
infeasible.

{\em Note}: The lecture notes contain only a few examples. For more
examples, please refer to worked examples in Sections 15.2 and 15.3.

\section*{Executive summary}

Words ...

\begin{enumerate}
\item The double integral of a function $f$ of two variables, over a
  domain $D$ in $\R^2$, is denoted $\int \int_D f(x,y) \, dA$ and
  measures an infinite analogue of the sum of $f$-values at all points
  in $D$.
\item Fubini's theorem for rectangles states that if $F$ is a function
  of two variables on a rectangle $R = [a,b] \times [p,q]$, such that
  $F$ is continuous except possibly at the union of finitely many
  smooth curves, then the integral equals either of these iterated
  integrals:

  $$\int \int_R F(x,y) \, dA = \int_a^b \int_p^q F(x,y) \, dy \, dx = \int_p^q \int_a^b F(x,y) \, dx \, dy$$

\item For a function $f$ defined on a closed connected bounded domain
  $D$ with a smooth boundary, we can make sense of $\int \int_D f(x,y)
  \, dA$ as being $\int \int_R F(x,y) \, dA$ where $R$ is a
  rectangular region containing $D$ and $F$ is a function that equals
  $f$ on $D$ and is $0$ on the rest of $R$.
\item Suppose $D$ is a Type I region, i.e., its intersection with
  every vertical line is either empty or a point or a line
  segment. Then, we can describe $D$ as $a \le x \le b$, $g_1(x) \le y
  \le g_2(x)$, where $g_1$ and $g_2$ are continuous functions. The
  integral $\int \int_D f(x,y) \, dA$ becomes:

  $$\int_a^b \int_{g_1(x)}^{g_2(x)} f(x,y) \, dy \, dx$$

\item Suppose $D$ is a Type II region, i.e., its intersection with
  every horizontal line is either empty or a point or a line
  segment. Then, we can describe $D$ as $p \le y \le q$, $g_1(y) \le x
  \le g_2(y)$, where $g_1$ and $g_2$ are continuous functions. The
  integral $\int \int_D f(x,y) \, dA$ becomes:

  $$\int_p^q \int_{g_1(y)}^{g_2(y)} f(x,y) \, dx \, dy$$

\item The double integral of $f + g$ over $D$ is the sum of the double
  integral of $f$ over $D$ and the double integral of $g$ over
  $D$. Similarly, scalars can be pulled out of double integrals.
\item The integral of the function $1$ over a domain is the area of
  the domain.
\item If $f(x,y) \ge 0$ on a domain $D$, the integral of $f$ over $D$
  is also $\ge 0$.
\item If $f(x,y) \ge g(x,y)$ on a domain $D$, the integral of $f$ over
  $D$ is $\ge$ the integral of $g$ over $D$.
\item If $m \le f(x,y) \le M$ over a domain $D$, then $\int \int_D
  f(x,y) \, dA$ is betweem $mA$ and $MA$ where $A$ is the area of $D$.
\item If $f(x,y)$ is odd in $x$ and the domain of integration is
  symmetric about the $y$-axis, the integral is zero. If $f(x,y)$ is
  odd in $y$ and the domain is symmetric about the $x$-axis, the
  integral is zero.
\end{enumerate}

Actions ...

\begin{enumerate}
\item To compute a double integral, compute it as an iterated
  integral. For a rectangle, we can choose either order of
  integration, as long as the integration is feasible. For other types
  of regions, we need to first determine whether the region is Type I
  or Type II, and break it up into pieces of those types.
\item For a multiplicatively separable function over a rectangular
  region (or for a sum of such multiplicatively separable functions),
  things are particularly easy.
\item Sometimes, an integral cannot be computed using a particular
  order of integration -- we might get stuck on the inner or the outer
  stage. However, it may be computable using the other order of
  integration.
\item We can often use symmetry-based techniques to argue that certain
  parts of the integrand integrate to zero.
\item Even in cases where the integral cannot be computed, we can
  bound it between limits using maximum or minimum values of function
  and/or using bigger or smaller regions on which the integral can be
  computed.
\end{enumerate}

\section{Double integral and iterated integral}

\subsection{What's a double integral?}

We will study the theory of double integrals (Section 16.1 of the
book) a little later in the course. For now, we provide an intuitive
idea of a double integral. Suppose $f$ is a function of two variables
$(x,y)$. The {\em double integral} of $f$ over a subset $D$ of $\R^2$
on which $f$ is defined is the {\em total contribution} of the
$f$-values at all points in the domain. One way of thinking of it is
as follows: we divide $D$ up into a lot of small regions, we pick a
point in each region, multiply the $f$-value at that point with the
area of the region and add up. This total gives the integral of $f$
over the region $D$.

The notation for the double integral of a function $f$ over a region
$D$ is:

$$\int \int_D f(x,y) \, dA$$

The $dA$ here represents an area element or area differential, and is
the two-dimensional analogue of $dx$ in one dimension. A detailed
exploration of the meaning is possible, but beyond our current scope.

The double integral does integration over a {\em region} in the same
way that the ordinary (single) integral does integration over an {\em
interval}. The region over which integration is being done is termed
the {\em region of integration} or {\em domain of integration} and the
function being integrated is termed the {\em integrand}.

For a function with nonnegative values, the double integral over a
region can also be interpreted as a volume. We will see this
interpretation a little later here.

\subsection{Linearity}

The double integral of a sum of two functions is the sum of their
double integrals:

$$\int \int_D [f(x,y) + g(x,y)] \, dA = \int \int_D f(x,y) \, dA + \int \int_D g(x,y) \, dA$$

Also, scalars can be pulled out of double integrals:

$$\int \int_D cf(x,y) \, dA = c \int \int_D f(x,y) \, dA$$

\subsection{What's an iterated integral?}

An iterated integral is an expression that involves an integral inside
another integral (and possibly even more integrals. For instance:

$$\int_a^b \left(\int_{p(x)}^{q(x)} f(x,y) \, dy\right) \, dx$$

What this means is:

\begin{itemize}
\item We first compute the inner integral by integrating with respect
  to $y$, treating $x$ as a constant. If $F(x,y)$ is an
  antiderivative, then the definite integral is $F(x,q(x)) -
  F(x,p(x))$.
\item The final answer computed above now depends only on $x$, the
  variable $y$ has been integrated over and thus discarded. We now
  integrate this function of $x$ between the limits $a$ and $b$.
\end{itemize}

A special case of this kind of iterated integral is one where the
limits for the inner function do not depend on the outer variable,
i.e., an integration of the form:

$$\int_a^b \left(\int_p^q F(x,y) \, dy \right) \, dx$$

Note that we could also consider an iterated integral where the inner
variable of integration is $x$ and the outer variable of integration
is $y$.

When things are reasonably clear, we can drop the parenthesization for
iterated integrals, so the above can be written as:

$$\int_a^b \int_p^q F(x,y) \, dy \, dx$$

\subsection{Fubini's theorem relating double and iterated integrals on rectangles}

Consider the filled rectangle $R = [a,b] \times [p,q]$ in the
$xy$-plane. This is a rectangle with vertices $(a,p)$, $(b,p)$,
$(a,q)$, and $(b,q)$. The region can be described as $\{ (x,y) : x \in
[a,b], y \in [p,q] \}$. Fubini's theorem for rectangles says that if
$F$ is a continuous function of two variables defined on this filled
rectangle, then:

$$\int \int_R F(x,y) dA = \int_a^b \int_p^q F(x,y) \, dy \, dx = \int_p^q \int_a^b F(x,y) \, dx \, dy$$

In other words, the double integral equals the iterated integral
computed in either order.

The assumption of continuity can be weakened somewhat: we only need to
assume that $f$ is bounded on $R$, and the set of points where it is
discontinuous is contained in a union of a finite number of smooth
curves. This generalization will help us deduce an important corollary
for functions whose domains are not rectangular.

\subsection{Intuitive explanation of Fubini's theorem}

Recall that the double integral of a function $F(x,y)$ can be thought
of as follows: $F(x,y)$ denotes the value at point $(x,y)$, and the
double integral is the total contribution of all points. For instance,
$F(x,y)$ could denote the pressure at the point $(x,y)$, and the
double integral over the rectangle/region is the total force exerted
on the region.

Iterated integration serves to break this integration up by {\em
slicing} horizontally or vertically. Let's be more specific:

\begin{itemize}
\item The iterated integral $\int_a^b \left(\int_p^q F(x,y) \,
  dy\right) \, dx$ can be interpreted as follows: the inner integral
  $\int_p^q F(x,y) \, dy$ is integrating {\em along a vertical slice}
  for a fixed value of $x$ (i.e., along a line parallel to the
  $y$-axis). The outer integral is then adding up the contributions of
  all the vertical slices.
\item The iterated integral $\int_p^q \left(\int_a^b F(x,y) \,
  dx\right) \, dy$ can be interpreted as follows: the inner integral
  $\int_a^b F(x,y) \, dx$ is integrating {\em along a horizontal
  slice} for a fixed value of $y$ (i.e., along a line parallel to the
  $x$-axis). The outer integral is then adding up the contributions of
  all the horizontal slices.
\end{itemize}

That all these values are the same is some infinite version of the
idea that addition is commutative and associative, i.e., we can
regroup summations by collecting all things with one common coordinate
and then adding up over that coordinate.

\subsection{The special case of multiplicatively separable functions}

A case worth noting is where $F(x,y)$ is of the form $F(x,y) =
f(x)g(y)$, i.e., we can separate it as the product of a function
purely of $x$ and a function purely of $y$.

Using the notation established above, if $f$ is continuous on $[a,b]$
and $g$ is continuous on $[p,q]$, then $F$ is continuous on $R = [a,b]
\times [p,q]$ and:

$$\int \int_R F(x,y) \, dA = \left(\int_a^b f(x) \, dx \right) \left(\int_p^q g(y) \, dy \right)$$

This is a corollary of Fubini's theorem, and can be deduced by using
either of the iterated integrals.

In particular, this means that if $F$ can be written as a {\em sum} of
multiplicatively separable functions, then its integral is a sum of
the products of integrals of these functions. In fancy notation, if $f
= \langle f_1, f_2, \dots, f_n \rangle$ and $g = \langle
g_1,g_2,\dots, g_n \rangle$, with all the $f_i$s continuous on $[a,b]$
and all the $g_i$s continuous on $[p,q]$, and if $F(x,y) =
\sum_{i=1}^n f_i(x)g_i(y)$, then:

$$\int \int_R F(x,y) \, dA = \sum_{i=1}^n \left[\left(\int_a^b f_i(x) \, dx \right) \left(\int_p^q g_i(y) \, dy \right)\right]$$

Note also that when calculating the integral of a multiplicatively
separable function, if either of the integrals of the pieces is zero,
the other one does not need to be computed and the product is zero. We
will see related ideas a little later when we cover symmetry.
\subsection{A concept of antiderivative}

Suppose $G$ is a function with the property that $G_{xy} = F$, i.e.,
$F$ is the mixed second-order partial derivative of $G$. Then, the
integral of $F$ over a rectangle $[a,b] \times [p,q]$ is:

$$G(b,q) - G(a,q) - G(b,p) + G(a,p)$$

Basically, the top right and bottom left values get added and the
bottom right and top left values get subtracted.

This is sort of like an antiderivative. But the approach is rarely
used for explicit computations and we usually try to find definite
integrals.
\section{Double integrals over regions other than rectangles}

\subsection{Defining such a double integral using a rectangle}

Suppose $D$ is a closed bounded region in the plane. In
particular, this means that $D$ can be enclosed inside a
rectangular region. Suppose $R$ is such a rectangular region. Then, we
define the double integral $\int \int_D f(x,y) \, dA$ as $\int
\int_R F(x,y) \, dA$ where $F(x,y)$ is defined as:

$$F(x,y) := \lbrace\begin{array}{rl} f(x,y), & (x,y) \in D\\ 0, & (x,y) \notin D \end{array}$$

In other words, we integrate the function that's $f$ on $D$ and $0$
outside. Note that $F$ need not be continuous, even if $f$ is. So, we
might be skeptical of applying results such as Fubini's theorem to
$F$. If, however, the boundary of $D$ is a piecewise smooth curve,
then by the slightly more general formulation of Fubini's theorem, it
turns out that the continuity of $f$ within $D$ allows us to apply
Fubini's theorem to $F$. This is great news because it means that we
can compute double integrals as iterated integrals.

\subsection{Type I and Type II regions}

For simplicity, we assume that the regions we are dealing with are all
connected, closed, bounded regions and their boundary curves are
piecewise smooth.

We call a region $D$ in the $xy$-plane a {\em Type I region} if its
intersection with every line parallel to the $y$-axis is either empty,
or a point, or a line segment, i.e., the intersection is always
connected. Such a region can be described as the region enclosed by
the graphs of two continuous functions $y = g_1(x)$ and $y = g_2(x)$,
with $g_1(x) \le g_2(x)$, for $x$ in an interval $[a,b]$. The function
$g_2$ is simply the $y$-coordinate value of the upper endpoint of the
line segment and the function $g_1$ is the $y$-coordinate value of the
lower endpoint of the line segnment. In other words:

$$D = \{ (x,y) : a \le x \le b, g_1(x) \le y \le g_2(x)\}$$

We can compute double integrals over Type I regions using iterated
integration. To integrate $f(x,y)$ over the Type I region of the kind
given above:

$$\int \int_D f(x,y) \, dA = \int_a^b \int_{g_1(x)}^{g_2(x)} f(x,y) \, dy \, dx$$

A {\em Type II region} is a region whose intersection with every
horizontal line is either empty or a point or a line segment. Such a
region can be described as the region enclosed by the the graphs of
continuous functions with $x$ expressed in terms of $y$, i.e., something of the form:

$$D = \{ (x,y) : p \le y \le q, g_1(y) \le x \le g_2(y) \}$$

To integrate over the Type II region of the kind given above, we can
do the integration:

$$\int \int_D f(x,y) \, dA = \int_p^q \int_{g_1(y)}^{g_2(y)} f(x,y) \, dx \, dy$$

Note that both these results follow from the general version of
Fubini's theorem for rectangles, using the trick of transitioning to
$F(x,y)$ from $f(x,y)$.

\subsection{Convex regions}

A {\em convex region} is a region with the property that for any two
points in the region, the line segment joining those two points lies
completely inside the region. Convex regions are both Type I and Type
II. In particular, this means we can use either of the integration
methods to compute integrals over convex regions.

Circular disks, triangular regions, and rectangular regions are all
examples of convex regions. A heart-shaped region is {\em not} a
convex region.

\subsection{Breaking up a region into Type I and Type II regions}

If a region $D$ is closed, connected, and bounded with a smooth
bounding curve, and $f$ is a continuous function of $D$, it may still
happen that $D$ is neither Type I nor Type II. There are still some
ways out. The first is to partition $D$ into finitely many pieces (chambers) such
that:

\begin{itemize}
\item Each piece is Type I or Type II
\item The intersection of any two of the pieces is one-dimensional and
  hence the restriction of the double integral over that intersection
  is zero.
\item The double integral on $D$ is now the sum of the values of
  double integrals on each piece, and each of the individual double
  integrals can be computed as an iterated integral by Fubini's
  theorem.
\end{itemize}

This is a two-dimensional analogue of chopping up an interval into
sub-intervals using a partition. Here, instead of sub-intervals, we
use subregions.

In the one-dimensional case, the slight overlap (isolated points)
between the partitioned pieces does not result in any double-counting,
i.e., the integral on the whole interval is the sum of the integrals
on the parts. In the two-dimensional cases, the slight overlap at
boundary curves (which are one-dimensional) does not result in any
double-counting, because the boundary curves are
infinitesimal/negligible.

\section{In practice: computing iterated and double integrals}

\subsection{Theory versus practice: the one-variable nightmare}

Let's recall the situation in one variable first and then we'll
discuss how the situation changes with more variables. We know that
any continuous function in one variable is integrable. This knowledge
does not always translate to actually being able to find expressions
for the integrals. There are three levels of difficulty:

\begin{itemize}
\item First, there are many functions expressible in terms of
  elementary functions but which do not have antiderivatives
  expressible in terms of elementary functions. To give names to the
  antiderivatives, we need to invent new branches of mathematics. For
  instance, logarithms were invented to integrate $1/x$, and
  trigonometry was invented to integrate $1/(x^2 + 1)$. But there's a
  lot more work to do -- some functions slip through the cracks and
  integrating them requires us to invent more branches of mathematics.

  Examples of elementarily expressible functions that do not have
  elementarily expressible antiderivatives are $e^{-x^2}$,
  $\sin(x^2)$, $(\sin x)/x$, $(e^x - 1)/x$, $1/\sqrt{x^4 + 1}$, and
  many others.
\item Second, the procedure for integrating a function does not break
  down into a bunch of deterministic rules. This is in sharp contrast
  with differentiation, where if we know how to differentiate a bunch
  of functions, we know how to differentiate all functions generated
  from them using the processes of pointwise combination, composition,
  inverses, and piecewise combination. For integration, all we have
  are heuristics. Thus, even if a neat antiderivative does exist, it
  can be hard to find.
\item Third, even if we are able to find antiderivatives, computing
  their values between limits can be difficult. Even to integrate a
  rational function, we need $\ln$ and $\arctan$ and computing the
  values of these is hard.
\end{itemize}

Each of these challenges continues to operate in many variables. With
multiple variables, there is some further bad news and some mitigating
good news. We turn to these.

\subsection{The further bad news}

The inner-most step of an iterated integral is something like:

$$\int_a^b f(x,y) \, dx$$

Here, we are treating $y$ as a constant temporarily while doing this
integration. However, we cannot put an actual value on $y$ -- it's an
{\em unknown known} for now, and in fact, when we have completed this
integration and are willing to move on outward, it will become a
variable again. Thus, this integration really is not integrating a
plain vanilla function but rather trying to do a large number of
integrations -- one for each fixed value of $y$ -- {\em
simultaneously} by getting a generic expression.

Now, it may turn out that there is no uniform general expression for
$y$. Consider the example:

$$\int_2^3 \frac{dx}{x^2 + \sin y}$$

When $\sin y > 0$, the integral becomes:

$$\frac{1}{\sqrt{\sin y}} \left[\arctan(x/\sqrt{\sin y})\right]_2^3$$

When $\sin y = 0$, the integral becomes:

$$[-1/x]_2^3 = (1/2) - (1/3) = 1/6$$

When $\sin y < 0$, the integral becomes:

$$\frac{1}{2\sqrt{-\sin y}}\left[\ln((x - \sqrt{-\sin y})/(x + \sqrt{-\sin y}))\right]_2^3$$

So, even though the original function had a single piece description,
the new function we get after integrating has a {\em piecewise
description}.

This will occur only rarely, and not in the routine examples that we
will see. Also, although it complicates matters, it does not make the
task any more impossible. To perform the outer integration for the
resultant piecewise function, we simply break the domain (for the
outer integration) into the various pieces and perform the integration
separately in each piece.

\subsection{More bad news for non-rectangular regions}

Another piece of bad news, that applies particularly to
non-rectangular regions, is that complications could arise not only
from the nature of the integrand, but also from the shape of the
region. For Type I or Type II regions, the nature of the bounding
functions that determine how the inner variable varies in terms of the
outer variable determine the expression to be integrated on the
outside. Thus, even for very easy functions $f(x,y)$, the actual
integration procedure may become difficult because of the complexity
arising from the shape of the region.

\subsection{The good news: use Fubini's to change order of integration}

The good news is that sometimes, an integral is impossible to do when
written as an iterated integral with a particular ordering of $x$ and
$y$, but can be done if the ordering of $x$ and $y$ were
reversed. Luckily, by Fubini's theorem, the answers have the same
value.

Let's consider a couple of examples.

Our first example is the function $x^y$ on the interval $[0,1] \times
[0,1]$. The domain is a square region with vertices $(0,0)$, $(0,1)$,
$(1,0)$ and $(1,1)$. Note that the function is undefined at the bottom
left vertex $(0,0)$. It takes the value $1$ on the lower edge, $0$ on
the left edge, $1$ on the right edge, and is equal to the function $x$
on the top edge. Note that everywhere in the square where it is
defined, the function takes a value in $[0,1]$. We want to integrate
it over the square.

We could set up the integral as an iterated integral in either of
these two ways:

$$\int_0^1 \int_0^1 x^y \, dy \, dx, \qquad \int_0^1 \int_0^1 x^y \, dx \, dy$$

Let's consider the first formulation of the integral:

$$\int_0^1 \int_0^1 x^y \, dy \, dx$$

The inner integral is:

$$\int_0^1 x^y \, dy$$

This simplifies to:

$$\left[\frac{x^y}{\ln x}\right]_0^1 = \frac{x - 1}{\ln x}$$

The new integral that we need to compute is thus:

$$\int_0^1 \frac{x - 1}{\ln x} \, dx$$

The {\em indefinite integral} of the integrand is not possible to
compute. So we're basically stuck.

On the other hand, if we use the other formulation ({\em FIXED ERROR
  BELOW!}:

$$\int_0^1 \int_0^1 x^y \, dx \, dy$$

The inner integral is:

$$\int_0^1 x^y \, dx$$

This simplifies to:

$$\left[\frac{x^{y+1}}{y + 1}\right]_0^1 = \frac{1}{y + 1}$$

We can now integrate this:

$$\int_0^1 \frac{1}{y + 1} \, dy = [\ln(y + 1)]_0^1 = \ln 2$$

Note that in this example, integrating in the wrong order got us into
problems at the {\em outer stage}, not at the inner stage. In some
cases, integrating in the wrong order can prevent us from getting
started. Here is an example:

$$\int_0^1 \int_x^1 \exp(-y^2) \, dy \, dx$$

This is the integral of the function $\exp(-y^2)$ over the triangular
region for the triangle with vertices $(0,0)$, $(1,1)$, and $(0,1)$,
i.e., the upper left half triangle in the unit square $[0,1] \times
[0,1]$. Unfortunately, as written here, the inner integral cannot be
computed in elementary terms.

Note that the region here is both a Type I and a Type II
region. This means that it can be sliced either vertically or
horizontally. If we slice horizontally instead, then for any fixed
$y$, the constraint on $x$ is $0 \le x \le y$, and we get:

$$\int_0^1 \int_0^y \exp(-y^2) \, dx \, dy$$

The inner integral is now:

$$\int_0^y \exp(-y^2) \, dx = y\exp(-y^2)$$

The outer integral now becomes:

$$\int_0^1 y \exp(-y^2) \, dy = \left[\frac{-1}{2}\exp(-y^2)\right]_0^1 = \frac{1}{2}\left(1 - \frac{1}{e}\right)$$

This is similar to Example 3 in the book.

\subsection{Integrating polynomials}

Polynomials are very easy to integrate {\em over rectangular regions}
because every polynomial is a sum of monomials, every monomial is
multiplicatively separable as a product of power functions, and each
power function can be integrated.

For instance, to integrate the polynomial $xy + 2x^5y^3$ over the
interval $[1,3] \times [4,6]$, we do:

$$\int_1^3 x \, dx \int_4^6 y \, dy + 2 \int_1^3 x^5 \, dx \int_4^6 y^3 \, dy$$

In other words, it is a sum of products of integrals of power
functions of one variable. The rest is just straightforward arithmetic.

To integrate a polynomial over a non-rectangular region is a little
trickier, and may not be feasible for all regions. First, note that we
can still additively separate the polynomial as a sum of monomials, so
it suffices to integrate each monomial, i.e., each expression of the
form $x^ay^b$. However, because the region is no longer rectangular,
we cannot use multiplicative separability.

Here's an example. Consider integrating $x^2y^2$ on the circular disk
$x^2 + y^2 \le 1$. This is both Type I and Type II. If we go for
horizontal slicing, then for $x \in [-1,1]$, we have $-\sqrt{1 - x^2}
\le y \le \sqrt{1 - x^2}$. The integral thus becomes:

$$\int_{-1}^1 \int_{-\sqrt{1 - x^2}}^{\sqrt{1 - x^2}} x^2y^2 \, dy \, dx$$

The inner integral becomes $x^2y^3/3$ which between limits is $2x^2(1
- x^2)^{3/2}/3$. This needs to be integrated on $[-1,1]$. Note how,
even though we started only with polynomials, the integrand for the
outer integration involves fractional powers. The fractional powers
are arising from the shape of the domain of integration.

{\em It is possible to complete the question in the case of circular
disks, but this is best done using double integrals in polar
coordinates, covered in Section 16.4 of the book. We will, however,
not cover this topic formally as part of the syllabus, although I will
explain it in class and give a few examples.}
\subsection{Integrating rational functions}

We first consider integrating rational functions over rectangular
regions. If the denominator of the rational function is of the form
$cx^ay^b$ (i.e., it is a monomial) then the rational function is a sum
of multiplicatively separable functions and can be integrated using the
same idea discussed above for polynomials.

More generally, if the denominator of the rational function can be
factorized as the product of a polynomial in $x$ and a polynomial in
$y$, we can use the multiplicatively separable approach.

For instance, for the rational function:

$$\frac{x^2 + y^2 - 2xy + 3}{x^2y^2 + 2x^2 + y^2 + 2}$$

The denominator can be factored as $(x^2 + 1)(y^2 + 2)$ and hence the rational function can be written as:

$$\frac{x^2}{x^2 + 1}\frac{1}{y^2 + 1} + \frac{1}{x^2 + 1}\frac{y^2}{y^2 + 2} - 2 \frac{x}{x^2 + 1}\frac{y}{y^2 + 2} + 3\frac{1}{x^2 + 1}\frac{1}{y^2 + 2}$$

This is a multiplicatively separable form, and can be integrated over
a rectangular region. Note: We are assuming knowledge of how to
integrate rational functions of one variable, something you saw in one
variable calculus.

In other cases, it is not completely obvious how to do the
integration, so we just try iterated integration and see how it works
out. For instance, consider:

$$\int_1^2 \int_1^2 \frac{1}{x + y} \, dy \, dx$$

The inner integral is $\ln(x + 2) - \ln(x + 1)$. The outer integral thus becomes:

$$\int_1^2 \ln(x + 2) - \ln(x + 1) \, dx$$

After some integration by parts (we skip steps) this becomes:

$$[(x + 2)\ln(x + 2) - (x + 1)\ln(x + 1)]_1^2 = 4\ln 4 - 3 \ln 3 - 3 \ln 3 + 2 \ln 2 = 10 \ln 2 - 6 \ln 3$$

In fact, it is possible to give a sketch for why this kind of
integration procedure will work for a wide variety of (all?) rational
functions.

\subsection{Exponential and trigonometric functions}

Again, for these, one thing to look for is multiplicative
separability, or expressibility as a sum of multiplicatively separable
functions, and hope that each of the constituent functions of one
variable can be integrated.

Consider $f(x,y) = \sin(x + y)$. We want to calculate:

$$\int_a^b \int_p^q \sin(x + y) \, dy \, dx$$

We could do this directly, integrating first with respect to $y$, to
get $-\cos(x + q) + \cos(x + p)$ and then integrating with respect to
$x$ to get $-\sin(b + q) + \sin(a + q) + \sin(b + p) - \sin(a + p)$.

Alternatively, we could rewrite $\sin (x + y) = \sin x \cos y + \cos x
\sin y$ and integrate by additively and then multiplicatively
splitting, to get:

$$(\cos a - \cos b)(\sin q - \sin p) + (\sin b - \sin a)(\cos p - \cos q)$$

It's possible to work out that both of these are the same. (Note: This
is easier to see if we use the ``antiderivative'' concept mentioned
earlier: the antiderivative by iterated integration is $-\sin(x + y)$
and the antiderivative by multiplicative separation is $-\cos x \sin y
- \sin x \cos y$ which becomes the same thing.

For exponential functions, note that $\exp(f(x) + g(y)) =
\exp(f(x))\exp(g(y))$ and is hence multiplicatively separable.


\section{Area and volume interpretations}

To make this course simpler, we will refrain from complicated volume
computations for the surfaces that are graphs of functions, but we
will go over the theoretical facts just in case you need them for the
future.

\subsection{Double integral equals volume}

Consider a function $z = f(x,y)$ on a closed connected bounded domain
$D$ such that $z \ge 0$ for all $(x,y) \in D$. Then, the integral
$\int \int_D f(x,y) \, dA$ equals the volume of the region between the
surface $z = f(x,y)$, the $xy$-plane. On the sides, this region is
bounded by line segments joining points in the boundary of $D$ and the
corresponding points on the graph of the surface above them.

The three-dimensional region can also be described as follows: it is
the union of all the line segments obtained by joining each point
$(x,y,0)$ with the point $(x,y,f(x,y))$ where $(x,y) \in D$. 

The fact that the double integral value equals the volume is the
three-dimensional analogue of the fact that the single integral value
equals the area under the graph of the function.

\subsection{Interpretation of slicing and iterated integration}

We can now interpret the horizontal and vertical slicing.

Computing the integral along a horizontal slice, i.e., a line parallel
to the $xy$-plane, correspondings to computing the area of the
intersection of the region with a plane parallel to the $xz$-plane
through that line. Specifically, computing the integral:

$$\int_a^b f(x,y_0) \, dx$$

means computing the area of the intersection of the region with the
plane $y = y_0$, or equivalently, computing the area under the graph
of the function $x \mapsto f(x,y_0)$ between $x = a$ and $x = b$.

The outer part of the integration then integrates this area function
along the other axis, to give the total volume.

If we perform the integration in the other order, we are computing the
areas of intersection with planes parallel to the $yz$-plane, and then
integrating this area function along the $x$-axis.

Note that all this fits in with the cross sectional method of
determining volume as the integral of the areas of the cross sections
along planes as we move along an axis perpendicular to these planes.

\section{Properties of double integrals}

\subsection{Inequalities that can be used for estimation}

These inequalities are a lot like those of single integrals:

\begin{itemize}
\item If $f(x,y) \ge 0$ on a domain $D$, then $\int \int_D f(x,y) \,
  dA \ge 0$.
\item If $f(x,y) \ge g(x,y)$ on a domain $D$, then $\int \int_D f(x,y)
  \, dA \ge \int \int_D g(x,y) \, dA$.
\item If $D_1 \subseteq D_2$, and $f$ is a {\em nonnegative function}
  defined on $D_2$, then $\int \int_{D_1} f(x,y) \, dA \le \int
  \int_{D_2} f(x,y) \, dA$: This last one is important because it
  means that to calculate integrals over an irregularly shaped region,
  we can bound from above and below by calculating integrals over a
  region contained inside it and over a region containing it.
\item If $D = D_1 \cup D_2$ and $D_1 \cap D_2$ is one-dimensional,
  then $\int \int_D f(x,y) dA = \int \int_{D_1} f(x,y) \, dA +
  \int \int_{D_2} f(x,y) \, dA$ (this was already discussed earlier).
\item The integral of the function $1$ over a domain $D$ is the area of $D$.
\item If $f(x,y)$ on a domain $D$ is bounded from above and below by
  $M$ and $m$ respectively, and $D$ has area $A$, then the integral of
  $f(x,y)$ over $D$ is between $mA$ and $MA$.
\end{itemize}

\subsection{Symmetry-based ideas}

These all build on the corresponding symmetry-based ideas for
functions of one variable:

\begin{itemize}
\item If $f$ is odd in the variable $x$, and the domain of integration
  is symmetric about the $y$-axis, then the integral is zero: If we do
  integration using horizontal slices, we see that each horizontal
  slice integrates to zero, so the overall integral is zero.

  Note that for a multiplicatively separable function, what matters is
  that the part depending on $x$ be odd, and the part depending on $y$
  does not matter.
\item If $f$ is odd in the variable $y$, and the domain of integration
  is symmetric about the $x$-axis, then the integral is zero: If we do
  integration using vertical slices, we see that each vertical slice
  integrates to zero, so the overall integral is zero.

  Note that for a multiplicatively separable function, what matters is
  that the part depending on $y$ be odd, and the part depending on $x$
  does not matter.
\end{itemize}

Often, a given function can be expressed as a sum of functions some of
which are odd in $x$ or in $y$, and hence, using symmetry of domain,
can be declared to be zero. Others may need to be computed.

Consider, for instance, the case $f(x,y) = x^3y^2 + \ln(x^2 + x +
1)\sin(y^3)$, being integrated over the circular disk $x^2 + y^2 \le
1$. Note that $f$ as given is not odd in either variable. However, it
is the sum of the functions $x^3y^2$ (which is odd in $x$) and
$\ln(x^2 + x + 1)\sin(y^3)$ (which is odd in $y$). Moreover, the
domain is symmetric about both axes. Thus, the integral for both these
functions is zero, hence the overall integral for $f$ is zero.


\end{document}
