\documentclass[10pt]{amsart}

%Packages in use
\usepackage{fullpage, hyperref, vipul, enumerate}

%Title details
\title{Class quiz: Wednesday February 6: Multivariable limit computations}
\author{Math 195, Section 59 (Vipul Naik)}
%List of new commands

\begin{document}
\maketitle

Your name (print clearly in capital letters): $\underline{\qquad\qquad\qquad\qquad\qquad\qquad\qquad\qquad\qquad\qquad}$

{\bf PLEASE FEEL FREE TO DISCUSS ALL QUESTIONS, BUT PLEASE ONLY ENTER
FINAL ANSWER OPTIONS THAT YOU PERSONALLY CONSIDER MOST LIKELY TO BE
CORRECT. DO NOT ENGAGE IN GROUPTHINK.}

\begin{enumerate}
\item (**) Consider the function $f(x,y) := x \sin(1/(x^2 + y^2))$,
  defined on all points other than the point $(0,0)$. What is the
  limit of the function at $(0,0)$? {\em Last time: $8/22$ correct}

  \begin{enumerate}[(A)]
  \item $0$
  \item $1/\sqrt{2}$
  \item $1$
  \item The limit is undefined, because the expression becomes unbounded around $0$.
  \item The limit is undefined, because the expression is oscillatory
    around $0$.
  \end{enumerate}

  \vspace{0.1in}
  Your answer: $\underline{\qquad\qquad\qquad\qquad\qquad\qquad\qquad}$
  \vspace{0.1in}

\item The typical $\varepsilon-\delta$ definition of limit in two
  dimensions makes use of open disks centered at the points on the
  domain and range side, where the open disk is the interior region
  bounded by a circle centered at the point. Which other geometric
  shapes can we use instead of a circle of specified radius centered
  at the point? Please see Options (D) and (E) before answering and
  make the most appropriate selection. {\em Last time: $12/22$
  correct}.

  \begin{enumerate}[(A)]
  \item A square of specified side length centered at the point
  \item An equilateral triangle of specified side length centered at the point
  \item A regular hexagon of specified side length centered at the point
  \item Any of the above
  \item None of the above
  \end{enumerate}

  \vspace{0.1in}
  Your answer: $\underline{\qquad\qquad\qquad\qquad\qquad\qquad\qquad}$
  \vspace{0.1in}

\item Here's a quick recap of the limit definition for a function of a
  vector variable. We say that $\displaystyle \lim_{\mathbf{x} \to \mathbf{c}}
  f(\mathbf{x}) = L$ if for every $\varepsilon > 0$ there exists
  $\delta > 0$ such that for all $\mathbf{x}$ satisfying $0 <
  |\mathbf{x} - \mathbf{c}| < \delta$, we have $|f(\mathbf{x}) - L| <
  \varepsilon$. We define $|\mathbf{x} - \mathbf{c}|$ as the Euclidean
  norm of $\mathbf{x} - \mathbf{c}$ where the Euclidean norm of a
  vector is the square root of the sum of the squares of its
  coordinates.

  We could replace the Euclidean norm by other measurements. For instance, we could use:

  (i) The {\em sum} of the absolute values of the coordinates of
  $\mathbf{x} - \mathbf{c}$.

  (ii) The {\em maximum} of the absolute values of the coordinates of
  $\mathbf{x} - \mathbf{c}$.

  (iii) The {\em minimum} of the absolute values of the coordinates of
  $\mathbf{x} - \mathbf{c}$.

  For any of (i) - (iii), we could replace $|\mathbf{x} - \mathbf{c}|$
  in our current definition of limit with that notion. The question
  is: for which of the replacements will our new notion of limit be
  the same as the old one? The deeper idea here is that limit depends
  upon a concept of what it means for two points to be close. So
  another way of phrasing the question is: which of the notions
  (i)-(iii) capture the same notion of closeness as the usual
  Euclidean distance?

  \begin{enumerate}[(A)]
  \item All of (i), (ii), and (iii).
  \item (i) and (ii) but not (iii).
  \item (i) and (iii) but not (ii).
  \item Only (i).
  \item None of (i), (ii), or (iii).
  \end{enumerate}

  \vspace{0.1in}
  Your answer: $\underline{\qquad\qquad\qquad\qquad\qquad\qquad\qquad}$
  \vspace{0.1in}

\item Suppose $f$ is a function of two variables $x,y$ and is defined
  on the whole $xy$-plane. Consider three conditions: (i) $f$ is
  continuous on the whole $xy$-plane, (ii) for every fixed value $x =
  x_0$, the function $y \mapsto f(x_0,y)$ is continuous in $y$ for all
  $y \in \R$, (iii) for every fixed value $y = y_0$, the function $x
  \mapsto f(x,y_0)$ is continuous in $x$ for all $x \in \R$, (iv) the
  function $t \mapsto f(p(t),q(t))$ is continuous for all $t \in \R$
  whenever $p$ and $q$ are both constant or linear functions (in other
  words, the restriction of $f$ to any straight line in $\R^2$ is
  continuous).

  Which of the following correctly describes the implications between
  (i), (ii), (iii), and (iv)?

  \begin{enumerate}[(A)]
  \item (i) implies both (ii) and (iii), and (ii) and (iii) together imply (iv).
  \item (i) implies (iv), and (iv) implies both (ii) and (iii).
  \item (iv) implies (ii) and (iii), and (ii) and (iii) together imply (i).
  \item (iv) implies (i), and (i) implies both (ii) and (iii).
  \item (ii) and (iii) together imply (iv), and (iv) implies (i).
  \end{enumerate}
  
  \vspace{0.1in}
  Your answer: $\underline{\qquad\qquad\qquad\qquad\qquad\qquad\qquad}$
  \vspace{0.1in}

\end{enumerate}

\end{document}