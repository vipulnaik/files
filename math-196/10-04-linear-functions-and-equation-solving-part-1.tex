\documentclass[10pt]{amsart}

%Packages in use
\usepackage{fullpage, hyperref, vipul, enumerate}

%Title details
\title{Take-home class quiz: due Friday October 4: Linear functions and equation-solving (part 1)}
\author{Math 196, Section 57 (Vipul Naik)}
%List of new commands

\begin{document}
\maketitle

Your name (print clearly in capital letters): $\underline{\qquad\qquad\qquad\qquad\qquad\qquad\qquad\qquad\qquad\qquad}$

{\bf PLEASE DO {\em NOT} DISCUSS ANY QUESTIONS EXCEPT THE STARRED OR DOUBLE-STARRED QUESTIONS. YOU CAN DISCUSS THE STARRED AND DOUBLE-STARRED QUESTIONS.}

This quiz covers some basics involving linear functions and
equation-solving (notes at {\tt Linear functions: a primer} and {\tt
  Equation-solving with a special focus on the linear case}). The quiz
tests for the following:

\begin{itemize}
\item What it means to be (affine) linear, and in particular, the
  significance of the intercept as an additional parameter to track.
\item The distinction between behavior relative to the variables (the
  inputs) and behavior relative to the parameters.
\item Using the linear paradigm to study functional forms that are not
  themselves linear.
\item A small taste of dealing with measurement uncertainty to obtain
  upper and lower bounds (not covered in the notes, so this is where
  your famed ability to think out of the box should manifest).
\item Solving ``triangular'' systems of equations.
\end{itemize}

\begin{enumerate}
\item (*) A function $f$ of $3$ variables $x$, $y$, $z$ defined
  everywhere is (affine) linear in the variables. (The ``affine'' is
  to indicate that the intercept may be nonzero). Based on the above
  information and some input-output pairs for $f$, we would like to
  determine $f$ uniquely. What is the minimum number of input-output
  pairs that we would need in order to achieve this?

  \begin{enumerate}[(A)]
  \item $1$
  \item $2$
  \item $3$
  \item $4$
  \item $5$
  \end{enumerate}

  \vspace{0.1in}
  Your answer: $\underline{\qquad\qquad\qquad\qquad\qquad\qquad\qquad}$
  \vspace{0.6in}

\item {\em Do not discuss this!}: Which of the following gives an
  example of a function $F$ of three variables $x,y,z$ whose
  third-order mixed partial derivative $F_{xyz}$ is zero everywhere,
  but for which none of the second-order mixed partial derivatives
  $F_{xy}$, $F_{xz}$, $F_{yz}$ is zero everywhere?

  \begin{enumerate}[(A)]
  \item $\sin(xy) - z^2$
  \item $\cos(x^2 + y^2) - \sin(y^2 + z^2)$
  \item $e^{xy} + (y - z)^2 + 3xz$
  \item $x^2 + y^2 + z^2$
  \item $xyz$
  \end{enumerate}

  \vspace{0.1in}
  Your answer: $\underline{\qquad\qquad\qquad\qquad\qquad\qquad\qquad}$
  \vspace{0.6in}

\item {\em Do not discuss this!}: Consider a function of the form
  $F(x,y) := Ca^xb^y$ where $C,a,b$ are all positive reals that serve
  as parameters and $x,y$ are restricted to the positive reals. We
  wish to study $F$ using the paradigm of linear functions. What is
  the best way of doing this?

  \begin{enumerate}[(A)]
  \item Express $\ln(F(x,y))$ in terms of $\ln x$ and $\ln y$
  \item Express $\ln(F(x,y))$ in terms of $x$ and $y$
  \item Express $F(x,y)$ in terms of $\ln x$ and $\ln y$
  \item Express $\ln(F(x,y))$ in terms of $a^x$ and $b^y$
  \item Express $F(x,y)$ in terms of $a^x$ and $b^y$
  \end{enumerate}

  \vspace{0.1in}
  Your answer: $\underline{\qquad\qquad\qquad\qquad\qquad\qquad\qquad}$
  \vspace{0.6in}

\item {\em Do not discuss this!}: Consider a function of the form
  $F(x,y) := Cx^ay^b$ where $C,a,b$ are all positive reals that serve
  as parameters and $x,y$ are restricted to the positive reals. We
  wish to study $F$ using the paradigm of linear functions. What is
  the best way of doing this?

  \begin{enumerate}[(A)]
  \item Express $\ln(F(x,y))$ in terms of $\ln x$ and $\ln y$
  \item Express $\ln(F(x,y))$ in terms of $x$ and $y$
  \item Express $F(x,y)$ in terms of $\ln x$ and $\ln y$
  \item Express $\ln(F(x,y))$ in terms of $x^a$ and $y^b$
  \item Express $F(x,y)$ in terms of $x^a$ and $y^b$
  \end{enumerate}

  \vspace{0.1in}
  Your answer: $\underline{\qquad\qquad\qquad\qquad\qquad\qquad\qquad}$
  \vspace{0.6in}

\item (**) {\em This is a hard question!} The population in the island
  of Andrognesia as a function of time is believed to be an
  exponential function. On January 1, 1984, the population was
  measured to be $3 * 10^5$ with a measurement error of up to $10^5$
  on either side, i.e., the population was measured to be between $2*
  10^5$ and $4 * 10^5$. On January 1, 1998, the population was
  measured to be $1.2 * 10^6$ with a measurememt error of up to $4 *
  10^5$ on either side, i.e., the population was measured to be
  between $8 * 10^5$ and $1.6 * 10^6$. If the population is an
  exponential function of time (i.e., the increment in population per
  year is a fixed proportion of the population that year), what is the
  {\bf range of possible values} of the population measured on January
  1, 2012? {\em Hint: Think of the umbral versus penumbral region for
    an eclipse}

  \begin{enumerate}[(A)]
  \item Between $3.2 * 10^6$ and $6.4 * 10^6$
  \item Between $3.2 * 10^6$ and $1.28 * 10^7$
  \item Between $1.6 * 10^6$ and $3.2 * 10^6$
  \item Between $1.6 * 10^6$ and $6.4 * 10^6$
  \item Between $1.6 * 10^6$ and $1.28 * 10^7$
  \end{enumerate}

  \vspace{0.1in}
  Your answer: $\underline{\qquad\qquad\qquad\qquad\qquad\qquad\qquad}$
  \vspace{0.6in}

\item {\em Do not discuss this!}: Suppose, according to our model, a
  particular function $f(x,y)$ is of the form $f(x,y) = a_1 + a_2x +
  a_3y + a_4x^2y^2$ where $a_1,a_2,a_3,a_4$ are parameters. Our goal
  is to determine the values of the parameters $a_1,a_2,a_3,a_4$. We
  do this by collecting a number of (input,output) pairs for the
  function $f$ and then setting up equations in terms of the
  parameters using the (input,output) pairs. What can we say about the
  nature of $f$ and the nature of the system of equations that we will
  need to solve?  {\em Note that ``nonlinear'' as used here simply
    means that the expression is not guaranteed to be linear, though
    it may turn out to be linear in some cases. Similarly,
    ``non-polynomial'' means not guaranteed to be polynomial, though
    it may turn out to be polynomial in some cases.}

  \begin{enumerate}[(A)]
  \item $f$ is a linear function of $x$ and $y$, hence we need to
    solve a linear system of equations to determine the parameters
    $a_1,a_2,a_3,a_4$.
  \item $f$ is a nonlinear polynomial function of $x$ and $y$, hence
    we need to solve a nonlinear polynomial system of equations to
    determine the parameters $a_1,a_2,a_3,a_4$.
  \item $f$ is a linear function of $x$ and $y$. However, we need to
    solve a nonlinear polynomial system of equations to determine the
    parameters $a_1,a_2,a_3,a_4$.
  \item $f$ is a nonlinear polynomial function of $x$ and
    $y$. However, we need to solve a linear system of equations to
    determine the parameters $a_1,a_2,a_3,a_4$.
  \item $f$ is a nonlinear polynomial function of $x$ and
    $y$. However, we need to solve a non-polynomial system of
    equations to determine the parameters $a_1,a_2,a_3,a_4$.
  \end{enumerate}

  \vspace{0.1in}
  Your answer: $\underline{\qquad\qquad\qquad\qquad\qquad\qquad\qquad}$
  \vspace{0.6in}

\item {\em Do not discuss this!}: Consider the system of equations:

  \begin{eqnarray*}
    x^2 - x & = & 2 \\
    y^2 + xy & = & x + 13 \\
  \end{eqnarray*}

  What is the number of solutions to this system for real $x$ and $y$?

  \begin{enumerate}[(A)]
  \item $0$
  \item $2$
  \item $4$
  \item $6$
  \item $8$
  \end{enumerate}

  \vspace{0.1in}
  Your answer: $\underline{\qquad\qquad\qquad\qquad\qquad\qquad\qquad}$
  \vspace{0.6in}

\item {\em Do not discuss this!}: Consider the system of equations:

  \begin{eqnarray*}
    x^2 - x & = & 2 \\
    y^2 + xy & = & x + 13 \\
    z^2 & = & xy \\
  \end{eqnarray*}

  What is the number of solutions to this system for real $x$, $y$,
  and $z$?

  \begin{enumerate}[(A)]
  \item $0$
  \item $2$
  \item $4$
  \item $6$
  \item $8$
  \end{enumerate}

  \vspace{0.1in}
  Your answer: $\underline{\qquad\qquad\qquad\qquad\qquad\qquad\qquad}$
  \vspace{0.6in}

\item {\em Do not discuss this!}: Consider the system of equations:

  \begin{eqnarray*}
    x^2 - x & = & 2 \\
    y^2 + xy & = & x + 13 \\
    z^2 & = & x^2 - y^2 \\
  \end{eqnarray*}

  What is the number of solutions to this system for real $x$, $y$,
  and $z$?

  \begin{enumerate}[(A)]
  \item $0$
  \item $2$
  \item $4$
  \item $6$
  \item $8$
  \end{enumerate}

  \vspace{0.1in}
  Your answer: $\underline{\qquad\qquad\qquad\qquad\qquad\qquad\qquad}$
  \vspace{0.6in}

\end{enumerate}

\end{document}
