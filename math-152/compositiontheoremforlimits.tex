\documentclass[10pt]{amsart}
\usepackage{fullpage,hyperref,vipul, graphicx}
\title{Composition theorem for limits}
\author{Math 152, Section 55 (Vipul Naik)}

\begin{document}
\maketitle

There is a composition theorem for continuous functions: if $g$ is
continuous at $c$ and $f$ is continuous at $g(c)$, then $f \circ g$ is
continuous at $c$.

We might suspect an analogous composition theorem for limits: if
$\lim_{x \to c} g(x) = L$ and $\lim_{v \to L} f(v) = M$, then $\lim_{x
\to c} f(g(x)) = M$.

However, the composition theorem for limits as stated above is not
strictly true.

The reason is very subtle, though it is in some sense similar to why
composites do not hold in the one-sided sense. It all has to do with
the $0 <$ in the limit definition, or, the fact that when we say
$\lim_{x \to c} g(x)$, we exclude the behavior at the point $c$.

Let's consider this more carefully. If $\lim_{x \to c} g(x) = L$, then
that means that as $x \to c$, then $g(x) \to L$. Next, we know that as
$v \to L$, then $f(v) \to M$. So, why doesn't it follow that as $x \to
c$, $f(g(x)) \to M$?

In words:

$x \to c$ implies $g(x) \to L$

and:

$v \to L$ implies $f(v) \to M$

Plugging $v = g(x)$, why doesn't the result follow?

The issue is that the $g(x) \to L$ on the right half of the first line
differs from the $v \to L$ on the left half of the second line. The
former conclusion is valid, for instance, if $g(x)$ is a constant
function $L$. On the other hand the $v \to L$ specifically includes
only a straightforward approach from the left or the right.

Thus, we can construct a counterexample to the ``composition theorem
for limits'' with the following features: (i) $g$ is constant around
$c$ with the value $L$ and (ii) $f$ is defined at $c$ but has a
removable discontinuity at $c$.

\end{document}
