\section{Background definitions and basic theory}\label{appsec:background-grouptheory}

\subsection{Rings and associativity}

A {\em ring} is an abelian group $R$ (with the abelian group operation
denoted additively) equipped with a $\mathbb{Z}$-bilinear operation $R
\times R \to R$ called the ring multiplication. Here,
``$\mathbb{Z}$-bilinear'' is equivalent to the left and right
distributivity laws. The multiplication operation of a ring may be
denoted by an explicit multiplicative symbol such as $*$ or $\cdot$
but it will often be denoted by concatenation (also known as
juxtaposition), i.e., we will denote by $ab$ the image of $(a,b)$.

An {\em associative} ring is a ring where the multiplication is
associative. With the notation above, $R$ is associative if the
following holds:

$$(ab)c = a(bc) \ \forall \ a,b,c \in R$$

Note that we do {\em not} assume associativity as part of the
definition of ring. Thus, our definition of ring includes associative
rings, Lie rings, and other kinds of rings. The use of concatenation
can be misleading in the non-associative case, because the string
$abc$ has two differing interpretations: $(ab)c$ versus $a(bc)$. When
considering products of length more than two in a non-associative
ring, we must take care to use parentheses to clarify the order of
operations.

Note that when we use the term ``commutative unital ring'' we are by
default referring to rings that are both commutative and associative
 and where the ring multiplication has an identity element, that we
denote as $1$.

\subsection{Algebras over a commutative unital ring}

Suppose $R$ is a commutative (associative) unital ring. An {\em algebra
  over $R$}, also called a $R$-algebra, is a $R$-module $A$ equipped
with a $R$-bilinear operation $A \times A \to A$ called the
multiplication or product. Note that any algebra over $R$ is naturally
also a ring. If the multiplication of $A$ is associative, we say that
$A$ is an {\em associative algebra} over $R$. We can in fact define a
ring as a $\Z$-algebra.

\subsection{Structure constants and multilinear identities}

Suppose $R$ is a commutative (associative) unital ring and $A$, $B$,
$C$ are $R$-modules, with $f:A \times B \to C$ a $R$-bilinear
map. Suppose $(a_i), (b_j),(c_k)$ form generating sets for $A$, $B$,
and $C$ as $R$-modules. We can find values $\lambda_{ij}^k \in R$,
called the {\em structure constants} of $f$, such that the following
holds for all $i,j$:

$$f(a_i,b_j) = \sum_k \lambda_{ij}^k c_k$$

Note that the structure constants need not be uniquely determined by
the knowledge of $f$ and the generating sets. However, in the case
that $C$ is a free $R$-module and the $(c_k)$ form a freely generating
set for $C$, the structure constants are uniquely determined by $f$
and the choice of generating sets. The converse is always true: if we
are given the structure constants and the generating sets, $f$ is
uniquely determined.

The typical case where we talk of structure constants is where $A$,
$B$, and $C$ are all free $R$-modules and each of the generating sets
is a {\em freely generating set}. In this case, {\em every} possible
tuple of values for $\lambda_{ij}^k$ can be realized via a bilinear
map. A further special case of this is where $R$ is a field. In this
case, all $R$-modules are free (explicitly, they are vector spaces
over the field, and each generating set is a basis for the
corresponding vector space).

In particular, given an algebra $A$ over a commutative unital ring $R$
such that the additive group of $A$ is a free $R$-module, we can
describe $A$ by choosing a freely generating set for it and then using
structure constants for the multiplication of $A$. In this case, we
will call the structure constants for the multiplication of $A$ the
structure constants of $A$.

\subsection{Lie rings and Lie algebras}\label{appsec:Lie}
A {\em Lie ring} is defined as an abelian group $L$ equipped with a
$\mathbb{Z}$-bilinear map $[ \ , \ ]:L \times L \to L$ called the {\em
  Lie bracket} satisfying the following conditions:

\begin{itemize}
\item {\em Alternating property}: $[x,x] = 0$ for all $x \in L$. It
  also follows from this that $[x,y] = -[y,x]$ for all $x,y \in
  L$. The latter condition is termed {\em skew symmetry}. The proof of
  the implication uses that $[ \ , \ ]$ is $\Z$-bilinear. Explicitly,
  the proof uses $\Z$-bilinearity to show that $[x,y] + [y,x] =
  [x+y,x+y] - [x,x] - [y,y]$, and we then note that the right side is
  zero by the alternating property.

  The converse implication does not hold, i.e., skew symmetry does not
  imply the alternating property, although it does so if $L$ is
  $2$-torsion-free. Explicitly, $[x,y] = -[y,x]$ for all $x,y$ implies
  that $[x,x] = -[x,x]$ for all $x$, so $2[x,x] = 0$. In the case that
  $L$ is torsion-free, this implies that $[x,x] = 0$.
\item {\em Jacobi identity}: There are two versions, both of which are
  equivalent under skew symmetry (and hence under the alternating
  property):
  \begin{itemize}
  \item {\em Left normed Jacobi identity}: $[[x,y],z] + [[y,z],x] +
    [[z,x],y] = 0$ for all $x,y,z \in L$.
  \item {\em Right normed Jacobi identity}: $[x,[y,z]] + [y,[z,x]] +
    [z,[x,y]] = 0$ for all $x,y,z \in L$.
  \end{itemize}
\end{itemize}

Let $R$ be a commutative (associative) unital ring. A {\em Lie
  algebra} over $R$ is a $R$-module $L$ equipped with a $R$-bilinear
map $[ \ , \ ]:L \times L \to L$ called the {\em Lie bracket}
satisfying the two conditions above, namely, the alternating property
and the Jacobi identity.

Note in particular that any $R$-Lie algebra can naturally be viewed as
a Lie ring. A Lie ring can be defined as a $\Z$-Lie algebra.

The following additional definitions are useful:

\begin{itemize}
\item {\em Subring of a Lie ring}: A subset $S$ of a Lie ring is
  termed a {\em subring} (or {\em Lie subring}) if it is an additive
  subgroup of $L$ and is closed under the Lie bracket, i.e., $[x,y]
  \in S$ for all $x,y \in S$.
\item {\em Ideal of a Lie ring}: A subset $I$ of a Lie ring $L$ is
  termed an {\em ideal} if $I$ is an additive subgroup of $L$ and
  $[x,y] \in I$ for all $x \in I$, $y \in L$.

  Any ideal is a subring, but a subring need not be an ideal.
\item {\em Homomorphism of Lie rings}: Given Lie ring $L_1$ and $L_2$,
  a {\em homomorphism} from $L_1$ to $L_2$ is a set map $\varphi:L_1
  \to L_2$ that is a homomorphism between $L_1$ and $L_2$ as groups
  and such that $\varphi([x,y]) = [\varphi(x),\varphi(y)]$ for all $x,y \in L$.
\item {\em Quotient Lie ring}: Given a Lie ring $L$ and an ideal $I$
  of $L$, the quotient group $L/I$ naturally inherits the structure of
  a Lie ring from $L$.
\end{itemize}

We can also define corresponding notions of subalgebra of a Lie
algebra, ideal of a Lie algebra, homomorphism of Lie algebras, and
quotient Lie algebra, all in the context of a $R$-Lie algebra for a
commutative (associative) unital ring $R$.

There are analogues in Lie rings (and also in $R$-Lie algebras) of the
four isomorphism theorems for groups.

\subsection{The relationship between Lie rings and Lie algebras}

Most of the basic ideas and definitions associated with Lie rings
(that we can think of as $\Z$-Lie algebras) generalize to $R$-Lie
algebras for any commutative (associative) unital ring $R$. In the
Appendix, we discuss how the definitions for Lie rings relate to
corresponding definitions for $R$-Lie algebras. However, all the
definitions and results in the main document are framed in terms of
Lie rings. 

Many of the results in the main document generalize to Lie
algebras. However, to correctly generalize the main results (which are
about correspondences between groups and Lie rings) we need to develop
a concept of a group powered over an arbitrary commutative
(associative) unital ring. %%TONOTDO: Link to appendix section

\subsection{Verification of multilinear identities: associativity and the Jacobi identity}\label{appsec:multilinear}

Suppose $R$ is a commutative unital ring and $M$ is a $R$-module. We
use the term $R$-{\em multilinear identity} for an identity of the
form:

$$F(x_1,x_2,\dots,x_n) = 0$$

where $F: M \times M \times \dots \times M \to M$ is $R$-linear in
each coordinate. We say that the identity holds in $M$ if the above
holds for all $x_1,x_2,\dots,x_n \in M$, i.e., $F$ is identically the
zero function.

For any commutative unital ring $R$ and any $R$-algebra $A$,
associativity of $A$ is a $R$-multilinear identity (that may or may
not hold for $A$). Similarly, the left-normed Jacobi identity for $A$
is a $R$-multilinear identity (that may or may not hold for $A$).

For any $R$-multilinear identity being considered on a $R$-module $M$,
the following are true:

\begin{itemize}
\item It suffices to verify the identity on a generating set for $M$
  as a $R$-module, i.e., it suffices to verify the identity for each
  $x_i$ varying arbitrarily over a generating set for $M$ as a
  $R$-module.
\item In the case that $M$ is a free $R$-module and the multilinear
  operation $F$ is built from a $R$-bilinear operation $f: M \times M
  \to M$, the multilinear identity reduces to a polynomial condition
  in terms of the structure constants for $f$ (this will be clearer
  from the examples below).
\end{itemize}

Suppose $R$ is a commutative unital ring and $A$ is a $R$-algebra with
multiplication $*$, i.e., $A$ is a $R$-module and $*:A \times A \to A$
is a $R$-bilinear map. The {\em associator} of $*$ is a $R$-trilinear
function $a: A \times A \times A \to A$ defined as:

$$a(x,y,z) := ((x * y) * z) - (x * (y * z))$$

$*$ is associative if and only if $a$ is the zero function. In order
to verify that $a$ is the zero function, it suffices to choose a
generating set $S$ for $A$ as a $R$-module and then verify that
$a(x,y,z) = 0$ for all $x,y,z \in S$.

In the case that $A$ is a free $R$-module with freely generating set
$e_i, i \in I$ and structure constants $\lambda_{ij}^k$ for the
multiplication, the associativity condition can be written as follows,
for all $i,j,k,l \in I$:

$$\sum_{m \in I} \lambda_{ij}^m\lambda_{mk}^l = \sum_{m \in I} \lambda_{im}^l \lambda_{jk}^m$$

Suppose $R$ is a commutative unital ring and $L$ is a $R$-algebra with
multiplication denoted by a bracket $[ \ , \ ]$. The left-normed and
right-normed Jacobi identities respectively mean that the following
$R$-trilinear functions are zero everywhere:

\begin{eqnarray*}
  J_l(x,y,z) & = & [[x,y],z] + [[y,z],x] + [[z,x],y]\\
  J_r(x,y,z) & = & [x,[y,z]] + [y,[z,x]] + [z,[x,y]]\\
\end{eqnarray*}

In order to verify either of these identities, it suffices to verify
it on a generating set for $L$ as a $R$-module. Explicitly, if $S$
generates $L$ as a $R$-module, then $J_l(x,y,z) = 0$ for all $x,y,z
\in L$ if and only if $J_l(x,y,z) = 0$ for all $x,y,z \in
S$. Similarly, $J_r(x,y,z) = 0$ for all $x,y,z \in L$ if and only if
$J_r(x,y,z) = 0$ for all $x,y,z \in S$.

In the case that $L$ is free as a $R$-module, the left-normed and
right-normed Jacobi identities can be verified in terms of the
structure constants. Explicitly, if $e_i, i \in I$ form a freely
generating set for $L$, and $\lambda_{ij}^k$ denote the structure
constants, then the following hold:

\begin{itemize}
\item The left-normed Jacobi identity is equivalent to the condition that:

  $$\sum_{m \in I} (\lambda_{ij}^m\lambda_{mk}^l + \lambda_{jk}^m\lambda_{mi}^l + \lambda_{ki}^m\lambda_{mj}^l) = 0$$

\item The right-normed Jacobi identity is equivalent to the condition
  that:

  $$\sum_{m \in I} (\lambda_{im}^l\lambda_{jk}^m + \lambda_{jm}^l\lambda_{ki}^m + \lambda_{km}^l\lambda_{ij}^m) = 0$$
\end{itemize}

\subsection{Derivation of a ring and of an algebra}\label{appsec:derivation}

We define derivations for rings, and also for algebras for commutative
unital rings.

\begin{definer}[Derivation]
  Suppose $A$ is a ring with multiplication denoted by $*$. A set map
  $d:A \to A$ is termed a {\em derivation} of $A$ if it satisfies {\em
    both} these conditions:

  \begin{itemize}
    \item $d$ is an endomorphism of the additive group of $A$.
    \item $d$ satisfies the following condition, called the {\em
      Leibniz condition}:

      $$d(x * y) = (d(x) * y) + (x * d(y)) \ \forall \ x,y \in R$$

      Note that we typically do not use parentheses for the inputs to
      derivations if they are single letters, so the above is often written as:

      $$d(x * y) = (dx * y) + (x * dy) \ \forall \ x,y \in R$$
  \end{itemize}
\end{definer}

The set of derivations of a ring forms a Lie ring, where the addition
is pointwise, and the Lie bracket is defined as $[d_1,d_2] = d_1 \circ
d_2 - d_2 \circ d_1$ (with $\circ$ denoting composition of set
maps). For a ring $A$, we will denote the Lie ring of derivations of
$A$ by $\operatorname{Der}(A)$.

We now consider the definition of derivation for an algebra over a
commutative unital ring.

\begin{definer}[Derivation of an algebra]
  Suppose $R$ is a commutative unital ring and $A$ is a $R$-algebra. A
  set map $d: A \to A$ is termed a derivation of $A$ as a $R$-algebra
  if $d$ is a $R$-module endomorphism of $A$ {\em and} satisfies the
  Leibniz condition (stated in the preceding definition).
\end{definer}

The set of derivations of a $R$-algebra forms a $R$-Lie algebra.
\subsection{Derivation of a Lie ring and of a Lie algebra}\label{appsec:derivation-lie}

Suppose $L$ is a Lie ring. For any $x \in L$, consider the set map
$\operatorname{ad}_x: L \to L$ given by $\operatorname{ad}_x(y) = [x,y]$. $\operatorname{ad}_x$ is termed the
{\em left adjoint map}, or simply the {\em adjoint map} corresponding
to $x$.

It follows from the distributivity of the Lie bracket that $\operatorname{ad}_x$ is
an additive group endomorphism, and the Jacobi identity along with
skew symmetry give us that $\operatorname{ad}_x$ satisfies the Leibniz
condition. Similarly, if $L$ is a Lie algebra over a commutative
unital ring $R$, then $\operatorname{ad}_x$ is a derivation in the $R$-algebra
sense.

The derivations of the form $\operatorname{ad}_x$ are termed {\em inner derivations}
of $L$. We can easily verify that two elements $x,y \in L$ satisfy
$\operatorname{ad}_x = \operatorname{ad}_y$ if and only if $\operatorname{ad}_{x-y} = 0$, and the set of values
$z \in L$ for which $\operatorname{ad}_z = 0$ is precisely the {\em center} of $L$
(defined as the set $ \{ z \in L \mid [z,y] = 0 \ \forall \ y \in L
\}$). We obtain the following:

\begin{itemize}
\item The inner derivations of $L$ form an ideal inside
  $\operatorname{Der}(L)$. We will denote this ideal by
  $\operatorname{Inn}(L)$.
\item The set map $x \mapsto \operatorname{ad}_x$ defines a Lie ring homomorphism
  from $L$ to $\operatorname{Der}(L)$ with image
  $\operatorname{Inn}(L)$ and kernel $Z(L)$. Thus, by the first
  isomorphism theorem, $L/Z(L)$ is canonically isomorphic to
  $\operatorname{Inn}(L)$.
\end{itemize}

%\newpage

\section{Abstract nonsense}\label{appsec:abstract-nonsense}

\subsection{Basic category theory: categories, functors, and natural transformations}\label{appsec:category}

We define here three foundational ideas of category theory:
categories, functors, and natural transformations.

\subsubsection{Definition of category}

A {\em category} $\mathcal{C}$ is the following data:

\begin{itemize}
\item {\em Objects}: A collection $\operatorname{Ob}\mathcal{C}$ of
  objects.\footnote{There are some set theory paradoxes due to which
    we are using the term ``collection'' rather than ``set'' here.}
\item {\em Morphisms}: For any objects $A,B \in
  \operatorname{Ob}\mathcal{C}$, a collection $\mathcal{C}(A,B)$ of
  morphisms. Every element in $\mathcal{C}(A,B)$ is termed a {\em
    morphism} from $A$ (i.e., with source or domain $A$) to $B$ (i.e.,
  with target or co-domain $B$). The morphism sets for different pairs
  of objects are disjoint. Note that $f \in \mathcal{C}(A,B)$ is also
  written as $f:A \to B$. The collection $\mathcal{C}(A,B)$ is
  sometimes also denoted $\operatorname{Hom}_{\mathcal{C}}(A,B)$ or
  simply $\operatorname{Hom}(A,B)$.
\item {\em Identity morphism}: For every object $A \in
  \operatorname{Ob}\mathcal{C}$, a distinguished morphism
  $\operatorname{id}_A \in \mathcal{C}(A,A)$. This is called the
  identity morphism of $A$.
\item {\em Composition rule}: For $A,B,C \in
  \operatorname{Ob}\mathcal{C}$, a map, called {\em composition of
  morphisms}, from $\mathcal{C}(B,C) \times \mathcal{C}(A,B)$ to
  $\mathcal{C}(A,C)$. This map is denoted by $\circ$.
\end{itemize}

satisfying the following compatibility conditions:

\begin{itemize}
\item {\em Associativity of composition}: For $A,B,C,D \in
  \operatorname{Ob}\mathcal{C}$, with $f \in \mathcal{C}(A,B), g \in
  \mathcal{C}(B,C), h \in \mathcal{C}(C,D)$, we have $h \circ (g \circ
  f) = (h \circ g) \circ f$.
\item {\em Identity behaves as an identity}: For $A,B \in
  \operatorname{Ob}\mathcal{C}$, with $f \in \mathcal{C}(A,B)$, we
  have $f \circ \operatorname{id}_A = \operatorname{id}_B \circ f =
  f$.
\end{itemize}

We will also use the following terms:

\begin{itemize}
\item For $A \in \operatorname{Ob}\mathcal{C}$, an {\em endomorphism}
  of $A$ is defined as an element of $\mathcal{C}(A,A)$. The
  endomorphisms of $A$ form a {\em monoid} under composition, denoted
  $\operatorname{End}_{\mathcal{C}}(A)$. The identity morphism
  $\operatorname{Id}_A$ is the identity element of this monoid.
\item For $A,B \in \operatorname{Ob}\mathcal{C}$, an {\em isomorphism}
  from $A$ to $B$ is defined as an element $f \in \mathcal{C}(A,B)$
  such that there exists an element $g \in \mathcal{C}(B,A)$ for which
  $g \circ f = \operatorname{Id}_A$ and $f \circ g =
  \operatorname{Id}_B$.
\item For $A \in \operatorname{Ob}\mathcal{C}$, an {\em automorphism}
  of $A$ is defined an an endomorphism of $A$ that is also an
  isomorphism. The automorphisms of $A$ form a {\em group} under
  composition, denoted $\operatorname{Aut}_{\mathcal{C}}(A)$. The
  identity morphism $\operatorname{Id}_A$ is the identity element of
  this group. $\operatorname{Aut}_{\mathcal{C}}(A)$ is the subgroup of
  $\operatorname{End}_{\mathcal{C}}(A)$ comprising all the elements
  with two-sided inverses.
\end{itemize}

The categories that we will consider in this document include:

\begin{itemize}
\item The category of groups, where the morphisms are group
  homomorphisms. The notions of endomorphism, isomorphism, and
  automorphism in this case coincide with our usual
  notions. Similarly, we consider the category of Lie rings where the
  morphisms are Lie ring homomorphisms. Note that these are the
  default category structures on group and Lie rings respectively.
\item The category of groups with homoclinisms, described in Section
  \ref{sec:isoclinism-and-homoclinism} (specifically, Section
  \ref{sec:homoclinism-category}). The isomorphisms in this category
  are isoclinisms of groups. Similarly, we consider the category of
  Lie rings with homoclinisms in Section
  \ref{sec:isoclinism-and-homoclinism-lie}.
\item The category of short exact sequences of groups, described in
  Section \ref{sec:ses-group}. Similarly, we consider the category of
  short exact sequences of Lie rings in Section \ref{sec:ses-lie}.
\item The category of central extensions of a group, described in
  Section \ref{sec:homomorphism-central-extensions}. Similarly, we
  consider the category of central extensions of a Lie ring in Section
  \ref{sec:homomorphism-central-extensions-lie}.
\item The category of central extensions of a group with homoclinisms,
  described in Section
  \ref{sec:homoclinism-central-extensions}. Similarly, we consider the
  category of central extensions of a Lie ring with homoclinisms in
  Section \ref{sec:homoclinism-central-extensions-lie}.
\item The category of $\pi$-powered groups for a prime set $\pi$,
  described in Section \ref{sec:group-powering}. Similarly, the
  category of $\pi$-powered Lie rings for a prime set $\pi$, described
  in Section \ref{sec:lie-ring-powering}.
\end{itemize}

\subsubsection{Definition of functor}\label{appsec:functor}

Suppose $\mathcal{C},\mathcal{D}$ are categories. A {\em functor}
(also called {\em covariant functor}) $\mathcal{F}$ from $\mathcal{C}$
to $\mathcal{D}$ comprises the following data:

\begin{itemize}
\item {\em Object level mapping}: A mapping $\mathcal{F}:
  \operatorname{Ob}\mathcal{C} \to \operatorname{Ob}\mathcal{D}$.
\item {\em Morphism level mapping}: For any $A,B \in \mathcal{C}$, a
  mapping $\mathcal{F}: \mathcal{C}(A,B) \to \mathcal{D}(\mathcal{F}A,\mathcal{F} B)$.
\end{itemize}

satisfying the following conditions:

\begin{itemize}
\item {\em It preserves the identity morphism}: For any $A \in
  \mathcal{C}$, $\mathcal{F}(\operatorname{id}_A) =
  \operatorname{id}_{\mathcal{F}(A)}$.
\item {\em It preserves composition}: For any $A,B,C \in \mathcal{C}$,
  and $f \in \mathcal{C}(A,B), g \in \mathcal{C}(B,C)$, we have $\mathcal{F}(g
  \circ f) = \mathcal{F}g \circ \mathcal{F}f$.
\end{itemize}

Suppose $\mathcal{C}, \mathcal{D}$ are categories. A {\em contravariant
functor} $\mathcal{F}:\mathcal{C} \to \mathcal{D}$ is defined by the
following data:

\begin{itemize}
\item A mapping $\mathcal{F}: \operatorname{Ob}\mathcal{C} \to
  \operatorname{Ob}\mathcal{D}$.
\item For every $A,B \in \operatorname{Ob}\mathcal{C}$, a mapping
  $\mathcal{F}: \mathcal{C}(A,B) \to
  \mathcal{D}(\mathcal{F}B,\mathcal{F}A)$.
\end{itemize}

satisfying the following conditions:

\begin{itemize}
\item It preserves the identity map: For any $A \in
  \operatorname{Ob}\mathcal{C}$, $\mathcal{F}(\operatorname{id}_A) =
  \operatorname{id}_{\mathcal{F}A}$.
\item It preserves composition, albeit reversing the order of
  composition: For any $A,B,C \in \operatorname{Ob}\mathcal{C}$, and
  $f \in \mathcal{C}(A,B), g \in \mathcal{C}(B,C)$, we have
  $\mathcal{F}(g \circ f) = \mathcal{F}f \circ \mathcal{F}g$.
\end{itemize}

We use the following terminology for functors. As above, let
$\mathcal{F}:\mathcal{C} \to \mathcal{D}$ be a functor.

\begin{itemize}
\item $\mathcal{F}$ is {\em faithful} if the induced set map
  $\mathcal{C}(A,B) \to \mathcal{D}(\mathcal{F}A,\mathcal{F}B)$ is
  injective for all $A,B \in \operatorname{Ob}\mathcal{C}$.
\item $\mathcal{F}$ is {\em full} if the induced set map
  $\mathcal{C}(A,B) \to \mathcal{D}(\mathcal{F}A,\mathcal{F}B)$ is
  surjective for all $A,B\in \operatorname{Ob}\mathcal{C}$.
\item $\mathcal{F}$ is {\em essentially surjective on objects} if
  every object of $\mathcal{D}$ is isomorphic in $\mathcal{D}$ to some
  object in the image of $\mathcal{F}$.
\item $\mathcal{F}$ is an {\em equivalence of categories} if it is
  full, faithful, and essentially surjective on objects. An
  equivalence of categories preserves many of the features we care
  about. In particular, it preserves the existence of initial objects
  and terminal objects, the nature of homomorphism sets, and the
  isomorphism types of endomorphism monoids and automorphism groups.
\item $\mathcal{F}$ is an {\em isomorphism of categories} if it is
  full, faithful, and bijective on objects.
\end{itemize}

We will use functors in one important way. For each of the
correspondences between suitably chosen groups and Lie rings, we will
first identify full subcategories of the category of groups and the
category of Lie rings respectively. Here, {\em full} subcategory means
a subcategory where the inclusion functor into the whole category is a
full functor. In other words, the full subcategory may not contain all
the objects of the big category, but given two objects of the big
category, it contains all the morphisms between them. Our
correspondence will then explicitly describe functors $\exp$ and
$\log$ between these full subcategories, where $\exp$ is from Lie
rings to groups and $\log$ is from groups to Lie rings. The functors
$\exp$ and $\log$ will turn out to be two-sided inverses of each
other, and the full subcategories will therefore turn out to be
isomorphic.

Section \ref{sec:abelian-lie-correspondence} describes all the details
for the abelian Lie correspondence. Similar ideas apply to the Baer
correspondence (described in Section
\ref{sec:baer-correspondence-isocat-consequences}), Malcev
correspondence, global Lazard correspondence, and Lazard
correspondence.

We also use the idea of equivalence of categories when relating two
alternate descriptions of the category of central extensions of a
group where the morphisms are homomorphisms of central
extensions. There are two alternative definitions of this category,
and these definitions do {\em not} define isomorphic
categories. However, they do define {\em equivalent} categories. In
fact, there is a forgetful functor from one category (that stores the
short exact sequence) to the other category (that simply stores the
quotient map of the short exact sequence) that serves as the
equivalence of categories. For more details, see Section
\ref{sec:homomorphism-central-extensions}.

Apart from the above fairly foundational uses of functors, we do not
use functors explicitly. 

However, many of the constructions we use
throughout the document are functorial, although we do not explicitly
use that fact. Some examples are below.

\begin{itemize}
\item The exterior square of a group, described in Section
  \ref{sec:exteriorsquare}, defines a functor from the category of
  groups to itself. In other words, given any homomorphism
  $\varphi:G_1 \to G_2$ of groups, there is an induced homomorphism
  $\varphi \wedge \varphi: G_1 \wedge G_1 \to G_2 \wedge G_2$
  satisfying the conditions for being a functor.
\item The Schur multiplier of a group, described in Section
  \ref{sec:exteriorsquare}, defines a functor from the category of
  groups to itself (or, alternatively, to the category of abelian
  groups).
\item The exterior square of a Lie ring, described in Section
  \ref{sec:exteriorsquare-lie}, defines a functor from the category
  of Lie rings to itself.
\item The Schur multiplier of a Lie ring, described in Section
  \ref{sec:exteriorsquare-lie}, defines a functor from the category of
  Lie rings to the category of abelian groups.
\item There are a number of free and forgetful functors we see. We
  discuss these a little later in the Appendix.
\end{itemize}

\subsubsection{Definition of natural transformation}

Given two functors $\mathcal{F},\mathcal{G}: \mathcal{C} \to
\mathcal{D}$, a {\em natural transformation} $\eta$ from $\mathcal{F}$
to $\mathcal{G}$ associates to each $A \in
\operatorname{Ob}\mathcal{C}$ a morphism $\eta_A: \mathcal{F}A \to
\mathcal{G}A$ such that for every morphism $f \in \mathcal{C}(A,B)$
for $A,B \in \operatorname{Ob}\mathcal{C}$, we have:

$$\eta_B \circ \mathcal{F}(f) = \mathcal{G}(f) \circ \eta_A$$

Equivalently, the following diagram commutes:

$$\begin{array}{llll}
\mathcal{F}(A) & \stackrel{\mathcal{F}(f)}{\to} & \mathcal{F}(B)\\
\eta_A \downarrow & & \eta_B \downarrow \\
\mathcal{G}(A) & \stackrel{\mathcal{G}(f)}{\to} & \mathcal{G}(B)\\
\end{array}$$

We call $\eta$ a {\em natural isomorphism} if $\eta_A$ is an
isomorphism in $\mathcal{D}$ for each $A \in
\operatorname{Ob}\mathcal{C}$.

We can also define an analogous concept of natural
transformation between contravariant functors.

%%TONOTDO: Say more clearly: for something that depends on multiple
%%variables, we can talk of being natural in some of the variables
%%(paves the way for the definition of left and right adjoint functors)

In our main proofs, we often use the term {\em canonical}. The
definition of {\em canonical transformation} is similar to that of
natural transformation, except that we impose the above condition only
when $f$ is an isomorphism. In other words, natural means that the
transformation behaves well (in the sense of giving a commutative
diagram) for {\em all} morphisms, whereas canonical means that the
transformation behaves well for {\em isomorphisms}. A {\em canonical
  isomorphism} is a canonical transformation that is an isomorphism in
every instance.

It turns out that most of the canonical constructions that we use in
our main proofs are also natural. This is not hard to prove, but we do
not demonstrate it because it is not necessary for our main proofs. We
list below some important instances of this.

\begin{itemize}
\item In Section \ref{sec:exteriorsquare}, we define the following
  short exact sequence associated with any group $G$:

  $$0 \to M(G) \to G \wedge G \to [G,G] \to 1$$

  All the groups appearing in the short exact sequence are functorial
  in $G$, and all the morphisms appearing in the short exact sequence
  are natural in $G$, i.e., they define natural transformation between
  the functors. Thus, we can view the short exact sequence itself as
  functorial in $G$, i.e., we can define a functor from the category
  of groups to the category of short exact sequences of groups that
  sends a group $G$ to the above short exact sequence.

  A similar observation applies to the corresponding short exact
  sequence for Lie rings described in Section
  \ref{sec:exteriorsquare-lie}.

\item In Section \ref{sec:ses-uct}, we define the following short
  exact sequence associated with a group $G$ and an abelian group $A$:

  $$0 \to \operatorname{Ext}^1_{\mathbb{Z}}(G;A) \to H^2(G;A) \to \operatorname{Hom}(M(G),A) \to 0$$

  All the groups appearing in the short exact sequence are {\em
    contravariant} functors of $G$ (holding $A$ constant) and
  (covariant) functors of $A$ (holding $G$ constant). Further, the
  morphisms of the short exact sequence are natural transformations
  with respect to both $G$ and $A$: they are natural transformations
  between the contravariant functors of $G$ (holding $A$ constant) and
  also between the covariant functors of $A$ (holding $G$ constant).

  A similar observation applies to the corresponding short exact
  sequence for Lie ring extensions described in Section
  \ref{sec:ses-uct-lie}.

\item In Section \ref{sec:ses-coboundary-cocycle-cohomology}, we
  consider the following short exact sequence associated with a group
  $G$ and an abelian group $A$:

  $$0 \to B^2(G;A) \to Z^2(G;A) \to H^2(G;A) \to 0$$

  All the groups appearing in the short exact sequence are {\em
    contravariant} functors of $G$ (holding $A$ constant) and
  (covariant) functors of $A$ (holding $G$ constant). Further, the
  morphisms of the short exact sequence are natural transformations
  with respect to both $G$ and $A$: they are natural transformations
  between the contravariant functors of $G$ (holding $A$ constant) and
  also between the covariant functors of $A$ (holding $G$ constant).

\end{itemize}

\subsection{Initial objects, terminal objects, and zero objects}

Suppose $\mathcal{C}$ is a category. We define the following:

\begin{itemize}
\item An {\em initial object} of $\mathcal{C}$ is an object $A$ of
  $\mathcal{C}$ such that for every object $B$ of $\mathcal{C}$, there
  is a unique morphism from $A$ to $B$, i.e., there is a unique
  element of $\mathcal{C}(A,B)$. Any two initial objects must be
  isomorphic and have a unique isomorphism between them. A category
  may or may not have an initial object. For a category that has
  initial objects, we therefore abuse notation by talking of {\em the}
  initial object of the category.
\item A {\em terminal object} of $\mathcal{C}$ is an object $A$ of
  $\mathcal{C}$ such that for every object $B$ of $\mathcal{C}$, there
  is a unique morphism from $B$ to $A$, i.e., there is a unique
  element of $\mathcal{C}(B,A)$. Any two terminal objects must be
  isomorphic and have a unique isomorphism between them. A category
  may or may not have a terminal object. For a category that has
  terminal objects, we therefore abuse notation by talking of {\em
    the} terminal object of the category.
\item A {\em zero object} of $\mathcal{C}$ is an object that is both
  an initial object and a terminal object. A category may or may not
  have a zero object. Any two zero objects must be isomorphic and have
  a unique isomorphism between them.

  Note that even if a category has an initial
  object and a terminal object, it may lack a zero object because the
  initial objects and terminal objects are not isomorphic. For
  instance, for the category of commutative unital rings, the initial
  object is $\Z$ and the terminal object is the zero ring, so there is
  no zero object. For a category that has terminal objects, we
  therefore abuse notation by talking of {\em the} terminal object of
  the category.

  For a category that has a zero object, we can define, between any
  two objects of the category, a morphism called the {\em zero
    morphism}. This is the morphism that is obtained by composing the
  map from the source object to the zero object and the map from the
  zero object to the target object. For instance, for the category of
  groups, the zero object is the trivial group, and the corresponding
  notion of zero morphism is the trivial homomorphism between any two
  groups.
\end{itemize}

The discussion in Sections \ref{sec:exteriorsquare} and
\ref{sec:exteriorsquare-lie} is motivated by the goal of finding an
initial object in a category, namely, the category of central
extensions of a particular group (or Lie ring) with the morphisms
being homoclinisms of central extensions. We show that initial objects
do exist for these categories, and these initial objects play an
important role in the study of the categories.

\subsection{Left and right adjoint functors}\label{appsec:adjoint}

Suppose $\mathcal{C},\mathcal{D}$ are categories and $F:\mathcal{C}
\to \mathcal{D}$ and $G:\mathcal{D} \to \mathcal{C}$ are covariant
functors. We say that $\mathcal{F}$ is left adjoint to $\mathcal{G}$,
or equivalently, that $\mathcal{G}$ is right adjoint to $\mathcal{F}$,
if there is a family of bijections:

$$\mathcal{D}(\mathcal{F} A, B) \cong \mathcal{C}(A, \mathcal{G}B)$$

for all $A \in \operatorname{Ob} \mathcal{C}$ and $B \in
\operatorname{Ob} \mathcal{D}$, such that the bijection is natural (in
the sense of being a natural transformation) in both the variables $A$
and $B$. The use of the word ``adjoint'' here is in analogy with the
idea of adjoint linear transformations with respect to an inner
product.

\subsection{Universal algebra}\label{appsec:univalg-basic}

Universal algebra is a branch of mathematics that provides a
unifying framework for the study of algebraic structures. Universal
algebra is fairly general and it is ill-suited to proving deep facts
about specific structures (such as advanced results in group theory or
in ring theory). Nonetheless, basic vocabulary from universal algebra
is useful in the study of specific algebraic structures.

The first concept we define is the concept of {\em signature}. A
signature is a function from a fixed set (that we call the {\em
  operator domain}) to $\N_0$ (the set of nonnegative integers). The
elements of the operator domain are the {\em operators} and the
signature sends each operator to a nonnegative integer that is its
{\em arity}. An {\em algebra} of a given signature is defined to be a
set equipped with operations as follows: for each operator in the
operator domain, there is a $n$-ary operation from the set to itself
where $n$ equals the arity of the operator. Note that a $0$-ary
operation is understood to mean a constant function.

An {\em identity} refers to a formal equality of two formal
expressions that are constructed using operators in the operator
domain, where the expressions make sense given what we know about the
arities of the operations. An algebra is said to satisfy the identity
if that identity holds for all choices of elements in that
algebra.

Given a signature and a collection of identities, the collection of
all algebras of the signature satisfying all the identities in the
collection will be called a {\em variety of algebras}.

For instance, consider the signature with operator domain $\Omega = \{
* \}$ where the arity of $*$ is $2$. We will use infix notation to
denote $*$, i.e., $*(x,y)$ will be denoted as $x * y$. An algebra with
this signature is equivalent to a set with a binary operation, also
called a {\em magma}.\footnote{Historically, the term ``groupoid'' was
  used but that term now has a somewhat different meaning.}

Then, associativity of $*$ is an identity. Explicitly, it is the identity:

$$(a * b) * c = a * (b * c)$$

The collection of all magmas that {\em satisfy the identity of
  associativity} is a {\em variety of algebras}. Algebras of this type
are termed {\em semigroups}, and the variety of algebras is termed
{\em the variety of semigroups}.

Groups also form a variety of algebras. The variety of groups has an
operator domain with three operations:

\begin{itemize}
\item An operation $*$ of arity $2$, called the {\em group
  multiplication} or {\em product} (denoted using infix notation)
\item An operation $e$ of arity $0$, called the {\em identity element}
  or {\em neutral element}.\footnote{We use $1$ to denote the identity
    element in this document, but it is more helpful to use the letter
    $e$ here for pedagogical reasons.}
\item An operation ${}^{-1}$ of arity $1$, called the {\em inverse
  map} (denoted using postfix superscript notation).
\end{itemize}

The identities that need to be universally satisfied are:

\begin{eqnarray*}
  (a * b) * c & = & a * (b * c) \ \text{ (Associativity)}\\
  a * e = e * a & = &  a \ \text{ (Identity element)}\\
  a * a^{-1} = a^{-1} * a & = & a \ \text{ (Inverse map)}\\
\end{eqnarray*}

The following notions can readily be defined in the context of
algebras of a given signature:

\begin{itemize}
\item {\em Subalgebra} refers to a subset of the underlying set of the
  algebra that is closed under all the operations defined for the
  algebra structure.
\item {\em Homomorphism of algebras} (between algebras of the same
  signature) refers to a set map between the algebras that commutes
  with all the algebra operations.
\item Based on the definition of homomorphism, we can define {\em
  isomorphism}, {\em endomorphism}, and {\em automorphism}.
\item {\em Direct product of algebras} (where all the algebras have
  the same signature) is defined as follows: we take the Cartesian
  product of the underlying sets, and define each of the operations
  coordinate-wise.
\end{itemize}

An important theorem about varieties of algebras is {\em Birkhoff's
  theorem}. The theorem states that a collection of algebras is a
variety of algebras if and only if it is closed under taking
subalgebras, homomorphic images, and direct products. One direction of
the proof is easy: any variety of algebras is closed under taking
subalgebras, homomorphic images, and direct products. The reverse
direction is hard, and we omit the proof.

The following are some examples of varieties of algebras that we
consider in this document:

\begin{itemize}
\item The variety of groups.
\item The variety of Lie rings.
\item The variety of abelian groups.
\item The variety of associative rings.
\item For a fixed commutative unital ring $R$, the variety of $R$-Lie algebras.
\item For a fixed commutative unital ring $R$, the variety of
  associative $R$-algebras.
\item For a fixed commutative unital ring $R$, the variety of
  $R$-modules.
\item For a set $\pi$ of primes, the variety of $\pi$-powered groups,
  described in Section \ref{sec:group-powering}.
\item For a set $\pi$ of primes, the variety of $\pi$-powered Lie
  rings, described in Section \ref{sec:lie-ring-powering}.
\item The variety of groups of nilpotency class $c$, for some fixed
  positive integer $c$.
\item The variety of Lie rings of nilpotency class $c$, for some fixed
  positive integer $c$.
\item The variety of $\pi$-powered groups of nilpotency class $c$.
\item The variety of $\pi$-powered Lie rings of nilpotency class $c$.
\end{itemize}

\subsection{Universal algebra and category theory combined: free and forgetful functors}\label{appsec:free-forgetful}

Every variety of algebras can naturally be interpreted as a {\em
  category}. The objects of the category are the algebras of the
variety. The morphisms of the category are homomorphisms between the
algebras of the variety. Note that the definition of homomorphism uses
only the signature and {\em not} the identities.

The following are some immediate observations about the category
corresponding to a particular variety of algebras:

\begin{itemize}
\item The notions of isomorphism, endomorphism, and automorphism
  defined in universal algebra coincide with the corresponding notions
  for the category.
\item The terminal object has an underlying set of size one, and all
  the operations are defined in the unique manner possible.
\item The category does have an initial object. If the variety has no
  0-ary operations, the initial object of the variety has an empty
  underlying set. If the varety has 0-ary operations, the initial
  object has a nonempty underlying set.
\item In general, the initial object and terminal object need not
  coincide, and therefore the zero object need not exist. Some
  varieties where they do coincide are the variety of groups, variety
  of Lie rings, variety of abelian groups, variety of $R$-modules for
  a commutative unital ring $R$, and variety of $R$-Lie algebras for a
  commutative unital ring $R$. A variety where they do not coincide is
  the variety of commutative unital rings. The initial object for this
  variety is $\Z$ and the terminal object is the zero ring.
\end{itemize}

Suppose $\mathcal{V}_1$ and $\mathcal{V}_2$ are two varieties of
algebras such that the operator domain of $\mathcal{V}_2$ is a subset
of the operator domain of $\mathcal{V}_1$, with the operators having
the same arities, and every identity true in $\mathcal{V}_2$ is also
true in $\mathcal{V}_1$. Formally, we say that the variety
$\mathcal{V}_2$ is a {\em reduct} of $\mathcal{V}_1$.

In this case, there is a natural {\em forgetful functor} from the
category corresponding to $\mathcal{V}_1$ to the category
corresponding to $\mathcal{V}_2$. This functor is faithful: for any
algebras $A$ and $B$ of $\mathcal{V}_1$, the set map
$\operatorname{Hom}_{\mathcal{V}_1}(A,B) \to
\operatorname{Hom}_{\mathcal{V}_2}(A,B)$ is injective. However, the
functor is not necessarily {\em full} because the set map above need
not be surjective. The functor also need not be {\em essentially
  surjective}: it is not necessary that every isomorphism type of
algebra in $\mathcal{V}_2$ must arise from an isomorphism type of
algebra in $\mathcal{V}_1$.

There is also a natural {\em free functor} from the category
corresponding to $\mathcal{V}_2$ to the category corresponding to
$\mathcal{V}_1$. This functor sends any algebra of $\mathcal{V}_2$ to
the algebra of $\mathcal{V}_1$ ``generated'' by it, with all the
identities of $\mathcal{V}_1$. The free functor is left adjoint to the
forgetful functor.

Below are some cases of interest:

\begin{enumerate}
\item $\mathcal{V}_2$ is the variety with {\em no}
  operations and no identities, so that the corresponding category is
  the category of sets. In this case, the forgetful functor simply
  sends an algebra to its underlying set. The free functor sends a set
  to the {\em free algebra} of the variety $\mathcal{V}_1$ generated
  by the set.
\item The varieties $\mathcal{V}_1$ and $\mathcal{V}_2$ have the same
  operator domain, so the only difference is that $\mathcal{V}_1$ may
  have more identities than $\mathcal{V}_2$. In this case, we say that
  $\mathcal{V}_1$ is a subvariety of $\mathcal{V}_2$, and the
  forgetful functor from the category for $\mathcal{V}_1$ to the
  category for $\mathcal{V}_2$ is a full functor in this case. An
  example is the case where $\mathcal{V}_1$ is the variety of groups
  of nilpotency class $c$ (for some positive integer $c$) and
  $\mathcal{V}_2$ is the variety of all groups.
\item Cases other than (2) where the forgetful functor is still full,
  i.e., every homomorphism in $\mathcal{V}_2$ between two algebras of
  $\mathcal{V}_1$ defines a homomorphism in $\mathcal{V}_1$. For
  instance, the forgetful functor from the variety of groups to the
  variety of semigroups that ``forgets'' the identity element and
  inverse map is full: any homomorphism of semigroups between two
  groups is also a homomorphism of groups. Similarly, for a prime set
  $\pi$, the forgetful functor from the variety of $\pi$-powered
  groups to the variety of groups is a full functor: any homomorphism
  of groups between two $\pi$-powered groups is also a homomorphism of
  $\pi$-powered groups.
\end{enumerate}

%\newpage

\section{Foundational results for manipulations in nilpotent groups}\label{appsec:nilpotent}

\subsection{Central series and nilpotency for groups and Lie rings}\label{appsec:group-basic}

Suppose $G$ is a group. A subgroup series:

$$G = K_1 \ge K_2 \ge \dots K_c \ge K_{c+1} = 1$$

is termed a {\em central series} if it satisfies the following conditions:

\begin{enumerate}
\item It is a normal series:\footnote{Some people use the term
  ``normal series'' for a series where each term is normal in its
  predecessor, and use the term ``strongly normal series'' for a
  series satisfying the condition stated here.} every $K_i$ is normal
  in $G$.
\item For every $i$, $K_i/K_{i+1}$ is contained in the center of $G/K_{i+1}$.
\end{enumerate}

Equivalently, it should satisfy the condition that for every $i$:

$$[G,K_i] \subseteq K_{i+1}$$

Note that the notation above uses a descending series, but the series
may also be described as an ascending series.

$G$ is termed a {\em nilpotent group} if it has a central series. The
{\em nilpotency class} of $G$ is defined as the smallest $c$ for which
there exists a central series of length $c$ (matching the above
notation). We say that $G$ is a group of nilpotency class (at most)
$c$ if the nilpotency class of $G$ is less than or equal to $c$. We
often drop the ``(at most)'' qualifier for nilpotency class and simply
say ``$G$ is a group of nilpotency class $c$'' to mean that $G$ is a
nilpotent group and its nilpotency class is at most $c$.

There is an ascending series termed the {\em upper central series} and
a descending series termed the {\em lower central series}. Both
of these can be defined for all groups. For each series, it reaches
the other end in finitely many steps if and only if the group is
nilpotent, and if so, its length is the nilpotency class of the group.

The {\em upper central series} of $G$ is an ascending series
$(Z^i(G))_{i \in \mathbb{N}_0}$ of subgroups of $G$ defined as
follows:

\begin{itemize}
\item $Z^0(G)$ is the trivial subgroup of $G$.
\item For $i > 0$, $Z^i(G)$ is defined as the unique subgroup of $G$
  such that $Z^i(G)/Z^{i-1}(G) = Z(G/Z^{i-1}(G))$, where
  $Z(G/Z^{i-1}(G))$ denotes the center of $G/Z^{i-1}(G)$. In
  particular, $Z^1(G) = Z(G)$ is the center of $G$. The subgroup
  $Z^2(G)$ is termed the second center of $G$, and so on.
\end{itemize}

The upper central series reaches the whole group $G$ in finitely many
steps if and only if $G$ is nilpotent. Further, if $G$ has nilpotency
class $c$, then $Z^c(G) = G$. More explicitly, if the nilpotency class
of $G$ is precisely $c$, then $Z^c(G) = G$ and $Z^i(G) \ne G$ for $i <
c$. In fact, the upper central series is the {\em fastest ascending}
central series possible.\footnote{Technically, although we can define
  the upper central series for a non-nilpotent group, and this
  definition is useful, the upper central series is {\em not} a
  central series if the group is non-nilpotent.

We can also define a {\em transfinite} upper central series, where we
define $Z^\alpha(G)$ for all ordinals $\alpha$. However, we do not
need to use the transfinite upper central series here, so we will not
define it.}

The {\em lower central series} of $G$ is a descending series of
subgroups $\gamma_i(G), i \in \mathbb{N}$, defined as follows:

\begin{itemize}
\item $\gamma_1(G) = G$
\item For $i \in \mathbb{N}$, $\gamma_{i+1}(G) = [G,\gamma_i(G)] =
  [\gamma_i(G),G]$ is the commutator of the subgroups $G$ and
  $\gamma_i(G)$.
\end{itemize}

The lower central series reaches the trivial subgroup in finitely many
steps if and only if $G$ is nilpotent. Further, if $G$ has nilpotency
class $c$, then $\gamma_{c+1}(G) = 1$. More explicitly, if the
nilpotency class of $G$ is precisely $c$, then $\gamma_{c+1}(G) = 1$
and $\gamma_{i+1}(G) \ne 1$ for $i < c$. In fact, the lower central
series is the {\em fastest descending} central series
possible.\footnote{Technically, although we can define the lower
  central series for a non-nilpotent group, and this definition is
  useful, the lower central series is {\em not} a central series if
  the group is non-nilpotent.

The subgroup $\gamma_2(G)$, also denoted $G'$, is the {\em derived
  subgroup} of $G$, and is sometimes also referred to as the {\em
  commutator subgroup} of $G$ (note that the term ``commutator
subgroup'' may also be used for a commutator of {\em two}
subgroups). The quotient group $G/\gamma_2(G) = G/G'$ is denoted
$G^{\operatorname{ab}}$ and is termed the {\em abelianization} of $G$.

We can also define a {\em transfinite} lower central series, where we
define $\gamma_\alpha(G)$ for all ordinals $\alpha$. However, we do
not need to use the transfinite lower central series here, so we will
not define it.}

\subsection{Witt's identity and the three subgroup lemma}\label{appsec:witt-three-subgroup}

Witt's identity (stated here with the right action convention) applies
to any elements $a,b,c$ in any group $G$. It says that:

$$[[a,b^{-1}],c]^b \cdot [[b,c^{-1}],a]^c \cdot [[c,a^{-1}],b]^a  = 1$$

The three subgroup lemma follows directly from Witt's identity:

\begin{lemma}[Three subgroup lemma]
  Suppose $G$ is a group and $A$, $B$, and $C$ are subgroups of
  $G$. Then, any two of these three statements implies the third:

  \begin{itemize}
  \item $[[A,B],C] = 1$
  \item $[[B,C],A] = 1$
  \item $[[C,A],B] = 1$
  \end{itemize}
\end{lemma}

\subsection{Commutator of element with products, and commutator as a homomorphism}

The following computational result regarding the commutator of an
element and a product is useful. The result has somewhat different explicit formulations depending on whether we use 

With the {\em left action convention}, where we denote by ${}^xg$ the
element $xgx^{-1}$ and where we denote by $[x,y]$ the commutator
$xyx^{-1}y^{-1}$, the result states that:

\begin{itemize}
\item $[x,yz] = [x,y](^y[x,z])$
\item $[xy,z] = ({}^x[y,z])[x,z]$
\end{itemize}

With the {\em right action convention}, where we denote by $g^x$ the
element $x^{-1}gx$ and where we denote by $[x,y]$ the commutator
$x^{-1}y^{-1}xy$,the result states that:

\begin{itemize}
\item $[x,yz] = [x,z][x,y]^z$
\item $[xy,z] = ([x,z]^y)[y,z]$
\end{itemize}

\begin{lemma}\label{lemma:iterated-commutator-is-multilinear}
  \begin{enumerate}
  \item {\em Independent of the action convention}: Suppose $G$ is a
    group of nilpotency class two. For any fixed $x \in G$, the
    commutator maps $y \mapsto [x,y]$ and $y \mapsto [y,x]$ are
    endomorphisms of $G$. Note that since the image of both
    endomorphisms is inside $[G,G]$, they can be viewed as
    homomorphisms to $[G,G]$.
  \item {\em Formulation specific to the left action convention}:
    Suppose $G$ is a group, $H$ is a subgroup of $G$, and $x$ is an
    element of $G$ such that $[x,H] \subseteq C_G(H)$, or
    equivalently, $[[x,h_1],h_2] = 1$ for all $h_1, h_2 \in H$. Then,
    the map $y \mapsto [x,y]$ is a homomorphism from $H$ to $G$. Note
    that with the right action convention, the map may become an
    anti-homomorphism, and the corresponding statement for the right
    action convention would use $y \mapsto [y,x]$.
  \item {\em Independent of the action convention}: Suppose $G$ is a
    group of nilpotency class $c$. Consider the set map:

    $$T: G \times G \times \dots \times G \to G$$

    where $G$ occurs $c$ times on the left, given by:

    $$T(x_1,x_2,\dots,x_c) := [[ \dots [[x_1,x_2],x_3],\dots,x_{c-1}],x_c]$$

    Choose any $i$ with $1 \le i \le c$ and fix the values of $x_j, j
    \ne i$. Then, $T$, viewed solely as a function of $x_i$, defines
    an endomorphism of $G$. The image of $T$ is inside $\gamma_c(G)$,
    so viewed this way, $T$ defines a homomorphism from $G$ to
    $\gamma_c(G)$. In fact, $T$ descends to a homomorphism from
    $G/Z^{c-1}(G)$ to $\gamma_c(G)$.
  \end{enumerate}
\end{lemma}

We will use the left action convention for our proofs, including
proofs of the statements that are independent of the action
convention.

\begin{proof}
  {\em Proof of (1)}: Showing that the map $y \mapsto [x,y]$ is an
  endomorphism is equivalent to showing that $[x,yz] =
  [x,y][x,z]$. This in turn follows from the identity $[x,yz] =
  [x,y]({}^y[x,z])$ and the assumption of class two allowing us to
  deduce that $[x,z] = {}^y[x,z]$. The proof for the other map is
  similar.

  {\em Proof of (2)}: This is similar to (1). Explicitly, we want to
  show that for $y,z \in H$, we have $[x,yz] = [x,y][x,z]$. This
  follows from the identity $[x,yz] = [x,y]({}^y[x,z])$ combined with
  the fact that, since $z \in H$, $[x,z] \in C_G(H)$ by assumption, so
  $[x,z]$ commutes with $y$.

  {\em Proof of (3)}: We prove the statement by induction on $c$. The
  base case $c = 2$ follows from (1). Suppose now that the statement
  is true for all smaller classes, and we want to prove it for class
  $c$.

  By the inductive hypothesis, the map:

  $$T_{c-1}: (x_1,x_2,\dots,x_{c-1}) \mapsto [ \dots [x_1,x_2],\dots,x_{c-1}] \mod \gamma_c(G)$$

  satisfies all the hypotheses: it is a homomorphism in each
  coordinate holding the other coordinates fixed. Further, the image
  of the map is in $\gamma_{c-1}(G)/\gamma_c(G)$.

  Now, note that the commutator map:

  $$\gamma_{c-1}(G) \times G \to \gamma_c(G)$$

  satisfies the condition that the image is in the center of $G$. In
  particular, by (2), we obtain that the commutator map is a
  homomorphism with respect to both coordinates. The map descends to
  the following map which is also a homomorphism in both coordinates:

  $$U: \gamma_{c-1}(G)/\gamma_c(G) \times G \to \gamma_c(G)$$

  We now see that:

  $$T(x_1,x_2,\dots,x_c) = U(T_{c-1}(x_1,x_2,\dots,x_{c-1}),x_c)$$

  The fact that both $T_{c-1}$ and $U$ are homomorphisms in each of
  their coordinates gives us that $T$ is a homomorphism in each
  coordinate.

  The proof of the statement that $T$ descends to a map from
  $G/Z^{c-1}(G)$ follows from the three subgroup lemma, discussed in
  Section \ref{appsec:witt-three-subgroup}.
\end{proof}

\subsection{Strongly central series}\label{appsec:strongly-central}

A descending series of subgroups:

$$G = G_1 \ge G_2 \ge G_3 \ge \dots \ge G_n = 1 = G_{n+1} = G_{n+2} = \dots$$

is termed a {\em strongly central series} if $[G_i,G_j] \le G_{i+j}$
for any positive integers $i$ and $j$.

\begin{lemma}
  For any group $G$ and any positive integers $i$ and $j$, we have
  that $[\gamma_i(G),\gamma_j(G)] \le \gamma_{i+j}(G)$. In particular,
  the lower central series of a nilpotent group is a strongly central series.
\end{lemma}

\begin{proof}
  This follows from the three subgroup lemma and proof by induction.
\end{proof}

%\newpage

\section{The use of grading}

\subsection{Graded ring}

Suppose $G$ is an abelian group with the group operation denoted
additively. Suppose $R$ is a commutative (associative) unital ring. A
$G$-graded $R$-algebra is a $R$-algebra $A$ whose additive group has a
direct sum decomposition as a sum of $R$-submodules indexed by the
group elements:

$$A = \bigoplus_{g \in G} A_g$$

such that for all $g,h \in G$, we have:

$$A_gA_h \subseteq A_{g+h}$$

An element of $A$ that lies inside $A_g$ for some $g \in G$ is termed
a {\em homogeneous element}.

A $G$-graded {\em ring} is a $G$-graded $\Z$-algebra.

The following cases are of particular interest:

\begin{itemize}
\item A $\Z$-graded $R$-algebra is a $R$-algebra graded over the group
  of integers $\Z$.
\item A $\N_0$-graded $R$-algebra is a $\Z$-graded $R$-algebra $A$
  where $A_i = 0$ for all $i < 0$.
\item A $\N$-graded $R$-algebra a $\Z$-graded $R$-algebra $A$ where
  $A_i = 0$ for all $i \le 0$.
\end{itemize}

Generally, the term {\em graded $R$-algebra} is used for a
$\mathbb{N}_0$-graded $R$-algebra, and the term {\em graded ring} is
used for a $\N_0$-graded $\Z$-algebra. We will follow this convention
for the rest of the appendix and in the main document.

\subsection{Verification of multilinear identities for graded rings}\label{appsec:multilinear-graded}

A $R$-multilinear identity holds in a graded $R$-algebra $A$ if and
only if it holds in the case where all inputs are homogeneous
elements. This is because the set of homogeneous elements forms a
generating set for $A$ as a $R$-module, and multilinear
identities can be verified on a generating set, as described in
Section \ref{appsec:multilinear}. In particular, this applies to
verifying associativity and the Jacobi identity.

\subsection{Associated graded Lie algebra for a Lie algebra}\label{appsec:associated-graded-lie}

Suppose $L$ is a Lie algebra over a commuative (associative) unital
ring $R$. The {\em associated graded $R$-Lie algebra} for $L$ is a
graded $R$-Lie algebra defined as follows. The underlying $R$-module
is:

$$\bigoplus_{i=1}^\infty \gamma_i(L)/\gamma_{i+1}(L)$$

where the $i^{th}$ component is the direct summand
$\gamma_i(L)/\gamma_{i+1}(L)$, where $\gamma_i(L)$ and
$\gamma_{i+1}(L)$ are the $i^{th}$ and $(i+1)^{th}$ members
respectively of the lower central series of $L$. We now define the Lie
bracket on the associated graded $R$-Lie algebra.

First, note that the Lie bracket restricts to the following
$R$-bilinear maps:

\begin{eqnarray*}
  \gamma_i(L) \times \gamma_j(L) & \to & \gamma_{i+j}(L)\\
  \gamma_{i+1}(L) \times \gamma_j(L) & \to & \gamma_{i+j+1}(L)\\
  \gamma_i(L) \times \gamma_{j+1}(L) & \to & \gamma_{i+j+1}(L)\\
\end{eqnarray*}

As a result, we get a canonical $R$-bilinear map:

$$\gamma_i(L)/\gamma_{i+1}(L) \times \gamma_j(L)/\gamma_{j+1}(L) \to \gamma_{i+j}(L)/\gamma_{i+j+1}(L)$$

The Lie bracket on the associated graded $R$-Lie algebra of $L$ is
defined as the Lie bracket whose restriction to the bracket of the
$i^{th}$ graded component and the $j^{th}$ graded component is the
above map for all $i$ and $j$. In order to show that this is a {\em
  Lie} bracket, we need to verify the Jacobi identity. Due to the
multilinearity of the Jacobi identity, it suffices to verify it in the
case that all three inputs are homogeneous (as explained in the
preceding section, Section \ref{appsec:multilinear-graded}), and in
that case, it follows from the Jacobi identity in the original $R$-Lie
algebra $L$.

Note in particular that if $L$ is a nilpotent $R$-Lie algebra of
nilpotency class at most $c$, then the direct sum above is the finite
direct sum:

$$\bigoplus_{i=1}^c \gamma_i(L)/\gamma_{i+1}(L)$$

For the case that $R = \Z$, we are thinking of $L$ simply as a Lie
ring, and the corresponding associated graded $\Z$-Lie algebra will be
called the {\em associated graded Lie ring}. Note that for any
commutative unital ring $R$ and any $R$-Lie algebra $L$, the following
coincide:

\begin{itemize}
\item The associated graded Lie ring of $L$, viewed as a Lie ring
  (ignoring its structure as a $R$-Lie algebra).
\item The underlying Lie ring for the associated graded $R$-Lie
  algebra of $L$.
\end{itemize}

The definition of associated graded Lie ring can also be generalized
to {\em any} strongly central series of a (nilpotent) Lie
ring. Strongly central series for Lie rings are defined analogously to
the definition for groups in Section \ref{appsec:strongly-central}. In
the context of this more general definition, the usual associated
graded Lie ring is the associated graded Lie ring for the lower
central series.
 
\subsection{Associated graded Lie ring for a group}\label{appsec:associated-graded-group}

Suppose $G$ is a group. The {\em associated graded Lie ring} for $G$
is a graded Lie ring defined as follows. The additive group is:

$$\bigoplus_{i=1}^\infty \gamma_i(G)/\gamma_{i+1}(G)$$

where the $i^{th}$ component is the direct summand
$\gamma_i(G)/\gamma_{i+1}(G)$.

First, note that the Lie bracket restricts to the following
$\mathbb{Z}$-bilinear maps:

\begin{eqnarray*}
  \gamma_i(G) \times \gamma_j(G) & \to & \gamma_{i+j}(G)\\
  \gamma_{i+1}(G) \times \gamma_j(G) & \to & \gamma_{i+j+1}(G)\\
  \gamma_i(G) \times \gamma_{j+1}(G) & \to & \gamma_{i+j+1}(G)\\
\end{eqnarray*}

As a result, we get a canonical $\mathbb{Z}$-bilinear map:

$$\gamma_i(G)/\gamma_{i+1}(G) \times \gamma_j(G)/\gamma_{j+1}(G) \to \gamma_{i+j}(G)/\gamma_{i+j+1}(G)$$

The Lie bracket on the associated graded Lie ring of $G$ is defined as
the Lie bracket whose restriction to the bracket of the $i^{th}$
graded component and the $j^{th}$ graded component is the above map
for all $i$ and $j$. In order to show that this is a {\em Lie}
bracket, we need to verify the Jacobi identity. Due to the
multilinearity of the Jacobi identity, it suffices to verify it in the
case that all three inputs are homogeneous (as explained in the
preceding section, Section \ref{appsec:multilinear-graded}). The
Jacobi identity in the homogeneous case follows from Witt's identity,
described in Section \ref{appsec:witt-three-subgroup}.

The definition of associated graded Lie ring can also be generalized
to {\em any} strongly central series of a (nilpotent) Lie
ring. Strongly central series were defined for groups in Section
\ref{appsec:strongly-central}. In the context of this more general
definition, the usual associated graded Lie ring is the associated
graded Lie ring for the lower central series.

\subsection{Associated graded Lie ring as a functor}\label{appsec:associated-graded-functor}

The associated graded Lie ring is functorial both for groups and for
Lie rings. Explicitly, we can define a functor from the category of
groups to the category of Lie rings such that its mapping on objects
sends a group to its associated graded Lie ring. Similarly, we can
define a functor from the category of Lie rings to itself such that
its mapping on objects sends a Lie ring to its associated graded Lie
ring.

We can think of taking the associated graded Lie ring as
``flattening'' the structure of the original object (the group or the
Lie ring) thereby forgetting some aspects of the cohomology. Note,
however, that taking the associated graded Lie ring {\em does}
preserve the order and the nilpotency class. We discover that the
following are true:

\begin{itemize}
\item If a group and Lie ring are in global class $c$ Lazard
  correspondence, they define the same associated graded Lie ring.
\item If two nilpotent groups are isoclinic to each other, then their
  associated graded Lie rings are isoclinic to each other. We can
  actually make a slightly stronger statement: if two groups are
  isoclinic as central extensions with the same base and same quotient
  group, then their associated graded Lie rings are {\em
    isomorphic}. Here, ``isoclinic as central extensions'' is being
  used in the sense of Sections \ref{sec:extensionsuptoisoclinism} and
  \ref{sec:beta-map}.
\item If two Lie rings are isoclinic to each other, then their
  associated graded Lie rings are isoclinic to each other. We can
  actually make a slightly stronger statement: if two Lie rings are
  isoclinic as central extensions with the same base and same quotient
  Lie rings, then their associated graded Lie rings are
  isomorphic. Here, ``isoclinic as central extensions'' is being used
  in the sense of Sections \ref{sec:extensionsuptoisoclinism-lie} and
  \ref{sec:beta-map-lie}.
\item If a group and Lie ring are in global class $c + 1$ Lazard
  correspondence up to isoclinism, then their associated graded Lie
  rings are isoclinic. We can make a slightly stronger statement: if a
  group and Lie ring are in global class $(c + 1)$ Lazard
  correspondence up to isoclinism {\em as central extensions} (as
  described in Section \ref{sec:glcuti-extensions-main}), then their
  corresponding associated graded Lie rings are isomorphic.
\end{itemize}

In the case of class two, the associated graded Lie ring offers an
essentially complete classification of the equivalence type up to
isoclinism for the extension. This is because the ``flattening'' it
does forgets only the abelian part of the group structure, and not the
commutator structure. For higher class, however, although the
associated graded Lie ring is invariant under taking isoclinisms of
extensions, it is not a complete invariant, because passing to the
associated graded Lie ring ``flattens'' out part of the commutator
structure as well.

\section{Homologism theory for arbitrary varieties}\label{appsec:homologism-theory}

The material discussed in this section of the Appendix is used in
Section \ref{sec:potential-extension-n-isoclinism} when discussing a
potential extension of the Lazard correspondence up to isoclinism. It
is not used elsewhere in the document, although it may help provide
better perspective on the material elsewhere in the document as well.

For basic background material on varieties of algebras, see the
Appendix, Sections \ref{appsec:univalg-basic} and
\ref{appsec:free-forgetful}.

The terms and concepts alluded to in this section can be found in the
paper \cite{Baerinvariantsandisolosigms} by Leedham-Green and
McKay. The concepts of $n$-isoclinism and $n$-homoclinism are also
described in \cite{Hekster} (for groups) and \cite{Moghaddametal} (for
Lie rings).

\subsection{Word maps: definition for an arbitrary variety of algebras}\label{appsec:word-maps}

Suppose $\mathcal{V}$ is a variety of algebras. A {\em word} in $n$
letters in $\mathcal{V}$ is defined as an element of the free algebra
(with respect to the variety $\mathcal{V}$) in terms of the $n$
letters. A word can be represented using an expression in terms of the
$n$ letters and the operations of the variety. Two expressions
describe the same word if they are formally equal, i.e., their
equality can be dedued from the formal identities of the variety. We
will often abuse notation by conflating the concepts of the word with
a formal expression used to describe the word.

Given a word $w$ in $n$ letters for the variety $\mathcal{V}$, and an
algebra $A$ in $\mathcal{V}$, $w$ defines a set map $A^n \to A$ as
follows. Denote by $F$ the free algebra on the $n$ letters, and denote
the letters by $g_1,g_2,\dots,g_n$. Then, for any tuple
$(x_1,x_2,\dots,x_n) \in A^n$, we can consider the unique homomorphism
$\varphi:F \to A$ such that $\varphi(g_i) = x_i$. The image
$\varphi(w)$ is the image of the tuple $(x_1,x_2,\dots,x_n)$ under the
word map. Abusing notation, we will use $w$ to denote the word itself,
the expression describing the word, {\em and} the word map, i.e., the
image of the tuple $(x_1,x_2,\dots,x_n)$ will be denoted
$w(x_1,x_2,\dots,x_n)$.

For some varieties, such as the variety of groups, there is a concept
of a {\em reduced word expression}. Any element of the free group on
$n$ letters has a unique {\em reduced expression}, and the equality of
two words can be determined by converting both of them to their
corresponding reduced expressions and then checking for literal
equality.

For the variety of groups, words involve group multiplication and
inverse map operations, and a word in reduced form is an expression
written as a product of the letters and their inverses, with no letter
and its inverse occurring adjacent to one another. For instance, the
following is an example word in two letters $g_1$ and $g_2$:

$$w(g_1,g_2) := g_1g_2g_1g_2^2g_1^{-1}$$

We say that a group $G$ satisfies a word $w$ in $n$ letters (with
respect to the variety of groups) if the image of the word map
corresponding to $w$ is the trivial subgroup of $G$, i.e.,
$w(x_1,x_2,\dots,x_n) = 1$ for all $x_1,x_2,\dots,x_n \in G$. 

The concept generalizes to any {\em variety of algebras with zero},
i.e., a variety of algebra with a distinguished 0-ary operation called
the {\em zero element}. Given a variety $\mathcal{V}$ with zero, an
algebra $A$ is said to {\em satisfy} a word $w$ in $n$ letters with
respect to the variety $\mathcal{V}$ if $w(x_1,x_2,\dots,x_n)$ is the
zero element of $A$ for all $x_1,x_2,\dots,x_n \in A$. For the variety
of groups, the zero element is the identity element.
\subsection{Word maps and subvarieties of the variety of groups}

To specify a subvariety of the variety of groups, we need to specify a
set of identities that define the subvariety. Note that the set of
identities we choose need not be an {\em exhaustive} list of
identities satisfied in the subvariety, but it needs to be sufficient
to generate all other identities satisfied in the subvariety.

An identity in the variety of groups has the form:

$$w_1(x_1,x_2,\dots,x_n) = w_2(x_1,x_2,\dots,x_n)$$

We can define a new word:

$$w(x_1,x_2,\dots,x_n) := w_1(x_1,x_2,\dots,x_n)(w_2(x_1,x_2,\dots,x_n))^{-1}$$

The identity can therefore be rewritten as:

$$w(x_1,x_2,\dots,x_n) = 1$$

Thus, any subvariety of the variety of groups can be described using a
set of words, and a group is in the subvariety if and only if all the
words are satisfied in the group.

\subsection{Some examples of subvarieties of the variety of groups}

The following are examples of subvarieties, each of which can be
defined using {\em one} word. Note that for all these definitions, it
does not matter whether we use the left or right convention for the
commutator: both define the same variety.

\begin{itemize}
\item The variety of abelian groups: This can be defined using the
  commutator word $w(x_1,x_2) = [x_1,x_2]$. 
\item The variety of groups of nilpotency class at most $c$: This can
  be defined using the iterated commutator word
  $w(x_1,x_2,\dots,x_c,x_{c+1}) = [[ \dots
      [x_1,x_2],\dots,x_c],x_{c+1}]$. 
\item The variety of groups of derived length at most $\ell$: This can
  be defined using an iterated balanced commutator of length
  $2^\ell$. For instance, when $\ell = 2$, the commutator is
  $[[x_1,x_2],[x_3,x_4]]$.
\end{itemize}

\subsection{Verbal and marginal subgroup corresponding to a subvariety}

Suppose $\mathcal{V}$ is a subvariety of the variety of groups. For a
group $G$, the $\mathcal{V}$-{\em verbal subgroup} of $G$ is defined
as the subgroup generated by the images of the word maps in $G$ for
all words that are satisfied in the subvariety
$\mathcal{V}$. Equivalently, it is the smallest normal subgroup of $G$
for which the quotient group is in $\mathcal{V}$.

Suppose $C$ is a set of words that generates the variety
$\mathcal{V}$, i.e., $\mathcal{V}$ is precisely the subvariety of the
variety of groups comprising those groups that satisfy all the words
in $C$. Then, the $\mathcal{V}$-verbal subgroup of $G$ is the subgroup
of $G$ generated by the union of the images of the word maps for all
words in $C$.

The $\mathcal{V}$-{\em marginal subgroup} of $G$ is defined as
follows. It is the set of elements $x$ in $G$ such that for every word
$w$ satisfied in $\mathcal{V}$, the following is true. Let $n$ be the number of letters used in $w$. Then, we want:

\begin{small}
$w(g_1,g_2,\dots,g_i, \dots,g_n) = w(g_1,g_2,\dots,xg_i,\dots,g_n) = w(g_1,g_2,\dots,g_ix,\dots,g_n) \ \forall \ g_1,g_2,\dots,g_n \in G, \ \forall \ i \in \{ 1,2,\dots,n\}$
\end{small}

In other words, the marginal subgroup is the set of elements that do not
affect the evaluation of any of the words that would become trivial in
the variety.

Suppose $C$ is a set of words that generates the variety
$\mathcal{V}$, i.e., $\mathcal{V}$ is precisely the subvariety of the
variety of groups comprising those groups that satisfy all the words
in $C$. Then, we can define the $\mathcal{V}$-marginal subgroup as the
set of elements of $G$ that satisfy the above condition for the words
in $C$ alone.

Some examples of verbal and marginal subgroups are below.

\begin{itemize}
\item Consider the variety of abelian groups. The verbal subgroup of a
  group $G$ corresponding to this variety is the derived subgroup
  $G'$, and the marginal subgroup is the center $Z(G)$.
\item Consider the variety of groups of nilpotency class at most
  $c$. The verbal subgroup of a group $G$ corresponding to this
  variety is the lower central series member subgroup
  $\gamma_{c+1}(G)$. The marginal subgroup is the upper central series
  member $Z^c(G)$.
%%\item {\em TONOTDO: Decide whether to insert more stuff}
\end{itemize}

\subsection{Homologism for a subvariety of the variety of groups}

Suppose $\mathcal{V}$ is a subvariety of the variety of groups. For any
group $G$, denote by $V^*(G)$ the $\mathcal{V}$-marginal subgroup and
by $V(G)$ the $\mathcal{V}$-verbal subgroup.

For any word $w$ satisfied in the variety $\mathcal{V}$, suppose $w$
uses $n_w$ letters. $w$ defines a word map:

$$\beta_{w,G}: G^{n_w} \to G$$

By the definitions of marginal and verbal subgroup, the map descends to a set map:

$$\omega_{w,G}: (G/V^*(G))^{n_w} \to V(G)$$

A {\em homologism} of groups $G_1$ and $G_2$ with respect to $\mathcal{V}$ is a pair $(\zeta,\varphi)$ where $\zeta: G_1/V^*(G_1) \to G_2/V^*(G_2)$, $\varphi: V(G_1) \to V(G_2)$ are homomorphisms, and for every $w$ satisfied in the variety $\mathcal{V}$, we have:

\begin{small}
$$\omega_{w,G_2}(\zeta(x_1), \zeta(x_2), \dots, \zeta(x_{n_w})) = \varphi(\omega_{w,G_1}(x_1,x_2,\dots, x_{n_w})) \ \forall \ (x_1,x_2,\dots,x_n) \in (G_1/V^*(G_1))^{n_w}$$
\end{small}

As with all the other definitions, it suffices to check this condition
on a set of words that {\em generates} the variety rather than on all
words satisfied in the variety.

A homologism is termed an {\em isologism} if both its component
homomorphisms are isomorphisms, or equivalently, if there is an
inverse map to it that is also a homologism.

For any subvariety $\mathcal{V}$ of the variety of groups, we can
define the {\em category of groups with
  $\mathcal{V}$-homologisms}. The objects of this category are groups
and the morphisms are $\mathcal{V}$-homologisms, with composition of
homologisms defined via composition of the corresponding
homomorphisms. The isomorphisms in this category are precisely the
$\mathcal{V}$-isologisms. Further, a group is isomorphic to the
trivial group in this category if and only if it is in the subvariety
$\mathcal{V}$. Thus, the category of groups with
$\mathcal{V}$-homologisms can be thought of as going ``modulo'' the
subvariety $\mathcal{V}$, in so far as the ``kernel'' is precisely
$\mathcal{V}$.

\subsection{Relation between homologism categories for subvarieties}

Consider two subvarieties $\mathcal{V}_1$ and $\mathcal{V}_2$ of the
variety of groups. We can consider the category of groups with
$\mathcal{V}_1$-homologisms as well as the category of groups with
$\mathcal{V}_2$-homologisms. Suppose $\mathcal{V}_1$ is a subvariety
within $\mathcal{V}_2$ (in other words, a group being in
$\mathcal{V}_1$ is a {\em stronger} condition than the group being in
$\mathcal{V}_2$).

We might naively expect that there is a natural forgetful functor of
some sort from the category of groups to $\mathcal{V}_1$-homologisms
and the category of groups with $\mathcal{V}_2$-homologisms. This,
however, is not true. In Section
\ref{sec:homomorphisms-and-homoclinisms}, we discussed the situation
in the case that $\mathcal{V}_1$ is the subvariety comprising only the
trivial group (so that $\mathcal{V}_1$-homologisms can be identified
with ordinary homomorphisms) and $\mathcal{V}_2$ is the subvariety of
abelian groups (so that $\mathcal{V}_2$-homologisms are the same as
homoclinisms). Using the same logic as we had used in Section
\ref{sec:homomorphisms-and-homoclinisms}, we obtain that a
$\mathcal{V}_1$-homologism gives rise to a $\mathcal{V}_2$-homologism
if and only if it sends the $\mathcal{V}_2$-marginal subgroup to
within itself. In particular, any {\em surjective}
$\mathcal{V}_1$-homologism induces a
$\mathcal{V}_2$-homologism. Therefore, every $\mathcal{V}_1$-{\em
  isologism} induces a $\mathcal{V}_2$-isologism.

\subsection{$n$-homoclinism and $n$-isoclinism}\label{appsec:n-homoclinism}

Suppose $n$ is a positive integer. A $n$-homoclinism is defined as a
homologism with respect to the subvariety of the variety of groups
comprising groups of nilpotency class at most $n$. The subvariety is
defined by the word $w(x_1,x_2,\dots,x_n,x_{n+1}) = [[ \dots
    [x_1,x_2],\dots,x_n],x_{n+1}]$. The {\em category of groups with
  $n$-homoclinisms} is the category where the objects are groups and
the homomorphisms are $n$-homoclinisms. A {\em $n$-isoclinism} is
defined as an isomorphism in this category.

The concepts of $n$-isoclinism and $c$-homoclinism are also
described in \cite{Hekster} (for groups) and \cite{Moghaddametal} (for
Lie rings).

\subsection{Baer invariant and nilpotent multiplier}

The Schur multiplier and exterior square are both defined based on the
commutator structure, and their definitions can be viewed as arising
from the choice of the subvariety of abelian groups and the notion of
homoclinism. There exist generalizations of these to arbitrary
subvarieties of the variety of nilpotent groups. The generalization of
the Schur multiplier is termed the {\em Baer invariant}. Explicitly,
for a subvariety $\mathcal{V}$ of the variety of groups, and any group
$G$ (not necessarily in $\mathcal{V}$), the {\em Baer invariant} of
$G$ for the subvariety $\mathcal{V}$ is an abelian group denoted
$\mathcal{V}M(G)$. The definition is similar to that for the Schur
multiplier in Section \ref{sec:exteriorsquare-and-homoclinism}, but
now using the word maps arising from words that define the subvariety
$\mathcal{V}$ instead of the commutator word. Alternatively, it can be
defined by generalizing Hopf's formula (described in Section
\ref{sec:hopf-formula-proofs}) as follows. Interested readers are encouraged
to read the paper \cite{Baerinvariantsandisolosigms} by Leedham-Green
and McKay and the paper \cite{Hekster} by Hekster.

In the case that $\mathcal{V}$ is the subvariety comprising all the
groups of nilpotency class at most $n$ (where $n$ is a fixed positive
integer), the corresponding Baer invariant is termed the $n$-nilpotent
multiplier and is denoted $M^{(n)}(G)$. In particular, the Schur
multiplier $M(G)$ is the group $M^{(1)}(G)$, i.e., it is the
$1$-nilpotent multiplier.

We can deduce a formula for the $n$-nilpotent multiplier similar to
Hopf's formula described in Section \ref{sec:hopf-formula-proofs}, and in
fact, many sources use that formula as the {\em definition} of the
$n$-nilpotent multiplier. Write $G$ as a quotient $F/R$ where $F$ is a
free group. Then:

$$M^{(n)}(G) = (R \cap \gamma_{n +1}(F))/\gamma_{n+1}(R,F)$$

Note that although it is not immediately obvious by looking at the
expression, we can see via the three subgroup lemma that $M^{(n)}(G)$
as defined above is an abelian group, as expected.

Here, $\gamma_{n+1}(F)$ denotes the $(n+1)^{th}$ member of the lower
central series of $F$, whereas $\gamma_{n+1}(R,F)$ is the $(n+1)^{th}$
member of the series given by $\gamma_1(R,F) = R$ and
$\gamma_{i+1}(R,F) = [F,\gamma_i(R,F)]$. Note that in the case $n =
1$, this givees us the formula we are already familiar with:

$$M(G) = (R \cap [F,F])/[R,F]$$

Note in particular that if $G = F/\gamma_{c+1}(F)$ for some positive
integer $c$ (i.e., $G$ is a free class $c$ nilpotent group), we get:

$$M^{(n)}(G) = \gamma_{\max \{ c,n \} + 1}(F)/\gamma_{c+n+1}(F)$$

