\documentclass[10pt]{amsart}

%Packages in use
\usepackage{fullpage, hyperref, vipul, amssymb, setspace, enumerate, pdfpages}

%Title details
\title{Lazard correspondence up to isoclinism: paper planning}
\author{Vipul Naik}

%List of new commands
\newcommand{\Skew}{\operatorname{Skew}}
\newcommand{\ad}{\operatorname{ad}}
\begin{document}
\maketitle
\onehalfspacing

Proposed outline for a journal paper.

Additional lines of research to possibly pursue before writing a
journal paper (these may or may not be included in the paper based on
space considerations):

\begin{itemize}
\item Complete the proof of existence for the Lazard correspondence up
  to isoclinism in the 3-local case.
\item Work out more details about the possibility of generalizing up
  to isologism, either to the point where I have a proof, or to the
  point where I have a clear explanation of what one would need to prove.
\item Work out in more detail some examples for finite $p$-groups of
  class $p$ for odd $p$, to highlight interesting and important
  features of these examples.
\end{itemize}

Section breakup of the paper:

\begin{itemize}
\item A brief section to review preliminaries (definition of
  homoclinism, powered groups etc.) other than those related to the
  Schur multiplier and extension theory.
\item One section that reviews the ideas of exterior square and Schur
  multiplier for groups and Lie rings. This will very briefly cover a
  few importnat ideas from Sections 4-12 of the thesis.
\item One section to review the setup for the Lazard correspondence
  (corresponds to Sections 24-30 of the thesis).
\item One section with the main proofs of the thesis (corresponds to
  Sections 31-37). This will probably be the longest, since details
  for important proofs should be included in full.
\item Sections on applications and possible extensions can be revised
  and improved somewhat compared with the versions in the thesis.
\end{itemize}
\end{document}
