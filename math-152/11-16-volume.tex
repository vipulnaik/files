\documentclass[10pt]{amsart}

%Packages in use
\usepackage{fullpage, hyperref, vipul, enumerate}

%Title details
\title{Class quiz: November 16: Volume}
\author{Math 152, Section 55 (Vipul Naik)}
%List of new commands

\begin{document}
\maketitle

Your name (print clearly in capital letters): $\underline{\qquad\qquad\qquad\qquad\qquad\qquad\qquad\qquad\qquad\qquad}$

\begin{enumerate}
\item Oblique cylinder:Right cylinder:: {\em Last year: $14/16$
  correct}

  \begin{enumerate}[(A)]
  \item Rectangle:Square
  \item Parallelogram:Rectangle
  \item Disk:Circle
  \item Triangle:Rectangle
  \item Triangle:Square
  \end{enumerate}

  \vspace{0.3in}
  Your answer: $\underline{\qquad\qquad\qquad\qquad\qquad\qquad\qquad}$
  \vspace{0.3in}

\item Right circular cone:Right circular cylinder:: {\em Last year:
  $13/16$ correct}

  \begin{enumerate}[(A)]
  \item Triangle:Square
  \item Rectangle:Square
  \item Isosceles triangle:Equilateral triangle
  \item Isosceles triangle:Rectangle
  \item Isosceles triangle:Square
  \end{enumerate}

  
  Your answer: $\underline{\qquad\qquad\qquad\qquad\qquad\qquad\qquad}$
  \vspace{0.3in}

\item Circular disk:Circle:: {\em Last year: $8/16$ correct}

  \begin{enumerate}[(A)]
  \item Hollow cylinder:Solid cylinder
  \item Solid cylinder:Hollow cylinder
  \item Cube:Cuboid (cuboid is a term for rectangular prism)
  \item Cube:Square
  \item Cube:Sphere
  \end{enumerate}

  
  \vspace{0.3in}
Your answer: $\underline{\qquad\qquad\qquad\qquad\qquad\qquad\qquad}$
  \vspace{0.3in}

\item Circular disk:Line segment:: {\em Last year: $14/16$ correct}

  \begin{enumerate}[(A)]
  \item Solid sphere:Circular disk
  \item Circle:Rectangle
  \item Sphere:Cube
  \item Cube:Right circular cylinder
  \item Square:Triangle
  \end{enumerate}

  
  \vspace{0.3in}
Your answer: $\underline{\qquad\qquad\qquad\qquad\qquad\qquad\qquad}$
  \vspace{0.3in}

\newpage
\item Suppose a filled triangle $ABC$ in the plane is revolved about
  the side $AB$. Which of the following best describes the solid of
  revolution thus obtained if both the angles $A$ and $B$ are acute
  (ignoring issues of boundary inclusion/exclusion)?  {\em Last year:
  $13/16$ correct}

  \begin{enumerate}[(A)]
  \item It is a right circular cone.
  \item It is the union of two right circular cones sharing a common
    disk as base.
  \item It is the set difference of two right circular cones sharing a
    common disk as base.
  \item It is the union of two right circular cones sharing a common
    vertex.
  \item It is the set difference of two right circular cones sharing a
    common vertex.
  \end{enumerate}

  
  \vspace{1in}
Your answer: $\underline{\qquad\qquad\qquad\qquad\qquad\qquad\qquad}$
  \vspace{1in}

\item Suppose a filled triangle $ABC$ in the plane is revolved about
  the side $AB$. Which of the following best describes the solid of
  revolution thus obtained if the angle $A$ is obtuse (ignoring issues
  of boundary inclusion/exclusion)? {\em Last year: $9/16$ correct}

  \begin{enumerate}[(A)]
  \item It is a right circular cone.
  \item It is the union of two right circular cones sharing a common
    disk as base.
  \item It is the set difference of two right circular cones sharing a
    common disk as base.
  \item It is the union of two right circular cones sharing a common
    vertex.
  \item It is the set difference of two right circular cones sharing a
    common vertex.
  \end{enumerate}

  
  \vspace{1in}
Your answer: $\underline{\qquad\qquad\qquad\qquad\qquad\qquad\qquad}$
  \vspace{1in}

\item What is the volume of the solid of revolution obtained by
  revolving the filled triangle $ABC$ about the side $AB$, if the
  length of the base $AB$ is $b$ and the height corresponding to this
  base is $h$? {\em Last year: $10/16$ correct}

  \begin{enumerate}[(A)]
  \item $(1/6) \pi b^{3/2}h^{3/2}$
  \item $(1/3) \pi b^2h$
  \item $(1/3) \pi bh^2$
  \item $(2/3) \pi b^2h$
  \item $(2/3) \pi bh^2$
  \end{enumerate}
  
  Your answer: $\underline{\qquad\qquad\qquad\qquad\qquad\qquad\qquad}$
  \vspace{1in}

  For the next two questions, suppose $\Omega$ is a region in a plane
  $\Pi$ and $\ell$ is a line on $\Pi$ such that $\Omega$ lies
  completely on one side of $\ell$ (in particular, it does not
  intersect $\ell$). Let $\Gamma$ be the solid of revolution obtained
  by revolving $\Omega$ about $\ell$. Suppose further that the
  intersection of $\Omega$ with any line perpendicular to $\ell$ is
  either empty or a point or a line segment.
\item (*) What is the intersection of $\Gamma$ with $\Pi$ (your answer
  should be always true)? {\em Last year: $6/16$ correct}

  \begin{enumerate}[(A)]
  \item It is precisely $\Omega$.
  \item It is the union of $\Omega$ and a translate of $\Omega$ along
    a direction perpendicular to $\ell$.
  \item It is the union of $\Omega$ and the reflection of $\Omega$
    about $\ell$.
  \item It is either empty or a rectangle whose dimensions depend on
    $\Omega$.
  \item It is either empty or a circle or circular disk or an annulus
    whose inner and outer radius depend on $\Omega$.
  \end{enumerate}

  
  \vspace{1in}
Your answer: $\underline{\qquad\qquad\qquad\qquad\qquad\qquad\qquad}$
  \vspace{1in}

\item What is the intersection of $\Gamma$ with a plane perpendicular
  to $\ell$ (your answer should be always true)? {\em Last year:
  $9/16$ correct}

  \begin{enumerate}[(A)]
  \item It is precisely $\Omega$.
  \item It is the union of $\Omega$ and a translate of $\Omega$ along
    a direction perpendicular to $\ell$.
  \item It is the union of $\Omega$ and the reflection of $\Omega$
    about $\ell$.
  \item It is either empty or a rectangle whose dimensions depend on
    $\Omega$.
  \item It is either empty or a circle or an annulus whose inner and
    outer radius depend on $\Omega$.
  \end{enumerate}

  
  \vspace{1in}
Your answer: $\underline{\qquad\qquad\qquad\qquad\qquad\qquad\qquad}$
  \vspace{1in}

\newpage
\item (*) Consider a fixed equilateral triangle $ABC$. Now consider,
  for any point $D$ outside the plane of $ABC$, the solid tetrahedron
  $ABCD$. This is the solid bounded by the triangles $ABC$, $BCD$,
  $ACD$, and $ABD$. The volume of this solid depends on $D$. What
  specific information about $D$ completely determines the volume?
  {\em Last year: $7/16$ correct}

  \begin{enumerate}[(A)]
  \item The perpendicular distance from $D$ to the plane of the
    triangle $ABC$.
  \item The minimum of the distances from $D$ to points in the filled
    triangle $ABC$.
  \item The location of the point $E$ in the plane of triangle $ABC$
    that is the foot of the perpendicular from $D$ to $ABC$.
  \item The distance from $D$ to the center of $ABC$ (here, you can
    take the center as any of the notions of center since $ABC$ is
    equilateral).
  \item None of the above.
  \end{enumerate}

  
  \vspace{1in}
  Your answer: $\underline{\qquad\qquad\qquad\qquad\qquad\qquad\qquad}$
  \vspace{1in}

\item (**) For $r > 0$, consider the region $\Omega_r(a)$ bounded by
  the $x$-axis, the curve $y = x^{-r}$, and the lines $x = 1$ and $x =
  a$ with $a > 1$. Let $V_r(a)$ be the volume of the region obtained
  by revolving $\Omega_r(a)$ about the $x$-axis. What is the precise
  set of values of $r$ for which $\lim_{a \to \infty} V_r(a)$ is
  finite? {\em Last year: $3/16$ correct}
 
  \begin{enumerate}[(A)]
  \item All $r > 0$
  \item $r > 1/2$
  \item $r > 1$
  \item $r > 2$
  \item No value of $r$
  \end{enumerate}

  \vspace{1in}
Your answer: $\underline{\qquad\qquad\qquad\qquad\qquad\qquad\qquad}$
  

\end{enumerate}
\end{document}