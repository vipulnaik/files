\documentclass[10pt]{amsart}
\usepackage{fullpage,hyperref,vipul, graphicx}
\title{Review sheet for final: basic}
\author{Math 196, Section 57 (Vipul Naik)}

\begin{document}
\maketitle

We will not be going over this sheet, but rather, we'll be going over
the advanced review sheet in the session. Please review this sheet on
your own time.

The summaries here are identical with the executive summaries you will
find at the beginning of the respective lecture notes PDF files. The
summaries are not intended to be exhaustive. Please review the
original lecture notes as well, especially if any point in the summary
is unclear.

\section{Linear dependence, bases, and subspaces}

\begin{enumerate}
\item A {\em linear relation} between a set of vectors is defined as a
  linear combination of these vectors that is zero. The {\em trivial}
  linear relation refers to the trivial linear combination being
  zero. A nontrivial linear relation is any linear relation other than
  the trivial one.
\item The trivial linear relation exists between any set of vectors.
\item A set of vectors is termed {\em linearly dependent} if there
  exists a nontrivial linear relation between them, and {\em linearly
    independent} otherwise.
\item Any set of vectors containing a linearly dependent subset is
  also linearly dependent. Any subset of a linearly independent set of
  vectors is a linearly independent set of vectors.
\item The following can be said of sets of small size:
  \begin{itemize}
  \item The empty set (the only possible set of size zero) is
    considered linearly independent.
  \item A set of size one is linearly dependent if the vector is the
    zero vector, and linearly independent if the vector is a nonzero
    vector.
  \item A set of size two is linearly dependent if either one of the
    vectors is the zero vector or the two vectors are scalar multiples
    of each other. It is linearly independent if both vectors are
    nonzero and they are not scalar multiples of one another.
  \item For sets of size three or more, a {\em necessary} condition
    for linear independence is that no vector be the zero vector and
    no two vectors be scalar multiplies of each other. However, this
    condition is not sufficient, because we also have to be on the
    lookout for other kinds of linear relations.
  \end{itemize}
\item Given a nontrivial linear relation between a set of vectors, we
  can use the linear relation to write one of the vectors (any vector
  with a nonzero coefficient in the linear relation) as a linear
  combination of the other vectors.
\item We can use the above to prune a spanning set as follows: given a
  set of vectors, if there exists a nontrivial linear relation between
  the vectors, we can use that to write one vector as a linear
  combination of the others, and then remove it from the set {\em
    without affecting the span}. The vector thus removed is termed a
  {\em redundant vector}.
\item A {\em basis} for a subspace of $\R^n$ is a linearly independent
  spanning set for that subspace. Any finite spanning set can be
  pruned down (by repeatedly identifying linear relations and removing
  vectors) to reach a basis.
\item The size of a basis for a subspace of $\R^n$ depends only on the
  choice of subspace and is {\em independent} of the choice of
  basis. This size is termed the {\em dimension} of the subspace.
\item Given an ordered list of vectors, we call a vector in the list
  {\em redundant} if it is redundant relative to the preceding
  vectors, i.e., if it is in the span of the preceding vectors, and
  {\em irredundant} otherwise. The irredundant vectors in any given
  list of vectors form a basis for the subspace spanned by those
  vectors.
\item Which vectors we identify as redundant and irredundant depends
  on how the original list was ordered. However, the {\em number} of
  irredundant vectors, insofar as it equals the dimension of the span,
  does not depend on the ordering.
\item If we write a matrix whose column vectors are a given list of
  vectors, the linear relations between the vectors correspond to
  vectors in the kernel of the matrix. Injectivity of the linear
  transformation given by the matrix is equivalent to linear
  independence of the vectors.
\item Redundant vector columns correspond to non-leading variables and
  irredundant vector columns correspond to leading variables if we
  think of the matrix as a coefficient matrix. We can row-reduce to
  find which variables are leading and non-leading, then look at the
  irredundant vector columns in the {\em original} matrix.
\item {\em Rank-nullity theorem}: The nullity of a linear transformation
  is defined as the dimension of the kernel. The nullity is the number
  of non-leading variables. The rank is the number of leading
  variables. So, the sum of the rank and the nullity is the number of
  columns in the matrix for the linear transformation, aka the
  dimension of the domain. See Section 3.7 of the notes for more details.
\item The problem of finding all the vectors orthogonal to a given set
  of vectors can be converted to solving a linear system where the
  rows of the coefficient matrix are the given vectors.
\end{enumerate}

\section{Coordinates (includes discussion of similarity of linear transformations)}

\begin{enumerate}
\item Given a basis $\mathcal{B} =
  (\vec{v}_1,\vec{v}_2,\dots,\vec{v}_m)$ for a subspace $V \subseteq
  \R^n$ (note that this forces $m \le n$), every vector $\vec{x} \in
  V$ can be written in a unique manner as a linear combination of the
  basis vectors $\vec{v}_1,\vec{v}_2,\dots,\vec{v}_m$. The fact that
  there exists a way of writing it as a linear combination follows
  from the fact that $\mathcal{B}$ spans $V$. The uniqueness follows
  from the fact that $\mathcal{B}$ is linearly independent. The
  coefficients for the linear combination are called the {\em
    coordinates} of $\vec{x}$ in the basis $\mathcal{B}$.
\item Continuing notation from point (1), finding the coordinates
  amounts to solving the linear system with coefficient matrix columns
  given by the basis vectors $\vec{v}_1,\vec{v}_2,\dots,\vec{v}_m$ and
  the augmenting column given by the vector $\vec{x}$. The linear
  transformation of the matrix is injective, because the vectors are
  linearly independent. The matrix, a $n \times m$ matrix, has full
  column rank $m$. The system is consistent if and only if $\vec{x}$ is
  actually in the span, and injectivity gives us uniqueness of the
  coordinates.
\item A canonical example of a basis is the {\em standard} basis, which
  is the basis comprising the standard basis vectors, and where the
  coordinates are the usual coordinates.
\item Continuing notation from point(1), in the special case that $m =
  n$, $V = \R^n$. So the basis is $\mathcal{B} =
  (\vec{v}_1,\vec{v}_2,\dots,\vec{v}_n)$ and it is an alternative
  basis for all of $\R^n$ (here, alternative is being used to contrast
  with the standard basis; we will also use ``old basis'' to refer to
  the standard basis and ``new basis'' to refer to the alternative
  basis). In this case, the matrix $S$ whose columns are the basis
  vectors $\vec{v}_1, \vec{v}_2, \dots, \vec{v}_n$ is a $n \times n$
  square matrix and is invertible. We will denote this matrix by $S$
  (following the book).
\item Continuing notation from point (4), if we denote by
  $[\vec{x}]_{\mathcal{B}}$ the coordinates of $\vec{x}$ in the new
  basis, then $[\vec{x}]_{\mathcal{B}} = S^{-1}\vec{x}$ and $\vec{x} =
  S[\vec{x}]_{\mathcal{B}}$.
\item For a linear transformation $T$ with matrix $A$ in the standard
  basis and matrix $B$ in the new basis, then $B = S^{-1}AS$ or
  equivalently $A = SBS^{-1}$. The $S$ on the right involves first
  converting from the new basis to the old basis, then we do the
  middle operation $A$ on the old basis, and then we do $S^{-1}$ to
  re-convert to the new basis.
\item If $A$ and $B$ are $n \times n$ matrices such that there exists
  an invertible $n \times n$ matrix $S$ satisfying $B = S^{-1}AS$, we
  say that $A$ and $B$ are {\em similar} matrices. Similar matrices
  have the same trace, determinant, and behavior with respect to
  invertibility and nilpotency. Similarity is an equivalence relation,
  i.e., it is reflexive, symmetric, and transitive.
\item Suppose $S$ is an invertible $n \times n$ matrix. The
  conjugation operation $X \mapsto SXS^{-1}$ from $\R^{n \times n}$ to
  $\R^{n \times n}$ preserves addition, scalar multiplication,
  multiplication, and inverses.
\end{enumerate}

\section{Abstract vector spaces and the concept of isomorphism}

General stuff ...

\begin{enumerate}
\item There is an abstract definition of real vector space that
  involves a set with a binary operation playing the role of addition
  and another operation playing the role of scalar multiplication,
  satisfying a bunch of axioms. The goal is to axiomatize the key
  aspects of vector spaces.
\item A subspace of an abstract vector space is a subset that contains
  the zero vector and is closed under addition and scalar
  multiplication.
\item A linear transformation is a set map between two vector spaces
  that preserves addition and preserves scalar multiplication. It also
  sends zero to zero, but this follows from its preserving scalar
  multiplication.
\item The {\em kernel} of a linear transformation is the subset of the
  domain comprising the vectors that map to zero. The kernel of a
  linear transformation is always a subspace.
\item The {\em image} of a linear transformation is its range as a set
  map. The image is a subspace of the co-domain.
\item The {\em dimension} of a vector space is defined as the size of
  any basis for it. The dimension provides an upper bound on the size
  of any linearly independent set in the vector space, with the upper
  bound attained (in the finite case) only if the linearly independent
  set is a basis. The dimension also provides a lower bound on the
  size of any spanning subset of the vector space, with the lower
  bound being attained (in the finite case) only if the spanning set
  is a basis.
\item Every vector space has a particular subspace of interest: the
  zero subspace.
\item The {\em rank} of a linear transformation is defined as the
  dimension of the image. The rank is the answer to the question:
  ``how much survives the linear transformation?''
\item The {\em nullity} of a linear transformation is defined as the
  dimension of the kernel. The nullity is the answer to the question:
  ``how much gets killed under the linear transformation?''
\item The sum of the rank and the nullity of a linear transformation
  equals the dimension of the domain. This fact is termed the {\em
    rank-nullity theorem}.
\item We can define the {\em intersection} and {\em sum} of
  subspaces. These are again subspaces. The intersection of two
  subspaces is defined as the set of vectors that are present in both
  subspaces. The sum of two subspaces is defined as the set of vectors
  expressible as a sum of vectors, one in each subspace. The sum of
  two subspaces also equals the subspace spanned by their union.
\item A linear transformation is {\em injective} if and only if its
  kernel is the zero subspace of the domain.
\item A linear transformation is {\em surjective} if and only if its
  image is the whole co-domain.
\item A {\em linear isomorphism} is a linear transformation that is
  {\em bijective}: it is both injective and surjective. In other
  words, its kernel is the zero subspace of the domain and its image
  is the whole co-domain.
\item The dimension is an isomorphism-invariant. It is in fact a {\em
  complete isomorphism-invariant}: two real vector spaces are
  isomorphic if and only if they have the same dimension. Explicitly,
  we can use a bijection between a basis for one space and a basis for
  another. In particular, any $n$-dimensional space is isomorphic to
  $\R^n$. Thus, by studying the vector spaces $\R^n$, we have
  effectively studied all finite-dimensional vector spaces up to
  isomorphism.
\end{enumerate}

Function spaces ...

\begin{enumerate}
\item For any set $S$, consider the set $F(S,\R)$ of {\em all}
  functions from $S$ to $\R$. With pointwise addition and scalar
  multiplication of functions, this set is a vector space over
  $\R$. If $S$ is finite ({\em not} our main case of interest) this
  space has dimension $|S|$ and is indexed by a basis of $S$. We are
  usually interested in {\em subspaces} of this space.
\item We can define vector spaces such as $\R[x]$ (the vector space of
  all polynomials in one variable with real coefficients) and $\R(x)$
  (the vector space of all rational functions in one variable with
  real coefficients). These are both infinite-dimensional spaces. We
  can study various finite-dimensional subspaces of these. For
  instance, we can define $P_n$ as the vector space of all polynomials
  of degree less than or equal to $n$. This is a vector space of
  dimension $n + 1$ with basis given by the monomials
  $1,x,x^2,\dots,x^n$.
\item There is a natural injective linear transformation $\R[x] \to
  F(\R,\R)$.
\item Denote by $C(\R)$ or $C^0(\R)$ the subspace of $F(\R,\R)$
  comprising the functions that are continuous everywhere. For $k$ a
  positive integer, denote by $C^k(\R)$ the subspace of $C(\R)$
  comrpising those functions that are at least $k$ times continuously
  differentiable, and denote by $C^\infty(\R)$ the subspace of $C(\R)$
  comprising all the functions that are {\em infinitely}
  differentiable. We have a descending chain of subspaces:

  $$C^0(\R) \supseteq C^1(\R) \supseteq C^2(\R) \supseteq \dots $$

  The image of $\R[x]$ inside $F(\R,\R)$ lands inside $C^\infty(\R)$.

\item We can view differentiation as a linear transformation $C^1(\R)
  \to C(\R)$. It sends each $C^k(\R)$ to $C^{k-1}(\R)$. It is
  surjective from $C^\infty(\R)$ to $C^\infty(\R)$. The kernel is
  constant functions, and the kernel of $k$-fold iteration is
  $P_{k-1}$. Differentiation sends $\R[x]$ to $\R[x]$ and is
  surjective to $\R[x]$.
\item We can also define a formal differentiation operator $\R(x) \to
  \R(x)$. This is not surjective.
\item Partial fractions theory can be formulated in terms of saying
  that some particular rational functions form a basis for certain
  finite-dimensional subspaces of the space of rational functions, and
  exhibiting a method to find the ``coordinates'' of a rational
  function in terms of this basis. The advantage of expressing in this
  basis is that the basis functions are particularly easy to integrate.
\item We can define a vector space of sequences. This is a special
  type of function space where the domain is $\mathbb{N}$. In other
  words, it is the function space $F(\mathbb{N},\R)$.
\item We can define a vector space of formal power series. The Taylor
  series operator and series summation operator are back-and-forth
  operators between this vector space (or an appropriate subspace
  therefore) and $C^\infty(\R)$.
\item Formal differentiation is a linear transformation $\R[[x]] \to
  \R[[x]]$. It is surjective but not injective. The kernel is the
  one-dimensional space of formal power series.
\item We can consider linear differential operators from
  $C^\infty(\R)$ to $C^\infty(\R)$. These are obtained by combining
  the usual differentiation operator and multiplication operators
  using addition, multiplication (composition) and scalar
  multiplication. Finding the kernel of a linear differential operator
  is equivalent to solving a homogeneous linear differential
  equation. Finding the inverse image of a particular function under a
  linear differential operator amounts to solving a non-homogeneous
  linear differential equation, and the solution set here is a
  translate of the kernel (the corresponding solution in the
  homogeneous case, also called the {\em auxilliary solution}) by a
  particular solution. The first-order case is particularly
  illuminative because we have an explicit formula for the fibers.
\end{enumerate}

\section{Ordinary least squares regression}

Words ...

\begin{enumerate}
\item Consider a model where the general functional form is linear in
  the parameters. Input-output pairs give a system of simultaneous
  linear equations in terms of the parameters. Each row of the
  coefficient matrix corresponds to a particular choice of input, and
  each corresponding entry of the augmenting column is the
  corresponding output. In the ``no-error'' case, what we would
  ideally like is that the coefficient matrix has full column rank
  (i.e., we get unique solutions for the parameters assuming
  consistency) and does {\em not} have full row rank (i.e., we have
  some extra input-output pairs so that consistency can be used as
  evidence in favor of the hypothesis that the given functional form
  is correct). If the model is correct, the system will be consistent
  (despite potential for inconsistency) and we will be able to deduce
  the values of the parameters.
\item Once we introduce measurement error, we can no longer find the
  parameters with certainty. However, what we {\em can} hope for is to
  provide a ``best guess'' for the parameter values based on the given
  data points (input-output pairs).
\item In the case where we have measurement error, we still aim to
  choose a large number of inputs so that the coefficient matrix has
  full column rank but does not have full row rank. Now, however, even
  if the model is correct, the system will probably be
  inconsistent. What we need to do is to replace the existin output
  vector (i.e., the existing augmenting column) by the vector closest
  to it that is in the image of the linear transformation given by the
  coefficient matrix. Explicitly, if $\vec{\beta}$ is the parameter
  vector that we are trying to find, $X$ is the coefficient matrix
  (also called the design matrix), and $\vec{y}$ is the output vector,
  the system $X\vec{\beta} = \vec{y}$ may not be consistent. We try to
  find a vector $\vec{\varepsilon}$ of minimum length subject to the
  constraint that $\vec{y} - \vec{\varepsilon}$ is in the image of the
  linear transformation given by $X$, so that the system $X\vec{\beta}
  = \vec{y} - \vec{\varepsilon}$ is consistent. Because in our setup
  (if we chose it well), $X$ had full column rank, this gives the
  unique ``best''choice of $\vec{\beta}$. Note also that ``length''
  here refers to Euclidean length (square root of sum of squares of
  coordinates) when we are doing an {\em ordinary least squares
    regression} (the default type of regression) but we could use
  alternative notions of length in other types of regressions.
\item In the particular case that the system $X \vec{\beta} = \vec{y}$
  is consistent, we choose $\vec{\varepsilon} = \vec{0}$. However,
  this does not mean that ths is the actual correct parameter
  vector. It is still only a guess.
\item In general, the more data points (i.e., input-output pairs) that
  we have, the better our guess becomes. However, this is true only in
  a probabilistic sense. It may well happen that a particular data
  point worsens our guess because that data point has a larger error
  than the others.
\item The corresponding geometric operation to finding the vector
  $\vec{\varepsilon}$ is orthogonal projection. Explicitly, the image
  of $X$ is a subspace of the vector space $\R^n$ (where $n$ is the
  number of input-output pairs). If there are $m$ parameters (i.e.,
  $\vec{\beta} \in \R^m$). and we chose $X$ wisely, the image of $X$
  would be a $m$-dimensional subspace of $\R^n$. In the no-error case,
  the vector $\vec{y}$ would be in this subspace, and we would be able
  to find $\vec{\beta}$ uniquely and correctly. In the presence of
  error, $\vec{y}$ may be outside the subspace. The vector $\vec{y} -
  \vec{\varepsilon}$ that we are looking for is the orthogonal
  projection of $\vec{y}$ onto this subspace. The error vector
  $\vec{\varepsilon}$ is the component of $\vec{y}$ that is
  perpendicular to the subspace.
\item A fit is impressive in the case that $m$ is much smaller than
  $n$ and yet the error vector $\vec{\varepsilon}$ is small. This is
  philosophically for the same reason that consistency becomes more
  impressive the greater the excess of input-output pairs over
  parameters. Now, the rigid notion of consistency has been replaced
  by the more loose notion of ``small error vector.''
\end{enumerate}

Actions ...

\begin{enumerate}
\item Solving $X\vec{\beta} = \vec{y} - \vec{\varepsilon}$ with
  $\vec{\varepsilon}$ (unknown) as the vector of minimum possible
  length is equivalent to solving $X^TX\vec{\beta} = X^T\vec{y}$. Note
  that this process does not involve finding the error vector
  $\vec{\varepsilon}$ directly. The matrix $X^TX$ is a square matrix
  that is symmetric.
\item In the case that $X$ has full column rank (i.e., we have unique
  solutions {\em if} consistent), $X^TX$ also has full rank (both full
  row rank and full column rank -- it is a square matrix), and we get a
  unique ``best fit'' solution.
\end{enumerate}

\end{document}
