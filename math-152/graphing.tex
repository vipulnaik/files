\documentclass[10pt]{amsart}
\usepackage{fullpage,hyperref,vipul,graphicx}
\title{Graphing}
\author{Math 152, Section 55 (Vipul Naik)}

\begin{document}
\maketitle

{\bf Corresponding material in the book}: Section 4.8

{\bf Difficulty level}: Hard.

{\bf What students should definitely get}: The main concerns in
graphing a function, how to figure out what needs figuring out. It is
important for students to go through all the graphing examples in the
book and do more hands-on practice. Transformations of graphs. Quickly
graphing constant, linear, quadratic graphs.

{\bf What students should hopefully get}: How all the issues of
symmetry, concavity, inflections, periodicity, and derivative signs
fit together in the grand scheme of graphing. The qualitative
characteristics of polynomial function and rational function graphs,
as well as graphs involving a mix of trigonometric and polynomial
functions.

{\bf Weird feature}: Ironically, there are very few pictures in this
document. The naive explanation is that I didn't have time to add many
pictures. The more sophisticated explanation is that since the purpose
here is to review how to graph functions, having actual pictures drawn
perfectly is counterproductive. Please keep a paper and pencil handy
and sketch pictures as you feel the need.

\section*{Executive summary}

\subsection{Symmetry yet again}

Words...

\begin{enumerate}
\item All mathematics is the study of symmetry (well, not all).
\item One interesting kind of symmetry that we often see in the graph
  of a function is {\em mirror symmetry} about a vertical line. This
  means that the graph of the function equals its reflection about the
  vertical line. If the vertical line is $x = c$ and the function is
  $f$, this is equivalent to asserting that $f(x) = f(2c - x)$ for all
  $x$ in the domain, or equivalently, $f(c + h) = f(c - h)$ whenever
  $c + h$ is in the domain. In particular, the domain itself must be
  symmetric about $c$.
\item A special case of mirror symmetry is the case of an {\em even
  function}. An even function is a function with mirror symmetry about
  the $y$-axis. In other words, $f(x) = f(-x)$ for all $x$ in the
  domain. (Even also implies that the domain should be symmetric about $0$).
\item Another interesting kind of symmetry that we often see in the
  graph of a function is {\em half-turn symmetry} about a point on the
  graph. This means that the graph equals the figure obtained by
  rotating it by an angle of $\pi$ about that point. A point $(c,d)$
  is a point of half-turn symmetry if $f(x) + f(2c - x) = 2d$ for all
  $x$ in the domain. In particular, the domain itself must be
  symmetric about $c$. If $f$ is defined at $c$, then $d = f(c)$.
\item A special case of half-turn symmetry is an odd function, which
  is a function having half-turn symmetry about the origin.
\item Another symmetry is {\em translation symmetry}. A function is
  {\em periodic} if there exists $h > 0$ such that $f(x + h) = f(x)$
  for all $x$ in the domain of the function (in particular, the domain
  itself should be invariant under translation by $h$). If a smallest
  such $h$ exists, then such an $h$ is termed the period of $f$.
\item A related notion is that of a function with {\em periodic
  derivative}. If $f$ is differentiable for all real numbers, and $f'$
  is periodic with period $h$, then $f(x + h) - f(x)$ is constant. If
  this constant value is $k$, then the graph of $f$ has a
  two-dimensional translational symmetry by $(h,k)$ and its multiples.
\end{enumerate}

Cute facts...

\begin{enumerate}
\item Constant functions enjoy mirror symmetry about every vertical
  line and half-turn symmetry about every point on the graph (can't
  get better).
\item Nonconstant linear functions enjoy half-turn symmetry about
  every point on their graph. They do not enjoy any mirror symmetry
  because they are everywhere increasing or everywhere decreasing.
\item Quadratic (nonlinear) functions enjoy mirror symmetry about the
  line passing through the vertex (which is the unique absolute
  maximum/minimum, depending on the sign of the leading
  coefficient). They do not enjoy any half-turn symmetry.
\item Cubic functions enjoy half-turn symmetry about the point of
  inflection, and no mirror symmetry. Either the first derivative does
  not change sign anywhere, or it becomes zero at exactly one point,
  or there is exactly one local maximum and one local minimum,
  symmetric about the point of inflection.
\item Functions of higher degree do not necessarily have either
  half-turn symmetry or mirror symmetry.
\item More generally, we can say the following for sure: a nonconstant
  polynomial of even degree greater than zero can have at most one
  line of mirror symmetry and no point of half-turn symmetry. A
  nonconstant polynomial of odd degree greater than one can have at
  most one point of half-turn symmetry and no line of mirror symmetry.
\item If a function is continuously differentiable and the first
  derivative has only finitely many zeros in any bounded interval,
  then the intersection of its graph with any vertical line of mirror
  symmetry is a point of local maximum or local minimum. The converse
  does not hold, i.e., points where local extreme values are attained
  do {\em not} usually give axes of mirror symmetry.
\item If a function is twice differentiable and the second derivative
  has only finitely many zeros in any bounded interval, then any point
  of half-turn symmetry is a point of inflection. The converse does
  not hold, i.e., points of inflection do {\em not} usually give rise
  to half-turn symmetries.
\item The sine function is an example of a function where the points
  of inflection and the points of half-turn symmetry are the same: the
  multiples of $\pi$. Similarly, the points with vertical axis of
  symmetry are the same as the points of local extrema: odd multiples
  of $\pi/2$.
\item For a periodic function, any translate by a multiple of the
  period of a point of half-turn symmetry is again a point of
  half-turn symmetry. (In fact, any translate by a multiple of half
  the period is also a point of half-turn symmetry).
\item For a periodic function, any translate by a multiple of the
  period of an axis of mirror symmetry is also an axis of mirror
  symmetry. (In fact, translation by multiples of half the period also
  preserve mirror symmetry).
\item A polynomial is an even function iff all its terms have even
  degree. Such a polynomial is termed an {\em even polynomial}. A
  polynomial is an odd function iff all its terms have odd
  degree. Such a polynomial is termed an {\em odd polynomial}.
\item Also, the derivative of an even function (if it exists) is odd;
  the derivative of an odd function (if it exists) is even.
\end{enumerate}

Actions ...

\begin{enumerate}
\item Worried about periodicity? Don't be worried if you only see
  polynomials and rational functions. Trigonometric functions should
  make you alert. Try to fit in the nicest choices of period. Check if
  smaller periods can work (e.g., for $\sin^2$, the period is
  $\pi$). Even if the function in and of itself is not periodic, it
  might have a periodic derivative or a periodic second
  derivative. The sum of a linear function and a periodic function has
  periodic derivative, and the sum of a quadratic function and a
  periodic function has a periodic second derivative.
\item Want to milk periodicity? Use the fact that for a periodic
  function, the behavior everywhere is just the behavior over one
  period translates over and over again. If the first derivative is
  periodic, the increase/decrease behavior is periodic. If the second
  derivative is periodic, the concave up/down behavior is periodic.
\item Worried about even and odd, and half-turn symmetry and mirror
  symmetry? If you are dealing with a quadratic polynomial, or a
  function constructed largely from a quadratic polynomial, you are
  probably seeing some kind of mirror symmetry. For cubic polynomials
  and related constructions, think half-turn symmetry.
\item Use also the cues about even and odd polynomials.
\end{enumerate}

\subsection{Graphing a function}

Actions ...

\begin{enumerate}
\item To graph a function, a useful first step is finding the domain
  of the function.
\item It is useful to find the intercepts and plot a few additional points.
\item Try to look for symmetry: even, odd, periodic, mirror symmetry,
  half-turn symmetry, and periodic derivative.
\item Compute the derivative. Use that to find the critical points,
  the local extreme values, and the intervals where the function
  increases and decreases.
\item Compute the second derivative. Use that to find the points of
  inflection and the intervals where the function is concave up and
  concave down.
\item Look for vertical tangents and vertical cusps. Look for vertical
  asymptotes and horizontal asymptotes. For this, you may need to
  compute some limits.
\item Connect the dots formed by the points of interest. Use the
  information on increase/decrease and concave up/down to join these
  points. To make your graph a little better, compute the first
  derivative (possibly one-sided) at each of these points and start
  off your graph appropriately at that point.
\end{enumerate}

Subtler points... (see the ``More on graphing'' notes for an
elaboration of these points; not all of them were covered in class):

\begin{enumerate}
\item When graphing a function, there may be many steps where you need
  to do some calculations and solve equations and you are unable to
  carry them out effectively. You can skip some of the steps and come
  back to them later.
\item If you cannot solve an equation exactly, try to approximate the
  locations of roots using the intermediate value theorem or other
  results such as Rolle's theorem.
\item In some cases, it is helpful to graph multiple functions
  together, on the same graph. For instance, we may be interested in
  graphing a function and its second and higher derivatives. There are
  other examples, such as graphing a function and its translates, or a
  function and its multiplicative shifts.
\item A graph can be used to suggest things about a function that are
  not obvious otherwise. However, the graph should not be used as
  conclusive evidence. Rather, the steps used in drawing the graph
  should be retraced and used to give an algebraic proof.
\item We are sometimes interested in sketching curves that are not
  graphs of functions. This can be done by locally expressing the
  curve piecewise as the graph of a function. Or, we could use many
  techniques similar to those for graphing functions.
\item For a function with a piecewise description, we plot each piece
  within its domain. At the points where the definition changes,
  determine the one-sided limits of the function and its first and
  second derivatives. Use this to make the appropriate open circles,
  asymptotes, etc.
\end{enumerate}

\section{Graphing in general}

The goal of this lecture is to make you more familiar with the tools
and techniques that can be used to graph a function. The book has a
list of points that you should keep in mind. The list in the book
isn't complete -- there are a number of additional points that tend
to come up for functions of particular kinds, but it is a good
starting point. But in this lecture, we'll focus on something more
than just the techniques -- we'll focus on the broad picture of why we
want to draw graphs and what information about the function we want
the graph to convey. Working from that, we will be able to reconstruct
much of the book's strategy.

\subsection{Graphs -- utility, sketching and plotting}

The graph of a function $f$ on a subset of the real numbers is the set
of points in $\R^2$ (the plane) of the form $(x,f(x))$, where $x$ is
in the domain of $f$. The graph of $f$ gives a geometric description
of $f$, and it completely determines $f$. For a given $x = x_0$,
$f(x_0)$ is the $y$-coordinate of the unique point of the graph that
is also on the line $x = x_0$.

Graphs are useful because they allow us to see many things about the
function at the same time, and enable us to use our visual instincts
to answer questions about the function. It is usually easy to look at
the graph and spot, without precise measurement, phenomena such as
periodicity, symmetry, increase, decrease, discontinuity, change in
direction, etc. Thus, the graph of a function, {\em if correctly
drawn}, is not only equivalent in information content to the function
itself, it makes that information content much more easy to read.

The problem is with the caveat {\em if correctly drawn}. The domains
of most of the functions we consider are unions of intervals, so they
contain infinitely many points. {\em Plotting the graph} in a complete
sense would involve evaluating the function at these infinitely many
points. In practice, {\em graph plotting} works by dividing the domain
into very small intervals (say, of length $10^{-3}$), calculating the
values of the function (up to some level of accuracy, say $10^{-4}$)
at the endpoints of the intervals, and then drawing a curve that
passes through all the graph points thus obtained. This last joining
step is typically done using straight line segments.\footnote{If you
have seen computer graphics in the old days where computer memory and
processing speed was limited, you would have seen that computer
renderings of geometric figures such as circles was done using small
line segments. As we improve the resolution, the line segments become
smaller and smaller until our eyes cannot make out the difference.}

Unfortunately, although softwares such as Mathematica are good for
plotting graphs, we humans would take too long to do the millions of
evaluations necessary to plot graphs. However, we have another asset,
which is our brains. We need to use our brains to find some substitute
for plotting the graph that still gives a reasonable approximation of
the graph and {\em captures the qualitative characteristics that make
the graph such an informative representation of the function}. The
process that we perform is called {\em graph sketching}.

A sketch of a graph is good if any information that is visually
compelling from the sketch (without requiring precise measurement) is
actually {\em correct} for the function. In other words, a good sketch
may mislead people into thinking that $f(2) = 2.4$ while it is
actually $2.5$, but it should not make people think that $f$ is
increasing on the interval $(2,3)$ if it is actually decreasing on the
interval.\footnote{This does raise an interesting point, which is that
the reason why sketches seem adequate even when inaccurate is because
of our limited observational power -- the correctly plotted graph
would not look too different in terms of compelling visual
information, and hence, we find the sketch good enough for our
purposes.}

\subsection{The domain of a function}

The domain of a function is easy to determine from its graph. Namely,
the domain is the subset of the $x$-axis obtained by orthogonally
projecting the graph onto the $x$-axis. In other words, it is the set
of possible $x$-coordinates of points on the graph.

So, the first step in drawing the graph is finding the domain. We
consider two main issues here:

\begin{enumerate}
\item Sometimes, the domain may contain an open interval without
  containing one or both of the endpoints of that interval. In other
  words, there may be points in the boundary of the domain but not in
  the domain. In such cases, try to determine the limits (left and/or
  right, as applicable) of the function at these boundary points. If
  finite, we have open circles. If equal to $+\infty$ or $-\infty$, we
  have vertical asymptotes.
\item In cases where the domain of the function stretches to $+\infty$
  and/or $-\infty$, determine the limit(s). Any finite limit thus
  obtained corresponds to a horizontal asymptote.
\end{enumerate}

\subsection*{Intercepts and a bit of plotting}

So that the graph is not completely wrong, it is helpful to make it
realistic using a bit of plotting. The book suggests computing the
$x$-intercepts and the $y$-intercept.

The $x$-intercepts are the points where the graph intersects the
$x$-axis, i.e., the points of the form $(x,0)$ where $f(x) = 0$. There
may be zero, one, or more than one $x$-intercepts. The $y$-intercept
is the unique point where the graph intersects the $y$-axis, i.e., the
point $(0,f(0))$. Note that if $0$ is not in the domain of the
function, then the $y$-intercept does not make sense.

In addition to finding the intercepts, it may also be useful to do a
bit of plotting, e.g., finding $f(x)$ for some values of $x$, or
finding solutions to $f(x) = y$ for a few values of $y$. The
intercepts are the bare minimum of plotting. They're important to
compute mainly because the values of the intercepts are visually
obvious and it would be misleading to people viewing the graph if
these values were obtained wrong.

\subsection{Symmetry/periodicity}

Another thing that is visually obvious from the graph is {\em patterns
of repetition}. There are two kinds of patterns of repetition that we
are interested in:

\begin{enumerate}
\item {\em Periodicity}: The existence of $h > 0$ such that $f(x + h)
  = f(x)$ for all $x$ in $\R$. Periodicity is graphically visible --
  the shape of the graph repeats after an interval of length $h$. Note
  that we can talk of periodicity even for functions that are not
  defined for all real numbers, as long as it is true that the domain
  itself is invariant under the addition of $h$. For instance, $\tan$
  has a period of $\pi$.
\item {\em Symmetry: even and odd}: An even function ($f(x) = f(-x)$
  for all $x \in \R$) exhibits a particular kind of symmetry: symmetry
  about the $y$-axis. An odd function ($f(x) = -f(-x)$ for all $x \in
  \R$) exhibits {\em half-turn symmetry} about the origin. Both these
  properties are geometrically visible. Note that we can talk of even
  and odd for functions not defined for all real numbers, as long as
  the domain is symmetric about $0$. For instance, $f(x) := 1/x$ is
  odd and $f(x) := 1/x^2$ is even.
\end{enumerate}

There are somewhat more sophisticated versions of this:

\begin{enumerate}
\item Periodicity with shift: This happens when there exists $h > 0$
  and $k \in \R$ such that $f(x + h) = f(x) + k$ for all $x \in
  \R$. Thus, the graph of $f$ repeats after an interval of length $h$,
  but it is shifted vertically by $k$. Note that the case of shift $0$
  is precisely the case where $f$ itself is periodic. If $f$ is also
  differentiable, this is equivalent to the derivative being periodic.

  A function is periodic with shift if and only if it is the sum of a
  periodic function and a linear function. The breakup as a sum is
  unique up to constants. The periodic function part can be thought of
  as representing the seasonal trend and the linear function part can
  be thought of as representing the secular trend.
\item Half turn symmetry about axes other than the $y$-axis.
\item Mirror symmetry about points other than the origin.
\end{enumerate}

With the exception of {\em periodicity with shift}, all the other
notions are discussed in detail in the second set of lecture notes on
functions (Functions: A Rapid Review (Part 2)) so we will not repeat
that discussion. Since the mirror symmetry and half turn symmetry
material was not covered in class at the time, we'll take a short
detour in class to cover that material.

\subsection{First derivative}

The next step in getting a better picture of the function is to use
the derivative. The derivative helps us find the intervals on which
the function is increasing and decreasing, the critical points, and
other related phenomena. We shall return in some time to the
application of this information to graph-sketching.

\subsection{Second derivative}

If the function is twice differentiable (at most points) the second
derivative is another useful tool. We can use the second derivative to
find intervals where the function is concave up, intervals where the
function is concave down, and inflection points of the
function. Combining this with information about the first derivative,
we can determine intervals where the function is increasing and
concave up (i.e., increasing at an increasing rate), increasing and
concave down (i.e., increasing at a decreasing rate), decreasing and
concave up (i.e., decreasing at a decreasing rate), or decreasing and
concave down (i.e., decreasing at an increasing rate).

\subsection{Classifying and understanding points of interest}

Some of the cases of interest are:

\begin{enumerate}
\item Point of discontinuity: Separately compute the left-hand limit,
  right-hand limit and value. If either one-sided limit is $\pm
  \infty$, we have a vertical asymptote. If a one-sided limit equals
  the value, the graph has a closed circle. If a one-sided limit
  exists but does not equal the value, the graph has an open circle.
\item Critical point where the function is continuous and not
  differentiable: Determine whether the left-hand derivative and
  right-hand derivative individually exist. If so, determine the
  values of these derivatives. If the left-hand and right-hand
  derivatives do not exist as finite values, try determining the
  left-hand limit and right-hand limit of the derivative. If the limit
  of the derivative is $+\infty$ from both sides or $-\infty$ from
  both sides, we have a vertical tangent at the point. if the limit of
  the derivative is $+\infty$ from one side and $-\infty$ from the
  other side, we have a vertical cusp at the point. In all cases,
  determine the value of the function at the point.
\item Critical points where the derivative of the function is zero:
  Determine whether this is a point of local maximum, a point of local
  minimum, a point of inflection, or none of these. In any case,
  determine the value of the function at the point.
\item Point of inflection: Determine the value of the function as well
  as the value of the first derivative at the point. Also, determine
  whether the graph switches from concave up to concave down or
  concave down to concave up at the point.
\end{enumerate}

Critical points, and phenomena related to the first derivative, are
usually geometrically compelling, so it is important to focus on
getting them right so as not to paint a misleading picture. The
precise location of points of inflection is less geometrically
compelling, except when such a point is also a critical
point. Generally, it is geometrically clear that there exists an
inflection point in the interval between two points, because the graph
is concave up at one point and concave down at the other. However, the
precise location of the critical point may be hard to determine. Thus,
getting the precise details of inflection points correct is desirable
but not as basic as getting the critical points correct.

\subsection{Sketching the graph}

We first plot the points of interest and values (including $\pm
\infty$, corresponding to vertical asymptotes), as well as the
horizontal asymptotes for points at infinity. Here, {\em points of
interest} includes the critical points and inflection points,
intercept points, and a few other points added in to get a preliminary
plot. In addition to plotting the graph points (which is the pair
$(x,f(x))$ where $x$ is the point of interest in the domain), it is
also useful to compute the one-sided derivatives at each of the points
of interest, and draw a short segment of the tangent line (or
half-line, if only one-sided derivatives exist) corresponding to that.

Next, we use the increase/decrase and concave up/down information, as
well as the tangent half-lines, to make the portions of the graph
between these points of interest. This is the step that involves some
guesswork. The idea is that because we are sure that the main
qualitative characteristics (increase versus decrease, concave up
versus concave down) are correct, errors in further shape details are
not a big problem.

Since these actual shapes are the result of guesswork, it is
particularly important that the issues of symmetry and periodicity be
taken into account while sketching. For a periodic function, it is
better to have a {\em somewhat less accurate shape repeated faithfully
in each period} than a number of different-looking shapes in different
periods. Similar remarks apply for symmetry and even/odd functions.

The book has a number of worked out examples, and you should go
through them. To keep your homework set of manageable size, I haven't
included graph sketching problems in the portion of the homework to be
submitted. But I have recommended a few graph sketching problems from
the book's exercises and you should try these (and others if you want)
and can check your answer against a graphing calculator or software.

\section{Graphing particular functions}

Here we discuss various simple classes of functions and how they can
be graphed. For functions in these well behaved classes, we do not
need to go through the entire rigmarole for graphing.

\subsection{Constant and linear functions}

We begin by looking at the constant function $f(x) := k$. This
function is soporific, because you know the graph of the function is a
straight horizontal line, the derivative of the function is zero
everywhere, it is constant everywhere. Every point is a local minimum
and a local maximum in the trivial sense. The limits $\lim_{x \to
\infty} f(x)$ and $\lim_{x \to -\infty} f(x)$ are also both equal to $k$.

Next, we look at the linear function $f(x) := ax + b$ where $a \ne
0$. This function has graph a straight line. The tangent line at any
point on the graph is the same straight line. The slope of the
straight line is $a$. If $a > 0$, the function is increasing
everywhere, and if $a < 0$, the function is decreasing everywhere. The
derivative is the constant $a$ and the second derivative is $0$.

If $a > 0$, then $\lim_{x \to \infty} f(x) = +\infty$ and $\lim_{x \to
  -\infty} f(x) = -\infty$. If $a < 0$, then $\lim_{x \to \infty} f(x)
= -\infty$ and $\lim_{x \to -\infty} f(x) = +\infty$.

\subsection{Quadratic functions}

Consider the function $f(x) := ax^2 + bx + c$, where $a \ne 0$. This
is a quadratic function. The derivative function $f'(x)$ is equal to
$2ax + b$, the second derivative $f''(x)$ is the constant function
$2a$, and the third derivative is $0$ everywhere. In other words, the
slope of the tangent line to the graph of this function is not
constant, but it is changing at a constant rate.

The graph of this function is called a {\em parabola}. We describe the
graph separately for the cases $a > 0$ and $a < 0$.

In the case $a > 0$, we have $\lim_{x \to \infty} f(x) = \lim_{x \to
  -\infty} f(x) = \infty$. The function attains a local as well as an
absolute minimum at the point $x = -b/2a$, and the value of the
minimum is $(4ac - b^2)/4a$. The point $(-b/2a,(4ac - b^2)/4a)$ is
termed the {\em vertex} of the parabola. $f$ is decreasing on the
interval $(-\infty,-b/2a]$ and increasing on the interval
  $[-b/2a,\infty)$. Also, the graph of $f$ is symmetric (i.e., a {\em
      mirror symmetry}) about the vertical line $x = -b/2a$.

In the case $a < 0$, we have $\lim_{x \to \infty} f(x) = \lim_{x \to
  -\infty} f(x) = -\infty$. The function attains a local as well as an
absolute maximum at the point $x = -b/2a$, and the value of the
maximum is $(4ac - b^2)/4a$. The point $(-b/2a,(4ac - b^2)/4a)$ is
termed the {\em vertex} of the parabola. The function is increasing on
the interval $(-\infty,-b/2a]$ and decreasing on the interval $[-b/2a,\infty)$.

Finally, note the following about the existence of zeros, based on
cases about the sign of the discriminant $b^2 - 4ac$:

\begin{enumerate}
\item Case $b^2 - 4ac > 0$ or $b^2 > 4ac$: In this case, there are two
  zeros, and they are located symmetrically about $-b/2a$. If $a >
  0$, the function $f$ is positive to the left of the smaller root,
  negative between the roots, and positive to the right of the larger
  root. If $a < 0$, the function $f$ is negative to the left of the
  smaller root, positive between the roots, and negative to the right
  of the larger root.
\item Case $b^2 - 4ac = 0$ or $b^2 = 4ac$: In this case, $-b/2a$ is a
  zero of multiplicity two. The vertex is thus a point on the $x$-axis
  with the $x$-axis a tangent line to it. Note that if $a > 0$, the
  parabola lies in the upper half-plane and if $a < 0$, the parabola
  lies in the lower half-plane.
\item Case $b^2 - 4ac < 0$ or $b^2 < 4ac$: In this case there are no
  zeros. If $a > 0$, the parabola lies completely in the upper
  half-plane, and if $a < 0$, the parabola lies completely in the
  lower half-plane.
\end{enumerate}

\subsection{Cubic functions}

We next look at the case of a cubic polynomial, $f(x) := ax^3 + bx^2 +
cx + d$, where $a \ne 0$. We carry out the discussion assuming $a >
0$. In case $a < 0$, maxima and minima get interchanged and the sign
of infinities on limits get flipped.\footnote{Another way of thinking of it is
that we can first plot the graph by taking out a minus sign on the
whole expression, then flip it about the $x$-axis.}

So let's discuss the case $a > 0$. We have $\lim_{x \to -\infty} f(x)
= -\infty$ and $\lim_{x \to \infty} f(x) = \infty$. Notice that, by
the intermediate-value theorem, the cubic polynomial takes all real
values. The derivative of the function is $f'(x) = 3ax^2 + 2bx + c$,
the second derivative is $f''(x) = 6ax + 2b$, the third derivative is
$f'''(x) = 6a$ and the fourth derivative is zero. This means that not
only is the slope changing, but it is changing at a changing rate, but
the rate at which that rate is changing isn't changing (yes, you read
that right).

We now try to determine where the function has local maxima and
minima, and where it is increasing or decreasing. For this, we need to
first find the critical points. The critical points are solutions to
$f'(x) = 0$. The discriminant of the quadratic polynomial $f'$ is
$4b^2 - 12ac$. We make three cases based on the sign of the
discriminant.

\begin{enumerate}
\item $4b^2 - 12ac > 0$, or $b^2 > 3ac$: In this case, there are two
  critical points, given by the two solutions to the quadratic
  equation. We also see that, since $a > 0$, $f'$ is positive to the
  left of the smaller root, negative between the two roots, and
  positive to the right of the larger root. Thus, $f$ is increasing
  from $-\infty$ to the smaller root, decreasing between the two
  roots, and increasing from the larger root to $\infty$. The smaller
  root is thus a point of local maximum and the larger root is a point
  of local minimum.
\item $4b^2 - 12ac = 0$, or $b^2 = 3ac$: In this case, there is one
  critical point, namely $-b/3a$. The function is increasing all the
  way through, so although this is a critical point, it is neither a
  local maximum nor a local minimum. In fact, it is a point of
  inflection, where both the first and the second derivative become
  zero.
\item $4b^2 - 12ac<0$, or $b^2 < 3ac$: In this case, the function has
  no critical points and is increasing all the way through.
\end{enumerate}

Any cubic polynomial enjoys a half-turn symmetry about the point
$(-b/3a,f(-b/3a))$, i.e., the graph is invariant under a rotation by
$\pi$ about this point. This center of half-turn symmetry is also the
unique point of inflection for the graph. In the case that $b^2 >
3ac$, the point of half-turn symmetry is the exact midpoint between
the point of local maximum and the point of local minimum.

\includegraphics[width=3in]{cubicwithmaxmin.png}

\includegraphics[width=3in]{cubefunction.png}

\includegraphics[width=3in]{cubicwithoutmaxmin.png}

\subsection{Polynomials of higher degree}

Here are some general guidelines to understanding polynomials of
higher degree:

\begin{enumerate}
\item The limits at $\pm \infty$ are determined by whether the
  polynomial has even or odd degree, and the sign of the leading
  coefficient. Positive leading coefficient and even degree mean a
  limit of $+\infty$ on both sides. Negative leading coefficient and
  even degree mean a limit of $-\infty$ on both sides. Positive
  leading coefficient and odd degree mean a limit of $+\infty$ as $x
  \to \infty$ and $-\infty$ as $x \to -\infty$. Negative leading
  coefficient and odd degree mean a limit of $+\infty$ as $x \to
  -\infty$ and $-\infty$ as $x \to \infty$.
\item The points where the function could potentially change direction
  are the zeros of the first derivative. For a polynomial of degree
  $n$, there are at most $n - 1$ of these points. For such a point, we
  can use the first-derivative test and/or second-derivative test to
  determine whether the point is a point of local maximum, local
  minimum, or a point of inflection. {\em Note that for polynomial
  functions, any critical point must be a point of local maximum,
  local minimum, or a point of inflection}. There are no other
  possibilities for polynomial functions, because the number of times
  the first and/or second derivative switch sign is finite, hence we
  cannot construct all those weird counterexamples involving
  oscillations when dealing with polynomial functions.
\item Between any two zeros of the polynomial there exists at least
  one zero of the derivative (this follows from Rolle's theorem). This
  can help us bound the number of zeros of a polynomial using
  information we have about the number of zeros of the derivative of
  that polynomial.
\item A polynomial of odd degree takes all real values, and in
  particular, intersects every horizontal line at least once.
\item A polynomial of even degree and positive leading coefficient has
  an absolute minimum value, and takes all values greater than or
  equal to that absolute minimum value at least once. A polynomial of
  even degree and negative leading coefficient has an absolute maximum
  value, and takes all values less than or equal to that absolute
  maximum value at least once.
\end{enumerate}

\subsection{Rational functions: the many concerns}

A lot of things are going on with rational functions, so we need to
think about them more carefully than we thought about polynomials.

Graphing the function requires putting these pieces together, each of
which we have dealt with separately:

\begin{enumerate}
\item Determine where the rational function is positive, negative,
  zero, and not defined.
\item At the points where the rational function is not defined,
  determine the left-hand and right-hand limits. In most cases, these
  limits are $\pm \infty$. The exceptions are for cases such as the
  $FORGET$ function, defined as $FORGET(x) = x/x$, which is not
  defined at $x = 0$, but has a finite limit at that point. {\em These
  exceptions only occur in situations where the rational function as
  originally expressed is not in reduced form.}
\item Determine the limits of the rational function at $\pm
  \infty$. Note that this depends on how the degrees of the numerator
  and denominator compare and the signs of the leading coefficients.
\item Consider the derivative $f'$, and do a similar analysis on the
  derivative. The regions where the derivative is positive are the
  regions where $f$ is increasing. The regions where the derivative is
  negative are the regions where $f$ is decreasing.
\item Consider the second derivative $f''$, and do a similar analysis
  on this. Use this to find the regions where $f$ is concave up and
  the regions where $f$ is concave down.
\end{enumerate}

We combine all of these to draw the graph of $f$. We can also use all
this information to determine where the function attains its local
maxima and local minima.

\subsection{Piecewise functions}

Let's now deal with functions that are piecewise polynomial or
rational functions. We'll also use this occasion to discuss general
strategies for handling functions with piecewise definitions.

First, we need a clear piecewise definition, i.e., a definition that
gives a polynomial or rational function expression on each part of the
domain. The original definition may not be in that form. Here are some
things we need to do:

\begin{enumerate}
\item Whenever the whole expression, or some component of it, is in
  the absolute value, we make cases based on whether the expression
  whose absolute value is being evaluated is positive or negative. The
  transitions usually occur either at points where the expression is
  not defined, or at points where the absolute value is zero.
\item Whenever the expression involves something like $\max \{ f(x),
  g(x) \}$, then we make cases based on whether $f(x) > g(x)$ or $f(x)
  < g(x)$. The transition occurs at points where $f(x) = g(x)$ or at
  points where one or both of $f$ and $g$ is undefined.
\end{enumerate}

Once we have the definition in piecewise form, we can differentiate,
with the rule being to use the formula for differentiating in each
piece where we have the expression. If the function is continuous at
the points where the definition changes, we can use these formal
expressions to calculate the left-hand derivative and right-hand
derivative. We can then combine all this information to get a
comprehensive picture of the function.

\subsection{A max-of-two-functions example}

{\em Note}: This or a very similar example appeared in a past
homework. You might want to revisit that homework problem.

Consider $f(x) := \max \{ x - 1, \frac{x}{x + 1}
\}$. We first need to get a piecewise description of $f$. For this, we
need to determine where $x - 1 > x/(x+1)$ and where $x - 1 <
x/(x+1)$. This reduces to determining where $(x^2 - x - 1)/(x + 1)$ is
positive, zero, and negative.

The expression is positive on $((1 + \sqrt{5})/2,\infty) \cup (-1,(1 -
\sqrt{5})/2)$, negative on $((1 - \sqrt{5})/2,(1 + \sqrt{5})/2) \cup
(-\infty,-1)$, zero at $(1 \pm \sqrt{5})/2$, and undefined at
$-1$. Thus, we get that:

$$f(x) = \lbrace \begin{array}{rl} x - 1, & x \in ((1 + \sqrt{5})/2,\infty) \cup (-1,(1 - \sqrt{5})/2)\\\frac{x}{x+1}, & x \in (-\infty,-1) \cup [(1 - \sqrt{5})/2,1 + \sqrt{5}/2]\end{array}$$

Next, we want to determine the limits of $f$ as $x \to \pm
\infty$. Since the definition for $x > (1 + \sqrt{5})/2$ is $x - 1$,
$\lim_{x \to \infty}f(x) = \lim_{x \to \infty} x - 1 = \infty$. On the
other hand, the definition for $x < -1$ is $x/(x+1)$, so $\lim_{x \to
-\infty} f(x) = \lim_{x \to -\infty} x/(x+1)$. This is a rational
function where the numerator and denominator have equal degrees, and
the leading coefficients are both $1$, so the limit as $x \to -\infty$
is $1$.

Next, we want to find out the left-hand limit and right-hand limit at
the point $x = -1$. The definition from the left side is
$x/(x+1)$. The denominator approaches $0$ from the left side and the
numerator approaches a negative number, so the quotient approaches
$+\infty$. The right-hand limit is $\lim_{x \to -1} x - 1 = -2$.

Next, let us try to determine where the function is increasing and
decreasing. For this, we need to differentiate the function on each
interval.  

On the intervals $(-\infty,-1)$ and $[(1 - \sqrt{5})/2,(1 +
\sqrt{5})/2]$, $f$ is equal to $x/(x + 1)$. The derivative is thus
$1/(x + 1)^2$, which is positive everywhere, and hence, in particular,
on this region. Thus, $f$ is increasing on $(-\infty,-1)$ as well as
on $[(1 - \sqrt{5})/2,(1 + \sqrt{5})/2]$. On the intervals $(-1,(1 -
\sqrt{5})/2)$ and $(1 + \sqrt{5}/2,\infty)$, $f$ is defined as $x -
1$. The derivative is $1$, so $f$ is increasing on these intervals as
well. In fact, since $f$ is continuous at $1 + \sqrt{5}/2$, $f$ is
increasing on $[(1 + \sqrt{5})/2,\infty)$. Combining all this
information, we obtain that $f$ is increasing on $(-\infty,-1)$ and on
$(-1,\infty)$.

We can now understand and graph $f$ better. As $x \to -\infty$, $f(x)
\to 1$, and as $x \to -1^-$, $f(x) \to +\infty$. Thus, on the interval
$(-\infty,-1)$, $f$ increases from $1$ to $\infty$. Since the
right-hand limit at $-1$ is $-2$ and the limit at $\infty$ is $\infty$,
we see that on the interval $(-1,\infty)$, $f$ increases from $-2$ to
$\infty$. There are two intermediate points where the definition
changes: $(1 \pm \sqrt{5})/2$. From $-1$ to $(1 - \sqrt{5})/2$, $f$
increases from $-2$ to $(-1 - \sqrt{5})/2$ in a straight line. Between
$(1 - \sqrt{5})/2$ and $(1 + \sqrt{5})/2$, $f$ increases from $(-1 -
\sqrt{5})/2$ to $(-1 + \sqrt{5})/2$, but not in a straight line. From
$(1 + \sqrt{5})/2$ onward, $f$ increases in a straight line again.

The critical points are $(1 \pm \sqrt{5})/2$. Neither of these is a
local minimum or a local maximum. There is no absolute maximum,
because the left-hand limit at $-1$ is $\infty$, so the function takes
arbitrarily large positive values.

The function does not take arbitrarily small values. In fact, a lower
bound on the function is $-2$. Despite this, the function has no
absolute minimum, because $-2$ arises only as the right-hand limit at
$-1$ and not as the value of the function at any specific point. 

Note that more careful graphing of the function would also take into
account concavity issues. Here are the pictures:

\includegraphics[width=3in]{twofunctionsformaxmin.png}

\includegraphics[width=3in]{piecewisemax.png}

\subsection{Trigonometric functions}

Trigonometric functions are somewhat more difficult to study because,
unlike the case of polynomials and rational functions, there could be
infinitely many zeros.

One technique that is sometimes helpful when dealing with periodic
functions is to concentrate on the behavior in an interval the length
of one period, draw conclusions from there, and then use that to
determine what happens everywhere. A very useful fact here is that $f$
is a periodic function with period $p$, then $f'$ (wherever it exists)
also has period $p$. Similarly, the points and values of local maxima,
local minima, absolute maxima and absolute minima all repeat after
period $p$. In particular, in order to find the absolute maximum or
absolute minimum, it suffices to find the absolute maximum or absolute
minimum over a closed interval whose length is one period.

Consider, for instance, the function $f(x) := \sin x \cos x$. Since
both $\sin$ and $\cos$ have a period of $2\pi$, $f$ repeats after
$2\pi$ (so the period divides $2\pi$). So, it suffices to find maxima
and minima over the interval $[0,2\pi]$. At the endpoints the value is
$0$. The derivative of the function is $\cos^2 x - \sin^2 x = \cos
(2x)$. For this to be zero, we need $2x$ to be an odd multiple of
$\pi/2$, so $x = \pi/4, 3\pi/4, 5\pi/4, 7\pi/4$. We can use the
second-derivative test to see that the points $\pi/4, 5\pi/4$ are
points of local maximum and the points $3\pi/4, 7\pi/4$ are points of
local minimum. The value of the local maximum is $1/2$ and the value
of the local minimum is $-1/2$.

(It turns out that the function $\sin x \cos x$ has a period of $\pi$,
and can also be thought of as $(1/2) \sin(2x)$.)

\subsection{Mix of polynomial and trigonometric functions}

When the function is a mix involving polynomial and trigonometric
functions, it is not usually periodic, nor is it a polynomial, so we
need to do some ad hoc work.

For instance, consider the function $f(x) := x - 2 \sin x$. The
derivative is $f'(x) = 1 - 2\cos x$. Note that although $f$ is not
periodic, $f'$ is periodic, so in order to find out where $f' > 0$,
$f' = 0$, and $f' < 0$, we can restrict attention to the interval
$[-\pi,\pi]$.

We have $f'(x) < 0$ for $x \in (-\pi/3,\pi/3)$, $f'(x) = 0$ for $x \in
\{ -\pi/3,\pi/3\}$, and $f'(x) > 0$ for $x \in (\pi/3,\pi) \cup
(-\pi,-\pi/3)$.

Translating this by multiples of $2\pi$, we obtain that $f'(x) < 0$
for $x \in (2n\pi - \pi/3, 2n\pi + \pi/3)$ for $n$ an integer, $f'(x)
= 0$ for $x \in \{ 2n\pi - \pi/3, 2n\pi + \pi/3 \}$, and $f'(x) > 0$
at other points. Thus, $f$ keeps shifting between increasing and
decreasing.

On the other hand, for the function $f(x) := 2x - \sin x$, the
derivative is $f'(x) = 2 - \cos x$. This is always positive, so $f$ is
increasing.

\subsection{Functions involving square roots and fractional powers}

For functions involving squareroots or other fractional powers, we
first need to figure out the domain. Then, we use the usual techniques
to handle things.

Consider, for instance, the function:

$$f(x) := \sqrt{x} + \sqrt{1 - x}$$

The domain of this function is the set of values of $x$ for which both
$\sqrt{x}$ and $\sqrt{1 - x}$ is defined. This turns out to be the set
$[0,1]$, since we need both $x \ge 0$ and $1 - x \ge 0$. We can
differentiate $f$ to get:

$$f'(x) = \frac{1}{2\sqrt{x}} - \frac{1}{2\sqrt{1 - x}}$$

Note that although $f$ is defined on the closed interval $[0,1]$, $f'$
is defined on the {\em open} interval $(0,1)$ -- it is not defined at
the endpoints. In fact, the right-hand limit at $0$ is $+\infty$ and
the left-hand limit at $1$ is $-\infty$.

Next, we want to determine where $f'(x) = 0$. Solving this, we get $x
= 1/2$. Thus, $x = 1/2$ is a critical point. We also see that for $x <
1/2$, $\sqrt{x} < \sqrt{1 - x}$, so the reciprocal $1/2\sqrt{x}$ is
greater than the reciprocal $1/2\sqrt{1 - x}$. Thus, the expression
for $f'(x)$ is greater than $0$. On the other hand, to the right of
$1/2$, $f'(x) < 0$. Thus, $f'$ is positive to the left of $1/2$ and
negative to the right of $1/2$, yielding that $f$ is increasing on
$[0,1/2]$ and decreasing on $[1/2,1]$. Thus, $f$ attains a unique
absolute maximum at $1/2$, with value $\sqrt{2}$.

\subsection*{A more complicated version of the coffee shop problem}

Remember the coffee shop problem, where there are two coffee shops
located at points $a < b$ on a two-way street, and our task was to
construct the function that describes distance to the nearest coffee
shop. Let's now look at a somewhat different version, where the coffee
shops are both located off the main street.

Suppose coffee shop $A$ is located at the point $(0,1)$ and coffee
shop $B$ is located at the point $(2,2)$, and our two-way street is
the $x$-axis. The goal is similar to before: write as a piecewise
function the distance from the nearest coffee shop.

Define $p(x)$ as the distance from $A$ and $q(x)$ as the distance from
$B$. Then, we have $p(x) = \sqrt{x^2 + 1}$ and $q(x) = \sqrt{(x - 2)^2
+ 4} = \sqrt{x^2 - 4x + 8}$. Our goal is to write down explicitly the
function $f(x) := \min \{ p(x), q(x) \}$.

In order to do this, we need to consider the function $p(x) - q(x)$
and determine where it is positive, zero and negative. Define $g(x) :=
p(x) - q(x)$. Then, for $g(x) = 0$, we need:

$$\sqrt{x^2 + 1} = \sqrt{x^2 - 4x + 8}$$

Squaring both sides and simplifying, we obtain that $x = 7/4$. Since
$g$ is continuous, we can see that it has constant sign to the left of
$7/4$ (which turns out to be negative, as we see by evaluating at $0$)
and constant sign to the right of $7/4$ (which turns out to be
positive, as we see by evaluating at $2$). Thus, our expression for
$f$ is given by:

$$f(x) = \lbrace \begin{array}{rl} \sqrt{x^2 + 1}, & x \in (-\infty,7/4]\\ \sqrt{x^2 - 4x + 8}, & x \in (7/4,\infty)\end{array}$$

We can now use this to calculate $f'$. $f'(x) = x/\sqrt{x^2 + 1}$ to
the left of $7/4$ and $(x - 2)/\sqrt{x^2 - 4x + 8}$ to the right of
$7/4$. At the point $7/4$, the left-hand derivative is $7/\sqrt{65}$
and the right-hand derivative is $-1/\sqrt{65}$. The function is not
differentiable at $7/4$.

Next, we want to determine where $f' > 0$, $f' = 0$ and $f' < 0$. For
$x < 7/4$, we see that $f'(x) < 0$ for $x \in (-\infty,0)$, $f'(0) =
0$, and $f'(x) > 0$ for $x \in (0,7/4)$. For $x > 7/4$, we see that
$f'(x) < 0$ for $x \in (7/4,2)$, $f'(2) = 0$, and $f'(x) > 0$ for $x
\in (2,\infty)$ (this should again be clear by looking at the picture
geometrically. The distance to coffee shop $A$ decreases till we get
to the same $x$-coordinate as $A$, then it increases. At some point,
$B$ starts becoming closer, whence the distance to $B$ starts
decreasing, till we reach the point with the same $x$-coordinate as
$B$, and then it starts increasing).

Thus, $f$ is decreasing on $(-\infty,0]$, increasing on $[0,7/4]$,
decreasing on $[7/4,2]$, and increasing on $[2,\infty)$. The critical
points are $0$, $7/4$, and $2$. There are local minima at $0$ (with
value $1$) and $2$ (with value $2$) and a local maximum at $7/4$ (with
value $\sqrt{65}/4$). The limits at $\pm \infty$ are both
$\infty$. Thus, there is no absolute maximum, but the absolute maximum
occurs at $0$, and it has value $1$.

Notice that although the picture here is qualitatively somewhat
similar to the case where both coffee shops are on the $x$-axis, there
are also some small differences -- the graph never touches the
$x$-axis, and the function is differentiable with derivative zero at
two of the three critical points.

Here are the pictures:

\includegraphics[width=3in]{offroadcoffeeshops.png}

Hre is the picture zoomed in (note: axes not centered at origin) near
the value $x = 7/4$.

\includegraphics[width=3in]{offroadcoffeeshopszoomin.png}

\section{Subtle issues}

\subsection{Equation-solving troubles}

In some cases, it is not computationally easy to do each of the
suggested steps. For instance, we may not have any known method for
solving $f(x) = 0$ for the given function $f$. Similarly, we may not
have any known method for solving $f'(x) = 0$ or $f''(x) = 0$. 

In cases where we do not have exact solutions, what we should do is
try to find the number of solutions and the intervals in which these
solutions lie, to as close an approximation as possible. Two useful
tools in this are the {\em intermediate-value theorem} and {\em
Rolle's theorem}.

For instance, consider the function $f(x) := x - \cos x$. $f$ is an
infinitely differentiable function, and its derivative, $1 + \sin x$,
is periodic with period $2\pi$. Thus, the graph of $f$ repeats after
$2\pi$, with a vertical upward shift of $2\pi$. We can further find
that $f$ is increasing everywhere, because $1 + \sin x \ge 0$ for all
$x$, with equality occurring only at isolated points. $f''(x) = \cos
x$, so $f$ is concave up on $(-\pi/2,\pi/2)$ and its
$2\pi$-translates, and $f$ is concave down on $(\pi/2,3\pi/2)$ and its
$2\pi$-translates. The inflection points of $f$ are precisely the odd
multiples of $\pi/2$. The $x$-intercept is $-1$.We thus have a fairly
complete picture of $f$, except that we do not know the
$x$-intercept(s).

\includegraphics[width=3in]{xminuscosx.png}

Although we do not know the $x$-intercept(s) precisely, we have some
qualitative information. First, there can be at most one
$x$-intercept, because $f$ is increasing on $\R$. The
intermediate-value theorem now reveals that the $x$-value must be
somewhere between $0$ ($f(0) = -1$) and $\pi/2$ ($f(\pi/2) =
\pi/2$). In other words, the zero occurs in the segment between the
$y$-intercept and the first inflection point after that. This is fine
for a rough visual guide, but for a more accurate graph, we might like
to narrow the location of the zero further. We can narrow it down
further to $(\pi/6,/\pi/4)$ using elementary trigonometric
computations. Further narrowing is best done with the aid of a
computer.

Note that even if we did not bother about knowing the $x$-intercept
before sketching the graph, our graph sketch would have been quite
okay and would in fact have {\em suggested} the location of the
$x$-intercept. This is an example of a general principle: {\em Often,
even if we are computationally unable to handle all the suggested
steps for graph-sketching, a preliminary sketch based on the steps we
could successfully execute gives enough valuable hints.} The moral of
the story is to not be discouraged about not executing a few steps and
instead to do as much as possible with the steps already executed, and
then seek alternative ways of tackling the recalcitrant steps.

See also Example 5 in the book.

\subsection{Graphing multiple functions together}

In many situations, it is necessary to be able to graph multiple
functions together. This is sometimes necessary to compare and
contrast these functions. Some examples include:

\begin{enumerate}
\item Graphing a function and its first, second and higher derivatives
  together: This is often visually useful in discerning patterns about
  the function, and helps with rapid switching between the global and
  local behavior of a function.
\item Graphing a function and another function obtained by scaling or
  shifting it: For instance, it may be helpful to graph $f(x)$ and
  $g(x) := f(x + h)$ on the same graph. This allows for easy visual
  insight into how the value of $f$ changes after an interval of
  length $h$.
\item Graphing two functions to determine their intersection points,
  angles of intersection, etc.
\end{enumerate}

When graphing multiple functions together, the procedure is similar to
that when graphing a single function, but the following additional
point needs to be kept in mind: It is important to make sure that any
visually obvious inferences made about the comparison of values of the
functions are correct. For instance, it is important to get right
which function is bigger where. The ideal way to do this is to find
precisely the points of intersection -- however, that may not be
possible because the equation involved cannot be solved. Nonetheless,
try to bound the locations of intersection points in small intervals
using the intermediate value theorem. (Note that for functions
obtained as derivatives, we can use Rolle's theorem and the mean value
theorem.)

\subsection{Transformations of functions/graphs}

Also, if the two functions are related in terms of a transform, it is
important that the geometric picture suggested by the transform is the
correct one. Here are some examples:

\begin{enumerate}
\item Suppose we have two functions $f$ and $g$ where $g(x) := f(x +
  h)$. Then, the graph of $g$ should be the graph of $f$ shifted left
  by $h$. If $h$ is negative, it is the graph of $f$ shifted right by
  $-h = |h|$.
\item Suppose we have $g(x) := f(x) + C$. Then, the graph of $g$
  equals the graph of $f$ shifted upward by $C$. If $C$ is negative,
  it is the graph of $f$ shifted downward by $-C = |C|$.
\item Suppose $g(x) := f(\alpha x)$. Then, the graph of $g$ should be
  the graph of $f$ shrunk along the $x$-dimension by a factor of
  $\alpha$. If $\alpha$ is negative, then this shrinking is a
  composite of a shrinking by $|\alpha|$ and a flip about the
  $y$-axis.
\item Suppose $g(x) := \alpha f(x)$. Then the graph of $g$ should be
  the graph of $f$ expanded along the $y$-dimension by a factor of
  $\alpha$. If $\alpha$ is negative, this involves an expansion by
  $|\alpha|$ and a flip about the $x$-axis.
\end{enumerate}

\subsection{Can a graph be used to prove things about a function?}

Yes and no. Remember that the way we drew the graph was using
algebraic information about the function. So anything we deduce from
the graph, we could directly deduce from that algebraic information,
without drawing the graph.

The importance of graphs is that {\em they suggest good guesses that
may not be obvious simply by looking at the algebra}. In other words,
they allow visual and spatial intuition to complement the formal,
symbolic intuition of mathematics. However, once the guess is made, it
should be possible to justify without resort to the graph. Such
justifications may use theorems such as the intermediate value
theorem, Rolle's theorem, the extreme value theorem, and the mean
value theorem. {\em In cases where things suggested by the graph
cannot be verified algebraically, it is possible that some unstated
and unjustified assumption was made while drawing the graph.}

\subsection{Sketching curves that are not graphs of functions}

Some curves are not in the form of functions, and cannot be expressed
in that form because there are multiple $y$-values for a given
$x$-value. To sketch such curves, we follow similar guidelines, but
there are some changes:

\begin{enumerate}
\item There is no clear concept of domain. However, it is still useful
  to determine the possible $x$-values for the curve and the possible
  $y$-values for the curve. This allows us to bound the curve in a
  rectangle or strip. For instance, consider the curve $x^4 + y^4 =
  16$. Then, the $x$-value is in the interval $[-2,2]$ and the
  $y$-value is in the interval $[-2,2]$.
\item We can use the techniques of sketching graphs of functions by
  breaking the curve down into graphs of functions. For instance, the
  curve $x^4 + y^4 = 16$ can be broken down as a union of graphs of
  two functions: $y = (16 - x^4)^{1/4}$ and $y = -(16 -
  x^4)^{1/4}$. We can sketch both graphs using the techniques of
  graph-sketching (in fact, it suffices to sketch the first graph and
  then construct the second graph as the reflection of the first graph
  about the $x$-axis).
\item In cases where this separation is not easy to do, we can still
  try to draw the graph using the general techniques: use implicit
  differentiation to find the first derivative and second derivative,
  determine the critical points, local extreme values, points of
  inflection, regions of increase and decrease, regions of concave up
  and concave down, and so on.
\end{enumerate}

\subsection{Piecewise descriptions, absolute values and max/min of two functions}

To graph a function explicitly given in piecewise form, we need to
keep in mind the following things:

\begin{enumerate}
\item Within the domain of each definition, plot the graph of the
  function the usual way.
\item At the points where the definition changes, determine the
  one-sided limits, one-sided limits of first derivatives, and
  one-sided limits of second derivatives. These points are likely
  candidates for discontinuity of the function, likely candidates for
  discontinuity of the derivative, and likely candidates for
  discontinuity of the second derivative of the function.
\item Piece this information together to draw the overall graph. Use
  open circles, closed circles etc. to mark clearly the limits at the
  points of definition changes.
\end{enumerate}

In some cases, it is helpful to draw the graphs of each of the pieces
over {\em all real values} and then pick out the requisite pieces from
the relevant domains of definition.

If a function is defined as the maximum of two functions or the
minimum of two functions, or in terms of absolute values, then we can
first express it as a piecewise function and then graph
it. Alternatively, we can graph both the functions (taking care of the
points of intersection) and then use a combination of visual insight
and algebra to graph the maximum and/or minimum of the two functions.

\section{Addenda}

\subsection{Addendum: Plotting graphs using Mathematica}

It is possible to plot the graph of a function using
Mathematica. Doing a few such plots can help reinforce your intuition
about the shape of graphs.

The Mathematica syntax is:

\begin{verbatim}
Plot[f[x],{x,a,b}]
\end{verbatim}

This plots the graph of $f(x)$ for $x \in [a,b]$.

For instance:

\begin{verbatim}
Plot[x^2,{x,0,1}]
\end{verbatim}

plots the graph of $x^2$ for $x \in [0,1]$.

The command:

\begin{verbatim}
Plot[x - Sin[x],{x,-3*Pi,3*Pi}]
\end{verbatim}

plots the graph of the function $x - \sin x$ on the interval
$[-3\pi,3\pi]$. Note that it is not possible to graph a function from
$-\infty$ to $\infty$, so we have to stay content with finite plots.

It is also possible to plot the graphs of multiple functions together. For instance:

\begin{verbatim}
Plot[{Sin[x],(Sin[x])^2},{x, -Pi,Pi}]
\end{verbatim}

This plots the graphs of the fnuctions $\sin$ and $\sin^2$ on the
interval $[-\pi,\pi]$. To learn more, see the Mathematica
documentation on the Plot function.

We can also use Mathematica to find where a function is positive,
zero, and negative. You can use the Solve, Reduce, and FindRoot
functions in Mathematica:

\begin{enumerate}
\item The Solve function only solves equalities, and may not find all
  solutions. It also uses formal methods, so may not find the
  solutions numerically. However, it will give a formal solution
  saying $\pi$ instead of $3.14 \dots$, for instance).
\item The Reduce function is more powerful. It solves both equalities
  and inequalities, and finds all solutions. Like Solve, it only works
  for certain kinds of functions where these analytical and formal
  methods can be applied.
\item The FindRoot function can be used to find points where a
  function is zero numerically. It is applicable to functions that
  involve a mixture of algebra and trigonometry. However, since it
  uses numerical methods, it may not give the exactly correct answer
  (for instance, it may compute $0.998$ instead of $1$).
\end{enumerate}

For instance, we can do:

\begin{verbatim}
Reduce[x^3 - x - 6 > 0,x]
\end{verbatim}

and find that the solution set to this is $x > 2$.

For something non-algebraic, we can find the roots:

\begin{verbatim}
FindRoot[x - Cos[x],{x,1}]
\end{verbatim}

find a solution to $\cos x = x$. Note that Solve and Reduce do not
work here because of the mixture of algebra and trigonometry. See the
documentation on Solve, Reduce, and FindRoot.

We can also find the derivative of a function. First, define the function, e.g.:

\begin{verbatim}
f[x_] := x - Sin[x]
\end{verbatim}

We can then refer to the derivative of $f$ as $f'$ and the second
derivative as $f''$. Thus, we can do:

\begin{verbatim}
Reduce[{f''[x] > 0,-Pi < x, x < Pi},x]
\end{verbatim}

This finds all solutions to $f''(x) > 0$ for $x$ in the open interval
$(-\pi,\pi)$.

These commands allow us to execute most of the computational aspects
needed for graph-sketching using Mathematica.

\subsection*{Addendum: using a graphing software or graphing calculator}

When using a graphing software or graphing calculator to plot the
graph of a function, please make sure you zoom in and out enough to
make sure that you are not fooled because of the scale chosen by the
calculator. For instance, plotting the graph of $x^2 \sin(1/x)$ using
a graphing software makes it seem like it crosses the $x$-axis at only
finitely many points. However, zooming in closer to zero shows a lot
of oscillation close to zero, and the more you zoom in, the more
oscillation you see. Thus, it is important to use graphing software as
a complement rather than a substitute for basic mathematical common
sense.

\end{document}