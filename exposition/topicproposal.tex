\documentclass{amsart}

%Packages in use
\usepackage{fullpage, hyperref, vipul}
%Title details
\title{Topic proposal: The structure of finite groups}
\author{Vipul Naik\\discussed with George Glauberman}

%List of new commands
\renewcommand{\definedind}[1]{{\bf #1}}
\makeindex

\begin{document}
\maketitle

The typical college course in group theory covers groups,
homomorphisms, normal subgroups, quotients, and group actions. This is
followed by two basic theorems about finite groups: Sylow's theorem
and the Jordan-Holder theorem. Sylow's theorem is the starting point
of the {\em arithmetic} structure of groups: it relates the
factorization of a natural number $n$ to subgroups in a group of order
$n$. The Jordan-Holder theorem is the starting point of the {\em
  normal} structure of groups: the use of normal subgroups and
quotient groups to study a group's structure. (For more discussion on
the arithmetic and normal structure, refer \cite{arithmeticandnormal}).

The Jordan-Holder theorem relates arbitrary finite groups to a
restricted class of finite groups called finite simple groups. A
\definedind{simple group} is a nontrivial group that
has no proper nontrivial normal subgroup. Unlike what the name
suggests, {\em simple} here does not mean straightforward, gullible,
or easy. Rather, simple means {\em unbreakable} or {\em
  indecomposable}.

By the eighties, finite group theorists had completed (modulo a snag
that was finally resolved in 2004) the classification of all finite
simple groups. (\cite{SolomonCFSG} is an excellent expository article
on the classification and its history). Unlike what some believe, this
does not render trivial the study of questions about {\em all finite
  groups}, since there are many different ways that finite
simple groups can be ``put together.'' For instance, all finite
$p$-groups are obtained by piecing together simple groups of order
$p$. However, there are many nontrivial questions about the structure
of finite $p$-groups. In fact, classifying all finite groups, or even
all finite $p$-groups, may not be meaningful in any sense because
there are just too many types.

With no large-scale classification in the works, group theorists are
trying to add that extra bit to our knowledge of how finite groups
behave. They're also building on the effort of the classification by
helping out in other areas of mathematics. Some examples worth
mentioning here:

\begin{itemize}

\item Geometric group theorists are interested in studying group
  actions on manifolds. Often, the study of such actions boils down to
  questions about $p$-groups and simple groups.

\item Pro-$p$-groups, and profinite groups in general, are of
  tremendous importance to questions in number theory, algebraic
  geometry, and Galois theory. For instance, the inverse Galois
  problem can be phrased as a question of computing the finite
  quotients of a given profinite group. Solving many of these
  questions requires exploration of questions involving finite groups.

\item $p$-local classifying spaces of finite groups are important
  objects in algebraic topology. The recent development of the notion
  of fusion system allows for the construction of classifying spaces
  of fusion systems, which is a generalization of $p$-local
  classifying spaces of finite groups. The construction of fusion
  systems is inspired by the fusion theorems proved by Alperin and
  others as work towards the classification. (More on this in section
  \ref{fusion}; also see \cite{fusionintro} for an introduction to
  fusion systems).

\end{itemize}

I hope to pursue research in group theory. My topic proposal
describes some old work (1900-1975) in group theory, done originally
with the classification in mind, that might help with such research.


My topic covers material largely from Chapters 5 to 8 of Gorenstein's
book  {\em Finite Groups} (\cite{gorenstein}).

\section{Groups of prime power order}

\subsection{Characteristic subgroups of finite $p$-groups}

Let $p$ be a prime. A finite $p$-group is a group whose order is a
power of $p$.\footnote{The trivial group is a finite $p$-group as
  well, but people often implicitly mean ``nontrivial'' when talking
  about finite $p$-groups.} A famous elementary result is that every
finite $p$-group is nilpotent: in particular, every nontrivial finite
$p$-group has a nontrivial center.

There are many important characteristic subgroups of finite
$p$-groups, many of them defined through subgroup-defining functions:
they're defined in a unique way starting from the big group. Examples
include the center, commutator subgroup, members of the lower central
series, members of the upper central series, the \definedind{Frattini
  subgroup} (defined as the intersection of all maximal subgroups),
and members of the Frattini series (a series obtained by taking
Frattini subgroups successively). Also important are the omega and
agemo subgroups: the subgroup $\Omega_j(P)$ of a $p$-group $P$ is
defined as the subgroup generated by all $x \in P$ for which the order
of $x$ divides $p^j$. $\mho^j(P)$ is defined as the subgroup generated
by all elements of the form $x^{p^j}$.

Why the plethora of subgroup-defining functions? The hope is to find
smaller characteristic subgroups inside the big group whose structure
gives information about the structure of the whole group.

\subsection{Measures of size}

How big is a $p$-group? The naive measure of size is the order. A
group of order $p^n$ thus has size $p^n$. However, size on its own
conveys little information about complexity. For instance, the
structure of the cyclic group of order $p$ is similar for all primes
$p$, and these groups might be thought of as equally complex in some
sense, even though their sizes differ widely. This suggests that
the measure of size is not $p^n$, but the exponent $n$.

However, for large values of $n$, the value of $p$ matters too. When
$n = 1$, all groups of order $p^1$ can be studied under the same
umbrella: cyclic groups of prime order. The same holds for $n = 2,3$
(with $p=2$ behaving slightly differently) and to a somewhat lesser
extent, for $n= 4$. But for $n \ge 5$, the number of groups of order
$p^n$, and their nature and behavior, depends very much on $p$.

We have two measures of size so far: $p^n$ and $n$. Other measures of
size look at the length of various series associated with the
group. For instance, we can consider the nilpotence class (the length
of the lower or upper central series), the solvable length (the length
of the derived series), or the Frattini length (the length of the
Frattini series). These measures are related in strange ways, and none
of them alone suffices. For instance:

\begin{itemize}

\item A nontrivial $p$-group is abelian if and only if it has
  nilpotence class one, if and only if it has solvable length
  one. Thus, groups of small nilpotence class and small solvable
  length can be thought of as being close to abelian.

\item A group of nilpotence class $c$ has solvable length {\em at
    most} $c$, but the solvable length provides no bound on the
  nilpotence class.

\item A nontrivial $p$-groups is elementary abelian\footnote{An
    \definedind{elementary abelian group} is a direct product of
    isomorphic cyclic groups of prime order; equivalently, it is the
    additive group of a vector space over a finite field.} if and only
  if it has Frattini length one.

\end{itemize}

Yet another measure of size is the exponent. The exponent of a finite
group is the smallest $d$ such that the order of every element divides
$d$. The exponent of a $p$-group is of the form $p^e$ for some natural
number $e$, so we may choose either $e$ or $p^e$ as the size measure.

Here are some interesting and unexpected facts:

\begin{itemize}

\item When $p = 2$, any group of exponent $p$ is abelian. When $p =
  3$, any group of exponent $p$ has nilpotence class at most
  three. When $p \ge 5$, there is no bound on the nilpotence class of
  a $p$-group of exponent $p$.

\item A group of order $p^n$ and nilpotence class $n-1$ (also called a
  \definedind{maximal class group}) is generally thought of as being
  very far from abelian. However, we can find maximal class groups
  with abelian subgroups of index $p$. For instance, the dihedral
  group of order $2^n$, or the wreath product of two cyclic
  $p$-groups, are maximal class groups, yet have abelian subgroups of
  prime index. In fact, any $2$-group of maximal class has a cyclic
  maximal subgroup (this follows from the classification of all
  $2$-groups of maximal class, see for instance \cite[Theorem 5.4.5,
  Page 194--195]{gorenstein}).

\end{itemize}

\subsection{Coprime automorphisms and Thompson's critical subgroup}

For simplicity of expression, we introduce a term. A subgroup $H$ of a
finite group $G$ is termed a \definedind{coprime automorphism-faithful
  subgroup} if, for any non-identity automorphism $\sigma$ of $G$ of
order relatively prime to the order of $G$, such that $\sigma(H) = H$,
the restriction of $\sigma$ to $H$ is not the identity map. In other
words, non-identity automorphisms of coprime order, when they
restrict, restrict to non-identity maps.

Thompson was interested in whether a small coprime
automorphism-faithful {\em characteristic} subgroup can be found. If
$H$ is both coprime automorphism-faithful and characteristic, we get a
homomorphism, by restriction:
$$\operatorname{Aut}(G) \to \operatorname{Aut}(H)$$
where the kernel contains no non-identity element of order coprime to
that of $G$. Thompson managed to prove, in his paper with Feit on the
odd-order theorem (\cite[Chapter 2, Lemma 8.2, Page
795]{feitthompson}) that every finite $p$-group has such a coprime
automorphism-faithful characteristic subgroup of nilpotence class
two. Nilpotence class two groups are very close to abelian groups:
they are ``small.'' Thompson used the faithful action of the coprime
automorphism group on this small group to get strong restrictions on
the nature of the coprime automorphism group.

The subgroup constructed by Thompson has many other
remarkable properties as well. These properties were abstracted by
Gorenstein (\cite[Theorem 5.3.11, Page 185--186]{gorenstein}), and a
subgroup satisfying these properties is termed a \definedind{critical
  subgroup}.

The typical temptation when trying to build a characteristic subgroup
with certain properties is to look for a subgroup-defining function:
something that pinpoints a subgroup uniquely. So, one might look at
the omega series, the agemo series, the Frattini series, the lower and
upper central series, or the derived series. Thompson's insight was to
{\em not} yield to this temptation. Rather, his construction of a
critical subgroup relies on certain {\em choices}, and in general, there
may be more than one critical subgroup. As yet, there is no known way
of {\em canonically} picking a critical subgroup, despite the fact
that critical subgroups are all characteristic. In fact, Thompson's
procedure does not even yield all the possible critical subgroups.

\subsection{Abelian to cyclic: groups in a real-life act}

Students of group cohomology may recognize the condition for finite
groups: {\em every abelian subgroup is cyclic}. This is precisely the
condition for having periodic cohomology, and is relevant to
topological questions such as group actions on spheres. The question
can be broken into two parts using Sylow's theorem: find the
$p$-groups with this property, and then, find all groups which have
$p$-Sylow subgroups within that list of $p$-groups with the
property. It turns out that when $p$ is odd, a $p$-group in which
every abelian subgroup is cyclic must itself by cyclic, and for $p =
2$, the group is either cyclic or generalized quaternion.\footnote{A
  \definedind{generalized quaternion group} of order $2^{n+1}$ is a group with a
  presentation $\langle a^{2^n} = 1, x^2 = a^{2^{n-1}}, xax^{-1} =
  a^{-1} \rangle$, $n \ge 2$. All generalized quaternion groups occur
  as subgroups of the skew field of Hamiltonian quaternions. The famed
  example is the quaternion group $\{ \pm 1, \pm 1, \pm j, \pm k \}$
  of order eight, obtained by setting $n = 2$.}

Thompson's critical subgroup theorem, and many related discoveries
that appeared in his paper with Feit on the odd-order theorem
(\cite{feitthompson}), can be used to answer more general versions of
this question. For instance, it is possible to considerably simplify
the proof of the classification of all $p$-groups in which {\em every
  abelian characteristic subgroup is cyclic} (originally done by
Philip Hall), and all $p$-groups in which {\em every abelian normal subgroup
  is cyclic}.

\subsection{Differentiate and exponentiate}

The last thing you might expect to be able to do with a finite group
is take logarithms, exponentials, and tangent spaces. And yet, it
turns out that for a restricted class of finite $p$-groups (the
subgroup generated by any three elements must have class at most
$p-1$), there exist ``Lie rings'' corresponding to the groups, with a
bijective map from the Lie ring to the group called the
``exponential''. This exponential behaves much the same way as the
usual exponential for a nilpotent Lie group (such as a group of
upper-triangular matrices). This establishes a correspondence between
a restricted class of $p$-groups and a restricted class of $p$-Lie
rings, called the Lazard correspondence.  Here are some interesting
observations about the correspondence:

\begin{itemize}

\item The exponential map is a bijection that preserves orders of
  elements, puts Lie subrings in correspondence with subgroups, and
  puts Lie ideals in correspondence with normal subgroups. It
  preserves automorphism groups as well.

\item For abelian groups, the corresponding Lie ring is abelian with
  the same additive group. In other words, the Lie bracket is zero.

\item For odd primes, the Lie ring corresponding to a group of class
  two is defined with the Lie bracket being the commutator in the
  group-theoretic sense.

\item When the group has exponent $p$, the Lie ring is a Lie algebra
  over the prime field. This allows us to do ``algebraic geometry''
  over the prime field.

\end{itemize}

The Lazard correspondence is more than a curiosity. Just as taking
logarithms helps in converting messy arithmetic computation involving
multiplication to relatively easier problems of addition, the Lazard
correspondence helps escape the messy noncommutativity of groups and
pass to the more pleasant and tractable Lie rings. The Lazard Lie ring
of a group of class two was used by Bender (in
\cite{benderonthompson}) to simplify a computationally involved proof
of Thompson about signalizers.

\section{Solvable groups}

\subsection{Schur-Zassenhaus and the shadow of odd order}

A subgroup $H$ of a group $G$ is termed a \definedind{Hall subgroup}
if the order and index of $H$ are relatively prime. Schur and
Zassenhaus proved that any normal Hall subgroup has a permutable
complement: in other words, if $H$ is normal Hall in $G$, $G$ splits
as a semidirect product involving $H$. The proof has many variations,
some using transversals, some making a direct appeal to group
cohomology. They further proved that if $H$ is normal Hall in $G$, and
either $H$ or $G/H$ is solvable, then any two complements to $H$ in
$G$ are conjugate.\footnote{Who contributed what part to the proof is
  unclear. It is definitely true that Schur set the ball rolling by
  proving existence, at least in the abelian case, and Zassenhaus
  stated the full theorem in his textbook later.}

This ``either one or the other is solvable'' assumption may seem
weak. However, given two coprime numbers, at least one of them is odd,
and the celebrated odd-order theorem of Feit and Thompson
(\cite{feitthompson}) shows that any group of odd order is
solvable. Thus, Zassenhaus' conclusion actually holds for any normal
Hall subgroup. Unfortunately, the proof of the Feit-Thompson theorem
is fairly involved (255 pages) and it feels bad to invoke such a big
theorem to prove a relatively minor result.

The Schur-Zassenhaus theorem has many applications, one of them being
a ``Sylow's theorem with operators'': a version of Sylow's theorem
that measures all the Sylow subgroups invariant under the action of a
coprime group of automorphisms. The usual existence, conjugacy, and
domination statements hold.

\section{Fusion and transfer}

\subsection{Alperin's fusion: hopping step by step locally}\label{fusion}

Fusion, despite its grand-sounding name, is about a simple question:
given two elements in a big group that are conjugate, are they
conjugate in some intermediate subgroup?  And more generally, can we
hop from one element to another using conjugates within intermediate
subgroups?

This question is related to the idea of ``local analysis''. Suppose
$G$ is a finite group with possibly many primes dividing it. Local
analysis basically says that certain behavior inside the big group $G$
can be studied by looking at a number of smaller subgroups dependent
on a prime $p$, along with information that describes how they piece
together. In the typical scenario, these smaller subgroups are
\definedind{$p$-local subgroups}: the normalizers of $p$-subgroups of
the whole group. Local analysis was originally developed to study the
local structure of simple groups, or candidates for simple
groups. Those who've played with Sylow numbers to prove that certain
groups aren't simple may think of local analysis as a much more
advanced tool that works partly in that direction.

Alperin's fusion theorem (\cite{alperinfusion}) proves a certain
result about selecting a family of subgroups of a $p$-Sylow subgroup,
such that two elements (or more generally, subsets) are conjugate in
the whole group if and only if there is a way from one element to
another by conjugation using elements in the normalizers of this
family of subgroups. In other words, it says that two subsets are
conjugate {\em globally in the group} if and only if they are
conjugate {\em locally, via normalizers of $p$-subgroups}. (For a more
detailed treatment of local analysis, refer \cite[Chapter 1]{GL}).

People working on the classification of finite simple groups came
across a few stubborn examples of collections of local information
that didn't give a group, but for which it was very hard to prove that
there was no group behind them. It turns out that these collections of
local information were actually something close to groups: they have
been christened {\em fusion systems}. For any finite simple group and
any prime $p$, we get a $p$-fusion system, but there exist $p$-fusion
systems that arise through other means. 

Algebraic topologists have long been interested in $p$-local
classifying spaces. The $p$-local classifying space of a group is the
$p$-localization of its classifying space. People working on fusion
systems came up with a notion of the classifying space corresponding
to a $p$-fusion system, such that if we take the $p$-fusion system
$\mathcal{F}$ corresponding to a finite group $G$, the classifying
space corresponding to $\mathcal{F}$ is equal to the $p$-local
classifying space corresponding to $G$. (For a quick introduction to
fusion systems, refer to \cite{fusionintro}).

\subsection{The Focal Subgroup Theorem}

The Focal Subgroup Theorem states something seemingly uninteresting:
if $P$ is a $p$-Sylow subgroup of $G$, then the intersection of $P$
with $[G,G]$ equals the subgroup generated by $xy^{-1}$, where $x,y
\in P$ and are conjugate in $G$. This theorem has many proofs, the
most typical being a proof using the transfer homomorphism. (The
theorem, along with many related facts, were proved by Donald Higman
as part of his doctoral thesis, \cite{higmanfocal}).

The Focal Subgroup Theorem has many consequences. For instance, define
a subgroup $H$ of a group $G$ to be a \definedind{conjugacy-closed
  subgroup} of $G$ if any two elements of $H$ that are conjugate in
$G$ are conjugate in $H$. In other words, fusion in $G$ is contained
completely inside $H$.

The Focal Subgroup Theorem, combined with results about conjugacy and
normalizers (such as Alperin's fusion theorem), yields that if an
abelian Sylow subgroup $S$ of $G$ is in the center of its normalizer,
it has a \definedind{normal complement}: a normal subgroup $N$ of $G$
such that $NS = G$ and $N \cap S$ is trivial. This result is called
Burnside's normal $p$-complement theorem. More work in this direction
shows that if $S$ is any conjugacy-closed Sylow subgroup of $G$, $S$
has a normal complement in $G$. A slightly more elaborate version of
this result is called the Frobenius normal $p$-complement theorem.

Here are some examples of applications of Burnside's and Frobenius'
normal $p$-complement theorems:

\begin{itemize}

\item Suppose $p$ is the least prime divisor of the order of a finite
  group $G$. if $G$ has a cyclic $p$-Sylow subgroup $P$, then $P$ has
  a normal complement in $G$.

\item Suppose $G$ is a group in which every Sylow subgroup is
  cyclic. Then, $G$ is solvable. The original proof used order
  computations involving elements, but the proof using the normal
  $p$-complement theorems is more intuitive.

\item If $G$ is a simple non-abelian group, either $12$ divides the
  order of $G$ or the cube of the smallest prime divisor of the order
  of $G$ divides the order of $G$.

\end{itemize}

\section{More powerful machinery}

\subsection{Replacement theorems, and more normal complement theorems}

The next level of normal complement theorems, proved by Thompson and
Glauberman, required a new array of techniques. We encountered
Thompson's out-of-the-box thinking in the critical subgroup
construction earlier.\footnote{Alas, out-of-the-box thinking ceases to
  be out-of-the-box pretty quickly. Thompson's methods were widely used by other group theorists.}

Here, we see {\em replacement}, yet another idea of Thompson. The idea
is: {\em if at first you pick a bad subgroup, replace it with a better
  one}. In \cite{thompsonsreplacement}, Thompson proved a replacement
theorem for abelian subgroups, showing that abelian subgroups could be
replaced by {\em better} abelian subgroups, using abelian normal
subgroups. Glauberman later proved a trickier replacement theorem
involving subgroups of class two. These replacement theorems led to
smarter choices that could prove stronger versions of normal
complement theorems. The upshot: better control than ever before on
the Sylow subgroups and $p$-structure of simple groups.

\subsection{Uniqueness subgroups and the maximal subgroup theorem}

As described earlier, $p$-local analysis studies the structure of
nontrivial $p$-subgroups and their normalizers, which are called
$p$-local subgroups. The structure of $p$-local subgroups can get
messy, and one of the many questions that may arise is whether the
collection of $p$-local subgroups has a largest element. The existence
of a largest element would make the study of questions of conjugacy
easier to handle.

One result in this direction is the maximal subgroup theorem due to
Thompson (see \cite[Theorem 8.6.3, Page 295--298]{gorenstein},for
instance). This gives condition under which certain subsets of the set
of normalizers of $p$-subgroups in a given $p$-Sylow subgroup have
maximal elements. When such a maximal element exists, it is termed a
\definedind{uniqueness subgroup}.

\begin{thebibliography}{99}

\bibitem{arithmeticandnormal}{\em Arithmetic and normal structure of
    finite groups} by Helmut Wielandt and Bertram Huppert,
  {\em Proc. Sympos. Pure Math.}, Vol. VI, Page 17--38 (Year 1962), MR
  \href{http://www.ams.org/mathscinet-getitem?mr=0147530}{0147530}

\bibitem{SolomonCFSG}{\em A brief history of the classification of the
    finite simple groups} by Ronald M. Solomon, {\em Bulletin of the
    American Mathematical Society}, ISSN 10889485 (electronic), ISSN
    02730979 (print), Volume 38, Page 315--352 (Year 2001): An
    expository paper by Ronald Mark Solomon describing the 110-year
    history of the classification of finite simple groups.

  \bibitem{gorenstein} {\em Finite Groups} by Daniel Gorenstein, {\em
      American Mathematical Society}, ISBN--10 0821843427, ISBN--13
    9780821843420

\bibitem{feitthompson} {\em Solvability of groups of odd order} by
  Walter Feit and John G. Thompson, {\em Pacific Journal of
  Mathematics}, Volume 13, Page 775--1029 (Year 1963), MR
  \href{http://www.ams.org/mathscinet-getitem?mr=0166261}{0166261}

\bibitem{benderonthompson}{\em Uber den grossten $p\sp{\prime}
    $-Normalteiler in $p$-auflosbaren Gruppen} by Helmut Bender,
  {\em Arch. Math. (Basel)}, Volume 18, Page 15--16 (1967), MR
  \href{http://www.ams.org/mathscinet-getitem?mr=0213439}{0213439}

\bibitem{alperinfusion}{\em Transfer and fusion in finite groups} by
  Jonathan L. Alperin and Daniel Gorenstein, {\em Journal of Algebra},
  ISSN 00218693, Volume 6, Page 242--255 (Year 1967), MR
  \href{http://www.ams.org/mathscinet-getitem?mr=0215914}{0215914}

\bibitem{GL}{\em Finite simple groups}, edited by Graham Higman and
  Martin B. Powell, {\em Academic Press}, ISBN--10 0125638507,
  ISBN--13 9780125638500

\bibitem{fusionintro}{\em Introduction to fusion systems} by Markus
  Linckelmann, URL \url{http://web.mat.bham.ac.uk/C.W.Parker/Fusion/fusion-intro.pdf}

\bibitem{higmanfocal}{\em Focal series in finite groups} by Donald
  G. Higman, {\em Canadian Journal of Mathematics}, Volume 5, Page
  477--497 (Year 1953), MR
  \href{http://www.ams.org/mathscinet-getitem?mr=0058597}{0058597}

\bibitem{thompsonsreplacement}{\em A replacement theorem for p-groups
    and a conjecture} by John G. Thompson, {\em Journal of Algebra},
  ISSN 00218693, Volume 13, Page 149--151 (Year 1969), MR
  \href{http://www.ams.org/mathscinet-getitem?mr=0245683}{0245683}

\end{thebibliography}

\end{document}
