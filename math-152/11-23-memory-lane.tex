\documentclass[10pt]{amsart}

%Packages in use
\usepackage{fullpage, hyperref, vipul, enumerate}

%Title details
\title{Class quiz: November 23: Memory lane}
\author{Math 152, Section 55 (Vipul Naik)}
%List of new commands

\begin{document}
\maketitle

Your name (print clearly in capital letters): $\underline{\qquad\qquad\qquad\qquad\qquad\qquad\qquad\qquad\qquad\qquad}$

\begin{enumerate}
\item For which of the following specifications is there {\bf no
  continuous function} satisfying the specifications?

  \begin{enumerate}[(A)]
  \item Domain $[0,1]$ and range $[0,1]$
  \item Domain $[0,1]$ and range $(0,1)$
  \item Domain $(0,1)$ and range $[0,1]$
  \item Domain $(0,1)$ and range $(0,1)$
  \item None of the above, i.e., we can get a continuous function for
    each of the specifications.
  \end{enumerate}

  \vspace{0.1in}
  Your answer: $\underline{\qquad\qquad\qquad\qquad\qquad\qquad\qquad}$
  \vspace{0.6in}

\item Suppose $f$ and $g$ are continuous functions on $\R$, such that
  $f$ is continuously differentiable everywhere and $g$ is
  continuously differentiable everywhere except at $c$, where it has a
  vertical tangent. What can we say is {\bf definitely true} about $f
  \circ g$?

  \begin{enumerate}[(A)]
  \item It has a vertical tangent at $c$.
  \item It has a vertical cusp at $c$.
  \item It has either a vertical tangent or a vertical cusp at $c$.
  \item It has neither a vertical tangent nor a vertical cusp at $c$.
  \item We cannot say anything for certain.
  \end{enumerate}

  \vspace{0.1in}
  Your answer: $\underline{\qquad\qquad\qquad\qquad\qquad\qquad\qquad}$
  \vspace{0.6in}

\item Consider the function $p(x) := x^{2/3}(x - 1)^{3/5} +
  (x-2)^{7/3}(x - 5)^{4/3}(x - 6)^{4/5}$. For what values of $x$ does
  the graph of $p$ have a vertical cusp at $(x,p(x))$?

  \begin{enumerate}[(A)]
  \item $x = 0$ only.
  \item $x = 0$ and $x = 5$ only.
  \item $x = 5$ and $x = 6$ only.
  \item $x = 0$ and $x = 6$ only.
  \item $x = 0$, $x = 5$, and $x = 6$.
  \end{enumerate}

  \vspace{0.1in}
  Your answer: $\underline{\qquad\qquad\qquad\qquad\qquad\qquad\qquad}$
  \vspace{0.6in}

\item Consider the function $f(x) := \lbrace\begin{array}{rl}x, & 0
  \le x \le 1/2 \\ x^2, & 1/2 < x \le 1\end{array}$. What is $f \circ f$?

  \begin{enumerate}[(A)]
  \item $x \mapsto \lbrace\begin{array}{rl} x, & 0 \le x \le 1/2\\ x^4, & 1/2 < x \le 1\end{array}$
  \item $x \mapsto \lbrace\begin{array}{rl} x, & 0 \le x \le 1/2\\ x^2, & 1/2 < x \le 1\end{array}$
  \item $x \mapsto \lbrace\begin{array}{rl} x, & 0 \le x \le 1/2\\ x^2, & 1/2 < x \le 1/\sqrt{2}\\ x^4, & 1/\sqrt{2} < x \le 1\end{array}$
  \item $x \mapsto \lbrace \begin{array}{rl} x, & 0 \le x \le 1/\sqrt{2}\\ x^2,& 1/\sqrt{2} < x \le 1\end{array}$
  \item $x \mapsto \lbrace\begin{array}{rl} x, & 0 \le x \le 1/\sqrt{2}\\ x^4, &1/\sqrt{2} < x \le 1\end{array}$
  \end{enumerate}

  \vspace{0.1in}
  Your answer: $\underline{\qquad\qquad\qquad\qquad\qquad\qquad\qquad}$
  \vspace{0.6in}

\item Suppose $f$ and $g$ are functions $(0,1)$ to $(0,1)$ that are
  both right continuous on $(0,1)$. Which of the following is {\em not}
  guaranteed to be right continuous on $(0,1)$?

  \begin{enumerate}[(A)]
  \item $f + g$, i.e., the function $x \mapsto f(x) + g(x)$
  \item $f - g$, i.e., the function $x \mapsto f(x) - g(x)$
  \item $f \cdot g$, i.e., the function $x \mapsto f(x)g(x)$
  \item $f \circ g$, i.e., the function $x \mapsto f(g(x))$
  \item None of the above, i.e., they are all guaranteed to be right
    continuous functions
  \end{enumerate}

  \vspace{0.1in}
  Your answer: $\underline{\qquad\qquad\qquad\qquad\qquad\qquad\qquad}$
  \vspace{0.6in}

\item For a partition $P = x_0 < x_1 < x_2 < \dots < x_n$ of $[a,b]$
  (with $x_0 = a$, $x_n = b$) define the norm $\| P \|$ as the maximum
  of the values $x_i - x_{i-1}$. Which of the following {\bf is always
  true} for any continuous function $f$ on $[a,b]$? (5 points)

  \begin{enumerate}[(A)]
  \item If $P_1$ is a finer partition than $P_2$, then $\| P_2 \| \le
    \| P_1 \|$ (Here, {\em finer} means that, as a set, $P_2 \subseteq
    P_1$, i.e., all the points of $P_2$ are also points of $P_1$).
  \item If $\| P_2 \| \le \| P_1 \|$, then $L_f(P_2) \le L_f(P_1)$
    (where $L_f$ is the lower sum).
  \item If $\| P_2 \| \le \| P_1 \|$, then $U_f(P_2) \le U_f(P_1)$
    (where $U_f$ is the upper sum).
  \item If $\| P_2 \| \le \| P_1 \|$, then $L_f(P_2) \le U_f(P_1)$.
  \item All of the above.
  \end{enumerate}

  \vspace{0.1in}
  Your answer: $\underline{\qquad\qquad\qquad\qquad\qquad\qquad\qquad}$
  \vspace{0.6in}

\item A disk of radius $r$ in the $xy$-plane is translated parallel to
  itself with its center moving in the $yz$-plane along the semicircle
  $y^2 + z^2 = R^2, y \ge 0$. The solid thus obtained can be thought
  of as a {\em cylinder of bent spine} with cross sections being disks
  of radius $r$ along the $xy$-plane and the centers forming a
  semicircle of radius $R$ in the $yz$-plane, with the $z$-value
  ranging from $-R$ to $R$. What is the volume of this solid?

  \begin{enumerate}[(A)]
  \item $2\pi r^2R$
  \item $\pi^2r^2R$
  \item $2\pi rR^2$
  \item $\pi^2rR^2$
  \item $\pi^2R^3$
  \end{enumerate}

  \vspace{0.1in}
  Your answer: $\underline{\qquad\qquad\qquad\qquad\qquad\qquad\qquad}$
  \vspace{0.6in}

\end{enumerate}
\end{document}