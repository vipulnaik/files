\documentclass[10pt]{amsart}

%Packages in use
\usepackage{fullpage, hyperref, vipul, enumerate}

%Title details
\title{Class quiz: November 21: One-one functions}
\author{Math 152, Section 55 (Vipul Naik)}
%List of new commands

\begin{document}
\maketitle

Your name (print clearly in capital letters): $\underline{\qquad\qquad\qquad\qquad\qquad\qquad\qquad\qquad\qquad\qquad}$

\begin{enumerate}
\item For one of these function types for a continuous function from
  $\R$ to $\R$, it is {\em possible} to also be a one-to-one
  function. What is that function type? {\em Last year: $15/15$
    correct}

  \begin{enumerate}[(A)]
  \item Function whose graph has mirror symmetry about a vertical line.
  \item Function whose graph has half turn symmetry about a point on it.
  \item Periodic function.
  \item Function having a point of local minimum.
  \item Function having a point of local maximum.
  \end{enumerate}

  \vspace{0.1in}
  Your answer: $\underline{\qquad\qquad\qquad\qquad\qquad\qquad\qquad}$
  \vspace{0.6in}

\item (**) Suppose $f$, $g$, and $h$ are continuous one-to-one
  functions whose domain and range are both $\R$. {\bf What can we
    say} about the functions $f + g$, $f + h$, and $g + h$? {\em Last
    year: $2/15$ correct}

  \begin{enumerate}[(A)]
  \item They are all continuous one-to-one functions with domain $\R$
    and range $\R$.
  \item At least two of them are continuous one-to-one functions with
    domain $\R$ and range $\R$ -- however, we cannot say more.
  \item At least one of them is a continuous one-to-one function with
    domain $\R$ and range $\R$ -- however, we cannot say more.
  \item Either all three sums are continuous one-to-one functions
    whose domain and range are both $\R$, or none is.
  \item It is possible that none of the sums is a continuous
    one-to-one function whose domain and range are both $\R$; it is
    also possible that one, two, or all the sums are continuous
    one-to-one functions whose domain and range are both $\R$.
  \end{enumerate}

  \vspace{0.1in}
  Your answer: $\underline{\qquad\qquad\qquad\qquad\qquad\qquad\qquad}$
  \vspace{0.6in}

\item (**) Suppose $f$ is a one-to-one function with domain a closed
  interval $[a,b]$ and range a closed interval $[c,d]$. Suppose $t$ is
  a point in $(a,b)$ such that $f$ has left hand derivative $l$ and
  right-hand derivative $r$ at $t$, with both $l$ and $r$
  nonzero. What is the left hand derivative and right hand derivative
  to $f^{-1}$ at $f(t)$? {\em Last year: $6/15$ correct}

  \begin{enumerate}[(A)]
  \item The left hand derivative is $1/l$ and the right hand
    derivative is $1/r$.
  \item The left hand derivative is $-1/l$ and the right hand
    derivative is $-1/r$.
  \item The left hand derivative is $1/r$ and the right hand
    derivative is $1/l$.
  \item The left hand derivative is $-1/r$ and the right hand
    derivative is $-1/l$.
  \item The left hand derivative is $1/l$ and the right hand
    derivative is $1/r$ if $l > 0$, otherwise the left hand derivative
    is $1/r$ and the right hand derivative is $1/l$.
  \end{enumerate}
  
  \vspace{0.1in}
  Your answer: $\underline{\qquad\qquad\qquad\qquad\qquad\qquad\qquad}$
  \vspace{0.6in}


\item (**) Which of these functions is one-to-one? {\em Last year:
    $2/15$ correct}

  \begin{enumerate}[(A)]
  \item $f_1(x) := \lbrace \begin{array}{rl} x, & x \text{ rational} \\ x^2, & x \text{ irrational}\\\end{array}$ 
  \item $f_2(x) := \lbrace \begin{array}{rl} x, & x \text{ rational} \\ x^3, & x \text{ irrational}\\\end{array}$
  \item $f_3(x) := \lbrace\begin{array}{rl} x, & x \text{ rational} \\ 1/(x - 1), & x \text{ irrational}\\\end{array}$
  \item All of the above
  \item None of the above
  \end{enumerate}

  \vspace{0.1in}
  Your answer: $\underline{\qquad\qquad\qquad\qquad\qquad\qquad\qquad}$
  \vspace{0.6in}

\item (**) Consider the following function $f:[0,1] \to [0,1]$ given
  by
  $f(x) := \lbrace\begin{array}{rl} \sin(\pi x/2), & 0 \le x \le 1/2 \\
    \sqrt{x}, & 1/2 < x \le 1\\\end{array}$. What is the correct
  expression for $(f^{-1})'(1/2)$? {\em Last year: $1/15$ correct}

  \begin{enumerate}[(A)]
  \item It does not exist, since the two one-sided derivatives of $f$ at
    $1/2$ do not match.
  \item $\sqrt{2}$
  \item $2\sqrt{2}/\pi$
  \item $4/\pi$
  \item $4/(\sqrt{3}\pi)$
  \end{enumerate}

  \vspace{0.1in}
  Your answer: $\underline{\qquad\qquad\qquad\qquad\qquad\qquad\qquad}$
  \vspace{0.6in}

\end{enumerate}
\end{document}