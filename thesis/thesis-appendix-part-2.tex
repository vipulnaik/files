\section{Some computational proofs related to the Baker-Campbell-Hausdorff formula}

\subsection{Baker-Campbell-Hausdorff formula: class two: full derivation}\label{appsec:bch-class-two}

In this case, we work with non-commuting variables $x_1,x_2$ such that
$x_1^3 = x_1^2x_2 = x_1x_2x_1 = x_1x_2^2 = x_2x_1^2 = x_2x_1x_2 =
x_2^2x_1 = x_2^3 = 0$. Thus:

$$ \exp(x_1) = 1 + x_1 + \frac{x_1^2}{2}$$

$$ \exp(x_2) = 1 + x_2 + \frac{x_2^2}{2}$$

We thus get:

$\exp(x_1)\exp(x_2) = \left(1 + x_1 + \frac{x_1^2}{2}\right)\left(1 + x_2 + \frac{x_2^2}{2}\right) = 1 + x_1 + \frac{x_1^2}{2} + x_2 + x_1x_2 + \frac{x_1^2x_2}{2} + \frac{x_2^2}{2} + \frac{x_1x_2^2}{2} + \frac{x_1^2x_2^2}{4}$

We drop all products of degree three or more and rearrange the remaining terms to get:

$$ \exp(x_1)\exp(x_2) = 1 + x_1 + x_2 + \frac{x_1^2}{2} + x_1x_2 + \frac{x_2^2}{2}$$

We thus get:

$$ w = \exp(x_1)\exp(x_2) - 1 = x_1 + x_2 + \frac{x_1^2}{2} + x_1x_2 + \frac{x_2^2}{2}$$

Finally, we compute $\log(1 + w)$. We have:

$$\log(1 + w) = w - \frac{w^2}{2}$$

We note that $w^2$ is the same as the square of its linear part
because the square of the degree two part, as well as products of the
degree two and the linear part, are degree three or more and hence
zero. Thus:

$\log(1 + w) = x_1 + x_2 + \frac{x_1^2}{2} + x_1x_2 + \frac{x_2^2}{2} - (x_1 + x_2)^2/2$

Simplifying, we get:

$\log(1 + w) = x_1 + x_2 + \frac{x_1^2 + 2x_1x_2 + x_2^2 - (x_1 + x_2)^2}{2}$

Now note that:

$ (x_1 + x_2)^2 = x_1^2 + x_1x_2 + x_2x_1 + x_2^2$

Plugging this in, we get:

$\log(1 + w) = x_1 + x_2 + \frac{x_1^2 + 2x_1x_2 + x_2^2 - (x_1^2 + x_1x_2 + x_2x_1 + x_2^2)}{2}$

Simplifying, we get:

$\log(1 + w) = x_1 + x_2 + \frac{x_1x_2 - x_2x_1}{2}$

Rewrite $x_1x_2 - x_2x_1 = [x_1,x_2]$ and we get the formula:

$ x_1 + x_2 + \frac{1}{2}[x_1,x_2]$

%% Recall that this {\em is the same as the formula for the group product
%%   of $x_1$ and $x_2$ in terms of the Lie ring operations} that we saw
%% in the Baer correspondence. This is not a coincidence. The
%% significance will become clearer soon.

\subsection{Baker-Campbell-Hausdorff formula: class three: full derivation}\label{appsec:bch-class-three}

Before proceeding to work out the formula in class three, we obtain a
more concise description of $w$ in terms of $x_1$ and $x_2$, thus
saving steps on the initial computation. If we are working in class $c$, then:

$$w = \sum_{k,l \ge 0, 0 < k + l \le c} \frac{x_1^kx_2^l}{k!l!} = \sum_{n=1}^c \frac{1}{n!} \sum_{k=0}^n \binom{n}{k} x_1^kx_2^{n-k}$$

We now deduce the class three Baker-Campbell-Hausdorff formula:

$w = \exp(x_1)\exp(x_2) -1 = (x_1 + x_2) + \frac{1}{2!}(x_1^2 + 2x_1x_2 + x_2^2) + \frac{1}{3!}(x_1^3 + 3x_1^2x_2 + 3x_1x_2^2 + x_2^3)$

Since the class is three, we have $w^4 = 0$, hence we get:

$\log(1 + w) = w - \frac{w^2}{2} + \frac{w^3}{3} = (x_1 + x_2) + \frac{1}{2!}(x_1^2 + 2x_1x_2 + x_2^2) + \frac{1}{3!}(x_1^3 + 3x_1^2x_2 + 3x_1x_2^2 + x_2^3) - \frac{1}{2}((x_1 + x_2) + \frac{1}{2!}(x_1^2 + 2x_1x_2 + x_2^2) + \frac{1}{3!}(x_1^3 + 3x_1^2x_2 + 3x_1x_2^2 + x_2^3))^2 + \frac{1}{3}((x_1 + x_2) + \frac{1}{2!}(x_1^2 + 2x_1x_2 + x_2^2) + \frac{1}{3!}(x_1^3 + 3x_1^2x_2 + 3x_1x_2^2 + x_2^3))^3$

The calculations for the degree one and degree two parts proceed
exactly as they did in the class two case covered in the preceding
section. We thus concentrate on the degree three part:

$\mbox{Degree three part} = \frac{1}{6}(x_1^3 + 3x_1^2x_2 + 3x_1x_2^2 + x_2^3) - \frac{1}{4}(x_1 + x_2)(x_1^2 + 2x_1x_2 + x_2^2) - \frac{1}{4}(x_1^2 + 2x_1x_2 + x_2^2) + \frac{1}{3}(x_1 + x_2)^3$

Instead of simplifying directly, we adopt the following procedure. We rewrite:

$x_1^2 + 2x_1x_2 + x_2^2 = (x_1 + x_2)^2 + [x_1,x_2]$

We similarly rewrite:

$x_1^3 + 3x_1^2x_2 + 3x_1x_2^2 + x_2^3 = (x_1 + x_2)^3 + 2x_1[x_1,x_2] + [x_1,x_2]x_1 + 2[x_1,x_2]x_2 + x_2[x_1,x_2]$

This can be further rewritten as:

$x_1^3 + 3x_1^2x_2 + 3x_1x_2^2 + x_2^3 = (x_1 + x_2)^3 + 3x_1[x_1,x_2] + 3[x_1,x_2]x_2 - [x_1,[x_1,x_2]] + [x_2,[x_1,x_2]]$

Now, we plug these into the expression 

$\mbox{Degree three part} = \frac{1}{6}((x_1 + x_2)^3 + 3x_1[x_1,x_2] + 3[x_1,x_2]x_2 - [x_1,[x_1,x_2]] + [x_2,[x_1,x_2]]) - \frac{1}{4}(x_1 + x_2)((x_1 + x_2)^2 + [x_1,x_2]) - \frac{1}{4}((x_1 + x_2)^2 + [x_1,x_2])(x_1 + x_2) + \frac{1}{3}(x_1 + x_2)^3$

We rearrange to obtain:

$\mbox{Degree three part} = \left(\frac{1}{6} - \frac{1}{4} - \frac{1}{4} + \frac{1}{3}\right)(x_1 + x_2)^3 + \frac{1}{2}(x_1[x_1,x_2] + [x_1,x_2]x_2) - \frac{1}{6}([x_1,[x_1,x_2]] - [x_2,[x_1,x_2]]) - \frac{1}{4}(x_1 + x_2)[x_1,x_2] - \frac{1}{4}([x_1,x_2](x_1 + x_2))$

The $(x_1 + x_2)^3$ term has zero coefficient and disappears, and we are left with:

$\mbox{Degree three part} = \frac{1}{2}(x_1[x_1,x_2] + [x_1,x_2]x_2) - \frac{1}{6}([x_1,[x_1,x_2]] - [x_2,[x_1,x_2]]) - \frac{1}{4}x_1[x_1,x_2] - \frac{1}{4}x_2[x_1,x_2] - \frac{1}{4}[x_1,x_2]x_1 - \frac{1}{4}[x_1,x_2]x_2$

We now use that $x_2[x_1,x_2] = [x_1,x_2]x_2 + [x_2,[x_1,x_2]]$ and $[x_1,x_2]x_1 = x_1[x_1,x_2] - [x_1,[x_1,x_2]]$ to get:

$\mbox{Degree three part} = \frac{1}{2}(x_1[x_1,x_2] + [x_1,x_2]x_2) - \frac{1}{6}([x_1,[x_1,x_2]] - [x_2,[x_1,x_2]]) - \frac{1}{4}x_1[x_1,x_2] - \frac{1}{4}[x_1,x_2]x_2 - \frac{1}{4}[x_2,[x_1,x_2]] - \frac{1}{4}x_1[x_1,x_2] + \frac{1}{4}[x_1,[x_1,x_2]] - \frac{1}{4}[x_1,x_2]x_2$

Combining coefficients, we find that the coefficients on $x_1[x_1,x_2]$ and $[x_1,x_2]x_2$ are zero, and we are left with:

$\mbox{Degree three part} = - \frac{1}{6}([x_1,[x_1,x_2]] - [x_2,[x_1,x_2]]) + \frac{1}{4}([x_1,[x_1,x_2]] - [x_2,[x_1,x_2]])$

We simplify $1/4 - 1/6 = 1/12$ to get:

$\mbox{Degree three part} = \frac{1}{12}([x_1,[x_1,x_2]] - [x_2,[x_1,x_2]])$

Plug this back in to the formula, and get the overall formula:

$x_1 + x_2 + \frac{1}{2}[x_1,x_2] + \frac{1}{12}([x_1,[x_1,x_2]] - [x_2,[x_1,x_2]])$

\subsection{Bounds on prime power divisors of the denominator}\label{appsec:bch-prime-power-divisor-bound}

For a prime $p$ and a natural number $c$, define $f(p,c)$ as
follows. Consider the class $c$ Baker-Campbell-Hausdorff
formula. $f(p,c)$ is defined as the largest positive integer $k$ such
that $p^k$ appears as a divisor of the denominator for one of the
coefficients for the formula.

It turns out that:

$$f(p,c) \le \left \lfloor \frac{c - 1}{p - 1} \right \rfloor$$

This was proved in Lazard's original paper (\cite{Lazardsoriginal}).
%% {\em TONOTDO: Insert position within paper}
The proof sketch for the {\em associative} version of the formula is
below. The result for the Lie version of the formula follows from
Khukhro's text \cite{Khukhro}, Theorem 5.39 (see more generally the
discussion in Sections 5.3 and 9.9 of the text).

Consider the ring of formal power series over $\mathbb{Q}$ in the two non-commuting variables $x_1$ and $x_2$.

Consider a $p$-adic valuation on $\mathbb{Q}$, i.e., a valuation $v_p: \mathbb{Q} \setminus \{ 0 \} \to \mathbb{Z}$ that sends a rational number $a/b$ to the integer $k$ such that $a/(bp^k)$ in reduced form has no divisor of $p$ for either the numerator or the denominator.

Extend the valuation to a value $1/(p-1)$ on the formal variables
$x_1,x_2$. The valuation can then be extended to the whole ring. We
then use multiplicativity to compute the valuation at various terms,
for $n \le c$:

$\! v_p(x_1^n/n!)=c/(p-1) - v_p(n!) \geq 1/(p-1)$

where we have used that $v_p(n!) \leq
\left \lfloor(n-1)/(p-1)\right\rfloor$. Then, $v_p(\exp x_1 - 1)\geq
1/(p-1)$. Denote $w = \exp(x_1)\exp(x_2) - 1$. It is easy to see that
$v_p(w) \geq 1/(p-1)$. The Baker-Campbell-Hausdorff formula is
obtained by expanding

$\! \log(1 + w)$

Note that $v_p(w^n/n)=nv_p(w)-v_p(c)\geq n/(p-1) - v_p(n!) \geq
1/(p-1)$. That is, $v_p(\log(1 + w))\geq 1/(p-1)$. For coefficients in
degree $n$ with $n \le c$, we obtain:

$$v_p(\mbox{coefficient in degree n}) \geq 1/(p-1) - n/(p-1)=-(n-1)/(p-1)$$

Since the above holds for all $n \le c$, we obtain that all the
coefficients in degree $\le c$ have prime power divisors of the
denominator less than or equal to $(c - 1)/(p - 1)$. Thus, $f(p,c) \le
(c - 1)/(p - 1)$. Since $f(p,c)$ is an integer, we obtain that:

$$f(p,c) \le \left \lfloor \frac{c - 1}{p - 1}\right \rfloor$$

\subsection{Finding the explicit formula $M_{c+1}$ for $c = 2$}\label{appsec:M3formula}

In Section \ref{sec:group-commutator-ito-lie-bracket}, we described
the general approach for computing a formula $M_{c+1}$ to describe the
group commutator in terms of the Lie bracket for the class $(c + 1)$
Lazard correspondence. The case $c = 1$ (and hence $c + 1 = 2$) was
discussed in detail in Section \ref{sec:baer-correspondence-basics}
(Lemma \ref{lemma:baer-correspondence-lie-ring-to-group} says this
explicitly). We consider the case $c = 2$, so that $c + 1 = 3$. In
other words, we are considering the class $3$ Lazard correspondence.

We mimic the general procedure of Section
\ref{sec:group-commutator-ito-lie-bracket}.

The class three Baker-Campbell-Hausdorff formula gives us that:

\begin{eqnarray*}
  xy & = & x + y + \frac{1}{2}[x,y] + \frac{1}{12}([x,[x,y]] - [y,[x,y]])\\
  yx & = & y + x + \frac{1}{2}[y,x] + \frac{1}{12}([y,[y,x]] - [x,[y,x]])\\
\end{eqnarray*}

We will denote the degree $i$ part as $t_i$, as in the discussion of
the Baker-Campbell-Hausdorff formula. In other words, we have:

\begin{eqnarray*}
  xy & = & t_1(x,y) + t_2(x,y) + t_3(x,y)\\
  yx & = & t_1(y,x) + t_2(y,x) + t_3(y,x)\\
\end{eqnarray*}

Here:

$$t_1(x,y) = t_1(y,x) = x + y = y + x$$

$$t_2(x,y) = -t_2(y,x) = \frac{1}{2}[x,y]$$

$$t_3(x,y) = t_3(y,x) = \frac{1}{12}([x,[x,y]] - [y,[x,y]]) = \frac{1}{12}([y,[y,x]] - [x,[y,x]])$$

Thus:

$$[x,y]_{\text{Group}} = (xy)(-(yx))$$

becomes:

$t_1(x,y) + t_2(x,y) + t_3(x,y) - (t_1(y,x) + t_2(y,x) + t_3(y,x)) +$

$t_2(xy,-(yx)) + t_3(xy,-(yx))$

Based on the relationships above, this simplifies to:

$[x,y]_{\text{Group}} = 2t_2(x,y) + t_2(xy,-(yx)) + t_3(xy,-(yx)) \qquad (*)$

We now expand each of the other terms. We have:

$t_2(xy,-(yx)) = \frac{1}{2}\left[t_1(x,y) + t_2(x,y) + t_3(x,y),- (t_1(y,x) + t_2(y,x) + t_3(y,x))\right]$

Expanding this out, we obtain:

$t_2(xy,-(yx)) = \frac{1}{2}[t_1(x,y),-t_2(y,x)] + \frac{1}{2}[t_2(x,y), -t_1(y,x)]$

$ = [t_1(x,y),t_2(x,y)] = \frac{1}{2}[x + y,[x,y]] \qquad (**)$

On the other hand:

$t_3(xy,-(yx)) = t_3(t_1(x,y) + t_2(x,y) + t_3(x,y),-(t_1(y,x) + t_2(y,x) + t_3(y,x))$

$ = t_3(t_1(x,y),-t_1(y,x)) = 0 \qquad (***) $

Plugging (**) and (***) into (*), we obtain that:

$[x,y]_{\text{Group}} = 2t_2(x,y) + \frac{1}{2}[x + y,[x,y]]$

This simplifies to:

$$[x,y]_{\text{Group}} = [x,y] + \frac{1}{2}[x + y,[x,y]]$$

Thus, we get:

$$M_3(x,y) = [x,y] + \frac{1}{2}[x + y,[x,y]]$$

Note that the formula would look somewhat different if we used the
right action convention for the group commutator. Explicitly, the
formula with the right action convention, which would be the formula
for the group commutator $x^{-1}y^{-1}xy$, would be:

$$x^{-1}y^{-1}xy = [x,y] - \frac{1}{2}[x + y,[x,y]]$$

\subsection{Finding the explicit formula $M_{c+1}$ for $c = 3$}\label{appsec:M4formula}

The steps here are very similar to the steps for the preceding
example, so we go over the steps very briefly. We have:

\begin{eqnarray*}
  xy & = & t_1(x,y) + t_2(x,y) + t_3(x,y) + t_4(x,y)\\
  yx & = & t_1(y,x) + t_2(y,x) + t_3(y,x) + t_4(y,x)\\
\end{eqnarray*}

Here:

$$t_1(x,y) = t_1(y,x) = x + y = y + x$$

$$t_2(x,y) = -t_2(y,x) = \frac{1}{2}[x,y]$$

$$t_3(x,y) = t_3(y,x) = \frac{1}{12}([x,[x,y]] - [y,[x,y]]) = \frac{1}{12}([y,[y,x]] - [x,[y,x]])$$

$$t_4(x,y) = -t_4(y,x) = - \frac{1}{24}[y,[x,[x,y]]]$$

Based on the above relationships, we get:

$[x,y]_{\text{Group}} = (xy)(-(yx)) = 2(t_2(x,y) + t_4(x,y)) + t_2(xy,-(yx)) + t_3(xy,-(yx)) + t_4(xy,-(yx))$

The last expression $t_4(xy,-(yx))$ is $0$ based on general reasons. We thus get:

$[x,y]_{\text{Group}} = 2(t_2(x,y) + t_4(x,y)) + t_2(xy,-(yx)) + t_3(xy,-(yx))$

We simplify the pieces separately. We have:

$t_2(xy,-(yx)) = \frac{1}{2}[t_1(x,y) + t_2(x,y) + t_3(x,y) + t_4(x,y),-(t_1(y,x) + t_2(y,x) + t_3(y,x) + t_4(y,x))]$

Note that any pair involving $t_4$ becomes zero, so this becomes:

$t_2(xy,-(yx)) = \frac{1}{2}[t_1(x,y) + t_2(x,y) + t_3(x,y),-t_1(x,y) + t_2(x,y) - t_3(x,y)]$

Note that $[t_1(x,y),t_3(x,y)]$ and $[t_3(x,y),t_1(x,y)]$
cancel. Also, the products $[t_i(x,y),t_j(x,y)]$ are zero for $i + j
\ge 5$, and also for $i = j$. Thus, the only products that survive are
$[t_1(x,y),t_2(x,y)]$ and $[t_2(x,y),-t_1(x,y)]$, and we obtain:

$t_2(xy,-(yx)) = \frac{1}{2}[x + y,[x,y]] \qquad (**)$

We now simplify $t_3(xy,-(yx))$:

$t_3(xy,-(yx)) = \frac{1}{12}[xy,[xy,-(yx)]] - \frac{1}{12}[-(yx),[xy,-(yx)]]$

The product $[xy,-(yx)]$ on the inside simplifies to $[x + y,[x,y]]$
based on the above calculations. Thus, we get:

$t_3(xy,-(yx)) = \frac{1}{12}[xy + yx,[x + y,[x,y]]$

We know that $xy + yx = 2t_1(x,y) + 2t_3(x,y)$. Thus:

$t_3(xy,-(yx)) = \frac{1}{12}[2t_1(x,y),[x+y,[x,y]]] + \frac{1}{12}[2t_3(x,y),[x+y,[x,y]]]$

The second term is zero because the degree is six. Simplifying the first term, we get:

$t_3(xy,-(yx)) = \frac{1}{6}[x + y,[x+y,[x,y]]] \qquad (***)$

Plugging (**) and (***) into the original formula (*), we obtain:

$[x,y]_{\text{Group}} = [x,y] - \frac{1}{12}[y,[x,[x,y]]] + \frac{1}{2}[x + y,[x,y]] + \frac{1}{6}[x + y,[x + y,[x,y]]]$ 

Rearranging, we obtain:

$[x,y]_{\text{Group}} = [x,y] + \frac{1}{2}[x + y,[x,y]] + \frac{1}{6}[x + y,[x + y,[x,y]]] - \frac{1}{12}[y,[x,[x,y]]]$ 

\subsection{Computing the second inverse Baker-Campbell-Hausdorff formula in the case $c = 2$, and an illustration of why it involves only strictly smaller primes}\label{appsec:lie-bracket-denominators-illustration}

We will use the case $c = 2$ to illuminate the discussion in Section
\ref{sec:lie-bracket-ito-group-commutator}, specifically the proof of
Lemma \ref{lemma:lie-bracket-denominators}.

We have:

$M_3(x,y) = [x,y] + \frac{1}{2}[x + y,[x,y]]$

Our goal is to find the expression for $h_{2,3}(x,y)$. In the class
two case, the group commutator and Lie bracket coincide, so we know
that:

$M_2(x,y) = [x,y], \qquad h_{2,2}(x,y) = [x,y]_{\text{Group}}$

Following the notation of Lemma \ref{lemma:lie-bracket-denominators}, we have that:

$(h_{2,2} \circ M_2)(x,y) = [x,y]$

Note that this case is remarkable because the equality holds {\em
  exactly}, rather than just modulo $\mathcal{A}^{c+1}$. In
particular, this means that the degree $(c + 1)$ expression
$\chi_{c+1}(x,y)$ defined in Lemma
\ref{lemma:lie-bracket-denominators} as the expression such that:

$(h_{2,2} \circ M_2)(x,y) = [x,y]_{\text{Lie}} + \chi_{c+1}(x,y) \pmod{\mathcal{A}^{c+2}}$

turns out to be zero, i.e., $\chi_{c+1}(x,y) = 0$.

We also have that:

$$M_3(x,y) = M_2(x,y) + \xi_{c+1}(x,y)$$

where $\xi_{c+1}(x,y) = \frac{1}{2}[x + y,[x,y]]$. Thus, we obtain that:

$$(h_{2,2} \circ M_3)(x,y) = [x,y]_{\text{Lie}} + \chi_{c+1}(x,y) + \xi_{c+1}(x,y)$$

or more explicity:

$$(h_{2,2} \circ M_3)(x,y) = [x,y]_{\text{Lie}} + \frac{1}{2}[x + y,[x,y]]$$

It therefore follows that:

$$h_{2,3}(x,y) = \frac{[x,y]}{\sqrt{[xy,[x,y]]}}$$

%%TONOTDO: Make this clearer

\section{Some results involving local nilpotency class}


\subsection{$3$-local class three implies global class three for Lie rings}\label{appsec:3-local-class-three-implies-global-class-three}

We prove that nilpotency class three is $3$-local.

\begin{lemma}
  Suppose $L$ is a Lie ring with the property that for any subset of
  $L$ of size at most three, the Lie subring generated by that subset
  is a subring of nilpotency class at most three. In other words, $L$
  has $3$-local nilpotency class at most three. Then, $L$ is a
  nilpotent Lie ring and its nilpotency class is at most three.
\end{lemma}

\begin{proof}
  The map $(w,x,y,z) \mapsto [w,[x,[y,z]]]$ is alternating and
  multi-linear, and hence skew-symmetric, in all pairs of inputs. The
  ``alternating'' condition follows from the $3$-local class three
  condition: whenever two inputs are the same, the product is a degree
  four product in a subring generated by three elements, hence it must
  equal zero. In particular, this means that the sign of the
  expression $[w,[x,[y,z]]]$ is reversed under any odd permutation of
  the inputs and is preserved under any even permutation of the
  inputs.

  We will show that for all $w$, $x$, $y$, $z \in L$, the Lie bracket
  $[w,[x,[y,z]]]$ equals $0$. The elements $w$, $x$, $y$, and $z$ are
  fixed but arbitrary for the duration of this proof.

  
  In particular, we obtain that $[w,[x,[y,z]]] = [w,[y,[z,x]]] =
  [w,[z,[x,y]]]$ for all $w,x,y,z \in L$. Combining with the Jacobi
  identity, we obtain that for all $w,x,y,z \in L$, we have that:

  $$3[w,[x,[y,z]]] = 0 \qquad (\dagger)$$

  We also obtain that 

  $$[w,[x,[y,z]]] = [[y,z],[w,x]] \qquad (*)$$

  The proof is as follows: Use the Jacobi identity to get
  $[w,[x,[y,z]]] + [x,[[y,z],w]] + [[y,z],[w,x]] = 0$. Now, the middle
  term $[x,[[y,z],w]]$ is the negative of $[x,[w,[y,z]]]$, which by
  the alternating condition we know to be the negative of
  $[w,[x,[y,z]]]$. So, the middle term equals $[w,[x,[y,z]]]$. Thus,
  the Jacobi identity expression gives $2[w,[x,[y,z]]] + [[y,z],[w,x]]
  = 0$. Combine with ($\dagger$) to obtain that $[w,[x,[y,z]]] =
  [[y,z],[w,x]]$.

  Similarly, we obtain that:

  $$[y,[z,[w,x]]] = [[w,x],[y,z]] \qquad (**)$$

  Combine (*), (**), and the fact that $[w,[x,[y,z]]] = [y,[z,[w,x]]]$
  because of the alternating nature of the map, and obtain that:

  $$[[y,z],[w,x]] = [[w,x],[y,z]] \qquad (***)$$

  On the other hand, we have, by the alternating nature of the Lie bracket, that:

  $$[[y,z],[w,x]] = -[[w,x],[y,z]] \qquad (****)$$

  Combining (***) and (****), we obtain that:

  $$2[[y,z],[w,x]] = 0$$

  Combining with (*), we obtain that:

  $$2[w,[x,[y,z]]] = 0$$

  Combining with ($\dagger$), we obtain that:

  $$[w,[x,[y,z]]] = 0$$

  as desired.
\end{proof}
%\newpage

%% {\em TONOTDO: Fill this in later}

%% %\newpage


%% \section{Operads}\label{appsec:operads}

%% {\em TONOTDO: Fill in}

%% %\newpage

%% \section{Multiplicative Lie rings}\label{appsec:multiplicative-lie-rings}

%% {\em TONOTDO: Fill in}
 
%% %\newpage

%% \section{Exploring groups of small order}

\section{Proofs related to isoclinism}\label{appsec:isoclinism-extra-proofs}

We begin with the proof of Theorem
\ref{isoclinic-same-proportions-conjugacy-class-sizes}. The theorem is
restated below.

\begin{quote}
  Suppose $G_1$ and $G_2$ are isoclinic groups. Suppose $c$ is a
  positive integer. Let $m_1$ be the number of conjugacy classes in
  $G_1$ of size $c$ (so that the {\em total} number of elements in
  such conjugacy classes is $m_1c$). Let $m_2$ be the number of
  conjugacy classes in $G_2$ of size $c$ (so that the {\em total}
  number of elements in such conjugacy classes is $m_2c$). Then, $m_1$
  is nonzero if and only if $m_2$ is nonzero, and if so, $m_1/m_2 =
  |G_1|/|G_2|$.

  In particular, if $G_1$ and $G_2$ additionally have the same order,
  then they have precisely the same multiset of conjugacy class sizes.

\end{quote}

\begin{proof}
  Let $W$ be the group identified with $\operatorname{Inn}(G_1) \cong
  \operatorname{Inn}(G_2)$, and $T$ be the group identified with $G_1'
  \cong G_2'$. Denote by $\alpha_1: G_1 \to W$ and $\alpha_2: G_2 \to
  W$ the respective quotient maps. Denote by $\omega: W \times W \to
  T$ the group commutator map. Note that the map $\omega$ is the same
  for both groups -- that's precisely the point of their being
  isoclinic.

  For $w \in W$, the centralizer in $G_1$ of any element in
  $\alpha_1^{-1}(w)$ is precisely $\alpha_1^{-1}(\mathcal{C}(w))$
  where
  
  $$\mathcal{C}(w) = \{ u \in W \mid \omega(u,w) \mbox{ is the identity element of } T \}$$

  Thus, the size of the conjugacy class in $G_1$ of any element in
  $\alpha_1^{-1}(w)$ is the index of the subgroup $\mathcal{C}(w)$ in $W$.
 
  From this, it follows that the set of elements of $G_1$ with
  conjugacy class size $c$ is $\alpha_1^{-1}(S)$ where $S$ is the set
  of $w \in W$ for which the index of the subgroup $\mathcal{C}(w) =
  \{ u \in W \mid \omega(u,w) \mbox{ is the identity element of } T
  \}$ in $W$ is $c$.

  Thus, we get the equality of the following two expressions for the
  number of elements of $G_1$ in conjugacy classes of size $c$:

  $$m_1c = |S||Z(G_1)|$$

  Analogously, we have:

  $$m_2c = |S||Z(G_2)|$$

  The crucial thing to note is that the subset $S$ of $W$ is the same
  in both cases.

  Taking the quotient, we get that $m_1$ is nonzero if and only if
  $m_2$ is nonzero, and if so:

  $$\frac{m_1}{m_2} = \frac{|Z(G_1)|}{|Z(G_2)|}$$

  Since $[G_1:Z(G_1)] = |W| = [G_2:Z(G_2)]$, we have 
  $|Z(G_1)|/|Z(G_2)| = |G_1|/|G_2|$, so we obtain:

  $$\frac{m_1}{m_2} = \frac{|G_1|}{|G_2|}$$

\end{proof}

We now turn to the proof of the result on irreducible
representations. We need some preliminary definitions.

\begin{definer}[Projective general linear group]
  Suppose $K$ is a field and $d$ is a positive integer. The {\em
    projective general linear group} of degree $d$ over $K$, denoted
  $PGL_d(K)$, is defined as the quotient group of the general linear
  group $GL_d(K)$ by the subgroup of scalar matrices in $GL_d(K)$. The
  subgroup of scalar matrices in $GL_d(K)$ is precisely the center of
  $GL_d(K)$. Hence, $PGL_d(K)$ is isomorphic to the inner automorphism
  group of $GL_d(K)$.
\end{definer}

\begin{definer}[Projective representation]
  Suppose $G$ is a group and $K$ is a field. A {\em projective
    representation} of $G$ over $K$ of degree $d$ is a homomorphism
  from $G$ to the projective general linear group $PGL_d(K)$ for some
  positive integer $d$. The value $d$ here is termed the {\em degree
    of the projective representation}.

  A projective representation $\rho:G \to PGL_d(K)$ is said to have a
  linear lift $\theta:G \to GL_d(K)$ if $\pi \circ \theta = \rho$,
  where $\pi: GL_d(K) \to PGL_d(K)$ is the natural quotient map. The
  term ``linear lift'' here refers to the fact that $\theta$ is a {\em
    linear} representation that serves as a ``lift'' of $\rho$.
\end{definer}

A projective representation may or may not admit a linear lift. The
next lemma describes the nature of the set of linear lifts assuming
that a linear lift exists.

\begin{lemma}\label{stabilizer-kernel-description}
  Suppose $G$ is a finite group and $\rho:G \to PGL_d(\mathbb{C})$ is
  a projective representation. Suppose $\theta:G \to GL_d(\mathbb{C})$
  a linear representation of $G$ that is a lift of $\rho$. In other
  words, if $\pi: GL_d(\mathbb{C}) \to PGL_d(\mathbb{C})$ is the
  natural quotient map, then we want that $\rho = \pi \circ
  \theta$. We know that the set of one-dimensional representations of
  $G$ (identified as the Pontryagin dual of $G/G'$) acts naturally on
  the set of irreducible representations of $G$. The claim is that the
  stabilizer of $\theta$ is precisely the set of one-dimensional
  representations of $G$ whose kernel contains the subgroup generated
  by $G'$ and all the elements $g$ of $G$ on which the trace of
  $\theta(g)$ takes a nonzero value.
\end{lemma}

\begin{proof}
  {\em One direction (one-dimensional representation whose kernel
    contains $G'$ and the elements with nonzero trace values for
    $\theta$ must be in the stabilizer of $\theta$)}: If a
  one-dimensional representation $\beta$ has a kernel containing all
  the points where $\theta$ has a nonzero-valued character, then that
  means that for any $g \in G$, either $\theta(g)$ has trace zero or
  $\beta(g)$ is the identity. Thus, in all cases, we have that
  $\beta(g)\theta(g)$ and $\theta(g)$ have the same trace. Thus,
  $\beta\theta$ and $\theta$ have the same character, hence, by basic
  character theory, are equivalent as representations.

  {\em Reverse direction (one-dimensional representation that
    stabilizes $\theta$ must have kernel containing $G'$ and the
    elements with nonzero trace values)}: Let $\beta$ be a
  one-dimensional representation of $G$ that stabilizes $\theta$. Note
  that the kernel of any one-dimensional representation already
  contains $G'$, so $G'$ is contained in the kernel of $\beta$. Thus,
  we only need to show it contains all the elements at which the trace
  of $\theta$ is nonzero. Suppose $g \in G$ is an element at which
  $\theta(g)$ has nonzero trace, and $\beta$ is a one-dimensional
  representation in the stabilizer of $\theta$. Then $\beta \theta$
  and $\theta$ are equivalent representations, hence they have the
  same character. Thus, $\beta(g)\theta(g)$ and $\theta(g)$ have the
  same trace. By assumption, $\theta(g)$ has nonzero trace, so this
  forces the complex number $\beta(g)$ to equal $1$, so $g$ is in the
  kernel of $\beta$, as desired.
\end{proof}

We can now turn to the proof of Theorem
\ref{isoclinic-same-proportions-irrep-degrees}, the main theorem about
irreducible representations.

\begin{quote}
  Suppose $G_1$ and $G_2$ are isoclinic finite groups. Suppose $d$ is
  a positive integer. Let $m_1$ denote the number of equivalence
  classes of irreducible representations of $G_1$ over $\mathbb{C}$
  that have degree $d$. Let $m_2$ denote the number of equivalence
  classes of irreducible representations of $G_2$ over $\mathbb{C}$
  that have degree $d$. Then, $m_1$ is nonzero if and only if $m_2$ is
  nonzero, and if so, $m_1/m_2 = |G_1|/|G_2|$.

  In particular, if $G_1$ and $G_2$ additionally have the same order,
  then they have precisely the same multiset of degrees of irreducible
  representations.
\end{quote}

\begin{proof}
  Let $W$ be the group identified with $\operatorname{Inn}(G_1) \cong
  \operatorname{Inn}(G_2)$, and $T$ be the group identified with $G_1'
  \cong G_2'$. Denote by $\alpha_1:G_1 \to W$ and $\alpha_2: G_2 \to
  W$ the respective quotient maps. $\omega: W \times W \to T$ the group
  commutator map, which is the same for both groups.

  We have short exact sequences:

  $$1 \to Z(G_1) \to G_1 \to W \to 1$$

  and

  $$1 \to Z(G_2) \to G_2 \to W \to 1$$

  We will show the following:

  \begin{enumerate}
  \item For any irreducible projective representation $\rho: W \to
    PGL_d(\mathbb{C})$, there exists a linear representation of $G_1$
    that descends to $\rho$ if and only if there exists a linear
    representation of $G_2$ that descends to $\rho$.
  \item Further, if so, the ratio of the number of linear
    representations of $G_1$ that descend to $\rho$ equals the number
    of linear representations of $G_2$ that descend to $\rho$ is
    $|G_1|/|G_2|$.
  \end{enumerate}

  Note that once we have (1) and (2), the result will follow: first,
  simply list all the projective representations of $W$ of degree $d$
  that lift to linear representations in the groups $G_1$ and/or
  $G_2$. For each, the number of lifts in the two groups is in the
  proportion $|G_1|:|G_2|$, so the overall proportion is also
  $|G_1|:|G_2|$.

  Proof of (1): This follows from Isaacs, Theorem 11.13, and the
  observation that the condition Isaacs specifies for the
  representation to lift is satisfied for $G_1$ if and only if it is
  satisfied for $G_2$. %% {\em TONOTDO: Add a little more detail here, since
  %% the notation used in Isaacs is a little different, so it may be hard
  %% for readers to follow this step; also, Suzuki's book might be a
  %% better reference since Isaacs never explicitly discusses isoclinism}.

  Proof of (2): If a projective representation lifts to $G_1$, then
  the set of lifts has a transitive action on it of the set of
  one-dimensional linear representations of $G_1$, which is the
  Pontryagin dual of $G_1/G_1'$. By the fundamental theorem of group
  actions, the size of the set of lifts equals the index of the
  stabilizer in this Pontryagin dual of any lift. So the question is:
  what is the necessary and sufficient condition for a one-dimensional
  representation $\chi$ of $G_1/G_1'$ to fix a linear lift of $\rho$
  to $G_1$?

  The notion of whether the trace is zero is a well-defined notion for
  $\rho$, even though the outputs are in $PGL_d(\mathbb{C})$ rather
  than being matrices themselves. Let $\mathcal{N}(\rho)$ be the
  subgroup of $W$ generated by $W'$ and all those elements of $W$ for
  which the trace of the image under $\rho$ is nonzero. By the
  preceding lemma (Lemma \ref{stabilizer-kernel-description}), the
  stabilizer of any lift of $\rho$ is precisely the set of
  one-dimensional representations of $G_1$ whose kernel contains
  $\alpha_1^{-1}(\mathcal{N}(\rho))$. Another way of putting it is
  that it is the Pontryagin dual of
  $G_1/\alpha_1^{-1}(\mathcal{N}(\rho))$, viewed as a subgroup of the
  Pontryagin dual of $G_1/G_1'$.


  The number of linear lifts is therefore:

  $$\frac{|G_1/G_1'|}{|G_1/\alpha_1^{-1}(\mathcal{N}(\rho))|}$$

  By the third isomorphism theorem of basic group theory, this is the
  same as:

  $$|\alpha_1^{-1}(\mathcal{N}(\rho))/G_1'|$$

  This simplifies to:

  $$\frac{|\mathcal{N}(\rho)||Z(G_1)|}{|G_1'|}$$

  Similarly, the number of lifts of $\rho$ to $G_2$ is:

  $$\frac{|\mathcal{N}(\rho)||Z(G_2)|}{|G_2'|}$$

  Note the crucial fact that $\mathcal{N}(\rho)$ is the same in both
  cases.

  Taking the quotient, we get $|Z(G_1)|/|Z(G_2)|$, which is the same
  as $|G_1/G_2|$ because the groups have isomorphic inner automorphism
  groups. This completes the proof of (2), and hence of the original
  statement.
\end{proof}

%\end{document}

%% Peter May back and forth:

%% Forth:

%% (1) The homology groups *with coefficients in the integers* for a Lie
%% ring L whose additive group is T-local are all T-local abelian
%% groups. I'm interested only in the nilpotent case and the second
%% homology group.

%% (2) Even better (though not necessary) would be the statement that for
%% any Lie ring L, if we denote by L_T the tensor product obtained by
%% extending the ring of scalars to Z_T, the following is true:

%% H_*(L;integers) -> H_*(L_T;integers)

%% Back:

%% I don't know the Lie ring literature, but I'm perfectly comfortable working
%% with Lie algebras over any commutative ring, and in your case I would
%% think of your L as a Lie algebra over $\bZ_{(p)}$, but we have a silly
%% choice: take the enveloping algebra as a Lie ring (Lie algebra over $\bZ$)
%% or a Lie algebra over $\bZ_{(p)}$.  Defining cohomology in the obvious
%% way using Ext over the universal enveloping algebra, these should differ
%% only in degree $0$, giving $\bZ$ with one choice and $\bZ_{(p)}$ with
%% the other there.  Modulo this distinction, the answer to (1) should be yes,
%% in positive degrees, with either choice.   The answer to (2) should be yes
%% in all degrees with either choice.   There should be nothing to prove.



%% \subsection{Summary of attributes preserved under Baer correspondence}

%% Let $G$ be a Baer Lie group and $L = \log G$ is its Baer Lie
%% ring. The following are true:

%% \begin{enumerate}
%% \item Under the Baer correspondence, endomorphisms of $G$ correspond
%%   to endomorphisms of $L$, as described in Section
%%   \ref{sec:baer-correspondence-isocat-consequences}.
%% \item Under the Baer correspondence, automorphisms of $G$ correspond
%%   to automorphisms of $L$, as described in Section
%%   \ref{sec:baer-correspondence-isocat-consequences}.
%% \item Under the Baer correspondence, the center of $G$ corresponds to
%%   the center of $L$.
%% \item Under the Baer correspondence, the derived subgroup of $G$
%%   corresponds to the derived subring of $L$.
%% \item As described in Section
%%   \ref{sec:baer-correspondence-sub-quot-dp}, we obtain the following
%%   correspondences:

%%   \begin{center}
%%     Baer Lie subgroups of $G$ $\leftrightarrown$ Baer Lie subrings of
%%     $L$
%%   \end{center}

%%   \begin{center}
%%     Normal Baer Lie subgroups of $G$ $\leftrightarrow$ Baer Lie ideals in $L$
%%   \end{center}

%%   \begin{center}
%%     Baer Lie quotient groups of $G$ $\leftrightarrow$ Baer Lie
%%     quotient rings of $L$
%%   \end{center}

%% \item The Baer correspondence gives a correspondence:

%%   \begin{center}
%%     $2$-powered characteristic subgroups of $G$ $\leftrightarrow$ $2$-powered characteristic subrings in $L$
%%   \end{center}

%%   Note that this also implies that any $2$-powered characteristic
%%   subring of $L$ is an ideal.

%% \item Conjecture \ref{conj:charpowering} (respectively, Conjecture
%%   \ref{conj:charpowering-lie}) stated that in a $\pi$-powered
%%   nilpotent group (respectively, $\pi$-powered nilpotent Lie ring),
%%   any characteristic subgroup (respectively, characteristic Lie
%%   subring) must be $\pi$-powered. We can consider restricted versions
%%   of Conjectures \ref{conj:charpowering} and
%%   \ref{conj:charpowering-lie} to the case of Baer Lie groups and Baer
%%   Lie rings respectively. The restricted versions of the conjectures
%%   are equivalent. Further, if the equivalent conjectures are true, the
%%   correspondence of the preceding point becomes a correspondence:

%%   \begin{center}
%%     Characteristic subgroups of $G$ $\leftrightarrow$ Characteristic
%%     subrings of $L$
%%   \end{center}
%% \end{enumerate}

%% \subsection{Some results about generating sets of the derived subgroup that we will use}\label{sec:derived-subgroup-results-to-use}

%% The following two results are easy to prove, but useful to keep in
%% mind.

%% \begin{itemize}
%% \item For a group $G$ with a generating set $S$, the derived subgroup
%%   of $G$ has the following generating set: all elements that can be
%%   expressed as commutator words (of length at least two) using the
%%   elements of $S$. To see this, note that $G$ is the normal closure of
%%   the set of commutators between elements of $S$, hence it is
%%   generated by the set of all conjugates of commutators between
%%   elements of $S$. It suffices to show that each such conjugate is
%%   expressible as a product of iterated commutators involving elements
%%   of $S$, and this is an easy exercise.
%% \item For a Lie ring $L$ with a generating set $S$, the derived
%%   subring of $L$ has the following generating set: all elements that
%%   can be expressed as Lie products (of length at least two) using the
%%   elements of $S$.
%% \end{itemize}

%% %\newpage


%% %% \subsection{Replacement theorems}

%% %% There is a considerable literature related to questions such as:

%% %% \begin{quote}
%% %%   Given a prime $p$ and positive integers $k \le n$, is it true
%% %%   that for every group $G$ of order $p^n$ having an abelian subgroup
%% %%   of order $p^k$, $G$ must have an abelian subring of order $p^k$?
%% %% \end{quote}

%% %% A major paper addressing questions of this type was a 1975 paper
%% %% \cite{JK75} by Jonah and Konvisser. The paper proved that for an odd
%% %% prime $p$ and for $0 \le k \le 5$, the answer to the question above
%% %% was always {\em yes}. In fact, their paper demonstrated a
%% %% substantially stronger result: they showed that if $G$ has an abelian
%% %% subgroup of order $p^k$ for some fixed $k$ satisfying $0 \le k \le 5$,
%% %% then the total number of abelian subgroup of $G$ of order $p^k$ is
%% %% congruent to $1$ mod $p$. The counting methods proposed by Jonah and
%% %% Konvisser were unsuited to settling the question above. In fact,
%% %% another paper \cite{JK75.2} by the same authors published in the same
%% %% journal issue constructed a generic example (valid for all odd primes
%% %% $p$) of a group of order $p^9$ that has exactly two abelian subgroups
%% %% of order $p^6$, both of which are elementary abelian and normal.

%% %% The paper \cite{AG98} by Alperin and Glauberman proves more powerful
%% %% results in the same direction. The paper uses the Lazard
%% %% correspondence for its proof, and also uses group-theoretic reasoning
%% %% to generalize the result somewhat to situations where the Lazard
%% %% correspondence does not directly apply. Explicitly, Theorem C of the
%% %% paper reads:

%% %% \begin{theorem}[Theorem C of \cite{AG98}]
%% %%   If $A$ is an abelian subalgebra of the Lie algebra $L$ over $F$ and
%% %%   $L$ is nilpotent of class at most $p$, then there is an abelian
%% %%   ideal $N$ of $L$, in the ideal closure of $A$ in $L$, of the same
%% %%   dimension as $A$.
%% %% \end{theorem}

