\documentclass[10pt]{amsart}
\usepackage{fullpage,hyperref,vipul,graphicx}
\title{Directional derivatives and gradient vectors}
\author{Math 195, Section 59 (Vipul Naik)}

\begin{document}
\maketitle

{\bf Corresponding material in the book}: Section 14.6.

{\bf What students should definitely get}: Definition of directional
derivative and gradient vector, gradient vector as direction with
maximum magnitude of directional derivative, directional derivative as
dot product of gradient vector and unit vector in the
direction. Formulas for tangent plane and normal line at a point to a
surface with a relational description.

\section*{Executive summary}

Words ...

\begin{enumerate}
\item The directional derivative of a scalar function
  $f$ of two variables along a {\em unit} vector $\mathbf{u} =
  a\mathbf{i} + b\mathbf{j}$ at a point $(x_0,y_0)$ is defined as the
  following limit of difference quotient, if the limit exists:

  $$\lim_{h \to 0} \frac{f(x_0 + ah, y_0 + bh) - f(x_0,y_0)}{h}$$

\item The directional derivative of a differentiable scalar function
  $f$ of two variables along a {\em unit} vector $\mathbf{u} =
  a\mathbf{i} + b\mathbf{j}$ at a point $(x_0,y_0)$ is
  $D_{\mathbf{u}}(f) = af_x(x_0,y_0) + bf_y(x_0,y_0)$.
\item The gradient vector for a differentiable scalar function $f$ of
  two variables at a point $(x_0,y_0)$ is $\nabla f(x_0,y_0) =
  f_x(x_0,y_0)\mathbf{i} + f_y(x_0,y_0)\mathbf{j}$.
\item The directional derivative of $f$ is the dot product of the
  gradient vector of $\nabla f$ and the unit vector $\mathbf{u}$. 
\item Suppose $\nabla f$ is nonzero. Then, if $\mathbf{u}$ makes an
  angle $\theta$ with $\nabla f$, then $D_{\mathbf{u}}(f)$ is
  $|\nabla_f|\cos \theta$. The directional derivative is maximum in
  the direction of the gradient vector, zero in directions orthogonal
  to the gradient vector, and minimum in the direction opposite to the
  gradient vector.
\item The level curves are orthogonal to the gradient vector.
\item Similar formulas for gradient vector and directional derivative
  work in three dimensions.
\item The level surfaces are orthogonal to the gradient vector for a
  function of three variables.
\item For a surface given by $F(x,y,z) = 0$, if $(x_0,y_0,z_0)$ is a
  point on the surface, and $F_x(x_0,y_0,z_0)$, $F_y(x_0,y_0,z_0)$,
  and $F_z(x_0,y_0,z_0)$ all exist and are nonzero, then the normal
  line is given by:

  $$\frac{x - x_0}{F_x(x_0,y_0,z_0)} = \frac{y - y_0}{F_y(x_0,y_0,z_0)} = \frac{z - z_0}{F_z(x_0,y_0,z_0)}$$

  The tangent plane is given by:

  $$F_x(x_0,y_0,z_0)(x - x_0) + F_y(x_0,y_0,z_0)(y - y_0) + F_z(x_0,y_0,z_0)(z - z_0) = 0$$
\end{enumerate}
\section{Directional derivatives: definition and key facts}

\subsection{Partial derivatives as derivatives along coordinate directions}

The partial derivative $f_x(x,y)$ is defined as the derivative of $f$
with respect to $x$, keeping $y$ constant. Thinking of the domain of
$f$ geometrically, this is the same as the derivative of $f$ along a
unit vector in the $x$-direction (which we archaically denote by
$\mathbf{i}$). Similarly, the partial derivative $f_y(x,y)$ is defined
as the derivative of $f$ with respect to $y$, keeping $x$
constant. This can be viewed as the derivative of $f$ with respect to
a unit vector in the $y$-direction (which we archaically denote by
$\mathbf{j}$).

We may be interested in derivatives with respect to mixed directions,
i.e., we may be interested in the question: if we move along the
direction of the vector $\mathbf{i} + \mathbf{j}$, how does the
function value change?

The correct notion is that of {\em directional derivative}. After
defining this notion, we consider its implication both for functions
that have physical significance and for the more abstract functions in
economics (such as production and demand functions).

\subsection{Definition of directional derivative}

For a function $f$, the directional derivative of $f$ at $(x_0,y_0)$
in the direction of a unit vector $\mathbf{u} = \langle a,b \rangle$
(i.e., $a^2 + b^2 = 1$) is denoted $D_{\mathbf{u}}f(x_0,y_0)$ and is
defined as:

$$D_{\mathbf{u}}f(x_0,y_0) = \lim_{h \to 0} \frac{f(x_0 + ha,y_0 + hb) - f(x_0,y_0)}{h}$$

if the limit exists.

The special case of $\mathbf{u} = \mathbf{i} = \langle 1,0 \rangle$
gives the partial derivative $f_x(x,y)$ and the case $\mathbf{u} =
\mathbf{j} = \langle 0,1 \rangle$ gives the partial derivative
$f_y(x,y)$.

\subsection{Existence and computation of directional derivative}

It turns out that there is a notion of ``differentiable'' for a
function of two variables (which are are avoiding discussion of) and
if a function is differentiable at a point in that sense, then it has
well-defined directional derivatives along all unit vectors.

Further, then the directional derivative in the direction of
$\mathbf{u} = \langle a,b \rangle$ is given by:

$$D_{\mathbf{u}}f(x_0,y_0) = af_x(x_0,y_0) + bf_y(x_0,y_0)$$

A {\em sufficient} condition for a function to be differentiable is
that both the partial derivative $f_x$ and $f_y$ exist and are
continuous {\em around} the point. 

\subsection{Geometric sense of directional derivatives}

Directional derivatives make direct geometric sense in cases where the
functions actually have physical significance. For instance, consider
a function whose input domain is a flat surface and whose output value
at any point on the surface is the temperature at that point. The {\em
directional} derivative at a point with respect to a direction at that
point can be thought of as the rate at which the temperature changes
if you move along that direction (physically) starting at that point.

More generally, suppose you move along a curve in the surface that's
the domain of the function. You want to find the rate at which the
temperature is changing along the curve that you are moving
along. This rate of temperature change is the product of the
(directional derivative along the unit vector tangent direction to the
curve at the point) times the (length of the tangent vector, i.e., the
speed of motion).

\subsection{Sense of directional derivatives in non-physical contexts}

Consider the case of a production function $f(L,K)$ with inputs $L$
(the expenditure on labor) and $K$ (the expenditure on capital). We
already made sense of the partial derivatives $f_L(L,K)$ and
$f_K(L,K)$. The partial derivative $f_L(L,K)$ is the marginal change
in output for a marginal change in the labor input (i.e., it is the
marginal product of labor). Similarly, the partial derivative
$f_K(L,K)$ is the marginal chane in output for a marginal change in
the capital input (i.e., it is the marginal product of capital).

Let's think of what it means to take the directional derivative along
the vector $\langle a, b \rangle$ with $a^2 + b^2 = 1$. What this
basically means is the following: we want to measure the marginal
change in output if the inputs $L$ and $K$ are changed marginally in
the ratio $a:b$. In other words, it measures the impact on output of a
particular {\em combined trajectory of change} in the values of $L$
and $K$.

\section{The gradient vector}

\subsection{The direction of change, the direction of no change}

The {\em gradient vector} for a differentiable function $f$ of two
variables is defined as follows:

$$\nabla f(x,y) = \langle f_x(x,y), f_y(x,y) \rangle = \frac{\partial f}{\partial x}\mathbf{i} + \frac{\partial f}{\partial y}\mathbf{j}$$

Note that the gradient vector, thus viewed, is a {\em vector-valued
function of two variables}, i.e., it has type $2 \to 2$. The gradient
vector at a particular point is, however, an actual vector, i.e., an
actual tuple of two real numbers.

If the gradient vector is nonzero, the {\em unit vector} in this
direction can be computed by dividing this vector by its length.

Here are some key observations that hold if the gradient vector is
nonzero:

\begin{itemize}
\item Of all the unit vector directions, the directional derivative is
  maximum (and positive) along the unit vector in the same direction
  as the gradient vector, and is minimum (and negative) along the unit
  vector in the opposite direction to the gradient vector.
\item The directional derivative is zero along the directions {\em
  perpendicular} to the gradient vector.
\end{itemize}

Intuitively the gradient vector is telling us the direction along
which the change/action is happening, and also telling us that there's
no action happening orthogonal to it.

\subsection{Special case of function depending on only one variable}

If the function $f$ depends only on the variable $x$ and has no
dependence on the variable $y$, then the gradient vector, where
nonzero, will always point parallel to the $x$-axis (in a positive or
negative direction, depending on whether the function is increasing or
decreasing).

\subsection{Writing directional derivative in terms of gradient vector}

The directional derivative along a unit vector $\mathbf{u}$ can be
defined as the dot product $(\nabla f) \cdot \mathbf{u}$. Since
$\mathbf{u}$ is a unit vector, this can be interpreted as the scalar
projection of $\nabla f$ along $\mathbf{u}$.

Intuitively, what this means is that the extent to which the function
is changing in a particular direction depends on the component of the
gradient vector that falls in that direction.

Another way of thinking of the directional derivative
$D_{\mathbf{u}}(f)$ for a unit vector $\mathbf{u}$ is as $|\nabla
f|\cos \theta$ where $\theta$ is the angle between $\nabla f$ and
$\mathbf{u}$. Note that this is equal to $|\nabla f|$ when
$\mathbf{u}$ is in the direction of $\nabla f$, and it is $-|\nabla
f|$ when $\mathbf{u}$ is opposite to $\nabla f$. It is $0$ when
$\mathbf{u}$ is orthogonal to $\nabla f$.

Note that if the gradient vector is zero, then there is no direction
to it, and the directional derivative along {\em every} direction is
zero.
\subsection{Relationship of gradient vector and level curves}

Recall that the gradient vector represents the direction along which
all the change in the function value is happening. It should thus come
as no surprise that at any point, the level curve through that point
is orthogonal to the gradient vector at that point (note that the
statement is trivially true if the gradient vector is zero, but gives
no geometric information in that case). Further, if we consider the
picture of level curves along with the function values for each curve,
then the gradient vector points in the direction of increasing
function values.

\section{Case of functions of three or more variables}

\subsection{Description}
The same expressions for gradient vector and directional derivative
apply to functions of three or more variables.

Basically:

\begin{itemize}
\item The directional derivative along a unit vector can be defined as
  a limit of a difference quotient. In three variables: For a function
  $f(x,y,z)$ at a point $(x_0,y_0,z_0)$ with unit vector $\mathbf{u} =
  \langle a,b,c \rangle$, this becomes:

  $$\lim_{h \to 0} \frac{f(x_0 + ah,y_0 + bh, z_0 + ch)}{h}$$

\item For a differentiable function, the gradient vector is defined as
  the vector obtained by adding, for each coordinate direction, the
  partial derivative in that direction times the unit vector in that
  direction. In three variables: For a function $f(x,y,z)$, this
  becomes:

  $$\nabla f(x,y,z) = f_x(x,y,z)\mathbf{i} + f_y(x,y,z)\mathbf{j} + f_z(x,y,z)\mathbf{k}$$

  This is a vector-valued function, i.e., it has type $3 \to 3$ (and
  more generally $n \to n$). At a particular point in the domain, it
  gives an actual vector, i.e., a tuple of $3$ (respectively $n$) real
  numbers.
\item For a differentiable function, and in particular for a function
  where all the first partials exist and are continuous around the
  point, the directional derivative along a unit vector is the dot
  product of the gradient vector and that unit vector.
\item The gradient vector at a point (if nonzero) is orthogonal to the
  level surface for the function at the point, or equivalently, it is
  orthogonal to the tangent plane for the level surface.
\end{itemize}

\subsection{Application to finding normal direction and tangent plane}

The fact that the gradient vector points in the normal direction and
is hence orthogonal to the tangent plane provides a strategy to
compute the tangent plane for any point on a surface in $\R^3$ given
by a top-down (relational) description of the form $F(x,y,z) =
0$. Namely, to find the normal vector at a point $(x_0,y_0,z_0)$, we
compute $\nabla F (x_0,y_0,z_0)$. If this is nonzero, it is a normal
vector. We can convert it to a unit vector if desired. Next, we use
the technique for finding the scalar equation of a plane to obtain the
scalar equation of the tangent plane.

Instead of working things out in each case using vectors, we can also
directly determine the scalar version and apply these directly. For a
relational description $F(x,y,z) = 0$ and a point $(x_0,y_0,z_0)$, the
symmetric equations of the normal line are:

$$\frac{x - x_0}{F_x(x_0,y_0,z_0)} = \frac{y - y_0}{F_y(x_0,y_0,z_0)} = \frac{z - z_0}{F_z(x_0,y_0,z_0)}$$

The equation of the tangent plane is:

$$F_x(x_0,y_0,z_0)(x - x_0) + F_y(x_0,y_0,z_0)(y - y_0) + F_z(x_0,y_0,z_0)(z - z_0) = 0$$

{\em Note that this method does not work if $\nabla F$ is $0$}. In
other words, it does not work if all the three first partials
$F_x(x_0,y_0,z_0)$, $F_y(x_0,y_0,z_0)$, and $F_z(x_0,y_0,z_0)$ equal
$0$. In this case, it may happen either that there is no tangent
plane, or it may happen that the tangent plane exists but cannot be
found through this procedure.

Recall that we had earlier determined the equation to the tangent
plane for $z = f(x,y)$, which is a special case of the above with
$F(x,y,z) = f(x,y) - z$. It can be verified (See the book) that the
earlier formula is consistent with the formula obtained above.

\subsection*{Note: application to temperature}

We consider the example of temperature.

Consider the temperature function defined on the surface of the earth
that sends a point to the surface of the earth to the surface
temperature at that point.

The level curves for this are the {\em isothermal lines} and represent
curves of constant temperature. The gradient vector at any point is a
vector tangential to the sphere and represents the direction in which
temperature is changing fastest. The positive direction along the
gradient vector is the direction of fastest temperature increase. The
negative direction is the direction of fastest temperature decrease.

The directional derivative along a direction tangential to the surface
of the earth describes the rate at which temperature changes if we
move along that direction.

The same example of temperature can be adapted to a three-dimensional
setting instead of the surface of the earth. For instance, in a
three-dimensional container that is not in thermal equilibrium (so
different parts have different temperatures) we can consider the
temperature as a function of the location in space. Then, the level
{\em surfaces} are the isothermal surfaces, and the gradient vector at
a point is orthogonal to the level surface and describes the direction
along which temperature is changing fastest. The directional
derivative along a particular direction is the rate at which
temperature changes if we move along that direction.
\end{document}
