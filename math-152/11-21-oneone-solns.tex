\documentclass[10pt]{amsart}

%Packages in use
\usepackage{fullpage, hyperref, vipul, enumerate}

%Title details
\title{Class quiz solutions: November 21: One-one functions}
\author{Math 152, Section 55 (Vipul Naik)}
%List of new commands

\begin{document}
\maketitle

\section{Performance review}

$12$ people attempted this quiz. The score distribution was as follows:

\begin{itemize}
\item Score of $1$: $3$ people.
\item Score of $2$: $6$ people.
\item Score of $3$: $3$ people.
\end{itemize}

The mean score was $2$.

Here are the problem wise solutions and scores:

\begin{enumerate}
\item Option (B): $11$ people
\item Option (C): $2$ people
\item Option (E): $9$ people
\item Option (C): $0$ people. {\em Whoops!}
\item Option (E): $2$ people
\end{enumerate}

\section{Solutions}

\begin{enumerate}
\item For one of these function types for a continuous function from
  $\R$ to $\R$, it is {\em possible} to also be a one-to-one
  function. What is that function type?

  \begin{enumerate}[(A)]
  \item Function whose graph has mirror symmetry about a vertical line.
  \item Function whose graph has half turn symmetry about a point on it.
  \item Periodic function.
  \item Function having a point of local minimum.
  \item Function having a point of local maximum.
  \end{enumerate}

  {\em Answer}: Option (B)

  {\em Explanation}: Think $f(x) = x$, $f(x) = x^3$, or $f(x) = x -
  \sin x$.

  For all the others, the violation of one-to-one is clear: periodic
  functions repeat completely after an interval, functions whose graph
  has mirror symmetry have the same value at equal distances from the
  axis of mirror symmetry, and the existence of a local extremum
  implies that values very close to that are attained both on the
  immediate left and the immediate right of the extremum.

  {\em Performance review}: $11$ out of $12$ got this correct. $1$
  left the question blank.

  {\em Historical note (last year)}: Everybody got this correct!

\item (**) Suppose $f$, $g$, and $h$ are continuous one-to-one
  functions whose domain and range are both $\R$. {\bf What can we
    say} about the functions $f + g$, $f + h$, and $g + h$?

  \begin{enumerate}[(A)]
  \item They are all continuous one-to-one functions with domain $\R$
    and range $\R$.
  \item At least two of them are continuous one-to-one functions with
    domain $\R$ and range $\R$ -- however, we cannot say more.
  \item At least one of them is a continuous one-to-one function with
    domain $\R$ and range $\R$ -- however, we cannot say more.
  \item Either all three sums are continuous one-to-one functions
    whose domain and range are both $\R$, or none is.
  \item It is possible that none of the sums is a continuous
    one-to-one function whose domain and range are both $\R$; it is
    also possible that one, two, or all the sums are continuous
    one-to-one functions whose domain and range are both $\R$.
  \end{enumerate}

  {\em Answer}: Option (C)

  {\em Explanation}: Since $f$, $g$, and $h$ are all continuous
  one-to-one functions with domain and range $\R$, each one of them is
  either increasing or decreasing. We consider various cases:

  \begin{itemize}
  \item If all three functions are increasing, so are all the pairwise
    sums, and hence, all the sums $f + g$, $f + h$, and $g + h$ are
    increasing. Further, the domain and range of all three pairwise
    sums is $\R$.
  \item If all three functions are decreasing, so are all the pairwise
    sums, and hence, all the sums $f + g$, $f + h$, and $g + h$ are
    decreasing. Further, the domain and range of all three pairwise
    sums is $\R$
  \item If two of the functions are increasing and the third function
    is decreasing, then we know for certain that the sum of the two
    increasing functions is increasing and hence one-to-one. But the
    sum of either of the increasing functions with the decreasing
    function may be increasing, decreasing, or neither. For instance,
    if $f(x) = g(x) = x$ and $h(x) = -x$, then $f + h$ and $g + h$ are
    both the zero function, which is neither increasing nor
    decreasing, and hence not one-to-one.
  \item If two of the functions are decreasing and the third function
    is increasing, then we know for certain that the sum of the two
    dcereasing functions is decreasing and hence one-to-one. We cannot
    say anything for sure about the other two sums, for the same
    reasons as in the previous case.
  \end{itemize}

  It's clear from all these that (C) is the right option.

  {\em Performance review}: $2$ out of $12$ got this. $6$ chose (E),
  $2$ chose (A), $1$ each chose (B) and (D).

  {\em Historical note (last year)}: $2$ out of $15$ people got this
  correct. $8$ people chose (A), $1$ person chose (B), and $4$ people
  chose (E).

  {\em Action point}: Please make sure you understand this solution
  really well! This kind of question should not trip you in the
  future.
\item (**) Suppose $f$ is a one-to-one function with domain a closed
  interval $[a,b]$ and range a closed interval $[c,d]$. Suppose $t$ is
  a point in $(a,b)$ such that $f$ has left hand derivative $l$ and
  right-hand derivative $r$ at $t$, with both $l$ and $r$
  nonzero. What is the left hand derivative and right hand derivative
  to $f^{-1}$ at $f(t)$?

  \begin{enumerate}[(A)]
  \item The left hand derivative is $1/l$ and the right hand
    derivative is $1/r$.
  \item The left hand derivative is $-1/l$ and the right hand
    derivative is $-1/r$.
  \item The left hand derivative is $1/r$ and the right hand
    derivative is $1/l$.
  \item The left hand derivative is $-1/r$ and the right hand
    derivative is $-1/l$.
  \item The left hand derivative is $1/l$ and the right hand
    derivative is $1/r$ if $l > 0$, otherwise the left hand derivative
    is $1/r$ and the right hand derivative is $1/l$.
  \end{enumerate}
  
  {\em Answer}: Option (E)

  {\em Explanation}: Although it isn't necessary to note this, a
  one-to-one function that satisfies the intermediate value property
  is continuous, so even though $f$ is not explicitly given to be
  continuous, it is in fact continuous on its domain.

  If $l > 0$, then, since we are dealing with a one-to-one function,
  the function is increasing throughout, and so $r \ge 0$ as
  well. Since we know $r \ne 0$, we conclude that $r > 0$
  strictly. The upshot is that as $x \to t^-$, $f(x) \to f(t)^-$ and
  as $x \to t^+$, $f(x) \to f(t)^+$. Thus, when we pass to the inverse
  function, the roles of left and right remain the same.

  On the other hand, if $l < 0$, then as $x \to t^-$, $f(x) \to
  f(t)^+$, and hence the roles of left and right get interchanged.

  {\em Performance review}: $9$ out of $12$ got this. $2$ chose (C)
  and $1$ chose (A).

  {\em Historical note (last year)}: $6$ out of $15$ people got this
  correct. $6$ people chose (A) and $3$ people chose (C).

  {\em Action point}: One-sided derivatives and increasing/decreasing
  functions are a potent mix. We've tried and failed to understand
  this mix many times in the past. But this might well be the time it
  finally clicks! Here's a repetition: when we apply an increasing
  function, {\em left remains left} and {\em right remains right}. But
  when we apply a decreasing function, {\em left becomes right} and
  {\em right becomes left}. Keep chanting this again and again until
  you understand, appreciate, and {\em believe} it.


\item (**) Which of these functions is one-to-one?

  \begin{enumerate}[(A)]
  \item $f_1(x) := \lbrace \begin{array}{rl} x, & x \text{ rational} \\ x^2, & x \text{ irrational}\\\end{array}$ 
  \item $f_2(x) := \lbrace \begin{array}{rl} x, & x \text{ rational} \\ x^3, & x \text{ irrational}\\\end{array}$
  \item $f_3(x) := \lbrace\begin{array}{rl} x, & x \text{ rational} \\ 1/(x - 1), & x \text{ irrational}\\\end{array}$
  \item All of the above
  \item None of the above
  \end{enumerate}

  {\em Answer}: Option (C)

  {\em Explanation}: Option (A) is easy to rule out: $\sqrt{2}$ and
  $-\sqrt{2}$ map to the same thing. Option (B) is a little harder to
  rule out, because the function is one-to-one within each piece,
  i.e., no two rationals map to the same thing and no two irrationals
  map to the same thing. However, a rational and an irrational can map
  to the same thing. For instance, $2$ and $2^{1/3}$ both map to $2$.

  For option (C), note that not only is the map one-to-one in each
  piece, but also, the image of the rationals stays inside the
  rationals and the image of the irrationals stays inside the
  irrationals. In particular, this means that a rational number and an
  irrational number cannot map to the same thing, so the function is
  globally one-to-one.

  {\em Performance review}: Nobody got this correct! $10$ chose (C),
  $2$ chose (A).

  {\em Historical note (last year)}: $2$ out of $15$ people got this
  correct. $7$ people chose (E), $4$ people chose (B), and $2$ people
  chose (D). It is possible that some of those who chose (E) lost hope
  after looking at (A) and (B) and concluded that checking (C) is
  pointless too.

  {\em Action point}: This one should not trip you in the future
  either!

\item (**) Consider the following function $f:[0,1] \to [0,1]$ given
  by
  $f(x) := \lbrace\begin{array}{rl} \sin(\pi x/2), & 0 \le x \le 1/2 \\
    \sqrt{x}, & 1/2 < x \le 1\\\end{array}$. What is the correct
  expression for $(f^{-1})'(1/2)$?

  \begin{enumerate}[(A)]
  \item It does not exist, since the two one-sided derivatives of $f$ at
    $1/2$ do not match.
  \item $\sqrt{2}$
  \item $2\sqrt{2}/\pi$
  \item $4/\pi$
  \item $4/(\sqrt{3}\pi)$
  \end{enumerate}

  {\em Answer}: Option (E)

  {\em Explanation}: We use:

  $$(f^{-1})'(1/2) = \frac{1}{f'(f^{-1}(1/2))}$$

  By inspection, we see that $f^{-1}(1/2)$ must be between $0$ and
  $1/2$. Thus, we must solve $\sin(\pi x/2) = 1/2$. This gives $\pi x
  / 2 = \pi/6$ (considering domain restrictions) so $x = 1/3$. Thus, we get:

  $$(f^{-1})'(1/2) = \frac{1}{f'(1/3)}$$

  The expression for the derivative is $(\pi/2)\cos(\pi x/2)$, which
  evaluated at $1/3$ gives $(\pi\sqrt{3})/4$. Taking the reciprocal,
  we get $4/(\pi\sqrt{3})$.

  Note that (A) is a sophisticated distractor in the sense that if you
  naively consider:

  $$(f^{-1})'(1/2) = \frac{1}{f'(1/2)}$$

  You will wrongly conclude (A). (B) and (C) are the one-sided
  derivative at $f(1/2)$, so these too are attractive propositions for
  the naive.

  {\em Performance review}: $2$ out of $12$ got this. $4$ chose (A),
  $3$ chose (D), $2$ chose (C), $1$ chose (B).

  {\em Historical note (last year)}: $1$ out of $15$ people got this
  correct. $7$ people chose (A), $5$ people chose (C), and $2$ people
  chose (B).

  {\em Action point}: I emphasized this point in class: $(f^{-1})'(x)
  = 1/(f'(f^{-1}(x)))$, not $1/(f'(x))$. However, the sting of getting
  it wrong on a quiz might be a greater spur to remember this fact
  forever. Make sure you never fall for this error again!
\end{enumerate}
\end{document}