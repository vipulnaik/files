\documentclass[10pt]{amsart}

%Packages in use
\usepackage{fullpage, hyperref, vipul, enumerate}

%Title details
\title{Class quiz solutions: March 30: Parametric stuff}
\author{Math 195, Section 59 (Vipul Naik)}
%List of new commands

\begin{document}

\maketitle

\section{Performance review}

$21$ people took this $6$-question quiz. The score distribution was as
follows:

\begin{itemize}
\item Score of $1$: $1$ person
\item Score of $2$: $2$ people
\item Score of $3$: $5$ people
\item Score of $4$: $7$ people
\item Score of $5$: $6$ people
\end{itemize}

The mean score was about $3.7$.

Here are the question wise answers and performance:

\begin{enumerate}
\item Option (D): $20$ people got this.
\item Option (A): $20$ people got this.
\item Option (C): $15$ people got this.
\item (*) Option (B): $7$ people got this.
\item (*) Option (D): $11$ people got this.
\item (+) Option (E): $5$ people got this.
\end{enumerate}

A (*) in front of a question indicates that there exist one or more
strong distractors in the answer choices, so you need to be more
careful. A (+) in front of a question indicates that one or more
incorrect answer choice was chosen more frequently than the correct
answer choice.

\section{Solutions}

\begin{enumerate}

\item Suppose $f$ and $g$ are both twice differentiable functions
  everywhere on $\R$. Which of the following is the correct formula
  for $(f \circ g)''$?

  \begin{enumerate}[(A)]

  \item $(f'' \circ g) \cdot g''$
  \item $(f'' \circ g) \cdot (f' \circ g') \cdot g''$
  \item $(f'' \circ g) \cdot (f' \circ g') \cdot (f \circ g'')$
  \item $(f'' \circ g) \cdot (g')^2 + (f' \circ g) \cdot g''$
  \item $(f' \circ g') \cdot (f \circ g) + (f'' \circ g'')$
  \end{enumerate}

  {\em Answer}: Option (D)

  {\em Explanation}: This question is tricky because it requires the
  application of both the product rule and the chain rule, with the
  latter being used twice. We first note that:

  $$(f \circ g)' = (f' \circ g) \cdot g'$$

  Now, we differentiate both sides:

  $$(f \circ g)'' = [(f' \circ g) \cdot g']'$$

  The expression on the right side that needs to be differentiated is
  a product, so we use the product rule:

  $$(f \circ g)'' = [(f' \circ g)' \cdot g'] + [(f' \circ g) \cdot g'']$$

  Now, the inner composition $f' \circ g$ needs to be
  differentiated. We use the chain rule and obtain that $(f' \circ g)'
  = (f'' \circ g) \cdot g'$. Plugging this back in, we get:

  $$(f \circ g)'' = (f'' \circ g) \cdot (g')^2 + (f' \circ g) \cdot g''$$

  {\em Remark}: What's worth noting here is that in order to
  differentiate composites of functions, you need to use both
  composites {\em and} products (that's the chain rule). And in order
  to differentiate products, you need to use both products {\em and}
  sums (that's the product rule). Thus, in order to differentiate a
  composite twice, we need to use composites, products, {\em and}
  sums.

  {\em Performance review}: $20$ out of $21$ people got this
  correct. $1$ person left the question blank.

  {\em Historical note}: I put this question in a quiz for Math 152
  back in October of this year, and $14$ of $14$ people who took that
  quiz got it correct.
\item Suppose $x = f(t)$ and $y = g(t)$ where $f$ and $g$ are both
  twice differentiable functions. What is $d^2y/dx^2$ in terms of $f$
  and $g$ and their derivatives evaluated at $t$?

  \begin{enumerate}[(A)]
  \item $(f'(t)g''(t) - g'(t)f''(t))/(f'(t))^3$
  \item $(f'(t)g''(t) - g'(t)f''(t))/(g'(t))^3$
  \item $(g'(t)f''(t) - f'(t)g''(t))/(f'(t))^3$
  \item $(g'(t)f''(t) - f'(t)g''(t))/(g'(t))^3$
  \item None of the above
  \end{enumerate}

  {\em Answer}: Option (A)

  {\em Explanation}: See lecture notes.

  {\em Performance review}: $20$ out of $21$ people got this
  correct. $1$ person chose (E).
\item Which of the following pair of bounds works for the arc length
  for the portion of the graph of the sine function between $(a,\sin
  a)$ and $(b, \sin b)$ where $a < b$?

  \begin{enumerate}[(A)]
  \item Between $(b - a)/\sqrt{3}$ and $(b - a)/\sqrt{2}$
  \item Between $(b - a)/\sqrt{2}$ and $b - a$
  \item Between $(b - a)$ and $\sqrt{2}(b - a)$
  \item Between $\sqrt{2}(b - a)$ and $\sqrt{3}(b - a)$
  \item Between $\sqrt{3}(b - a)$ and $2(b - a)$
  \end{enumerate}

  {\em Answer}: Option (C)

  {\em Explanation}: The derivative function is $\cos$, so the
  corresponding arc length formula gives:

  $$\int_a^b \sqrt{1 + \cos^2x} \, dx$$

  The integrand is always between $1$ and $\sqrt{2}$, so the integral
  must be between $1 \cdot (b - a)$ and $\sqrt{2} \cdot (b - a)$.

  {\em Performance review}: $15$ out of $21$ people got this
  correct. $2$ chose (B), $2$ left the question blank, $1$ each chose
  (A) and (D).

\item (*) Consider the parametric curve $x = e^t$, $y = e^{t^2}$. How does
  $y$ grow in terms of $x$ as $x \to \infty$?

  \begin{enumerate}[(A)]
  \item $y$ grows like a polynomial in $x$.
  \item $y$ grows faster than any polynomial in $x$ but slower than an
    exponential function of $x$.
  \item $y$ grows exponentially in $x$.
  \item $y$ grows super-exponentially in $x$ but slower than a double
    exponential in $x$.
  \item $y$ grows like a double exponential in $x$.
  \end{enumerate}

  {\em Answer}: Option (B)

  {\em Explanation}: Note that a polynomial in $x$ is still
  exponential in $t$, and not in $t^2$, i.e., it is too slow to be
  $y$. Thus $y$ grows faster than any polynomial in $x$. On the other
  hand, an exponential in $x$ is doubly exponential in $t$, which is
  faster in growth than $e^{t^2}$. Thus, option (B).

  {\em Performance review}: $7$ out of $21$ people got this
  correct. $4$ each chose (A) and (E), $3$ chose (C), $1$ chose (D),
  and $2$ left the question blank.

\item (*) We say that a curve is {\em algebraic} if it admits a
  parameterization of the form $x = f(t)$, $y = g(t)$, where $f$ and
  $g$ are rational functions and $t$ varies over some subset of the real
  numbers. Which of the following curves is {\em not} algebraic?

  \begin{enumerate}[(A)]
  \item $x = \cos t$, $y = \sin t$, $t \in \R$
  \item $x = \cos t$, $y = \cos(3t)$, $t \in \R$
  \item $x = \cos t$, $y = \cos^2t$, $t \in \R$
  \item $x = \cos t$, $y = \cos(t^2)$, $t \in \R$
  \item None of the above, i.e., they are all algebraic
  \end{enumerate}

  {\em Answer}: Option (D)

  {\em Explanation}: In all the other cases, we can elucidate an
  algebraic relationship between the variables.

  For option (A), we can set both $\cos t$ and $\sin t$ as rational
  functions in $\tan(t/2)$. In fact, the rational functions in
  $\tan(t/2)$ approach works for options (B) and (C) as well, though
  there are simpler approaches in those cases. The approach does not
  work for option (D).

  {\em Performance review}: $11$ out of $21$ people got this
  correct. $8$ chose (E), $1$ chose (B), $1$ left the question blank.

\item (+) Suppose $x = f(t)$, $y = g(t)$, $t \in \R$ is a parametric
  description of a curve $\Gamma$ and both $f$ and $g$ are continuous
  on all of $\R$. If both $f$ and $g$ are even, what can we conclude
  about $\Gamma$ and its parameterization?

  \begin{enumerate}[(A)]
  \item $\Gamma$ is symmetric about the $y$-axis
  \item $\Gamma$ is symmetric about the $x$-axis
  \item $\Gamma$ is symmetric about the line $y = x$
  \item $\Gamma$ has half turn symmetry about the origin
  \item The parameterizations of $\Gamma$ for $t \le 0$ and for $t \ge
    0$ both cover all of $\Gamma$, and in directions mutually reverse
    to each other.
  \end{enumerate}

  {\em Answer}: Option (E)

  {\em Explanation}: See lecture notes.

  {\em Performance review}: $5$ out of $21$ people got this
  correct. $7$ chose (A), $4$ chose (D), $2$ each chose (B) and (C),
  $1$ left the question blank.
\end{enumerate}
\end{document}