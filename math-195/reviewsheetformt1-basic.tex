\documentclass[10pt]{amsart}
\usepackage{fullpage,hyperref,vipul, graphicx}
\title{Review sheet for midterm 1: basic}
\author{Math 195, Section 59 (Vipul Naik)}

\begin{document}
\maketitle

We will not be going over this sheet, but rather, we'll be going over
the advanced review sheet in the session. Please review this sheet on your own time.

\section{Formula summary}

\subsection{Parametric}

Set $x = f(t)$, $y = g(t)$, parametric curve in $\R^2$.
\begin{itemize}
\item $dy/dt = g'(t)$ and $dx/dt = f'(t)$.
\item $\frac{dy}{dx} = \frac{g'(t)}{f'(t)}$.
\item $\frac{d^2y}{dx^2} = \frac{f'(t)g''(t) - g'(t)f''(t)}{(f'(t))^3}$
\item Arc length: $\int \sqrt{(f'(t))^2 + (g'(t))^2} \, dt$
\end{itemize}

\subsection{Polar}

Set $r = F(\theta)$, polar equation of a curve.

\begin{itemize}
\item $y = F(\theta)\sin \theta$ and $x = F(\theta)\cos \theta$.
\item $dy/d\theta = F'(\theta)\sin \theta + F(\theta)\cos \theta$ and
  $dx/d\theta = F'(\theta) \cos \theta - F(\theta) \sin \theta$.
\item $\frac{dy}{dx} = \frac{F'(\theta)\sin \theta + F(\theta)\cos
  \theta}{F'(\theta)\cos \theta - F(\theta)\sin \theta}$
\item Arc length: $\int \sqrt{(F(\theta))^2 + (F'(\theta))^2} \,
  d\theta$
\end{itemize}

\subsection{Three-dimensional geometry}

\begin{itemize}
\item Distance formula between $(x_1,y_1,z_1)$ and $(x_2,y_2,z_2)$:
  $\sqrt{(x_2 - x_1)^2 + (y_2 - y_1)^2 + (z_2 - z_1)^2}$.
\item Sphere with center having coordinates $(h,k,l)$ and radius $r$
  is $(x - h)^2 + (y - k)^2 + (z - l)^2 = r^2$.
\end{itemize}

\subsection{Vectors}

\begin{itemize}
\item Vector dot product: $\langle v_1,v_2,\dots, v_n \rangle \cdot
  \langle w_1,w_2,\dots,w_n \rangle = v_1w_1 + v_2w_2 + \dots + v_nw_n$.
\item Length of vector $\langle v_1,v_2,\dots,v_n \rangle$ is
  $\sqrt{v_1^2 + v_2^2 + \dots + v_n^2}$.
\item Unit vector in the direction of a vector $v$ is $v/|v|$. Unit
  vector in opposite direction but along same line (so parallel) is
  $-v/|v|$.
\item Vector cross product: $\langle a_1, a_2, a_3 \rangle \times
  \langle b_1, b_2, b_3 \rangle = \langle a_2b_3 - a_3b_2, a_3b_1 -
  a_1b_3, a_1b_2 - a_2b_1 \rangle$.
\item For nonzero vectors $v$ and $w$ in three dimensions, we have $|v
  \times w| = |v||w|\sin \theta$ where $\theta$ is the angle between
  $v$ and $w$.
\item Scalar triple product is $a \cdot (b \times c)$.
\item Angle between nonzero vectors $v$ and $w$ is
  $\arccos\left(\frac{v \cdot w}{|v||w|}\right)$.
\item Scalar projection of $b$ onto $a$ is $(a \cdot b)/|a|$. {\em
  Note: Be careful what is being projected onto what}.
\item Vector projection of $b$ onto $a$ is $((a \cdot b)/|a|^2)a$.
\item Area of triangle with vertices $P$, $Q$ and $R$ is $(1/2)|PQ
  \times PR|$. Need to: (i) compute difference vectors, (ii) take
  cross product, (iii) compute length of the cross product, (iv)
  divide by 2.
\item Area of parallelogram with vertices $P$, $Q$, $R$, $S$ is $|PQ
  \times PR|$ or $|PQ \times PS|$ (same number). Steps (i)-(iii) of above.
\item Volume of parallelepiped is {\em absolute value of} scalar triple
  product of vectors for adjacent triple of edges.
\end{itemize}

\section{Quickly: what you should know from one-variable calculus}

You need to be able to do the following from one-variable calculus and
before:

\begin{enumerate}
\item Finding domains of functions
\item Basic algebraic manipulation and trigonometric identities
\item Graphing: Know equation of circle centered at origin, graph
  linear functions, sine, cosine.
\item Differentiation and integration: Everything you saw in
  one-variable calculus. However, for this midterm, you will get only
  simple integrations that rely on the very basic formulas and not,
  for instance, those that use integration by parts.
\end{enumerate}

\section{Parametric stuff}

Words ...

\begin{enumerate}
\item A parametric description of a curve is one where both
  coordinates are expressed as functions of of a parameter, typically
  denoted $t$. Parametric descriptions offer an alternative to
  functional and implicit (relational) descriptions of curves. Here,
  $t$ varies over some subset of the real numbers. In symbols, we have
  something like $x = f(t)$, $y = g(t)$, where $t$ varies over some
  subset $D$ of the real numbers.
\item Descriptions where $x$ is a function of $y$ or $y$ is a function
  of $x$ are special cases of parametric descriptions.
\item The same curve may admit multiple parametrizations, and
  different parameterizations may correspond to different speeds and
  different orderings of traversal of the point. The curve itself only
  contains the information of {\em what} points were traversed, not
  the information of the {\em sequence} and {\em pace} in which they
  were traversed.
\item The curve traced by a parameterization depends not only on the
  coordinate functions but also the domain for the parameter. The
  larger the domain, in general, the larger the curve traced. However,
  in some cases, expanding the domain may not make the curve strictly
  larger. This happens in cases where both coordinate functions are
  even or have commensurable periods.
\item A parameterization of a curve may involve self-intersections,
  retracings (e.g., tracing back for even function pairs), or even
  wrapping around itself (for periodic function pairs).
\item Function composition allows us to switch between multiple
  parameterizations.
\item In some cases, it is possible to move back and forth between
  parametric and relational descriptions.
\item Parametric differentiation: if $x = f(t)$ and $y = g(t)$, then
  $dy/dx = (dy/dt)/(dx/dt) = g'(t)/f'(t)$. This can also be used to
  differentiate repeatedly. Note that the derivative is a function of
  $t$ rather than of $(x,y)$, so to find the derivative given the
  point $(x,y)$ we need to go back and determine $t$.
\item Higher derivatives can be computed iteratively using parametric
  differentiation. But note that it is {\em not} true that $d^2y/dx^2
  = (d^2y/dt^2)/(d^2x/dt^2)$. The actual formula/procedure is more
  complicated (see lecture notes or formula summary).
\item Arc length: The formula for arc length from $t = a$ to $t = b$
  (with $a < b$) is $\int_a^b \sqrt{(dx/dt)^2 + (dy/dt)^2} \, dt$.
\end{enumerate}

Actions ...

\begin{enumerate}
\item {\em Parametric to relational: elimination of parameter}: In
  many cases, it is possible to eliminate a parameter from a
  parametric description. The idea is to use some well known
  identities or manipulation techniques to try to directly relate $x$
  and $y$ by finding some equation between them that is true for all
  $t$. However, this is not the full story. We next need to see if
  there are additional restrictions on $x$ and $y$ deducible from the
  fact that they arose as function of $t$, also keeping in mind the
  domain restrictions on $t$.

  For instance, the parameterization $x = t^2, y = t^4$ for $t \in \R$
  can be rewritten as $y = x^2$, but we need the additional condition
  that $x \ge 0$.

  See more examples in the lecture notes, quizzes, and homeworks.
\item {\em Relational to parametric}: Here, we see a relation between
  $x$ and $y$, and try to choose a parametric description that would
  give rise to the relation. Again, the domain of choice for the
  parameter needs to be chosen wisely.

  See more examples in the lecture notes and quizzes.
\item {\em Parametric differentiation and geometric consequences}: We
  use the formula $(dy/dt)/(dx/dt)$. If $x = f(t)$ and $y = g(t)$,
  then this becomes $g'(t)/f'(t)$. This is valid for all $t$ in the
  interior of the domain of definition where both $f'$ and $g'$ are
  defined and $f' \ne 0$. If $f'(t) = 0$ but $g'(t) \ne 0$, we have a
  vertical tangent situation. If $g'(t) = 0$ but $f'(t) \ne 0$, we
  have a horizontal tangent situation.
\end{enumerate}

\section{Polar coordinates}

Words ...

\begin{enumerate}
\item {\em Specifying a polar coordinate system}: To specify a polar
  coordinate system, we need to choose a point (called the {\em
  origin} or {\em pole}), a half-line starting at the point (called
  the {\em polar axis} or {\em reference line}) and an orientation of
  the plane (chosen counter-clockwise in the usual depictions).
\item {\em Finding the polar coordinates of a point and vice versa}:
  The radial coordinate $r$ is the distance between the point and the
  pole. The angular coordinate $\theta$ is the angle (measured in the
  counter-clockwise direction) from the polar axis to the line segment
  from the pole to the point. Note that $\theta$ is uniquely defined
  up to addition of multiples of $2\pi$, and it becomes truly unique
  if we restrict it to a half-open half-closed interval of length
  $2\pi$. {\em The exception is the pole itself, for which $\theta$ is
  undefined} in the sense that any value of $\theta$ could be chosen.
\item {\em Converting between Cartesian and polar coordinates}: If we
  take the polar axis as the positive $x$-axis and the axis at an
  angle of $+\pi/2$ from it as the positive $y$-axis, we get a
  Cartesian coordinate system. The point defined by polar coordinates
  $(r,\theta)$ has Cartesian coordinates $(r\cos \theta,r \sin
  \theta)$. Conversely, given a point with Cartesian coordinates
  $(x,y)$ the corresponding polar coordinates are $r = \sqrt{x^2 +
  y^2}$ and $\theta$ is the unique angle (up to addition of multiples
  of $2\pi$) such that $x = r \cos \theta$, $y = r\sin \theta$.
\end{enumerate}

Actions ...

\begin{enumerate}
\item A functional description of the form $r = F(\theta)$ gives rise
  to a parametric description in Cartesian coordinates: $x =
  F(\theta)\cos \theta$ and $y = F(\theta)\sin \theta$. We can do the
  usual things (like find slopes of tangent lines) using this
  parametric description. Note that here, $\theta$ is typically
  allowed to vary over all of $\R$ rather than simply being restricted
  to an interval of length $2\pi$. The slope of the tangent line in
  Cartesian terms is given by:

$$\frac{dy}{dx} = \frac{dy/d\theta}{dx/d\theta} = \frac{d(F(\theta)\sin \theta)/d\theta}{d(F(\theta)\cos \theta)/d\theta} = \frac{F'(\theta)\sin \theta + F(\theta)\cos \theta}{F'(\theta) \cos \theta - F(\theta)\sin \theta}$$

\item The arc length is given by integrating $\sqrt{(F(\theta))^2 +
  (F'(\theta))^2}$ for $\theta$ in the suitable interval. (See quiz
  question on this).
\item An implicit (relational) description in Cartesian coordinates
  can be converted to a description in polar coordinates by replacing
  $x$ by $r\cos \theta$ and $y$ by $r\sin \theta$.
\item An implicit (relational) description in polar coordinates can
  sometimes be converted to a description in Cartesian coordinates,
  but with some ambiguity. General idea: replace $r$ by $\sqrt{x^2 +
  y^2}$, $\cos \theta$ by $x/\sqrt{x^2 + y^2}$, and $\sin \theta$ by
  $y/\sqrt{x^2 + y^2}$.
\end{enumerate}

\section{Three-dimensional geometry}

Words ...

\begin{enumerate}
\item Three-dimensional space is coordinatized using a Cartesian
  coordinate system by selecting three mutually perpendicular axes
  passing through a point called the origin: the $x$-axis, $y$-axis,
  and $z$-axis. These satisfy the right-hand rule. The coordinates of
  a point are written as a $3$-tuple $(x,y,z)$.
\item There are $2^3 = 8$ octants based on the signs of each of the
  coordinates. There are three coordinate planes, each corresponding
  to the remaining coordinate being zero (the $xy$-plane corresponds
  to $z = 0$, etc.). There are three axes, each corresponding to the
  other two coordinates being zero (e.g., the $x$-axis corresponds to
  $y = z = 0$).
\item The distance formula between points with coordinates
  $(x_1,y_1,z_1)$ and $(x_2,y_2,z_2)$ is: $$\sqrt{(x_2 - x_1)^2 + (y_2
  - y_1)^2 + (z_2 - z_1)^2}$$ This is similar to the formula in two
  dimensions and the squares and square root arises from the
  Pythagorean theorem.
\item The equation of a sphere with center having coordinates
  $(h,k,l)$ and radius $r$ is $(x - h)^2 + (y - k)^2 + (z - l)^2 =
  r^2$. Given an equation, we can try completing the square to see if
  it fits the model for the equation of a sphere.
\end{enumerate}

\section{Introduction to vectors and relation with geometry}

\subsection{$n$-dimensional generality}

Words ...

\begin{enumerate}
\item A vector is an ordered $n$-tuple of real numbers (or quantities
  measured using real numbers). The space of such $n$-tuples is a
  $n$-dimensional vector space over the real numbers. Vectors can be
  used to store tuples of prices, probabilities, and other kinds of
  quantities.
\item There is a zero vector. We can add vectors and we can multiply a
  vector by a scalar. Note that these operations may or may not have
  an actual meaning based on the thing we are storing using the vector.
\item We can take the dot product $v \cdot w$ of two vectors $v$ and
  $w$ in $n$-dimensional space. if $v = \langle v_1, v_2, \dots, v_n
  \rangle$ and $w = \langle w_1, w_2, \dots, w_n \rangle$, then $v
  \cdot w = \sum_{i=1}^n v_iw_i$. The dot product is a real number
  (though if we put units to the coordinates of the vector, it gets
  corresponding squared units).
\item The length or norm of a vector $v$, denoted $|v|$, is defined as
  $\sqrt{v \cdot v}$. It is a nonnegative real number.
\item The correlation between two vectors $v$ and $w$ is defined as
  $(v \cdot w)/(|v||w|)$. It is in $[-1,1]$. (For geometric
  interpretation, see the three-dimensional case).
\item {\em Properties of the dot product}: The dot product is
  symmetric, the dot product of any vector with the zero vector is
  $0$, the dot product is additive (distributive) in each coordinate
  and scalars can be pulled out.
\item {\em Properties of length}: The only vector with length zero is
  the zero vector, all other vectors have positive length. The length
  of $\lambda v$ is $|\lambda|$ times the length of $v$. We also have
  $|v + w| \le |v| + |w|$ for any vectors $v$ and $w$, with equality
  occurring if either is a positive scalar multiple of the other or
  one of them is the zero vector.
\end{enumerate}

\subsection{Three-dimensional geometry}

Words ...

\begin{enumerate}
\item We can identify points in three-dimensional space with
  three-dimensional vector as follows: the vector corresponding to a
  point $(x,y,z)$ is the vector $\langle x,y,z \rangle$. Physically,
  this can be thought of as a directed line segment or arrow from the
  origin to the point $(x,y,z)$.
\item We can also consider vectors starting at any point in
  three-dimensional space and ending at any point. The corresponding
  vector can be obtained by subtracting the coordinates of the
  points. The vector from point $P$ to point $Q$ is denoted
  $\overline{PQ}$.
\item There are unit vectors $\mathbf{i} = \langle 1,0,0 \rangle$,
  $\mathbf{j} = \langle 0,1,0 \rangle$, and $\mathbf{k} = \langle
  0,0,1 \rangle$. These are thus the vectors of length $1$ along the
  positive $x$, $y$, and $z$ directions respectively. A vector
  $\langle x,y,z \rangle$ can be written as $x\mathbf{i} + y\mathbf{j}
  + z\mathbf{k}$.
\item Vectors can be added geometrically using the {\em parallelogram
  law}. This procedure gives the same answer as the usual
  coordinate-wise addition.
\item Scalar multiplication also has a geometric interpretation -- the
  length gets scaled by the scalar multiple, and the direction remains
  the same or is reversed depending on the scalar's sign.
\item For vectors $v$ and $w$, we have $v \cdot w = |v||w|\cos \theta$
  where $\theta$ is the angle between $v$ and $w$. We can use this
  procedure to find the angle between two vectors. The correlation
  between the vectors is thus $\cos \theta$. We can interpret this
  specifically for $\theta = 0$, $\theta$ an acute angle, $\theta =
  \pi/2$, $\theta$ an obtuse angle, and $\theta = \pi$ (see the table
  in the lecture notes).
\item We can define the vector cross product $v \times w$ using a
  matrix determinant. Equivalently, if $v = \langle v_1,v_2,v_3
  \rangle$ and $w = \langle w_1,w_2,w_3 \rangle$, then $v \times w =
  \langle v_2w_3 - v_3w_2, v_3w_1 - v_1w_3, v_1w_2 - v_2w_1
  \rangle$. {\em This is a specifically three-dimensional construct}.
\item The cross product has the property that cross product of any two
  collinear vectors is zero, cross product of any vector with the
  zero vector is zero, the cross product is skew-symmetric,
  distributive in each variable, and allows scalars to be pulled
  out. It is not associative in general. There is an identity relating
  cross product and dot product: $a \times (b \times c) = (a \cdot c)b
  - (a \cdot b)c$. Also, the cross product satisfies the relation: $$a
  \times (b \times c) + b \times (c \times a) + c \times (a \times b)
  = 0$$
\item The cross product of $a$ and $b$ satisfies $|a \times b| =
  |a||b| \sin \theta$ where $\theta$ is the angle between $a$ and $b$,
  and further, the cross product vector is perpendicular to both $a$
  and $b$, and its direction is given by the right hand rule.
\item There is a scalar triple product. The scalar triple product of
  vectors $a$, $b$, and $c$ is defined as the number $a \cdot (b
  \times c)$. It can also be viewed as the determinant of a matrix
  whose rows are the coordinates of $a$, $b$, and $c$
  respectively. The scalar triple product is preserved under cyclic
  permutations of the input vectors and gets negated under flipping
  two of the input vectors. It is linear in each input variable (i.e.,
  distributive and pulls out scalars). The scalar triple product is
  zero if and only if the three input vectors can all be made to lie
  in the same plane.
\item {\em Added for clarification}: In particular, $a \cdot (a \times
  b) = 0$ and $b \cdot (a \times b) = 0$ for any vectors $a$ and $b$
  in three dimensions.
\end{enumerate}

Actions ...

\begin{enumerate}
\item Vector and scalar projections: Given vectors $a$ and $b$, the
  {\em vector projection} of $b$ onto $a$, denoted
  $\operatorname{proj}_ab$, is given by the vector $\frac{a \cdot
    b}{|a|^2} a$. The scalar projection or component of
  $b$ along $a$, denoted $\operatorname{comp}_ab$, is given by
  $\frac{a \cdot b}{|a|}$. The vector projection is what we obtain by
  taking the vector from the origin to the foot of the perpendicular
  from the head of $b$ to the line of $a$. The scalar projection is
  the {\em directed} length of this vector, measured positive in the
  direction of $a$.
\item Finding the angle between vectors: This is done using the dot
  product. The angle between vectors $v$ and $w$ is $\arccos((v \cdot
  w)/|v||w|)$.
\item Finding the area of a triangle or a parallelogram: We first find
  two adjacent sides as vectors both with the same starting vertex (by
  taking the differences of coordinates of endpoints). For the
  parallelogram, we take the length of the cross product of these two
  vectors. For the triangle, we take {\em half} the length.
\item Finding the volume of a parallelopiped: We find three sides as
  vectors, all with the same starting vertex. Then we take the {\em
  absolute value of} the scalar triple product of these sides.
\item Finding a vector orthogonal to two given vectors: Simply take
  the cross product if they are linearly independent. Otherwise, just
  pick anything that dots with one of them to zero.
\item Testing orthogonality: We check whether the dot product is zero.
\item Testing coplanarity of points: We take one point as the
  basepoint, compute difference vectors to it from the other three
  points. We then take the scalar triple product of these three
  vectors. If we get zero, then the four points are coplanar,
  otherwise they are not.
\end{enumerate}

\section{Vector-valued functions}

\subsection{Vector-valued functions, limits, and continuity}

\begin{enumerate}
\item {\em Not for review discussion}: A vector-valued function is a
  function from $\R$, or a subset of $\R$, to a vector space
  $\R^n$. It comprises $n$ scalar functions, one for each of the
  coordinates. For instance, given scalar functions $f_1, f_2, \dots,
  f_n$, we can construct a vector-valued function $f = \langle f_1,
  f_2, \dots, f_n \rangle$ defined by $t \mapsto \langle f_1(t),
  f_2(t), \dots, f_n(t) \rangle$.
\item {\em Not for review discussion}: A vector-valued function in $n$
  dimensions corresponds to a parametric description of a curve in
  $\R^n$ whose points are just the heads of the corresponding
  vectors. The vector-valued function from the previous observation
  has corresponding curve $\{ (f_1(t),f_2(t),\dots,f_n(t)): t \in D \}$
  where $D$ is the appropriate domain.
\item To add two vector-valued functions in $n$ dimensions, we add
  them coordinate-wise, where the corresponding scalar functions are
  added pointwise as usual. This sum is also a vector-valued function
  in $n$ dimensions.
\item We can multiply a scalar function and a vector-valued function
  to get a new vector-valued function. At each point in the domain,
  this is just multiplication of the corresponding scalar number and
  the corresponding vector.
\item We can take the dot product of two vector-valued functions in
  $n$ dimensions. The dot product is a scalar-valued function. At each
  point in the domain, it is obtained by taking the dot product of the
  corresponding vector values.
\item For $n = 3$, we can take the cross product of two vector-valued
  functions and get a vector-valued function. This cross product is
  taken pointwise.
\item To calculate the limit of a vector-valued function at a point,
  we calculate the limit separately for each coordinate. We use this
  idea to define the {\em limit}, {\em left hand limit}, and {\em
  right hand limit} at any point in the domain.
\item Limit theorems: Limit of sum is sum of limits, constant scalars
  pull out of limits, limit of scalar-vector product is product of
  scalar limit and vector limit, limit of dot product is dot product
  of limits, limit of cross product (case $n = 3$) is cross product of limits.
\item A vector-valued function is {\em continuous} at a point in its
  domain if each coordinate function is continuous, or equivalently,
  if the limit equals the value. We say it is continuous on its
  interval if it is continuous at every point in the interior of the
  interval and has one-sided continuity at one of the endpoints.
\item Continuity theorems: Sum of continuous vector-valued functions
  is continuous, product of continuous scalar function and continuous
  vector-valued function is continuous, dot product of continuous
  vector-valued functions is continuous, cross product (case $n = 3$)
  of continuous vector-valued functions is continuous.
\item There is no $n$-dimensional analogue of the intermediate value
  theorem, multiple things fail.
\end{enumerate}

Actions ...

\begin{enumerate}
\item If no domain is specified, the domain of a vector-valued
  function is the intersection of the domains of all the constituent
  scalar functions.
\end{enumerate}

\subsection{Top-down and bottom-up descriptions}

Words ...

\begin{enumerate}
\item A top-down description of a subset of $\R^n$ is in terms of a
  system of equations and inequality constraints. Each equation
  (equality constraint) is expected to reduce the dimension by $1$ (we
  start from $n$) whereas inequality constraints usually have no
  effect on the dimension. So if there are $k$ independent equality
  constraints describing a subset of $\R^n$, we expect the subset to
  have dimension $n - k$.
\item A bottom-up description is a parametric description with
  possibly more than one parameter. The number of parameters needed is
  the dimension of the subset. The parametric descriptions we have
  seen so far are $1$-parameter descriptions and hence they describe
  curves -- $1$-dimensional subsets.
\item The codimension of a $m$-dimensional subset is $n - m$.
\item When intersecting, codimensions are expected to add. If the
  total codimension we get after adding is greater than the dimension of
  the space, the intersection is expected to be empty.
\item In $\R^3$, curves are one-dimensional, surfaces are
  two-dimensional. Thus, curves are not expected to intersect each
  other, but curves and surfaces are expected to intersect at finite
  collections of points (in general).
\end{enumerate}

Actions ...

\begin{enumerate}
\item Strategy for finding intersection of subsets in $\R^n$
  (specifically, curves and surfaces in $\R^3$) given with top-down
  descriptions: Take all the equations together and solve
  simultaneously.
\item Strategy for finding intersection of curve given parametrically
  and curve or surface given by top-down description: Plug in the
  functions of the parameter for the coordinates in the top-down description.
\item Strategy for finding intersection of curves given
  parametrically: Choose different letters for parameter values, and
  then equate coordinate by coordinate. We get a bunch of equations in
  two variables (the two parameter values).
\item Strategy for finding collision of curves given parametrically:
  Just equate coordinates, using the same letter for parameter
  values. Get a bunch of equations all in one variable.
\end{enumerate}

\subsection{Differentiation, tangent vectors, integration}

\begin{enumerate}
\item The derivative of a $n$-dimensional vector-valued function is
  again a $n$-dimensional vector-valued function. It can be defined by
  differentiating each coordinate with respect to the parameter, or by
  using a difference quotient expression. These definitions are
  equivalent.
\item This derivative operation satisfies the sum rule, pulling out
  constant scalars, and product rules for scalar-vector
  multiplication, dot product, and cross product (case $n = 3$).
\item As a free vector, the tangent vector at $t = t_0$ to a
  parametric description of a curve is just the derivative vector for
  the corresponding vector-valued function. As a localized vector, it
  starts off at the corresponding point in $\R^n$.
\item The tangent vector for a curve with parametric description
  depends on the choice of parameterization. The {\em unit tangent
  vector} does not, apart from the issue of direction (forward or
  backward). The unit tangent vector is a unit vector (i.e., length
  $1$ vector) in the direction of the tangent vector. It is unique for
  a given curve (independent of parameterization) up to
  forward-backward issues.
\item To perform definite or indefinite integration of a vector-valued
  function, we perform the integration coordinate-wise.
\end{enumerate}

\end{document}