\documentclass[10pt]{amsart}
\usepackage{fullpage,hyperref,vipul,graphicx}
\title{Equations of lines and planes}
\author{Math 195, Section 59 (Vipul Naik)}

\begin{document}
\maketitle

{\bf Corresponding material in the book}: Section 12.5.

{\bf What students should definitely get}: Parametric equation of line
given in point-direction and two-point form, symmetric equations of
line, degenerate cases where direction vector has one or more
coordinate zero, intersecting lines, equation of plane, angle between
planes, line of intersection of planes, distance of point and plane.

{\bf What students should hopefully get}: How the equation setup
relates to the general setup for curves and surfaces. Understanding of
the degenerate cases. Role of parameter restrictions in defining a
line segment. Deeper understanding of relationship of direction vector
and direction cosines.


\section*{Executive summary}

\subsection{Direction cosines}

\begin{enumerate}
\item For a nonzero vector $v$, there are two unit vectors parallel to
  $v$, namely $v/|v|$ and $-v/|v|$.
\item The direction cosines of $v$ are the coordinates of $v/|v|$. if
  $v/|v| = \langle \ell,m,n \rangle$, then the direction cosines are
  $\ell$, $m$, and $n$. We have the relation $\ell^2 + m^2 + n^2 =
  1$. Further, if $\alpha$, $\beta$, and $\gamma$ are the angles made
  by $v$ with the positive $x$, $y$, and $z$ axes, then $\ell = \cos
  \alpha$, $m = \cos \beta$, and $n = \cos \gamma$.
\end{enumerate}

\subsection{Lines}
Words ...

\begin{enumerate}
\item A line in $\R^3$ has dimension $1$ and codimension $2$. A
  parametric description of a line thus requires $1$ parameter. A
  top-down equational description requires two equations.
\item Given a point with radial vector $\mathbf{r}_0$ and a direction
  vector $\mathbf{v}$ along a line, the parametric description of the
  line is given by $\mathbf{r}(t) = \mathbf{r}_0 + t\mathbf{v}$. If
  $\mathbf{r}_0 = \langle x_0, y_0,z_0 \rangle$ and $\mathbf{v} =
  \langle a,b,c \rangle$, this is more explicitly described as $x =
  x_0 + ta$, $y = y_0 + tb$, $z = z_0 + tc$.
\item Given two points with radial vectors $\mathbf{r_0}$ and
  $\mathbf{r_1}$, we obtain a vector equation for the line as
  $\mathbf{r}(t) = t\mathbf{r_1} + (1 - t)\mathbf{r_0}$. If we
  restrict $t$ to the interval $[0,1]$, then we get the line segment
  joining the points with these radial vectors.
\item If the line is not parallel to any of the coordinate planes,
  this parametric description can be converted to symmetric equations
  by eliminating the parameter $t$. With the above notation, we get:

  $$\frac{x - x_0}{a} = \frac{y - y_0}{b} = \frac{z - z_0}{c}$$

  This is actually {\em two} equations rolled into one.

\item If $c = 0$ and $ab \ne 0$, the line is parallel to
  the $xy$-plane, and we get the equations:

  $$\frac{x - x_0}{a} = \frac{y - y_0}{b}, \qquad z = z_0$$

  Similarly for the other cases where precisely one coordinate is zero.
\item If $a = b = 0$ and $c \ne 0$, the line is parallel to the
  $z$-axis, and we get the equations:

  $$x = x_0, \qquad y = y_0$$
\end{enumerate}

Actions ...

\begin{enumerate}
\item To intersect two lines both given parametrically: Choose
  different letters for parameters, equate coordinates, solve $3$
  equations in $2$ variables. {\em Note: Expected dimension of
  solution space is $2 - 3 = -1$}.
\item To intersect a line given parametrically and a line given by
  equations: Plug in the coordinates as functions of parameters into
  both equations, solve. Solve $2$ equations in $1$ variable. {\em
  Note: Expected dimension of solution space is $1 - 2 = -1$}.
\item To intersect two lines given by equations: Combine equations,
  solve $4$ equations in $3$ variables. {\em Note: Expected dimension
  of solution space is $3 - 4 = -1$}.
\end{enumerate}

\subsection{Planes}

Words ...

\begin{enumerate}
\item Vector equation of a plane (for the radial vector) is
  $\mathbf{n} \cdot (\mathbf{r} - \mathbf{r_0}) = 0$ where
  $\mathbf{n}$ is a normal vector to the plane and $\mathbf{r_0}$ is
  the radial vector of any fixed point in the plane. This can be
  rewritten as $\mathbf{n} \cdot \mathbf{r} = \mathbf{n} \cdot
  \mathbf{r_0}$. Using $\mathbf{n} = \langle a,b,c \rangle$,
  $\mathbf{r} = \langle x,y,z \rangle$, and $\mathbf{r}_0 = \langle
  x_0,y_0,z_0 \rangle$, we get the corresponding scalar equation $ax +
  by + cz = ax_0 + by_0 + cz_0$. Set $d = -(ax_0 + by_0 + cz_0)$ and
  we get $ax + by + cz + d = 0$.
\item The ``direction'' or ``parallel family'' of a plane is
  determined by its normal vector. The angle between planes is the
  angle between their normal vectors. Two planes are parallel if their
  normal vecors are parallel. And so on.
\end{enumerate}

Actions ...

\begin{enumerate}
\item Given three non-collinear points, we find the equation of the
  unique plane containing them as follows: first we find a normal
  vector by taking the cross product of two of the difference
  vectors. Then we use any of the three points to calculate the dot
  product with the normal vector in the above vector equation or the
  corresponding scalar equation.

  Note that if the points are collinear, there is no unique plane
  through them -- any plane containing their line is a plane
  containing them.
\item We can compute the angle of intersection of two planes by
  computing the angle of intersection of their normal vectors.
\item The line of intersection of two planes that are not parallel can
  be computed by simply taking the equations for {\em both}
  planes. This, however, is not a standard form for a line in
  $\R^3$. To find a standard form, either find two points by
  inspection and join them, or find one point by inspection and
  another point by taking the cross product of the normal vectors to
  the plane.
\item To intersect a plane and a line, plug in parametric expressions
  for the coordinates arising from the line into the equation of the
  plane. We get one equation in the one parameter variable. In
  general, this is expected to have a unique solution for the
  parameter. Plug in the value of the parameter into the parametric
  expressions for the line and get the coordinates of the point of
  intersection.
\item For a point with coordinates $(x_1,y_1,z_1)$ and a plane $ax +
  by + cz + d = 0$, the distance of the point from the plane is given
  by $|ax_1 + by_1 + cz_1 + d|/\sqrt{a^2 + b^2 + c^2}$.
\end{enumerate}
\section{Lines and planes}

\subsection{Lines: dimension and codimension}

A line in $\R^n$ has dimension one and codimension $n - 1$. In
particular, a line in Euclidean space $\R^3$ has dimension $1$ and
codimension $3 - 1 =2$. In particular, based on what we know of
dimension and codimension, we expect that:

\begin{itemize}
\item In a top-down or relational description, we should need {\em
  two} independent equations to define a line.
\item In a bottom-up or parametric description, we should need {\em
  one} parameter to define a line.
\end{itemize}

\subsection{Planes: dimension and codimension}

A plane in $\R^3$ is $2$-dimensional, and it has codimension $3 - 2 =
1$. In particular, based on what we know of dimension and codimension,
we expect that:

\begin{itemize}
\item In a top-down or relational description, we should need {\em
  one} equation to define a plane.
\item In a bottom-up or parametric description, we should need {\em
  two} parameters to define a plane. {\em This gets into the realm of
  functions of two variables, so we will defer the actual
  $2$-parameter description of planes for now}.
\end{itemize}

\subsection{Intersection theory}

We have the following basic intersection facts:

\begin{tabular}{|l|l|l|l|l|}
  \hline
  Intersect & Generic case & Special case 1 & Special case 2 & Special case 3 \\
  \hline
  Plane, plane & Line & Empty (parallel planes) & Plane (equal planes) & \\
  \hline
  Plane, line & Point & Empty (line parallel to, not on plane) & Line (line on plane) & \\
  \hline
  Line, line & Empty (skew lines) & Point (intersecting lines) & Empty (parallel lines) & Line (equal lines) \\
  \hline
\end{tabular}

The {\em generic case} here represents the case that is most likely,
i.e., the case that would arise if the things being intersected were
chosen randomly. There are mathematical ways of making this precise,
but these are beyond the current scope.

In particular, it is worth pointing out that the generic case is
exactly as intersection theory predicts. Let's consider the three
generic cases:

\begin{itemize}
\item {\em Generic intersection of plane and plane}: A plane has
  codimension $1$, so the intersection of two planes (generically) has
  codimension $1 + 1 = 2$. We know that a line has codimension $2$, so
  this makes sense.
\item {\em Generic intersection of plane and line}: A plane has
  codimension $1$ and a line has codimension $2$, so the intersection
  of a plane and a line (generically) has codimension $1 + 2 = 3$, so
  it is zero-dimensional. A point is zero-dimensional.
\item {\em Generic intersection of line and line}: A line has
  codimension $2$, so the intersection of two lines (generically) has
  codimension $2 + 2 = 4$, so it has dimension $3 - 4 = -1$. Negative
  dimension indicates that the intersection is generically empty.
\end{itemize}

After we study the intersection theory in detail for lines and planes,
we will be in a position to acquire a better understanding of the {\em
general principles} of intersection theory. Specifically, we will
acquire a better grasp of the {\em non-generic} cases where the
intersections don't work out as they generically do.

\section{Equations of lines}

\subsection{The point-direction form}

The general principle behind this is the same as it is with the {\em
point-slope form}. Basically, to describe a line, it suffices to
specify a point on the line, and the {\em direction} of the line.

The {\em direction} is specified by specifying any vector parallel to
the line. Specifically, given a line with points $A$ and $B$ on it,
the direction of the line is given by taking the vector $AB$. Note
that any two vectors that are scalar multiples of each other (i.e.,
parallel to each other) specify the same direction.

Suppose $\mathbf{r_0}$ is the radial vector for one point on the line,
and $\mathbf{v}$ is any nonzero vector along the line. Then the radial
vector (i.e., vector from the origin to a point) for points on the
line can be defined by the parametric equation:

$$\mathbf{r}(t) = \mathbf{r_0} + t\mathbf{v}$$

where $t$ varies over the real numbers. For each value of $t$, we get
a radial vector for some point on the line, and every point on the
line is covered this way.

Suppose $\mathbf{r_0} = \langle x_0, y_0, z_0 \rangle$ and $\mathbf{v} =
\langle a,b,c \rangle$. Then $\mathbf{r_0} + t\mathbf{v}$ is the
vector:

$$\langle x_0 + ta, y_0 + tb, z_0 + tc \rangle$$

The corresponding parametric description of a curve is:

$$\{ (x_0 + ta, y_0 + tb, z_0 + tc) : t \in \R \}$$

Note that the {\em choice} of parametric description depends on the
choice of basepoint in the line and the choice of vector (which can be
varied up to scalar multiples).

By the way, here is some terminology (which we overlooked
earlier). The {\em direction cosines} for a particular direction are
defined as the coordinates of the {\em unit vector} in that
direction. The direction cosines of a particular direction are denoted
$\ell$, $m$, and $n$. For instance, if a direction vector is $\langle
1,2,3 \rangle$, then the corresponding unit vector is $\langle
1/\sqrt{14}, 2/\sqrt{14}, 3/\sqrt{14} \rangle$, so the direction
cosines are $\ell = 1/\sqrt{14}$, $m = 2/\sqrt{14}$, and $n =
3/\sqrt{14}$.

The direction cosines are also the cosines of the angles made by the
vectors with the $x$-axis, $y$-axis, and $z$-axis. They satisfy the
relation:

$$\ell^2 + m^2 + n^2 = 1$$

\subsection{The two-point form}

Suppose $\mathbf{r_0}$ and $\mathbf{r_1}$ are the radial vectors of two
points on a line. Then, we can get a line in the point-direction form
by setting $\mathbf{v} = \mathbf{r_1} - \mathbf{r_0}$. We thus get the form:

$$\mathbf{r}(t) = \mathbf{r_0} + t(\mathbf{r_1} - \mathbf{r_0})$$

This simplifies to:

$$\mathbf{r}(t) = t\mathbf{r_1} + (1 - t)\mathbf{r_0}$$

As $t$ varies over all of $\R$, this gives the whole line. When $t =
0$, we get the point with radial vector $\mathbf{r_0}$ and when $t = 1$,
we get the point with radial vector $\mathbf{r_1}$. If we allow only $0
\le t \le 1$, we get the {\em line segment} joining the two points.

\subsection{Top-down description: symmmetric equations}

To obtain the symmetric equations, we start with the parametric
equations and then eliminate the parameter. In other words, with the
parametric description:

$$\{ (x_0 + ta, y_0 + tb, z_0 + tc) : t \in \R \}$$

We note that:

$$x = x_0 + ta, \qquad \implies \qquad t = \frac{x - x_0}{a}$$

Similarly, we get $t = (y - y_0)/b$ and $t = (z - z_0)/c$. Eliminating
$t$, we get:

$$\frac{x - x_0}{a} = \frac{y - y_0}{b} = \frac{z - z_0}{c}$$

Note that while this looks like a single long equation, it is actually
{\em two} equations:

$$\frac{x - x_0}{a} = \frac{y - y_0}{b}$$

and

$$\frac{y - y_0}{b} = \frac{z - z_0}{c}$$

This is in keeping with what we expect/hope -- that to describe a
$1$-dimensional subset in $3$-dimensional space, we need $3 - 1 = 2$
equations.

Intuitively, what these equations are saying is that the coordinate
changes are in the ratio $a:b:c$.

\subsection{Exceptional case of lines parallel to one of the coordinate planes}

The symmetric equations formulation breaks down if one of the
coordinates of the direction vector $\langle a,b,c \rangle$ is
zero. In this case, the line is parallel to one of the three
coordinate planes, with the third coordinate being unchanged (e.g., if
$c = 0$, then the line is parallel to the $xy$-plane, because its
$z$-coordinate is unchanged).

They break down even more when two coordinates of the direction vector
are zero, which means that the line is parallel to one of the axes.

In this case, the symmetric equations given above do not work, and we
instead do the following.

\begin{itemize}
\item If only one coordinate of the direction vector is zero: If $c =
  0$ and $a,b \ne 0$, then we get the two equations:

  $$\frac{x - x_0}{a} = \frac{y - y_0}{b}, \qquad z = z_0$$

  Similarly for the other cases.
\item If two coordinates are zero: If, say $a = b = 0$, then we get
  the two equations:

  $$x = x_0, \qquad y = y_0$$

  $z$ does not appear in the equations because it can vary
  freely. This line is parallel to the $z$-axis.
\end{itemize}

\subsection{Pairs of lines: questions about intersection}

As we noted earlier, lines in $\R^3$ have codimension $2$, so the
intersection of two lines is expected to be empty. There are
qualitatively four possibilities:

\begin{enumerate}
\item The lines are skew lines: This is the most ``independent'' case
  possible. Here, the equations describing the two lines are as
  independent of each other as possible and the two lines thus do not
  lie in the same plane. They do not intersect.
\item The lines are intersecting lines in the same plane: This is a
  somewhat less independent case. Here, there is a plane (not
  necessarily containing the origin) that contains both lines, and the
  lines are not parallel, so they intersect at a point.
\item The lines are parallel lines in the same plane: Here, the
  equations for the line are inconsistent in a specific way, so they
  lie in the same plane but are parallel. They do not intersect. {\em
  Although the conclusion about intersection is the same both for
  pairs of parallel lines and for pairs of skew lines, the reasons
  behind this conclusion are different.}
\item The two lines are actually the same line: In this case, their
  intersection is the same line. This is the most dependent case
  possible.
\end{enumerate}

We now examine how to find the intersection of two lines. The approach
is simply a special case of finding the intersection of two
curves. Since the equations are all linear, we can actually devise
specific procedures to solve the equations.

\begin{itemize}
\item {\em Both lines are given parametrically}: In this case, we
  first make sure we have different letters for the parameters for
  each line. Then we equate coordinate-wise and solve the system of
  $3$ linear equations in $2$ variables (the parameter variables for
  the two lines). Note that the number of equations is more than the
  number of variables -- unsurprising since the generic case is one of
  skew lines.

  After finding solutions for the two parameters, plug back to find
  the points.
\item {\em One line is given parametrically in terms of $t$, the other
  using symmetric equations}: We substitute the parametric expressions
  into the values of $x$, $y$, and $z$ in the symmetric equations and
  solve the system of two equations in the one (parameter) variable
  $t$. After finding solutions for $t$, plug back to find the points.
\item {\em Both lines are given by symmetric equations}: We solve all
  the four symmetric equations.
\end{itemize}

\section{Planes}

\subsection{Vector description in terms of dot product}

For a given plane in $\R^3$, it either already passes through the
origin, or there is a unique plane parallel to it that passes through
the origin. We say that two planes are {\em parallel} if they either
coincide or they do not intersect -- equivalently, if for every line
in one plane, there is a line in the other plane parallel to it.

A family of parallel planes can be thought of as sharing a
direction. But how do we specify the direction of a plane, which is a
two-dimensional object? The idea is to look at the {\em complement},
or the {\em codimension}, of the plane. Specifically, we look at the
direction that is {\em orthogonal} to the plane.

There is a unique direction vector (up to scalar multiples) orthogonal
to a family of parallel planes. Further, the dot product of this
vector with the radial vector in any fixed plane in the family is a
constant, and this constant differs for each plane in the family. This
allows us to give equations for planes as follows.

Let $\mathbf{n}$ be a normal vector (orthogonal vector) to a plane and
let $\mathbf{r_0}$ be the radial vector for a fixed point in the
plane. Then, if $\mathbf{r}$ is the radial vector for an arbitrary
point in the plane, we have:

$$\mathbf{n} \cdot (\mathbf{r} - \mathbf{r_0}) = 0$$

Rearranging, we get:

$$\mathbf{n} \cdot \mathbf{r} = \mathbf{n} \cdot \mathbf{r_0}$$

Note that the right side is an actual real number.

If $\mathbf{n} = \langle a,b,c \rangle$ and $\mathbf{r_0} = \langle
x_0,y_0,z_0 \rangle$, we get the scalar equation:

$$ax + by + cz = ax_0 + by_0 + cz_0$$

If we define $d = -(ax_0 + by_0 + cz_0)$, we can rewrite the above as:

$$ax + by + cz + d = 0$$

Conversely, any equation of the above sort, where at least one of the
numbers $a$, $b$, and $c$ is nonzero, gives a plane.

\subsection{Plane parallel to the coordinate axes and planes}

We say that a plane and a line are parallel if either the line lies on
the plane or they do not intersect at all.

If $a = 0$, the plane is parallel to the $x$-axis. If $b = 0$, the
plane is parallel to the $y$-axis. If $c = 0$, the plane is parallel
to the $z$-axis.

If $a = b = 0$, the plane is parallel to the $xy$-plane. If $b = c =
0$, the plane is parallel to the $yz$-plane. If $a = c = 0$, the plane
is parallel to the $xz$-plane.

\subsection{Finding the equation of a plane given three points}

To specify a plane, we need to provide at least three points on the
plane. Given these three points, we can find the equation of the plane
as follows:

\begin{itemize}
\item We first take two difference vectors and take their cross
  product to find a normal vector to the plane: If the points are $P$,
  $Q$, and $R$, we take the difference vectors $PQ$ and $PR$ and
  compute their cross product.
\item We now use the vector equation, and hence from that the scalar
  equation, taking any of of the three points $P$, $Q$, or $R$ as the
  basepoint.
\end{itemize}

Note that if the three points given are {\em collinear}, then they do
not define a unique plane. Rather, any plane through the line joining
these three points works. It is no surprise that the above procedure
fails at the stage where we need to take cross product, because the
cross product turns out to be the zero vector.
\subsection{Intersecting two planes: line of intersection}

Given two planes, the typical case is that they intersect in a
line. If we have scalar equations for both planes, then the
intersection line can be described by taking the two equations
together.

Unfortunately, this pair of two equations together, while it does
define a line, is not directly one of the {\em standard} descriptions
of a line.

There are many ways of obtaining the line in standard form. One of
these is as follows: first, find normal vectors to the planes. For
instance, if the equations for the planes are:

\begin{eqnarray*}
  a_1x + b_1y + c_1z + d_1 & = & 0\\
  a_2x + b_2y + c_2z + d_2 & = & 0\\
\end{eqnarray*}

Then the normal vectors to these planes are $\langle
a_1,b_1,c_1\rangle$ and $\langle a_2,b_2,c_2\rangle$. A direction
vector along the line of intersection must be perpendicular to {\em
both} these normal vectors, hence, it must be in the line of the {\em
cross product}. Hence, we take the cross product $\langle a_1,b_1,c_1
\rangle \times \langle a_2,b_2,c_2 \rangle$.

Now that we've found the direction vector along the intersection of
these planes, we need to find just one point along the intersection
and we can then use the point-direction form. One way of finding a
point is to set $z = 0$ in both equations and solve the system for $x$
and $y$ (this is assuming that neither is parallel to the $xy$-plane;
otherwise choose some other coordinate).

Note that if the planes are parallel or coincide, then their normal
vectors are parallel and thus the cross product of the normal vectors
becomes zero. Conversely, the cross product becoming zero means the
planes are parallel, so there is a good reason for the line of
intersection to not make sense.

\subsection{Intersecting two planes: angle of intersection}

The {\em angle of intersection} between two planes is the angle of
intersection between their normal vectors. As for the line of
intersection, we can extract the normal vector from the scalar
equation of the planes. To compute the angle of intersection, we use
the formula as arc cosine of the quotient of the dot product by the
product of the lengths.

\subsection{Intersecting a plane and a line}

Given a plane and a line, we can intersect them as follows: If the
plane is given by a scalar equation and the line is given
parametrically using a parameter $t$, then to compute the
intersection, we plug in all coordinates as functions of the parameter
into the scalar equation for the plane, and solve one equation in the
one variable $t$. After finding the solution $t$, we plug this into
the parametric equation of the line to find the coordinates of the
point of intersection.

There are three possibilities:

\begin{itemize}
\item The typical case is that we have a linear equation in one
  variable, and it has a unique solution. In other words, the plane
  and line intersect at a point.
\item Another case is that the equation simplifies to something
  nonsensical, such as $0 = 1$. In this case, there is no
  intersection. Geometrically, this means the line is parallel to but
  not on the plane.
\item The final case is that the equation simplifies to a tautology,
  such as $0 = 0$. In this case, all real $t$ give
  solutions. Geometrically, this means that the line is on the plane.
\end{itemize}

\subsection{Distance of a point from a plane}

We will not have much occasion to use this formula, but we note it
briefly nonetheless. Given a point with coordinates $(x_1,y_1,z_1)$
and a plane $ax + by + cz + d = 0$, the distance from the point to the
plane is given by the formula:

$$\frac{|ax_1 + by_1 + cz_1 + d|}{\sqrt{a^2 + b^2 + c^2}}$$

\end{document}
