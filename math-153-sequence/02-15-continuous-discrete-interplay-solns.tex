\documentclass[10pt]{amsart}

%Packages in use
\usepackage{fullpage, hyperref, vipul, enumerate}

%Title details
\title{Take-home class quiz solutions: due February 15: Interplay of continuous and discrete}
\author{Math 153, Section 55 (Vipul Naik)}
%List of new commands

\begin{document}
\maketitle

\section{Performance review}

$11$ people took this quiz. The score distribution was as follows:

\begin{itemize}
\item Score of $4$: $1$ person
\item Score of $5$: $1$ person
\item Score of $6$: $7$ people
\item Score of $7$: $2$ people
\end{itemize}

The mean score was $5.91$. The question wise answers and performance
review were as follows:

\begin{enumerate}
\item Option (C): $11$ people (everybody)
\item Option (D): $11$ people (everybody)
\item Option (B): $9$ people
\item Option (B): $2$ people (the groupthink folk all got it wrong!)
\item Option (D): $2$ people (the groupthink folk all got it wrong!)
\item Option (E): $11$ people (everybody)
\item Option (D): $9$ people
\item Option (D): $10$ people
\end{enumerate}

\section{Solutions}
\begin{enumerate}

\item Consider a function $f$ defined on all real numbers. Consider
  also the sequence $a_n = f(n)$ defined for $n$ a natural
  number. Which of the following is true?

  \begin{enumerate}[(A)]
  \item $\lim_{x \to \infty} f(x)$ is finite if and only if $\lim_{n
    \to \infty} a_n$ is finite, and if so, both limits are equal.
  \item $\lim_{x \to \infty} f(x)$ is finite if and only if $\lim_{n
    \to \infty} a_n$ is finite, but the limits need not be equal.
  \item If $\lim_{x \to \infty} f(x)$ is finite, then $\lim_{n \to
    \infty} a_n$ is finite, but the converse is not true. Moreover, if
    both limits are finite, they must be equal.
  \item If $\lim_{n \to \infty} a_n$ is finite, then $\lim_{x \to
    \infty} f(x)$ is finite, but the converse is not true. Moreover, if
    both limits are finite, they must be equal.
  \item It is possible for either of the limits $\lim_{x \to \infty}
    f(x)$ and $\lim_{n \to \infty} a_n$ to be finite, but for the
    other one not to be finite. Moreover, even if both limits exist,
    they need not be equal.
  \end{enumerate}

  {\em Answer}: Option (C)

  {\em Explanation}: The key idea is that the values at the natural
  numbers only form a part of the behavior of the function. If the
  function as a whole has a finite limit $L$ at infinity, then that
  means that for every $\epsilon$ there exists a value of $A$ such
  that $|f(x) - L| < \epsilon$ for all real $x > A$.

  This in turn forces that all the values that the function takes at
  {\em integers} bigger than $A$ is also within $\epsilon$-distance of
  $L$. Thus, $\lim_{n \to \infty} a_n = L$.

  The converse is not true because the function outside of the
  integers could behave in a completely different way. For instance,
  take $f(x) =\sin(\pi x)$. We get $a_n = 0$ for all $n$. $\lim_{n \to
  \infty} a_n = 0$ but $\lim_{x \to \infty} f(x)$ does not exist.

  See the lecture notes on the interplay between continuous and
  discrete.

  {\em Performance review}: All $11$ people got this.

\item Consider a function $f:\R \to \R$. Restricting the
  domain of $f$ to the natural numbers, obtain a sequence whose
  $n^{th}$ member $a_n$ is defined as $f(n)$. Which of the following
  statements is {\bf false} about the relationship between $f$ and the
  sequence $(a_n)$?

  \begin{enumerate}[(A)]
  \item If $f$ is an increasing function, then $(a_n)$ form an
    increasing sequence.
  \item If $f$ is a decreasing function, then $(a_n)$ form a
    decreasing sequence.
  \item If $f$ is a bounded function, (i.e., its range is a bounded
    set) then $(a_n)$ form a bounded sequence.
  \item If $f$ is a periodic function, then $(a_n)$ form a periodic sequence.
  \item If $f$ has a limit at infinity, then $(a_n)$ is a convergent
    sequence.
  \end{enumerate}

  {\em Answer}: Option (D)

  {\em Explanation}: (A), (B), (C), and (E) are immediately true (see
  the lecture notes for more information). As for option (D), the
  problem with it is that $f$ may not have an {\em integer} period
  even though it is periodic. For instance, if we set $f = \sin$, then
  $f$ is periodic, but its period is $2\pi$ which has no multiple that
  is an integer, on account of $\pi$ being irrational.

  {\em Performance review}: All $11$ people got this.

\item We are given a sequence $a_1, a_2, \dots, a_n, \dots$ of real
  numbers. The goal is to find a {\em continuous} function $f$ on all
  of $\R$ such that $f(n) = a_n$ for all $n \in \N$. Which of the
  following is true?

  \begin{enumerate}[(A)]
  \item There is a unique choice of $f$ that works.
  \item There exist infinitely many different choices of $f$ that work.
  \item The number of possible choices of $f$ depends on the
    sequence. Depending on the sequence, the number of possible
    choices of $f$ may be zero, one, or infinite.
  \item The number of possible choices of $f$ depends on the
    sequence. Depending on the sequence, the number of possible
    choices of $f$ may be zero or one. It can never be infinite.
  \item The number of possible choices of $f$ depends on the
    sequence. Depending on the sequence, the number of possible
    choices of $f$ may be one or infinite. It can never be zero.
  \end{enumerate}

  {\em Answer}: Option (B)

  {\em Explanation}: Imagine the graph of the sequence, i.e., we plot
  the points $(n,a_n)$ in the coordinate plane for all $n \in \N$. The
  goal is to find a continuous function whose graph passes through all
  these points. We could do this in many ways. For instance, for each
  pair of adjacent points, we could join them up by a line segment or
  some other continuous curve. And to the left of $1$ we could do any
  of a number of things.

  {\em Performance review}: $9$ out of $11$ got this. $1$ chose (C),
  $1$ chose (E).
\item We are given a sequence $a_1, a_2, \dots, a_n, \dots$ of real
  numbers. The goal is to find an {\em infinitely differentiable}
  function $f$ on all of $\R$ such that $f(n) = a_n$ for all $n \in
  \N$. Which of the following is true?

  \begin{enumerate}[(A)]
  \item There is a unique choice of $f$ that works.
  \item There exist infinitely many different choices of $f$ that work.
  \item The number of possible choices of $f$ depends on the
    sequence. Depending on the sequence, the number of possible
    choices of $f$ may be zero, one, or infinite.
  \item The number of possible choices of $f$ depends on the
    sequence. Depending on the sequence, the number of possible
    choices of $f$ may be zero or one. It can never be infinite.
  \item The number of possible choices of $f$ depends on the
    sequence. Depending on the sequence, the number of possible
    choices of $f$ may be one or infinite. It can never be zero.
  \end{enumerate}

  {\em Answer}: Option (B)

  {\em Explanation}: The reasoning is similar to the previous problem,
  albeit there is more subtletly to this one. Stay tuned for more on
  this later in the course!

  {\em Performance review}: $2$ out of $11$ got this. $8$ chose (A),
  $1$ chose (E).

  {\em Action point}: Avoid the perils of groupthink. Follow your
  conscience.
\item We are given a sequence $a_1, a_2, \dots, a_n, \dots$ of real
  numbers. The goal is to find a {\em polynomial}
  function $f$ on all of $\R$ such that $f(n) = a_n$ for all $n \in
  \N$. Which of the following is true?

  \begin{enumerate}[(A)]
  \item There is a unique choice of $f$ that works.
  \item There exist infinitely many different choices of $f$ that work.
  \item The number of possible choices of $f$ depends on the
    sequence. Depending on the sequence, the number of possible
    choices of $f$ may be zero, one, or infinite.
  \item The number of possible choices of $f$ depends on the
    sequence. Depending on the sequence, the number of possible
    choices of $f$ may be zero or one. It can never be infinite.
  \item The number of possible choices of $f$ depends on the
    sequence. Depending on the sequence, the number of possible
    choices of $f$ may be one or infinite. It can never be zero.
  \end{enumerate}

  {\em Answer}: Option (D)

  {\em Explanation}: Imagine that there are two polynomials $f$ and
  $g$ that both satisfy $f(n) = g(n) = a_n$ for all $n \in \N$. Then,
  the polynomial $f - g$ is zero at all $n\in \N$. A polynomial can
  have infinitely many roots only if it is the zero polynomial, so $f
  - g = 0$ and $f = g$.

  This shows that there is {\em at most} one polynomial function
  fitting the sequence. It is, however, possible for there to be no
  polynomial function. For instance, if we take a sequence that grows
  exponentially, such as $a_n = 2^n$, there will be no polynomial
  function fitting it.

  {\em Performance review}: $2$ out of $11$ got this. $7$ chose (B),
  $1$ each chose (A) and (E).

  For the remaining questions: For a function $f:\N \to \R$, define
  $\Delta f$ as the function $n \mapsto f(n+1) -f(n)$. Denote by
  $\Delta^k f$ the function obtained by applying $\Delta$ $k$ times to
  $f$.
\item  If $f(n) = n^2$, what is $(\Delta f)(n)$?

  \begin{enumerate}[(A)]
  \item $1$
  \item $n$
  \item $2n - 1$
  \item $2n$
  \item $2n + 1$
  \end{enumerate}

  {\em Answer}: Option (E)

  {\em Explanation}: We get $f(n + 1) - f(n) = (n + 1)^2 - n^2 = n^2 +
  2n + 1 - n^2 = 2n + 1$.

  {\em Performance review}: All $11$ people got this.

\item If $f$ is expressible as a polynomial function of degree $d >
  0$, what is the smallest $k$ for which $\Delta^k f$ is identically
  the zero function? {\em Hint: Think of the analogous question using
  continuous derivatives. Although $\Delta$ differs from the
  continuous derivative, much of the qualitative behavior is the same.}

  \begin{enumerate}[(A)]
  \item $d - 2$
  \item $d - 1$
  \item $d$
  \item $d + 1$
  \item $d + 2$
  \end{enumerate}

  {\em Answer}: Option (D)

  {\em Explanation}: Every time we apply $\Delta$, the degree of the
  polynomial goes down by one. After $d$ applications to a polynomial
  of degree $d$, we get a constant polynomial. The $(d + 1)^{th}$
  application should therefore yield the zero polynomial.

  {\em Performance review}: $9$ out of $11$ got this. $1$ each chose (B) and (C).

\item If $f$ is a function such that $\Delta f = af$ for some positive
  constant $a$, and $f(1)$ is positive, which of the following best
  describes the nature of growth of $f$? {\em Hint: Think of the
  analogous differential equation using continuous derivatives. The
  precise solution is different but the nature of the solution is
  similar.}

  \begin{enumerate}[(A)]
  \item $f$ grows like a sublinear function of $n$.
  \item $f$ grows like a linear function of $n$.
  \item $f$ grows like a superlinear but subexponential function of
    $n$.
  \item $f$ grows like an exponential function of $n$.
  \item $f$ grows like a superexponential function of $n$.
  \end{enumerate}

  {\em Answer}: Option (D)
  
  {\em Explanation}: The condition tells us that $f(n+1) - f(n) =
  af(n)$, so $f(n+1) = (a+1)f(n)$. Thus, each term is $(a + 1)$ times
  its predecessor. Thus, the sequence grows exponentially, and the
  general term is $f(n) = (a+1)^{n-1}f(1)$.

  {\em Performance review}: $10$ out of $11$ got this. $1$ chose (B).
\end{enumerate}
\end{document}