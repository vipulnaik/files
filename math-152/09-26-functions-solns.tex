\documentclass[10pt]{amsart}

%Packages in use
\usepackage{fullpage, hyperref, vipul, enumerate}

%Title details
\title{Class quiz solutions: September 26; Topic: Functions}
\author{Vipul Naik}
%List of new commands

\begin{document}
\maketitle

\section{Performance review}

$10$ people took this $5$-question quiz. The performance was as
follows:

\begin{enumerate}

\item (A): Everybody got this correct.
\item (E): Everybody got this correct.
\item (B): Everybody got this correct.
\item (B): Everybody got this correct.
\item (A): $1$ person got this incorrect.
\end{enumerate}

There were thus $9$ full scores and $1$ score of $4$.
\section{Solutions}
\begin{enumerate}

\item Consider the function $f(x) := |x + 1| - |x|$. For which of the
  following values of $x$ is $f(x)$ equal to $0$?

  \begin{enumerate}[(A)]
  \item $-\frac{1}{2}$
  \item $-\frac{1}{3}$
  \item $0$
  \item $\frac{1}{3}$
  \item $\frac{1}{2}$
  \end{enumerate}

  {\em Answer}: Option (A)

  {\em Explanation}: When we set $x = -1/2$, we get $f(x) = |(-1/2) +
  1| - |-1/2| = |1/2| - |-1/2|$, which becomes $1/2 - 1/2$, which is
  equal to $0$.

  We can also solve the equation formally, but this is a little
  trickier, and we will get to it at a later stage.

 {\em The other choices}: All the other choices are incorrect:

  Option (B): $f(-1/3) = 2/3 - 1/3 = 1/3$.

  Option (C): $f(0) = 1 - 0 = 1$.

  Option (D): $f(1/3) = 4/3 - 1/3 = 1$.

  Option (E): $f(1/2) = 3/2 - 1/2 = 1$.

  {\em Performance review}: Everybody got it correct.

  {\em Historical note}: When this same quiz question was asked last
  year, everybody got it correct.

\item Consider the function $f(x) := x^2 + 1$. What is the polynomial
  describing $f(f(x))$?

  \begin{enumerate}[(A)]
  \item $x^2 + 2$
  \item $x^4 + x^2 + 1$
  \item $x^4 + x^2 + 2$
  \item $x^4 + 2x^2 + 1$
  \item $x^4 + 2x^2 + 2$
  \end{enumerate}

  {\em Answer}: Option (E)

  {\em Explanation}: We have $f(f(x)) = f(x^2 + 1) = (x^2 + 1)^2 + 1 =
  x^4 + 2x^2 + 1 + 1$, which simplifies to option (E).

  {\em The other choices}:

  Option (A) is $(x^2 + 1) + 1 = x^2 + 2$. The error here is is not
  squaring the $x^2 + 1$ expression.

  Option (D) is $(x^2 + 1)^2 = x^4 + 2x^2 + 1$. The error here is in
  forgetting to add the $1$ at the end.

  Options (B) and (C) are like options (D) and (E), with an error in
  the coefficient of $x^2$.

  {\em Performance review}: Everybody got it correct.

  {\em Historical note}: When this same quiz question was asked last
  year, everybody got it correct.

\item Consider the function $f(x) := \frac{x}{x^2 + 1}$. What is $f(f(1))$?

  \begin{enumerate}[(A)]
  \item $1/5$
  \item $2/5$
  \item $4/5$
  \item $5/4$
  \item $5/8$
  \end{enumerate}

  {\em Answer}: Option (B)

  {\em Explanation}: We have:

  $$f(1) = \frac{1}{1^2 + 1} = \frac{1}{2}$$

  Thus, $f(f(1)) = f(1/2)$, and we get:

  $$f(1/2) = \frac{1/2}{(1/2)^2 + 1} = \frac{1/2}{5/4} = \frac{1}{2} \cdot \frac{4}{5} = \frac{2}{5}$$

  {\em Performance review}: Everybody got it correct.

  {\em Historical note}: When the same question appeared last year, $1$
  person chose (A), everybody else got this correct.

\item Consider the function $f(x) := x + 1$. What is $f(f(x))$?

  \begin{enumerate}[(A)]
  \item $x$
  \item $x + 2$
  \item $2x + 1$
  \item $(x + 1)^2$
  \item $x^2 + 1$
  \end{enumerate}

  {\em Answer}: Option (B)

  {\em Explanation}: We have $f(f(x)) = f(x + 1) = (x + 1) + 1 = x +
  2$.

  {\em Performance review}: Everybody got it correct.

  {\em Historical note}: When this same quiz question was asked last
  year, everybody got it correct.


\item If a circle has radius $r$, the area of the circle is $\pi
  r^2$. What is the area of a circle with diameter $d$?

  \begin{enumerate}[(A)]
  \item $\pi d^2/4$
  \item $\pi d^2/2$
  \item $\pi d^2$
  \item $2\pi d^2$
  \item $4\pi d^2$
  \end{enumerate}

  {\em Answer}: Option (A)

  {\em Explanation}: The diameter is twice the radius, so the radius
  is half the diameter, i.e., $r = d/2$. Plugging this in, we get that the area is:

  $$\pi r^2 = \pi (d/2)^2 = \pi d^2/4$$

  {\em The other choices}: Option (B) is the best distractor. It could
  arise if we forget to square the $2$ in the denominator in the above
  calculation.

  The other options could arise through erroneous starting assumptions
  such as $r = d$ or $r = 2d$.

  {\em Performance review}: $1$ person chose (B), the best
  distractor. Everybody else got it correct.

  {\em Historical note}: When this same quiz question was asked last
  year, everybody got it correct.

\end{enumerate}
\end{document}