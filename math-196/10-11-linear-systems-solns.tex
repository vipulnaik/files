\documentclass[10pt]{amsart}

%Packages in use
\usepackage{fullpage, hyperref, vipul, enumerate}

%Title details
\title{Take-home class quiz solutions: due Friday October 11: Linear systems}
\author{Math 196, Section 57 (Vipul Naik)}
%List of new commands

\begin{document}
\maketitle

\section{Performance review}

29 people took this $10$-question quiz. The score distribution was
as follows:

\begin{itemize}
\item Score of 2: 1 person
\item Score of 3: 2 people
\item Score of 4: 2 people
\item Score of 5: 4 people
\item Score of 6: 3 people
\item Score of 7: 7 people
\item Score of 8: 3 people
\item Score of 9: 3 people
\item Score of 10: 4 people
\end{itemize}

The mean score was 6.69.

The question-wise answers and performance review were as follows:

\begin{enumerate}
\item Option (B): 12 people
\item Option (A): 20 people
\item Option (B): 17 people
\item Option (C): 15 people
\item Option (D): 19 people
\item Option (D): 19 people
\item Option (B): 27 people
\item Option (A): 23 people
\item Option (C): 22 people
\item Option (B): 20 people
\end{enumerate}

\section{Solutions}

The quiz questions here, although not hard {\em per se}, are
conceptually demanding: answering them requires a clear understanding
of multiple concepts and an ability to execute them
conjunctively. Even if you feel that you've understood the material as
presented in class, you will need to think through each question
carefully. Some of the questions are related to similar homework
problems (Homeworks 1 and 2), and they test a conceptual understanding
of the solutions to these problems. You might want to view them in
conjunction with the homework problems. Other questions sow the seeds
of ideas we will explore later. The quiz should seem relatively easier
when you review it later, assuming that you work hard on attempting
the questions right now and read the solutions once they're put up.

\begin{enumerate}
\item (*) Rashid and Riena are trying to study a function $f$ of two
  variables $x$ and $y$. Rashid is convinced that the function is
  linear (i.e., it is of the form $f(x,y) := ax + by + c$) but has no
  idea what $a$, $b$, and $c$ are. Riena thinks a linear model is
  completely out-of-place. Rashid is eager to find $a$, $b$, and $c$,
  whereas Riena is eager to disprove Rashid's linear
  model. Unfortunately, all they have is a black box that will output
  the value of the function for a given input pair $(x,y)$, and that
  black box can only be called three times. What should Rashid and
  Riena try for?

  \begin{enumerate}[(A)]
  \item Rashid and Riena would both like to provide three input pairs that
    are non-collinear as points in the $xy$-plane
  \item Rashid would like to provide three input pairs that are
    non-collinear, while Riena would like to provide three input pairs
    that are collinear as points in the $xy$-plane.
  \item Rashid and Riena would both like to provide three input pairs
    that are collinear as points in the $xy$-plane.
  \item Rashid would like to provide three input pairs that are
    collinear, while Riena would like to provide three input pairs
    that are non-collinear as points in the $xy$-plane.
  \item Both Rashid and Riena are indifferent regarding how the three
    input pairs are picked.
  \end{enumerate}

  {\em Answer}: Option (B)

  {\em Explanation}: Rashid wants to pick three independent
  constraints, i.e., pick three linear equations that are independent
  of each other. This would give him three independent equations in
  three variables, allowing him to uniquely determine the values of
  the parameters $a$, $b$, and $c$. Thus, he wishes to pick three
  non-collinear points. If he picked three collinear points, one of
  his equations would be redundant (deducible from the other two).

  Riena, on the other hand, wants to pick three points with some
  redundancy between them. If she picks three collinear points, then,
  assuming Rashid's model is correct, the values of the outputs at two
  of the three points determine the value of the output at the
  third. If it turns out that the actual value differs from the
  predicted value, this {\em falsifies} Rashid's model.

  The reason Riena wants them to be collinear is thus the same as the
  reason Rashid wants them to be non-collinear. Rashid already thinks
  he knows the predicted result from the collinear case, so he is not
  interested in it. Riena wants to show Rashid that his prediction is
  wrong, so she cares about it. The non-collinear case would not be
  interesting to Riena because it offers no evidence against Rashid.

  Note that if $4$ or more inputs can be queried, then both Rashid's
  and Riena's concerns can be addressed. {\em Scarcity breeds
    conflict}.

  More about this is discussed in the notes on {\tt Hypothesis
    testing, rank, and overdetermination} that we intend to discuss
  next week.

  {\em Not clear to you?}: Pick some function of two variables that is
  not linear, and put yourself in Riena's shoes. How would you
  convinced Rashid that the function is not linear? What types of
  inputs would you choose? Similarly, put yourself in Rashid's shoes
  and try to determine what types of inputs will help you determine
  uniqueness.

  {\em Performance review}: 12 out of 29 got this. 7 each chose (A)
  and (D), 2 chose (E), 1 chose (C).

  {\em Historical note (last time)}: $18$ out of $29$ got this. $6$ chose (D),
  $4$ chose (C), $1$ chose (A).

\item (*) Let $m$ and $n$ be natural numbers with $m \ge 3$. We are
  given a bunch of numbers $x_1< x_2< \dots<x_m$ and another bunch of
  numbers $y_1,y_2,\dots,y_m$. We want to find a continuous function
  $f$ on $[x_1,x_m]$, such that $f(x_i) = y_i$ for all $1 \le i \le
  m$, and such that the restriction of $f$ to any interval of the form
  $[x_i,x_{i+1}]$ (for $1 \le i \le m - 1$) is a polynomial of degree
  $\le n$. What is the smallest value of $n$ for which we are
  guaranteed to be able to find such a function $f$?

  \begin{enumerate}[(A)]
  \item $1$
  \item $2$
  \item $3$
  \item $4$
  \item $5$
  \end{enumerate}

  {\em Answer}: Option (A)

  {\em Explanation}: The brief explanation is that we can construct a
  piecewise linear function: a function whose graph comprises straight
  line segments joining $(x_i,y_i)$ to $(x_{i+1},y_{i+1})$.

  The more sophisticated explanation is that for each pair of adjacent
  points, we have to choose a functional form between the two points,
  and there are two input-output pairs constraining the functional
  form (namely, the two endpoints). We thus need a functional form
  with two parameters. The linear functional form works well for that
  purpose. If we were to count the {\em total} number of parameters,
  it would work out to $2(m - 1)$ ($2$ parameters for each of the $m -
  1$ intervals $(x_1,x_2)$, $(x_2,x_3)$, $\dots$,
  $(x_{m-1},x_m)$). The number of equations would also work out to
  $2(m - 1)$. The number of parameters equals the number of
  equations. However, the equality of the number of parameters and the
  number of equations in the abstract is not {\em completely
    sufficient} to argue that we can always find a solution, because
  of concerns about redundant equations. In this case, it is
  sufficient, based on the explanation offered earlier. It is still a
  good working heuristic, though.

  {\em Note on subtle difference in goals from previous situations}:
  The situation of this question, and the next three questions, is
  different in a subtle but important way from earlier
  situations. Broadly, consider two types of situations.

  \begin{enumerate}
  \item {\em Prediction/forecasting/extrapolation situation}: The
    situation where we are trying to model some pre-existing
    phenomenon and we have a model with a general functional form
    (involving not-yet-known parameters) that we believe applies to
    the phenomenon. We are trying to find the parameters using
    input-output pairs. In this case, we are trying to find
    information about a function that, in some sense, already
    exists. This means that out job isn't just to find {\em some
      function that fits the known data}, but to find {\em the right
      function}. In this situation, having more input-output pairs
    (and hence equations) than parameters (that become our new
    variables) is desirable.
  \item {\em Engineering solution situation}: The situation where our
    goal is just to find some function that fits the existing data,
    rather than to find a particular function. In this case, having
    more variables than equations, resulting in the system being
    underdetermined (i.e., the solution being non-unique and having
    degrees of freedom) is a {\em good} thing.
  \end{enumerate}

  The type of situation we are dealing with in these questions is type
  (b). We are interested in just finding some function satisfying
  certain constraints, not in finding a specific function that already
  exists.

  Note that all situations that involve prediction and trend
  forecasting fall under (a) rather than (b). Situation (b) arises in
  cases where we are solving specific engineering problems to come up
  with one-time constructs that satisfy constraints (such as the
  roller-coaster ride example in the book). From the social science
  perspective, the more common situation is situation (a), where we
  are trying to predict or forecast for an existing model that we do
  not yet fully understand.

  {\em Not clear to you?}: Make a picture with points marked for the
  values of the function at $x_1,x_2,\dots,x_m$. Now notice that we
  can make a graph that uses a straight line segment for each interval
  between adjacent $x_i$s.

  {\em Performance review}: 20 out of 29 people got this. 8 chose (B),
  1 chose (C).

  {\em Historical note (last time)}: $17$ out of $29$ got this. $10$ chose (C),
  $2$ chose (D).

\item (*) Let $m$ and $n$ be natural numbers with $m \ge 3$. We are
  given a bunch of numbers $x_1< x_2< \dots<x_m$ and another bunch of
  numbers $y_1,y_2,\dots,y_m$. We want to find a continuous function
  $f$ on $[x_1,x_m]$, such that $f(x_i) = y_i$ for all $1 \le i \le
  m$, and such that the restriction of $f$ to any interval of the form
  $[x_i,x_{i+1}]$ (for $1 \le i \le m - 1$) is a polynomial of degree
  $\le n$. In addition, we want to make sure that $f$ is
  differentiable on the open interval $(x_1,x_m)$. What is the
  smallest value of $n$ for which we are guaranteed to be able to find
  such a function $f$?

  \begin{enumerate}[(A)]
  \item $1$
  \item $2$
  \item $3$
  \item $4$
  \item $5$
  \end{enumerate}

  {\em Answer}: Option (B)

  {\em Explanation}: We can take the first piece to be linear (simply
  join the points $(x_1,y_1)$ and $(x_2,y_2)$). For each subsequent
  piece, we have three constraints. For the $i^{th}$ piece, the
  constraints include the value at $x_i$, the value at $x_{i+1}$, and
  the derivative at $x_i$ (which must equal the corresponding
  derivative from the preceding piece). In order to be able to fit all
  three constraints, we need to use $n = 2$ so as to have three
  parameters. It is also straightforward to check that the system we
  get this way is not redundant.

  Here is the accounting regarding parameters and equations. The
  number of equations is $2(m - 1) + (m - 2)$. The $2(m - 1)$
  equations arise from the values at the endpoints of each of the $m -
  1$ intervals. The extra $m - 2$ equations arise from the equality of
  the formal expressions for the left hand derivative and right hand
  derivative at each of the $m-2$ transition points $x_2,x_3,\dots,
  x_{m-1}$. The total number of equations is therefore $3m -
  4$. Whatever value of $n$ we choose, the total number of parameters
  is $(n + 1)(m - 1)$ (we need $n + 1$ parameters for the piece
  definition on each of the $m - 1$ intervals $(x_1,x_2)$,
  $(x_2,x_3)$, $\dots$, $(x_{m-1},x_m)$). We want to choose $n$ such
  that the number of parameters is greater than or equal to the number
  of equations, and the smallest value that works is $n = 2$, in which
  case we get $(n + 1)(m - 1) = 3m - 3 \ge 3m - 4$. Note that we have
  one extra parameter over the number of equations, so we have a bit
  of ``slack'' here, at least in principle. Again, the crude
  comparison of parameters and equations is not conclusive because of
  the potential for redundancy, so we need an explanation of the sort
  offered in the preceding paragraph to be sure. Note that the slight
  excess of parameters over equations corresponds to the fact that for
  the piece definition in the very first interval $(x_1,x_2)$, we have
  some flexibility.

  {\em Note on subtle difference in goals from previous situations}:
  See the note at the end of the answer to Question 2.

  {\em Not clear to you?}: Make a picture similar to the preceding
  question, but notice now that from the second interval onward, you
  have a constraint on the initial slope.

  {\em Performance review}: 17 out of 29 got this. 10 chose (C), 2 chose (A).

  {\em Historical note (last time)}: $25$ out of $29$ got this. $2$
  chose (C), $1$ each chose (A) and (D).
\item (*) Let $m$ and $n$ be natural numbers with $m \ge 3$. We are given
  a bunch of numbers $x_1< x_2< \dots<x_m$ and another bunch of
  numbers $y_1,y_2,\dots,y_m$. We want to find a continuous function
  $f$ on $[x_1,x_m]$, such that $f(x_i) = y_i$ for all $1 \le i \le
  m$, and such that the restriction of $f$ to any interval of the form
  $[x_i,x_{i+1}]$ (for $1 \le i \le m - 1$) is a polynomial of degree
  $\le n$. In addition, we want to make sure that $f$ is
  differentiable on the open interval $(x_1,x_m)$. In addition, we are
  told the value of the right hand derivative of $f$ at $x_1$ and the
  left hand derivative of $f$ at $x_m$. What is the smallest value of
  $n$ for which we are guaranteed to be able to find such a function
  $f$?

  \begin{enumerate}[(A)]
  \item $1$
  \item $2$
  \item $3$
  \item $4$
  \item $5$
  \end{enumerate}

  {\em Answer}: Option (C)

  {\em Explanation}: This is similar to the cubic spline question on
  your advanced homework.

  The key reason why this differs from the preceding question is that,
  here, we have a specification of the one-sided derivatives at the
  endpoint. The specification at $x_1$ constrains how we start, but
  that is stil acceptable. We'll need something quadratic to begin
  with, but we can keep choosing quadratics all the way, since at each
  piece we need to satisfy three conditions. The problem occurs in the
  final stage, i.e., the interval $[x_{m-1},x_m]$. Here, we have {\em
    four} constraints: the values at the endpoints, and the
  derivatives at both endpoints. So, we need a functional form with
  four parameters. In other words, we need a cubic for the last piece.

  It would be acceptable to use quadratics for all except the last
  piece, and a cubic for the last piece. The options available to us,
  however, require us to declare a single degree that works for all,
  so the smallest available is $3$.

  Using the accounting for parameters and equations similar to the
  preceding question, we obtain $3m - 2$ equations (the $3m - 4$
  equations of the preceding question, plus $2$ additional equations
  for the one-sided derivative conditions at the endpoints). We
  therefore need to choose $n$ such that $(n + 1)(m - 1) \ge 3m -
  2$. The value $n = 2$ {\em just} falls short: it would give $3m -3$
  parameters, and this corresponds to the fact that if we tried to fit
  a quadratic, we could run into trouble in the very last piece. The
  value $n = 3$, on the other hand, offers us considerable slack.

  {\em Note on subtle difference in goals from previous situations}:
  See the note at the end of the answer to Question 2.

  {\em Performance review}: 15 out of 29 got this. 7 chose (B), 3
  chose (A), 2 chose (D), 1 chose (E), 1 left the question blank.

  {\em Historical note (last time)}: $11$ out
  of $29$ got this. $10$ chose (B), $6$ chose (D), $2$ chose (A).

\item (*) Let $k$, $m$, and $n$ be natural numbers with $m \ge 3$. We are
  given a bunch of numbers $x_1< x_2< \dots<x_m$ and another bunch of
  numbers $y_1,y_2,\dots,y_m$. We want to find a continuous function
  $f$ on $[x_1,x_m]$, such that $f(x_i) = y_i$ for all $1 \le i \le
  m$, and such that the restriction of $f$ to any interval of the form
  $[x_i,x_{i+1}]$ (for $1 \le i \le m - 1$) is a polynomial of degree
  $\le n$. In addition, we want to make sure that $f$ is at least $k$
  times differentiable on the open interval $(x_1,x_m)$. What is the
  smallest value of $n$ for which we are guaranteed to be able to find
  such a function $f$?

  \begin{enumerate}[(A)]
  \item $k - 2$
  \item $k - 1$
  \item $k$
  \item $k + 1$
  \item $k + 2$
  \end{enumerate}

  {\em Answer}: Option (D)

  {\em Explanation}: For the part $[x_1,x_2]$, we can simply fit a
  straight line. For each subsequent piece, we have $k + 1$
  constraints: the left and right endpoint values, and all the
  derivative values up to the $k^{th}$ derivative at the left
  endpoint. This gives a total of $k + 2$ constraints. A polynomial of
  degree $k + 1$ has the necessary number of parameters, namely $k +
  2$. We can also use Taylor polynomials to see that the system can
  always be solved.

  If we account for the total number of parameters and equations, the
  numbers we get are as follows. We have $2(m - 1) + k(m - 2)$
  equations. The $2(m - 1)$ equations are from the values at the
  endpoins of the $m - 1$ intervals. The $k(m - 2)$ equations are from
  the equality of the first $k$ derivatives of the piece functions at
  the transition points $x_2,x_3,\dots,x_{m-1}$. The total number of
  equations is $(k + 2)(m - 1) - k$. We have $(n + 1)(m - 1)$
  parameters. Therefore, choosing $n = k + 1$ gives more parameters
  than equations. On the other hand, choosing $n = k$ gives $(k + 1)(m
  - 1)$ parameters, so that the number of parameters - the number of
  equations is $k - (m - 1)$. Since we are not given any concrete
  information about the sign relationship between $k$ and $m - 1$, we
  cannot be sure that this will work.

  {\em Note on subtle difference in goals from previous situations}:
  See the note at the end of the answer to Question 2.

  {\em Performance review}: 19 out of 29 got this. 5 chose (B), 3
  chose (E), 2 chose (C).

  {\em Historical note (last time)}: $12$ out of $29$ got this. $14$
  chose (B), $2$ chose (C), $1$ chose (A).

  \vspace{0.1in}

  The next few questions are framed deterministically, though similar
  real-world applications would be probabilistic, with some square
  roots floating around. Unfortunately, we do not have the tools yet
  to deal with the probabilistic versions of the questions.

\item (*) A function $f$ of one variable is known to be linear. We know
  that $f(1) = 1.5 \pm 0.5$ and $f(2) = 2.5 \pm 0.5$. Assume these are
  the full ranges, without any probability distribution known. Assuming
  nothing is known about how the measurement errors for $f$ at
  different points are related, what can we say about $f(3)$?

  \begin{enumerate}[(A)]
  \item $f(3) = 3.5$ (exactly)
  \item $f(3) = 3.5 \pm 0.5$
  \item $f(3) = 3.5 \pm 1$
  \item $f(3) = 3.5 \pm 1.5$
  \item $f(3) = 3.5 \pm 2.5$
  \end{enumerate}

  {\em Answer}: Option (D)

  {\em Explanation}: The highest value occurs if we take $f(1) = 1$
  and $f(2) = 3$, giving $f(3) = 5$. The lowest value occurs if we
  take $f(1) = f(2) = 2$, giving $f(3) = 2$. Overall, $f(3)$ is
  between $2$ and $5$, so $3.5 \pm 1.5$ is a reasonable description.

  {\em Graphical interpretation}: Make a picture of the coordinate
  $xy$-plane with a vertical line segment joining the points with
  coordinates $(1,1)$ and $(1,2)$ (depicting the range of
  possibilities for $f(1)$) and another line segment joining the
  points with coordinates $(2,2)$ and $(2,3)$ (depicting the range of
  possibilities for $f(2)$). The actual value of $f(1)$ could be
  anywhere on the first line segment and the actual value of $f(2)$
  could be anywhere on the second line segment. The slowest growth
  case is the case where $f(1) = 2$ and $f(2) = 2$, so in fact we get
  a constant function $f(x) = 2$ in this case, and the graph of this
  is the line $y = 2$, and $f(3) = 2$ in this case.. The fastest
  growth case is the case $f(1) = 1$ and $f(2) = 3$, and we get the
  function $f(x) = 2x - 1$ in this case, and the graph is the line $y
  = 2x - 1$. We get $f(3) = 5$ in this case. Thus, $f(3)$ could be any
  value between $2$ and $5$, or equivalently, the range of values for
  $f(3)$ is given by the line segment in the $xy$-plane joining the
  points $(3,2)$ and $(3,5)$.

  The picture might remind you of eclipses. The region in the
  ``shadow'' so to speak is the penumbral region of the eclipse.

  {\em Performance review}: 19 out of 29 got this. 6 chose (B), 3
  chose (C), 1 chose (A).

  {\em Historical note (last time)}: $25$ out of $29$ got this. $3$ chose (C),
  $1$ chose (B).
\item (*) A function $f$ of one variable is known to be linear. We know
  that $f(1) = 1.5 \pm 0.5$ and $f(2) = 2.5 \pm 0.5$. Assume these are
  the full ranges, without any probability distribution known. Assume
  also that the measurement error for $f$ at all points is the same in
  magnitude and sign. What can we say about $f(3)$?

  \begin{enumerate}[(A)]
  \item $f(3) = 3.5$ (exactly)
  \item $f(3) = 3.5 \pm 0.5$
  \item $f(3) = 3.5 \pm 1$
  \item $f(3) = 3.5 \pm 1.5$
  \item $f(3) = 3.5 \pm 2.5$
  \end{enumerate}

  {\em Answer}: Option (B)

  {\em Explanation}: $f(1) = 1.5 + m$ where $m$ is measurement error
  with $|m| \le 0.5$. Since all measurement errors are equal, $f(2) =
  2.5 + m$. Thus, $f(3) = 3.5 + m$, with $|m| \le 0.5$, giving the
  answer.

  {\em Graphical interpretation}: The vertical line segments are the
  same as before: one joins $(1,1)$ and $(1,2)$, the other joins
  $(2,2)$ and $(2,3)$. However, since we now know that the errors are
  the same, the lower bounding line connects the lowest ends of both
  vertical line segments, and the upper bounding line connects the
  upper ends of both vertical line segments. In other words, we get a
  pair of parallel lines. The lower line passes through $(3,3)$ and
  the upper line passes through $(3,4)$. The range of possibilities
  for $f(3)$ is therefore the set of values between $3$ and $4$, i.e.,
  it is $3.5 \pm 0.5$.

  {\em Performance review}: 27 out of 29 got this. 2 chose (C).
 
  {\em Historical note (last time)}: $26$ out of $29$ got this. $2$ chose (D),
  $1$ chose (C).
\item (*) Suppose $f$ is a linear function on a bounded interval $[a,b]$
  but our measurement of outputs for given inputs has some measurement
  error (with the range of measurement error the same regardless of
  the input, and no known correlation between the magnitude of
  measurement error at different points). Assume we can get the
  outputs for any two specified inputs we desire, and we will then fit
  a line through the (input,output) pairs to get the graph of $f$. How
  should we choose our inputs?

  \begin{enumerate}[(A)]
  \item Choose the inputs as far as possible from each other, i.e.,
    choose them as $a$ and $b$.
  \item Choose the inputs to be as close to each other as possible,
    i.e., choose them to be nearby points but not equal to each other.
  \item It does not matter. Any choice of two distinct inputs is good
    enough.
  \end{enumerate}

  {\em Answer}: Option (A)

  {\em Explanation}: If the inputs are chosen close together, then
  even small errors in the input can cause large errors in the
  measurement of the slope. The error in slope is the signed
  difference of measurement errors divided by the distance between the
  inputs. The former is beyond our control, because we noted that the
  magnitude and sign of measurement error does not depend on where we
  choose the inputs. Thus, choosing the inputs as far as possible
  brings us a larger denominator, and therefore keeps the slope error
  at a minimum.

  {\em Performance review}: 23 out of 29 got this. 4 chose (B), 2
  chose (C).

  {\em Historical note (last time)}: $11$ out of $29$ got this. $15$ chose (C),
  $3$ chose (B).
\item (*) $f$ is a function of one variable defined on an interval
  $[a,b]$. You are trying to find an explicit function that fits $f$
  well. You initially try a straight line fit that works at the points
  $a$ and $b$. It turns out that this fit systematically overestimates
  $f$ for points in between (i.e., the actual function $f$ is below
  the linear function) with the maximum magnitude of discrepancy
  occurring at the midpoint $(a + b)/2$. Based on this information,
  what kind of fit should you try to look for?

  \begin{enumerate}[(A)]
  \item Try to fit $f$ using a logarithmic function
  \item Try to fit $f$ using an exponential function
  \item Try to fit $f$ using a quadratic function
  \item Try to fit $f$ using a polynomial of degree at most $3$
  \item Try to fit $f$ using the reciprocal of a linear function
  \end{enumerate}

  {\em Answer}: Option (C)

  {\em Explanation}: Let $L$ be the linear function obtained and let
  $g = f - L$. $g$ is zero at both endpoints $a$ and $b$, below zero
  in between, and has its minimum at the midpoint. These
  characteristics strongly suggest that $g$ is a quadratic function
  with positive leading coefficient. Any quadratic function with
  positive leading coefficient that has zeros has its absolute minimum
  precisely at the midpoint between its zeros.

  Since $g$ is expected to be quadratic, $f = g + L$ is also expected to be
  quadratic.
  
  Note that we could try fitting using a polynomial of degree at most
  $3$. This, however, might run us into overfitting problems. As a
  general rule, we should try to fit using a function with as few
  parameters as possible and where the functional form is justified by
  broad theoretical considerations. If after fitting the quadratic, we
  discover some systemic errors that are best explained by a cubic
  type of discrepancy, we could then try a cubic.

  {\em Performance review}: 22 out of 29 got this. 3 each chose (B)
  and (D), 1 chose (A).

  {\em Historical note (last time)}: $25$ out of $29$ got this. $3$ chose (B),
  $1$ chose (D).
\item (*) Recall the Leontief input-output model. Recall that the GDP is
  defined as the total money value of all the {\em final} goods and
  services produced in the economy, which in this case means only
  those that go into meeting consumer demand, not interindustry demand
  (note that we are assuming away the existence of investment and
  government spending, which complicate the GDP calculation). Assuming
  that the unit prices of the goods are constant (a very unrealistic
  assumption given that price itself responds to supply and demand,
  but fortunately it does not affect the conclusion we draw here) what
  might be a way of increasing GDP while keeping the magnitude of
  output of each industry the same?

  \begin{enumerate}[(A)]
  \item Increase interindustry dependence, i.e., increase the amount
    needed from each industry that is necessary to produce a given
    amount in another industry.
  \item Reduce interindustry dependence, i.e., reduce the amount
    needed from each industry that is necessary to produce a given
    amount in another industry.
  \item Changes in interindustry dependence have no effect.
  \end{enumerate}

  {\em Answer}: Option (B)

  {\em Explanation}: The lower the interindustry dependence, the
  larger the share of the industries' output that can be used to meet
  consumer demand, i.e., contribute to GDP.

  {\em Performance review}: 20 out of 29 got this. 6 chose (C), 3
  chose (A).

  {\em Historical note (last time)}: $26$ out of $29$ got this. $3$ chose (C).
\end{enumerate}

\end{document}
