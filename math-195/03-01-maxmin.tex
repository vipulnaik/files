\documentclass[10pt]{amsart}

%Packages in use
\usepackage{fullpage, hyperref, vipul, enumerate}

%Title details
\title{Take-home class quiz: due Friday March 1: Max/min values: one-variable recall}
\author{Math 195, Section 59 (Vipul Naik)}
%List of new commands

\begin{document}
\maketitle

Your name (print clearly in capital letters): $\underline{\qquad\qquad\qquad\qquad\qquad\qquad\qquad\qquad\qquad\qquad}$

{\bf YOU ARE FREE TO DISCUSS ALL QUESTIONS, BUT PLEASE ONLY ENTER FINAL ANSWER CHOICES THAT YOU PERSONALLY ENDORSE.}

\begin{enumerate}

\item Suppose $f$ is a function defined on a closed interval
  $[a,c]$. Suppose that the left-hand derivative of $f$ at $c$ exists
  and equals $\ell$. Which of the following implications is {\bf true
  in general}?

  \begin{enumerate}[(A)]
  \item If $f(x) < f(c)$ for all $a \le x < c$, then $\ell < 0$.
  \item If $f(x) \le f(c)$ for all $a \le x < c$, then $\ell \le 0$.
  \item If $f(x) < f(c)$ for all $a \le x < c$, then $\ell > 0$.
  \item If $f(x) \le f(c)$ for all $a \le x < c$, then $\ell \ge 0$.
  \item None of the above is true in general.
  \end{enumerate}

  \vspace{0.1in}
  Your answer: $\underline{\qquad\qquad\qquad\qquad\qquad\qquad\qquad}$
  \vspace{0.5in}

\item Suppose $f$ is a continuous function defined on an open interval
  $(a,b)$ and $c$ is a point in $(a,b)$. Which of the following
  implications is {\bf true}?
  \begin{enumerate}[(A)]

  \item If $c$ is a point of local minimum for $f$, then there is a
    value $\delta > 0$ and an open interval $(c - \delta, c + \delta)
    \subseteq (a,b)$ such that $f$ is non-increasing on $(c -
    \delta,c)$ and non-decreasing on $(c,c+\delta)$.
  \item If there is a value $\delta > 0$ and an open interval $(c -
    \delta, c + \delta) \subseteq (a,b)$ such that $f$ is
    non-increasing on $(c - \delta,c)$ and non-decreasing on
    $(c,c+\delta)$, then $c$ is a point of local minimum for $f$.
  \item If $c$ is a point of local minimum for $f$, then there is a
    value $\delta > 0$ and an open interval $(c - \delta, c + \delta)
    \subseteq (a,b)$ such that $f$ is non-decreasing on $(c -
    \delta,c)$ and non-increasing on $(c,c+\delta)$.
  \item If there is a value $\delta > 0$ and an open interval $(c -
    \delta, c + \delta) \subseteq (a,b)$ such that $f$ is
    non-decreasing on $(c - \delta,c)$ and non-increasing on
    $(c,c+\delta)$, then $c$ is a point of local minimum for $f$.
  \item All of the above are true.
  \end{enumerate}

  \vspace{0.1in}
  Your answer: $\underline{\qquad\qquad\qquad\qquad\qquad\qquad\qquad}$
  \vspace{0.5in}

\item Consider all the rectangles with perimeter equal to a fixed
  length $p > 0$. Which of the following {\bf is true} for the unique
  rectangle which is a square, compared to the other rectangles?

  \begin{enumerate}[(A)]
  \item It has the largest area and the largest length of diagonal.
  \item It has the largest area and the smallest length of diagonal.
  \item It has the smallest area and the largest length of diagonal.
  \item It has the smallest area and the smallest length of diagonal.
  \item None of the above.
  \end{enumerate}

  \vspace{0.1in}
  Your answer: $\underline{\qquad\qquad\qquad\qquad\qquad\qquad\qquad}$
  \vspace{0.5in}

  {\bf PLEASE TURN OVER FOR REMAINING QUESTIONS}

  \newpage
\item Suppose the total perimeter of a square and an equilateral
  triangle is $L$. (We can choose to allocate all of $L$ to the
  square, in which case the equilateral triangle has side zero, and we
  can choose to allocate all of $L$ to the equilateral triangle, in
  which case the square has side zero). Which of the following
  statements {\bf is true} for the sum of the areas of the square and
  the equilateral triangle? (The area of an equilateral triangle is
  $\sqrt{3}/4$ times the square of the length of its side).
  \begin{enumerate}[(A)]
  \item The sum is minimum when all of $L$ is allocated to the square.
  \item The sum is maximum when all of $L$ is allocated to the square.
  \item The sum is minimum when all of $L$ is allocated to the
    equilateral triangle.
  \item The sum is maximum when all of $L$ is allocated to the
    equilateral triangle.
  \item None of the above.
  \end{enumerate}

  \vspace{0.1in}
  Your answer: $\underline{\qquad\qquad\qquad\qquad\qquad\qquad\qquad}$
  \vspace{0.5in}

\item Suppose $x$ and $y$ are positive numbers such as $x + y =
  12$. For {\bf what values} of $x$ and $y$ is $x^2y$ maximum?

  \begin{enumerate}[(A)]
  \item $x = 3$, $y = 9$
  \item $x = 4$, $y = 8$
  \item $x = 6$, $y = 6$
  \item $x = 8$, $y = 4$
  \item $x = 9$, $y = 3$
  \end{enumerate}

  \vspace{0.1in}
  Your answer: $\underline{\qquad\qquad\qquad\qquad\qquad\qquad\qquad}$
  \vspace{0.5in}

\item Consider the function $p(x) := x^2 + bx + c$, with $x$
  restricted to integer inputs. Suppose $b$ and $c$ are integers. The
  minimum value of $p$ is attained either at a single integer or at
  two consecutive integers. Which of the following is a {\bf
  sufficient condition} for the minimum to occur at two consecutive
  integers?

  \begin{enumerate}[(A)]
  \item $b$ is odd
  \item $b$ is even
  \item $c$ is odd
  \item $c$ is even
  \item None of these conditions is sufficient.
  \end{enumerate}

  \vspace{0.1in}
  Your answer: $\underline{\qquad\qquad\qquad\qquad\qquad\qquad\qquad}$
  \vspace{0.5in}
\end{enumerate}

\end{document}