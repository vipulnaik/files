\documentclass[10pt]{amsart}

%Packages in use
\usepackage{fullpage, hyperref, vipul, enumerate}

%Title details
\title{Take-home class quiz solutions: due Monday October 28: Matrix multiplication and inversion as computational problems}
\author{Math 196, Section 57 (Vipul Naik)}
%List of new commands

\begin{document}
\maketitle

\section{Performance review}

26 people took this 16-question quiz. The score distribution is as
follows:

\begin{itemize}
\item Score of 3: 1 person
\item Score of 6: 2 people
\item Score of 7: 1 person
\item Score of 8: 9 people
\item Score of 9: 3 people
\item Score of 10: 5 people
\item Score of 11: 3 people
\item Score of 12: 1 person
\item Score of 15: 1 person
\end{itemize}

The question-wise answers and performance review are below:

\begin{enumerate}
\item Option (A): 25 people%$21$ people
\item Option (C): 23 people%$23$ people
\item Option (B): 12 people%$12$ people
\item Option (A): 15 people%$20$ people
\item Option (C): 15 people%$16$ people
\item Option (E): 20 people%$14$ people
\item Option (B): 9 people%$19$ people
\item Option (B): 11 people%$22$ people
\item Option (A): 8 people%$10$ people
\item Option (B): 5 people%$13$ people
\item Option (B): 21 people%$21$ people
\item Option (A): 24 people%$23$ people
\item Option (D): 10 people%$12$ people
\item Option (E): 1 person%$8$ people
\item Option (D): 18 people%$20$ people
\item Option (A): 14 people%$13$ people
\end{enumerate}

\section{Solutions}
This quiz tests for a strong {\em conceptualization} (i.e., a
metacognition) of the processes used for matrix multiplication and
inversion. It is based on part of the {\tt Matrix multiplication and
  inversion} notes and is related to Sections 2.3 and 2.4. It does
not, however, test all aspects of that material.

{\bf PLEASE FEEL FREE TO DISCUSS {\em ALL} QUESTIONS.}

\begin{enumerate}
\item How many arithmetic operations are needed for naive matrix
  multiplication of a $m \times n$ matrix and a $n \times p$ matrix?

  \begin{enumerate}[(A)]
  \item $O(mnp)$ additions and $O(mnp)$ multiplications
  \item $O(m + n + p)$ additions and $O(mnp)$ multiplications
  \item $O(mn)$ additions and $O(np)$ multiplications
  \item $O(mn + mp)$ additions and $O(mnp)$ multiplications
  \item $O(m + n + p)$ additions and $O(m + n + p)$ multiplications
  \end{enumerate}

  {\em Answer}: Option (A)

  {\em Explanation}: The actual number of multiplications is $mnp$ and
  the actual number of additions is $m(n - 1)p$, both of which are
  $O(mnp)$. In fact, it is $\Theta(mnp)$. (The $O$ notation means an
  upper bound on order, the $\Theta$ notation means upper and lower
  bounds on order).

  {\em Performance review}: 25 out of 26 got this. 1 person chose (D).

  {\em Historical note (last time)}: $21$ out of $26$ got this. $5$ chose (D).

\item What is the arithmetic complexity (in terms of total number of
  arithmetic operations needed) for naive matrix multiplication of two
  generic $n \times n$ matrices?

  \begin{enumerate}[(A)]
  \item $\Theta(n)$
  \item $\Theta(n^2)$
  \item $\Theta(n^3)$
  \item $\Theta(n^4)$
  \item $\Theta(n^5)$
  \end{enumerate}

  {\em Answer}: Option (C)

  {\em Explanation}: Based on Question 1, where $m,n,p$ are all equal
  to $n$.

  {\em Performance review}: 23 out of 26 got this. 3 people
  chose (D).

  {\em Historical note (last time)}: $23$ out of $26$ got this. $2$
  chose (B), $1$ chose (D).
\item Which of the following is the tightest ``obvious'' lower bound
  on the possible arithmetic complexity of any generic algorithm for
  multiplying two $n \times n$ matrices? We use $\Omega$ to denote
  {\em at least that order}.

  \begin{enumerate}[(A)]
  \item $\Omega(n)$
  \item $\Omega(n^2)$
  \item $\Omega(n^3)$
  \item $\Omega(n^4)$
  \item $\Omega(n^5)$
  \end{enumerate}

  {\em Answer}: Option (B)

  {\em Explanation}: The product matrix has $n^2$ entries, all of
  which could in principle be different and require at least some
  nonzero computation. In other words, just filling in the output
  matrix takes $n^2$ steps. Thus, the obvious lower bound for matrix
  multiplication is $\Omega(n^2)$.

  {\em Performnance review}: 12 out of 26 got this. 8 chose
  (A), 6 chose (C).

  {\em Historical note (last time)}: $12$ out of $26$ got this. $10$
  chose (C), $4$ chose (A).
\item What is the minimum number of arithmetic operations needed to
  compute the product of two generic diagonal $n \times n$ matrices?

  \begin{enumerate}[(A)]
  \item $n$
  \item $n + 1$
  \item $2n - 1$
  \item $2n$
  \item $n^2$
  \end{enumerate}

  {\em Answer}: Option (A)

  {\em Explanation}: All the off-diagonal entries are $0$. The
  diagonal entries are obtained by entry-wise
  multiplication. Explicitly, if $AB = C$, then:

  $$c_{ii} = a_{ii}b_{ii}$$

  Thus, each of the $n$ diagonal entries of the matrix requires one
  multiplication to compute, so the total number of operations
  necessary is $n$.

  {\em Performance review}: 15 out of 26 got this. 8 chose (E),
  2 chose (C), 1 chose (D).

  {\em Historical note (last time)}: $20$ out of $26$ got this. $4$
  chose (E), $2$ chose (C).
\item What is the minimum number of arithmetic operations needed to
  compute the product of a generic $1 \times n$ matrix and a generic
  $n \times 1$ matrix?

  \begin{enumerate}[(A)]
  \item $n$
  \item $n + 1$
  \item $2n - 1$
  \item $2n$
  \item $n^2$
  \end{enumerate}

  {\em Answer}: Option (C)

  {\em Explanation}: This is a dot product computation. It involves
  $n$ multiplications and $n - 1$ additions, so a total of $2n - 1$
  operations. Alternatively, recall that in general, for multiplying a
  $m \times n$ matrix and a $n \times p$ matrix, we require $mnp$
  multiplications and $m(n -1)p$ additions. Here, $m = p = 1$.

  {\em Performance review}: 15 out of 26 got this. 8 chose (A),
  2 chose (B), 1 chose (E).

  {\em Historical note (last time)}: $16$ out of $26$ got this. $9$
  chose (A), $1$ chose (D).
\item What is the minimum number of arithmetic operations needed to
  compute the product of a generic $n \times 1$ matrix and a generic
  $1 \times n$ matrix?

  \begin{enumerate}[(A)]
  \item $n$
  \item $n + 1$
  \item $2n - 1$
  \item $2n$
  \item $n^2$
  \end{enumerate}

  {\em Answer}: Option (E)

  {\em Explanation}: This is the Hadamard product or outer
  product. The product is a $n \times n$ matrix and each entry is
  simply one product. Thus, there are $n^2$ multiplications and $0$
  additions, so a total of $n^2$ operations.

  We can think of this in terms of the general setting of
  multiplication of a $m \times n$ matrix and a $n \times p$
  matrix. The $n$ in our current situation equals both the $m$ and the
  $p$ of the generic setup and the $n$ of our generic setup equals
  $1$. The number of multiplications is $mnp = n(1)n = n^2$ and the
  the number of additions is $n(1 - 1)n = 0$.
  
  {\em Performance review}: 20 out of 26 got this. 4 chose (A), 2 chose (C).

  {\em Historical note (last time)}: $14$ out of $26$ got this. $6$ chose (A),
  $3$ chose (C), $2$ chose (D), $1$ chose (B).

\item What is the minimum number of arithmetic operations needed to
  compute the product of a generic $n \times n$ diagonal matrix and a
  generic $n \times n$ upper triangular matrix?  The upper triangular
  matrix has zero entries below the diagonal. The entries on or above
  the diagonal may be nonzero (and generically, they will be nonzero).

  \begin{enumerate}[(A)]
  \item $n(n - 1)/2$
  \item $n(n + 1)/2$
  \item $n(n - 1)$
  \item $n^2$
  \item $n(n + 1)$
  \end{enumerate}

  {\em Answer}: Option (B)

  {\em Explanation}: To compute the $(ij)^{th}$ entry of the product,
  we need to take the dot product of the $i^{th}$ row of the diagonal
  matrix and the $j^{th}$ column of the other matrix. There is only
  one nonzero term in the corresponding summation if $i \le j$, and no
  nonzero term if $i > j$. So, we have to do $0$ additions, and the
  number of multiplications is the number of entries in the upper
  triangular part.

  So, we need to calculate the number of entries in the upper triangle
  including the diagonal. There are $n$ positions on the
  diagonal. There are thus $n^2 - n$ off-diagonal entries, with half
  of them above the diagonal and half of them below the
  diagonal. Thus, there are $(n^2 - n)/2 = n(n - 1)/2$ entries above
  the diagonal, so a total of $n(n - 1)/2 + n = n(n + 1)/2$ entries on
  and above the diagonal.

  {\em Performance review}: 9 out of 26 got this. 7 each chose (A) and
  (C), 2 chose (D), 1 left the question blank.

  {\em Historical note (last time)}: $19$ out of $26$ got
  this. $5$ chose (D), $2$ chose (B).

  \vspace{1in}

  Adding $n$ numbers to each other requires $n - 1$ addition
  operations. In a non-parallel setting, there is no way of improving
  this.

  However, using the associativity of addition, we can write a faster
  parallelizable algorithm. A simple parallelization is to split the
  list being added into two sublists of length about $n/2$
  each. Delegate the task of adding up within each sublist to
  different processors running in parallel. Then, add up the numbers
  obtained. This takes about half the time, with a little overhead (of
  collecting and adding up). This type of strategy is called a {\em
    divide and conquer} strategy. Using a divide and conquer strategy
  repeatedly, we can demonstrate that the parallelized arithmetic
  complexity of this approach is $\Theta(\log_2n)$.

\item Suppose $A$ is a $1 \times n$ matrix and $B$ is a $n \times 1$
  matrix. Assume an unlimited number of processors that all have free
  read access to both $A$ and $B$, free write access to the product
  matrix, and a shared workspace where they can store intermediate
  results. What is the arithmetic complexity in this context (i.e.,
  the parallelized arithmetic complexity) for computing $AB$? What we
  mean here is: what is the smallest depth of a computational tree to
  compute $AB$?

  \begin{enumerate}[(A)]
  \item $\Theta(1)$
  \item $\Theta(\log_2 n)$
  \item $\Theta(n \log_2 n)$
  \item $\Theta(n^2)$
  \item $\Theta(n^2 \log_2 n)$
  \end{enumerate}

  {\em Answer}: Option (B)

  {\em Explanation}: Computing the dot product requires computing $n$
  individual products, and then adding them up. The computation of the
  individual products can be done in parallel, so that takes time
  $\Theta(1)$. The adding up involves the addition of $n$ numbers,
  which takes then $\Theta(\log_2n)$.  The total time taken is thus
  $\Theta(\log_2 n)$.

  {\em Performance review}: 11 out of 26 got this. 11 chose (C), 1
  chose (D), 3 chose (A).

  {\em Historical note (last time)}: $22$ out of $26$ got this. $2$ chose (C),
  $1$ each chose (A) and (D).

\item Suppose $A$ is a $n \times 1$ matrix and $B$ is a $1 \times n$
  matrix. Assume an unlimited number of processors that all have free
  read access to both $A$ and $B$, free write access to the product
  matrix, and a shared workspace where they can store intermediate
  results. What is the arithmetic complexity in this context (i.e.,
  the parallelized arithmetic complexity) for computing $AB$? What we
  mean here is: what is the smallest depth of a computational tree to
  compute $AB$?

  \begin{enumerate}[(A)]
  \item $\Theta(1)$
  \item $\Theta(\log_2 n)$
  \item $\Theta(n \log_2 n)$
  \item $\Theta(n^2)$
  \item $\Theta(n^2 \log_2 n)$
  \end{enumerate}

  {\em Answer}: Option (A)

  {\em Explanation}: Since the number of columns in $A$ = the number
  of rows in $B$ is $1$, we do not need to perform any
  additions. Rather, we need to do $n^2$ multiplications. All these
  multiplications can be performed in parallel, so the time taken is
  $\Theta(1)$.

  {\em Performance review}: 8 out of 26 got this. 7 chose (B), 6 chose
  (D), 4 chose (C), 1 chose (E).

  {\em Historical note (last time)}: $10$ out of $26$ got this. $10$ chose (C),
  $4$ chose (B), $2$ chose (D).
\item Suppose $A$ and $B$ are two $n \times n$ matrices. Assume an
  unlimited number of processors that all have free read access to
  both $A$ and $B$, free write access to the product matrix, and a
  shared workspace where they can store intermediate results. What is
  the arithmetic complexity in this context (i.e., the parallelized
  arithmetic complexity) for computing $AB$? What we mean here is:
  what is the smallest depth of a computational tree to compute $AB$?
  Use naive matrix multiplication and speed it up using the
  parallelized processes discused here.

  \begin{enumerate}[(A)]
  \item $\Theta(1)$
  \item $\Theta(\log_2 n)$
  \item $\Theta(n \log_2 n)$
  \item $\Theta(n^2)$
  \item $\Theta(n^2 \log_2 n)$
  \end{enumerate}

  {\em Answer}: Option (B)

  {\em Explanation}: We need to calculate $n^2$ matrix entries in the
  product, but all the entries can be computed in parallel, and the
  time taken to compute a matrix entry is the time taken to perform a
  single dot product. As we saw earlier, this is $\Theta(\log_2 n)$.

  We are given a $n \times n$ matrix $A$ and we want to use {\em
    repeated squaring} to calculate powers of $A$. For instance, to
  calculate $A^4$, we can simply calculate $(A^2)^2$, which requires
  two multiplications. To calculate $A^5$, we calculate $(A^2)^2A$,
  which requires three multiplications. Assume that we can store any
  number of intermediate matrices, i.e., storage space is not a
  constraint.

  {\em Performance review}: 5 out of 26 got this. 19 chose (E), 2
  chose (C).

  {\em Historical note (last time)}: $13$ out of $26$ got this. $9$
  chose (E), $2$ chose (D), $1$ each chose (A) and (C).
\item What is the smallest number of matrix multiplications needed to
  calculate $A^7$ using repeated squaring?

  \begin{enumerate}[(A)]
  \item $3$
  \item $4$
  \item $5$
  \item $6$
  \item $7$
  \end{enumerate}

  {\em Answer}: Option (B)

  {\em Explanation}: We compute $A^2$ (first operation), then compute
  $(A^2)^2 = A^4$ (second operation), then multiply them to get $A^6$
  (third operation), then multiply that by $A$ to get $A^7$ (fourth
  operation).

  {\em Performance review}: 21 out of 26 got this. 3 chose (A), 2 chose (C).

  {\em Historical note (last time)}: $21$ out of $26$ got this. $3$ chose (C),
  $2$ chose (A).
\item What is the smallest number of matrix multiplications needed to
  calculate $A^8$ using repeated squaring?

  \begin{enumerate}[(A)]
  \item $3$
  \item $4$
  \item $5$
  \item $6$
  \item $7$
  \end{enumerate}

  {\em Answer}: Option (A)

  {\em Explanation}: Three squarings will do the trick.

  {\em Performance review}: 24 out of 26 got this. 2 chose (B).

  {\em Historical note (last time)}: $23$ out of $26$ got this. $3$ chose (C).

\item What is the smallest number of matrix multiplications needed to
  calculate $A^{21}$ using repeated squaring?

  \begin{enumerate}[(A)]
  \item $3$
  \item $4$
  \item $5$
  \item $6$
  \item $7$
  \end{enumerate}

  {\em Answer}: Option (D)

  {\em Explanation}: Square $A$ four times to get to $A^{16}$. In the
  process, we have also found $A^2$, $A^4$, and $A^8$. Multiply
  $A^{16}$ by $A^4$ (fifth operation) and then multiply by $A$ (sixth
  operation).

  {\em Performance review}: 10 out of 26 got this. 15 chose (E), 1
  left the question blank.

  {\em Historical note (last time)}: $12$ out of $26$ got this. $8$ chose (E),
  $5$ chose (C), $1$ chose (A).

  \vspace{1in}

  Suppose $A$ is an {\em invertible} $n \times n$ matrix. It is
  possible to invert $A$ using $\Theta(n^3)$ (worst-case) arithmetic
  operations via Gauss-Jordan elimination. We can thus add computation
  of the inverse to our toolkit when calculating powers. It is helpful
  even when calculating positive powers.

  Count each matrix multiplication and each matrix inversion as one
  ``matrix operation.''

\item What is the smallest positive $r$ where we can achieve a saving
  on the total number of matrix operations to calculate $A^r$ by also
  computing $A^{-1}$, rather than just using repeated squaring?

  \begin{enumerate}[(A)]
  \item $3$
  \item $7$
  \item $15$
  \item $23$
  \item $31$
  \end{enumerate}

  {\em Answer}: Option (E)

  {\em Explanation}: The case where we are likely to get the best
  saving is where $r = 2^s - 1$. In this case, doing the calculation
  without finding the inverse takes $2s - 2$ steps and doing the
  calculation using the inverse takes $s + 2$ steps: $s$ to calculate
  $A^{2^s}$, $1$ to calculate $A^{-1}$, and $1$ more to multiply them
  and get $A^{2^s - 1}$.. For $s + 2 < 2s - 2$, we need $s \ge 5$, so
  the smallest number that works is $2^5 - 1 = 31$.

  {\em Performance review}: 1 out of 26 got this. 16 chose (C), 4
  chose (D), 3 chose (B), 2 chose (A), and 1 left the question blank.

  {\em Historical note (last time)}: $8$ out of $26$ got this. $9$ chose (C),
  $5$ chose (B), $2$ chose (D), $1$ chose (A).
\item {\em Strassen's algorithm} is a {\em fast matrix multiplication}
  algorithm that can multiply two $n \times n$ matrices using
  $O(n^{\log_27})$ arithmetic operations. In practice, however, a lot
  of existing computer code for matrix multiplication, written long
  after Strassen's algorithm was discovered, uses naive matrix
  multiplication. Which of the following reasons explain this? Please
  see Options (D) and (E) before answering.

  \begin{enumerate}[(A)]
  \item Strassen's algorithm becomes faster than naive matrix
    multiplication only for very large matrix sizes.
  \item Strassen's algorithm is more complicated to code.
  \item Strassen's algorithm is not as easily parallelizable as naive
    matrix multiplication.
  \item All of the above.
  \item None of the above.
  \end{enumerate}

  {\em Answer}: Option (D)

  {\em Explanation}: Strassen's algorithm for the $2 \times 2$ case
  requires $7$ multiplications and $14$ additions. The number of
  additions is more. The saving on multiplication does dominate for
  large enough matrix sizes, but this does not occur immediately.

  Strassen's algorithm is definitely more complicated to code. If it
  were easy to code, it would also be easier to explain, and we would
  have discussed it in class.

  Parallelization is harder because on the ``conquer'' part of the
  divide and conquer strategy. With the naive approach, each entry can
  be computed totally separately. With Strassen's, there are many
  different types of computations that need to be done and then pieced
  together.

  {\em Performance review}: 18 out of 26 got this. 5 chose (C), 3
  chose (A).

  {\em Historical note (last time)}: $20$ out of $26$ got this. $3$ chose (C),
  $2$ chose (E), $1$ chose (A).
  \vspace{1in}

  There exist even faster algorithms for matrix multiplication than
  Strassen's algorithm. The best known algorithm currently is the {\em
    Coppersmith-Winograd algorithm}, which can multiply two $n \times
  n$ matrices in time $O(n^{2.3727})$. However, the
  Coppersmith-Winograd algorithm is even more rarely implemented than
  Strassen's for practical matrix multiplication code (according to
  some sources, Coppersmith-Winograd has {\em never} been
  implemented). The same reasons as those cited above for the
  reluctance to use Strassen's algorithm apply. There are some
  additional obstacles to practical implementations of the
  Coppersmith-Winograd algorithm that make it even more difficult to
  use.

  \vspace{1in}
\item Suppose $A$ and $B$ are $n \times n$ matrices. What is the
  minimum number of matrix multiplications needed generically to
  compute the product $ABABABABA$?

   \begin{enumerate}[(A)]
   \item $4$
   \item $5$
   \item $6$
   \item $7$
   \item $8$
   \end{enumerate}

   {\em Answer}: Option (A)

   {\em Explanation}: We calculate $AB$ (first), then square it to get
   $ABAB$ (second), then square again to get $ABABABAB$ (third), then
   multiply by $A$ to get $ABABABA$ (fourth).
   
   {\em Performance review}: 14 out of 26 got this. 7 chose (B), 3
   chose (E), 1 each chose (C) and (D).

   {\em Historical note (last time)}: $13$ out of $26$ got this. $8$ chose (B),
   $2$ chose (C), $2$ chose (E), $1$ chose (D).
\end{enumerate}
\end{document}
