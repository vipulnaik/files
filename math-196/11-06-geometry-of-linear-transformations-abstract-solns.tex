\documentclass[10pt]{amsart}

%Packages in use
\usepackage{fullpage, hyperref, vipul, enumerate}

%Title details
\title{Take-home class quiz solutions: due Wednesday November 6: Geometry of linear transformations (abstract)}
\author{Math 196, Section 57 (Vipul Naik)}
%List of new commands

\begin{document}
\maketitle

\section{Performance review}

27 people took this 14-question quiz. The score distribution was
as follows:

\begin {itemize}
\item Score of 2: 1 person
\item Score of 4: 1 person
\item Score of 6: 3 people
\item Score of 7: 2 people
\item Score of 8: 4 people
\item Score of 9: 4 people
\item Score of 10: 5 people
\item Score of 11: 4 people
\item Score of 12: 2 people
\item Score of 14: 1 person
\end{itemize}

The mean score was 8.8.

The question-wise answers and performance review are as follows:

\begin{enumerate}
\item Option (B): 18 people
\item Option (C): 22 people
\item Option (C): 17 people
\item Option (C): 18 people
\item Option (A): 10 people
\item Option (E): 22 people
\item Option (C): 25 people
\item Option (E): 16 people
\item Option (C): 13 people
\item Option (B): 17 people
\item Option (B): 12 people
\item Option (A): 20 people
\item Option (C): 15 people
\item Option (C): 13 people
\end{enumerate}

\section{Solutions}

{\bf PLEASE FEEL FREE TO DISCUSS {\em ALL} QUESTIONS.}

This quiz tests for a deep abstract understanding of linear
transformations and their geometry. It is related to Section 2.2 of
the book and also to the {\tt Geometry of linear transformations}
lecture notes.

For the questions here, please use the following terminology.

Suppose $n$ is a fixed natural number greater than $1$. For ease of
geometric visualization, you can take $n = 2$ for the discussion.

\begin{itemize}
\item A {\em linear automorphism} of $\R^n$ is defined as
  a bijective linear transformation from $\R^n$ to $\R^n$.
\item An {\em affine linear automorphism} of $\R^n$ is defined as a
  bijective function from $\R^n$ to itself that preserves
  collinearity, i.e., it sends lines to lines. In addition, it
  preserves the ratios of lengths within each line. This can be
  included as part of the definition or deduced from the fact that
  collinearity is preserved for $n > 1$.
\item A {\em self-isometry} of $\R^n$ is defined as a bijective
  function from $\R^n$ to itself that preserves Euclidean distance:
  for all pairs of points $\vec{x},\vec{y} \in \R^n$, the Euclidean
  distance between $\vec{x}$ and $\vec{y}$ equals the Euclidean
  distance between $T(\vec{x})$ and $T(\vec{y})$.
\item A {\em self-homothety} (or {\em similitude transformation} or
  {\em similarity transformation}) of $\R^n$ is defined as a bijective
  function from $\R^n$ to itself that multiplies all distances by a
  fixed number called the {\em factor of similitude} (dependent on the
  transformation): if the factor of similitude is $\lambda$, then for
  all pairs of points $\vec{x},\vec{y} \in \R^n$, the distance between
  $T(\vec{x})$ and $T(\vec{y})$ equals $\lambda$ times the distance
  between $\vec{x}$ and $\vec{y}$.
\end{itemize}

\begin{enumerate}
\item What is the relationship between linear automorphisms and affine
  linear automorphisms of $\R^n$?

  \begin{enumerate}[(A)]
  \item Being a linear automorphism is precisely the same as being an
    affine linear automorphism.
  \item Every linear automorphism is an affine linear automorphism,
    but not every affine linear automorphism is a linear automorphism.
  \item Every affine linear automorphism is a linear automorphism, but
    not every linear automorphism is an affine linear automorphism.
  \item A linear automorphism need not be affine linear, and an affine
    linear automorphism need not be linear.
  \end{enumerate}

  {\em Answer}: Option (B)

  {\em Explanation}: Linear automorphisms are precisely those affine
  linear automorphisms that send the origin to itself. In general, an
  affine linear automorphism can be expressed as a composite of a
  translation and a linear automorphism.

  {\em Performance review}: 18 out of 27 got this. 8 chose (C), 1 chose (A).

  {\em Historical note (last time)}: $16$ out of $27$ got this. $10$ chose (C),
  $1$ chose (D).

\item What is the relationship between self-homotheties and
  self-isometries of $\R^n$?

  \begin{enumerate}[(A)]
  \item Being a self-homothety is precisely the same as being a
    self-isometry.
  \item Every self-homothety is a self-isometry, but not every
    self-isometry is a self-homothety.
  \item Every self-isometry is a self-homothety, but not every
    self-homothety is a self-isometry.
  \item A self-homothety need not be a self-isometry, and a
    self-isometry need not be a self-homothety.
  \end{enumerate}

  {\em Answer}: Option (C)

  {\em Explanation}: We can define a self-isometry as a self-homothety
  where the factor of homothety (the factor of similitude) is $1$.

  {\em Performance review}: 22 out of 27 got this. 4 chose (B), 1 chose (D).

  {\em Historical note (last time)}: $21$ out of $27$ got this. $4$ chose (D),
  $1$ each chose (A) and (B).

\item What is the relationship between self-homotheties and affine
  linear automorphisms of $\R^n$?

  \begin{enumerate}[(A)]
  \item Being an affine linear automorphism is precisely the same as
    being a self-homothety.
  \item Every affine linear automorphism is a self-homothety, but not
    every self-homothety is an affine linear automorphism.
  \item Every self-homothety is an affine linear automorphism, but not
    every affine linear automorphism is a self-homothety.
  \item An affine linear automorphism need not be a self-homothety,
    and a self-homothety need not be affine linear.
  \end{enumerate}

  {\em Answer}: Option (C)

  {\em Explanation}: A self-homothety must preserve ratios of
  distances. In particular, thanks to the triangle inequality, it must
  preserve collinearity with betweennness. In other words, if $T$ is a
  self-homothety of $\R^n$ and $A,B,C$ are points in $\R^n$ with $B$
  between $A$ and $C$, then $T(A)$, $T(B)$, and $T(C)$ are also points
  in $\R^n$ with $T(B)$ between $T(A)$ and $T(C)$, because of the
  triangle inequality.

  There are affine linear automorphisms, such as shear automorphisms,
  that are not self-homotheties. Alternatively, we could also consider
  a dilation along one axis keeping the other axis fixed.

  {\em Performance review}: 17 out of 27 got this. 6 chose (B), 4 chose (A).

  {\em Historical note (last time)}: $18$ out of $27$ got this. $6$ chose (B),
  $2$ chose (D), $1$ chose (A).

\item What is the relationship between affine linear automorphisms and
  self-isometries of $\R^n$?

  \begin{enumerate}[(A)]
  \item Being an affine linear automorphism is precisely the same as
    being a self-isometry.
  \item Every affine linear automorphism is a self-isometry, but not
    every self-isometry is an affine linear automorphism.
  \item Every self-isometry is an affine linear automorphism, but not
    every affine linear automorphism is a self-isometry.
  \item An affine linear automorphism need not be a self-isometry, and
    a self-isometry need not be affine linear.
  \end{enumerate}

  {\em Answer}: Option (C)

  {\em Explanation}: This can be deduced from the two preceding
  questions, and can also be seem directly.

  {\em Performance review}: 18 out of 27 got this. 3 each chose (B)
  and (D), 2 chose (A), 1 left the question blank.

  {\em Historical note (last time)}: $22$ out of $27$ got this. $4$ chose (D),
  $1$ chose (A).
\item There is a special kind of bijection from $\R^n$ to $\R^n$
  called a {\em translation}. A translation with translation vector
  $\vec{v}$ is defined as the bijection $\vec{x} \mapsto \vec{x} +
  \vec{v}$. A {\em nontrivial} translation is a translation whose
  translation vector is not the zero vector. Which of the following is
  an automorphism type that nontrivial translations are {\em not}?
  Please see Option (E) before answering.

  \begin{enumerate}[(A)]
  \item Linear automorphism
  \item Affine linear automorphism
  \item Self-isometry
  \item Self-homothety
  \item None of the above, i.e., nontrivial translations are of all
    these types
  \end{enumerate}

  {\em Answer}: Option (A)

  {\em Explanation}: Translations are self-isometries. Hence, they are
  also self-homotheties and affine linear automorphisms. They are not
  linear automorphisms because they do not preserve the origin.

  {\em Performance review}: 10 out of 27 got this. 7 chose (E), 5
  chose (C), 4 chose (D), 1 chose (B).

  {\em Historical note (last time)}: $12$ out of $27$ got this. $11$ chose (E),
  $3$ chose (C), and $1$ chose (D).

\item A collection of bijections from $\R^n$ to itself is said to form
  a {\em group} if it satisfies all these three conditions:

  \begin{itemize}
  \item The composite of any two (possibly equal, possibly distinct)
    bijections in the collection is also in the collection.
  \item The identity bijection (i.e., the map sending every vector to
    itself) is in the collection.
  \item For every bijection in the collection, the inverse bijection is
    also in the collection.
  \end{itemize}

  For fixed $n$, which of the following collections of bijections from
  $\R^n$ to itself does {\em not} form a group? Please see Option (E)
  before answering.

  \begin{enumerate}[(A)]
  \item The collection of all linear automorphisms of $\R^n$
  \item The collection of all affine linear automorphisms of $\R^n$
  \item The collection of all self-isometries of $\R^n$
  \item The collection of all self-homotheties of $\R^n$
  \item None of the above, i.e., each of them is a group
  \end{enumerate}

  {\em Answer}: Option (E)

  {\em Explanation}: This is obvious from the definition and concept
  of symmetry.

  {\em Performance review}: 22 out of 27 got this. 2 each chose (B)
  and (D), 1 chose (C).

  {\em Historical note (last time)}: $24$ out of $27$ got this. $2$ chose (C),
  $1$ chose (D).

  For the remaining questions, we deal with the case $n = 2$.

  We consider two special types of bijections from $\R^2$ to $\R^2$:
  {\em rotations} (a rotation is specified by the center of rotation
  and the angle of rotation) and {\em reflections} (a reflection is
  specified by the line of reflection).
  
  The identity map (i.e., the map sending every point to itself) is
  considered both a translation and a rotation. It is the translation
  by the zero vector. It can be viewed as a rotation about any point
  by the zero angle.

  Note that for a rotation, the angle of rotation is determined
  uniquely up to additive multiples of $2\pi$. The center of rotation
  is determined uniquely for all nontrivial rotations.

 \item What is the composite of two rotations centered at the same
   point in $\R^2$? Assume for simplicity that the composite is not the
   identity, i.e., the two rotations do not cancel each other. Note
   that the rotations must commute, so the order of operation does
   not matter.

   \begin{enumerate}[(A)]
   \item It must be a reflection about a line passing through that center point.
   \item It must be a reflection about a line {\em not} passing through
     that center point.
   \item It must be a rotation centered at the same point
   \item It must be a rotation but it need not be centered at the same point.
   \item It must be a translation.
   \end{enumerate}

   {\em Answer}: Option (C)

   {\em Explanation}: This is physically obvious.

   {\em Performance review}: 25 out of 27 got this. 1 chose (D), 1
   left the question blank.

   {\em Historical note (last time)}: $18$ out of $27$ got this. $5$ chose (D),
   $3$ chose (E), $1$ chose (A).

 \item What is the composite of two reflections about lines in $\R^2$,
   if the two lines of reflection are known to be parallel but
   distinct? Although the two reflections do not commute, the {\em
     type} of their composite does not depend upon the order in which
   we compose them.

   \begin{enumerate}[(A)]
   \item It must be a reflection about a third line which is parallel
     to both the lines and is equidistant from them.
   \item It must be a reflection about a third line which is
     perpendicular to both the lines.
   \item It must be a rotation about a point that is equidistant from
     both lines.
   \item It must be a translation by a vector parallel to the lines
     about which we are reflecting.
   \item It must be a translation by a vector perpendicular to the
     lines about which we are reflecting.
   \end{enumerate}

   {\em Answer}: Option (E)

   {\em Explanation}: The vector will in fact be twice the
   perpendicular difference vector between the lines.

   You can think about it in terms of double mirrors.

   {\em Performance review}: 16 out of 27 got this. 6 chose (A), 3
   chose (D), 1 chose (B).

   {\em Historical note (last time)}: $20$ out of $27$ got this. $3$ chose (D),
   $2$ each chose (A) and (C).
 \item What is the composite of two reflections about lines in $\R^2$,
   if the two lines of reflection are distinct and intersect? Once
   again, the reflections do not in general commute, but the {\em type}
   of the composite does not depend on the order of composition.

   \begin{enumerate}[(A)]
   \item It must be a reflection about a third line which passes
     through the point of intersection of the two lines of reflection.
   \item It must be a reflection about a third line which does not pass
     through the point of intersection of the two lines of reflection.
   \item It must be a rotation about the point of intersection.
   \item It must be a translation by a vector that makes equal angles
     with both the lines.
   \item It need not be a translation, rotation, or reflection.
   \end{enumerate}

   {\em Answer}: Option (C)

   {\em Explanation}: In fact, it will be a rotation with the angle of
   rotation equal to {\em twice} the angle between the lines. We can
   easily see this using basic Euclidean geometry. The simplest case
   to think of is the case of perpendicular lines of reflection. In
   this case, the composite is a rotation by $\pi$, also known as a
   half-turn.

   {\em Performance review}: 13 out of 27 people got this. 8 chose
   (A), 2 each chose (D) and (E), 1 chose (B).

   {\em Historical note (last time)}: $18$ out of $27$ got this. $5$
   chose (A), $2$ each chose (B) and (E).
 \item What is the composite of a nontrivial rotation in $\R^2$ (i.e.,
   the angle of rotation is not a multiple of $2\pi$) and a nontrivial
   translation?

   \begin{enumerate}[(A)]
   \item It must be a rotation with the same center of rotation but
     with a different angle of rotation.
   \item It must be a rotation with the same angle of rotation but with a
     different center of rotation.
  \item It must be a reflection about a line passing through the
    center of rotation.
  \item It must be a reflection about a line {\em not} passing through
    the center of rotation.
  \item It must be a translation.
  \end{enumerate}

  {\em Answer}: Option (B)

  {\em Explanation}: The angle of rotation must remain the same. We
  can see this by imagining some kind of physical figure that is made
  to undergo the rotation and then the translation. The rotation
  changes the direction of the figure by the angle of rotation. The
  translation then preserves the direction of the figure. The
  composite should also change the direction of the figure by the same
  angle, hence must be a rotation by the same angle. The center of
  rotation may well be different, and in fact, must be different if
  the translation is nontrivial.

  {\em Performance review}: 17 out of 27 got this. 3 each chose
  (A) and (E), 2 chose (D), 1 chose (C), 1 left the question blank.

  {\em Historical note (last time)}: $16$ out of $27$ got this. $7$ chose (A),
  $2$ chose (C), $1$ each chose (D) and (E).
\item An affine linear automorphism of $\R^2$ is termed {\em
  area-preserving} if it preserves areas, i.e., the area of the image
  of any triangle under the automorphism is the same as the area of
  the original triangle. What is the relation between being a
  self-isometry and being an area-preserving affine linear
  automorphism of $\R^2$?

  \begin{enumerate}[(A)]
  \item Being a self-isometry is precisely the same as being an
    area-preserving affine linear automorphism.
  \item Every self-isometry is area-preserving, but not every
    area-preserving affine linear automorphism is a self-isometry.
  \item Every area-preserving affine linear automorphism is a
    self-isometry, but not every self-isometry is area-preserving.
  \item A self-isometry need not be an area-preserving affine linear
    automorphism, and an area-preserving affine linear automorphism
    need not be a self-isometry.
  \end{enumerate}

  {\em Answer}: Option (B)

  {\em Explanation}: If something preserves lengths, it sends
  triangles to congruent triangles, hence it also preserves areas.

  There can be area-preserving affine linear automorphisms that are
  not self-isometries. Shear operations, such as the one with this
  matrix:

  $$\left[\begin{matrix} 1 & 1 \\ 0 & 1 \\\end{matrix}\right]$$

  are examples. The point is that areas still get preserved. For
  instance, a triangle based on the $x$-axis gets bent out of shape,
  but its base and height remain the same, hence its area remains the
  same.

  {\em Performance review}: 12 out of 27 got this. 9 chose (A), 4
  chose (C), 1 chose (D), 1 left the question blank.

  {\em Historical note (last time)}: $15$ out of $27$ got this. $5$ chose (A),
  $3$ each chose (C) and (D), $1$ left the question blank.

\item An affine linear automorphism of $\R^2$ is termed {\em
  orientation-preserving} if it preserves orientation, i.e., it does
  not interchange left with right. An affine linear automorphism of
  $\R^2$ is termed {\em orientation-reversing} if it reverses
  orientation, i.e., it interchanges the roles of left and
  right. Obviously, the composite of two orientation-preserving affine
  linear automorphisms is orientation-preserving. What can we say
  about the composite of two orientation-reversing affine linear
  automorphisms?

  \begin{enumerate}[(A)]
  \item It must be orientation-preserving
  \item It must be orientation-reversing
  \item It may be orientation-preserving or orientation-reversing
  \end{enumerate}

  {\em Answer}: Option (A)

  {\em Explanation}: Reversing the orientation means switching the
  roles of left and right. Reversing the orientation a second time
  means switching the roles back. Thus, the composite of
  orientation-preserving.

  {\em Performance review}: 20 out of 27 got this. 6 chose (C), 1 left
  the question blank.

  {\em Historical note (last time)}: $24$ out of $27$ got this. $2$ chose (B),
  $1$ chose (C).
\item The linear automorphism of $\R^2$ with matrix:

  $$\left[\begin{matrix} 1 & 1 \\ 0 & 1 \\\end{matrix}\right]$$

  is an example of a {\em shear automorphism}. Which of the following
  is this automorphism {\em not}? Please see options (D) and (E)
  before answering.

  \begin{enumerate}[(A)]
  \item Area-preserving
  \item Orientation-preserving
  \item Self-isometry
  \item None of the above, i.e., it is area-preserving,
    orientation-preserving, and a self-isometry
  \item All of the above, i.e., it is not area-preserving, not
    orientation-preserving, and not a self-isometry of $\R^2$.
  \end{enumerate}

  {\em Answer}: Option (C)

  {\em Explanation}: The vector $\left[\begin{matrix} 0 \\ 1
      \\\end{matrix}\right]$ maps to the vector $\left[\begin{matrix}
      1 \\ 1 \\\end{matrix}\right]$. The original vector had length
  $1$, but the new vector has length $\sqrt{2}$, so the map does not
  preserve length. It does preserve area and orientation.

  {\em Performance review}: 15 out of 27 got this. 6 chose (E), 3
  chose (D), 2 chose (A).

  {\em Historical note (last time)}: $16$ out of $27$ got this. $4$ each chose
  (A) and (D), $3$ chose (B).

\item Which of the following is guaranteed to send any triangle in
  $\R^2$ to a similar triangle? Please see Options (D) and (E) before
  answering.

  \begin{enumerate}[(A)]
  \item Linear automorphism
  \item Affine linear automorphism
  \item Self-homothety
  \item All of the above
  \item None of the above
  \end{enumerate}

  {\em Answer}: Option (C)

  {\em Explanation}: Follows from the definition.

  {\em Performance review}: 13 out of 27 got this. 10 chose (D), 2
  chose (E), 1 chose (B), 1 left the question blank.

  {\em Historical note (last time)}: $20$ out of $27$ got this. $5$ chose (D),
  $1$ each chose (A) and (E).
\end{enumerate}
\end{document}
