\documentclass[10pt]{amsart}

%Packages in use
\usepackage{fullpage, hyperref, vipul, enumerate}

%Title details
\title{Take-home class quiz: due February 24: Sequences and series, miscellaneous stuff}
\author{Math 153, Section 55 (Vipul Naik)}
%List of new commands

\begin{document}
\maketitle

Your name (print clearly in capital letters): $\underline{\qquad\qquad\qquad\qquad\qquad\qquad\qquad\qquad\qquad\qquad}$

{\bf YOU ARE FREE TO DISCUSS ALL QUESTIONS, BUT PLEASE MAKE SURE TO
ONLY ENTER ANSWER CHOICES THAT YOU PERSONALLY ENDORSE}

\begin{enumerate}
\item {\em Forward difference operators and partial sums}: Recall that
  for a function $g: \N \to \R$, the forward difference operator of
  $g$, denoted $\Delta g$, is defined as the function $(\Delta g)(n) =
  g(n+1) - g(n)$. Suppose we have two functions $f,g: \N \to \R$ such
  that $g(n) = \sum_{k=1}^n f(k)$. What is the relationship between
  $\Delta g$ and $f$? {\em This is a discrete version of the
  fundamental theorem of calculus.}

  \begin{enumerate}[(A)]
  \item $(\Delta g)(n) = f(n)$ for all $n \in \N$
  \item $(\Delta g)(n) = f(n + 1)$ for all $n \in \N$
  \item $(\Delta g)(n + 1) = f(n)$ for all $n \in \N$
  \item $(\Delta g)(n) = f(n + 2)$ for all $n \in \N$
  \item $(\Delta g)(n + 2) = f(n)$ for all $n \in \N$
  \end{enumerate}

  \vspace{0.1in}
  Your answer: $\underline{\qquad\qquad\qquad\qquad\qquad\qquad\qquad}$
  \vspace{0.15in}

\item Which of the following is the correct definition of $\lim_{x \to
  \infty} f(x) = L$ for $L$ a finite number?

  \begin{enumerate}[(A)]
  \item For every $\epsilon > 0$ there exists $a \in \R$ such that if
    $0 < |x - L| < \epsilon$ then $f(x) > a$.
  \item For every $\epsilon > 0$ there exists $a \in \R$ such that if
    $x > a$ then $|f(x) - L| < \epsilon$.
  \item For every $a \in \R$ there exists $\epsilon > 0$ such that if
    $x > a$ then $|f(x) - L| < \epsilon$.
  \item For every $a \in \R$ there exists $\epsilon > 0$ such that if
    $0 < |x - L| < \epsilon$ then $f(x) > a$.
  \item There exists $a \in \R$ and $\epsilon > 0$ such that if $x >
    a$ then $|f(x) - L| < \epsilon$.
  \end{enumerate}

  \vspace{0.1in}
  Your answer: $\underline{\qquad\qquad\qquad\qquad\qquad\qquad\qquad}$
  \vspace{0.15in}

\item {\em Horizontal asymptote and limit of derivative}: Suppose
  $\lim_{x \to \infty} f'(x)$ is finite. Which of the following is
  true (be careful about $f$ versus $f'$ when reading the choices)?
  \begin{enumerate}[(A)]
  \item If $\lim_{x \to \infty} f'(x)$ is zero, then $\lim_{x \to
    \infty} f(x)$ is finite.
  \item If $\lim_{x \to \infty} f(x)$ is finite, then $\lim_{x \to
    \infty} f'(x)$ is zero.
  \item If $\lim_{x \to \infty} f(x)$ is finite, then $\lim_{x \to
    \infty} f(x)$ is zero.
  \item All of the above.
  \item None of the above.
  \end{enumerate}

  \vspace{0.1in}
  Your answer: $\underline{\qquad\qquad\qquad\qquad\qquad\qquad\qquad}$
  \vspace{0.15in}

\item {\em Convergent sequence and limit of forward difference
  operator}: Suppose $f: \N \to \R$ is a function (so we can think of
  it as a sequence). Which of the following is true? Here $(\Delta
  f)(n) =f(n+1) - f(n)$.

  \begin{enumerate}[(A)]
  \item If $\lim_{n \to \infty} (\Delta f)(n)$ is zero, then $\lim_{n \to
    \infty} f(n)$ is finite.
  \item If $\lim_{n \to \infty} f(n)$ is finite, then $\lim_{n \to
    \infty} (\Delta f)(n)$ is zero.
  \item If $\lim_{n \to \infty} f(n)$ is finite, then $\lim_{n \to
    \infty} f(n)$ is zero.
  \item All of the above.
  \item None of the above.
  \end{enumerate}

  \vspace{0.1in}
  Your answer: $\underline{\qquad\qquad\qquad\qquad\qquad\qquad\qquad}$
  \vspace{0.15in}

\item {\em Function iteration converges at infinity}: Suppose $(a_n)$
  is a sequence whose terms are given by the relation $a_n =
  f(a_{n-1})$, with $a_1$ specified separately and $f$ is a continuous
  function on $\R$. Further, suppose we know that $\lim_{n \to \infty}
  a_n = L$ for some finite $L$. What can we conclude is true about
  $L$?

  \begin{enumerate}[(A)]
  \item $f(L) = L$
  \item $f(L) = 0$
  \item $f'(L) = L$
  \item $f'(L) = 0$
  \item $f''(L) = 0$
  \end{enumerate}

  \vspace{0.1in}
  Your answer: $\underline{\qquad\qquad\qquad\qquad\qquad\qquad\qquad}$
  \vspace{0.15in}

\item {\em Equilibrium at infinity}: Suppose a function $y$ of time
  $t$ satisfies the differential equation $y' = f(y)$ for all time
  $t$, where $f$ is a continuous function on $\R$. Further, suppose we
  know that $\lim_{t \to \infty} y = L$ for some finite $L$. What can
  we conclude is true about $L$?  {\em Note: Although the question is
  conceptually similar to the preceding question, you have to reason
  about the question differently.}
  \begin{enumerate}[(A)]
  \item $f(L) = L$
  \item $f(L) = 0$
  \item $f'(L) = L$
  \item $f'(L) = 0$
  \item $f''(L) = 0$
  \end{enumerate}

  \vspace{0.1in}
  Your answer: $\underline{\qquad\qquad\qquad\qquad\qquad\qquad\qquad}$
  \vspace{0.15in}

\item A sequence $a_n$ is found to satisfy the recurrence $a_{n+1} =
  2a_n(1 - a_n)$. Assume that $a_1$ is strictly between $0$ and
  $1$. What can we say about the sequence $(a_n)$?
  \begin{enumerate}[(A)]
  \item It is monotonic non-increasing, and its limit is $0$.
  \item It is monotonic non-decreasing, and its limit is $1$.
  \item From $a_2$ onward, it is monotonic non-decreasing, and its
    limit is $1/2$.
  \item From $a_2$ onward, it is monotonic non-increasing, and its
    limit is $1/2$.
  \item It is either monotonic non-decreasing or monotonic
    non-increasing everywhere, and its limit is $1/2$.
  \end{enumerate}

  \vspace{0.1in}
  Your answer: $\underline{\qquad\qquad\qquad\qquad\qquad\qquad\qquad}$
  \vspace{0.15in}

\item Suppose $f$ is a continuous function on $\R$ and $(a_n)$ is a
  sequence satisfying the recurrence $f(a_n) = a_{n+1}$ for all
  $n$. Further, suppose the limit of the $a_n$s for odd $n$ is $L$ and
  the limit of the $a_n$s for even $n$ is $M$. What can we say about
  $L$ and $M$?
  \begin{enumerate}[(A)]
  \item $f(L) = L$ and $f(M) = M$
  \item $f(L) = M$ and $f(M) = L$
  \item $f(L) = f(M) = 0$
  \item $f'(L) = f'(M) = 0$
  \item $f'(L) = M$ and $f'(M) = L$
  \end{enumerate}

  \vspace{0.1in}
  Your answer: $\underline{\qquad\qquad\qquad\qquad\qquad\qquad\qquad}$
  \vspace{0.15in}

\item Consider a function $f$ on the natural numbers defined as
  follows: $f(m) = m/2$ if $m$ is even, and $f(m) = 3m + 1$ if $m$ is
  odd. Consider a sequence where $a_1$ is a natural number and we
  define $a_n := f(a_{n-1})$. It is conjectured (see {\em Collatz
  conjecture}) that $(a_n)$ is eventually periodic, regardless of the
  starting point, and that there is only one possibility for the
  eventual periodic fragment. Which of the following can be the
  eventual periodic fragment?

  \begin{enumerate}[(A)]
  \item $(1,2,3)$
  \item $(1,3,2)$
  \item $(1,2,4)$
  \item $(1,4,2)$
  \item $(1,3,4)$
  \end{enumerate}

  \vspace{0.1in}
  Your answer: $\underline{\qquad\qquad\qquad\qquad\qquad\qquad\qquad}$
  \vspace{0.15in}

\item For which of the following properties $p$ of sequences of real
  numbers does $p$ equal {\em eventually} $p$?
  \begin{enumerate}[(A)]
  \item Monotonicity
  \item Periodicity
  \item Being a polynomial sequence (i.e., given by a polynomial function)
  \item Being a constant sequence
  \item Boundedness
  \end{enumerate}

  \vspace{0.1in}
  Your answer: $\underline{\qquad\qquad\qquad\qquad\qquad\qquad\qquad}$
  \vspace{0.15in}

  The remaining questions are based on a rule which we call the {\em
  degree difference rule}. This states the following. Consider a
  rational function $p(x)/q(x)$, and suppose $a \in \R$ is such that
  $q$ has no roots in $[a,\infty)$. Then, the improper integral
  $\int_a^\infty \frac{p(x)}{q(x)} \, dx$ is finite if and only if the
  degree of $q$ minus the degree of $p$ is {\em at least} two, or in
  other words, is strictly greater than one. The same rule applies to
  $\int_{-\infty}^\infty \frac{p(x)}{q(x)} \, dx$ if $q$ has no zero.

  The degree difference rule has a slight variation: we can apply it
  to situation where $p$ and $q$ are not quite polynomials, but rather
  their growth rates are of the same order as that of some polynomial
  or power function. For instance, $(x^2 + 1)^{3/2}$ has ``degree''
  three with this more liberal interpretation. {\em Added: In this
    more liberal interpretation, we require that the degree difference
    (which could now be a non-integer), be strictly greater than
    one. For instance, a degree difference of $3/2$ means that the
    integral converges.}

  Consider a probability distribution on $\R$ with density function
  $f$. In particular, this means that $\int_{-\infty}^\infty f(x) \,
  dx = 1$. Further, assume that $f$ has mean zero and is an even
  function, i.e., the probability distribution is symmetric about
  zero.

  The {\em mean deviation} of the distribution is defined as
  $\int_{-\infty}^\infty |x|f(x) \, dx$. On account of the fact that
  $f$ is an even function, this can be rewritten as $2
  \int_0^\infty xf(x) \, dx$.

  The {\em standard deviation} of the distribution, denoted $\sigma$, of $f$ is defined as
  $\sqrt{\int_{-\infty}^\infty x^2 f(x) \, dx}$.

  The {\em kurtosis} of the distribution is defined as $-3 +
  (\int_{-\infty}^\infty x^4 f(x) \, dx)/\sigma^4$. Note that the
  kurtosis does not make sense if the standard deviation is infinite.

\item Consider the distribution with density function $f(x):=
  (x^2+1)^{-1}/\pi$. (We divide by $\pi$ so that the integral on
  $(-\infty,\infty)$ is $1$). Which of the following is true?

  \begin{enumerate}[(A)]
  \item The mean deviation is finite but the standard deviation is infinite.
  \item The standard deviation and kurtosis are finite, but the mean deviation is infinite.
  \item The standard deviation, mean deviation, and kurtosis are all finite.
  \item The standard deviation and mean deviation are finite, but the kurtosis is infinite.
  \item The standard deviation and mean deviation are both infinite.
  \end{enumerate}

  \vspace{0.1in}
  Your answer: $\underline{\qquad\qquad\qquad\qquad\qquad\qquad\qquad}$
  \vspace{0.15in}

\item Consider the distribution with density function $f(x):=
  (x^2+1)^{-3/2}/2$. (We divide by $2$ so that the integral on
  $(-\infty,\infty)$ is $1$). Which of the following is true?

  \begin{enumerate}[(A)]
  \item The mean deviation is finite but the standard deviation is infinite.
  \item The standard deviation and kurtosis are finite, but the mean deviation is infinite.
  \item The standard deviation, mean deviation, and kurtosis are all finite.
  \item The standard deviation and mean deviation are finite, but the kurtosis is infinite.
  \item The standard deviation and mean deviation are both infinite.
  \end{enumerate}

  \vspace{0.1in}
  Your answer: $\underline{\qquad\qquad\qquad\qquad\qquad\qquad\qquad}$
  \vspace{0.15in}

\item Consider the distribution with density function $f(x):=
  (x^2+1)^{-2}/(\pi/2)$. (We divide by $\pi/2$ so that the integral on
  $(-\infty,\infty)$ is $1$). Which of the following is true?

  \begin{enumerate}[(A)]
  \item The mean deviation is finite but the standard deviation is infinite.
  \item The standard deviation and kurtosis are finite, but the mean deviation is infinite.
  \item The standard deviation, mean deviation, and kurtosis are all finite.
  \item The standard deviation and mean deviation are finite, but the kurtosis is infinite.
  \item The standard deviation and mean deviation are both infinite.
  \end{enumerate}

  \vspace{0.1in}
  Your answer: $\underline{\qquad\qquad\qquad\qquad\qquad\qquad\qquad}$
  \vspace{0.15in}

\item Consider the distribution with density function $f(x):=
  (x^2+1)^{-5/2}/(4/3)$. (We divide by $4/3$ so that the integral on
  $(-\infty,\infty)$ is $1$). Which of the following is true?

  \begin{enumerate}[(A)]
  \item The mean deviation is finite but the standard deviation is infinite.
  \item The standard deviation and kurtosis are finite, but the mean deviation is infinite.
  \item The standard deviation, mean deviation, and kurtosis are all finite.
  \item The standard deviation and mean deviation are finite, but the kurtosis is infinite.
  \item The standard deviation and mean deviation are both infinite.
  \end{enumerate}

  \vspace{0.1in}
  Your answer: $\underline{\qquad\qquad\qquad\qquad\qquad\qquad\qquad}$
  \vspace{0.15in}

\item Consider the distribution with density function $f(x):=
  (x^2+1)^{-3}/(3\pi/8)$. (We divide by $3\pi/8$ so that the integral on
  $(-\infty,\infty)$ is $1$). Which of the following is true?

  \begin{enumerate}[(A)]
  \item The mean deviation is finite but the standard deviation is infinite.
  \item The standard deviation and kurtosis are finite, but the mean deviation is infinite.
  \item The standard deviation, mean deviation, and kurtosis are all finite.
  \item The standard deviation and mean deviation are finite, but the kurtosis is infinite.
  \item The standard deviation and mean deviation are both infinite.
  \end{enumerate}

  \vspace{0.1in}
  Your answer: $\underline{\qquad\qquad\qquad\qquad\qquad\qquad\qquad}$
  \vspace{0.15in}


\end{enumerate}
\end{document}