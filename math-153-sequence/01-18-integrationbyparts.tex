\documentclass[10pt]{amsart}

%Packages in use
\usepackage{fullpage, hyperref, vipul, enumerate}

%Title details
\title{Take-home class quiz: Integration by parts: due Wednesday January 18}
\author{Math 153, Section 55 (Vipul Naik)}
%List of new commands

\begin{document}
\maketitle

Your name (print clearly in capital letters): $\underline{\qquad\qquad\qquad\qquad\qquad\qquad\qquad\qquad\qquad\qquad}$

{\bf YOU ARE FREE TO DISCUSS ALL QUESTIONS, BUT PLEASE MAKE SURE TO
ONLY ENTER ANSWER CHOICES THAT YOU PERSONALLY ENDORSE}

In the questions below, we say that a function is {\em expressible in
terms of elementary functions} or {\em elementarily expressible} if it
can be expressed in terms of polynomial functions, rational functions,
radicals, exponents, logarithms, trigonometric functions and inverse
trigonometric functions using pointwise combinations, compositions,
and piecewise definitions. We say that a function is {\em elementarily
integrable} if it has an elementarily expressible antiderivative.

Note that if a function is elementarily expressible, so is its
derivative on the domain of definition.

We say that a function $f$ is $k$ times elementarily integrable if
there is an elementarily expressible function $g$ such that $f$ is the
$k^{th}$ derivative of $g$.

We say that the integrals of two functions are {\em equivalent up to
elementary functions} if an antiderivative for one function can be
expressed using an antiderivative for the other function and
elementary function, again piecing them together using pointwise
combination, composition, and piecewise definitions.

\begin{enumerate}

\item Consider the statements $P$ and $Q$, where $P$ states that every
  rational function is elementarily integrable, and $Q$ states that
  any rational function is $k$ times elementarily integrable for all
  positive integers $k$.

  Which of the following additional observations is {\bf correct} and
  {\bf allows us to deduce} $Q$ given $P$? {\em Last year: $18/27$ correct}

  \begin{enumerate}[(A)]
  \item There is no way of deducing $Q$ from $P$ because $P$ is true
    and $Q$ is false.
  \item The antiderivative of a rational function can always be chosen
    to be a rational function, hence $Q$ follows from a repeated
    application of $P$.
  \item Using integration by parts, we see that repeated integration
    of a function $f$ is equivalent to integrating $f$, $f^2$, $f^3$,
    and higher powers of $f$ (the powers here are pointwise products,
    not compositions). If $f$ is a rational function, each of these is
    also a rational function. Applying $P$, each of these is
    elementarily integrable, hence $f$ is $k$ times elementarily
    integrable for all $k$.
  \item Using integration by parts, we see that repeated integration
    of a function $f$ is equivalent to integrating $f$, $f'$, $f''$,
    and higher derivatives of $f$. If $f$ is a rational function, each
    of these is also a rational function. Applying $P$, each of these
    is elementarily integrable, hence $f$ is $k$ times elementarily
    integrable for all $k$.
  \item Using integration by parts, we see that repeated integration
    of a function $f$ is equivalent to integrating each of the
    functions $f(x)$, $xf(x)$, $\dots$. If $f$ is a rational function,
    each of these is also a rational function. Applying $P$, each of
    these is elementarily integrable, hence $f$ is $k$ times
    elementarily integrable for all $k$.
  \end{enumerate}

  \vspace{0.1in}
  Your answer: $\underline{\qquad\qquad\qquad\qquad\qquad\qquad\qquad}$
  \vspace{0.6in}

\item Suppose $f$ is a continuous function on all of $\R$ and is the
  third derivative of an elementarily expressible function, but is not
  the fourth derivative of any elementarily expressible function. In
  other words, $f$ can be integrated three times but not four times
  within the collection of elementarily expressible functions. What is
  the {\bf largest positive integer} $k$ such that $x \mapsto x^kf(x)$
  is {\em guaranteed to be} {\bf elementarily integrable}? {\em Last
  year: $23/27$ correct}

  \begin{enumerate}[(A)]
  \item $1$
  \item $2$
  \item $3$
  \item $4$
  \item $5$
  \end{enumerate}

  \vspace{0.1in}
  Your answer: $\underline{\qquad\qquad\qquad\qquad\qquad\qquad\qquad}$
  \vspace{0.6in}

\item Suppose $f$ is a continuous function on $(0,\infty)$ and is the
  third derivative of an elementarily expressible function, but is not
  the fourth derivative of any elementarily expressible function. In
  other words, $f$ can be integrated three times but not four times
  within the collection of elementarily expressible functions. What is
  the {\bf largest positive integer} $k$ such that the function $x
  \mapsto f(x^{1/k})$ with domain $(0,\infty)$ is {\em guaranteed to
  be} {\bf elementarily integrable}? {\em Last year: $14/27$ correct}

  \begin{enumerate}[(A)]
  \item $1$
  \item $2$
  \item $3$
  \item $4$
  \item $5$
  \end{enumerate}

  \vspace{0.1in}
  Your answer: $\underline{\qquad\qquad\qquad\qquad\qquad\qquad\qquad}$
  \vspace{0.6in}

  
\item Of these five functions, four of the functions are elementarily
  integrable and can be integrated using integration by parts. The
  other one function is {\bf not elementarily integrable}. Identify
  this function. {\em Last year: $22/27$ correct}

  \begin{enumerate}[(A)]
  \item $x \mapsto x \sin x$
  \item $x \mapsto x \cos x$
  \item $x \mapsto x \tan x$
  \item $x \mapsto x \sin^2x$
  \item $x \mapsto x \tan^2x$
  \end{enumerate}

  \vspace{0.1in}
  Your answer: $\underline{\qquad\qquad\qquad\qquad\qquad\qquad\qquad}$
  \vspace{0.6in}

\item Consider the four functions $f_1(x) = \sqrt{\sin x}$, $f_2(x) =
  \sin \sqrt{x}$, $f_3(x) = \sin^2 x$ and $f_4(x) = \sin(x^2)$, all
  viewed as functions on the interval $[0,1]$ (so they are all well
  defined). Two of these functions are elementarily integrable; the
  other two are not. Which are {\bf the two elementarily integrable
  functions}? {\em Last year: $17/27$ correct}

  \begin{enumerate}[(A)]
  \item $f_3$ and $f_4$.
  \item $f_1$ and $f_3$.
  \item $f_1$ and $f_4$. 
  \item $f_2$ and $f_3$.
  \item $f_2$ and $f_4$.
  \end{enumerate}

  \vspace{0.1in}
  Your answer: $\underline{\qquad\qquad\qquad\qquad\qquad\qquad\qquad}$
  \vspace{0.6in}

\item Suppose $f$ is an elementarily expressible and infinitely
  differentiable function on the positive reals (so all derivatives of
  $f$ are also elementarily expressible). An antiderivative for
  $f''(x)/x$ is {\bf not equivalent} up to elementary functions to
  {\bf which one} of the following? {\em Last year: $10/27$ correct}

  \begin{enumerate}[(A)]
  \item An antiderivative for $x \mapsto f''(e^x)$, domain all of $\R$.
  \item An antiderivative for $x \mapsto f'(e^x/x)$, domain positive reals.
  \item An antiderivative for $x \mapsto f'''(x)(\ln x)$, domain positive
    reals.
  \item An antiderivative for $x \mapsto f'(1/x)$, domain positive
    reals.
  \item An antiderivative for $x \mapsto f(1/\sqrt{x})$, domain positive reals.
  \end{enumerate}

  \vspace{0.1in}
  Your answer: $\underline{\qquad\qquad\qquad\qquad\qquad\qquad\qquad}$
  \vspace{0.6in}

\item Of the five functions below, four of them have antiderivatives
  that are equivalent up to elementary functions, i.e., an
  antiderivative for any one of them can be used to provide an
  antiderivative for the other three. The fifth function has an
  antiderivative that is {\bf not equivalent} to any of these. Identify
  the fifth function. {\em Last year: $7/27$ correct}

  \begin{enumerate}[(A)]
  \item $x \mapsto e^{e^x}$, domain all reals
  \item $x \mapsto \ln(\ln x)$, domain $(1,\infty)$
  \item $x \mapsto e^x/x$, domain $(0,\infty)$
  \item $x \mapsto 1/(\ln x)$, domain $(1,\infty)$
  \item $x \mapsto 1/(\ln(\ln x))$, domain $(e,\infty)$
  \end{enumerate}

  \vspace{0.1in}
  Your answer: $\underline{\qquad\qquad\qquad\qquad\qquad\qquad\qquad}$
  \vspace{0.6in}

\item Which of the following functions has an antiderivative that is
  {\bf not equivalent} up to elementary functions to the antiderivative of
  $x \mapsto e^{-x^2}$? {\em Last year: $10/27$ correct}

  \begin{enumerate}[(A)]
  \item $x \mapsto e^{-x^4}$
  \item $x \mapsto e^{-x^{2/3}}$
  \item $x \mapsto e^{-x^{2/5}}$
  \item $x \mapsto x^2e^{-x^2}$
  \item $x \mapsto x^4e^{-x^2}$
  \end{enumerate}

  \vspace{0.1in}
  Your answer: $\underline{\qquad\qquad\qquad\qquad\qquad\qquad\qquad}$
  \vspace{0.6in}

\item Which of the following has an antiderivative that is {\bf not
  equivalent} up to elementary functions to the antiderivative of the
  function $f(x) := e^x/x, x > 0$? {\em Last year: $5/27$ correct}

  \begin{enumerate}[(A)]
  \item $x \mapsto e^x/\sqrt{x}, x > 0$
  \item $x \mapsto e^x/x^2, x > 0$
  \item $x \mapsto e^x(\ln x), x > 0$
  \item $x \mapsto e^{1/\sqrt{x}}, x > 0$
  \item $x \mapsto e^{1/x}, x > 0$
  \end{enumerate}

  \vspace{0.1in}
  Your answer: $\underline{\qquad\qquad\qquad\qquad\qquad\qquad\qquad}$
  \vspace{0.6in}


\end{enumerate}

\end{document}
