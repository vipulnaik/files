\documentclass[10pt]{amsart}

%Packages in use
\usepackage{fullpage, hyperref, vipul, enumerate}

%Title details
\title{Class quiz solutions: November 11: Whoppers}
\author{Math 152, Section 55 (Vipul Naik)}
%List of new commands

\begin{document}
\maketitle

\section{Performance review}

$12$ people took this quiz. The score distribution was as follows:

\begin{itemize}
\item Score of $1$: $2$ people
\item Score of $2$: $3$ people
\item Score of $3$: $3$ people
\item Score of $4$: $2$ people
\item Score of $8$: $2$ people
\end{itemize}

The mean score was $3.42$. Here are the problem wise answers:

\begin{enumerate}
\item Option (C): $3$ people.
\item Option (A): $8$ people. {\em This was an exact replica, so good performance was expected.}
\item Option (D): $3$ people.
\item Option (C): $4$ people.
\item Option (B): $5$ people.
\item Option (C): $4$ people.
\item Option (B): $7$ people. {\em This was an exact replica, so good
  performance was expected.}
\item Option (A): $6$ people. {\em This was an exact replica, so good
  performance was expected.}
\item Option (D): $1$ person.
\end{enumerate}
\section{Solutions}

\begin{enumerate}
\item Suppose $g:\R \to \R$ is a continuous function such that
  $\lim_{x \to 0} g(x)/x^2 = A$ for some constant $A \ne 0$. What is
  $\lim_{x \to 0} g(g(x))/x^4$? {\em Similar to (but trickier than)
  October 4 Question 2: only 1 person got that right}.

  \begin{enumerate}[(A)]
  \item $A$
  \item $A^2$
  \item $A^3$
  \item $A^2g(A)$
  \item $g(A)/A^2$
  \end{enumerate}

  {\em Answer}: Option (C)

  {\em Explanation}: First, note that since $g(x)/x^2 \to A$ as $x \to
  0$, we must have $g(x) \to 0$ as $x \to 0$. In particular, $g(0) =
  0$.

  Now, consider:

  $$\lim_{x \to 0} \frac{g(g(x))}{x^4} = \lim_{x \to 0} \frac{g(g(x))}{(g(x))^2} \cdot \frac{(g(x))^2}{x^4}$$

  Splitting the limit, we get:

  $$\lim_{x \to 0} \frac{g(g(x))}{(g(x))^2} \lim_{x \to 0} \left(\frac{g(x)}{x^2}\right)^2$$

  Setting $u = g(x)$ for the first limit, and using the fact that as
  $x \to 0$, $u \to 0$ we see that the first limit is $A$. For the
  second limit, pulling the square out yields that the second limit is
  $A^2$. The overall limit is thus $A \cdot A^2 = A^3$.

  We can also use an actual example to solve this problem. For
  instance, consider the extreme case where $g(x) = Ax^2$ (identically). In
  this case, $g(g(x)) = A(Ax^2)^2 = A^3x^4$. Thus, $g(g(x))/x^4 =
  A^3$, and the limit is thus $A^3$.

  Even more generally, if $\lim_{x \to 0} g(x)/x^n = A$, then $\lim_{x
  \to 0} g(g(x))/x^{n^2} = A^{n + 1}$.

  {\em Performance review}: $3$ out of $12$ got this correct. $7$
  chose (D), $1$ chose (B), $1$ chose (E).

  {\em Historical note (last year)}: $4$ out of $16$ people got this
  correct. $5$ people chose (A), $4$ people chose (B), $2$ people
  chose (D), and $1$ person chose (E).

  Also, this appeared in one of the error-spotting exercises for
  Midterm 1.
\item Which of the following statements is {\bf always true}? {\em
  Exact replica of a past question}.

  \begin{enumerate}[(A)]

  \item The range of a continuous nonconstant function on a closed
    bounded interval (i.e., an interval of the form $[a,b]$) is a
    closed bounded interval (i.e., an interval of the form $[m,M]$).
  \item The range of a continuous nonconstant function on an open
    bounded interval (i.e., an interval of the form $(a,b)$) is an
    open bounded interval (i.e., an interval of the form $(m,M)$).
  \item The range of a continuous nonconstant function on a closed
    interval that may be bounded or unbounded (i.e., an interval of
    the form $[a,b]$, $[a,\infty)$, $(-\infty,a]$, or
    $(-\infty,\infty)$) is also a closed interval that may be bounded
    or unbounded.
  \item The range of a continuous nonconstant function on an open
    interval that may be bounded or unbounded (i.e., an interval of
    the form $(a,b)$,$(a,\infty)$, $(-\infty,a)$, or
    $(-\infty,\infty)$), is also an open interval that may be bounded
    or unbounded.
  \item None of the above.
  \end{enumerate}

  {\em Answer}: Option (A)

  {\em Explanation}: This is a combination of the extreme-value
  theorem and the intermediate-value theorem. By the extreme-value
  theorem, the continuous function attains a minimum value $m$ and a
  maximum value $M$. By the intermediate-value theorem, it attains
  every value between $m$ and $M$. Further, it can attain no other
  values because $m$ is after all the minimum and $M$ the maximum.

  {\em The other choices}:

  Option (B): Think of a function that increases first and then
  decreases. For instance, the function $f(x) := \sqrt{1 - x^2}$ on
  $(-1,1)$ has range $(0,1]$, which is not open. Or, the function
  $\sin x$ on the interval $(0,2\pi)$ has range $[-1,1]$.

  Option (C): We can get counterexamples for unbounded intervals. For
  instance, consider the function $f(x) := 1/x$ on $[1,\infty)$. The
  range of this function is $(0,1]$, which is not closed. The idea is
  that we make the function approach but not reach a finite value as
  $x \to \infty$ (we'll talk more about this when we deal with
  asymptotes).

  Option (D): The same counterexample as for option (B) works.

  {\em Performance review}: $8$ out of $12$ got this correct. $1$ each
  chose (B), (C), (D), and (E).

  {\em Historical note (last year)}: $9$ out of $16$ people got it
  correct. $3$ people chose (C), $2$ people chose (D), and $1$ person
  each chose (A) and (E).

  {\em Historical note (last year, previous quiz)}: When the question
  appeared in a previous quiz, $2$ out of $11$ people got it correct.

\item Suppose $f$ is a continuously differentiable function on $\R$
  and $c \in \R$. Which of the following implications is {\bf false}?
  {\em Similar to (but trickier than) October 13 question 4.}

  \begin{enumerate}[(A)]
  \item If $f$ has mirror symmetry about $x = c$, $f'$ has half turn
    symmetry about $(c,f'(c))$.
  \item If $f$ has half turn symmetry about $(c,f(c))$, $f'$ has
    mirror symmetry about $x = c$.
  \item If $f'$ has mirror symmetry about $x = c$, $f$ has half turn
    symmetry about $(c,f(c))$.
  \item If $f'$ has half turn symmetry about $(c,f'(c))$, $f$ has
    mirror symmetry about $x = c$.
  \item None of the above, i.e., they are all true.
  \end{enumerate}

  {\em Answer}: Option (D)

  {\em Explanation}: We can construct a number of examples, but
  instead of doing that, we make a general note. If $f$ has mirror
  symmetry about $x = c$, then not only must $f'$ have half turn
  symetry about $(c,f'(c))$, we must {\em also} have $f'(c) =
  0$. Therefore, in any situation where $f'$ has half turn symmetry
  about a point where it does not take the value $0$, $f$ will not
  have mirror symmetry about that point. More details below.

  Option (A) is true. In fact, the following stronger claim is true:
  if $f$ has mirror symmetry about $x = c$ and $f'$ is defined on all
  of $\R$, then $f'(c) = 0$ and $f'(c + h) + f'(c - h) = 0$ for all $h
  \in \R$. This can be proved as follows. By mirror symmetry:

  $$f(c + h) = f(c - h)$$

  This is true as an identity in $h$. Thus, differentiating both sides
  with respect to $h$ and using the chain rule, we get:

  $$f'(c + h) = -f'(c - h)$$

  The negative sign arises due to the chain rule.

  Option (B) is true and the justification is similar to that for
  option (A). Namely:

  $$f(c + h) + f(c - h) = 2f(c)$$

  Differentiating both sides with respect to $h$ gives:

  $$f'(c + h) - f'(c - h) = 0$$

  Option (C) requires more justification. The idea is that:

  $$f(c + h) - f(c) - \int_c^{c + h} f'(t) \, dt$$

  and:

  $$f(c) - f(c - h) = \int_{c - h}^c f'(t) \, dt$$

  Now, since $f'$ has mirror symmetry about $c$, the two definite
  integrals are equal, so we get:

  $$f(c + h) - f(c) = f(c) - f(c - h)$$

  which is precisely the condition for half turn symmetry of $f$ about
  $(c,f(c))$.

  Option (D) is false, because, as mentioned earlier, for $f$ to have
  mirror symmetry given $f'$ having half turn symmetry, we need the
  additional condition that $f'(c) = 0$.

  For instance, any cubic polynomial has half turn symmetry, but most
  polynomials of degree four do not have mirror symmetry. In fact, a
  polynomial of degree four has mirror symmetry iff its derivative
  cubic has the property that its point of inflection is a critical
  point.

  {\em Additional note}: As we differentiate things, we obtain more
  and more symmetry. Here is the overall summary (where each step is
  true assuming that we can differentiate):

  Half turn symmetry $\stackrel{\text{diff}}{\to}$ Mirror symmetry
  $\stackrel{\text{diff}}{\to}$ Half turn symmetry about point on $x$-axis.

  Further, these arrows are reversible: any antiderivative of a
  function with half turn symmetry about a point on the $x$-axis has
  mirror symmetry about the same point, and any antiderivative of that
  has half turn symmetry for a point with the same $x$-coordinate.

  The even and odd functions are special cases where the
  $x$-coordinate is $0$, so in that special case:

  Half turn symmetry about point on $y$-axis $\stackrel{\text{diff}}{\to}$
  Even function $\stackrel{\text{diff}}{\to}$ Odd function

  {\em Performance review}: $3$ out of $12$ got this correct. $5$
  chose (E), $3$ chose (C), $1$ left the question blank.

  {\em Historical note (last year)}: $2$ out of $16$ people
  got it correct. $7$ people chose (E), $5$ people chose (C), and $2$
  people chose (A).
\item Consider the function $f(x) := \lbrace\begin{array}{rl} x, & 0
  \le x \le 1/2 \\ x - (1/5), & 1/2 < x \le 1 \\\end{array}$. Define by
  $f^{[n]}$ the function obtained by iterating $f$ $n$ times, i.e.,
  the function $f \circ f \circ f \circ \dots \circ f$ where $f$
  occurs $n$ times. What is the smallest $n$ for which $f^{[n]} =
  f^{[n + 1]}$? {\em Similar to a question on the previous midterm.}

  \begin{enumerate}[(A)]
  \item $1$
  \item $2$
  \item $3$
  \item $4$
  \item $5$
  \end{enumerate}

  {\em Answer}: Option (C)

  {\em Explanation}: We need to iterate $f$ enough times that
  everything gets inside $[0,1/2]$, after which it becomes
  stable. Note that each time, the value goes down by $0.2$. Thus, for
  any $x \le 1$, we need at most three steps to bring it in $[0,1/2]$,
  with the upper bound of $3$ being attained for $1$.

  {\em Performance review}: $4$ out of $12$ got this correct. $3$
  chose (D), $2$ each chose (A) and (E), $1$ chose (B).

  {\em Historical note (last year)}: $3$ out of $16$ people
  got this correct. $8$ people chose (A), $3$ people chose (B), and
  $2$ people chose (D).

  {\em Action point}: It seems that most people did not figure out how
  composition works. Please make sure you understand at least one
  question like this at some point in your life.

\item With $f$ as in the previous question, what is the set of points
  in $(0,1)$ where $f \circ f$ is not continuous?

  \begin{enumerate}[(A)]
  \item $0.5$ only
  \item $0.5$ and $0.7$
  \item $0.5$, $0.7$, and $0.9$
  \item $0.7$ and $0.9$
  \item $0.9$ only
  \end{enumerate}

  {\em Answer}: Option (B)

  {\em Explanation}: The piecewise definition is:

  $$(f \circ f)(x) = \lbrace\begin{array}{rl} x, & 0 \le x \le 0.5\\ x - (1/5), & 0.5 < x \le 0.7 \\ x - (2/5), & 0.7 < x \le 1 \\\end{array}$$

  We see that that points of discontinuity are $0.5$ and $0.7$.

  Note that if we considered $f \circ f \circ f$ instead, $0.9$ would
  also be a point of discontinuity, since, the definition to the right
  of $0.9$ would be $x - (3/5)$.
  
  {\em Performance review}: $5$ out of $12$ got this correct. $4$
  chose (C), $2$ chose (A), $1$ chose (D).

  {\em Historical note (last year)}: $4$ out of $16$ people got this
  correct. $9$ people chose (A), $2$ people chose (C), and $1$ person
  chose (D).

  {\em Action point}: Same as for previous problem -- make sure you
  understand this clearly at least once in your life.
\item Consider the graph of the function $f(x) := x\sin(1/(x^2 -
  1))$. What can we say about the vertical and horizontal asymptotes?
  {\em This resembles a future whopper.}

  \begin{enumerate}[(A)]
  \item The graph has vertical asymptotes at $x = +1$ and $x = -1$
    and horizontal asymptote (in both directions) $y = 0$.
  \item The graph has vertical asymptotes at $x = +1$ and $x = -1$
    and horizontal asymptote (in both directions) $y = 1$.
  \item The graph has no vertical asymptotes and horizontal
    asymptote (in both directions) $y = 0$.
  \item The graph has no vertical asymptotes and horizontal
    asymptote (in both directions) $y = 1$.
  \item The graph has no vertical or horizontal asymptotes.
  \end{enumerate}

  {\em Answer}: Option (C)

  {\em Explanation}: The points where $f$ is undefined are $x = \pm
  1$. At both these points, the limit is undefined, but the function
  is bounded, because the $x$-part has a finite limit and the
  $\sin(1/(x^2 - 1))$ part is bounded in $[-1,1]$. Thus, the function
  cannot have a vertical asymptote (in fact, it is oscillatory with no
  limit at both these points).

  For the horizontal asymptote, we rewrite the limit at $+\infty$ as:

  $$\lim_{x \to \infty} \frac{x}{x^2 - 1} \lim_{x \to \infty} (x^2 - 1)\sin(1/(x^2 - 1))$$

  The first limit is $0$. As for the second limit, putting $t = 1/(x^2
  - 1)$, we see that $t \to 0^+$ as $x \to \infty$, so the second
  limit becomes $\lim_{t \to 0^+} (\sin t)/t = 1$. The overall limit
  at $+\infty$ is thus $0$. A similar argument works for
  $-\infty$. Note that since $x^2 - 1$ has even degree, we get
  $\lim_{t \to 0^+} (\sin t)/t$ in this case as well.

  {\em Performance review}: $4$ out of $12$ got this correct. $4$
  chose (A), $2$ chose (E), $1$ each chose (B) and (D).
  
  {\em Historical note (last year)}: $3$ out of $16$ people
  got this correct. $6$ people chose (A), $5$ people chose (B), and
  $1$ person each chose (D) and (E).

  {\em Action point}: Since the most commonly chosen incorrect answer
  was (A), it seems that most people figured out the horizontal
  asymptotes correctly. However, the vertical asymptotes confused
  people -- not surprisingly, because they confused me too when I dug
  up this question from last year. Once you read the solution,
  however, you should be able to understand it.
\item Suppose $f$ and $g$ are increasing functions from $\R$ to
  $\R$. Which of the following functions is {\em not} guaranteed to be
  an increasing functions from $\R$ to $\R$? {\em An exact replica of
  a past question.}

  \begin{enumerate}[(A)]

  \item $f + g$
  \item $f \cdot g$
  \item $f \circ g$
  \item All of the above, i.e., none of them is guaranteed to be increasing.
  \item None of the above, i.e., they are all guaranteed to be increasing.
  \end{enumerate}

    {\em Answer}: Option (B)

  {\em Explanation}: The problem with option (B) arises when one or
  both functions take negative values. For instance, consider the case
  $f(x) := x$ and $g(x) := x$. Both are increasing functions on all of
  $\R$. However, the pointwise product is the function $x \mapsto
  x^2$, which is a decreasing function for negative $x$.

  Formally, the issue is that we cannot multiply inequalities of the
  form $A < B$ and $C < D$ unless we are guaranteed to be working with
  positive numbers.

  {\em The other choices}:

  Option (A): For any $x_1 <
  x_2$, we have $f(x_1) < f(x_2)$ and $g(x_1) < g(x_2)$. Adding up, we
  get $f(x_1) + g(x_1) < f(x_2) + g(x_2)$, so $(f + g)(x_1) < (f + g)(x_2)$.

  Option (C): For any $x_1 < x_2$, we have $g(x_1) < g(x_2)$ since $g$
  is increasing. Now, we use the factthat $f$ is increasing to compare
  its values at the two points $g(x_1)$ and $g(x_2)$, and we get
  $f(g(x_1)) < f(g(x_2))$. We thus get $(f \circ g)(x_1) < (f \circ
  g)(x_2)$.

  {\em Performance review}: $7$ out of $12$ got this correct. $4$
  chose (C), $1$ chose (E).

  {\em Historical note (last year)}: $9$ out of $16$ people got this
  correct. $3$ people chose (C) and $2$ people each chose (D) and (E).

  {\em Historical note (last year, previous quiz)}: When this question
  appeared earlier on October 20, only $1$ out of $15$ people got it
  correct.

\item Suppose $F$ and $G$ are continuously differentiable functions on
  all of $\R$ (i.e., both $F'$ and $G'$ are continuous). Which of the
  following is {\bf not necessarily true}? {\em Exact replica of a
  previous question.}

  \begin{enumerate}[(A)]
  \item If $F'(x) = G'(x)$ for all integers $x$, then $F - G$ is a
    constant function when restricted to integers, i.e., it takes the
    same value at all integers.
  \item If $F'(x) = G'(x)$ for all numbers $x$ that are not integers,
    then $F - G$ is a constant function when restricted to the set of
    numbers $x$ that are not integers.
  \item If $F'(x) = G'(x)$ for all rational numbers $x$, then $F - G$
    is a constant function when restricted to the set of rational
    numbers.
  \item If $F'(x) = G'(x)$ for all irrational numbers $x$, then $F -
    G$ is a constant function when restricted to the set of irrational
    numbers.
  \item None of the above, i.e., they are all necessarily true.
  \end{enumerate}

  {\em Answer}: Option (A).

  {\em Explanation}: The fact that the derivatives of two functions
  agree at integers says nothing about how the derivatives behave
  elsewhere -- they could differ quite a bit at other places. Hence,
  (A) is not necessarily true, and hence must be the right option. All
  the other options are correct as statements and hence cannot be the
  right option. This is because in all of them, the set of points
  where the derivatives agree is {\em dense} -- it intersects every
  open interval. So, continuity forces the functions $F'$ and $G'$ to
  be equal everywhere, forcing $F - G$ to be constant everywhere.

  {\em Performance review}: $6$ out of $12$ got this correct. $5$
  chose (E), $1$ chose (D).

  {\em Historical note (last year)}: $10$ out of $16$ people got this
  correct. $2$ people each chose (B), (D), and (E).

  {\em Historical note (last year, previous quiz)}: Nobody got it
  correct.

  {\em Action point}: It's possible that many of you just
  remembered/revisited the question and saw that the correct answer
  option is (A). However, you should make sure you understand {\em
  why} the correct answer option is (A).
\item Consider the four functions $\sin(\sin x)$, $\sin(\cos x)$,
  $\cos(\sin x)$, and $\cos(\cos x)$. Which of the following
  statements are true about their periodicity?

  \begin{enumerate}[(A)]
  \item All four functions are periodic with a period of $2\pi$.
  \item All four functions are periodic with a period of $\pi$.
  \item $\sin(\sin x)$ and $\sin(\cos x)$ have a period of $\pi$,
    whereas $\cos(\sin x)$ and $\cos(\cos x)$ have a period of $2\pi$.
  \item $\cos(\sin x)$ and $\cos(\cos x)$ have a period of $\pi$,
    whereas $\sin(\sin x)$ and $\sin(\cos x)$ have a period of $2\pi$.
  \item $\sin(\sin x)$ has a period of $2\pi$, the other three
    functions have a period of $\pi$.
  \end{enumerate}

  {\em Answer}: Option (D)

  {\em Explanation}: Since the inner functions in all cases have a
  period of $2\pi$, it is clear that all the four functions have a
  period of at most $2\pi$, in fact, the period of each divides
  $2\pi$. The crucial question is which of them have the smaller
  period $\pi$.

  Let's look at $\sin \circ \sin$ first. We have:

  $$\sin(\sin(x + \pi)) = \sin(-\sin x) = - \sin(\sin x)$$

  So, we see that that value at $x + \pi$ is the negative, and hence
  usually not the equal, of the value at $x$. Similarly:

  $$\sin(\cos(x + \pi)) = \sin(-\cos x) = -\sin(\cos x)$$

  On the other hand, for the functions that have a $\cos$ on the
  outside, the negative sign on the inside gets eaten up by the even
  nature of the outer function. For instance:

  $$\cos(\sin(x + \pi)) = \cos(-\sin x) = \cos(\sin x)$$

  and:

  $$\cos(\cos(x + \pi)) = \cos(-\cos x) = \cos(\cos x)$$

  Now, this is not a proof that $\pi$ is strictly the smallest period
  for these functions, but that can be proved using other methods. In
  any case, given the choices presented, it is now easy to single out
  (D) as the only correct answer.

  The key feature here is that both $\sin$ and $\cos$ (viewed as the
  inner functions of the composition) have {\em anti-period} $\pi$:
  their value gets negated after an interval of $\pi$.

  The outer function $\cos$ is even, hence it converts an anti-period
  for the inner function into a period for the overall function. The
  outer function $\sin$ is odd, so it keeps anti-periods anti-periods.

  {\em Performance review}: $1$ out of $12$ got this correct. $5$
  chose (A), $4$ chose (B), $1$ chose (C), $1$ chose (E).


  {\em Historical note (last year)}: $5$ out of $16$ people got this
  correct. $1$ person left the question blank. $7$ people chose (A),
  $1$ person chose (B), and $2$ people chose (C).

  {\em Action point}: This is fairly tricky to get at first sight, but
  you should be able to read and understand the solution.
\end{enumerate}
\end{document}
