\documentclass[10pt]{amsart}

%Packages in use
\usepackage{fullpage, hyperref, vipul, enumerate}

%Title details
\title{Take-home class quiz: due February 8: Limits at infinity and improper integral}
\author{Math 153, Section 55 (Vipul Naik)}
%List of new commands

\begin{document}
\maketitle

Your name (print clearly in capital letters): $\underline{\qquad\qquad\qquad\qquad\qquad\qquad\qquad\qquad\qquad\qquad}$

{\bf YOU ARE FREE TO DISCUSS ALL QUESTIONS, BUT PLEASE MAKE SURE TO
ONLY ENTER ANSWER CHOICES THAT YOU PERSONALLY ENDORSE}

\begin{enumerate}

\item If $\lim_{x \to \infty} f(x) = L$ for some finite $L$, this tells
  us that the graph of $f$ has a:

  \begin{enumerate}[(A)]
  \item vertical asymptote
  \item horizontal asymptote
  \item vertical tangent
  \item horizontal tangent
  \item vertical cusp
  \end{enumerate}

  \vspace{0.1in}
  Your answer: $\underline{\qquad\qquad\qquad\qquad\qquad\qquad\qquad}$
  \vspace{0.15in}

\item If $\lim_{x \to \infty} f(x) = L$ and $\lim_{x \to \infty} f'(x)
  = M$, where both $L$ and $M$ are finite, then:

  \begin{enumerate}[(A)]
  \item $L = 0$ but $M$ need not be zero
  \item $M = 0$ but $L$ need not be zero
  \item Both $L$ and $M$ must be zero.
  \item Neither $L$ nor $M$ need be zero.
  \item At least one of $L$ and $M$ must be zero, but it could be
    either one.
  \end{enumerate}

  \vspace{0.1in}
  Your answer: $\underline{\qquad\qquad\qquad\qquad\qquad\qquad\qquad}$
  \vspace{0.15in}

  Consider the following $\epsilon-\delta$ definition of limit at
  $\infty$: $\lim_{x \to \infty} f(x) = L$ if for all $\epsilon > 0$,
  there exists $a \in \R$ such that for all $x > a$, $|f(x) - L| <
  \epsilon$. 
\item What is the smallest $a$ that can be picked for the function $f
  = \arctan$ with $L$ being its limit at $\infty$ and $\epsilon =
  \pi$?
  \begin{enumerate}[(A)]
  \item $\sqrt{3}$
  \item $1$
  \item $0$
  \item $-1$
  \item There is no smallest $a$. Any $a \in \R$ will do.
  \end{enumerate}

  \vspace{0.1in}
  Your answer: $\underline{\qquad\qquad\qquad\qquad\qquad\qquad\qquad}$
  \vspace{0.15in}

\item What is the smallest $a$ that can be picked for the function $f
  = \arctan$ with $L$ being its limit at $\infty$ and $\epsilon =
  \pi/6$?
  \begin{enumerate}[(A)]
  \item $1/2$
  \item $1/\sqrt{3}$
  \item $1$
  \item $\sqrt{3}$
  \item $2$
  \end{enumerate}

  \vspace{0.1in}
  Your answer: $\underline{\qquad\qquad\qquad\qquad\qquad\qquad\qquad}$
  \vspace{0.15in}

\item Suppose $f(x) := p(x)/q(x)$ is a rational function in reduced form
  (i.e., the numerator and denominator are relatively prime) and
  $\lim_{x \to c} f(x) = \infty$. Which of the following can you
  conclude about $f$?
  \begin{enumerate}[(A)]
  \item $x - c$ divides $p(x)$, and the largest $r$ such that $(x -
    c)^r$ divides $p(x)$ is even.
  \item $x - c$ divides $q(x)$, and the largest $r$ such that $(x -
    c)^r$ divides $q(x)$ is even.
  \item $x - c$ divides $p(x)$, and the largest $r$ such that $(x -
    c)^r$ divides $p(x)$ is odd.
  \item $x - c$ divides $q(x)$, and the largest $r$ such that $(x -
    c)^r$ divides $q(x)$ is odd.
  \item $x - c$ does not divide either $p(x)$ or $q(x)$.
  \end{enumerate}

  \vspace{0.1in}
  Your answer: $\underline{\qquad\qquad\qquad\qquad\qquad\qquad\qquad}$
  \vspace{0.15in}

\item Suppose $f(x) := p(x)/q(x)$ is a rational function in reduced
  form (i.e., the numerator and denominator are relatively prime) and
  $\lim_{x \to c^-} f(x) = \infty$ and $\lim_{x \to c^+} f(x) =
  -\infty$. Which of the following can you conclude about $f$?
  \begin{enumerate}[(A)]
  \item $x - c$ divides $p(x)$, and the largest $r$ such that $(x -
    c)^r$ divides $p(x)$ is even.
  \item $x - c$ divides $q(x)$, and the largest $r$ such that $(x -
    c)^r$ divides $q(x)$ is even.
  \item $x - c$ divides $p(x)$, and the largest $r$ such that $(x -
    c)^r$ divides $p(x)$ is odd.
  \item $x - c$ divides $q(x)$, and the largest $r$ such that $(x -
    c)^r$ divides $q(x)$ is odd.
  \item $x - c$ does not divide either $p(x)$ or $q(x)$.
  \end{enumerate}

  \vspace{0.1in}
  Your answer: $\underline{\qquad\qquad\qquad\qquad\qquad\qquad\qquad}$
  \vspace{0.15in}

  Suppose $F$ is a function of two real variables, say $x$
  and $t$, so $F(x,t)$ is a real number for $x$ and $t$ restricted to
  suitable open intervals in the real number. Suppose, further, that $F$
  is jointly continuous (whatever that means) in $x$ and $t$.

  Define $f(t) := \int_0^\infty F(x,t) \, dx$. Here, while doing the
  integration, $t$ is treated as a constant. $x$, the variable of
  integration, is being integrated on $[0,\infty)$.
    
  Suppose further that $f$ is defined and continuous for $t$ in
  $(0,\infty)$.

  In the next few questions, you are asked to compute the function $f$
  explicitly given the function $F$, for $t \in (0,\infty)$.

\item $F(x,t) := e^{-tx}$. Find $f$.

  \begin{enumerate}[(A)]
  \item $f(t) = e^{-t}/t$
  \item $f(t) = e^t/t$
  \item $f(t) = 1/t$
  \item $f(t) = -1/t$
  \item $f(t) = -t$
  \end{enumerate}

  \vspace{0.1in}
  Your answer: $\underline{\qquad\qquad\qquad\qquad\qquad\qquad\qquad}$
  \vspace{0.15in}

\item $F(x,t) := 1/(t^2 + x^2)$. Find $f$.

  \begin{enumerate}[(A)]
  \item $f(t) = \pi/(2t)$
  \item $f(t) = \pi/t$
  \item $f(t) = 2\pi/t$
  \item $f(t) = \pi t$
  \item $f(t) = 2\pi t$
  \end{enumerate}

  \vspace{0.1in}
  Your answer: $\underline{\qquad\qquad\qquad\qquad\qquad\qquad\qquad}$
  \vspace{0.15in}

\item $F(x,t) := 1/(t^2 + x^2)^2$. Find $f$.

  \begin{enumerate}[(A)]
  \item $f(t) = \pi/t^3$
  \item $f(t) = \pi/(2t^3)$
  \item $f(t) = \pi/(4t^3)$
  \item $f(t) = \pi/(8t^3)$
  \item $f(t) = 3\pi/(8t^3)$
  \end{enumerate}

  \vspace{0.1in}
  Your answer: $\underline{\qquad\qquad\qquad\qquad\qquad\qquad\qquad}$
  \vspace{0.15in}

\item $F(x,t) = \exp(-(tx)^2)$. Use that $\int_0^\infty \exp(-x^2) \, dx=
  \sqrt{\pi}/2$. Find $f$.

  \begin{enumerate}[(A)]
  \item $f(t) = t^2\sqrt{\pi}/2$
  \item $f(t) = t\sqrt{\pi}/2$
  \item $f(t) = \sqrt{\pi}/2$
  \item $f(t) = \sqrt{\pi}/(2t)$
  \item $f(t) = \sqrt{\pi}/(2t^2)$
  \end{enumerate}

  \vspace{0.1in}
  Your answer: $\underline{\qquad\qquad\qquad\qquad\qquad\qquad\qquad}$
  \vspace{0.15in}

\item In the same general setup as above (but with none of these
  specific $F$s), which of the following is a {\em sufficient}
  condition for $f$ to be an increasing function of $t$?

  \begin{enumerate}[(A)]
  \item $t \mapsto F(x_0,t)$ is an increasing function of $t$ for
    every choice of $x_0 \ge 0$.
  \item $x \mapsto F(x,t_0)$ is an increasing function of $x$ for
    every choice of $t_0 \in (0,\infty)$.
  \item $t \mapsto F(x_0,t)$ is a decreasing function of $t$ for
    every choice of $x_0 \ge 0$.
  \item $x \mapsto F(x,t_0)$ is a decreasing function of $x$ for
  every choice of $t_0 \in (0,\infty)$.
  \item None of the above.
  \end{enumerate}

  \vspace{0.1in}
  Your answer: $\underline{\qquad\qquad\qquad\qquad\qquad\qquad\qquad}$
  \vspace{0.15in}

\end{enumerate}

\end{document}