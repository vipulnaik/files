\documentclass[10pt]{amsart}

%Packages in use
\usepackage{fullpage, hyperref, vipul}

%Title details
\title{Research summary as of April 2011 end}
\author{Vipul Naik}

%List of new commands
\newcommand{\Skew}{\operatorname{Skew}}

\begin{document}
\maketitle

I am currently pursuing some research questions related to $p$-groups,
partly in collaboration with John Wiltshire-Gordon, and with some
guidance from my advisor George Glauberman. In this document, I
describe the research, its current state, and future directions.

This is a rough sketch document lacking in footnotes/citations or
precise mathematical statements.

\section*{Overall aim of research}

\subsection*{Baer correspondence and proposed generalizations}

The goal of this research is to discover new results that clarify the
relationship between finite (hence nilpotent) $p$-groups and finite
nilpotent $p$-Lie rings.

Baer originally noted (in an appendix to a 1939 paper) that there is a
correspondence:

Uniquely $2$-divisible groups of nilpotency class at most two
$\leftrightarrow$ Uniquely $2$-divisible Lie rings of nilpotency class
at most two

In particular, this works for odd-order class two $p$-groups.

The correspondence proceeds by a formula which is a truncation of the
Baker-Campbell-Hausdorff formula, roughly as follows:

$$x \cdot y = x + y + \frac{1}{2}[x,y], \qquad x + y :=  \frac{x \cdot y}{\sqrt{[x,y]}}, \qquad [x,y] = [x,y]$$

Note that under the correspondence any Lie ring gets identified with a
group {\em on the same underlying set}, and homomorphisms of Lie rings
get identified with homomorphisms of the corresponding groups. The
upshot of all this is that the above establishes what category
theorists term an ``isomorphism of categories over Set'' between the
two sides. (The category theoretic interpretation is not central to
our current direction of thinking, but it may turn out to be important
at a later stage).

The Baer correspondence can be generalized in some obvious ways: {\em
first}, we note that the formula actually requires only unique
$2$-divisibility within the derived subgroup/subring. This slight
generalization does not provide any essentially new examples in the
finite case, apart from the inclusion of abelian $2$-groups.

The remaining generalizations help us cover {\em some} non-abelian
$2$-groups of class two.

{\em Second}, we can get over the restriction of unique
$2$-divisibility by merely looking for ``linear halves'', i.e.,
instead of demanding that there be a unique half for every element in
the derived subring/subgroup, we can seek an alternating bilinear map
that doubles to the Lie bracket/commutator. Allowing for linear halves
helps us include some $2$-groups not covered earlier (the smallest
example is of order $2^6$). However, we lose out on the uniqueness
aspect since there may be many difference choices of linear half.

{\em Third}, we can get rid of the ``linear'' part in {\em linear
half} and simply look for a ``half'' that satisfies enough conditions
for key features of the proof of the Baer correspondence to go
through. We require the half to be a $2$-cocycle satisfying certain
conditions (namely, it must be skew symmetric and
cyclicity-preserving\footnote{This means that whenever the two inputs
to the cocycle generate a cyclci subgroup, the output is zero}). This
gives us new examples, the smallest being of order $2^4$.

{\em Fourth}, we could drop the ``half'' thing altogether and simply
look for a $2$-cocycle satisfying certain conditions (namely, it must
be cyclicity-preserving) that skews to the Lie
bracket/commutator. This gives us new examples, the smallest being of
order $2^5$.

All these generalizations have the property that they induce
isomorphisms on cyclic subgroups on both sides. In particular, this
means that any group covered under any of these generalizations has
the same cyclic subgroup structure as a particular abelian group,
namely the additive group of its Lie ring.

\subsection*{What we lose in the generalizations}

With the generalizations listed above, we lose out on many of the
features that make the Baer correspondence work so well. Most
importantly, we lose the canonical nature of the bijection and the
``isomorphism of categories'' aspect that allows us to identify the
endomorphisms, automorphisms, subrings/subgroups, ideals/normal
subgroups, characteristic subrings/characteristic subgroups,
etc.

\subsection*{New terminology: class two Lie cring}

We introduce some terminology: A class two (near-)Lie cring is an
abelian group equipped with a binary operation that is a $2$-cocycle
satisfying certain conditions, specifically, the condition that $x * y
= 0$ whenever $\langle x,y \rangle$ is cyclic, $((x * y) * z) = (x *
(y * z)) = 0$ for all $x$, $y$, and $z$, and $x * y = -(y * x)$ for
all $x$ and $y$. For near-Lie crings, the last condition (skew
symmetry) is dropped.

Class two Lie crings can be viewed as the {\em enriched intermediate
structures that we use to pass between Lie rings and groups}. We can
think of a mapping:

$$\text{Class two Lie ring} \leftarrow \text{Class two (near-)Lie
cring} \rightarrow \text{Group}$$

The same class two Lie ring could arise from multiple class two
near-Lie crings, or from none. An analogous observation holds on the
group side.

\subsection*{Good news: uniqueness}

The {\em uniqueness theorem} roughly states the following: regardless
of the choice of the class two (near-)Lie cring used as intermediate,
the isomorphism class of the {\em group} that we get starting from a
Lie ring depends only on the isomorphism class of the Lie
ring. Conversely, the isomorphism class of the Lie ring that we can
get starting from a group depends only on the isomorphism class of the
group. The uniqueness theorem actually states something stronger, but
we will avoid the technical details here.

\subsection*{Studying extensions and cohomology}

We can also study the theory of extensions with a particular central
subgroup $A$ and a particular abelian quotient group $G$. On the one
hand, we can study the {\em Lie ring} extensions where $A$ is a
central subring and $G$ is an abelian quotient ring. On the other
hand, we can study the group extensions where $A$ is a central
subgroup and $G$ is an abelian quotient group.

The group of possible Lie ring extensions can be viewed as the direct
sum of the group of abelian extensions $H^2_{sym}(G,A)$ and the group
of alternating biadditive maps from $G$ to $A$ (the group
$\bigwedge^2(G,A)$). In other words:

$$\text{Lie ring extensions} \cong H^2_{sym}(G,A) \oplus \bigwedge^2(G,A)$$

The $H^2_{sym}(G,A)$ part captures how the addition extends, and the
$\bigwedge^2(G,A)$ part captures the Lie bracket.

The group of possible group extensions can be identified with
$H^2(G,A)$. The subgroup $H^2_{sym}(G,A)$, standing for symmetric
$2$-cocycles, corresponds to abelian extensions. It turns out that (at
least in the finite case) the quotient $H^2(G,A)/H^2_{sym}(G,A)$ can
be naturally identified with $\bigwedge^2(G,A)$. Thus, $H^2(G,A)$ has
a normal subgroup and a quotient that can be identified with the two
direct summands that describe the group of Lie ring extensions.

In particular, we see that the {\em number} of group extensions equals
the number of Lie ring extensions, and if the normal subgroup
$H^2_{sym}(G,A)$ has a complement in $H^2(G,A)$, then the group of
group extensions is isomorphic to the group of Lie ring extensions.

Our generalization of the Baer correspondence relates to the above
discussion as follows. Denote by $H^2_{CP}(G,A)$ the group of
cohomology classes represented by cyclicity-preserving $2$-cocycles $G
\times G \to A$. Then, by a strong version of the uniqueness theorem,
$H^2_{sym}(G,A)$ and $H^2_{CP}(G,A)$ have trivial intersection. Under
additional assumptions, these generate all of $H^2(G,A)$. {\em If this
happens}, then $H^2(G,A)$ is an internal direct sum of these
subgroups, and the result is that {\em we get a canonical isomorphism
between the group of group extensions and the group of Lie ring
extensions}. It turns out that the identification of group and Lie
ring under {\em this correspondence coincides with the generalization
of the Baer correspondence} that we considered a while back.

The smallest example where this phenomenon occurs for $2$-groups is
where $G$ is a Klein four-group and $A$ is cyclic of order
four. $H^2(G,A)$ is elementary abelian of order $8$, and
$H^2_{sym}(G,A)$ is elementary abelian of order $4$. $H^2_{CP}(G,A)$
is cyclic of order two and $H^2(G,A)$ splits as an internal direct
sum. We can use this decomposition to find (generalized Baer) Lie
rings for some non-abelian groups of order $16$.

\subsection*{Embedding theorem}

This theorem states that for any finite $2$-group of class at most
two, we can embed it as a subgroup of a finite $2$-group that has a
corresponding Lie ring under the generalized Baer correspondence. The
key idea is to take a central product with a large enough cyclic group
in which there is enough space to divide by two.


\subsection*{Further directions}

Some further directions of research related to these generalizations
of the Baer correspondence are:

\begin{itemize}
\item {\em Reduce the gap between necessary and sufficient conditions
  for the existence of a Lie ring for a group or a group for a Lie
  ring}: As of now, we have some easily verifiable sufficient
  conditions, and some easily verifiable necessary conditions, for a
  group to have a Lie ring under the generalized Baer
  correspondence. However, there is a considerable gap between the
  sufficient conditions and the necessary conditions. In particular,
  there are many examples that do not fall within the ``sufficient
  conditions'' and are worthy of investigations.
\item {\em Study the intermediate structures more extensively}: The
  ``intermediate structures'' refers to the structure of the group or
  Lie ring along with the chosen $2$-cocycle that satisfies the
  desired conditions. These structures have been provisionally named
  ``class two Lie crings.'' A basic theory and classification of these
  structures can help better understand the relationship between
  groups and Lie rings.
\item {\em Get clear results relating the size and structure of
  automorphism groups of the group and Lie ring}: There are some
  potential leads as to possible approaches to do this. In particular,
  it seems likely that the automorphism groups have the same order and
  partly similar structure.
\item {\em Get clear results on the relation between the
  subgroup/subring lattices on both sides}
\end{itemize}

\section*{Generalizations to class more than two}

In addition to better understanding the class two case, there are
generalizations of this idea to class more than two. To do this, we
need to generalize the {\em Lazard correspondence} rather than the
Baer correspondence.

There are two aspects to such generalizations:

\begin{itemize}
\item The list of ``bad primes'', i.e., the primes that we have
  difficulty dividing by because of problems with existence and
  uniqueness.
\item The nilpotency class at which we are operating.
\end{itemize}

Of course, trouble arises only with the ``bad primes'' that are less
than or equal to the nilpotency class at which we are operating. For
simplicity, we will assume that there is only one bad prime. This
assumption is valid in the finite case because we can separate out the
effects of different primes and restrict attention to $p$-groups. So,
we are interested in generalizations to situations of $p$-groups of
class $c$, where $p \le c$, such as:

\begin{itemize}
\item $2$-groups of class three
\item $3$-groups of class three
\item $2$-groups of class four
\item $3$-groups of class four
\item $2$-groups of class five
\item $3$-groups of class five
\item $5$-groups of class five
\end{itemize}

\subsection*{Bad prime $p = 2$, higher class}

This case has been resolved theoretically for the Lie ring to group
direction. The approach is as follows: start with the Lie ring. Now,
we look for a $2$-cocycle that skews/doubles to the Lie bracket, but
the conditions that the $2$-cocycle needs to satisfy are now somewhat
different, depending on the class at which we are operating. After
obtaining this $2$-cocycle, we consider a version of the
Baker-Campbell-Hausdorff formula where the Lie bracket is replaced by
this $2$-cocycle and the denominators are cleared of their powers of
$2$. For instance, the usual class three formula is:

$$x \cdot y = x + y + \frac{1}{2}[x,y] + \frac{1}{12}[x,[x,y]] - \frac{1}{12}[y,[x,y]]$$

The new formula is:

$$x \cdot y = x + y + (x * y) + \frac{1}{3}(x * (x * y)) - \frac{1}{3}(y * (x * y))$$

We have managed to prove that with this new formula, we can proceed
from the Lie ring to a group. However, much work needs to be done here:

\begin{itemize}
\item We don't have a cohomology interpretation that can be used to
  generate examples.
\item We aren't sure if this is the most general version possible.
\item We have identified some possible examples computationally, but
  have not been able to compute the $2$-cocycles and understand them
  conceptually.
\item We have not yet worked out the reverse formula from the group to
  the Lie ring (though this should not be hard).
\end{itemize}

\subsection*{Higher class intermediaries}

Some of the examples we have suggest that it is possible for the group
and the Lie ring to both have class $c$ but for the intermediate
structure (cring) that we go through to have higher class than
$c$. The smallest example here is a group of order $32$, which we do
not yet fully understand.

Roughly speaking, what is happening is that for the cring structure
(i.e., the $2$-cocycle we get after ``halving'') the image of each
application of the ``halved'' operation is potentially bigger than the
image of the actual Lie bracket. If we denote the Lie bracket by $[,]$
and the $2$-cocycle obtained after halving by $*$, then $L * L$ could
be bigger than $[L,L]$. But this may mean that we need more iterations
$((L * L) * L) * L \dots$ to reach the trivial subring than we do for
the Lie bracket.

\subsection*{The case of $p = 3$ and other primes}

For $p = 3$, there is no problem for groups and Lie rings of class up
to two. The problem begins when we are trying to interpret the degree
three terms, i.e., for groups/Lie rings of class three or higher.

As of now, the best level of generalization for tackling this in the
case of $3$-groups is unclear. Some ideas are:

\begin{itemize}
\item We find linear functions of $3$ variables $f(x,y,z)$ with the
  property that $3f(x,y,z) = [x,[y,z]]$ (and similarly for the other
  primes). If we can find such functions, we can interpret every
  instance of $(1/3)[x,[y,z]]$ as simply $f(x,y,z)$ and proceed from that.
\item We directly try to find functions of two variables $t(x,y)$ with
  the property that $12t(x,y) = [x,[x,y]] - [y,[x,y]]$, and satisfying
  some additional conditions.
\end{itemize}

\section*{Broader implications}

If we can obtain more results in this direction, there are a number of
possible broader implications and lines of research that could be
opened up:

\begin{itemize}
\item We have two novel approaches to the problem of division by bad
  primes. The first is to shift from ``doubles'' to ``skews'' (which
  unfortunately works only for $p = 2$). The second is to divide {\em
  non-uniquely} but impose some {\em global uniformity conditions} on
  the function we get after dividing. These uniformity conditions
  could be linearity or a cocycle condition.
\item We give prominence to an approach that has not been widely used
  in this area of mathematics: {\em generalize from linearity to a
  cocycle condition}. In particular, we look at a generalization from
  {\em rings} (which are characterized by an additive group and a {\em
  bilinear map} on top of that called multiplication) to {\em crings}
  (which are characterized by an additive group and a {\em 2-cocycle}
  on top of that). Generalizing from linear things to 2-cocycles is a
  theme in some other parts of mathematics, but it is not common in
  the study of algebraic structures of the sort we are dealing with.
\item As a raw fraction, the groups for which we can obtain Lie rings
  by this approach is very small. However, the embedding theorem and
  similar results show that many groups can be studied {\em by means
  of such correspondences} by putting them in bigger groups. It may be
  possible to come up with measurements of how far a group is from
  having a Lie ring of its own, and of bounding the nature and
  structure of the subgroups and automorphisms of groups.
\item Another related direction of study (that has been shelved for
  now) is the relation between derivations and automorphisms obtained
  by exponentiating them. Some preliminary calculations suggest that
  the derivation-automorphism relationship will play on $1$-cocycles
  in much the same way as our current work plays on $2$-cocycles.
\end{itemize}

\end{document}