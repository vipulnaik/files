\documentclass[10pt]{amsart}

%Packages in use
\usepackage{fullpage, hyperref, vipul, enumerate}

%Title details
\title{Class quiz: January 9: Inverse trigonometric functions}
\author{Math 153, Section 55 (Vipul Naik)}
%List of new commands

\begin{document}
\maketitle

Your name (print clearly in capital letters): $\underline{\qquad\qquad\qquad\qquad\qquad\qquad\qquad\qquad\qquad\qquad}$

\begin{enumerate}
\item What is the domain of $\arcsin \circ \arcsin$? Here, {\em
  domain} refers to the maximal possible subset of $\R$ on which the
  function is defined. {\em Last year: $18/28$ correct}

  \begin{enumerate}[(A)]
  \item $[-1,1]$
  \item $[-\sin 1,\sin 1]$
  \item $[-\arcsin 1, \arcsin 1]$
  \item $[-\sin(2/\pi),\sin(2/\pi)]$
  \item $[-\arcsin(2/\pi),\arcsin(2/\pi)]$
  \end{enumerate}

  \vspace{0.1in}
  Your answer: $\underline{\qquad\qquad\qquad\qquad\qquad\qquad\qquad}$
  \vspace{0.4in}

\item One of these five functions has a horizontal asymptote as $x \to
  +\infty$ and a horizontal asymptote as $x \to -\infty$, with the
  limiting values for $+\infty$ and $-\infty$ being {\em
  different}. Identify the function. {\em This didn't appear in last
  year's quiz, but appeared in a midterm two years ago in somewhat
  modified form. At the time, $11/15$ got it correct.}

  \begin{enumerate}[(A)]
  \item $f(x) := \ln|x|$.
  \item $f(x) := \arctan x$.
  \item $f(x) := e^{-x}$.
  \item $f(x) := e^{-x^2}$.
  \item $f(x) := (\sin x)/(x^2 + 1)$.
  \end{enumerate}

  \vspace{0.1in}
  Your answer: $\underline{\qquad\qquad\qquad\qquad\qquad\qquad\qquad}$
  \vspace{0.4in}

\item Suppose $f$ is a polynomial with degree at least one and
  positive leading coefficient. Consider the function $g(x) :=
  \arctan(f(x))$. What can we say about the horizontal asymptotes of
  the graph $y = g(x)$? {\em Last year: $22/28$ correct}

  \begin{enumerate}[(A)]
  \item The horizontal asymptote is $y = \pi/2$ both for $x \to +\infty$
    and for $x \to -\infty$, regardless of $f$.
  \item The horizontal asymptote is $y = \pi/2$ for $x \to +\infty$ and
    $-\pi/2$ for $x \to -\infty$, regardless of $f$.
  \item The horizontal asymptote is $y = \pi/2$ for $x \to +\infty$,
    and as $x \to -\infty$, it is $y = \pi/2$ if $f$ has even degree
    and $y = -\pi/2$ if $f$ has odd degree.
  \item The horizontal asymptote is $y = f(\pi/2)$ both for $x \to
    +\infty$ and for $x \to -\infty$.
  \item The horizontal asymptote is $y = f(\pi/2)$ for $x \to +\infty$
    and as $x \to -\infty$, it is $y = f(\pi/2)$ if $f$ has even
    degree and $y = f(-\pi/2)$ if $f$ has odd degree.
  \end{enumerate}

  \vspace{0.1in}
  Your answer: $\underline{\qquad\qquad\qquad\qquad\qquad\qquad\qquad}$
  \vspace{0.4in}

\item Consider the function $f(x) := \arcsin(\sin x)$ on the domain
  $[\pi/2,3\pi/2]$. Which of the following is $f(x)$ equal to on that
  domain? {\em Last year: $20/28$ correct}

  \begin{enumerate}[(A)]
  \item $\pi + x$
  \item $\pi - x$
  \item $x - \pi$
  \item $(3\pi/2) - x$
  \item $x - (\pi/2)$
  \end{enumerate}

  \vspace{0.1in}
  Your answer: $\underline{\qquad\qquad\qquad\qquad\qquad\qquad\qquad}$
  \vspace{0.4in}

\item Consider the function $f(x) := \arccos(\sin x)$ on all of
  $\R$. What can we say about the function $f$? {\em Last year:
  $21/28$ correct}

  \begin{enumerate}[(A)]
  \item $f$ is periodic, continuous, and piecewise linear.
  \item $f$ is periodic and continuous but is not piecewise linear.
  \item $f$ is continuous and piecewise linear but not periodic.
  \item $f$ is periodic but not continuous.
  \item $f$ is continuous but not periodic or piecewise linear.
  \end{enumerate}

  \vspace{0.1in}
  Your answer: $\underline{\qquad\qquad\qquad\qquad\qquad\qquad\qquad}$
  \vspace{0.4in}

\end{enumerate}

\end{document}
