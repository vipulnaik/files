\documentclass[a4paper]{amsart}

%Packages in use
\usepackage{fullpage, hyperref, vipul}

%Title details
\title{Why I joined CMI}
\author{Vipul Naik}
\thanks{\copyright Vipul Naik, B.Sc. (Hons) Math and C.S., Chennai Mathematical Institute}

%List of new commands

\makeindex

\begin{document}
\maketitle
%\tableofcontents

\begin{abstract}
  I have often been asked by people why I joined Chennai Mathematical
  Institute (CMI) for its National Undergraduate Programme in
  Mathematics and Computer Science, forsaking other options such as
  the Indian Institute of Technology and the Indian Statistical
  Institute. Here, I describe the important factors that led to my
  decision to join CMI.
\end{abstract}

\section{A little about my academic background}

\subsection{Schooling}

I did my schooling (starting from Nursery right until 12th standard)
at Delhi Public School, Noida. The school is affiliated to the Central
Board of Secondary Education, and is well-known (though not
outstanding) for its achievements both in academics and in other
areas. At the time when I was in my school, it was undergoing radical
transformations and improvements in terms of competition-orientation.
However, awareness regarding Olympiads was still fairly low. Before
me, no student from my school had made it through the Indian National
Mathematical Olympiad, and I don't know whether anybody ever made it
through the Regional Mathematical Olympiad.

\subsection{Coaching institutes}

In my +2 years (standards 11 and 12) I attended Vidyamandir
classes\footnote{\url{http:///www.vidyamandirclasses.com}} to improve
my basics in Physics, Chemistry and Mathematics. Vidyamandir Classes
is a coaching institute for the Indian Institute of Technology Joint
Entrance Examination (IIT-JEE) and as such, focusses on
problem-solving techniques. Although the Vidyamandir administration
was not aware of and did not attach too much importance to Olympiads,
a large fraction of the students from Delhi who qualified for
Olympiads in Mathematics, Physics and Chemistry, were from Vidyamandir
Classes.  Thus, Vidyamandir Classes was an important means for me to
network with others interested in Olympiad mathematics.

\subsection{Olympiads}

I had heard of the Olympiads since a very young age, but there was no
particular awareness about Olympiads in my school. The Olympiad
programme in Delhi was (and is) coordinated by Dr. Amitabh Tripathi,
who works at the Indian Institute of Technology, Delhi. The earlier
coordinator for this event was Dr. S. P. Arya. Dr. Arya, after leaving
the official coordination for the Olympiad programme, has continued to
organize Olympiads for junior classes, such as EMO, SMO, and
JMO. These Olympiads are of a somewhat different flavour from the RMO
(Regional Mathematical Olympiad\footnote{\url{http://en.wikipedia.org/wiki/Regional_Mathematical_Olympiad}}).

Thanks to this confusion, some students (including me) were under the
false impression that the EMO, JMO and SMO were initial steps leading up
to the RMO. Luckily, however, through an indirect personal contact of
my mother's, I came to hear from Professor Rajeeva L. Karandhikar
(then working at Indian Statistical Institute, Delhi). At the time,
Professor Karandhikar was the National Coordinator for the Olympiad
programme, and he informed me that the RMO is independent of contests
organized by Dr. Arya. He also said that people can and do give the
RMO even before 11th standard (which is the highest standard for
unrestricted competition).  I had not seriously considered the
possibility of appearing for the RMO before 11th standard, and
unfortunately, I did not appear in 10th standard even after hearing
from Professor Karandhikar. This was due to a clash with the National
Talent Search
Examination\footnote{\url{http://en.wikipedia.org/wiki/National_Talent_Search_Examination}}
for which I had been preparing and because I felt I could not prepare
sufficiently for the Olympiad in such a short span of time.

At the end of 10th standard, I visited Dr. Amitabh Tripathi at Indian
Institute of Technology, Delhi to purchase the book {\em Mathematical
Circles} from him. Dr. Tripathi strongly urged me to prepare for the
Olympiads.

In 11th standard, I cleared the Regional Mathematical Olympiad, coming
first in Delhi region. I followed this up by clearing the Indian
National Mathematical
Olympiad\footnote{\url{http://en.wikipedia.org/wiki/INMO}}, coming 4th
across India. I then proceeded to make it through the rigourous
selection procedure at the International Mathematical Olympiad
Training Camp\footnote{\url{http://en.wikipedia.org/wiki/IMOTC}} and
thus went for the IMO in 2003. I went to the IMO again in 2004 based on good
performance in IMOTC 2004.

I have chronicled my Olympiad-related experiences at my blog:

\url{http://olympiadsandi.blogspot.com}

\section{My decision process, sources and factors}

\subsection{Before the camp}

Keen to pursue study and research in mathematics, I sought inputs on
good places for pursuing the subject from as many sources as
possible. Most sources pointed to two names -- the B.Sc. (Hons)
programme at Chennai Mathematical Institute (CMI) and the B.Mat.
programme at Indian Statistical Institute (ISI), Bangalore.

\begin{enumerate}

\item Professor Shiva Shankar, who had been a batchmate of my mother
  in the B.Tech. programme at IIT, Delhi, shifted from his job at the
  Indian Institute of Technology, Powai, to Chennai Mathematical
  Institute.  He also informed my mother of an undergraduate programme
  for mathematics that CMI has recently started.

\item Professor Rajeeva L. Karandhikar, at the time I consulted
  him regarding Olympiads, said that Chennai Mathematical Institute (CMI)
  and Indian Statistical Institute (ISI) Bangalore, were the two top
  places for pursuing an undergraduate mathematics programme within India.

\item As topper of a privately organized competition called the National
  Science Olympiad, I was invited to a dinner with the Minister for Science
  and Technology. There, one of the Secretaries, Dr. Ramamurthy, also
  said that the two best places for pursuing mathematics in India
  are CMI and ISI Bangalore.

\end{enumerate}

\subsection{During and after the 2003 camp}

In the IMO Training Camp, I discussed future prospects for studying
mathematics both with students and with teachers. The Mathematical
Olympiad Cell teachers, including Dr. C.R. Pranesachar and
Dr. B. J. Venkatachala, strongly recommended Chennai Mathematical
Institute and Indian Statistical Institute, Bangalore. They cited past students
with Olympiad background who had joined both the places.

In the 2003 International Mathematical Olympiad, one of my teammates
from India was Swarnendu Datta. This was Swarnendu's fourth time at
the IMO, and he was well-known for his deep passion and interest in
mathematics. Swarnendu had initially been in favour of joining Indian
Statistical Institute, Bangalore. However, at the International
Olympiad in Informatics Training Camp (IOITC), faculty members from
CMI such as Madhavan Mukund and K. Narayan Kumar (who are the main organizers of the IOITC)
convinced Swarnendu to join CMI. Swarnendu promised to write to me after some time
with input on how CMI is as a place.

In September, I got a mail from Swarnendu telling me that CMI was a
good place to join because of the great academic freedom enjoyed by
students and the helpfulness of the faculty. Till that time, I did not have any inside sources about
ISI, Bangalore.

\subsection{2004}

By the end of 2003, I was still undecided on whether to join CMI or ISI, Bangalore. The main factor
in ISI Bangalore's favour was that ISI was a better established place at the time with a full campus and hostel,
whereas CMI was still operating from ten rooms in an office complex in T. Nagar with rented accommodation given
to students. Further, CMI was not a deemed university and its degrees were granted by Madhya Pradesh (Bhoj) Open University.
\footnote{Both these factors are no longer applicable. CMI now has its own campus with hostel and deemed university status has been
  approved for CMI}

One of the main attractions of CMI at the time was that it seemed to
be actively seeking students, unlike ISI Bangalore. This included
policies of direct admission to students attending the International
Mathematical Olympiad Training Camp as well as their attempts to get
students into CMI during the IOITC. Another attraction was the greater
``academic freedom'' that CMI offered its students.

As of February 2004, I was still undecided, so I applied both to CMI
and ISI. Application to CMI was largely a formality, because my
attending the IMOTC gave me direct admission. Application to ISI Bangalore
was through a written test followed by an interview. I gave the written test in mid-May,
still unsure of where I would eventually land up.

\subsection{2004 Training Camp}

In April-May 2004, after having successfully completed my CBSE
examinations, I started reading higher mathematics textbooks,
particularly in algebra. I was getting fascinated with group theory
and with the many variations to the concept of group.

My fascination with Euclidean geometry, which began with Olympiad
preparation, had also grown stronger, and I had started getting
glimpses of the ways in which this related to ``algebraic geometry''
and ``commutative algebra''. I was curious to learn more on these
areas.

During the 2004 IMO Training Camp, I talked to some visiting
mathematics teachers, including a person from ISI Delhi and the late
Professor C Musili (University of Hyderabad).  They both recommended
CMI as being more tuned to my areas of interest. They also said that
with my proven mathematical background and an education from CMI, I
had a very high chance of getting admitted for mathematics research to a top place
in India such as Tata Institute of Fundamental Research.

\subsection{The final decision}

Apart from the inputs I received regarding mathematics at CMI, another
major factor that influenced me to join CMI was its strong computer
science programme. I had been interested in algorithms and coding,
though I had not devoted too much time to these areas at high school. The fact
that the main trainers and organizers for the Indian contingent to the IMO Training Camp,
as well as the strong mix of computer science courses at CMI, convinced me that an education
in CMI will give me a good foundation in both mathematics and computer science.

I had put all these factors together by the middle of June, 2004 and
decided to join CMI. Accordingly, I sent CMI a confirmation of my
admission. Although I qualified for the interview to ISI Bangalore, I
decided not to attend the interview as I had already made up my mind
on CMI.





\printindex

\end{document}
