\documentclass[10pt]{amsart}

%Packages in use
\usepackage{fullpage, hyperref, vipul, enumerate}

%Title details
\title{Class quiz solutions: February 3: Differential equations}
\author{Math 153, Section 55 (Vipul Naik)}
%List of new commands

\begin{document}
\maketitle

\section{Performance review}

$10$ people took this $8$-question quiz. The score distribution was as
follows:

\begin{itemize}
\item Score of $4$: $3$ people.
\item Score of $5$: $1$ person.
\item Score of $6$: $2$ people.
\item Score of $7$: $4$ people.
\end{itemize}

The mean score was $5.18$. The question wise answers are given
below. {\em I strongly recommend reviewing all solutions, even for
questions that you got right.}

\begin{enumerate}
\item Option (D): $5$ people
\item Option (A): $10$ people
\item Option (B): $7$ people
\item Option (B): $9$ people
\item Option (A): $7$ people
\item Option (C): $9$ people
\item Option (A): $7$ people
\item Option (E): $3$ people
\end{enumerate}
\section{Solutions}

\begin{enumerate}
\item It takes time $T$ for $1/10$ of a radioactive substance to
  decay. How much does it take for $3/10$ of the same substance to
  decay?
  
  \begin{enumerate}[(A)]
  \item Between $T$ and $2T$
  \item Between $2T$ and $3T$
  \item Exactly $3T$
  \item Between $3T$ and $4T$
  \item Between $4T$ and $5T$
  \end{enumerate}

  {\em Answer}: Option (D)

  {\em Explanation}: In time $T$, the material reduces to $0.9$ of its
  original value. In time $3T$, it reduces to $0.9^3 = 0.729$ of its
  original value. In time $4T$, it reduces to $0.9^4 = 0.6561$ of its
  original value. The time taken for $3/10$ to decay, which means that
  it must reduce to $0.7$ of its original value, is thus between $3T$
  and $4T$.

  {\em Intuitive rationale}: The time taken for
  the first $1/10$ to decay is less than the time taken for the next
  $1/10$ to decay, because the next $1/10$ is, as a fraction, $1/9$ of
  what is left. Thus, the total time taken for $2/10$ to decay is
  slightly more than twice the time taken for $1/10$ to
  decay. Reasoning similarly, we see that the total time taken for
  $3/10$ to decay is slightly more than thrice the time taken for
  $1/10$ to decay.

  {\em Performance review}: $5$ out of $10$ got this. $4$ chose (E),
  $1$ chose (C).

  {\em Historical note (last year)}: $22$ out of $26$ people got this
  correct. $2$ people chose (A) and $1$ person each chose (B) and
  (C). Many people did lengthy calculations involving $\ln$.

  {\em Action point}: Please make sure you understand the {\em
  intuitive rationale} presented above, so that you can answer this
  question faster.

\item Suppose a function $f$ satisfies the differential equation
  $f''(x) = 0$ for all $x \in \R$. Which of the following is true
  about $\lim_{x \to \infty} f(x)$ and $\lim_{x \to -\infty} f(x)$?

  \begin{enumerate}[(A)]
  \item If either limit is finite, then both are finite and they are
    equal. Otherwise, both the limits are infinities of opposite
    signs.
  \item If either limit is finite, then both are finite and they are
    equal. Otherwise, both the limits are infinities of the same sign.
  \item One of the limits is finite and the other is infinite.
  \item Both the limits are finite and unequal.
  \item Both the limits are infinite but they may be of the same or of
    opposite signs.
  \end{enumerate}

  {\em Answer}: Option (A)

  {\em Explanation}: Solving, we see that $f(x)$ is a function of the
  form $ax + b$, where $a$ and $b$ are constants. There are three
  cases: $a = 0$, in which case $f$ is a constant function, $a > 0$,
  in which case $f$ approaches $+\infty$ as $x \to \infty$ and
  approaches $-\infty$ as $x$ approaches $-\infty$, and $a < 0$, in
  which case $f$ approaches $-\infty$ as $x$ approaches $+\infty$ and
  approaches $-\infty$ as $x$ approaches $+\infty$.

  {\em Performance review}: All $10$ got this.

  {\em Historical note (last year)}: $13$ out of $26$ people got this
  correct. $6$ people chose (B) and $7$ people chose (E). Of the
  people who chose (B), some seem to have mistakenly considered the
  general solution to be quadratic rather than linear.
\item For $y$ a function of $x$, consider the differential equation
  $(y')^2 - 3yy' + 2y^2 = 0$. What is the description of the {\bf
  general solution} to this differential equation?

  \begin{enumerate}[(A)]
  \item $y = C_1e^x + C_2e^{2x}$, where $C_1$ and $C_2$ are arbitrary
    real numbers.
  \item $y = C_1e^x + C_2e^{2x}$, where $C_1$ and $C_2$ are real
    numbers satisfying $C_1C_2 = 0$ (i.e., at least one of them is
    zero)
  \item $y = C_1e^x + C_2e^{2x}$, where $C_1$ and $C_2$ are real
    numbers satisfying $C_1 + C_2 = 0$.
  \item $y = C_1e^x + C_2e^{2x}$, where $C_1$ and $C_2$ are real
    numbers satisfying $C_1C_2 = 1$.
  \item $y = C_1e^x + C_2e^{2x}$, where $C_1$ and $C_2$ are real
    numbers satisfying $C_1 + C_2 = 1$.
  \end{enumerate}

  {\em Answer}: Option (B)

  {\em Explanation}: Factorize to obtain:

  $$(y' - y)(y' - 2y) = 0$$

  Thus, either $y' = y$ or $y' = 2y$. Note that for both these
  solutions to hold together, we must have $y = 0$ at some point, in
  which case it is identically zero. Thus, it cannot shift from one
  solution to the other. So, either $y' = y$ identically or $y' = 2y$
  identically.

  In case $y' = y$ identically, we get $y = C_1e^x$ and in case $y' =
  2y$ identically, we get $y = C_2e^{2x}$. The general solution can be
  written as $C_1e^x + C_2e^{2x}$, with the proviso that at least one
  among $C_1$ and $C_2$ is zero.

  {\em Performance review}: $7$ out of $10$ got this. $1$ each chose
  (A), (C), and (E).

  {\em Historical note (last year)}: $12$ out of $26$ people got this
  correct. $8$ people chose (C), $4$ people chose (A), $1$ person
  chose (E), and $1$ person left the question blank.

\item Suppose $F(t)$ represents the number of gigabytes of disk space
  that can be purchased with one dollar at time $t$ in commercially
  available disk drive formats (not adjusted for inflation). Empirical
  observation shows that $F(1980) \approx 5 * 10^{-6}$, $F(1990) \approx
  10^{-4}$, $F(2000) \approx 10^{-1}$, and $F(2010) \approx 10$. From
  these data, what is a good estimate for the ``doubling time'' of
  $F$, i.e., the time it takes for the number of gigabytes
  purchaseable with a dollar to double?
  \begin{enumerate}[(A)]
  \item Between $6$ months and $1$ year.
  \item Between $1$ year and $2$ years.
  \item Between $2$ years and $4$ years.
  \item Between $4$ years and $5$ years.
  \item Between $5$ years and $6$ years.
  \end{enumerate}

  {\em Answer}: Option (B)

  {\em Explanation}: From the given data, the amount by which $F$
  multiplies in ten years is roughly $100$. Note that the first
  doubling time is about $20$, the next one is about $1000$, and the
  next one is $100$. That's just the way real-world data is messy!

  Overall, it seems to be greater than $30$ (although the 1980-1990
  period comes slightly less than that) and less than $1000$.

  The first period (1980-1990) could be a little misleading in this
  sense. To get the best estimate, it makes sense to look at a longer
  time period, so looking at the overall time period of 30 years from
  1980 to 2010 gives a total multiplication of $2 * 10^6$ over 30
  years, which is closest to multiplication by $10^2$ every $10$
  years.

  If the doubling time is $1$ year, then in ten years, $F$ would
  multiply by $2^{10} = 1024$, which is too lrage. If the doubling
  time is $2$ years, then in ten years, $F$ would multiply by $2^5 =
  32$, which is too small. The right doubling time is likely to
  therefore be between $1$ and $2$ years.
  
  {\em Real world thinking}: Do you remember what USB drives, external
  disk drives, etc. used to cost two years ago per GB? Compare those
  costs with today. Do you remember how much disk space was there in a
  typical iPod, iPhone, or other smartphone? Compare that disk space
  with today. Do you see the doubling?

  By the way, this is related to (but not the same as)``Kryder's law''
  which in turn is analogous to Moore's law.

  {\em Performance review}: $9$ out of $10$ got this. $1$ chose (C).

  {\em Historical note (last year)}: $10$ out of $26$ people got this
  correct. $7$ people chose (A) (mild optimism!), $6$ people chose (C)
  (mild pessimism!), $2$ people chose (E) (superstrong pessimism!),
  and $1$ person chose (D) (strong pessimism!).

  {\em Action point}: This is a real-life question with real-world
  data! These are the kinds of questions for which you should have an
  intuitive feel. 
\item The size $S$ of an online social network satisfies the
  differential equation $S'(t) = kS(t)(1 - (S(t))/(W(t)))$ where
  $W(t)$ is the world population at time $t$. Suppose $W(t)$ itself
  satisfies the differential equation $W'(t) = k_0W(t)$ where $k_0$ is
  positive but much smaller than $k$. How would we expect $S$ to
  behave, assuming that initially, $S(t)$ is positive but much smaller
  than $W(t)$?

  \begin{enumerate}[(A)]
  \item It initially appears like an exponential function with
    exponential growth rate $k$, but over time, it slows down to
    (roughly) an exponential function with exponential growth rate
    $k_0$.
  \item It initially appears like an exponential function with
    exponential growth rate $k_0$, but over time, it speeds up to
    (roughly) an exponential function with exponential growth rate $k$.
  \item It behaves roughly like an exponential function with growth
    rate $k_0$ for all time.
  \item It behaves roughly like an exponential function with growth
    rate $k$ for all time.
  \item It initially behaves like an exponential function with
    exponential growth rate $k$ but then it starts declining.
  \end{enumerate}

  {\em Answer}: Option (A)

  {\em Explanation}: Here is a conceptual explanation. Initially, the
  growth of the social network is not directly or visibly constrained
  by the size of the world population. The factor $1 - (S(t))/(W(t))$
  is very close to $1$ because $S(t)$ is much smaller than
  $W(t)$. Thus, the differential equation is approximately $S'(t)
  \approx kS(t)$, which is exponential with exponential growth rate
  $k$.

  When $S$ starts becoming comparable to $W$, then $1 - S(t)/W(t)$
  becomes notably smaller than $1$. The asymptotic steady state would
  occur when $1 - S(t)/W(t) = k_0/k$, i.e., $S(t) = W(t)(1 -
  (k_0/k))$. If this state is achieved, then we would get $S'(t) =
  k_0S(t)$, and also $W'(t) = k_0W(t)$. Thus, the size of the social
  network and the world population are growing at the same exponential
  rate, which means that the social network is used by a constant
  fraction of the world's population.

  This equilibrium steady state will not in practice be achieved in
  finite time, but the asymptotic tendency will be to approach
  this. Note that the smaller $k_0$ is compared to $k$, the larger the
  equilibrium fraction $S/W$. If $k_0 = 0$ (so world population is
  static), then $S/W \to 1$.

  {\em Real world thinking}: For instance, think of Facebook, which
  opened at Harvard in February 2004. Here, the $k$ of Facebook is
  much higher than the $k_0$ for world population, so Facebook's
  initial growth was viral, reaching about 150,000 in about three
  months. Then the (exponential) growth rate sort of slowed down, so
  Facebook reached a million users by about November 2004. If the same
  (or even a slightly lower) {\em exponential} growth rate had
  continued, Facebook would have already saturated the human
  population well before the end of 2009, and would have had to start
  looking at non-human ``people'' to maintain exponential growth.

  Things have been (by and large) getting slower when measured in
  terms of {\em exponential} growth rates, though faster in terms of
  {\em linear} growth rates. In other words, Facebook's number of new
  users per month is much higher today than it was in 2004, but its
  {\em proportion} of new users per month is much lower (less than
  $5\%$). Now that Facebook has about $12\%$ of the world's population,
  it may soon be getting to the stage where the rate of Facebook
  growth is limited by the rate of population growth.

  {\em Performance review}: $7$ out of $10$ got this. $1$ each chose
  (B), (D), and (E).

  {\em Historical note (last year)}: $11$ out of $26$ people got this
  correct. $8$ people chose (B), $3$ people chose (E), $2$ people
  chose (D), $1$ person chose (C), and $1$ person left the question
  blank.

\item Suppose the growth of a population $P$ with time is described by
  the equation $dP/dt = aP^{1 - \beta}$ with $a > 0$ and $0 < \beta <
  1$. What can we say about the nature of the population as a function
  of $t$, assuming that the population at time $0$ is positive?

  \begin{enumerate}[(A)]
  \item The population grows as a sub-linear power function of $t$,
    i.e., roughly like $t^\gamma$ where $0 < \gamma < 1$.
  \item The population grows as a linear power function of $t$, i.e.,
    roughly like $t$.
  \item The population grows as a superlinear power function of $t$,
    i.e., roughly like $t^\gamma$ where $\gamma > 1$.
  \item The population grows like an exponential function of $t$,
    i.e., roughly like $e^{kt}$ for some $k > 0$.
  \item The population grows super-exponentially, i.e., it eventually
  surpasses any exponential function.
  \end{enumerate}
  
  {\em Answer}: Option (C)

  {\em Explanation}: Rearranging, we get:

  $$P^{\beta - 1} dP = dt$$

  Integrating both sides, we get:

  $$P^\beta/\beta = t + C$$

  Rearranging, we get:

  $$P = (\beta(t + C))^{1/\beta}$$

  Since $0 < \beta < 1$, $1/\beta > 1$. This form most closely matches
  (C), with $\gamma = 1/\beta$.

  {\em Performance review}: $9$ out of $10$ got this. $1$ chose (A).

  {\em Historical note (last year)}: $8$ out of $26$ students got this
  correct. $9$ people chose (A), which is probably because they got to
  $P^\beta/\beta = t + C$ but failed to rearrange to express $P$ in
  terms of $t$. $4$ people each chose (B) and (D) and $1$ person left
  the question blank.

  {\em Action point}: Please consider re-attempting this problem prior
  to reviewing course material for the next midterm or final.
\item Suppose the growth of a population $P$ with time is described by
  the equation $dP/dt = aP^{1 + \theta}$ with $0 < \theta$ and $a >
  0$. What can we say about the nature of the population as a function
  of $t$, assuming that the population at time $0$ is positive?

  \begin{enumerate}[(A)]
  \item The population approaches infinity in finite time, and the
    differential equation makes no sense beyond that.
  \item The population increases at a decreasing rate and approaches a
    horizontal asymptote, i.e., it proceeds to a finite limit as time
    approaches infinity.
  \item The population grows linearly.
  \item The population grows super-linearly but sub-exponentially.
  \item The population grows exponentially.
  \end{enumerate}

  {\em Answer}: Option (A)

  {\em Explanation}: Rearranging, we get:

  $$\int P^{-\theta - 1} \, dP = \int dt$$

  Integrating from time $0$, we get:

  $$\frac{P(0)^{-\theta} - P(t)^{-\theta}}{\theta} = t$$

  Thus, we get:

  $$P(t)^{-\theta} = P(0)^{-\theta} - t\theta$$

  Thus, we get:

  $$P(t) = [P(0)^{-\theta} - t\theta]^{-1/\theta}$$

  In particular, as $t \to P(0)^{-\theta}/\theta$, $P(t) \to \infty$.

  Using the specific value $\theta = 1$ may make the preceding
  discussion easier to follow.

  {\em Performance review}: $7$ out of $10$ got this. $2$ chose (B),
  $1$ chose (D).

  {\em Historical note (last year)}: $3$ out of $26$ students got this
  correct. $10$ people chose (B) (which would be sort of correct, if
  it weren't the case that the population had already gone off to
  infinity), $7$ people chose (E), $5$ people chose (D), and $1$
  person chose (C).

  {\em Action point}: Please consider re-attempting this problem
  during review for the next midterm and final.

\item Let $r(t)$ denote the fractional growth rate per annum in per
  capita income, which we denote by $I(t)$. In other words, $r(t) =
  I'(t)/I(t)$, measured in units of (per year). It is observed that,
  over a certain time period, $r'(t) = kr(t)$ for a positive constant
  $k$. Assuming that the initial values of $I(t)$ and $r(t)$ are
  positive, what best describes the nature of the function $I(t)$?

  \begin{enumerate}[(A)]
  \item $I(t)$ is a linear function of $t$, i.e., per capita income is
    getting incremented by a constant {\em amount} (rather than a
    constant proportion).
  \item $I(t)$ is a super-linear but sub-exponential function of $t$,
    i.e., per capita income is rising, but less than exponentially.
  \item $I(t)$ is an exponential function of $t$, i.e., per capita
    income is rising by a constant proportion per year.
  \item $I(t)$ is a super-exponential function of $t$ but slower than
    a doubly exponential function of $t$.
  \item $I(t)$ is a doubly exponential function of $t$.
  \end{enumerate}

  {\em Answer}: Option (E)
  
  {\em Explanation}: Intuitively, the exponential rate of growth is
  itself growing exponentially, so the overall growth is doubly
  exponential.

  Formally, $r(t) = r(0)e^{kt}$. Then, we have:

  $$\frac{dI}{I dt} = r(t)$$

  Thus, we get:

  $$\frac{dI}{Idt} = r(0)e^{kt}$$

  Rearranging, we get:

  $$\frac{dI}{I} = r(0)e^{kt} \, dt$$

  Integrating from $0$ to $t$, we get:

  $$\ln(I(t)/I(0)) = \frac{r(0)(e^{kt} - 1)}{k}$$

  Exponentiating, we get:

  $$I(t) = I(0)\exp[\frac{r(0)(e^{kt} - 1)}{k}]$$

  This is doubly exponential in $t$.

  {\em Real world thinking}: This is a very important question since
  the answer to it reflects the way you think about growth. When you
  see a {\em constant} growth rate for income measured in percentage
  terms, that means that income is growing exponentially. When the
  growth rate {\em itself} is growing exponentially, then income is
  growing doubly exponentially.

  Does doubly exponential growth occur in the real world? Possibly,
  but one reason why it goes unnoticed is that the exponential rate of
  growth of the exponential rate of growth is too slow and is hidden
  by seasonal fluctuations. Taking the long view, we see that rates of
  growth {\em have} increased. Annual per capita income growth before
  1800 was less than $1\%$ almost everywhere in the world, now it is
  $2\%$ or higher in developed countries and often more than $5\%$ in
  developing countries. Similarly, the ``doubling time'' of a number
  of technologies (Moore's law-related) has been falling, albeit very
  slowly. It may not be the case, however, that this double
  exponentiality will be a continuing feature.

  {\em Performance review}: $3$ out of $10$ got this. $5$ chose (B),
  $2$ chose (D).

  {\em Historical note (last year)}: $2$ people got this correct, $5$
  people chose option (D), $10$ people chose option (C) (pessimists!),
  $6$ people chose option (B) (strong pessimists)!, $2$ people chose
  option (A) (super-strong pessimists!), and $1$ person left the
  question blank.

  {\em Action point}: Get some real world intuition! Please re-attempt
  this problem some time and review the solution.
\end{enumerate}

\end{document}
