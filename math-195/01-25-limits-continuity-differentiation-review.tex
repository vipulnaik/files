\documentclass[10pt]{amsart}

%Packages in use
\usepackage{fullpage, hyperref, vipul, enumerate}

%Title details
\title{Take-home class quiz: due Friday January 25: Limits, continuity, differentiation review}
\author{Math 195, Section 59 (Vipul Naik)}
%List of new commands

\begin{document}
\maketitle

Your name (print clearly in capital letters): $\underline{\qquad\qquad\qquad\qquad\qquad\qquad\qquad\qquad\qquad\qquad}$

These questions are all related to single variable
calculus. Specifically, these are some of the harder questions from
material typically covered in Math 151/152. I've given these questions
in past quizzes/tests in Math 152 and Math 153 and the scores
indicated here are the scores in appearances of these questions in
previous quizzes/tests.

{\bf PLEASE FEEL FREE TO DISCUSS THESE QUESTIONS WITH OTHERS, BUT YOUR
FINAL ANSWERS SHOULD BE THE ANSWERS YOU ARE PERSONALLY CONVINCED
ABOUT}.

\begin{enumerate}

\item Which of the following statements is {\bf always true}? {\em
  Earlier scores: $2/11$, $9/16$, $16/28$}

  \begin{enumerate}[(A)]
  \item The range of a continuous nonconstant function on an open
    bounded interval (i.e., an interval of the form $(a,b)$) is an
    open bounded interval (i.e., an interval of the form $(m,M)$).
  \item The range of a continuous nonconstant function on a closed
    bounded interval (i.e., an interval of the form $[a,b]$) is a
    closed bounded interval (i.e., an interval of the form $[m,M]$).
  \item The range of a continuous nonconstant function on an open
    interval that may be bounded or unbounded (i.e., an interval of
    the form $(a,b)$,$(a,\infty)$, $(-\infty,a)$, or
    $(-\infty,\infty)$), is also an open interval that may be bounded
    or unbounded.
  \item The range of a continuous nonconstant function on a closed
    interval that may be bounded or unbounded (i.e., an interval of
    the form $[a,b]$, $[a,\infty)$, $(-\infty,a]$, or
    $(-\infty,\infty)$) is also a closed interval that may be bounded
    or unbounded.
  \item None of the above.
  \end{enumerate}

  \vspace{0.05in}
  Your answer: $\underline{\qquad\qquad\qquad\qquad\qquad\qquad\qquad}$
  \vspace{0.05in}

\item For which of the following specifications is there {\bf no
  continuous function} satisfying the specifications? {\em Earlier
  score: $7/14$, $21/28$}

  \begin{enumerate}[(A)]
  \item Domain $(0,1)$ and range $(0,1)$
  \item Domain $[0,1]$ and range $(0,1)$
  \item Domain $(0,1)$ and range $[0,1]$
  \item Domain $[0,1]$ and range $[0,1]$
  \item None of the above, i.e., we can get a continuous function for
    each of the specifications.
  \end{enumerate}

  \vspace{0.05in}
  Your answer: $\underline{\qquad\qquad\qquad\qquad\qquad\qquad\qquad}$
  \vspace{0.05in}

\item Suppose $f$ is a continuously differentiable function from the
  open interval $(0,1)$ to $\R$. Suppose, further, that there are
  exactly $14$ values of $c$ in $(0,1)$ for which $f(c) = 0$. What can
  we say is {\bf definitely true} about the number of values of $c$ in
  the open interval $(0,1)$ for which $f'(c) = 0$? {\em Earlier scores:
  $7/15$, $19/28$}

  \begin{enumerate}[(A)]
  \item It is at least $13$ and at most $15$.
  \item It is at least $13$, but we cannot put any upper bound on it
    based on the given information.
  \item It is at most $15$, but we cannot put any lower bound (other
  than the meaningless bound of $0$) based on the given information.
  \item It is at most $13$.
  \item It is at least $15$.
  \end{enumerate}

  \vspace{0.05in}
  Your answer: $\underline{\qquad\qquad\qquad\qquad\qquad\qquad\qquad}$
  \vspace{0.05in}
  
\item Consider the function $f(x) := \left \lbrace\begin{array}{rl} x, & 0
  \le x \le 1/2 \\ x - (1/7), & 1/2 < x \le 1 \\\end{array} \right.$. Define by
  $f^{[n]}$ the function obtained by iterating $f$ $n$ times, i.e.,
  the function $f \circ f \circ f \circ \dots \circ f$ where $f$
  occurs $n$ times. What is the smallest $n$ for which $f^{[n]} =
  f^{[n + 1]}$? {\em Earlier scores: $3/16$, $10/28$}

  \begin{enumerate}[(A)]
  \item $1$
  \item $2$
  \item $3$
  \item $4$
  \item $5$
  \end{enumerate}

  
  \vspace{0.05in}
  Your answer: $\underline{\qquad\qquad\qquad\qquad\qquad\qquad\qquad}$
  \vspace{0.05in}

\item Suppose $f$ and $g$ are functions $(0,1)$ to $(0,1)$ that are
  both right continuous on $(0,1)$. Which of the following is {\em
  not} guaranteed to be right continuous on $(0,1)$? {\em Earlier
  scores: $3/11$, $9/14$, $20/28$}

  \begin{enumerate}[(A)]
  \item $f + g$, i.e., the function $x \mapsto f(x) + g(x)$
  \item $f - g$, i.e., the function $x \mapsto f(x) - g(x)$
  \item $f \cdot g$, i.e., the function $x \mapsto f(x)g(x)$
  \item $f \circ g$, i.e., the function $x \mapsto f(g(x))$
  \item None of the above, i.e., they are all guaranteed to be right
    continuous functions
  \end{enumerate}

  \vspace{0.05in}
  Your answer: $\underline{\qquad\qquad\qquad\qquad\qquad\qquad\qquad}$
  \vspace{0.05in}

\item Suppose $f$ and $g$ are increasing functions from $\R$ to
  $\R$. Which of the following functions is {\em not} guaranteed to be
  an increasing function from $\R$ to $\R$? {\em Earlier scores:
  $1/15$, $9/16$, $18/28$}

  \begin{enumerate}[(A)]

  \item $f + g$
  \item $f \cdot g$
  \item $f \circ g$
  \item All of the above, i.e., none of them is guaranteed to be increasing.
  \item None of the above, i.e., they are all guaranteed to be increasing.
  \end{enumerate}

  \vspace{0.05in}
  Your answer: $\underline{\qquad\qquad\qquad\qquad\qquad\qquad\qquad}$
  \vspace{0.05in}

\item Suppose $F$ and $G$ are two functions defined on $\R$ and $k$ is
  a natural number such that the $k^{th}$ derivatives of $F$ and $G$
  exist and are equal on all of $\R$. Then, $F - G$ must be a
  polynomial function. What is the {\bf maximum possible degree} of $F
  - G$?  (Note: Assume constant polynomials to have degree zero) {\em
  Earlier scores: $6/16$, $10/28$}

  \begin{enumerate}[(A)]
  \item $k - 2$
  \item $k - 1$
  \item $k$
  \item $k + 1$
  \item There is no bound in terms of $k$.
  \end{enumerate}

  \vspace{0.05in}
  Your answer: $\underline{\qquad\qquad\qquad\qquad\qquad\qquad\qquad}$
  \vspace{0.05in}

\item Suppose $f$ is a continuous function on $\R$. Clearly, $f$ has
  antiderivatives on $\R$. For all but one of the following
  conditions, it is possible to guarantee, without any further
  information about $f$, that there exists an antiderivative $F$
  satisfying that condition. {\bf Identify the exceptional condition}
  (i.e., the condition that it may not always be possible to
  satisfy). {\em Earlier scores: $3/16$, $10/28$}

  \begin{enumerate}[(A)]
  \item $F(1) = F(0)$.
  \item $F(1) + F(0) = 0$.
  \item $F(1) + F(0) = 1$.
  \item $F(1) = 2F(0)$.
  \item $F(1)F(0) = 0$.
  \end{enumerate}

  \vspace{0.05in}
  Your answer: $\underline{\qquad\qquad\qquad\qquad\qquad\qquad\qquad}$
  \vspace{0.05in}

\item Suppose $F$ is a function defined on $\R \setminus \{ 0 \}$ such
  that $F'(x) = -1/x^2$ for all $x \in \R \setminus \{ 0 \}$. Which of
  the following pieces of information is/are {\bf sufficient} to
  determine $F$ completely? Please see options (D) and (E) before
  answering. {\em Earlier scores: $4/16$, $15/28$}
  \begin{enumerate}[(A)]
  \item The value of $F$ at any two positive numbers.
  \item The value of $F$ at any two negative numbers.
  \item The value of $F$ at a positive number and a negative number.
  \item Any of the above pieces of information is sufficient, i.e., we
    need to know the value of $F$ at any two numbers.
  \item None of the above pieces of information is sufficient.
  \end{enumerate}

  \vspace{0.05in}
  Your answer: $\underline{\qquad\qquad\qquad\qquad\qquad\qquad\qquad}$
  \vspace{0.05in}

\item Suppose $F$ and $G$ are continuously differentiable functions on
  all of $\R$ (i.e., both $F'$ and $G'$ are continuous). Which of the
  following is {\bf not necessarily true}? {\em Earlier scores: $0$,
  $10/16$, $11/28$}

  \begin{enumerate}[(A)]
  \item If $F'(x) = G'(x)$ for all integers $x$, then $F - G$ is a
    constant function when restricted to integers, i.e., it takes the
    same value at all integers.
  \item If $F'(x) = G'(x)$ for all numbers $x$ that are not integers,
    then $F - G$ is a constant function when restricted to the set of
    numbers $x$ that are not integers.
  \item If $F'(x) = G'(x)$ for all rational numbers $x$, then $F - G$
    is a constant function when restricted to the set of rational
    numbers.
  \item If $F'(x) = G'(x)$ for all irrational numbers $x$, then $F -
    G$ is a constant function when restricted to the set of irrational
    numbers.
  \item None of the above, i.e., they are all necessarily true.
  \end{enumerate}
  
  \vspace{0.05in}
  Your answer: $\underline{\qquad\qquad\qquad\qquad\qquad\qquad\qquad}$
  \vspace{0.05in}
\end{enumerate}

\end{document}