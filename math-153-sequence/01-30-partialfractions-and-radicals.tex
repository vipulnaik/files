\documentclass[10pt]{amsart}

%Packages in use
\usepackage{fullpage, hyperref, vipul, enumerate}

%Title details
\title{Class quiz: January 30: Partial fractions and radicals}
\author{Math 153, Section 55 (Vipul Naik)}
%List of new commands

\begin{document}
\maketitle

Your name (print clearly in capital letters): $\underline{\qquad\qquad\qquad\qquad\qquad\qquad\qquad\qquad\qquad\qquad}$

\begin{enumerate}
\item Which of these functions of $x$ is {\em not} elementarily
  integrable? {\em Last year: $22/27$ correct}

  \begin{enumerate}[(A)]
  \item $x\sqrt{1 + x^2}$
  \item $x^2\sqrt{1 + x^2}$
  \item $x(1 + x^2)^{1/3}$
  \item $x\sqrt{1 + x^3}$
  \item $x^2\sqrt{1 + x^3}$
  \end{enumerate}

  \vspace{0.1in}
  Your answer: $\underline{\qquad\qquad\qquad\qquad\qquad\qquad\qquad}$
  \vspace{0.1in}
\item For which of these functions of $x$ does the antiderivative
  necessarily involve {\em both} $\arctan$ {\em and} $\ln$? {\em Last
  year: $21/27$ correct}

  \begin{enumerate}[(A)]
  \item $1/(x + 1)$
  \item $1/(x^2 + 1)$
  \item $x/(x^2 + 1)$
  \item $x/(x^3 + 1)$
  \item $x^2/(x^3 + 1)$
  \end{enumerate}

  \vspace{0.1in}
  Your answer: $\underline{\qquad\qquad\qquad\qquad\qquad\qquad\qquad}$
  \vspace{0.1in}

\item (**) Consider the function $f(k) := \int_1^2 \frac{dx}{\sqrt{x^2 +
  k}}$. $f$ is defined for $k \in (-1,\infty)$. What can we say about
  the nature of $f$ within this interval? {\em Last year: $4/27$ correct}

  \begin{enumerate}[(A)]
  \item $f$ is increasing on the interval $(-1,\infty)$.
  \item $f$ is decreasing on the interval $(-1,\infty)$.
  \item $f$ is increasing on $(-1,0)$ and decreasing on $(0,\infty)$.
  \item $f$ is decreasing on $(-1,0)$ and increasing on $(0,\infty)$.
  \item $f$ is increasing on $(-1,0)$, decreasing on $(0,2)$, and
    increasing again on $(2,\infty)$.
  \end{enumerate}

  \vspace{0.1in}
  Your answer: $\underline{\qquad\qquad\qquad\qquad\qquad\qquad\qquad}$
  \vspace{0.1in}


\item (*) Suppose $F$ is a (not known) function defined on $\R \setminus
  \{ -1,0,1\}$, differentiable everywhere on its domain, such that
  $F'(x) = 1/(x^3 - x)$ everywhere on $\R \setminus \{-1,0,1\}$. For
  which of the following sets of points is it true that knowing the
  value of $F$ at these points {\bf uniquely} determines $F$? {\em
  Last year: $14/27$ correct}

  \begin{enumerate}[(A)]
  \item $\{ -\pi, -e, 1/e,1/\pi \}$
  \item $\{ -\pi/2, -\sqrt{3}/2, 11/17,\pi^2/6 \}$
  \item $\{ -\pi^3/7,-\pi^2/6,\sqrt{13},11/2 \}$
  \item Knowing $F$ at any of the above determines the value of $F$
    uniquely.
  \item None of the above works to uniquely determine the value of
    $F$.
  \end{enumerate}

  \vspace{0.1in}
  Your answer: $\underline{\qquad\qquad\qquad\qquad\qquad\qquad\qquad}$
  \vspace{0.1in}

\item (*) Consider a rational function $f(x) := p(x)/q(x)$ where $p$
  and $q$ are nonzero polynomials and the degree of $p$ is strictly
  less than the degree of $q$. Suppose $q(x)$ is monic of degree $n$
  and has $n$ distinct real roots $a_1,a_2,\dots,a_n$, so $q(x) =
  \prod_{i=1}^n (x - a_i)$. Then, we can write:

  $$f(x) = \frac{c_1}{x - a_1} + \frac{c_2}{x - a_2} + \dots + \frac{c_n}{x - a_n}$$

  for suitable constants $c_i \in \R$. What can we say about the sum
  $\sum_{i=1}^n c_i$? {\em Last year: $12/27$ correct}

  \begin{enumerate}[(A)]
  \item The sum is always $0$.
  \item The sum equals the leading coefficient of $p$.
  \item The sum is $0$ if $p$ has degree $n - 1$. If the degree of $p$
  is smaller, the sum equals the leading coefficient of $p$.
  \item The sum is $0$ if $p$ has degree smaller than $n - 1$. If $p$
    has degree equal to $n - 1$, the sum is the leading coefficient of
    $p$.
  \item The sum is $0$ if $p$ is a constant polynomial. Otherwise, it
    equals the leading coefficient of $p$.
  \end{enumerate}

  \vspace{0.1in}
  Your answer: $\underline{\qquad\qquad\qquad\qquad\qquad\qquad\qquad}$
  \vspace{0.1in}

\item (**) {\em Hard right now, will become easier later}: Suppose $F$
  is a continuously differentiable function whose domain contains
  $(a,\infty)$ for some $a \in \R$, and $F'(x)$ is a rational function
  $p(x)/q(x)$ on the domain of $F$. Further, suppose that $p$ and $q$
  are nonzero polynomials. Denote by $d_p$ the degree of $p$ and by
  $d_q$ the degree of $q$. Which of the following is a {\bf necessary
  and sufficient condition} to ensure that $\lim_{x \to \infty} F(x)$
  is finite? {\em Last year: $3/27$ correct}

  \begin{enumerate}[(A)]
  \item $d_p - d_q \ge 2$
  \item $d_p - d_q \ge 1$
  \item $d_p = d_q$
  \item $d_q - d_p \ge 1$
  \item $d_q - d_p \ge 2$
  \end{enumerate}

  \vspace{0.1in}
  Your answer: $\underline{\qquad\qquad\qquad\qquad\qquad\qquad\qquad}$
  \vspace{0.1in}

  For the remaining questions, we build on the observation: For any
  nonconstant monic polynomial $q(x)$, there exists a finite
  collection of transcendental functions $f_1, f_2, \dots, f_r$ such
  that the antiderivative of any rational function $p(x)/q(x)$, on an
  open interval where it is defined and continuous, can be expressed
  as $g_0 + f_1g_1 + f_2g_2 + \dots + f_rg_r$ where $g_0, g_1, \dots,
  g_r$ are rational functions.

\item (*) For the polynomial $q(x) = 1 + x^2$, what collection of
  $f_i$s works (all are written as functions of $x$)? {\em Last year:
  $15/27$ correct}

  \begin{enumerate}[(A)]
  \item $\arctan x$ and $\ln|x|$
  \item $\arctan x$ and $\arctan(1 + x^2)$
  \item $\ln|x|$ and $\ln(1 + x^2)$ 
  \item $\arctan x$ and $\ln(1 + x^2)$
  \item $\ln|x|$ and $\arctan(1 + x^2)$
  \end{enumerate}

  \vspace{0.1in}
  Your answer: $\underline{\qquad\qquad\qquad\qquad\qquad\qquad\qquad}$
  \vspace{0.1in}

\item (**) For the polynomial $q(x) := 1 + x^2 + x^4$, what is the
  size of the smallest collection of $f_i$s that works? {\em Last
  year: $7/27$ correct}

  \begin{enumerate}[(A)]
  \item $1$
  \item $2$
  \item $3$
  \item $4$
  \item $5$
  \end{enumerate}

  \vspace{0.1in}
  Your answer: $\underline{\qquad\qquad\qquad\qquad\qquad\qquad\qquad}$
  \vspace{0.1in}
\end{enumerate}

\end{document}
