\documentclass[10pt]{amsart}

%Packages in use
\usepackage{fullpage, hyperref, vipul, enumerate}

%Title details
\title{Class quiz solutions: October 21: Max-min problems}
\author{Math 152, Section 55 (Vipul Naik)}
%List of new commands

\begin{document}
\maketitle

\section{Performance review}

{\em Note}: The quiz was submitted in class on October 26.

$12$ students took this quiz. The score distribution was as follows:

\begin{itemize}
\item Score of $3$: $2$ persons.
\item Score of $4$: $5$ persons.
\item Score of $6$: $4$ persons.
\item Score of $7$: $1$ person.
\end{itemize}

The mean score was $4.75$. Here are the problem solutions and problem
wise scores:

\begin{enumerate}
\item Option (B): Everybody
\item Option (B): Everybody
\item Option (D): Everybody
\item Option (A): $6$ people
\item Option (A): $6$ people
\item Option (C): $4$ people
\item Option (C): $5$ people
\end{enumerate}
\section{Solutions}

\begin{enumerate}

\item Consider all the rectangles with perimeter equal to a fixed
  length $p > 0$. Which of the following {\bf is true} for the unique
  rectangle which is a square, compared to the other rectangles?

  \begin{enumerate}[(A)]
  \item It has the largest area and the largest length of diagonal.
  \item It has the largest area and the smallest length of diagonal.
  \item It has the smallest area and the largest length of diagonal.
  \item It has the smallest area and the smallest length of diagonal.
  \item None of the above.
  \end{enumerate}

  {\em Answer}: Option (B)

  {\em Explanation}: We can see this easily by doing calculus, but it
  can also be deduced purely by thinking about how a square and a long
  thin rectangle of the same perimeter compare in terms of area and
  diagonal length.

  {\em Performance review}: Everybody got this correct.

  {\em Historical note (last year)}: Everybody got this correct.

  {\em Historical note (two years ago)}: This question appeared on
  last year's 151 final, and $31$ out of $33$ people got it correct.

\item Suppose the total perimeter of a square and an equilateral
  triangle is $L$. (We can choose to allocate all of $L$ to the
  square, in which case the equilateral triangle has side zero, and we
  can choose to allocate all of $L$ to the equilateral triangle, in
  which case the square has side zero). Which of the following
  statements {\bf is true} for the sum of the areas of the square and
  the equilateral triangle? (The area of an equilateral triangle is
  $\sqrt{3}/4$ times the square of the length of its side).
  \begin{enumerate}[(A)]
  \item The sum is minimum when all of $L$ is allocated to the square.
  \item The sum is maximum when all of $L$ is allocated to the square.
  \item The sum is minimum when all of $L$ is allocated to the
    equilateral triangle.
  \item The sum is maximum when all of $L$ is allocated to the
    equilateral triangle.
  \item None of the above.
  \end{enumerate}

  {\em Answer}: Option (B)

  {\em Quick explanation}: The problem can also be solved
  using the rough heuristic that works for these kinds of problems:
  the maximum occurs when everything is allocated to the most
  efficient use, but the minimum typically occurs somewhere in
  between.

  {\em Full explanation}: Suppose $x$ is the part allocated to the
  square. Then $L - x$ is the part allocated to the equilateral
  triangle. The total area is:

  $$A(x) = x^2/16 + (\sqrt{3}/4)(L - x)^2/9$$

  Differentiating, we obtain:

  $$A'(x) = \frac{x}{8} - \frac{\sqrt{3}}{18} (L - x) = x \left(\frac{1}{8} + \frac{\sqrt{3}}{18} \right) - \frac{\sqrt{3}}{18}L$$

  We see that $A'(x) = 0$ at

  $$x = \frac{L(\sqrt{3}/18)}{(1/8) + (\sqrt{3}/18)}$$

  This number is indeed within the range of possible values of $x$.

  Further, $A'(x) > 0$ for $x$ greater than this and $A'(x) < 0$ for
  $x$ less than this. Thus, this point is a local minimum and the
  maximum must occur at one of the endpoints. We plug in $x = 0$ to
  get $(\sqrt{3}/36)L^2$ and we plug in $x = L$ to get $L^2/16$. Since
  $1/16 > \sqrt{3}/36$, we obtain the the maximum occurs when $x = L$,
  which means that all the perimeter goes to the square.

  {\em Performance review}: Everybody got this correct.

  {\em Historical note (last year)}: $9$ out of $15$ people got this
  correct. $2$ people each chose (E) and (C), $1$ person chose (D),
  and $1$ person left the question blank.

  {\em Historical note}: This question appeared in a 152 midterm last
  year, and $20$ of $29$ people got it correct. This is a somewhat
  better showing than you lot, but that midterm occurred after several
  homeworks, lectures, and two review sessions covering max-min
  problems. Also, in that midterm, option (E) wasn't there, so things
  became a little easier.

\item Suppose $x$ and $y$ are positive numbers such as $x + y =
  12$. For {\bf what values} of $x$ and $y$ is $x^2y$ maximum?

  \begin{enumerate}[(A)]
  \item $x = 3$, $y = 9$
  \item $x = 4$, $y = 8$
  \item $x = 6$, $y = 6$
  \item $x = 8$, $y = 4$
  \item $x = 9$, $y = 3$
  \end{enumerate}

  {\em Answer}: Option (D).

  {\em Quick explanation}: This is a special case of the general
  Cobb-Douglas situation where we want to maximize $x^a(C - x)^b$. The
  general solution is to take $x = Ca/(a + b)$, i.e., to take $x$ and
  $C - x$ in the proportion of $a$ to $b$.

  {\em Full explanation}: We need to maximize $f(x) := x^2(12 - x)$,
  subject to $0 < x < 12$. Differentiating, we get $f'(x) = 3x(8 -
  x)$, so $8$ is a critical point. Further, we see that $f'$ is
  positive on $(0,8)$ and negative on $(8,12)$, so $f$ attains its
  maximum (in the interval $(0,12)$) at $8$.

  {\em Performance review}: Everybody got this correct.

  {\em Historical note (last year)}: $12$ out of $15$ people got this
  correct. $2$ people chose (E) and $1$ person chose (B). Of the
  people who got this correct, some seem to have computed the
  numerical values and others seem to have used calculus. Some who did
  not show any work may have used the general result of the
  Cobb-Douglas situation.
\item Consider the function $p(x) := x^2 + bx + c$, with $x$
  restricted to integer inputs. Suppose $b$ and $c$ are integers. The
  minimum value of $p$ is attained either at a single integer or at
  two consecutive integers. Which of the following is a {\bf
  sufficient condition} for the minimum to occur at two consecutive
  integers?

  \begin{enumerate}[(A)]
  \item $b$ is odd
  \item $b$ is even
  \item $c$ is odd
  \item $c$ is even
  \item None of these conditions is sufficient.
  \end{enumerate}

  {\em Answer}: Option (A)

  {\em Explanation}: The graph of $f$ is symmetric about the
  half-integer axis value $-b/2$. It is an upward-facing parabola. For
  odd $b$, it attains its minimum among integers at the two
  consecutive integers $-b/2 + 1/2$ and $-b/2 - 1/2$. When $b$ is
  even, the minimum is attained uniquely at $-b/2$, which is itself an
  integer. $c$ being odd or even tells us nothing.

  {\em Performance review}: $6$ out of $12$ got this correct. $5$
  chose (E), $1$ chose (B).

  {\em Historical note (last year)}: $4$ out of $15$ people got this
  correct. $8$ people chose (E) and $3$ people chose (B).

  {\em Action point}: While this problem can be solved using calculus,
  it is much easier if you already know and remember important facts
  about the graphs of quadratic functions. Please review basic facts
  about quadratic functions.
\item Consider a hollow cylinder with no top and bottom and total
  curved surface area $S$. What can we say about the {\bf maximum and
  minimum} possible values of the {\bf volume}? (for radius $r$ and height
  $h$, the curved surface area is $2\pi rh$ and the volume is $\pi
  r^2h$).

  \begin{enumerate}[(A)]
  \item The volume can be made arbitrarily small (i.e., as close to
    zero as we desire) and arbitrarily large (i.e., as large as we
    want).
  \item There is a positive minimum value for the volume, but it can
    be made arbitrarily large.
  \item There is a finite maximum value for the volume, but it can be
    made arbitrarily small.
  \item There is both a finite positive minimum and a finite positive
    maximum for the volume.
  \end{enumerate}

  {\em Answer}: Option (A)

  {\em Explanation}: We have $h = S/2\pi r$. Thus, the volume is
  $rS/2$. We see that as $r \to \infty$, the volume goes to infinity,
  and as $r \to 0$, the volume tends to zero. Thus, the volume can be
  made arbitrarily large as well as arbitrarily small.

  {\em Performance review}: $6$ out of $12$ got this correct. $4$
  chose (C), $2$ chose (D).

  {\em Historical note (last year)}: $6$ out of $15$ people got this
  correct. $3$ people chose (B) and $6$ people chose (C).

  {\em Action point}: Please make sure you understand this problem,
  and also how it differs in nature from the next two problems.

\item Consider a hollow cylinder with a bottom but no top and total
  surface area (curved surface plus bottom) $S$. What can we say about
  the {\bf maximum and minimum} possible values of the {\bf volume}?
  (for radius $r$ and height $h$, the curved surface area is $2\pi rh$
  and the volume is $\pi r^2h$).

  \begin{enumerate}[(A)]
  \item The volume can be made arbitrarily small (i.e., as close to
    zero as we desire) and arbitrarily large (i.e., as large as we
    want).
  \item There is a positive minimum value for the volume, but it can
    be made arbitrarily large.
  \item There is a finite maximum value for the volume, but it can be
    made arbitrarily small.
  \item There is both a finite positive minimum and a finite positive
    maximum for the volume.
  \end{enumerate}

  {\em Answer}: Option (C)

  {\em Quick explanation}: We see that the radius cannot be expanded
  too much, otherwise the area of the bottom will itself exceed
  $S$. This puts a constraint on the total volume. On the other hand,
  the cylinder can be made arbitrarily thin and thus have arbitrarily
  small volume.

  {\em Full explanation}: Try it yourself! There is a worked example
  in the book that essentially computes the maximum with specific
  numerical values -- locate it!

  {\em Performance review}: $4$ out of $12$ got this correct. $3$ each
  chose (B) and (D), $2$ chose (A).

  {\em Historical note (last year)}: $8$ out of $15$ people got this
  correct. $1$ person chose (A), $4$ people chose (B), and $2$ people
  chose (D).
\item Consider a hollow cylinder with a bottom and a top and total
  surface area (curved surface plus bottom and top) $S$. What can we
  say about the {\bf maximum and minimum} possible values of the {\bf
  volume}?  (for radius $r$ and height $h$, the curved surface area is
  $2\pi rh$ and the volume is $\pi r^2h$).

  \begin{enumerate}[(A)]
  \item The volume can be made arbitrarily small (i.e., as close to
    zero as we desire) and arbitrarily large (i.e., as large as we
    want).
  \item There is a positive minimum value for the volume, but it can
    be made arbitrarily large.
  \item There is a finite maximum value for the volume, but it can be
    made arbitrarily small.
  \item There is both a finite positive minimum and a finite positive
    maximum for the volume.
  \end{enumerate}

  {\em Answer}: Option (C)

  {\em Quick explanation}: Identical to the previous problem.

  {\em Full explanation}: Try it yourself!

  {\em Performance review}: $5$ out of $12$ got this correct. $4$
  chose (D), $2$ chose (B), and $1$ chose (A).

  {\em Historical note (last year)}: $5$ out of $15$ people got this
  correct. $8$ people chose (D) and $2$ people chose (A).

  {\em Action point}: Please try to understand, both conceptually and
  computationally, why the qualitative conclusion for this problem is
  the same as for the previous problem.
\end{enumerate}

\end{document}