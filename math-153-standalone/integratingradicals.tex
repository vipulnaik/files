\documentclass[10pt]{amsart}
\usepackage{fullpage,hyperref,vipul}
\title{Integrating radicals}
\author{Math 153, Section 55 (Vipul Naik)}

\begin{document}
\maketitle

{\bf Corresponding material in the book}: Section 8.4.

{\bf What students should already know}: The definitions of inverse
trigonometric functions. The differentiation and integration formulas
for these. The differentiation formulas for the straight-up
trigonometric functions.

{\bf What students should definitely get}: The three key integral
formulations: $a^2 - x^2$, $a^2 + x^2$, and $x^2 - a^2$, and the
mechanics of the trigonometric substitution for each. The procedure
for completing the square term in quadratic functions and using this
to integrate functions which have quadratics in denominators or with
half-integer powers.

{\bf What students should hopefully get}: The interpretation in terms
of homogeneous degree, the contours of the relationship with inverse
hyperbolic trigonometry.

\section*{Executive summary}

Words ...

\begin{enumerate}
\item Expressions of the form $a^2 + x^2$ (with $a > 0$) in the
  denominator or under the radical sign suggest the substitution
  $\theta = \arctan(x/a)$. With this substitution, $x = a \tan
  \theta$, $dx = a \sec^2 \theta \, d\theta$, $a^2 + x^2 = a^2 \sec^2
  \theta$, and $\sqrt{a^2 + x^2} = a \sec \theta$. In the end, when
  substituting back, we use $\theta = \arctan(x/a)$, $\tan \theta =
  x/a$, $\sec \theta = \sqrt{a^2 + x^2}/a$, $\cos \theta = a/\sqrt{a^2
  + x^2}$, and $\sin \theta = x/\sqrt{a^2 + x^2}$. The first sentence
  of substitutions is useful when converting the given integrand into
  a trigonometric integrand. The second sentence is useful when
  converting the integrated answer back at the end. (This latter step
  is unnecessary when we are dealing with a definite integral and we
  transform limits simultaneously).
\item For $a^2 - x^2$ under a squareroot, we have a similar
  substitution $\theta = \arcsin(x/a)$. For $x^2 - a^2$, we take
  $\theta = \arccos(a/x)$. It is useful to work out the forward and
  backward substitutions for these. (See the notes for the details of
  these substitutions). {\em It is strongly suggested that you
  internalize both the forward and the backward substitutions to the
  point where they become automatic. Memorization helps, but you
  should also be able to re-derive things on the spot as the need
  arises.}
\item There is a little subtlety in these substitutions. When we take
  $\theta$ as $\arcsin$, we know that $\cos \theta$ is
  nonnegative. Hence, when we simplify $\sqrt{a^2 - x^2}$, we get
  $\sqrt{a^2\cos^2 \theta}$. Because by assumption $a$ is positive,
  and because $\cos \theta$ is nonnegative, we can write the answer as
  $a\cos \theta$. In other words, we know how exactly we can lift off
  the squareroot. Something similar happens when we are dealing with
  the tangent and secant functions: secant is nonnegative on the range
  of arc tangent. Unfortunately, tangent is {\em not} nonnegative on
  the entire range of arc secant, so we need to actually look at the
  region where we are carrying out the integration. In case both the
  upper and lower bounds of integration are greater than $a$, we know
  that we will in fact get $\tan \theta$.
\end{enumerate}

{\em Note}: Some of you may find it useful to draw right triangles, as
suggested in the book, if reading trigonometric ratios off triangles
is easier for you than algebraic manipulation of trigonometric
expressions.

Actions ...

\begin{enumerate}
\item Trigonometric substitutions allow us to integrate things like
  $x^m(a^2 + x^2)^{n/2}$. However, some special cases of these can be
  integrated without resort to trigonometric substitutions. For
  instance, when $n$ is a nonnegative even integer, this is a sum of
  powers of $x$ and can be integrated term wise. Also, if $m$ is odd,
  we can do a $u$-substitution with $u = a^2 + x^2$.
\item Similar remarks apply to expressions involving $\sqrt{a^2 -
  x^2}$ and $\sqrt{x^2 - a^2}$.
\item To apply this or similar techniques to more general quadratics,
  we need to use a technique known as {\em completing the
  square}. Here, we rewrite:

  $$Ax^2 + Bx + C = A(x + (B/2A))^2 + (C - B^2/4A)$$

  The special case where $A = 1$ is given by:

  $$x^2 + Bx + C = (x + (B/2))^2 + (C - (B/2)^2)$$

  Note that the left-over constant term after completing the square is
  $-D/4A$ where $D$ is the discriminant of the quadratic
  polynomial. In the case $A = 1$, when the polynomial has positive
  discriminant, this left-over term is negative, whereas when the
  polynomial has negative discriminant, this left-over term is
  positive. In the latter case, we can write it as the square of
  something. We would thus have written our original polynomial as $(x
  - \beta)^2 + \gamma^2$, whereupon we can make the substitution
  $\theta = \arctan((x - \beta)/\gamma)$ (or directly apply the
  integration formula).
\end{enumerate}

\section{The key idea of substitution}

We are often faced with situations where the integrand is an algebraic
expression that involves a squareroot sign. The key idea is to use a
trigonometric substitution that converts the problem to a
trigonometric integration. We then use the plethora of trigonometric
identities to simplify this integral.

\subsection{Substitutions involving $\sqrt{a^2 - x^2}$}

We first recall the following basic facts that will provide context
for the trigonometric substitutions that follow:

\begin{enumerate}
\item If $\theta = \arcsin(u)$, then $\sin \theta = u$ and $\cos
  \theta = \sqrt{1 - u^2}$. Note that it is the nonnegative squareroot
  because {\em cosine is nonnegative on the range of the arcsine
  function}.
\item We have $\int dx/\sqrt{1 - x^2} = \arcsin(x)$. We obtained this
  result by noting that $\arcsin'(x) = 1/\sin'(\arcsin(x))$ and
  simplifying.
\item In general, if we make the substitution $\theta = \arcsin(u)$ in
  an integration problem, then $du = \cos \theta \, d\theta$, $u = \sin
  \theta$, and $\sqrt{1 - u^2} = \cos \theta$.
\item Even more generally, if we put $\theta = \arcsin(x/a)$ in an
  integration problem (with $a > 0$ a constant), then $dx = a \cos
  \theta \, d\theta$, $x = a \sin \theta$, and $\sqrt{a^2 - x^2} = a \cos
  \theta$. In reverse, $\sin \theta = x/a$, $\cos \theta = \sqrt{1 -
  (x/a)^2} = \sqrt{a^2 - x^2}/a$.
\end{enumerate}

This brings us to the key idea of integration: if we see the
expression $\sqrt{a^2 - x^2}$ in the integrand, we should consider the
substituion $\theta = \arcsin(x/a)$, further getting $dx = a \cos
\theta d\theta$, $x = a \sin \theta$, and $\sqrt{a^2 - x^2} = a\cos
\theta$. For instance, consider:

$$\int \sqrt{a^2 - x^2} \, dx$$

Using the substitution $\theta = \arcsin(x/a)$, we get:

$$\int a \cos \theta (a \cos \theta d \theta) = \int a^2 \cos^2 \theta \, d\theta$$

We can simplify this further, using the well-memorized formula for the
antiderivative of $\cos^2\theta$. We get:

$$a^2\left[\frac{\theta}{2} + \frac{\sin(2\theta)}{4} \right] = a^2\left[\frac{\theta + \sin \theta \cos \theta}{2}\right]$$

Putting $\theta = \arcsin(x/a)$, and using $\sin \theta = x/a$ and
$\cos \theta = \sqrt{1 - (x/a)^2}$, we simplify and obtain:

$$\frac{a^2}{2} \arcsin(x/a) + \frac{x}{2} \sqrt{a^2 - x^2}$$

This should all be familiar to you, since it was a homework problem.

The idea works more generally for half-integer powers of $a^2 -
x^2$. For instance, consider:

$$\int (a^2 - x^2)^{3/2} \, dx$$

After the trigonometric substitution, we obtain:

$$\int a^3 \cos^3 \theta (a \cos \theta) \, d\theta$$

This reduces to integrating $\cos^4 \theta$. We can now go a number of
routes -- we can use the reduction formula to reduce it to the
integral of $\cos^2 \theta$, or we can use a bunch of trigonometric
identities using double angle formulas.

More generally:

$$\int (a^2 - x^2)^{n/2} \, dx$$

becomes, after the appropriate trigonometric substitution $\theta =
\arcsin(x/a)$:

$$a^{n+1} \int \cos^{n+1} \theta \, d\theta$$

Note that this formula works for negative $n$ as well. The particular
case $n = -1$ is the familiar integration formula $\int dx/\sqrt{a^2 -
x^2} = \arcsin(x/a)$.

\subsection{A slight complication}

Consider an integral of the form (as before, $a > 0$):

$$\int x^m (a^2 - x^2)^{n/2} \, dx$$

There are three cases of note:

\begin{enumerate}
\item $n$ is even and nonnegative: In this case, we can expand using
  the binomial theorem and integrate termwise.
\item $m$ is odd: In this case, we can make a $u$-substitution $u =
  a^2 - x^2$, and solve the integral in a purely algebraic fashion. We
  could also do the trigonometric substitution if we so desire, but to
  solve that problem we would end up doing an algebraic substitution
  back again..
\item Other cases: We can use the trigonometric substitution $\theta =
  \arcsin(x/a)$ and simplify. We basically reduce to the case of
  integrating the product of a power of the sine function and a power
  of the cosine function.
\end{enumerate}

\subsection{The two other key substitution ideas}

We will now state the two other key ideas for substitutions:

\begin{enumerate}
\item An expression of the form $a^2 + x^2$ with a negative power or
  squareroot to it should suggest $\theta = \arctan(x/a)$, giving $dx
  = a \sec^2 \theta \, d\theta$, $x = a \tan \theta$, and $a^2 + x^2 =
  a^2 \sec^2 \theta$. Also, $\sqrt{a^2 + x^2} = a \sec \theta$. Also,
  $\tan \theta = x/a$, $\sec \theta = \sqrt{1 + (x/a)^2} = \sqrt{a^2 +
  x^2}/a$. Here, we are using the fact that $\sec$ is positive on the
  range of the arc tangent function.
\item An expression of the form $\sqrt{x^2 - a^2}$ should suggest
  $\theta = \operatorname{arcsec}(x/a) = \arccos(a/x)$, i.e., $x = a
  \sec \theta$, giving $dx = a \sec \theta \tan \theta \, d\theta$ and
  $\sqrt{x^2 - a^2} = a|\tan \theta|$. We cannot dispense with the
  absolute value sign because tangent is not positive throughout the
  range of the arc secant function (which is the same as the range of
  the arc cosine function).
\end{enumerate}

We use this to calculate some important trigonometric integrals:

$$\int \frac{dx}{\sqrt{a^2 + x^2}}$$

We use the substitution $\theta = \arctan(x/a)$ and obtain:

$$\int \frac{a \sec^2 \theta}{a \sec \theta} \, d\theta$$

After some cancellation, this reduces to:

$$\int \sec \theta \, d \theta$$

This gives:

$$\ln|\sec \theta + \tan \theta|$$

Note that $\tan \theta = x/a$, and $\sec \theta = \sqrt{1 + (x/a)^2}$,
so we get:

$$\ln\left|\sqrt{1 + \frac{x^2}{a^2}} + \frac{x}{a} \right|$$

Note that, interestingly, the final answer can be written in a manner
that is completely devoid of trigonometry. However, the trigonometric
route was useful in obtain this answer.

Here's another example:

$$ \int \sqrt{a^2 + x^2} \, dx$$

We use the substitution $\theta = \arctan(x/a)$ and obtain:

$$\int a\sec \theta a \sec^2 \theta \, d\theta$$

This reduces to $a^2$ times the integral of $\sec^3 \theta$, which we
have seen using integration by parts. The answer is:

$$\frac{a^2}{2} \left[ \sec \theta \tan \theta + \ln|sec \theta + \tan \theta|\right]$$

We can now substitute back and simplify, writing functions of $\theta$
in terms of $x$ and $a$.

\subsection{Application to higher powers of $x^2 + a^2$}

Next, consider an integral of the form:

$$\int \frac{dx}{(x^2 + a^2)^{n/2}}$$

We use the substitution $\theta = \arctan(x/a)$, and simplify to obtain:

$$\frac{1}{a^{n - 1}} \int \cos^{n - 2} \theta \, d\theta$$

The special case $n = 2$ just gives $\theta/a = (1/a)
\arctan(x/a)$. The case $n = 3$ gives:

$$\frac{1}{a^2} \int \cos \theta \, d\theta = (\sin \theta)/(a^2)$$

We can now rewrite $\sin \theta$ in terms of $x$ and $a$.

The case $n = 4$ gives:

$$\frac{1}{a^3} \int \cos^2 \theta \, d\theta$$

which we can solve using the well memorized integral of $\cos^2$, and
then substitute back in terms of $x$ and $a$.

\subsection*{Aside: quick as a fox}

To get really good at these integration problems, it helps to memorize
the way the substitutions typically work. But it also helps to
memorize key integration results in a manner that they can be easily
applied to problems directly. Thus, I recommend that, once you have a
basic mastery of the methods, you memorize the integrals of various
half-integer powers of $x^2 - a^2$, $x^2 + a^2$, and $a^2 - x^2$. This
memorization should include remembering the final answer clearly,
remembering the steps used to reach it, and being able to quickly
apply the learned formula to specific numerical values.

\subsection*{Using triangles}

If you find it hard to deal with ratios when doing trigonometric
substitutions in forward and reverse, you may benefit from using right
triangles. The book uses this approach. We'll briefly sketch it here,
and you can see worked examples in the book.

For instance, when doing the $u$-substitution $\theta = \arcsin(x/a)$,
consider a right triangle with hypotenuse $a$, base angle $\theta$,
and height $x$ (so $x$ is the side opposite $\theta$). Now use the
Pythagorean theorem to deduce that the other side is $\sqrt{a^2 -
x^2}$. It is now possible to compute all the trigonometric functions
for $\theta$ in terms of the sides of the triangle.

This approach is not really different from what we discussed, but some
people find it more intuitive. Refer to examples in the book.

\section{Interpretation of formulas in terms of homogeneous functions}

This is optional material, in the sense that it will not be directly
tested, but it may help you understand things.

Here, we briefly discuss the concept of homogeneous degree and how it
can be used to obtain a qualitative understanding of some of the
integration formulas.

Consider a function $F(x,a)$ of two variables. We say that $F$ is
homogeneous of degree $d$ if, for any $\lambda$, we have:

$$F(\lambda x, \lambda a) = \lambda^d F(x,a)$$

A homogeneous function of degree zero is sometimes called {\em
dimensionless}, and it depends only on the quotient $x/a$.

A {\em homogeneous polynomial} of degree $d$ is a polynomial in which
each monomial has total degree $d$ in the two variables. A homogeneous
polynomial of degree $d$ is also a homogeneous function of degree $d$.

We also have the following:

\begin{itemize}
\item The zero function can be viewed as homogeneous of any positive
degree, but more properly, it is just treated as an anomaly.
\item If $F_1$ and $F_2$ are homogeneous of the same degree $d$, so is
  any {\em linear combination} $a_1F_1 +a_2F_2$ (unless that linear
  combination is identically the zero function), where $a_1$ and $a_2$
  are real constants. In particular, $F_1 + F_2$ and $F_1 - F_2$ are
  homogeneous of degree $d$.
\item If $F_1$ and $F_2$ are homogeneous of degrees $d_1$ and $d_2$
  respectively, then $F_1 \cdot F_2$ is homogeneous of degree $d_1 +
  d_2$ and $F_1/F_2$ is homogeneous of degree $d_1 - d_2$.
\item If $F$ is homogeneous of degree $d$, then $F^m$ (where the power
  denotes a pointwise power) is homogeneous of degree $dm$. Here $m$
  could be an integer or a rational number.
\item Applying any function to something homogeneous of degree zero
  gives something homogeneous of degree zero.
\end{itemize}

Combining these two observations, we see that $1/\sqrt{a^2 - x^2}$ is
homogeneous of degree $-1$, $1/(x^2 + a^2)$ is homogeneous of degree
$-2$, and $(x^2 + a^2)^{3/2}$ is homogeneous of degree two. Now, we
make the key observations relevant to the differentiation and
integration formulas:

\begin{itemize}
\item Differentiation of a homogeneous function in $x$ and $a$ with
  respect to $x$ gives a homogeneous function with degree one less.
\item Conversely, integration of a homogeneous function in $x$ and $a$
  with respect to $x$ usually gives a homogeneous function with degree
  one more than the integrand, plus a constant.

  {\em Note, please, that not every antiderivative expression is a
  homogeneous function}. Rather, all we are saying is that {\em one}
  of the antiderivatives is homogeneous, so every antiderivative is a
  constant plus a homogeneous function. However, the {\em usual}
  methods we use to integrate will naturally yield the homogeneous
  antiderivative.
\item The upshot is that when integrating some radically thing which
  is homogeneous in $x$ and $a$ of degree $d$, we will get something
  which is homogeneous in $x$ and $a$ of degree $d + 1$. {\em
  Further}, all the parts that involve inverse trigonometric functions
  will be of the form $a^{d+1}$ times some inverse trigonometric
  function (or variant) of $x/a$. Any expression involving logarithms
  should involve the logarithm of some function of $x/a$ (i.e., should
  have degree zero).
\end{itemize}

Here are some examples (in all of which we assume $a > 0$):

\begin{itemize}
\item The integral of $1/\sqrt{a^2 - x^2}$ is $\arcsin(x/a)$. The
  integrand is homogeneous of degree $-1$, and the expression we get
  after integrating is homogeneous of degree $0$.
\item The integral of $1/(x^2 + a^2)$ is $(1/a) \arctan(x/a)$. The
  integrand is homogeneous of degree $-2$, and the expression we get
  after integrating is homogeneous of degree $-1$ -- namely, it is the
  product of $1/a$ and the dimensionless quantity $\arctan(x/a)$.
\item The integral of $1/\sqrt{x^2 + a^2}$ is $\ln[(x/a) +
  \sqrt{(x/a)^2 + 1}]$. The integrand is homogeneous of degree $-1$
  and after integrating we get something that is homogeneous of degree
  $0$.
\item The integral of $\sqrt{a^2 - x^2}$ is $(a^2/2) \arcsin(x/a) +
  x\sqrt{a^2 - x^2}/2$. The integrand is homogeneous of degree one and
  after integrating we get something that is homogeneous of degree two
  -- it is the sum of two terms each of which is homogeneous of degree
  two.
\end{itemize}

\section{Alternative interpretation: inverse hyperbolic trigonometry}

This section is optional, and is not officially part of the course,
but is included to help offer another perspective to these
integrations.

Recall that we did the integration:

$$\int \frac{dx}{\sqrt{x^2 + a^2}} = \ln\left[\frac{x}{a} + \sqrt{(x/a)^2 + 1}\right]$$

We did this integration using a trigonometric substitution and then
using the formula for integrating the secant function. That, however,
is {\em not} the {\em natural} approach to this problem. The natural
approach is to consider the {\em arc hyperbolic sine} function,
briefly discussed here.

Recall that $\sinh$ is a one-to-one function with domain and range
$\R$. Thus, we can define an inverse function, which we denote
$\sinh^{-1}$, on all of $\R$. If $\sinh x = y$, then we have:

$$\cosh^2x = y^2 + 1$$

Since $\cosh$ is positive, we get $\cosh x = \sqrt{y^2 + 1}$, so
$\exp(x)$ becomes $\sinh x + \cosh x$, which is $y + \sqrt{y^2 + 1}$. Thus, we get:

$$x = \ln[y + \sqrt{y^2 + 1}]$$

Interchanging the roles of $x$ and $y$ to get the explicit expression
for $\sinh^{-1}$, we get $\sinh^{-1} x = x + \sqrt{x^2 + 1}$. (This was
seen in one of the quiz problems).

Now, getting back to the integration problem (with $a > 0$):

$$\int \frac{dx}{\sqrt{x^2 + a^2}}$$

Put $t = \sinh^{-1}(x/a)$. Then $x = a \sinh t$, $dx = a \cosh t \, dt$, and we get:

$$\int \frac{a \cosh t \, dt}{\sqrt{a^2(\sinh^2t + 1)}}$$

Using that $\sqrt{\sinh^2 t + 1} = \cosh t$, we get:

$$\int 1 \, dt$$

which gives that the integral is $\sinh^{-1}(x/a)$. Now using the
explicit expression for $\sinh^{-1}$ worked out above, we get the result
indicated.

In general, any integration that involves a half-integer power of $a^2
+ x^2$ is best done using inverse hyperbolic sine, and anything that
involves an integer power (possibly negative) is best done using the
arc tangent. However, as we have seen, it is possible (though messy)
to do all these types of integrations using arc tangent alone, as long
as we are prepared to integrate odd powers of secant using integration
by parts. We'll stick to using only inverse circular trigonometry for
this course.

If you want to learn more on hyperbolic trigonometry, go through
Section 7.9 of the book, which we're not including in the
syllabus. $\sinh^{-1}$ and other related functions are discussed on
Pages 394 and 395 of the book.

\section{Dealing with quadratics: square completion}

\subsection{The basics}
We have looked at quadratics in the past, but we now need to consider
them from a somewhat different perspective.

Given a quadratic function $f(x) := Ax^2 + Bx + C$ with $A \ne 0$, we
can write:

$$f(x) = A\left(x + \frac{B}{2A}\right)^2 + \frac{4AC - B^2}{4A}$$

The expression $B^2 - 4AC$ is termed the {\em discriminant} of the
quadratic, and we will, for convenience, denote it by the letter
$D$. We thus have:

$$f(x) = A\left(x + \frac{B}{2A}\right)^2 - \frac{D}{4A}$$

The derivative is:

$$f'(x) = 2A\left(x + \frac{B}{2A}\right) = 2Ax + B$$

The only critical point is $x = -B/2A$. We now consider various cases:

\begin{enumerate}
\item If $A > 0$, the quadratic goes to $+\infty$ as $x \to \pm
  \infty$, it is decreasing for $x$ in $(-\infty,-B/2A)$ and it is
  increasing for $x$ in $(-B/2A,\infty)$. The minimum is $-D/4A$,
  attained at $-B/2A$.
\item If $A < 0$, the quadratic goes to $-\infty$ as $x \to \pm
  \infty$, it is increasing for $x$ in $(-\infty,-B/2A)$, and it is
  decreasing for $x$ in $(-B/2A,\infty)$. The maximum is $-D/4A$,
  attained at $-B/2A$.
\end{enumerate}

We also see from the above that if $D < 0$, then the function has
constant sign on all of $\R$, and never becomes zero. If $D = 0$, the
function attains the value $0$ at its vertex $-B/2A$ and has constant
sign everywhere else. If $D > 0$, the function attains the value $0$
at two distinct points. Note also that the symmetry of the graph about
the line $x = -B/2A$ is clear from the context.

\subsection{The upshot}

The upshot of the above is that every quadratic function can be written as:

$$[\text{constant}] \times [\text{square of ($x - $ something)}] + \text{constant}$$

For simplicity, we will assume, through a change of variable, that
that $x - $ something is just $x$, i.e., that $B = 0$ in the original
quadratic. We can do this change of variable. The upshot is that every
quadratic can, after this transformation, be written as:

$$px^2 + q = p(x^2 + q/p)$$

There are now the following three possibilities relevant to
integration situations:

\begin{enumerate}
\item $p$ and $q$ are both positive or both negative: In this case, we
  can find $a$ such that $a^2 = q/p$ and then use the substitution
  $\theta = \arctan(x/a)$.
\item $p$ and $q$ have opposite signs: In this case, we can find $a$
  such that $a^2 = -q/p$. We can now put $\theta = \arcsin(x/a)$ (if
  that makes sense in the context) or put $\theta = \arccos(a/x)$ (if
  that makes sense in the context).
\item $q = 0$: Here, integration poses no challenges.
\end{enumerate}

\subsection{Completing the square: some examples}

For instance, consider:

$$\int \frac{dx}{x^2 + x + 1}$$

By the general discussion above, we can write this as:

$$\int \frac{dx}{(x + 1/2)^2 + 3/4}$$

We see that here, the second term is positive, and we obtain:

$$\int \frac{dx}{(x + 1/2)^2 + (\sqrt{3}/2)^2}$$

The trigonometric substitution is now clear: $\theta = \arctan{(x +
1/2)/(\sqrt{3}/2)}$. In this case, we can directly apply one of the
antiderivative formulas (so we don't even need to formally do the
substitution) and we get:

$$\frac{2}{\sqrt{3}} \arctan\left(\frac{x + (1/2)}{\sqrt{3}/2}\right) + C$$

Here is another example:

$$\int \frac{dx}{\sqrt{1 - 2x - x^2}}$$

We can use the square completion to obtain:

$$\int \frac{dx}{\sqrt{2 - (x + 1)^2}}$$

We can thus write it as:

$$\int \frac{dx}{\sqrt{(\sqrt{2})^2 - (x + 1)^2}}$$

We now put $\theta = \arcsin((x + 1)/\sqrt{2})$ and obtain, after
simplification, that the integral is:

$$\arcsin\left(\frac{x + 1}{\sqrt{2}}\right) + C$$

Note that in both the above cases, we have been lucky in the sense
that the square term had a coefficient of $\pm 1$. Let's consider an
example where it does not.

$$\int \frac{dx}{2x^2 + 3x + 4}$$

We can complete the square as:

$$\int \frac{dx}{2(x + (3/4))^2 + (4 - (9/8))}$$

We can now proceed, but it is usually easier if we take the
coefficient of the square term outside the integral sign, obtaining,
in this case:

$$\frac{1}{2} \int \frac{dx}{(x + (3/4))^2 + 23/16}$$

We now write $23/16 = (\sqrt{23}/4)^2$ and obtain the result in terms
of the arc tangent function, namely:

$$\frac{2}{\sqrt{23}} \arctan\left(\frac{x + (3/4)}{\sqrt{23}/4}\right)$$

Mathematics is beautiful, but it is not always pretty.

\end{document}