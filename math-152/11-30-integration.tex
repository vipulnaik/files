\documentclass[10pt]{amsart}

%Packages in use
\usepackage{fullpage, hyperref, vipul, enumerate}

%Title details
\title{Take home class quiz: Turn in November 30: Integration}
\author{Math 152, Section 55 (Vipul Naik)}
%List of new commands

\begin{document}
\maketitle

Your name (print clearly in capital letters): $\underline{\qquad\qquad\qquad\qquad\qquad\qquad\qquad\qquad\qquad\qquad}$

{\bf THIS IS A TAKE HOME QUIZ.}

{\bf FEEL FREE TO DISCUSS ALL QUESTIONS, BUT ATTEMPT THEM YOURSELF FIRST.}

\begin{enumerate}
\item What is the limit $\lim_{x \to \infty} \left[\left(\int_0^x
  \sin^2 \theta d\theta\right)/x\right]$? {\em Last year: $13/16$
  correct}

  \begin{enumerate}[(A)]
  \item $1/2$
  \item $1$
  \item $1/\pi$
  \item $2/\pi$
  \item $1/(2\pi)$
  \end{enumerate}

  \vspace{0.1in}
  Your answer: $\underline{\qquad\qquad\qquad\qquad\qquad\qquad\qquad}$
  \vspace{0.6in}

\item Consider the substitution $u = -1/x$ for the integral $\int
  \frac{dx}{x^2 + 1}$. What is the {\bf new integral}? {\em Last year:
  $8/16$ correct}

  \begin{enumerate}[(A)]
  \item $\int \frac{du}{u(u^2 + 1)}$
  \item $\int \frac{du}{u^2 + 1}$
  \item $\int \frac{ u du}{u^2 + 1}$
  \item $\int \frac{u^2 du}{u^2 + 1}$
  \item $\int \frac{u^2 du}{(u^2 + 1)^2}$ 
  \end{enumerate}

  \vspace{0.1in}
  Your answer: $\underline{\qquad\qquad\qquad\qquad\qquad\qquad\qquad}$
  \vspace{0.6in}

\item {\em Hard}: What is the {\bf value} of $c \in (0,\infty)$ such
  that $\int_0^c \frac{dx}{x^2 + 1} = \lim_{a \to \infty} \int_c^a
  \frac{dx}{x^2 + 1}$? {\em Last year: $8/16$ correct}

  \begin{enumerate}[(A)]
  \item $\frac{1}{\sqrt{3}}$
  \item $\frac{1}{\sqrt{2}}$
  \item $1$
  \item $\sqrt{2}$
  \item $\sqrt{3}$
  \end{enumerate}

  \vspace{0.1in}
  Your answer: $\underline{\qquad\qquad\qquad\qquad\qquad\qquad\qquad}$
  \vspace{0.6in}

\item Suppose $f$ is a continuous nonconstant even function on
  $\R$. Which of the following is {\bf true}? {\em Last year: $4/16$
  correct}
  
  \begin{enumerate}[(A)]
  \item Every antiderivative of $f$ is an even function.
  \item $f$ has exactly one antiderivative that is an even function.
  \item Every antiderivative of $f$ is an odd function.
  \item $f$ has exactly one antiderivative that is an odd function.
  \item None of the antiderivatives of $f$ is either an even or an odd
    function.
  \end{enumerate}

  \vspace{0.1in}
  Your answer: $\underline{\qquad\qquad\qquad\qquad\qquad\qquad\qquad}$
  \vspace{0.6in}

\item Suppose $f$ is a continuous nonconstant odd function on
  $\R$. Which of the following is {\bf true}? {\em Last year: $13/16$
  correct}

  \begin{enumerate}[(A)]
  \item Every antiderivative of $f$ is an even function.
  \item $f$ has exactly one antiderivative that is an even function.
  \item Every antiderivative of $f$ is an odd function.
  \item $f$ has exactly one antiderivative that is an odd function.
  \item None of the antiderivatives of $f$ is either an even or an odd
    function.
  \end{enumerate}

  \vspace{0.1in}
  Your answer: $\underline{\qquad\qquad\qquad\qquad\qquad\qquad\qquad}$
  \vspace{0.6in}

\item Suppose $f$ is a continuous nonconstant periodic function on
  $\R$ with period $h$. Which of the following is {\bf true}? {\em
  Last year: $5/16$ correct}

  \begin{enumerate}[(A)]
  \item Every antiderivative of $f$ is a periodic function with period
    $h$, regardless of the choice of $f$.
  \item For some choices of $f$, every antiderivative of $f$ is a
    periodic function; for all others, $f$ has no periodic antiderivative.
  \item $f$ has exactly one periodic antiderivative for every choice of $f$.
  \item For some choices of $f$, $f$ has exactly one periodic
    antiderivative; for all others, $f$ has no periodic
    antiderivative.
  \item Regardless of the choice of $f$, no antiderivative of $f$ can
    be periodic.
  \end{enumerate}

  \vspace{0.1in}
  Your answer: $\underline{\qquad\qquad\qquad\qquad\qquad\qquad\qquad}$
  \vspace{0.6in}

\item Consider a continuous increasing
  function $f$ defined on the nonnegative real numbers. Define $m_f(a)$,
  for $a > 0$, as the unique value $c \in [0,a]$ such that $f(c)$ is the
  mean value of $f$ on the interval $[0,a]$.
  
  If $f(x) := x^n$, $n$ an integer greater than $1$, what kind of
  function is $m_f$ (your answer should be valid for all $n$)? {\em
  Last year: $1/16$ correct}

  \begin{enumerate}[(A)]
  \item $m_f(a)$ is a constant $\lambda$ dependent on $n$ but
    independent of $a$.
  \item It is a function of the form $m_f(a) = \lambda a$, where
    $\lambda$ is a constant depending on $n$.
  \item It is a function of the form $m_f(a) = \lambda a^{n-1}$, where
    $\lambda$ is a constant depending on $n$.
  \item It is a function of the form $m_f(a) = \lambda a^n$, where
    $\lambda$ is a constant depending on $n$.
  \item It is a function of the form $m_f(a) = \lambda a^{n+1}$, where
    $\lambda$ is a constant depending on $n$.
  \end{enumerate}

  \vspace{0.1in}
  Your answer: $\underline{\qquad\qquad\qquad\qquad\qquad\qquad\qquad}$
  \vspace{0.6in}

\end{enumerate}
\end{document}
