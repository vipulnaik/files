\documentclass[10pt]{amsart}
\usepackage{fullpage,hyperref,vipul,graphicx}
\title{Max-min problems}
\author{Math 152, Section 55 (Vipul Naik)}

\begin{document}
\maketitle

{\bf Corresponding material in the book}: Section 4.5

{\bf Difficulty level}: Moderate to hard. This is material that you
have probably seen at the AP level, but it is very important and there
will be many additional subtleties that you may have glossed over
earlier.

{\bf What students should definitely get}: The basic procedure for
converting a verbal or real-world optimization problem into a
mathematical problem seeking absolute maxima and absolute minima,
solving that problem, and reinterpreting the solution in real-world
terms.

{\bf What students should hopefully get}: Important facts about
area-perimeter optima. The idea that the maximum is determined by the
minimum, or most binding, constraint. The intuiton of tangency (as
seen in the tapestry problem). The multiple use heuristic. The idea of
transforming a function into an equivalent function that is easier to
optimize. The procedure for and subtleties in integer
optimization. How single-variable optimization fits into the broader
optimization context.

\section*{Executive summary}

Words...

\begin{enumerate}
\item In real-world situations, maximization and minimization problems
  typically involve multiple variables, multiple constraints on those
  variables, and some objective function that needs to be maximized or
  minimized.
\item The only thing we know to solve such problems is to reduce
  everything in terms of one variable. This is typically done by {\em
  using up} some of the constraints to express the other variables in
  terms of that variable.
\item The problem then typically boils down to a
  maximization/minimization problem of a function in a single variable
  over an interval. We use the usual techniques for understanding this
  function, determining the local extreme values, determining the
  endpoint extreme values, and determining the absolute extreme
  values.
\end{enumerate}

Actions... (think of examples; also review the notes on max-min problems)

\begin{enumerate}
\item Extremes sometimes occur at endpoints but these endpoints could
  correspond to degenerate cases. For instance, of all the rectangles
  with given perimeter, the square has the maximum area, and the
  minimum occurs in the degenerate case of a rectangle where one side
  has length zero.
\item Some constraints on the variables we have are explicitly stated,
  while others are implicit. Implicit constraints include such things
  as nonnegativity constraints. {\em Some of these implicit
  constraints may be on variables other than the single variable in
  terms of which we eventually write everything.}
\item After we have obtained the objective function in terms of one
  variable, we are in a position to throw out the other
  variables. However, before doing so, it is {\em necessary to
  translate all the constraints into constraints on the one variable
  that we now have}. 
\item When our intent is to maximize a function, it is sometimes
  useful to maximize an equivalent function that is easier to
  visualize or differentiate. For instance, to maximize $\sqrt{f(x)}$
  is equivalent to maximizing $f(x)$ if $f(x)$ is nonnegative. With
  this way of thinking about equivalent functions, we can make sure
  that the actual function that we differentiate is easy to
  differentiate. The main criterion is that the two functions should
  rise and fall together. (Analogous observations apply for
  minimizing) Remember, however, that to calculate the {\em value} of
  the maximum/minimum, you should go back to the original function.
\item Sometimes, there are other parameters in the
  maximization/minimization problem that are {\em unknown constants},
  and the final solution is expected to be in terms of those
  constants. In rare cases, the nature of the function, and hence the
  nature of maxima and minima, depends on whether those constants fall
  in particular intervals. {\em If you find this to be the case, go
  back to the original problem and see whether the real-world
  situation it came from constrains the constants to one of the
  intervals}.
\item For some geometrical problems, the maximization/minimization can
  be done trigonometrically. Here, we make a clever choice of an angle
  that controls the {\em shape} of the figure and then use the
  trigonometric functions of that angle. This could provide alternate
  insight into maximization.
\end{enumerate}

Smart thoughts for smart people ...

\begin{enumerate}
\item Before getting started on the messy differentiation to find
  critical points, think about the constraints and the endpoints. Is
  it obvious that the function will attain a minimum/maximum at one of
  the endpoints? What are the values of the function at the endpoints?
  (If no endpoints, take limiting values as you go in one direction of
  the domain). Is there an intuitive reason to believe that the
  function attains its optimal value somewhere {\em in between} rather
  than at an endpoint? Is there some kind of trade-off to be made? Are
  there some things that can be said qualitatively about where the
  trade-off is likely to occur?
\item Feel free to convert your function to an equivalent function
  such that the two functions rise and fall together. This reduces the
  burden of messy expressions.
\item It is useful to remember the fact that the function $x^p(1 -
  x)^q$ attains a local maximum at $p/(p + q)$. That's because this
  function appears in disguise all the time (e.g., maximizing area of
  rectangle with given perimeter, etc.)
\item A useful idea is that when dividing a resource into two
  competing uses, and one use is hands-down better than the other, the
  {\em best} use happens when the entire resource is devoted to the
  better use. However, the {\em worst} may well happen somewhere in
  between, because divided resources often perform even worse than
  resources devoted wholeheartedly to a bad use. This is seen in
  perimeter allocation to boundaries with the objective function being
  the total area, and area allocation to surfaces with the objective
  function being the total volume.
\item When we want to {\em maximize} something subject to a collection
  of many constraints, the most relevant constraint is the {\em
  minimum} one. Think of the ladder-through-the-hallway problem, or
  the truck-going-under-bridges problem. 
\end{enumerate}

\section{Motivation and basic terminology}

In the previous lecture, we discussed how to compute points and values
of absolute maximum and absolute minimum. Our focus now shifts to
using these tools and techniques for real-world (or pseudo-real-world)
optimization problems. Because the techniques we have developed are so
limited, we will be very selective about the nature of the real-world
problems that we pick. Nonetheless, we'll see that even with the
modest machinery we have built, we have ways of effectively
understanding and tackling many real-world problems.

\subsection{Notion of constraints and objective function}

In a typical real-world situation, we usually have multiple things
interacting. Many of these items can be measured quantitatively, i.e.,
they can be measured using real numbers. The values of these real
numbers may be subjected to further constraints. Those making
decisions may have control over some of the variables. Those making
decisions are also tasked with trying to maximize some kind of utility
that is dependent on these variables, or minimize some kind of cost
function that is dependent on these variables. The task is to {\em
choose the variables subject to constraints in the manner that best
maximizes that particular utility function or minimizes that
particular cost function.}

For example, think of money management, something that you might be
familiar with. You have a certain limited amount of money, and there
are various things you want to buy with that money. Each thing that
you buy gives you a certain amount of satisfaction; however, for most
things, the amount of satisfaction varies with how much you buy
it. The question is: how do you allocate money between the many
competing things in the market so as to get the best deal for
yourself? The number of variables in this case is just the number of
different items that you can buy in variable quantities. At a broad
level, you may choose to spend $A$ on food, $B$ on clothing, and $C$
on extra books to study calculus. If the total money quantity with you
is $M$, then, assuming that you're not one of those who likes to live
on credit, you'll have the constraints $A + B + C \le M$.

Now, there are going to be three functions $f$, $g$, and $h$, where
$f(A)$ is the happiness that spending $A$ on food gives you, $g(B)$ is
the happiness that spending $B$ on clothing gives you, and $h(C)$ is
the happiness (?) that spending on extra calculus books gives
you. Assuming that happiness is additive, what you want to maximize is
$f(A) + g(B) + h(C)$. Given specific functional forms for $f$, $g$,
and $h$, we hope to use the tools of calculus to tackle this problem.

\subsection{The extremes and the middle path}

There are two schools of philosophical thought that shall contend for
our attention here: the school that says that extremes are good, and
the school that urges you to follow a middle path -- a bit of this and
a bit of that. Which of them is right? Depends.

The extremists would say that among the three things: food, clothing,
and calculus books, one of them is the best value for money (for
instance, in our case, it may be calculus books). This means that
every additional unit of money that you spend should go on calculus
books. Thus suggests that the way you'll be ``happiest'' is if you
spend all your money on calculus books.

In reality, however, we know that that isn't how things work. The
problem? We need a bit of food, a bit of clothing, and a bit of
calculus, but beyond a point, more food, clothing and calculus is less
helpful. This is obvious in the case of food -- too much food at your
disposal means that you either eat more than your body can handle or
throw food away. It is also obvious in the case of clothing. It may
not be that obvious in the case of calculus, but you'll have to take
my word for it that there does come a point after which more calculus
may not be worth it.

So, basically, this is a three-way trade-off game, and we need to
figure out where to make the trade-offs between food, clothing, and
calculus. This comes somewhere in between -- a local maximum, where
shifting resources from any one sector to any other sector reduces
utility.

On the other hand, there are situations where extremes are
better. Those are situations where it's just no contest between the
two options -- more of one thing is better than more of the other no
matter how much of either you have. So the extremes could be the best
option.

Thus, the maximum could occur at the endpoints, but it could also
occur in between, as a local maximum, where diverting resources a bit
in either direction makes things worse.

\subsection{Humbler matters: one-variable ambitions}

After suggesting that I could help you with managing money, I have to
retreat to humbler ground. All the tools we have developed so far are
tools that specifically deal with one variable -- we've talked of {\em
functions of one variable}, and developed concepts of limits,
continuity, and differentiation all in this context. Thus, the kind of
budgeting and allocation problems that we encounter in the real world,
that involve a plethora of variables, are simply too hard for us to
handle with this machinery. This is also a reason why you shouldn't
just stop with the 150s, and should go on to study multivariable
calculus, but let's now talk of what {\em can} be done using the
one-variable approach.

A priori, you might expect that the one-variable approach can only
work when there is only one variable involved. It's actually a little
more general.

The one-variable approach can be used for situations where we can use
some of the constraints to express all variables in terms of a single
variable, wherein the optimization problem simply becomes a problem in
terms of that variable. So, even though the problem has more than one
variable, we are able to tackle it as a one-variable problem. Here is
one example.

For instance, consider the following problem: For all the rectangles
with diagonal length $c$, find the dimensions of the one with the largest
area.

To solve this problem, we try to figure out what variables we have
control over, and what constraints these variables satisfy. A
rectangle is specified by specifying its length and breadth, i.e., the
two side lengths. If we call these $l$ and $b$, the goal is to
maximize $lb$. Also, $l$ and $b$ are subject to the constraint $l^2 +
b^2 = c^2$, and $l > 0, b > 0$.

The problem is that we have two variables, and we only know how to
tackle situations with one variable. In order to solve the problem, we
need to write one of the variables in terms of the other one. Note
that the relation $l^2 + b^2 = c^2$, along with the fact that $l > 0$
and $b > 0$, allows us to write $b = \sqrt{c^2 - l^2}$. Thus, the area
is given by a function of $l$, namely:

$$A(l) := l\sqrt{c^2 - l^2}$$

Note that $\sqrt{c^2 - l^2} > 0$ implies that $l < c$. Thus, the goal
is to maximize this function on the interval $(0,c)$.

We compute the derivative:

$$A'(l) = \sqrt{c^2 - l^2} - \frac{l^2}{\sqrt{c^2 - l^2}}$$

Setting $A'(l) = 0$, we obtain that:

$$\sqrt{c^2 - l^2} = \frac{l}{\sqrt{c^2 - l^2}}$$

Simplifying, we obtain:

$$l = b = c/\sqrt{2}$$

and thus:

$$A(l) = c^2/2$$

Thus, the only critical point for the function is at $l =
c/\sqrt{2}$. Note that for $l < c/\sqrt{2}$, we have $l < \sqrt{c^2 -
l^2}$, so the expression for $A'(l)$ is positive, and for $l >
c/\sqrt{2}$, the expression is negative. Thus, the point $l =
c/\sqrt{2}$ is a point of local maximum.

To determine whether it is a point of absolute maximum, we need to
verify that the value of the function at this point is greater than
the limits at the two endpoints. It is easy to see that the limits at
both endpoints are zero, so indeed, we have a local maximum at $l =
c/\sqrt{2}$.

Here is an alternative approach, that involves a different way of
choosing variables. Let $\theta$ be the angle made by the diagonal
with the base of the rectangle. Then, $0 < \theta < \pi/2$, and the
two sides of the rectangle have length $c \cos \theta$ and $c \sin
\theta$. Thus, the area is given by the function:

$$f(\theta) = c^2 \sin \theta \cos \theta = (c^2/2) \sin (2\theta)$$

\includegraphics[width=3in]{rectanglesidesintermsofdiagonal.png}

This attains an absolute maximum at $\theta = \pi/4$, where $2\theta =
\pi/2$, so that $\sin (2\theta) = 1$. Note that we can solve the
problem in this case even without using calculus, but if you don't
notice that $\sin \theta \cos \theta = (1/2) \sin (2\theta)$, you can
solve the problem the calculus way and obtain that the absolute
maximum is at $\theta = \pi/4$. Thus, the area is $c^2/2$ and the two
side lengths are $c/\sqrt{2}$.

Thus, the maximum occurs for a square.

\subsection{Geometrical and visual optimization}

In most of the situations that we'll be dealing with, it is helpful to
draw a figure, label all the lengths and/or angles involved in the
figure, and then write down the various constraints as well as the
objective function that needs to be maximized. Then, try to get
everything in terms of one variable, using the constraints, and
finally, do the maximization for that variable. The book gives the
following five-point procedure on Page 183:

\begin{enumerate}
\item Draw a representative figure and assign labels to the relevant
  quantities.
\item Identify the quantity to be maximized or minimized and find a
  formula for it.
\item Express the quantity to be maximized or minimized in terms of a
  single variable; use the conditions given in the problem to
  eliminate the other variable(s).
\item Determine the domain of the function generated by Step 3.
\item Apply the techniques of the preceding sections to find the
  extreme value(s).
\end{enumerate}

One of the important things in this is to notice that we usually need
to maximize the function with the input variable restricted to a
certain domain. Thus, there are often situations where the absolute
maximum or minimum occurs at an endpoint of the domain, i.e., it is an
endpoint maximum/minimum.

Here are some examples that you should keep in mind, both in terms of
the final results and the methods we use to get them:

\begin{enumerate}
\item Of all the rectangles with a given perimeter, the square has the
  largest area. This boils down to maximizing $l((p/2) - l)$. There is
  no rectangle with minimum area -- the minimum area occurs in the
  {\em degenerate rectangle} where one of the sides has zero length
  and the other side has length half the perimeter. The degenerate
  rectangle isn't ordinarily considered a rectangle. Here are pictures
  of a collection of rectangles with the same perimeter. It is
  visually clear that the square has the largest area.

  \includegraphics[width=3in]{equiperimeterrectangles.png}
\item Conversely, of all the rectangles with a given area, the square
  has the smallest perimeter. This boils down to minimizing $2(l +
  (A/l))$. There is no rectangle with the largest perimeter -- we can
  keep getting longer and thinner rectangles.
\item Of all the rectangles with a given diagonal length, the square
  is the one with the largest area. This boils down to maximizing
  $l\sqrt{c^2 - l^2}$. Trigonometrically, it involves maximizing $\cos
  \theta \sin \theta$. The minimum again occurs for the degenerate
  rectangle, hence does not occur for any actual rectangle.
\item Of all the rectangles with a given diagonal length, the square
  is the one with the largest perimeter. This boils down to maximizing
  $l + \sqrt{c^2 - l^2}$. Trigonometrically, it involves maximizing
  $\cos \theta + \sin \theta$. The minimum again occurs for the
  degenerate rectangle, hence does not occur for any actual rectangle.
\end{enumerate}

\subsection{Applications to real-world physical situations}

We see a common concern that is apparent with all these
maximization/minimization problems. Maximizing the area for a given
perimeter, or minimizing the perimeter for a given area, is a concern
that arises when trying to create containers with as little material
used for the boundary as possible. Maximizing the area for a given
diagonal length or constraints on lengths occurs in situations where
concerns of space availability and fitting stuff are paramount.

Here are some of the quantities and formulas that are useful:

\begin{enumerate}
\item For a right circular cylinder with base radius $r$ and height
  $h$, the total volume (or capacity) is $\pi r^2h$. The curved
  surface area is $2\pi rh$. Each of the disks at the ends has area
  $\pi r^2$. Thus, a right circular cylinder closed at one end has
  surface area $\pi r (2h + r)$ and a right circular cylinder closed
  at both ends has surface area $2\pi r(r + h)$. Based on the
  situation at hand, we need to figure out which of these three
  surface areas is being refered to.
\item For a right circular cone with base radius $r$, vertical height
  $h$ and slant height $l$, the volume is $(1/3) \pi r^2h$. The curved
  surface area is $\pi rl$ and the surface area of the base is $\pi
  r^2$, so the total surface area is $\pi r(r + l)$. Again, we need to
  figure out, based on the situation, which of the surface areas is
  being refered to. Also note that $r$, $h$, and $l$ are related by
  the Pythagorean theorem: $l^2 = r^2 + h^2$.
\item For a semicircle of radius $r$, the area is $(1/2) \pi r^2$. The
  length of the curved part is $\pi r$ and the length of the straight
  part (the diameter) is $2r$, so the total perimeter is $r(\pi +
  2)$. More generally, for a sector of the circle bounded by two radii
  and an arc, where the radii make an angle of $\theta$, the perimeter
  is $r(2 + \theta)$ and the area is $(1/2) \theta r^2$.
\item For a sphere, the surface area is $4 \pi r^2$ and the volume is
  $(4/3) \pi r^3$. For a hemisphere, the surface area is $3\pi r^2$
  ($2\pi r^2$ for the curved part and $\pi r^2$ for the bounding disk)
  and the volume is $(2/3) \pi r^3$.
\end{enumerate}

Here are some important results on optimization in these various examples:

\begin{enumerate}
\item For a right circular cylinder with volume $V$, there is no
  minimum and no maximum for the curved surface area. This is because
  for given radius $r$, the expression for the curved surface area is
  $2V/r$, which approaches $\infty$ as $r \to 0$ (smaller and smaller
  radius, larger and larger height) and approaches $0$ as $r \to
  \infty$ (larger and larger radius, smaller and smaller height). If,
  however, we have additional boundary constraints on the radius or
  height, the maximum/minimum will occur at these boundaries.
\item For a right circular cylinder with volume $V$, there is an
  absolute minimum for the surface area of the base plus curved part
  (i.e., only one bounding disk is included). The expression is $2V/r
  + \pi r^2$. As $r \to 0$ or $r \to \infty$, this expression tends to
  $\infty$. The absolute minimum occurs at the point $r =
  (2V/\pi)^{1/3}$. (see also Example 1 from the book).
\item For a right circular cylinder with volume $V$, there is an
  absolute minimum for the total surface area (including both
  disks). The expression is $2V/r + 2\pi r^2$. As $r \to 0$ or $r \to
  \infty$, this tends to infinity. The absolute minimum occurs at the
  point $r = (V/\pi)^{1/3}$.
\end{enumerate}

\section{Important tricks in real-world problems}

\subsection{The maximum is determined by the tightest constraint}

Let me first state this mathematically (where it's obvious) and then
non-mathematically (where again it's obvious).

Suppose $x$ is a real number subject to the constraints $x \le a_1$,
$x \le a_2$, and $x \le a_3$. What is the {\em maximum} value that $x$
can take? Clearly, it is the {\em minimum} among $a_1$, $a_2$, and
$a_3$, because that is the tightest, most limiting constraint on $x$.

Here are some non-mathematical formulations:

\begin{enumerate}
\item A truck has to go on a highway. As part of its journey, the
  truck needs to negotiate three underpasses, with clearances of $10$
  feet, $9$ feet, and $11$ feet respectively. What is the {\em
  maximum} possible height of the truck? (Hint: You want to make sure
  you don't get into a problem anywhere).
\item Hydrogen and oxygen combine in a ratio of $1:8$ by mass to
  produce water. Assuming that we have $50$ grams of hydrogen and
  $220$ grams of oxygen, what is the {\em maximum} possible amount of
  water that can be produced from these? (Hint: Limiting reagent).
\end{enumerate}

To repeat: {\em the maximum value that something can take is
determined by the tightest of the upper bounds on it.} The importance
of this idea cannot be over-emphasized. On the one hand, it is a
staple of a whole branch of graph theory/network theory results called
max-min theorems. All of them have the flavor that the upper end of
what's possibility coincides with the lower end of the constraints. On
the other hand, it is also the whole basis for the theory of least
upper bounds and greatest lower bounds that we will see in 153 and
that forms the basis for a rigorous study of the reals (which you
might see if you proceed to study real analysis).

\subsection{Some random tricks}

A real-world optimization problem is not usually given in a
ready-to-solve form. Rather, some decisions and judgments need to be
made about the procedure and the general form of the solution in order
to obtain a mathematical setup.

The initial judgment may use general rules: for instance, the rule
that straight line paths, where possible, are shorter than
non-straight line paths. Thus, when asked to find a shortest path
subject to certain constraints, we may be able to narrow it down to a
straight line path and then do the optimization within that.

As a somwhat trickier example, consider the following problem, which
appears on your homework:

\begin{quote}
  Two hallways, one $8$ feet wide and the other $6$ feet wide, meet at
  right angles. Determine the length of the longest ladder that can be
  carried horizontally from one hallway to the other.
\end{quote}

Here, the significance of {\em horizontally} is simply that the ladder
cannot be tilted vertically, a strategy that would enable one to carry
a longer ladder. This problem is a hard one because the nature of the
constraint is not clear. How does the width of the hallways constrain
the length of the ladder that can be passed through?

We need to role-play the {\em process of carrying the ladder}. When a
ladder is being carried along a corridor, it makes the most sense to
align the ladder parallel to the walls of the corridor. When the
direction of the corridor changes, the ladder needs to be rotated to
align it with the new corridor. We must be able to rotate the ladder
through every angle. This leads to the constraint: for every angle,
the ladder must fit in. We then try to find, for every angle $\theta$,
the maximum length of ladder that can fit in at the junction between
the two corridors. Each of these imposes a constraint on the length of
ladder. The most relevant binding constraint is the {\em minimum} of
these lengths.

\subsection{The intuition of tangency}

Let's now look at another problem that also appears on your homework:

\begin{quote}
  A tapestry $7$ feet high hangs on a wall. The lower edge is $9$ feet
  above an observer's eye. How far from the wall should the observer
  stand in order to obtain the most favorable view? Namely, what
  distance from the wall maximizes the visual angle of the observer?
\end{quote}

Here's the intuition behind this problem. If you stand right under the
tapestry, it seems {\em foreshortened}. If, however, you go very far,
then it simply seems small. The quantity that measures how large the
tapestry appears is the visual angle, or the angle between the lines
joining your eyes to the top and bottom of the tapestry. This angle is
zero if you are right under the tapestry, and it approaches zero as
you go out far from the tapestry. Where is it maximum? Somewhere in
between. But where exactly?

You can find this using calculus -- which is what you are expected to
do in this homework. But there is an alternative, related approach
that is more geometric.

The main geometric fact used is that the angle subtended by a chord of
a circle at any two points on the circle on the same side of the chord
is the same.

Now, consider a circle passing through the two ends of the
tapestry. If this circle intersects the horizontal line of possible
locations of your eye at two points $P$ and $Q$, then by the result I
mentioned, the visual angle at $P$ equals the visual angle at
$Q$. Note that for a very large circle, $P$ is very close to the base
of the tapestry and $Q$ is very far away. This helps explain why small
visual angles are achieved both very close and very far away from the
tapestry.

We also see that the smaller the circle, the larger the visual
angle. Thus, the goal is to find the smallest circle passing through
the two ends of the tapestry that intersects the horizontal line of
possible locations of the eye. A little thought reveals that this
occurs when the circle is tangent to the horizontal line. If you imagine
starting with a very large circle and shrinking it further and
further, the circle that is tangent to the horizontal line is the one
at which the circle just leaves the horizontal line. (Having deduced
this, it is possible to determine the precise point using geometry and
algebra, without any calculus. You can verify the answer you obtain
using calculus via this method).

Here's the picture with lots of such circles drawn. Such a system of
circles is called a {\em coaxial system of circles}.

\includegraphics[width=3in]{statueproblem.png}

Here's the same picture with just the circle of tangency and another
circle drawn:

\includegraphics[width=3in]{statueproblemdetailed.png}

{\em Note}: We can use another result of geometry to calculate the
distance of the point of tangency from the foot of the
tapestry. Namely, the result says that if $P$ is a point outside a
circle, $PT$ is a tangent to the circle with point of tangency $T$,
and a secant line through $P$ intersects the circle at $A$ and $B$,
then $PA \cdot PB = PT^2$. We can use this to calculate $PT$ as the
square root of the product of the distances from the base of the
bottom and top of the tapestry.
\subsection{The heuristic of multiple uses}

Suppose a resource (such as fencing wire, which plays the role of
perimeter) is to be divided among two alternative uses: say a square
fence, and a circular fence. It is a fact that, of all possible shapes
with a given perimeter, the circle encloses the largest area. (This is
called the {\em isoperimetric problem}, and although we will not show
it, it is useful to remember). In particular, devoting all the fencing
to the circle yields a larger area than devoting all the fencing to
the square. (This can be checked easily by algebra, and the fact that
$\pi < 4$).

Given this, what is the way of allocating fencing so as to get the
maximum and minimum possible total area? It turns out that for the
{\em maximum possible}, we allocate all resources to the hands-down
better use, which is in this case the circle. However, for the {\em
minimum possible}, the strategy is {\em not} that of allocating
everything to the square. Why not? It turns out that we can do even
worse by providing some fencing to the square and some fencing to the
circle? Why? Because there is some wastage that arises simply from
having two fences. Even though a square is less efficient than a
circle, devoting everything to the square is a little more efficient
than devoting mostly to the square and a bit for the circle. A problem
of this kind appears in Homework 5.

In the good old days when everybody farmed, each time a farmer with
multiple sons died, his land was divided among the sons. As a result,
fencing costs and wastage kept increasing. One solution to this was
{\em primogeniture laws}, which stated that the eldest son was
entitled to the land. While not fair, these laws helped combat the
problem of fragmentation of land holdings.
\subsection{Integer optimization}

A few brief notes on integer optimization may be worthwhile. In many
real-world situations, the possible values that a variable can take
are constrained to be integers. For instance: {\em how many passengers
can travel on this vehicle}? The optimization here thus requires one
to optimize {\em subject to the integer value constraint on the
variables}.

It may seem reasonable at first to believe that the best integer
solution is the integer closest to the best real solution. This is not
always the case. In fact, computer scientists have shown that even
solving systems of linear equations and inequalities in integer
variables has no general-purpose algorithm that runs quickly (subject
to a long-standing conjecture called $P \ne NP$). This is despite the
fact that the analogous problem is very easy to solve for real
variables.

The problem here is that the value of a function can change very
rapidly between a real number and the integers closest to it. See, for
instance, this picture for a function where the maximum value among
values at integers is {\em not} attained at the integer closest to the
absolute maximum:

\includegraphics[width=3in]{integeroptimizationillustration.png}

However, it is useful to look at the behavior of a function over all
real numbers in order to determine the integers where it attains
maxima and minima. For instance, if we can determine where the
function is increasing and decreasing, we can use this information
along with testing some values in order to find out the maxima and
minima. Specifically, what we first do is {\em extend the function to
all real numbers} (by considering the definition of the function
applied to all real numbers) and find the intervals where the function
is increasing and decreasing as a function with real inputs. Then:

\begin{enumerate}
\item If $f$ is increasing on an interval, then the minimum of $f$ on
  that interval occurs at the smallest integer in the interval and the
  maximum occurs at the largest integer in the interval.
\item If $f$ is decreasing on an interval, then the minimum of $f$ on
  that interval occurs at the largest integer in the interval and the
  maximum occurs at the smallest integer in the interval.
\item If we need to find the absolute maximum of $f$ over all
  integers, we can first break up into intervals where $f$ is
  increasing/decreasing, find the maximum over each of those
  intervals, and finally compare the values of all these maxima to
  find which is the largest one.
\end{enumerate}

Here are some simple examples:

\begin{enumerate}
\item Consider a function that is decreasing on $(-\infty,1.3]$ and
  increasing on $[1.3,\infty)$. Then, if viewed as a function on
  reals, this function has a unique absolute minimum at $1.3$. As a
  function on integers, we know that the function is increasing from
  $2$ onwards, so the value for any integer greater than $2$ is
  greater than the value at $2$. Similarly, we know that the value at
  any integer less than $1$ is greater than the value at $1$. So,
  there are two candidates for the absolute minimum among integers:
  the values at $1$ and $2$. We now calculate the values at $1$ and
  $2$ and find which one is smaller.
\item Consider a function that decreases on $(-\infty,-1.1]$,
  increases on $[-1.1,0.1]$, decreases on $[0.1,0.9]$, and then
  increases on $[0.9,\infty)$. On the interval $(-\infty,-1.1]$, the
  minimum among integers is at $-2$. On the interval $[-1.1,0.1]$, the
  minimum among integers is at $-1$. On the interval $[0.1,0.9]$,
  there are no integers. On the interval $[0.9,\infty)$, the minimum
  among integers is at $1$. Thus, the three candidates for the point
  of absolute minimum are $-2$, $-1$, and $1$.
\end{enumerate}

Note that it is {\em not} necessarily true that the integer where the
absolute minimum among integers is attained is the closest integer to
the real number where the absolute minimum among real numbers is
attained. This is because the function can change very rapidly between
a real number and the closest integer. For certain special kinds of
functions (such as quadratic functions), it {\em is} true that the
integer for absolute minimum is the closest integer to the real number
for absolute minimum. But this is due to the symmetric nature of
quadratic functions -- the graph of a quadratic function with positive
leading coefficient is symmetric about the vertical line through its
point of absolute minimum.\footnote{For negative leading coefficient,
the corresponding statement is true if we replace absolute minimum by
absolute maximum.}

\section{Some notes from social and natural sciences}

\subsection{An important maximization: Cobb-Douglas, fair share, and kinetics}

Let's now go to a question considered by some economists in the early
twentieth century. We'll then talk about how a similar question comes
up in chemical reactions.

Suppose a factory is producing some goods using two kinds of inputs:
labor and capital. For a given production process, if the factory
spends $L$ on labor and $K$ on capital, the output of the factory is
given by $L^aK^b$, where $a$ and $b$ are positive numbers. The goal of
the factory is to maximize output for a given expenditure ($L +
K$). In other words, if the factory is spending a total $E$ on labor
and capital put together, how should it allocate $E$ between labor and
capital to obtain the maximum output?

Since $E$ is fixed, we can choose $L$ as the variable and write $K = E
- L$. We thus get that the output is $L^a(E - L)^b$. If we further let
$x = L/E$ (the fraction on labor), then the output is given by $E^{a +
  b}x^a(1 - x)^b$. Thus, in order to maximize output, we need to
maximize $x^a(1 - x)^b$, where $x \in [0,1]$.

A maximization of this sort appears on your homework, and we find
there that the absolute maximum on the interval $[0,1]$ occurs at the
point $a/(a + b)$. Thus, the maximum occurs when $L = Ea/(a + b)$ and
$K = Eb/(a + b)$. Thus, the labor-to-capital expenditure ratio $L/K$
is $a/b$ -- the same as the ratio of exponents.

What this result shows is that the ratio of exponents on labor and
capital represents the relative contributions of labor and capital to
production. Optimization occurs when the allocation of resources is
done according to these relative contributions: a fraction of $a/(a +
b)$ to labor and a fraction of $b/(a + b)$ to capital. In hindsight,
this makes intuitive sense: the larger the value of $a$, the more
sense it makes to invest in labor, because the return on investment in
labor is higher. However, after some point, it also makes sense to
invest a bit in capital, otherwise that becomes a bottleneck. The
proportion should have something to do with the ratio of $a$ and
$b$. Mathematically, we have shown that these two proportions in fact
coincide.

This raises the question of what determines $a$ and $b$ in the first
place. This has something to do with the nature of the production
process. A {\em labor-intensive process} would be one where $a$
dominates and a capital-intensive process would be one where $b$
dominates.

All production functions do not look like the function above. However,
it was the argument of Cobb and Douglas that assuming functions to be
of the above form is a useful simplification and many phenomena of
relative allocation of resources to factors of production can be
understood this way. In many parts of economics and the social
sciences, people wanting to do a simple analysis often begin by
assuming that a given production function is Cobb-Douglas, in order to
get a clear handle on the relative contribution of different factors.

Another place where a similar formulation pops up is chemical
kinetics. Suppose we have a chemical reaction between two substances
$A$ and $B$, with equation of the form $mA + nB \to $ products. The
theory of chemical kinetics suggests that, assuming this reaction is
elementary, the rate of forward reaction is given by $k_f[A]^m[B]^n$
where $k_f$ is a constant (with suitable dimensions), $[A]$ is the
concentration of $A$ and $[B]$ is the concentration of $B$.

Now, the question may be: for a given total concentration, how do you
decide the proportions in which to mix $A$ and $B$ to get the fastest
reaction? This is the same problem in a new guise, and it turns out
that the maximum occurs when $[A]:[B] = m:n$. This is poetic justice,
because this is precisely the {\em right} ratio from the
{\em stoichiometric} viewpoint.

\subsection{Frontier curves and optimal allocation}

An important concept, which you may first see in economics courses,
but which also occurs elsewhere, is that of a {\em production
possibility frontier} or {\em production possibility curve}. Let's
understand these curves in the language of optimization.

Suppose you are running a farm that can produce only two things: wheat
and rice. Now, let's say that you decide to produce $50,000$ bushels
of wheat. Given this constraint, your goal is to produce as much rice
as possible. This is now an optimization problem and you somehow solve
it and find out that you can produce at most $25,000$ bushels of rice
if you want to produce $50,000$ bushels of wheat.

Now, if you instead wanted to produce only $40,000$ bushels of wheat,
it is possible that you can produce more -- say $40,000$ bushels of
rice. Thus, {\em for each quantity of wheat that you choose to
produce}, there is a maximum quantity of rice you can produce with the
given resources. We can thus define a {\em function} that takes as
input the quantity of wheat and outputs the maximum quantity of rice
that can be produced alongside. This is a {\em decreasing} function
(the more wheat you produce, the less resources you can devote to
producing rice) and its domain is from $0$ to the maximum amount of
wheat that you can produce. The largest value in the domain is the
maximum amount of wheat you can produce if you devote all resources to
wheat production, and the value of the function at $0$ if the maximum
amount of rice you can produce if you devote all your resources to
rice production.

The graph of this function is called the {\em production possibility
curve} or {\em production possibility frontier}. The important thing
to note about this graph is that {\em every point on the graph is an
optimal point in some sense} -- there is no way to unambiguously
improve from any of these points. Any point below a point on the
curve, or on the inside of the curve, is achievable but non-optimal,
in the sense that it is possible to increase the production of one or
both the outputs without decreasing the other one. A point above or
outside the production possibility frontier is a point that cannot be
achieved, reflecting the {\em reality of scarcity} or the {\em
limitations of current technology}, depending on your perspective.

\subsection{Can spontaneous processes solve optimization problems?}

First, a little clarification on what the question means. In all the
situations we have seen so far, there is a conscious agent that is
using calculus to find an optimal allocation or optimal value by
explicitly considering constraints. But optimization has been a goal
for living creatures and for nature long before the advent of
calculus. How did they do it?

For instance, bubbles tend to be spherical in order to minimize their
surface area. More generally, the shapes that soap films can take are
minimal surfaces -- they minimize surface area. But are bubbles and
soap films solving a complicated optimization problem by choosing a
spherical shape? Do we need to posit a theory of consciousness and
calculus ability every time we see such optimization in the physical
or biological world?

No. Physical entities (and most primitive biological entities) are not
trying to reach an optimal state -- they simply {\em keep moving
around until} they hit upon a {\em stable equilibrium}, which is
{\em locally optimal}. In fact, the same is true for humans interacting in a
large market. This point is extremely important.

For instance, here are some crude heuristics:

\begin{enumerate}
\item In the world of physics, the reason why mechanical or
  physical systems tend to certain ``optimal'' configurations is that in
  these configurations, there are no forces rending them apart or
  causing them further change.
\item In the world of chemistry, materials keep reacting until they
  reach a configuration where the push to the reaction in one direction
  equals the push to reaction in the other direction.
\item In the world of biology, living creatures explore the space
  around them till they hit on something that's better than the stuff
  around it.
\item In the word of economics, each individual keeps making changes
  in the variables under his/her economic control until reaching a
  situation where a change in either direction is not to his/her
  advantage.
\end{enumerate}

The upshot is that local optima tend to be places of stability simply
because there isn't a tendency to deviate either way, not because
anybody did calculus. You can think of it like an ant moving along the
graph of a curve and stopping when it gets to a peak and would need to
go down both ways.

But this also has a flip side:

\begin{itemize}
\item Local optima need not be absolute optima. That was the whole
  point of our earlier lecture on the subject! But given their
  stability, natural systems may stay stuck at these local optima. To
  get to an even bigger global optima, a {\em push} may be
  needed. (For instance, activation energy in the context of a
  chemical reaction, or the entry of a new competitor in a stagnating
  and non-innovating industry).
\item In some cases, there may be so many different local optima, or
  the situation may be so shaky, that there is never any place to
  settle down at. In some cases, inertia may prevent settling
  down. This causes such phenomena as {\em oscillatory} and {\em
  chaotic} behavior.
\end{itemize}
\end{document}