\documentclass[10pt]{amsart}

%Packages in use
\usepackage{fullpage, hyperref, vipul, enumerate}

%Title details
\title{Diagnostic in-class quiz solutions: due Monday November 25: Subspace, basis, and dimension}
\author{Math 196, Section 57 (Vipul Naik)}
%List of new commands

\begin{document}
\maketitle

\section{Performance review}

23 people took this 3-question quiz. The score distribution was as follows:

\begin{itemize}
\item Score of 0: 2 people
\item Score of 1: 8 people
\item Score of 2: 6 people
\item Score of 3: 7 people
\end{itemize}

The mean score was 1.78.

The question-wise answers and performance review were as follows:

\begin{enumerate}
\item Option (A): 15 people %$15$ people
\item Option (E): 11 people %$12$ people
\item Option (C): 15 people %$8$ people
\end{enumerate}

\section{Solutions}

{\bf PLEASE DO NOT DISCUSS {\em ANY} QUESTIONS.}

This quiz covers material related to the {\tt Linear dependence, bases
  and subspaces} notes corresponding to Sections 3.2 and 3.3 of the
text.

Keep in mind the following facts. Suppose $T:\R^m \to \R^n$ is a
linear transformation. Suppose $A$ is the matrix for $T$, so that
$T(\vec{x}) = A\vec{x}$ for all $\vec{x} \in \R^m$. Then, $A$ is a $n
\times m$ matrix. Further, the following are true:

\begin{itemize}
\item The dimension of the image of $T$ equals the rank of $A$.
\item The dimension of the kernel of $T$, called the {\em nullity} of
  $A$, is $m$ minus the rank of $A$.
\end{itemize}

\begin{enumerate}
\item {\em Do not discuss this!}: Suppose $T:\R^m \to \R^n$ is a
  linear transformation. What is the best we can say about the
  dimension of the image of $T$?

  \begin{enumerate}[(A)]
  \item It is at least $0$ and at most $\min \{m,n\}$. However, we
    cannot be more specific based on the given information.
  \item It is at least $0$ and at most $\max \{m,n \}$. However, we
    cannot be more specific based on the given information.
  \item It is at least $\min \{ m,n \}$ and at most $\max \{ m,n
    \}$. However, we cannot be more specific based on the given information.
  \item It is at least $\min \{ m,n \}$ and at most $m + n$. However,
    we cannot be more specific based on the given information.
  \item It is at least $\max \{ m,n \}$ and at most $m + n$. However,
    we cannot be more specific based on the given information.
  \end{enumerate}

  {\em Answer}: Option (A)

  {\em Explanation}: The dimension of the image equals the rank of the
  matrix for $T$, which is a $n \times m$ matrix, hence is at most
  $\min \{ m, n\}$. It is at least $0$ for obvious reasons. It is easy
  to see that there exist examples for each possible dimension ranging
  from $0$ to $\min \{ m,n \}$.

  {\em Performance review}: 15 out of 23 people got this. 5 chose (E),
  2 chose (B), 1 chose (C).

  {\em Historical note (last time)}: $15$ out of $26$ got this. $5$ chose (C),
  $3$ each chose (B) and (D).

\item {\em Do not discuss this!}: Suppose $T_1,T_2:\R^m \to \R^n$ are
  linear transformations. Suppose the images of $T_1$ and $T_2$ have
  dimensions $d_1$ and $d_2$ respectively. What can we say about the
  dimension of the image of $T_1 + T_2$? Assume that both $m$ and $n$
  are larger than $d_1 + d_2$.

  \begin{enumerate}[(A)]
  \item It is precisely $|d_2 - d_1|$.
  \item It is precisely $\min \{ d_1, d_2 \}$.
  \item It is precisely $\max \{ d_1, d_2 \}$.
  \item It is precisely $d_1 + d_2$.
  \item Based on the information, it could be any integer $r$ with
    $|d_2 - d_1| \le r \le d_1 + d_2$.
  \end{enumerate}

  {\em Answer}: Option (E)

  {\em Explanation}: The image of $T_1 + T_2$ is contained in the sum
  of the images of $T_1$ and $T_2$, hence its dimension is at most the
  dimension of the sum of the images of $T_1$ and $T_2$. The dimension
  of the sum of subspaces is at most equal to the sum of the
  dimensions (because we can take the union of the spanning
  sets). Thus, the dimension of the image of $T_1 + T_2$ is at most
  equal to $d_1 + d_2$.

  For the lower bound of $|d_2 - d_1|$, note that the dimension of the
  image of $T_1$ is at most the sum of the dimensions of the images of
  $T_1 + T_2$ and of $T_2$ and also that the dimension of the image of
  $T_2$ is at most the sum of the dimensions of the images of $T_1 +
  T_2$ and of $T_1$. This gives lower bounds of $d_1 - d_2$ and $d_2 -
  d_1$ respectively on the dimension of the image of $T_1 + T_2$. The
  maximum of these is $|d_2 - d_1|$, which is the lower bound.

  It is easy to construct examples of diagonal matrices with $0$, $1$
  and $-1$ as the diagonal entries to realize any $r$ with $|d_2 -
  d_1| \le r \le d_1 + d_2$.

  {\em Performance review}: 11 out of 23 people got this. 7 chose (C),
  3 chose (B), 2 chose (D).

  {\em Historical note (last time)}: $12$ out of $26$ got this. $8$ chose (B),
  $2$ each chose (A) and (C), $1$ chose (D).

\item {\em Do not discuss this!}: Suppose $V_1$ and $V_2$ are
  subspaces of $\R^n$. We define the sum $V_1 + V_2$ as the subset of
  $\R^n$ comprising all vectors that can be expressed as a sum of a
  vector in $V_1$ and a vector in $V_2$. Define $V_1 \cup V_2$ as the
  set-theoretic union of $V_1$ and $V_2$, i.e., the set of all vectors
  that are either in $V_1$ or in $V_2$. What can we say about these?

  \begin{enumerate}[(A)]
  \item $V_1 \cup V_2 = V_1 + V_2$ and it is a subspace of $\R^n$.
  \item $V_1 \cup V_2$ is contained in $V_1 + V_2$ and both are subspaces of $\R^n$.
  \item $V_1 \cup V_2$ is contained in $V_1 + V_2$, and $V_1 + V_2$ is
    a subspace of $\R^n$. $V_1 \cup V_2$ is generally not a subspace
    of $\R^n$ (though it might be in special cases).
  \item $V_1 \cup V_2$ contains $V_1 + V_2$, and both are subspaces of $\R^n$.
  \item $V_1 \cup V_2$ contains $V_1 + V_2$, and $V_1 \cup V_2$ is a
    subspace of $\R^n$. $V_1 + V_2$ is generally not a subspace of
    $\R^n$ (though it might be in special cases).
  \end{enumerate}

  {\em Answer}: Option (C)

  {\em Explanation}: $V_1 + V_2$ is defined as the set of all vectors
  that can be expressed as the sum of a vector in $V_1$ and a vector
  in $V_2$. In particular, it contains both $V_1$ and $V_2$. The
  reason it contains $V_1$ is that any vector in $V_1$ can be written
  as itself plus the {\em zero vector} of $V_2$. The reason it
  contains $V_2$ is that any vector in $V_2$ can be written as the
  {\em zero vector} of $V_1$ plus that vector itself.

  Thus, $V_1 \cup V_2$ is contained in $V_1 + V_2$. They need not be
  equal. For instance, consider the case that $V_1$ is the span of
  $\vec{e}_1$ and $V_2$ is the span of $\vec{e}_2$ inside $\R^n$. The
  union of these is the set of vectors that are on either of the
  axes. It is a union of two perpendicular lines. The sum on the other
  hand is the plane spanned by the first two coordinates. The sum
  includes linear combinations where both spaces contribute
  nontrivially.

  This also hints at why $V_1 \cup V_2$ does not need to be a
  subspace: it contains both subspaces, but not the vectors that are
  obtained by combining nonzero vectors from both subspaces. In fact,
  $V_1 \cup V_2$ is a subspace if and only if it equals $V_1 + V_2$,
  and this happens if and only if either $V_1 \subseteq V_2$ or $V_2
  \subseteq V_1$.

  {\em Performance review}: 15 out of 23 got this. 5 chose (D), 2
  chose (B), 1 chose (E).

  {\em Historical note (last time)}: $8$ out of $26$ got this. $10$ chose (B),
  $4$ chose (E), $2$ chose (D), $1$ chose (A).

\end{enumerate}
\end{document}
