\documentclass[10pt]{amsart}

%Packages in use
\usepackage{fullpage, hyperref, vipul, enumerate}

%Title details
\title{Take-home class quiz: due Monday October 14: Matrix computations}
\author{Math 196, Section 57 (Vipul Naik)}
%List of new commands

\begin{document}
\maketitle

Your name (print clearly in capital letters): $\underline{\qquad\qquad\qquad\qquad\qquad\qquad\qquad\qquad\qquad\qquad}$

{\bf PLEASE DO {\em NOT} DISCUSS ANY QUESTIONS EXCEPT THE STARRED OR DOUBLE-STARRED QUESTIONS.}

This quiz has a few questions on the mechanics of the computational
execution of Gauss-Jordan elimination, and it has one question on
setting up a linear system.

Suppose $f$ is a function on the positive integers that takes positive
integer values. Suppose $n$ is a parameter related to the input size
of an algorithm. We say that the running time of an algorithm
(respectively, the space requirement of the algorithm) is:

\begin{itemize}
\item $O(f(n))$ if, for large enough $n$, it can be bounded from above
  by a positive constant times $f(n)$.
\item $\Omega(f(n))$ if, for large enough $n$, it can be bounded from
  below by a positive constant times $f(n)$.
\item $\Theta(f(n))$ if it is both $O(f(n))$ and $\Omega(f(n))$.
\end{itemize}

You can read more at:

\url{http://en.wikipedia.org/wiki/Big_O_notation}

\begin{enumerate}
\item (*) If you treat each arithmetic operation (addition,
  subtraction, multiplication, division) of numbers as taking constant
  time, and all entry rewrites and changes as again taking constant
  time per entry, what would be the best description of the worst-case
  running time of the algorithm to convert a $n \times n$ matrix to
  reduced row-echelon form? (Note that this complexity is termed {\em
    arithmetic complexity} and can be distinguished from the {\em bit
    complexity} of the algorithm, which could be considerably
  higher).

  \begin{enumerate}[(A)]
  \item $\Theta(n)$
  \item $\Theta(n^2)$
  \item $\Theta(n^3)$
  \item $\Theta(n^4)$
  \item $\Theta(n^5)$
  \end{enumerate}

  \vspace{0.1in}
  Your answer: $\underline{\qquad\qquad\qquad\qquad\qquad\qquad\qquad}$
  \vspace{0.1in}

\item (*) If you treat each arithmetic operation (addition,
  subtraction, multiplication, division) of numbers as taking constant
  space, and all matrix entries as taking constant space, what would
  be the best description of the worst-case space requirement of the
  algorithm to convert a $n \times n$ matrix to reduced row-echelon
  form? Assume that space is reusable, i.e., it is possible to rewrite
  over existing space used.

  \begin{enumerate}[(A)]
  \item $\Theta(n)$
  \item $\Theta(n^2)$
  \item $\Theta(n^3)$
  \item $\Theta(n^4)$
  \item $\Theta(n^5)$
  \end{enumerate}

  \vspace{0.1in}
  Your answer: $\underline{\qquad\qquad\qquad\qquad\qquad\qquad\qquad}$
  \vspace{0.1in}

\item (*) Suppose the coefficient matrix of a linear system with $n$
  variables and $n$ equations is known in advance, and we can spend as
  much time processing it as we desire in advance (this time will not
  count towards the running time of the algorithm). In other words, we
  can use Gauss-Jordan elimination to row-reduce the coefficient
  matrix in advance. However, we do not have the output column with us
  in advance. What is the worst-case running time of the part of the
  algorithm that runs after the output column is known?

  \begin{enumerate}[(A)]
  \item $\Theta(n)$
  \item $\Theta(n^2)$
  \item $\Theta(n^3)$
  \item $\Theta(n^4)$
  \item $\Theta(n^5)$
  \end{enumerate}

  \vspace{0.1in}
  Your answer: $\underline{\qquad\qquad\qquad\qquad\qquad\qquad\qquad}$
  \vspace{0.1in}

\item {\em Do not discuss this!}: Which of the following matrices does
  {\em not} have the identity matrix as its reduced row-echelon form?

  \begin{enumerate}[(A)]
  \item $$\left[\begin{matrix} 2 & 0 & 0 \\ 0 & 5 & 0 \\ 0 & 0 & -1 \\\end{matrix}\right]$$
  \item $$\left[\begin{matrix} 1 & 2 & 3 \\ 0 & 3 & 5 \\ 0 & 0 & 7 \\\end{matrix} \right]$$
  \item $$\left[\begin{matrix} 4 & 0 & 0 \\ 3 & 1 & 0 \\ 0 & 5 & -6 \\\end{matrix}\right]$$
  \item $$\left[\begin{matrix} 1 & 2 & -3 \\ 4 & -3 & -1 \\ -2 & 1 & 1 \\\end{matrix}\right]$$
  \item $$\left[\begin{matrix} 1 & 2 & 3 \\ 2 & 4 & 7 \\ 3 & 7 & 11 \\\end{matrix}\right]$$
  \end{enumerate}

  \vspace{0.1in}
  Your answer: $\underline{\qquad\qquad\qquad\qquad\qquad\qquad\qquad}$
  \vspace{0.1in}

\item {\em Do not discuss this!}: A number of different consumer price
  indices have been constructed. All of them use the market prices for
  an existing collection of commodities (though not all of them use
  every commodity in the collection) and take a different ``weighted''
  linear combination of those. For instance, one price index might be
  3 times (the price per ton of wheat on the Chicago wheat market) + 4
  times (the price of 1 gallon of unleaded gasoline at a particular
  gas station) + 17 times (the price of Burt's chapstick). Another
  price index might use 30 times (the price of Transcend's 32 GB flash
  drive) + 14 times (the price of 1 gallon of gasoline at a particular
  gas station).

  What is a good way of modeling these?

  \begin{enumerate}[(A)]
  \item The prices of the various goods are stored in a matrix, the
    different weightings used in various indices are stored in a
    vector, and the consumer price indices arise as the output vector
    of the matrix-vector product.
  \item The different weightings used in various indices are stored in
    a matrix, the prices of the various goods are stored in a vector,
    and the consumer price indices arise as the output vector of the
    matrix-vector product.
  \item The prices of the various goods are stored in a matrix, the
    consumer price indices are stored as a vector, and the weightings
    used in the indices arise as the output vector of the
    matrix-vector product.
  \item The different weightings used in various indices are stored in
    a matrix, the consumer price indices are stored in a vector, and
    the prices of the various goods arise as the output vector of the
    matrix-vector product.
  \item The consumer price indices are stored in a matrix, the prices
    of the various goods are stored in a vector, and the weightings
    used in the indices arise as the output vector of the
    matrix-vector product.
  \end{enumerate}

  \vspace{0.1in}
  Your answer: $\underline{\qquad\qquad\qquad\qquad\qquad\qquad\qquad}$
  \vspace{0.1in}
\end{enumerate}
\end{document}
