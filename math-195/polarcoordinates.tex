\documentclass[10pt]{amsart}
\usepackage{fullpage,hyperref,vipul,graphicx}
\title{Polar coordinates}
\author{Math 195, Section 59 (Vipul Naik)}

\begin{document}
\maketitle

{\bf Corresponding material in the book}: Section 10.3.

{\bf What students should definitely get}: The information needed to
set up a polar coordinate system, how to go between a physical point
and its polar coordinates, conversion back and forth between Cartesian
and polar coordinates. Thinking of polar coordinate descriptions as
parametric descriptions when viewed in Cartesian coordinates. Using
this to compute the slope of the tangent line.

{\bf What students should hopefully get}: The dimensionality of the
plane is $2$, so $2$ parameters are needed to describe a point in any
decent coordinate system. What happens when we fix one coordinate in a
Cartesian or polar coordinate system. Why spirals are easy to describe
in polar coordinates. How various symmetries in implicit and
functional descriptions correspond to geometric symmetries.

\section*{Executive summary}

Words ...

\begin{enumerate}
\item {\em Specifying a polar coordinate system}: To specify a polar
  coordinate system, we need to choose a point (called the {\em
  origin} or {\em pole}), a half-line starting at the point (called
  the {\em polar axis} or {\em reference line}) and an orientation of
  the plane (chosen counter-clockwise in the usual depictions).
\item {\em Finding the polar coordinates of a point and vice versa}:
  The radial coordinate $r$ is the distance between the point and the
  pole. The angular coordinate $\theta$ is the angle (measured in the
  counter-clockwise direction) from the polar axis to the line segment
  from the pole to the point. Note that $\theta$ is uniquely defined
  up to addition of multiples of $2\pi$, and it becomes truly unique
  if we restrict it to a half-open half-closed interval of length
  $2\pi$. {\em The exception is the pole itself, for which $\theta$ is
  undefined} in the sense that any value of $\theta$ could be chosen.
\item {\em Converting between Cartesian and polar coordinates}: If we
  take the polar axis as the positive $x$-axis and the axis at an
  angle of $+\pi/2$ from it as the positive $y$-axis, we get a
  Cartesian coordinate system. The point defined by polar coordinates
  $(r,\theta)$ has Cartesian coordinates $(r\cos \theta,r \sin
  \theta)$. Conversely, given a point with Cartesian coordinates
  $(x,y)$ the corresponding polar coordinates are $r = \sqrt{x^2 +
  y^2}$ and $\theta$ is the unique angle (up to addition of multiples
  of $2\pi$) such that $x = r \cos \theta$, $y = r\sin \theta$.
\end{enumerate}

Actions ...

\begin{enumerate}
\item A functional description of the form $r = F(\theta)$ gives rise
  to a parametric description in Cartesian coordinates: $x =
  F(\theta)\cos \theta$ and $y = F(\theta)\sin \theta$. We can do the
  usual things (like find slopes of tangent lines) using this
  parametric description. Note that here, $\theta$ is typically
  allowed to vary over all of $\R$ rather than simply being restricted
  to an interval of length $2\pi$. The slope of the tangnet line in
  Cartesian terms is given by:

$$\frac{dy}{dx} = \frac{dy/d\theta}{dx/d\theta} = \frac{d(F(\theta)\sin \theta)/d\theta}{d(F(\theta)\cos \theta)/d\theta} = \frac{F'(\theta)\sin \theta + F(\theta)\cos \theta}{F'(\theta) \cos \theta - F(\theta)\sin \theta}$$

\item An implicit (relational) description in Cartesian coordinates
  can be converted to a description in polar coordinates by replacing
  $x$ by $r\cos \theta$ and $y$ by $r\sin \theta$.
\item An implicit (relational) description in polar coordinates can
  sometimes be converted to a description in Cartesian coordinates,
  but with some ambiguity. General idea: replace $r$ by $\sqrt{x^2 +
  y^2}$, $\cos \theta$ by $x/\sqrt{x^2 + y^2}$, and $\sin \theta$ by
  $y/\sqrt{x^2 + y^2}$.
\end{enumerate}
\section{Review of Cartesian coordinates}

\subsection{Descartes' achievement}

We're so used to Cartesian coordinates that we don't often give them
much thought. But the idea of using Cartesian coordinates to describe
a plane, or to describe three-dimensional space, was a fundamental
breakthrough when it did occur. The idea here being that something as
geometric as a plane or space could be represented purely by a tuple
of real numbers.

There are some aspects of the {\em rectangular Cartesian coordinate
system} that are worth disaggregating:

\begin{enumerate}
\item The {\em number} of parameters used is equal to the {\em
  dimensionality} of the system being studied. The concept of
  dimensionality as the {\em number of free parameters} or {\em number
  of degrees of freedom} is probably not something totally new to you.
\item In order to actually specify a Cartesian coordinate system, we
  need to choose a pair of orthogonal lines, an ordering of these
  lines, and a direction to be labeled positive within each line.
\item Once we choose an origin and an ordered pair of orthogonal
  directed lines through it, we can use Cartesian coordinates to
  proceed from an ordered pair of real numbers to a point in the
  plane, and back. The two procedures are reverses of each other.
\item Different choices of origin and different choices of direction
  for the pair of perpendicular lines give different choices of
  Cartesian coordinate systems. Moving from one to the other {\em
  geometrically} corresponds to translations, rotations, and
  reflections. Moving from one to the other {\em algebraically}
  corresponds to some specific algebraic operations on the
  coordinates.
\end{enumerate}

The most important of these ideas is (1). In some sense, the
dimensionality of the plane -- namely $2$, is far more fundamental
than the specific choice of coordinate system used. We will soon
construct a new kind of coordinate system called a {\em polar
coordinate system}. This looks very different from a Cartesian
coordinate system, but the number of real parameters needed to
describe a point remains $2$.

\subsection{Fixing one coordinate in the Cartesian system}

Let's consider another aspect of the Cartesian coordinate system. A
point in the Cartesian coordinate system is given by a pair of
coordinates $(x,y)$. What happens if we fix one coordinate and let the
other vary over $\R$. Specifically:

\begin{itemize}
\item If we fix a value of $x$ to $x_0$ and let $y$ vary over $\R$, we
  get a vertical line given by $x = x_0$. For different choices of
  $x_0$, we get parallel lines. Overall, we get a family of parallel
  lines.
\item If we fix a value of $y$ to $y_0$ and let $x$ vary over $\R$, we
  get a horizontal line given by $y = y_0$. Overall, we get a family
  of parallel lines.
\end{itemize}

\section{Polar coordinates}

\subsection{The key definitions}

The key idea behind polar coordinates is to specify the {\em distance}
from a specified origin and the {\em direction} of the line joining
the point to the origin.

To create a polar coordinate system, we need the following pieces of data:

\begin{itemize}
\item A point selected as the {\em pole} or {\em origin} for the
  coordinate system.
\item A half-line (ray) with the point at its endpoint. We will call
  this the {\em polar axis} or {\em reference line}.
\item An orientation (counter-clockwise) on the plane.
\end{itemize}

Every point has two coordinates:

\begin{itemize}
\item The {\em radial coordinate}, denoted $r$, which is the distance
  from the origin to that point.
\item The {\em angular coordinate}, denoted $\theta$, which is the
  angle made between the reference line and the line segment joining the
  origin to that point, measured counter-clockwise.
\end{itemize}

Some important notes:

\begin{itemize}
\item The radial coordinate is a nonnegative real number.
\item The angular coordinate for the pole is not defined. In fact,
  {\em any} value of $\theta$ could be used for the pole and it would
  serve to describe the pole. This is a kind of degeneracy or
  singularity.
\item For any other point, the angular coordinate is unique up to
  multiples of $2\pi$. To make it truly unique, we usually adopt the
  convention that the angle $\theta$ must satisfy $0 \le \theta <
  2\pi$.
\end{itemize}

The role of coordinates in a polar coordinate system is
asymmetric. The $r$-coordinate is a length coordinate, and the
$\theta$-coordinate is a dimensionless angle coordinate. This
contrasts with the Cartesian coordinate system where both coordinates
play a symmetric role as lengths.

\subsection{What happens if we fix one coordinate?}

We note that:

\begin{itemize}
\item If we fix a given value of $r$ but allow $\theta$ to vary
  freely, we get a circle centered at the origin. The exceptional case
  $r = 0$ gives us the single point namely the origin. Thus, we get a
  family of concentric circles centered at the origin.
\item If we fix a given value of $\theta$ but allow $r$ to vary
  freely, we get a ray (half-line) starting at the origin.
\end{itemize}

\subsection{Polar and Cartesian coordinates: conversion}

Every polar coordinate system has a corresponding Cartesian coordinate
system, and vice versa. For a Cartesian coordinate system, we convert
to a polar coordinate system by selecting the same origin, taking the
reference line as the positive $x$-axis, and choosing the orientation
as counter-clockwise, from the positive $x$-axis to the positive
$y$-axis.

With this back-and-forth, here are the conversion rules:

\begin{itemize}
\item The Cartesian coordinates $(x,y)$ gives $r = \sqrt{x^2 + y^2}$
  and $\theta$ is the angle $0 \le \theta < 2\pi$ such that $r \cos
  \theta = x$ and $r \sin \theta = y$.
\item The polar coordinates $(r,\theta)$ gives the Cartesian
  coordinates $x = r\cos \theta$ and $y = r\sin \theta$.
\end{itemize}

With these conversion rules, we can derive formulas involving polar
coordinates from the corresponding formulas involving Cartesian
coordinates.

\subsection{Polar coordinates with negative radial coordinate}

A slight variation on the polar coordinate theme is the case where we
consider polar coordinates with {\em negative} radial coordinate
value. Here $r$ is the negative of the distance from the pole to the
point, and $\theta$ is the usual $\theta$ shifted by $\pi$, so it is
the angle made from the polar axis to the half-line in the {\em
opposite} direction to that joining the pole to the point.
\section{Descriptions of curves in polar coordinates}

\subsection{$r$ as a function of $\theta$}

When we give this kind of description, we {\em usually} allow $\theta$
to vary over all real numbers, rather than just restrict it to an
interval of length $2\pi$. Typically, the curves for which these
descriptions work well are {\em spirals} starting out at the origin
and spiraling outward. For instance, the curve with equation $r =
e^{\theta}$ is a spiral.

One way of thinking of these curves in terms of the {\em usual
Cartesian coordinate system} is in parametric terms -- $\theta$ is a
parameter. If we denote $r = F(\theta)$, then the two coordinate
functions $x$ and $y$ are given by $x = F(\theta)\cos \theta$ and $y =
F(\theta)\sin \theta$.

\subsection{Case of negative $r$ values}

In some cases, the expression $r = F(\theta)$ gives rise to negative
$r$-values for some values of $\theta$. In this case, we interpret
these the way we discussed the interpretation of negative $r$-values.

\subsection{Slope of tangent line in Cartesian terms}

We can determine the slope of the tangent line as follows:

$$\frac{dy}{dx} = \frac{dy/d\theta}{dx/d\theta} = \frac{d(F(\theta)\sin \theta)/d\theta}{d(F(\theta)\cos \theta)/d\theta} = \frac{F'(\theta)\sin \theta + F(\theta)\cos \theta}{F'(\theta) \cos \theta - F(\theta)\sin \theta}$$

Using this, we can determine the equation of the tangent line in a
Cartesian coordinate system.

\subsection{Relational description}

In addition to functional descriptions of the form $r = F(\theta)$, we
could more generally have implicit (relational) descriptions between
$r$ and $\theta$, i.e., descriptions of the form $H(r,\theta) = 0$
where $H$ is a function of two variables. These are more general than
functional descriptions.

Note that it is generally possible to convert a relational description
in a Cartesian coordinate system to a relational description in a
polar coordinate system. Starting with a relational description
$G(x,y) = 0$ in a Cartesian coordinate system we get the corresponding
polar relation by setting $x = r \cos \theta$ and $y = r \sin
\theta$. For instance, the parabola $y = x^2$, in polar coordinates,
becomes:

$$r \sin \theta = r^2 \cos^2 \theta$$

which simplifies to:

$$r(\sin \theta - r\cos^2 \theta) = 0$$

It turns out that the $r$ solution is subsumed in the other, so we get:

$$\sin \theta = r\cos^2 \theta$$

Similarly if we have the equation:

$$xy = \sin(x^2 + y^2)$$

Then, in polar coordinates, this becomes:

$$r^2 \cos \theta \sin \theta = \sin(r^2 \cos^2 \theta + r^2\sin^2\theta)$$

This simplifies to:

$$r^2 \cos \theta \sin \theta = \sin(r^2)$$

Going the other way around, the rule is to plug in $r = \sqrt{x^2 +
y^2}$, $\cos \theta = x/\sqrt{x^2 + y^2}$ and $\sin \theta =
y/\sqrt{x^2 + y^2}$. Note that directly plugging $\theta$ in reverse
is tricky, because there is no single shorthand expression for
$\theta$ -- the expression depends on the signs of $x$ and $y$ and
gets a little messy.

\subsection{Symmetries for polar equations}

Here are two symmetries of note:

\begin{itemize}
\item {\em Mirror symmetry}: A functional or relational description is
  symmetric about the polar axis (the reference line) if it satisfies
  the condition that the condition is satisfied by replacing $\theta$
  with $-\theta$. More generally, if replacing $\theta$ by $2\alpha -
  \theta$ preserves the condition, then there is symmetry about the
  line $\theta = \alpha$. Thus, for instance, if replacing $\theta$ by
  $\pi - \theta$ preserves the condition, the curve has mirror
  symmetry about the $y$-axis.
\item {\em Half turn symmetry}: If replacing $\theta$ by $\theta +
  \pi$ preserves the condition, then the curve has half turn symmetry
  about the pole (origin). Similarly, if replacing $r$ by $-r$
  preserves the condition, then the curve has half turn symmetry about
  the pole (origin).
\end{itemize}

\end{document}