\documentclass[10pt]{amsart}

%Packages in use
\usepackage{fullpage, hyperref, vipul, enumerate}

%Title details
\title{Class quiz solutions: January 18: Mathematical induction (actual date Jan 20)}
\author{Math 153, Section 55 (Vipul Naik)}
%List of new commands

\begin{document}
\maketitle

\section{Performance review}

$11$ students took this $4$-question quiz. The score distribution was
as follows:

\begin{itemize}
\item Score of $0$: $3$ people
\item Score of $1$: $1$ preson
\item Score of $2$: $2$ people
\item Score of $3$: $2$ people
\item Score of $4$: $3$ people
\end{itemize}

The mean score was $2.1$.

Here are the question wise solutions and performance:

\begin{enumerate}
\item Option (C): $6$ people
\item Option (C): $6$ people
\item Option (D): $7$ people
\item Option (E): $4$ people
\end{enumerate}

\section{Solutions}

\begin{enumerate}
\item Suppose $S$ is a subset of the natural numbers with the property
  that $1 \in S$ and $k \in S \implies k + 2 \in S$. What can we
  conclude is {\bf definitely true} about $S$?

  \begin{enumerate}[(A)]
  \item $S$ contains all natural numbers.
  \item $S$ contains all natural numbers other than $2$. It may or may
    not contain $2$.
  \item $S$ contains all odd natural numbers.
  \item $S$ contains all even natural numbers.
  \item $S$ does not contain any natural number other than $1$.
  \end{enumerate}

  {\em Answer}: Option (C)

  {\em Explanation}: Once we know that $1 \in S$, then we get $3 \in
  S$, and then $5 \in S$, and this way, we get $1,3,5,7,\dots$ are all
  in $S$. However, there is no way to conclude anything about any of
  the even numbers.

  {\em Performance review}: $6$ out of $11$ got this correct. $2$ each
  chose (A) and (B), $1$ chose (D).

  {\em Historical note (last year)}: $18$ people got this correct. $6$
  people chose (B) and $5$ people chose (A).
\item Suppose $S$ is a subset of the natural numbers with the property
  that whenever $k \in S$, we have $k + 5 \in S$. Which of these is
  the {\bf smallest subset} $T$ with the property that checking $T
  \subseteq S$ is sufficient to show that $S$ is the set of all
  natural numbers?

  \begin{enumerate}[(A)]
  \item $\{ 1,2,3 \}$
  \item $\{ 1,2,3,4 \}$
  \item $\{ 1,2,3,4,5 \}$
  \item $\{ 1,4 \}$
  \item $\{ 1,3,5 \}$
  \end{enumerate}

  {\em Answer}: Option (C)

  {\em Explanation}: The fact that $k \in S \implies k + 5 \in S$ does
  not say anything about the numbers $1,2,3,4,5$ because subtracting
  $5$ from any of these gives something that is not a natural
  number. Thus, we need to guarantee this subset to be in $S$. Once we
  have all these in $S$, everything else is automatically in $S$
  because it is of the form $5k + r$ where $r \in \{ 1,2,3,4,5 \}$.

  This is like an extended/enhanced base case.

  {\em Performance review}: $6$ out of $11$ got this correct. $3$
  chose (B), $1$ each chose (D) and (E).

  {\em Historical note (last year)}: $21$ people got this correct. $3$
  people chose (B), $3$ people chose (D), and $2$ people chose (A).
\item Consider the function $f(x) := a \sin x + b \cos x$, with $a$,
  $b$ nonzero reals. The $n^{th}$ derivative of $f$ is denote
  $f^{(n)}$. The association $n \mapsto f^{(n)}$ is periodic, i.e.,
  there is a unique smallest positive integer $h$ such that $f^{(n +
  h)} = f^{(n)}$ for all $n$. What is {\bf this value} of $h$?

  \begin{enumerate}[(A)]
  \item $1$
  \item $2$
  \item $3$
  \item $4$
  \item $5$
  \end{enumerate}

  {\em Answer}: Option (D)

  {\em Explanation}: For both $\sin$ and $\cos$ have derivative cycles
  of size four; hence, so does any nontrivial linear combination of
  these.

  {\em Performance review}: $7$ out of $11$ got this. $2$ chose (A),
  $1$ each chose (B) and (C).

  {\em Historical note (last year)}: $25$ people got this correct. $3$
  people chose (B) and $1$ person chose (C).

\item What is the {\bf sum} $\sum_{k=2}^n \frac{1}{k^2 - 1}$ for a
  positive integer $n \ge 2$?

  \begin{enumerate}[(A)]
  \item $\frac{3}{2} - \frac{2n + 3}{2(n+1)}$
  \item $\frac{3}{2} - \frac{2n + 3}{n(n+1)}$
  \item $\frac{3}{4} - \frac{2n + 1}{(n + 1)(n+2)}$
  \item $\frac{3}{4} - \frac{2n - 1}{2n(n-1)}$
  \item $\frac{3}{4} - \frac{2n + 1}{2n(n+1)}$
  \end{enumerate}

  {\em Answer}: Option (E)

  {\em Explanation}: We give below the full proof by induction.

    {\em Base case for induction}: This is the case $n = 2$. In this
    case, the left side is $1/(2^2 - 1) = 1/3$ and the right side is
    $3/4 - (2 \cdot 2 + 1)/(2 \cdot 2 \cdot (2 + 1)) = 3/4 - 5/12 =
    1/3$. Thus, the two sides are equal for $n = 2$ and the base case
    is settled.

    {\em Inductive step}: Suppose the statement is true for $k$. We
    want to show it is true for $k + 1$.

    In other words, we are given that:

    \begin{equation*}
      \frac{1}{2^2 - 1} + \frac{1}{3^2 - 1} + \dots + \frac{1}{k^2 - 1} = \frac{3}{4} - \frac{2k + 1}{2k(k+1)} \tag{*}
    \end{equation*}

    and we want to show that:

    \begin{equation*}
      \frac{1}{2^2 - 1} + \frac{1}{3^2 - 1} + \dots + \frac{1}{k^2 - 1} + \frac{1}{(k + 1)^2 - 1} = \frac{3}{4} - \frac{2(k+1) + 1}{2(k+1)((k+1)+1)} \tag{**}
    \end{equation*}

    Let's do this. Add $1/((k+1)^2 - 1)$ to both sides of (*):

    \begin{eqnarray*}
      \frac{1}{2^2 - 1} + \frac{1}{3^2 - 1} + \dots + \frac{1}{k^2 - 1} + \frac{1}{(k + 1)^2 - 1} & = & \frac{3}{4} - \frac{2k + 1}{2k(k+1)} + \frac{1}{(k+1)^2 - 1} \\
      \implies \frac{1}{2^2 - 1} + \frac{1}{3^2 - 1} + \dots + \frac{1}{k^2 - 1} + \frac{1}{(k + 1)^2 - 1} & = & \frac{3}{4} - \frac{2k + 1}{2k(k+1)} + \frac{1}{k(k+2)}\\
      \implies \frac{1}{2^2 - 1} + \frac{1}{3^2 - 1} + \dots + \frac{1}{k^2 - 1} + \frac{1}{(k + 1)^2 - 1}& = &\frac{3}{4} - \frac{(2k + 1)(k+2) - 2(k + 1)}{2k(k+1)(k+2)}\\
      \implies \frac{1}{2^2 - 1} + \frac{1}{3^2 - 1} + \dots + \frac{1}{k^2 - 1} + \frac{1}{(k + 1)^2 - 1}& = &\frac{3}{4} - \frac{2k^2 + 3k}{2k(k+1)(k+2)}\\
      \implies \frac{1}{2^2 - 1} + \frac{1}{3^2 - 1} + \dots + \frac{1}{k^2 - 1} + \frac{1}{(k + 1)^2 - 1}& = &\frac{3}{4} - \frac{2k + 3}{2(k+1)(k+2)}\\
      \implies \frac{1}{2^2 - 1} + \frac{1}{3^2 - 1} + \dots + \frac{1}{k^2 - 1} + \frac{1}{(k + 1)^2 - 1}& = &\frac{3}{4} - \frac{2(k+1) + 1}{2(k+1)((k+1)+1)}
    \end{eqnarray*}

    which is precisely what we want, namely (**). This completes the
    inductive step and we thus have the result by the principle of
    mathematical induction.

    {\em Performance review}: $4$ out of $11$ got this. $3$ chose (D),
    $2$ each chose (A) and (B).

    {\em Historical note (last year)}: $18$ people got this correct. $4$
    people chose (D), $2$ people chose (A), $2$ people chose (C), $2$
    people chose (B), and $1$ left the question blank.
\end{enumerate}

\end{document}
