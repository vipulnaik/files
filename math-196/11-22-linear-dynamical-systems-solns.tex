\documentclass[10pt]{amsart}

%Packages in use
\usepackage{fullpage, hyperref, vipul, enumerate}

%Title details
\title{Take-home class quiz solutions: due Friday November 22: Linear dynamical systems}
\author{Math 196, Section 57 (Vipul Naik)}
%List of new commands

\begin{document}
\maketitle

\section{Performance review}

24 people took this 12-question quiz. The score distribution was as follows:

\begin{itemize}
\item Score of 3: 3 people
\item Score of 4: 5 people
\item Score of 5: 6 people
\item Score of 6: 4 people
\item Score of 7: 3 people
\item Score of 10: 3 people
\end{itemize}

The mean score was about 5.58.

The question-wise answers and performance review were as follows:

\begin{enumerate}
\item Option (A): 21 people %$21$ people
\item Option (D): 11 people %$5$ people
\item Option (E): 8 people %$5$ people
\item Option (E): 10 people %$3$ people
\item Option (A): 11 people %$4$ people
\item Option (B): 13 people %$9$ people
\item Option (C): 14 people %$17$ people
\item Option (B): 15 people %$13$ people
\item Option (A): 9 people %$8$ people
\item Option (C): 7 people %$5$ people
\item Option (A): 8 people %$7$ people
\item Option (A): 7 people %$6$ people
\end{enumerate}

\section{Solutions}

{\bf PLEASE FEEL FREE TO DISCUSS {\em ALL} QUESTIONS.}

This quiz covers a topic that we will not be able to get to formally
in the course due to time constraints. The corresponding section of
the book is Section 7.1, and there is more relevant material discussed
in the later sections of Chapter 7. However, you do not need to read
those sections in order to attempt this quiz. Also, simply mastering
the computational techniques in those sections of the book will not
help you much with the quiz questions.

The questions here consider a linear dynamical system. Consider a
linear transformation $T:\R^n \to \R^n$. Let $A$ be the matrix of $T$,
so that $A$ is a $n \times n$ matrix. For any positive integer $r$,
the matrix $A^r$ is the matrix for the linear transformation $T^r$
(note here that $T^r$ refers to the $r$-fold {\em composite} of
$T$). The goal is to determine, starting off with an arbitrary vector
$\vec{x} \in \R^n$, how the following sequence behaves:

$$\vec{x}, T(\vec{x}), T^2(\vec{x}), T^3(\vec{x}), \dots$$

More explicitly, each term of the sequence is obtained by applying $T$
to the preceding term. In other words, the sequence is:

$$\vec{x}, T(\vec{x}), T(T(\vec{x})), T(T(T(\vec{x}))), \dots$$

\begin{enumerate}
\item What is the necessary and sufficient condition on $A$ such that
  for {\em every} choice of $\vec{x} \in \R^n$, the sequence described
  above eventually reaches, and stays at, the zero vector? Note that
  if it reaches the zero vector, it must do so in at most $n$
  steps. Please see Option (E) before answering.

  \begin{enumerate}[(A)]
  \item $A$ is a nilpotent matrix.
  \item $A$ is an idempotent matrix.
  \item $A$ is an invertible matrix.
  \item $A$ is a non-invertible matrix.
  \item None of the above.
  \end{enumerate}

  {\em Answer}: Option (A)

  {\em Explanation}: First, note that if $A$ is a nilpotent matrix,
  there exists a positive integer $r$ such that $A^r$ is the zero
  matrix, which is equivalent to requiring that $T^r$ be the zero
  linear transformation. In particular, this means that $T^r(\vec{x})$
  is the zero vector for all initial vectors $\vec{x}$, so every
  sequence eventually reaches the zero vector.

  Other direction: From the fact stated, if the sequence reaches the
  zero vector, it must do so in $n$ steps, so that this means that if
  the condition holds for every $\vec{x} \in \R^n$, then $T^n$ sends
  every vector to the zero vector. In particular, this means that
  $A^n$ is the zero matrix, so $A$ is nilpotent.

  {\em Performance review}: 21 out of 24 got this. 3 chose (E).

  {\em Historical note (last time)}: $21$ out of $24$ got this. $2$ chose (E),
  $1$ chose (B).

\item What is the necessary and sufficient condition on $A$ such that
  there {\em exists} a nonzero vector $\vec{x} \in \R^n$ for which the
  sequence described above eventually reaches, and stays at, the zero
  vector?  Note that if it reaches the zero vector, it must do so in
  at most $n$ steps. Please see Option (E) before answering.

  \begin{enumerate}[(A)]
  \item $A$ is a nilpotent matrix.
  \item $A$ is an idempotent matrix.
  \item $A$ is an invertible matrix.
  \item $A$ is a non-invertible matrix.
  \item None of the above.
  \end{enumerate}

  {\em Answer}: Option (D)

  {\em Explanation}: Suppose $\vec{x}$ is a nonzero vector and $r$ is
  the smallest positive integer such that the vector $A^r\vec{x}$ is
  the zero vector. Then, $A(A^{r-1}\vec{x}) = 0$ but $A^{r-1}\vec{x}
  \ne 0$, so that $A^{r-1}\vec{x}$ is a nonzero vector in the kernel
  of $T$. Thus, $T$ has a nonzero kernel, so it must be
  non-invertible, hence $A$ must be a non-invertible matrix.

  Conversely, if $A$ is non-invertible, there is a nonzero vector, say
  $\vec{x}$, in the kernel of $A$. This vector can be used.

  {\em Performance review}: 11 out of 24 got this. 7 chose (A), 5
  chose (E), 1 chose (B).

  {\em Historical note (last time)}: $5$ out of $24$ got this. $15$ chose (A),
  $3$ chose (C), and $1$ chose (E).

\item What is the necessary and sufficient condition on $A$ such that
  for {\em every} choice of $\vec{x} \in \R^n$, the sequence described
  above returns to $\vec{x}$ after a finite and positive number of
  steps? Please see Option (E) before answering.

  \begin{enumerate}[(A)]
  \item $A$ is a nilpotent matrix.
  \item $A$ is an idempotent matrix.
  \item $A$ is an invertible matrix.
  \item $A$ is a non-invertible matrix.
  \item None of the above.
  \end{enumerate}

  {\em Answer}: Option (E)

  {\em Explanation}: It is definitely a {\em necessary} condition that
  $A$ be invertible, otherwise there would be a nonzero vector in its
  kernel for which the sequence could never return to the
  vector. However, this is not sufficient. Consider the case where $n
  = 1$ and the matrix $A$ is $[2]$. This is invertible, but applying
  it repeatedly to a nonzero vector can never get us back to the
  original vector.

  {\em Performance review}: 8 out of 24 got this. 16 chose (B).

  {\em Historical note (last time)}: $5$ out of $24$ got this. $16$ chose (B),
  $2$ chose (C), $1$ chose (A).

\item What is the necessary and sufficient condition on $A$ such that
  there {\em exists} a nonzero vector $\vec{x} \in \R^n$ for which the
  sequence described above returns to $\vec{x}$ after a finite and
  positive number of steps? Please see Option (E) before answering.

  \begin{enumerate}[(A)]
  \item $A$ is a nilpotent matrix.
  \item $A$ is an idempotent matrix.
  \item $A$ is an invertible matrix.
  \item $A$ is a non-invertible matrix.
  \item None of the above.
  \end{enumerate}

  {\em Answer}: Option (E)

  {\em Explanation}: Similar to the preceding question, except now
  that $A$ does not even need to be invertible.

  {\em Performance review}: 10 out of 24 got this. 7 chose (C), 6
  chose (B), 1 chose (A).

  {\em Historical note (last time)}: $3$ out of $24$ got this. $11$ chose (A),
  $9$ chose (B), $1$ chose (C).

\item Suppose $n = 2$ and $T$ is a rotation by an angle that is a
  rational multiple of $\pi$. What can we say about the range of
  the sequence

  $$\vec{x}, T(\vec{x}), T^2(\vec{x}), T^3(\vec{x}), \dots$$

  starting from a nonzero vector $\vec{x}$?

  \begin{enumerate}[(A)]
  \item The range is finite, i.e., there are only finitely many
    distinct vectors in the sequence.
  \item The range is infinite and forms a dense subset of the circle
    centered at the origin and with radius equal to the length of the
    vector $\vec{x}$. However, it is not the entire circle.
  \item The range is infinite and is the entire circle centered at the
    origin and with radius equal to the length of the vector
    $\vec{x}$.
  \item The range is infinite and forms a dense subset of the line of
    the vector $\vec{x}$ (excluding the origin), but is not the entire
    line (excluding the origin).
  \item The range is infinite and is the entire line of the vector
    $\vec{x}$, excluding the origin.
  \end{enumerate}

  {\em Answer}: Option (A)

  {\em Explanation}: If the angle of rotation for $T$ is $\theta$,
  then the angle of rotation for $T^r$ is $r\theta$. This is because
  angles of rotation add up when we compose the rotations.

  Since the angle of rotation $\theta$ is a rational multiple of
  $\pi$, it is of the form $p\pi/q$ where $p$ and $q$ are integers
  with $q \ne 0$. Then, $2q\theta = 2\pi p$. This implies that
  $T^{2q}$ is rotation by an integer multiple of $2\pi$, and hence, is
  the identity transformation. In particular, this means that for any
  nonzero vector $\vec{x}$, the sequence $\vec{x}, T(\vec{x}),
  T^2(\vec{x}), \dots$ returns to $\vec{x}$ at $T^{2q}(\vec{x})$. Beyond
  that point, it will just cycle the same set of vectors.

  {\em Performance review}: 11 out of 24 got this. 9 chose (C), 3
  chose (B), 1 chose (D).

  {\em Historical note (last time)}: $4$ out of $24$ got this. $12$ chose (B),
  $5$ chose (C), $2$ chose (E), $1$ chose (D).

\item Suppose $n = 2$ and $T$ is a rotation by an angle that is a
  irrational multiple of $\pi$. What can we say about the range of
  the sequence

  $$\vec{x}, T(\vec{x}), T^2(\vec{x}), T^3(\vec{x}), \dots$$

  starting from a nonzero vector $\vec{x}$?

  \begin{enumerate}[(A)]
  \item The range is finite, i.e., there are only finitely many
    distinct vectors in the sequence.
  \item The range is infinite and forms a dense subset of the circle
    centered at the origin and with radius equal to the length of the
    vector $\vec{x}$. However, it is not the entire circle.
  \item The range is infinite and is the entire circle centered at the
    origin and with radius equal to the length of the vector
    $\vec{x}$.
  \item The range is infinite and forms a dense subset of the line of
    the vector $\vec{x}$ (excluding the origin), but is not the entire
    line (excluding the origin).
  \item The range is infinite and is the entire line of the vector
    $\vec{x}$, excluding the origin.
  \end{enumerate}

  {\em Answer}: Option (B)

  {\em Explanation}: Since $T$ is rotation by an {\em irrational}
  multiple $\theta$ of $\pi$, there is no positive integer multiple of
  $\theta$ that is also an integer multiple of $2\pi$. Thus, $T^r$ is
  not the identity map for any positive integer $r$, so there is no
  cycling back, so we do get an infinite set of vectors. All of them
  have the same length as $\vec{x}$, because rotations preserve
  length. Thus, they lie on the circle centered at the origin with
  length equal to the length of the vector $\vec{x}$. In fact, we can
  show that they form a dense subset of the circle, i.e., they come
  arbitrarily close to every point of the circle. Note, however, that
  we do not get all points on the circle. For instance, $-\vec{x}$ is
  not in the range, because achieving it would require a nonzero
  rational (in fact, odd integer) multiple of $\pi$, and no integer
  multiple of an irrational multiple of $\pi$ is of that form.

  {\em Performance review}: 13 out of 24 got this. 8 chose (C), 2
  chose (D), 1 chose (A).

  {\em Historical note (last time)}: $9$ out of $24$ got this. $4$ chose (A),
  $7$ chose (C), $3$ chose (D), $1$ chose (E).

  \vspace{0.6in}

  We return to generic $n$ now.

\item A nonzero vector $\vec{x}$ is termed an {\em eigenvector} for a
  linear transformation $T: \R^n \to \R^n$ with {\em eigenvalue} a
  real number $\lambda \in \R$ if $T(\vec{x}) = \lambda\vec{x}$. Note
  that $\lambda$ is allowed to be $0$. We sometimes conflate the roles
  of $T$ and its matrix $A$, so that we call $\vec{x}$ an eigenvector
  for $A$ and $\lambda$ an eigenvalue for $A$.

  If $\vec{x}$ is an eigenvector of $T$ (or equivalently, of $A$) with
  eigenvalue $\lambda$, which of the following is true? We denote by
  $I_n$ the identity transformation from $\R^n$ to itself.

  \begin{enumerate}[(A)]
  \item $\vec{x}$ must be in the kernel of the linear transformation
    $T + \lambda I_n$
  \item $\vec{x}$ must be in the image of the linear transformation $T
    + \lambda I_n$
  \item $\vec{x}$ must be in the kernel of the linear transformation
    $T - \lambda I_n$
  \item $\vec{x}$ must be in the image of the linear transformation $T
    - \lambda I_n$
  \item $\vec{x}$ must be in the kernel of the linear transformation
    $\lambda T$
  \end{enumerate}

  {\em Answer}: Option (C)

  {\em Explanation}: We are trying to find the vectors $\vec{x}$ such
  that $T(\vec{x}) = \lambda \vec{x}$. This can be rewritten as
  $T\vec{x} = \lambda I_n \vec{x}$ and hence as $(T - \lambda
  I_n)\vec{x} = 0$, so that $\vec{x}$ is in the kernel of $T - \lambda
  I_n$.

  Note that the other options are false for the following reasons:

  \begin{itemize}
  \item Option (A): Consider the case that $T = I_n$, $\vec{x}$ is any
    nonzero vector, and $\lambda = 1$. Then, $T + \lambda I_n = 2I_n$,
    the kernel of which is zero, so $\vec{x}$ is not in the kernel.

    In fact, Option (A) holds true if and only if $\lambda = 0$.
  \item Option (B): Consider the case that $T = 0$, $\vec{x}$ is any
    nonzero vector, and $\lambda = 0$. Then, $T + \lambda I_n = 0$,
    the image of which is zero, so $\vec{x}$ is not in the image.

    In fact, Option (B) holds true if $\lambda \ne 0$. If $\lambda \ne
    0$, then $\vec{x} = (T + \lambda I_n)(\vec{x}/(2\lambda))$. For
    $\lambda = 0$, Option (B) may or may not hold. The example of $T$
    being the zero linear transformation gives a situation where it
    does not hold. Here is an example where it does hold. Consider:

    $$A = \left[\begin{matrix} 0 & 1 \\ 0 & 0 \\\end{matrix}\right]$$

    Then, $\vec{e}_1$ is an eigenvector for $T$ with eigenvalue $0$,
    and it {\em is} in the image of $T + \lambda I_n = T$, since
    $T(\vec{e}_2) = \vec{e}_1$.
  \item Option (D): Consider the case that $T = I_n$, $\vec{x}$ is any
    nonzero vector, and $\lambda = 1$. Then, $T - \lambda I_n = 0$,
    the image of which is zero, so $\vec{x}$ is not in the image.

    Option (D) is often true and often false, but its truth or
    falsehood does not have any direct relation with whether $\lambda$
    is zero or nonzero.
  \item Option (E): Consider the case that $T = I_n$, $\vec{x}$ is any
    nonzero vector, and $\lambda = 1$. Then, $\lambda T = I_n$,
    the kernel of which is zero, so $\vec{x}$ is not in the kernel.

    In fact, Option (E) holds true if and only if $\lambda = 0$.
  \end{itemize}

  {\em Performance review}: 14 out of 24 got this. 4 chose (E), 3
  chose (B), 2 chose (A), 1 chose (D).

  {\em Historical note (last time)}: $17$ out of $24$ got this. $4$ chose (A),
  $2$ chose (B), $1$ chose (E).

\item As above, let $T: \R^n \to \R^n$ be a linear transformation with
  matrix $A$. Use the terminology of eigenvector and eigenvalue from
  the preceding question. Which of the following is a characterization
  of the situation that $A$ is a diagonal matrix?

  \begin{enumerate}[(A)]
  \item Every nonzero vector in $\R^n$ is an eigenvector for $T$.
  \item Every standard basis vector in $\R^n$ is an eigenvector for $T$.
  \item Every vector with at least one zero coordinate in $\R^n$ is an
    eigenvector for $T$.
  \item $T$ has a unique eigenvector (up to scalar multiples, i.e.,
    all eigenvectors of $T$ are scalar multiples of each other).
  \item $T$ has no eigenvector.
  \end{enumerate}

  {\em Answer}: Option (B)

  {\em Explanation}: Suppose the matrix of $T$ has $c_i$ as the
  $i^{th}$ diagonal entry. Then, $T(\vec{e}_i)$ is the $i^{th}$ column
  of the diagonal matrix, which has $0$s everywhere except in the
  diagonal entry, which is $c_i$. Thus, $T(\vec{e}_i) = c_i\vec{e}_i$,
  so that all standard basis vectors are eigenvectors.

  Conversely, if every standard basis vector is an eigenvector, then
  for each $i$, $T(\vec{e}_i) = c_i\vec{e}_i$ for some $c_i$, which
  forces the $i^{th}$ column to have the value $c_i$ in the diagonal
  position and the value $0$ elsewhere. This gives a diagonal matrix.

  {\em Performance review}: 15 out of 24 got this. 5 chose (D), 2
  chose (A), 1 each chose (C) and (E).

  {\em Historical note (last time)}: $13$ out of $24$ got this. $5$ chose (E),
  $3$ chose (D), $2$ chose (C), $1$ chose (A).

\item As above, let $T: \R^n \to \R^n$ be a linear transformation with
  matrix $A$. Use the terminology of eigenvector and eigenvalue from
  the preceding question. Which of the following is a characterization
  of the situation that $A$ is a scalar matrix (i.e., a diagonal matrix
  with all diagonal entries equal)?

  \begin{enumerate}[(A)]
  \item Every nonzero vector in $\R^n$ is an eigenvector for $T$.
  \item Every standard basis vector in $\R^n$ is an eigenvector for $T$.
  \item Every vector with at least one zero coordinate in $\R^n$ is an
    eigenvector for $T$.
  \item $T$ has a unique eigenvector (up to scalar multiples, i.e.,
    all eigenvectors of $T$ are scalar multiples of each other).
  \item $T$ has no eigenvector.
  \end{enumerate}

  {\em Answer}: Option (A)

  {\em Explanation}: If the matrix for $T$ is scalar with scalar value
  $\lambda$, then $T(\vec{x}) = \lambda \vec{x}$ for all vectors
  $\vec{x} \in \R^n$. Thus, all nonzero vectors are eigenvectors.

  To establish the converse, we need to show that if every nonzero
  vector of $T$ is an eigenvector, then all of them have the {\em
    same} eigenvalue. Note that two vectors in the same line have the
  same eigenvalue, so the statement is trivial for $n = 1$.

  For $n \ge 2$, we already know that the matrix of $T$ is diagonal on
  account of the preceding question. We want to show that all diagonal
  entries are equal to each other. Consider the $i^{th}$ and $j^{th}$
  diagonal entries. Let's say these are $c_i$ and $c_j$
  respectively. Then, $T(\vec{e}_i + \vec{e}_j) = T(\vec{e}_i) +
  T(\vec{e}_j)$. This simplifies to $c_i\vec{e}_i + c_j\vec{e}_j$. For
  this to be a multiple of $\vec{e}_i + \vec{e}_j$, we need that $c_i
  = c_j$. Since this is true for every pair of $i$ and $j$, we get
  that all the diagonal entries are equal and that the matrix is a
  scalar matrix.

  {\em Performance review}: 9 out of 24 got this. 8 chose (B), 7 chose
  (D).

  {\em Historical note (last time)}: $8$ out of $24$ got this. $11$ chose (B),
  $2$ each chose (C) and (E), $1$ chose (D).

\item Suppose $A$ is a strictly upper-triangular $n \times n$ matrix,
  i.e., all entries of $A$ that are on or below the main diagonal are
  zero. $T$ is the linear transformation corresponding to $A$. It will
  turn out that the only eigenvalue for $T$ is $0$. What can we say
  about the eigenvectors for $T$ for this eigenvalue?

  \begin{enumerate}[(A)]
  \item All nonzero vectors in $\R^n$ are eigenvectors for $T$ with
    eigenvalue $0$.
  \item All standard basis vectors in $\R^n$ are eigenvectors for $T$
    with eigenvalue $0$.
  \item The vector $\vec{e}_1$ is an eigenvector for $T$ with
    eigenvalue $0$. The information presented is not sufficient to
    determine whether any of the other standard basis vectors is an
    eigenvector.
  \item The vector $\vec{e}_n$ is an eigenvector for $T$ with
    eigenvalue $0$. The information presented is not sufficient to
    determine whether any of the other standard basis vectors is an
    eigenvector.
  \item At least one of the standard basis vectors is an eigenvector
    for $T$ with eigenvalue $0$. However, the information presented is
    not sufficient to say definitively for any particular standard
    basis vector that it is an eigenvector.
  \end{enumerate}

  {\em Answer}: Option (C)

  {\em Explanation}: The first column is the zero column, so the
  vector $\vec{e}_1$ gets mapped to zero. As for the remaining
  standard basis vectors, we do not have enough information to know
  whether they are sent to zero. This is because there are entries
  above the diagonal in those columns, and these are allowed to be
  nonzero.

  For instance, consider:

  $$\left[\begin{matrix} 0 & 1 \\ 0 & 0 \\\end{matrix}\right]$$

  This sends $\vec{e}_1$ to the zero vector and sends $\vec{e}_2$ to
  $\vec{e}_1$.

  {\em Performance review}: 7 out of 24 got this. 5 each chose (A) and
  (D), 3 each chose (B) and (E), 1 left the question blank.

  {\em Historical note (last time)}: $5$ out of $24$ got this. $9$ chose (A),
  $6$ chose (E), $2$ each chose (B) and (D).

\item Suppose $A$ is a strictly upper-triangular $n \times n$ matrix,
  i.e., all entries of $A$ that are on or below the main diagonal are
  zero. $T$ is the linear transformation corresponding to $A$. Which
  of the following is $A$ guaranteed to be? Please see Options (D) and
  (E) before answering.

  \begin{enumerate}[(A)]
  \item Nilpotent
  \item Idempotent
  \item Invertible
  \item All of the above
  \item None of the above
  \end{enumerate}

  {\em Answer}: Option (A)

  {\em Explanation}: The image of $T$ is contained in the span of the
  vectors $\vec{e}_1, \vec{e}_2, \dots, \vec{e}_{n-1}$. The image of
  $T^2$ is in the span of
  $\vec{e}_1,\vec{e}_2,\dots,\vec{e}_{n-2}$. Each time we apply $T$,
  we lose the last basis vector. Thus, $T^n$ is the zero
  transformation, so $A^n$ is the zero matrix, so $A$ is nilpotent.

  For instance, consider:

  $$\left[\begin{matrix} 0 & 1 & 2 \\ 0 & 0 & 3 \\ 0 & 0 & 0 \\\end{matrix}\right]$$

  The image of $T$ is spanned by the column vectors. The first column
  is the zero column, so it is spanned by the other two column vectors:

  $$\left[\begin{matrix} 1 \\ 0 \\ 0 \\\end{matrix}\right], \left[\begin{matrix} 2 \\ 3 \\ 0 \\\end{matrix}\right]$$

  This is the same as the span of $\vec{e}_1$ and $\vec{e}_2$.

  The image of $T^2$ is thus the image of the span of these two
  vectors. This is the span of the first two columns of $A$. The first
  column is zero, so the image of $T^2$ is simply the span of the
  second column vector:

  $$\left[\begin{matrix} 1 \\ 0 \\ 0 \\\end{matrix}\right]$$

  This is the span of $\vec{e}_1$. The image of this under $T$ again
  is the zero space. Thus, the image of $T^3$ is zero, so $A^3 = 0$.

  {\em Performance review}: 8 out of 24 got this. 10 chose (E), 5
  chose (C), 1 chose (D).

  {\em Historical note (last time)}: $7$ out of $24$ got this. $10$ chose (C),
  $5$ chose (E), $1$ chose (D). $1$ left the question blank.

\item Consider the case $n = 2$ and let $T:\R^2 \to \R^2$ be a
  rotation by an angle that is {\em not} an {\em integer} multiple of
  $\pi$. What can we say about the set of eigenvectors and eigenvalues
  for $T$?

  \begin{enumerate}[(A)]
  \item $T$ has no eigenvectors
  \item $T$ has one eigenvector (up to scalar multiples) with
    eigenvalue $1$
  \item $T$ has one eigenvector (up to scalar multiples) and the
    eigenvalue depends on the angle of rotation
  \item $T$ has two linearly independent eigenvectors (so that the set
    of all eigenvectors is obtained as the set of scalar multiples of
    either one of these vectors) with the same eigenvalue
  \item $T$ has two linearly independent eigenvectors (so that the set
    of all eigenvectors is obtained as the set of scalar multiples of
    either one of these vectors) with distinct eigenvalues
  \end{enumerate}

  {\em Answer}: Option (A)

  {\em Explanation}: Every nonzero vector gets rotated, so no nonzero
  vector goes to a scalar multiple of itself. Note that the case that
  the angle of rotation is an integer multiple of $\pi$ differs: in
  that case, the transformation is either the identity or the negative
  identity and hence is scalar, so all nonzero vectors are
  eigenvectors for it.

  {\em Performance review}: 7 out of 24 got this. 6 chose (C), 4 each
  chose (B) and (D), 3 chose (E).

  {\em Historical note (last time)}: $6$ out of $24$ got this. $10$ chose (E),
  $4$ chose (C), $2$ chose (D), $1$ chose (B).
\end{enumerate}
\end{document}
