\documentclass[10pt]{amsart}
\usepackage{fullpage,hyperref,vipul, graphicx}
\title{Review sheet for midterm 1: advanced}
\author{Math 153, Section 55 (Vipul Naik)}

\begin{document}
\maketitle

{\bf To maximize efficiency, please bring a copy (print or readable
electronic) of this review sheet, the basic review sheet, AND the
integration worksheet to the review session.}

\section{Exponential growth and decay}

Combined error-spotting exercises...

\begin{enumerate}
\item The growth of a population is exponential. In one year, the
  population increases by $300\%$. The population increase in two
  years should thus be the square of $300\%$, which is $900\%$.
\item With exponential growth, the time taken for a quantity to
  increase by $10\%$ is five years. Thus, the time taken for the
  quantity to increase by $30\%$ must be fifteen years.
\item With exponential decay, the time taken for a quantity to
  decrease by $10\%$ is five years. Thus, the time taken for the
  quantity to decrease by $30\%$ must be fifteen years.
\item For a function undergoing exponential growth, the ratio of its
  tripling time to its doubling time is $3/2$.
\item To determine whether a quantity has exponential growth with
  respect to time and to find the rate of exponential growth, it
  suffices to observe the quantity at two points in time.
\end{enumerate}

\section{Inverse trigonometric functions}

\subsection{Main points}

Error-spotting exercises ...

\begin{enumerate}
\item The function $f(x):= \arcsin(\sin x)$ coincides with the
  function $x \mapsto  x$ everywhere.
\item The function $f(x) := \arccos(\sin x)$ coincides with the
  function $x \mapsto \sqrt{1 - x^2}$ everywhere.
\item The function $f(x) := \cos(\arcsin x)$ coincides with the
  function $x \mapsto (\pi/2) - x$ everywhere.
\end{enumerate}

\subsection{The formulas for indefinite integration}

Error-spotting exercises ...

\begin{enumerate}
\item We have:

  $$\int_1^{\sqrt{3}} \frac{dx}{1 + x^2} = [\arctan x]_1^{\sqrt{3}} = \arctan \sqrt{3} - 1 = \frac{\pi}{6} -1$$
\item We have:

  $$\int_0^1 \frac{dx}{\sqrt{2 - x^2}} = (1/2)[\arcsin(x/2)]_0^1 = (1/2)(\pi/6) = \pi/12$$

\item Consider:

  $$\int \frac{\cos x \, dx}{3 - 2 \cos^2x} = \int \frac{\cos x \, dx}{3 - 2(1 - \sin^2x)} = \int \frac{\cos x \, dx}{1 + \sin^2x}$$

  Now, with the $u$-substitution $u = \sin x$, we get:

  $$\int \frac{du}{1 + u^2}$$

  This becomes $\arctan u = \arctan (\sin x)$.

\end{enumerate}

\section{Hyperbolic functions}

Error-spotting exercises ...

\begin{enumerate}
\item We have $\cosh(\ln x) - \sinh(\ln x) = \exp(-\ln x) = -\exp(\ln x) = -x$.
\item We have $\cosh(2x) = 2\cosh^2x - 1 = 1 - 2\sinh^2x = \cosh^2x - \sinh^2x$
\end{enumerate}

\section{Integration by parts}

Error-spotting exercises ...

\begin{enumerate}

\item Using integration by parts, knowledge of how to integrate both
  $f$ and $g$ is sufficient to know how to integrate the product
  function $fg$.
\item The $u$-substitution method for integration is the correct
  strategy for integrating the composite of two functions. Integration
  by parts is the correct strategy for integrating the product of
  two functions.
\item We can use integration by parts to show that integrating a
  function $f$ twice is equivalent to integrating the function
  $xf(x)$.
\item We can use integration by parts to show that integrating a
  function $f$ on the interval $[a,b]$ is equivalent to integrating
  $f^{-1}$ on the same interval $[a,b]$.
\item The function $x \mapsto \ln(\sin x)$ can be integrated using the
  $u$-substitution $u = \sin x$ and then performing integration by
  parts (recursive version).
\end{enumerate}

\section{Induction}

No error-spotting exercises.

\section{Quickly}

This section lists things you should be able to do quickly.

\subsection{Our common values}

Preferably remember these (or be capable of computing quickly) to at
least one digit. Generally, you will {\em not} be asked to do any
numerical computations using these. In practice, the main way this is
useful is to figure out whether something is positive or negative. For
instance, is $3\sqrt{2} - 4$ positive? What about $e^2 - 8$? Often,
there are other ways of answering such questions, but remembering the
numerical values is a quick and dirty approach.

\begin{enumerate}
\item Square roots of $2$, $3$, $5$, $6$, $7$, $10$.
\item Natural logarithms of $2$, $3$, $5$, $7$, and $10$.
\item Value of $\pi$, $1/\pi$, $\sqrt{\pi}$, and $\pi^2$.
\item Value of $e$, $1/e$.
\item Some relative logarithms, such as $\log_23$ or
  $\log_2(10)$. Although you don't need these values to a significant
  degree of precision, it is useful to have some idea of their
  magnitude.
\end{enumerate}

\subsection{Adding things up: arithmetic}

You should be able to:

\begin{enumerate}
\item Do quick arithmetic involving fractions.
\item Sense when an expression will simplify to $0$.
\item Compute approximate values for square roots of small numbers,
  $\pi$ and its multiples, etc., so that you are able to figure out,
  for instance, whether $\pi/4$ is smaller or bigger than $1$, or two
  integers such that $\sqrt{39}$ is between them.
\item Know or quickly compute small powers of small positive
  integers. This is particularly important for computing definite
  integrals. For instance, to compute $\int_2^3 (x + 1)^3 \, dx$, you
  need to know/compute $3^4$ and $4^4$.
\end{enumerate}

\subsection{Computational algebra}

You should be able to:

\begin{enumerate}
\item Add, subtract, and multiply polynomials.
\item Factorize quadratics or determine that the quadratic cannot be
  factorized.
\item Factorize a cubic if at least one of its factors is a small and
  easy-to-spot number such as $0$, $\pm 1$, $\pm 2$, $\pm 3$. {\em
  This could be an area for potential improvement for many people.}
\item Factorize an even polynomial of degree four. {\em This could be
  an area for potential improvement for many people.}
\item Do polynomial long division.
\item Solve simple inequalities involving polynomial and rational
  functions once you've obtained them in factored form.
\end{enumerate}

\subsection{Computational trigonometry}

You should be able to:

\begin{enumerate}
\item Determine the values of $\sin$, $\cos$, and $\tan$ at multiples
  of $\pi/2$.
\item Determine the intervals where $\sin$ and $\cos$ are positive and
  negative.
\item Remember the formulas for $\sin(\pi \pm x )$ and $\cos(\pi \pm x)$,
  as well as formulas for $\sin(-x)$ and $\cos(-x)$.
\item Recall the values of $\sin$ and $\cos$ at $\pi/6$, $\pi/4$, and
  $\pi/3$, as well as at the corresponding obtuse angles or other
  larger angles.
\item Reverse lookup for these, for instance, you should quickly
  identify the acute angle whose $\sin$ is $1/2$.
\item Formulas for double angles, half angles: $\sin(2x)$, $\cos(2x)$
  in terms of $\sin$ and $\cos$; also the reverse: $\sin^2x$ and
  $\cos^2x$ in terms of $\cos(2x)$.
\item Remember the formulas for $\sin(A + B)$, $\cos(A + B)$, $\sin(A
  - B)$, and $\cos(A - B)$.
\item Convert between products of $\sin$ and $\cos$ functions and
  their sums: for instance, the identity $2\sin A \cos B = \sin(A + B)
  + \sin (A - B)$. You don't have to remember these identities
  separately since they follow from the identities covered in the
  previous point, but you should be comfortable going back and forth.
\end{enumerate}

\subsection{Computational limits}

You should be able to: size up a limit, determine whether it is of the
form that can be directly evaluated, of the form that we already know
does not exist, or indeterminate.

\subsection{Computational differentiation}

You should be able to:

\begin{enumerate}
\item Differentiate a polynomial (written in expanded form) once or
  twice on sight, without rough work.
\item Differentiate sums of powers of $x$ on sight (without rough
  work).
\item Differentiate rational functions with a little thought.
\item Do multiple differentiations of expressions whose derivative
  cycle is periodic, e.g., $a \sin x + b \cos x$ or $a \exp(-x)$.
\item Do multiple differentiations of expressions whose derivative
  cycle is periodic up to constant factors, e.g. $a \exp(mx + b)$ or $a
  \sin(mx + \varphi)$.
\item Differentiate simple composites without rough work (e.g.,
  $\sin(x^3)$).
\item Differentiate $\ln$, $\exp$, and expressions of the form $f^g$
  and $\log_f(g)$.
\end{enumerate}

\subsection{Computational integration}

You should be able to:

\begin{enumerate}
\item Compute the indefinite integral of a polynomial (written in
  expanded form) on sight without rough work.
\item Compute the definite integral of a polynomial with very few
  terms within manageable limits quickly.
\item Compute the indefinite integral of a sum of power functions
  quickly.
\item Know that the integral of sine or cosine on any quadrant is $\pm
  1$.
\item Compute the integral of $x \mapsto f(mx)$ if you know how to
  integrate $f$. In particular, integrate things like $(a + bx)^m$.
\item Integrate $\sin$, $\cos$, $\sin^2$, $\cos^2$, $\tan^2$,
  $\sec^2$, $\cot^2$, $\csc^2$, $\sin^3$, $\cos^3$, $\tan^3$,
  $\sec^3$, $\cot^3$, $\csc^3$, and other higher powers of the basic
  trigonometric functions.
\item Integrate on sight things such as $x\sin(x^2)$, getting the
  constants right without much effort.
\item {\em By parts}: Integrate $\int xf(x) \, dx$ on sight if it is
  easy to integrate $f$ twice, without having to think much (the
  answer is $x \int f - \int \int f$). Similarly, do $\int x^2f(x) \,
  dx$ if you know how to integrate $f$ twice.
\item Using the previous point, integrate $\int f(\sqrt{x}) \, dx$
  with minimal stress and effort.
\item {\em By parts}: Integrate $\int f(x) \, dx$ on sight if $xf'(x)$
  is easy to integrate, e.g., $\int \ln x \, dx$.
\item Remember the integrals for formats such as $e^x \cos x$ and $e^x
  \sin x$.
\item If there is an easy choice of $f$ such that $f + f' = g$,
  integrate $\int e^xg(x) \, dx = e^xf(x)$ on sight and similarly
  integrate $\int g(\ln x) \, dx = xf(\ln x)$ on siht.
\end{enumerate}
\subsection{Being observant}

You should be able to look at a function and:

\begin{enumerate}
\item Sense if it is odd (even if nobody pointedly asks you whether it
  is).
\item Sense if it is even (even if nobody asks you whether it is).
\item Sense if it is periodic and find the period (even if nobody asks
  you about the period).
\end{enumerate}

\subsection{Graphing}

You should be able to:

\begin{enumerate}
\item Mentally graph a linear function.
\item Mentally graph a power function $x^r$ (see the list of things to
  remember about power functions). Sample cases for $r$: $1/3$, $2/3$,
  $4/3$, $5/3$, $1/2$, $1$, $2$, $3$, $-1$, $-1/3$ $-2/3$.
\item Graph a piecewise linear function with some thought.
\item Mentally graph a quadratic function (very approximately) --
  figure out conditions under which it crosses the axis etc.
\item Graph a cubic function after ascertaining which of the cases for
  the cubic it falls under.
\item Mentally graph $\sin$ and $\cos$, as well as functions of the $A
  \sin(mx)$ and $A\cos(mx)$.
\item Graph a function of the form linear + trigonometric, after doing
  some quick checking on the derivative.
\item Graph the inverse trigonometric functions $\arctan$, $\arcsin$,
  and $\arccos$.
\end{enumerate}

\subsection{Graphing: transformations}

Given the graph of $f$, you should be able to quickly graph the following:

\begin{enumerate}
\item $f(mx)$, $f(mx + b)$: pre-composition with a linear function;
  how does $m < 0$ differ from $m > 0$?
\item $Af(x) + C$: post-composition with a linear function, how does
  $A > 0$ differ from $A < 0$?
\item $f(|x|)$, $|f(x)|$, $f(x^+)$, and $(f(x))^+$: pre- and
  post-composition with absolute value function and positive part
  fnuction.
\item More slowly: $f(1/x)$, $1/f(x)$, $\ln(|f(x)|)$, $f(\ln|x|)$,
  $\exp(f(x))$, and other popular composites.
\end{enumerate}

\end{document}
