\documentclass[10pt]{amsart}

%Packages in use
\usepackage{fullpage, hyperref, vipul, enumerate}

%Title details
\title{Take-home class quiz solutions: due Friday November 15: Image and kernel}
\author{Math 196, Section 57 (Vipul Naik)}
%List of new commands

\begin{document}
\maketitle

\section{Performance review}

27 people took this 18-question quiz. The score distribution was as follows:

\begin{itemize}
\item Score of 2: 3 people
\item Score of 5: 3 people
\item Score of 6: 1 person
\item Score of 7: 4 people
\item Score of 8: 2 people
\item Score of 9: 2 people
\item Score of 10: 3 people
\item Score of 12: 2 people
\item Score of 14: 3 people
\item Score of 15: 2 people
\item Score of 16: 1 person
\item Score of 18: 1 peron
\end{itemize}

The mean score was 9.22.

The question-wise answers and performance review are below:

\begin{enumerate}
\item Option (E): 18 people
\item Option (D): 22 people
\item Option (E): 18 people
\item Option (D): 9 people
\item Option (E): 16 people
\item Option (D): 23 people
\item Option (C): 9 people
\item Option (E): 13 people
\item Option (D): 12 people
\item Option (E): 18 people
\item Option (C): 12 people
\item Option (B): 8 people
\item Option (E): 7 people
\item Option (C): 11 people
\item Option (D): 12 people
\item Option (B): 16 people
\item Option (D): 15 people
\item Option (C): 10 people
\end{enumerate}
\section{Solutions}

{\bf PLEASE FEEL FREE TO DISCUSS {\em ALL} QUESTIONS.}

The purpose of this quiz is to review in greater depth the ideas
behind image and kernel. The goal of the first seven questions is to
review the ideas of injectivity, surjectivity, and bijectivity in the
context of arbitrary functions between sets. The purpose is two-fold:
(i) to give a functions-based approach to justifying, intuitively and
formally, facts about the effect of matrix multiplication on rank, and
(ii) to hint at ways in which linear transformations behave better
than other types of functions.

The corresponding lecture notes are titled {\tt Image and kernel of a
  linear transformation} and the corresponding section of the text is
Section 3.1.

Just as a reminder, a function $f:A \to B$ between sets $A$ and $B$ is
said to be:

\begin{itemize}
\item {\em injective} if for every $b \in B$, there is {\em at most}
  one value of $a$ such that $f(a) = b$. In other words, if we denote
  by $f^{-1}(b)$ the set $\{ a \in A \mid f(a) = b \}$, then
  $|f^{-1}(b)| \le 1$ for all $b \in B$ (here $|f^{-1}(b)|$ denotes
  the size of the set $f^{-1}(b)$).
\item {\em surjective} if for every $b \in B$, there is {\em at least}
  one value of $a$ such that $f(a) = b$. In other words, if we denote
  by $f^{-1}(b)$ the set $\{ a \in A \mid f(a) = b \}$, then
  $|f^{-1}(b)| \ge 1$ for all $b \in B$.
\item {\em bijective} if for every $b \in B$, there is {\em exactly}
  one value of $a$ such that $f(a) = b$. In other words, if we denote
  by $f^{-1}(b)$ the set $\{ a \in A \mid f(a) = b \}$, then
  $|f^{-1}(b)| = 1$ for all $b \in B$.
\end{itemize}

\begin{enumerate}
\item Suppose $g:A \to B$ and $f:B \to C$ are functions. The composite
  $f \circ g$ is a function from $A$ to $C$. What can we say the
  relationship between the injectivity of $f \circ g$, the injectivity
  of $f$, and the injectivity of $g$?

  \begin{enumerate}[(A)]
  \item $f \circ g$ is injective if and only if $f$ and $g$ are both injective.
  \item If $f$ and $g$ are both injective, then $f \circ g$ is
    injective. However, $f \circ g$ being injective does not imply
    anything about the injectivity of either $f$ or $g$.
  \item If $f$ and $g$ are both injective, then $f \circ g$ is
    injective. If $f \circ g$ is injective, then at least one of $f$
    and $g$ is injective, but we cannot conclusively say for any
    specific one of the two that it must be injective.
  \item If $f$ and $g$ are both injective, then $f \circ g$ is
    injective. If $f \circ g$ is injective, then $f$ is injective, but
    we do not have enough information to deduce whether $g$ is
    injective.
  \item If $f$ and $g$ are both injective, then $f \circ g$ is
    injective. If $f \circ g$ is injective, then $g$ is injective, but
    we do not have enough information to deduce whether $f$ is
    injective.
  \end{enumerate}

  {\em Answer}: Option (E)

  {\em Explanation}: See the lecture notes for more details (note that
  the roles of $f$ and $g$ are reversed in the lecture notes). The
  hard part is the direction from $f \circ g$ to $g$. To see this,
  note that if $g$ has a collision $g(a_1) = g(a_2)$ (i.e., is
  non-injective) then that collision continues for $f \circ g$, i.e,
  we still have $f(g(a_1)) = f(g(a_2))$.

  {\em Performance review}: 18 out of 27 got this. 6 chose (D), 1 each
  chose (A), (B), and (C).

  {\em Historical note (last time)}: $24$ out of $26$ got this. $1$ each chose
  (A) and (D).

\item Suppose $g:A \to B$ and $f:B \to C$ are functions. The composite
  $f \circ g$ is a function from $A$ to $C$. What can we say the
  relationship between the surjectivity of $f \circ g$, the surjectivity
  of $f$, and the surjectivity of $g$?

  \begin{enumerate}[(A)]
  \item $f \circ g$ is surjective if and only if $f$ and $g$ are both
    surjective.
  \item If $f$ and $g$ are both surjective, then $f \circ g$ is
    surjective. However, $f \circ g$ being surjective does not imply
    anything about the surjectivity of either $f$ or $g$.
  \item If $f$ and $g$ are both surjective, then $f \circ g$ is
    surjective. If $f \circ g$ is surjective, then at least one of $f$
    and $g$ is surjective, but we cannot conclusively say for any
    specific one of the two that it must be surjective.
  \item If $f$ and $g$ are both surjective, then $f \circ g$ is
    surjective. If $f \circ g$ is surjective, then $f$ is surjective,
    but we do not have enough information to deduce whether $g$ is
    surjective.
  \item If $f$ and $g$ are both surjective, then $f \circ g$ is
    surjective. If $f \circ g$ is surjective, then $g$ is surjective,
    but we do not have enough information to deduce whether $f$ is surjective.
  \end{enumerate}

  {\em Answer}: Option (D)

  {\em Explanation}: See the lecture notes for more details (note that
  the roles of $f$ and $g$ are reversed in the lecture notes).

  {\em Performance review}: 22 out of 27 got this. 3 chose (A), 1 each
  chose (B) and (E).

  {\em Historical note (last time)}: $23$ out of $26$ got this. $2$ chose (E),
  $1$ chose (A).

\item Suppose $g:A \to B$ and $f:B \to C$ are functions. The composite
  $f \circ g$ is a function from $A$ to $C$. Suppose $f \circ g$ is
  bijective. What can we say about $f$ and $g$ individually?

  \begin{enumerate}[(A)]
  \item Both $f$ and $g$ must be bijective.
  \item Both $f$ and $g$ must be injective, but neither of them need be surjective.
  \item Both $f$ and $g$ must be surjective, but neither of them need be injective.
  \item $f$ must be injective but need not be surjective. $g$ must be
    surjective but need not be injective.
  \item $f$ must be surjective but need not be injective. $g$ must be
    injective but need not be surjective.
  \end{enumerate}

  {\em Answer}: Option (E)

  {\em Explanation}: The composite $f \circ g$ is bijective, so it is
  both injective and surjective. The results of the previous two
  questions now take care of things.

  {\em Performance review}: 18 out of 27 got this. 4 chose (A), 2 each
  chose (C) and (D), 1 left the question blank.

  {\em Historical note (last time)}: $23$ out of $26$ got this. $2$ chose (D),
  $1$ chose (A).

\item $g:A \to B$ and $f:B \to C$ are functions. The composite $f
  \circ g$ is a function from $A$ to $C$. Suppose both $f$ and $g$ are
  surjective. Further, suppose that for every $b \in B$, $g^{-1}(b)$
  has size $m$ (for a fixed positive integer $m$) and for every $c \in
  C$, $f^{-1}(c)$ has size $n$ (for a fixed positive integer
  $n$). Then, what can we say about the sizes of the fibers (i.e., the
  inverse images of points in $C$) under the composite $f \circ g$?

  \begin{enumerate}[(A)]
  \item The size is $\min \{ m,n \}$
  \item The size is $\max \{ m,n \}$
  \item The size is $m + n$
  \item The size is $mn$
  \item The size is $m^n$
  \end{enumerate}

  {\em Answer}: Option (D)

  {\em Explanation}: For any $c \in C$, $(f \circ g)^{-1}(c) =
  g^{-1}(f^{-1}(c))$ is the union, for all $b \in f^{-1}(c)$, of the
  sets $g^{-1}(b)$. Each of these sets has size $m$, and there is a
  total of $n$ such sets, so we get a total of $mn$ elements. {\em
    Remember, as you learned in kindergarten, that multiplication is
    repeated addition}.

  {\em Performance review}: 9 out of 27 got this. 7 chose (A), 6 chose
  (B), 3 chose (E), 1 chose (C), 1 left the question blank.

  {\em Historical note (last time)}: $21$ out of $26$ got this. $4$ chose (A),
  $1$ chose (B).

\item {\bf PLEASE READ THIS VERY CAREFULLY AND CONSIDER A WIDE VARIETY
  OF POLYNOMIAL EXAMPLES}: Suppose $f$ is a polynomial function of
  degree $n > 2$ from $\R$ to $\R$. What can we say about the fibers
  of $f$, i.e., the sets of the form $f^{-1}(x)$, $x \in \R$?

  {\em Hint}: At the one extreme, consider a polynomial of the form
  $x^n$. Consider the sizes of the fibers $f^{-1}(0)$ and $f^{-1}(x)$
  for a positive value of $x$ (the fiber size for the latter will
  depend on whether $n$ is even or odd). Alternatively, consider a
  polynomial of the form $(x - 1)(x - 2) \dots (x - n)$. Consider the
  size of the fiber $f^{-1}(0)$.

  \begin{enumerate}[(A)]
  \item Every fiber has size $n$.
  \item The minimum of the sizes of fibers is exactly $n$, but every
    fiber need not have size $n$.
  \item The maximum of the sizes of fibers is exactly $n$, but every
    fiber need not have size $n$.
  \item The minimum of the sizes of fibers is at least $n$, but need
    not be exactly $n$.
  \item The maximum of the sizes of fibers is at most $n$, but need not
    be exactly $n$.
  \end{enumerate}

  {\em Answer}: Option (E)

  {\em Explanation}: Suppose we are trying to calculate the size of
  the fiber $f^{-1}(x)$ for a particular value of $x$. This is
  equivalent to solving the equation $f(t) = x$ in the variable
  $t$. This is a polynomial equation of degree $n$, so it has at most
  $n$ roots. Thus, the size of each fiber is at most $n$. Thus, the
  maximum of the sizes of the fibers is at most $n$.

  For $n = 2$, the maximum of the fiber sizes is always $2$. However,
  for $n \ge 3$, there are examples where the maximum of the sizes of
  the fibers is less than $n$. Specifically, consider the example of
  $x^n$. The maximum of the fiber sizes here is $1$ if $n$ is odd and
  $2$ if $n$ is even. In both cases, it is less than $n$ for $n \ge 3$.

  {\em Performance review}: 16 out of 27 got this. 6 chose (C), 3
  chose (B), 1 chose (A), 1 left the question blank.

  {\em Historical note (last time)}: $1$ out of $26$ got this (!!!). $16$ chose
  (C), $5$ chose (B), $3$ chose (D), $1$ chose (A).

\item Suppose $f$ is a continuous injective function from $\R$ to
  $\R$. What can we say about the nature of $f$?

  \begin{enumerate}[(A)]
  \item $f$ must be an increasing function on all of $\R$.
  \item $f$ must be a decreasing function on all of $\R$.
  \item $f$ must be a constant function on all of $\R$.
  \item $f$ must be either an increasing function on all of $\R$ or a
    decreasing function on all of $\R$, but the information presented
    is insufficient to decide which case occurs.
  \item $f$ must be either an increasing function or a decreasing
    function or a constant function on all of $\R$, but the
    information presented is insufficient for deciding anything
    stronger.
  \end{enumerate}

  {\em Answer}: Option (D)

  {\em Explanation}: This is easy to see pictorially, though a
  rigorous proof would invoke the intermediate value theorem and the
  extreme value theorem.

  {\em Performance review}: 23 out of 27 got this. 4 chose (E).

  {\em Historical note (last time)}: $21$ out of $26$ got this. $5$ chose (E).

\item {\bf PLEASE READ THIS CAREFULLY, MAKE CASES, AND CHECK YOUR
  REASONING}: Suppose $f$, $g$, and $h$ are continuous bijective
  functions from $\R$ to $\R$. What can we say about the functions $f
  + g$, $f + h$, and $g + h$?

  {\em Hint}: Based on the preceding question, you know something
  about the nature of $f$, $g$, and $h$ individually as functions, but
  there is some degree of ambiguity in your knowledge. Make cases
  based on the possibilities and see what you can deduce in the best
  and worst case.

  \begin{enumerate}[(A)]
  \item They are all continuous bijective functions from $\R$ to $\R$.
  \item At least two of them are continuous bijective functions from
    $\R$ to $\R$. However, we cannot say more.
  \item At least one of them is a continuous bijective function from
    $\R$ to $\R$. However, we cannot say more.
  \item Either all three sums are continuous bijective functions from
    $\R$ to $\R$, or none is.
  \item It is possible that none of the sums is a continuous
    bijective functions from $\R$ to $\R$; it is
    also possible that one, two, or all the sums are continuous
    bijective functions from $\R$ to $\R$.
  \end{enumerate}

  {\em Answer}: Option (C)

  {\em Explanation}: Since $f$, $g$, and $h$ are all continuous
  bijective functions $\R \to \R$, each one of them is either
  increasing or decreasing. Further, the functions that are increasing
  must have a limit of $-\infty$ at $-\infty$ and a limit of $\infty$
  at $\infty$, whereas the functions that are decreasing must have a
  limit of $\infty$ at $-\infty$ and a limit of $-\infty$ at
  $\infty$. Thus:

  \begin{itemize}
  \item A sum of two continuous increasing surjective functions is
    also a continuous increasing surjective function, and hence is
    bijective: To see this, use the fact that the limit of the sum is
    the sum of the limits to deduce that for the sum, the limit at
    $-\infty$ is $-\infty$ and the limit at $\infty$ is $\infty$, so
    that the function must be surjective.
  \item A sum of two continuous decreasing surjective functions is
    also a continuous decreasing surjcetive function, hence is
    bijective. To see this, use the fact that the limit of the sum is
    the sum of the limits to deduce that for the sum, the limit at
    $-\infty$ is $\infty$ and the limit at $\infty$ is $-\infty$, so
    that the function must be surjective.
  \end{itemize}

  We consider various cases:

  \begin{itemize}
  \item If all three functions are increasing, so are all the pairwise
    sums, and hence, all the sums $f + g$, $f + h$, and $g + h$ are
    bijective.
  \item If all three functions are decreasing, so are all the pairwise
    sums, and hence, all the sums $f + g$, $f + h$, and $g + h$ are
    decreasing.
  \item If two of the functions are increasing and the third function
    is decreasing, then we know for certain that the sum of the two
    increasing functions is bijective. But the sum of either of the
    increasing functions with the decreasing function may be
    increasing, decreasing, or neither. For instance, if $f(x) = g(x)
    = x$ and $h(x) = -x$, then $f + h$ and $g + h$ are both the zero
    function, which is neither increasing nor decreasing, and hence
    not one-to-one.
  \item If two of the functions are decreasing and the third function
    is increasing, then we know for certain that the sum of the two
    dcereasing functions is bijective. We cannot say anything for sure
    about the other two sums, for the same reasons as in the previous
    case (specifically, we can just use the negatives of the functions
    in the preceding example).
  \end{itemize}

  It's clear from all these that (C) is the
  right option.

  {\em Performance review}: 9 out of 27 got this. 10 chose (E), 4
  chose (B), 3 chose (A), 1 chose (D).

  {\em Historical note (last time)}: $1$ out of $26$ got this. $15$ chose
  (E), $8$ chose (A), $2$ chose (D).

  \vspace{0.5in}

  The questions that follow tripped up students quite a bit last time,
  so I urge you to proceed with caution. You can do each of these
  questions in either of two ways:

  \begin{itemize}
  \item Using abstract, general reasoning.
  \item Constructing concrete examples.
  \end{itemize}

  While the former approach is one you should eventually be able to
  embrace without trepidation, feel free to rely on the latter
  approach for now. For this, consider matrices describing the linear
  transformations and use matrix multiplication to compute the
  composite where needed. Compute the kernel, image, and rank using
  the methods known to you. Take matrices such as those arising from
  finite state automata (as described in the ``linear transformations
  and finite state automata'' quiz) or their generalizations to
  rectangular matrices.

  For instance, you might try taking a matrix such as
  $\left[ \begin{matrix} 1 & 0 & 0 & 0 & 0 \\ 0 & 1 & 0 & 0 & 0\\ 0 &
      0 & 1 & 0 & 0 \\ 0 & 0 & 0 & 0 & 0 \\\end{matrix}\right]$. This
  describes a linear transformation $\R^5 \to \R^4$ and has rank
  three. The dimension of the kernel (inside $\R^5$) is 2 (explicitly,
  the kernel is precisely the set of vectors in $\R^5$ whose first
  three coordinates are zero) and the dimension of the image (inside
  $\R^4$) is 3 (explicitly, the image is precisely the set of vectors
  in $\R^4$ whose fourth coordinate is $0$).

\item {\em This is the analogue for linear transformations of Question
  1}: Suppose $m,n,p$ are positive integers. Suppose $A$ is a $m
  \times n$ matrix and $B$ is a $n \times p$ matrix. The product $AB$
  is a $m \times p$ matrix. Denote by $T_A$, $T_B$, and $T_{AB}$
  respectively the linear transformations corresponding to $A$, $B$,
  and $AB$. We have $T_A:\R^n \to \R^m$, $T_B: \R^p \to \R^n$, and
  $T_{AB}: \R^p \to \R^m$. Note that $T_{AB} = T_A \circ T_B$.

  Recall that a matrix has full column rank if and only if the
  corresponding linear transformation is injective.

  Which of the following describes correctly the relationship between
  $A$ having full column rank (i.e., rank $n$), $B$ having full column
  rank (i.e., rank $p$), and $AB$ having full column rank (i.e., rank
  $p$)?

  \begin{enumerate}[(A)]
  \item $AB$ has full column rank (i.e., rank $p$) if and only if $A$
    and $B$ both have full column rank (ranks $n$ and $p$
    respectively).
  \item If $A$ and $B$ both have full column rank, then $AB$ has full
    column rank. However, $AB$ having full column rank does not imply
    anything (separately or jointly) regarding whether $A$ or $B$ has
    full column rank.
  \item If $A$ and $B$ both have full column rank, then $AB$ has full
    column rank. If $AB$ has full column rank, then at least one of
    $A$ and $B$ has full column rank, but we cannot definitively say
    for any particular one of $A$ and $B$ that it must have full column
    rank.
  \item If $A$ and $B$ both have full column rank, then $AB$ has full
    column rank. $AB$ having full column rank implies that $A$ has
    full column rank, but it does not tell us for sure that $B$ has
    full column rank.
  \item If $A$ and $B$ both have full column rank, then $AB$ has full
    column rank. $AB$ having full column rank implies that $B$ has
    full column rank, but it does not tell us for sure that $A$ has
    full column rank.
  \end{enumerate}

  {\em Answer}: Option (E)

  {\em Explanation}: This is a special case of Question 1 (note that
  the letters do not match). Essentially, $T_A$ plays the role of $f$
  and $T_B$ plays the role of $g$.

  An alternative way of thinking of this is that the rank of a product
  is less than or equal to the rank of each of the matrices being
  multiplied. If the rank of $AB$ is $p$ (full column rank), then that
  means that the rank of $B$ is at least $p$. Since the number of
  columns of $B$ equals $p$, this forces $B$ to have full column rank
  $p$.

  Note that $A$ need not have full column rank. For instance, consider:

  $$A = \left[ \begin{matrix} 1 & 0 \\\end{matrix}\right], B = \left[\begin{matrix} 1 \\ 0 \\\end{matrix}\right]$$

  {\em Performance review}: 13 out of 27 got this. 6 chose (D), 3 each
  chose (B) and (C), 2 chose (A).

\item {\em This is the analogue for linear transformations of Question
  2}: Suppose $m,n,p$ are positive integers. Suppose $A$ is a $m
  \times n$ matrix and $B$ is a $n \times p$ matrix. The product $AB$
  is a $m \times p$ matrix. Denote by $T_A$, $T_B$, and $T_{AB}$
  respectively the linear transformations corresponding to $A$, $B$,
  and $AB$. We have $T_A:\R^n \to \R^m$, $T_B: \R^p \to \R^n$, and
  $T_{AB}: \R^p \to \R^m$. Note that $T_{AB} = T_A \circ T_B$.

  Recall that a matrix has full row rank if and only if the
  corresponding linear transformation is surjective.

  Which of the following describes correctly the relationship between
  $A$ having full row rank (i.e., rank $m$), $B$ having full row rank
  (i.e., rank $n$), and $AB$ having full row rank (i.e., rank $m$)?

  \begin{enumerate}[(A)]
  \item $AB$ has full row rank if and only if $A$ and $B$ both have
    full row rank.
  \item If $A$ and $B$ both have full row rank, then $AB$ has full row
    rank. However, $AB$ having full row rank does not imply anything
    (separately or jointly) regarding whether $A$ or $B$ has full row
    rank.
  \item If $A$ and $B$ both have full row rank, then $AB$ has full
    row rank. If $AB$ has full row rank, then at least one of
    $A$ and $B$ has full row rank, but we cannot definitively say
    for any particular one of $A$ and $B$ that it must have full row
    rank.
  \item If $A$ and $B$ both have full row rank, then $AB$ has full
    row rank. $AB$ having full row rank implies that $A$ has
    full row rank, but it does not tell us for sure that $B$ has
    full row rank.
  \item If $A$ and $B$ both have full row rank, then $AB$ has full
    row rank. $AB$ having full row rank implies that $B$ has
    full row rank, but it does not tell us for sure that $A$ has
    full row rank.
  \end{enumerate}

  {\em Answer}: Option (D)

  {\em Explanation}: This follows from Question 2. We can also think
  of it in terms of the rank of a product being less than or equal to
  the ranks of the individual matrices. This forces the rank of $A$ to
  be at least $m$, and therefore exactly $m$.

  Also, for an example of a situation where $B$ does not have full row
  rank, we can use the same example as in the preceding question.

  {\em Performance review}: 12 out of 27 got this. 6 chose (E), 4
  chose (A), 3 chose (B), 2 chose (C).
\item {\em This is the analogue for linear transformations of Question
  3}: Suppose $m$ and $n$ are positive integers. Suppose $A$ is a $m
  \times n$ matrix and $B$ is a $n \times m$ matrix. The product $AB$
  is a $m \times m$ matrix. The corresponding linear transformations
  are $T_A: \R^n \to \R^m$, $T_B: \R^m \to \R^n$, and $T_{AB}: \R^m
  \to \R^m$. 

  Suppose the square matrix $AB$ has full rank $m$. What can we deduce
  about the ranks of $A$ and $B$?

  \begin{enumerate}[(A)]
  \item Both $A$ and $B$ have full row rank, {\em and} both $A$ and
    $B$ have full column rank.
  \item Both $A$ and $B$ have full column rank, but neither of them
    need have full row rank.
  \item Both $A$ and $B$ have full row rank, but neither of them need
    have full column rank.
  \item $A$ must have full column rank but need not have full row
    rank. $B$ must have full row rank but need not have full column
    rank.
  \item $A$ must have full row rank but need not have full column
    rank. $B$ must have full column rank but need not have full row
    rank.
  \end{enumerate}

  {\em Answer}: Option (E)

  {\em Explanation}: Follows from Question 3. We can use the same
  example as for the preceding two questions.

  Also note that in this case, we must have $m \le n$, and therefore,
  the rank of both $A$ and $B$, since it's $\le \min \{ m, n \}$ but
  also $\ge m$, must equal $m$. This means full row rank in the case
  of $A$, and full column rank in the case of $B$.
  
  {\em Performance review}: 18 out of 27 got this. 5 chose (D), 2
  chose (A), 1 each chose (B) and (C).

  \vspace{0.3in}

  For the coming questions, we will denote vector spaces by letters
  such as $U$, $V$, and $W$. You can, however, consider them to be
  finite-dimensional vector spaces of the form $\R^n$. However, you
  should take care not to use a letter for the dimension of a vector
  space if the letter is already in use elsewhere in the
  question. Also, you should take care to use different letters for
  the dimensions of different vector spaces, unless it is given to you
  that the vector spaces have the same dimension. The results also
  hold for infinite-dimensional vector spaces, but you can work on all
  the problems assuming you are working in the finite-dimensional
  setting.

\item {\em This is an analogue for linear transformations of Question
  4}: Suppose $T_1: U \to V$ and $T_2:V \to W$ are linear
  transformations. The composite $T_2 \circ T_1$ is also a linear
  transformation, this time from $U$ to $W$. Suppose the kernel of
  $T_1$ has dimension $m$ and the kernel of $T_2$ has dimension
  $n$. Suppose both $T_1$ and $T_2$ are surjective. What can you say
  about the dimension of the kernel of $T_2 \circ T_1$?

  {\em Please note this carefully}: Although this question is
  analogous to Question 4, the correct answer options differ for the
  two questions. Here is an intuitive explanation for the relationship
  between the questions. Question 4 asked about the {\em sizes} of the
  fibers. This question asks about the dimensions of the kernels. The
  fibers do correspond to the kernels. But the relationship between
  dimension and size is of a {\em logarithmic nature}. What we mean is
  that the dimension can be thought of as the logarithm of the
  size. This isn't literally true, because the size is infinite. But
  metaphorically, it makes sense, because, for instance, the dimension
  of $\R^p$ is the exponent $p$, and that comports with the laws of
  logarithms (similar to how the $\log_2(2^p) = p$).

  \begin{enumerate}[(A)]
  \item The dimension is $\min \{ m,n \}$.
  \item The dimension is $\max \{ m,n \}$.
  \item The dimension is $m + n$.
  \item The dimension is $mn$.
  \item The dimension is $m^n$.
  \end{enumerate}

  {\em Answer}: Option (C)

  {\em Explanation}: See the lecture notes for more.

  {\em Performance review}: 12 out of 27 got this. 5 chose (D), 4
  chose (A), 3 chose (B), 2 chose (E), 1 left the question blank.

  {\em Historical note (last time)}: $3$ out of $26$ got this. $13$ chose (B),
  $7$ chose (A), $3$ chose (D).

\item Suppose $T_1: U \to V$ and $T_2:V \to W$ are linear
  transformations. The composite $T_2 \circ T_1$ is also a linear
  transformation, this time from $U$ to $W$. Suppose the kernel of
  $T_1$ has dimension $m$ and the kernel of $T_2$ has dimension
  $n$. However, unlike the preceding question, we are not given any
  information about the surjectivity of either $T_1$ or $T_2$. The
  answer to the preceding question gives an (inclusive) {\em upper}
  bound on the dimension of the kernel of $T_2 \circ T_1$. Which of
  the following is the best {\em lower} bound we can manage in
  general?

  \begin{enumerate}[(A)]
  \item $|m - n|$
  \item $m$
  \item $n$
  \item $m + n$
  \end{enumerate}

  {\em Answer}: Option (B)

  {\em Explanation}: The kernel of $T_2 \circ T_1$ contains the kernel
  of $T_1$, so $m$ is a lower bound on the dimension.

  {\em Performance review}: 8 out of 27 got this. 11 chose (A), 4
  chose (C), 3 chose (D), 1 left the question blank.

  {\em Historical note (last time)}: $11$ out of $26$ got this. $10$ chose (A),
  $4$ chose (C), $1$ chose (D).

\item Suppose $T_1,T_2:U \to V$ are linear transformations. Which of
  the following is true? Please see Options (D) and (E) before
  answering and select the single option that best reflects your view.

  \begin{enumerate}[(A)]
  \item If both $T_1$ and $T_2$ are injective, then $T_1 + T_2$ is injective.
  \item If both $T_1$ and $T_2$ are surjective, then $T_1 + T_2$ is surjective.
  \item If both $T_1$ and $T_2$ are bijective, then $T_1 + T_2$ is bijective.
  \item All of the above
  \item None of the above
  \end{enumerate}

  {\em Answer}: Option (E)

  {\em Explanation}: We can get counterexamples in one dimension:
  consider the situation where $T_1$ has matrix $[1]$ and $T_2$ has
  matrix $[-1]$. Then, both $T_1$ and $T_2$ are bijective (hence also
  injective and surjective) but the sum $T_1 + T_2$, which is the zero
  map, is neither injective nor surjective.

  {\em Performance review}: 7 out of 27 got this. 13 chose (D), 4
  chose (B), 2 chose (C), 1 left the question blank.

  {\em Historical note (last time)}: $6$ out of $26$ got this. $19$ chose (D),
  $1$ chose (C).

\item Suppose $T_1, T_2: U \to V$ are linear transformations. Which of
  the following best describes the relation between the kernels of
  $T_1$, $T_2$, and $T_1 + T_2$?

  \begin{enumerate}[(A)]
  \item The kernel of $T_1 + T_2$ equals the intersection of the
    kernel of $T_1$ and the kernel of $T_2$.
  \item The kernel of $T_1 + T_2$ is contained inside the intersection
    of the kernel of $T_1$ and the kernel of $T_2$, but need not be
    equal to the intersection.
  \item The kernel of $T_1 + T_2$ contains the intersection of the
    kernel of $T_1$ and the kernel of $T_2$, but need not be equal to
    the intersection.
  \item The kernel of $T_1 + T_2$ is contained inside the sum of the
    kernel of $T_1$ and the kernel of $T_2$, but need not be equal to
    the sum.
  \item The kernel of $T_1 + T_2$ contains the sum of the kernel of
    $T_1$ and the kernel of $T_2$, but need not be equal to the sum.
  \end{enumerate}

  {\em Answer}: Option (C)

  {\em Explanation}: We will show this in steps:

  \begin{itemize}
  \item Suppose $\vec{u}$ is in the intersection of the kernel of
    $T_1$ and the kernel of $T_2$. Then, $\vec{u}$ is in the kernel of
    $T_1 + T_2$: This is easy to see: $(T_1 + T_2)(\vec{u}) =
    T_1(\vec{u}) + T_2(\vec{u}) = 0 + 0 = 0$.
  \item It is possible to have a situation where the kernel of $T_1 +
    T_2$ does not contain the sum of the kernels of $T_1$ and
    $T_2$. For instance, consider the case that $T_1$ has matrix $[0]$
    and $T_2$ has matrix $[1]$.
  \item It is possible to have a situation where the kernel of $T_1 +
    T_2$ is not even contained inside the sum of the kernels of $T_1$
    and $T_2$, let alone the intersection: Consider $T_1$ to be the
    linear transformation with matrix $[1]$ and $T_2$ to be the linear
    transformation with matrix $[-1]$. Both have zero kernels, so the
    sum of the kernels is also zero. But the sum $T_1 + T_2$ is a
    linear transformation with matrix $[0]$, so its kernel is all of $\R$.
  \end{itemize}

  {\em Performance review}: 11 out of 27 got this. 6 chose (A), 4 each
  chose (B) and (D), 1 chose (E), 1 left the question blank.

  {\em Historical note (last time)}: $3$ out of $26$ got this. $10$ chose (B),
  $6$ each chose (A) and (D), $1$ chose (E).

\item Suppose $T_1, T_2: U \to V$ are linear transformations. Which of
  the following best describes the relation between the images of
  $T_1$, $T_2$, and $T_1 + T_2$?

  \begin{enumerate}[(A)]
  \item The image of $T_1 + T_2$ equals the intersection of the
    image of $T_1$ and the image of $T_2$.
  \item The image of $T_1 + T_2$ is contained inside the intersection
    of the image of $T_1$ and the image of $T_2$, but need not be
    equal to the intersection.
  \item The image of $T_1 + T_2$ contains the intersection of the
    image of $T_1$ and the image of $T_2$, but need not be equal to
    the intersection.
  \item The image of $T_1 + T_2$ is contained inside the sum of the
    image of $T_1$ and the image of $T_2$, but need not be equal to
    the sum.
  \item The image of $T_1 + T_2$ contains the sum of the image of
    $T_1$ and the image of $T_2$, but need not be equal to the sum.
  \end{enumerate}

  {\em Answer}: Option (D)

  {\em Explanation}: We show the following:

  \begin{itemize}
  \item Any vector in the image of $T_1 + T_2$ can be expressed as the
    sum of a vector in the image of $T_1$ and a vector in the image of
    $T_2$: Suppose $\vec{v}$ is in the image of $T_1 + T_2$. Thus,
    $\vec{v} = (T_1 + T_2)(\vec{u})$ which simplifies to $T_1(\vec{u})
    + T_2(\vec{u})$, thus it is in the sum of the images.
  \item The image of $T_1 + T_2$ need not contain the intersection of
    the images of $T_1$ and $T_2$: We can again use the example of
    linear transformations with matrices $[1]$ and $[-1]$
    respectively.
  \item The image of $T_1 + T_2$ need not be contained in the
    intersection of the images of $T_1$ and $T_2$: We can again use
    the example of linear transformations with matrices $[1]$ and
    $[0]$. 
  \end{itemize}

  {\em Performance review}: 12 out of 27 got this. 8 chose (B), 4
  chose (C), 2 chose (E), 1 left the question blank.

  {\em Historical note (last time)}: $11$ out of $26$ got this. $11$ chose (B),
  $2$ chose (C), $1$ each chose (A) and (E).

\item Suppose $T$ is a linear transformation from a vector space $V$
  to itself. Note that $V$ may be an infinite-dimensional space, such
  as $C^\infty(\R)$ (with $T$ being differentiation), but for
  convenience, you can imagine $V$ to be finite-dimensional (we will
  not reference the dimension of $V$ in this question,
  however). Suppose the kernel of $T$ has dimension $n$. What can you
  say from this information about the dimension of the kernel of $T^r$
  for a positive integer $r$?

  \begin{enumerate}[(A)]
  \item It is at least $n$ and at most $n + r$.
  \item It is at least $n$ and at most $nr$.
  \item It is at least $n + r$ and at most $nr$.
  \item It is at least $n + r$ and at most $n^r$.
  \end{enumerate}

  {\em Answer}: Option (B)

  {\em Explanation}: We know the kernel of a composite contains the
  kernel of the very first operation, so the dimension is at least
  $n$. But it could be bigger. Recall that the dimensions of the
  kernels could at worst add up, so the worst case scenario (which
  occurs if each time the kernel is also contained in the image) is
  that the total dimension is $nr$.

  {\em Performance review}: 16 out of 27 got this. 9 chose (A), 1
  chose (C), 1 left the question blank.
 
  {\em Historical note (last time)}: $13$ out of $26$ got this. $7$ chose (A),
  $5$ chose (C), $1$ chose (D).

  \vspace{0.4in}

  The next few questions deal with the relationship between the rows
  and columns of the matrix on the one hand, and the image and kernel
  of the linear transformation on the other hand.

\item Suppose $A$ is a $n \times m$ matrix and $T_A: \R^m \to \R^n$ is
  the corresponding linear transformation. Which of the following
  correctly describes the relationship between the rows and columns of
  $A$ and the image and kernel of $T_A$?

  \begin{enumerate}[(A)]
  \item The kernel of $T_A$ is precisely the subspace of $\R^m$
    spanned by the rows of $A$. The image of $T_A$ is precisely the
    subspace of $\R^n$ spanned by the columns of $A$.
  \item The kernel of $T_A$ is precisely the subspace of $\R^m$
    spanned by the columns of $A$. The image of $T_A$ is precisely the
    subspace of $\R^n$ spanned by the rows of $A$.
  \item The kernel of $T_A$ is precisely the subspace of $\R^m$
    comprising the vectors that are {\em orthogonal} to the rows of
    $A$. The image of $T_A$ is precisely the subspace of $\R^n$
    comprising the vectors that are {\em orthogonal} to the columns of
    $A$.
  \item The kernel of $T_A$ is precisely the subspace of $\R^m$
    comprising the vectors that are {\em orthogonal} to the rows of
    $A$. The image of $T_A$ is the subspace of $\R^n$ spanned by the
    columns of $A$.
  \item The kernel of $T_A$ is precisely the subspace of $\R^m$
    spanned by the rows of $A$. The image of $T_A$ is precisely the
    subspace of $\R^n$ comprising the vectors that are {\em
      orthogonal} to the columns of $A$.
  \end{enumerate}

  {\em Answer}: Option (D)
  
  {\em Explanation}: The kernel is the set of vectors $\vec{x} \in
  \R^m$ such that $A\vec{x} = \vec{0}$. The entries of $A\vec{x}$ are
  the dot products of the rows of $A$ with $\vec{x}$. Therefore, a
  particular entry is zero if and only if the corresponding dot
  product is zero, i.e., $\vec{x}$ is orthogonal to the corresponding
  row of $A$. Thus, all the entries of the output vector are zero iff
  $\vec{x}$ is orthogonal to all rows of $A$ (there were also similar
  questions in the preceding quiz).

  The part of the statement about the image is a standard fact about
  the image that you have seen in lecture.

  {\em Performance review}: 15 out of 27 got this. 5 chose (B), 4
  chose (C), 2 chose (A), 1 left the question blank.

\item Suppose $A$ and $B$ are $n \times m$ matrices, $T_A:\R^m \to
  \R^n$ is the linear transformation corresponding to $A$, and
  $T_B:\R^m \to \R^n$ is the linear transformation corresponding to
  $B$. Which of the following correctly describes the relation between
  the rows, columns, image and kernel? Please see Option (E) before
  answering.

  \begin{enumerate}[(A)]
  \item If $B$ can be obtained from $A$ by a sequence of row
    interchange operations, then $T_A$ and $T_B$ have the same kernel
    as each other and also the same image as each other.
  \item If $B$ can be obtained from $A$ by a sequence of column
    interchange operations, then $T_A$ and $T_B$ have the same kernel
    as each other and also the same image as each other.
  \item If $B$ can be obtained from $A$ by a sequence of row
    interchange operations, then $T_A$ and $T_B$ have the same kernel
    as each other. If $B$ can be obtained from $A$ by a sequence of
    column interchange operations, then $T_A$ and $T_B$ have the same
    image as each other.
  \item If $B$ can be obtained from $A$ by a sequence of row
    interchange operations, then $T_A$ and $T_B$ have the same image
    as each other. If $B$ can be obtained from $A$ by a sequence of
    column interchange operations, then $T_A$ and $T_B$ have the same
    kernel as each other.
  \item None of the above.
  \end{enumerate}

  {\em Answer}: Option (C)

  {\em Explanation}: Interchanging the rows is a legitimate operation
  for computing the reduced row-echelon form, and the process does not
  affect the solution set for the system of linear equations, aka the
  kernel.

  Also, the columns of the matrix are a spanning set for the image, so
  interchanging the columns should not affect the image.

  However, interchanging the rows can alter the image. For instance:

  $$A = \left[\begin{matrix} 1 & 1 \\ 0 & 0 \\\end{matrix}\right], B = \left[\begin{matrix} 0 & 0 \\ 1 & 1 \\\end{matrix}\right]$$

  have the same set of rows, but different images.

  Similarly, interchanging the columns can alter the kernel. For
  instance:

  $$A = \left[\begin{matrix} 1 & 0 \\ 1 & 0 \\\end{matrix}\right], B = \left[\begin{matrix} 0 & 1 \\ 0 & 1 \\\end{matrix}\right]$$

  have the same set of columns (interchanged) but different kernels.

  {\em Performance review}: 10 out of 27 got this. 6 each chose (D)
  and (E), 3 chose (B), 1 chose (A), 1 left the question blank.
\end{enumerate}
\end{document}
