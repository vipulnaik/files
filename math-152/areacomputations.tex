\documentclass[10pt]{amsart}
\usepackage{fullpage,hyperref,vipul,graphicx}
\title{Area computations using integration}
\author{Math 152, Section 55 (Vipul Naik)}

\begin{document}
\maketitle

{\bf Corresponding material in the book}: Section 5.5, 6.1.

{\bf Difficulty level}: Moderate. The basic computational
ideas are of easy to moderate difficulty, but some of the slicing
ideas at the end are somewhat hard.

{\bf What students should definitely get}: The application of integral
to computing areas. The idea of slicing and integration of slice
lengths.

\section*{Executive summary}

Words ...

\begin{enumerate}

\item We can use integration to determine the area of the region
  between the graph of a function $f$ and the $x$-axis from $x = a$ to
  $x = b$: this integral is $\int_a^b f(x) \, dx$. The integral
  measures the signed area: parts where $f \ge 0$ make positive
  contributions and parts where $f \le 0$ make negative
  contributions. The magnitude-only area is given as $\int_a^b |f(x)|
  \, dx$. The best way of calculating this is to split $[a,b]$ into
  sub-intervals such that $f$ has constant sign on each sub-interval,
  and add up the areas on each sub-interval.
\item Given two functions $f$ and $g$, we can measure the area between
  $f$ and $g$ between $x = a$ and $x = b$ as $\int_a^b |f(x) - g(x)|
  \,dx$. For practical purposes, we divide into sub-intervals so that
  on each sub-interval one function is bigger than the other. We then
  use integration to find the magnitude of the area on each
  sub-interval and add up. If $f$ and $g$ are both continuous, the
  points where the functions {\em cross} each other are points where
  $f = g$.
\item Sometimes, we may want to compute areas against the
  $y$-axis. The typical strategy for doing this is to interchange the
  roles of $x$ and $y$ in the above discussion. In particular, we try
  to express $x$ as a function of $y$.
\item An alternative strategy for computing areas against the $y$-axis
  is to use formulas for computing areas against the $x$-axis, and
  then compute differences of regions.
\item A general approach for thinking of integration is in terms of
  slicing and integration. Here, integration along the $x$-axis is
  based on the following idea: divide the region into vertical slices,
  and then integrate the lengths of these slices along the horizontal
  dimension. Regions for which this works best are the regions called
  {\em Type I regions}. These are the regions for which the
  intersection with any vertical line is either empty or a point or a
  line segment, hence it has a well-defined length.
\item Correspondingly, integration along the $y$-axis is based on
  dividing the region into horizontal slices, and integrating the
  lengths of these slices along the vertical dimension. Regions for
  which this works best are the regions called {\em Type II regions}.
  These are the regions for which the intersection with any horizontal
  line is either empty or a point or a line segment, hence it has a
  well-defined length.
\item Generalizing from both of these, we see that our general
  strategy is to choose two perpendicular directions in the plane, one
  being the direction of our slices and the other being the direction
  of integration.
\end{enumerate}

Actions ...

\begin{enumerate}
\item In some situations we are directly given functions and/or curves
  and are asked to find areas. In others, we are given real-world
  situations where we need to find areas of regions. Here, we have to
  find functions and set up the integration problem as an intermediate
  step.
\item In all these situations, it is important to draw the graphs in a
  reasonably correct way. This brings us to all the ideas that are
  contained in graph drawing. Remember, here we may be interested in
  simultaneously graphing more than one function. Thus, in addition to
  being careful about each function, we should also correctly estimate
  where one function is bigger than the other, and find (approximately
  or exactly) the intersection points. (Go over the notes on
  graph-drawing, and some additional notes on graphing that weren't
  completely covered in class).
\item In some situations, we are asked to find the area(s) of
  region(s) bounded by the graphs of one, two, three, or more
  functions. Here, we first need to sketch the figure. Then, we need
  to find the interval of integration, and if necessary, split this
  interval into sub-intervals, such that on each sub-interval, we know
  exactly what integral we need to do. For instance, consider the
  region between the graphs of $\sin$, $\cos$, and the
  $x$-axis. Basically, the idea is to find, for all the vertical
  slices, the upper and lower limits of the slice.
\end{enumerate}

\section{Integral and area: against the $x$-axis}

\subsection{Definite integral as the signed area between the graph and the $x$-axis}

Suppose $f$ is a continuous function on a closed interval $[a,b]$. The
graph of $f$ forms a curve in the plane $\R^2$. Consider the signed
area between this curve and the $x$-axis. This is the area of the
region bounded by the graph, the $x$-axis, and the vertical lines $x =
a$ and $x = b$.

\includegraphics[width=3in]{areaundergraph.png}

The basic result of integration is that this area equals the definite
integral

$$\int_a^b f(x) \, dx$$

If $f(x) \ge 0$ for all $x \in [a,b]$, i.e., if the graph is entirely
in the upper half-plane (possibly hitting the boundary $x$-axis), then
this integral is nonnegative, and its value is the magnitude of the
area. If $f(x) \le 0$ for all $x \in [a,b]$, i.e., if the graph is
entirely in the lower half-plane (possibly hitting the boundary
$x$-axis), then this integral is zero or negative, and its value is
the {\em negative} of the magnitude of the area. If the function has
parts where it is positive and parts where it is negative, then the
parts where it is positive make positive contributions and the parts
where it is negative make negative contributions.

For instance, consider the function $f(x) := 1 - x^2$. We want to find
the area between the $x$-axis and the graph of the part of this
function that is above the $x$-axis.

First, note that the graph is above the $x$-axis on $(-1,1)$. Thus, in
order to find the area, we need to perform the integration:

$$\int_{-1}^1 (1 - x^2) \, dx$$

\includegraphics[width=3in]{1minusxsquared.png}

We can do this integration by finding an antiderivative and evaluating
it between limits. We take $x - x^3/3$ as the
antiderivative. Evaluating it between limits gives the value
$4/3$. Thus, the area of the region we are interested in is $4/3$.

\subsection{Measuring unsigned area}

\includegraphics[width=3in]{sinefilled.png}

Suppose we want to measure the total area between the graph of the
sine curve and the $x$-axis over one period, say $[0,2\pi]$. In other
words, we want to compute the integral

$$\int_0^{2\pi} \sin x \, dx$$

We know that $- \cos$, which is an antiderivative for $\sin$, also has
a period of $2\pi$. Hence, its value between limits is zero, so the
above integral is zero. Thus, the total {\em signed area} between the
graph of the sine curve and the $x$-axis is zero. This makes sense
graphically. The positive area between the sine curve and the $x$-axis
on the interval $[0,\pi]$ is canceled by a negative area of equal
magnitude between the $x$-axis and the sine curve on the interval
$[\pi,2\pi]$. Why are the two areas the same? There are plenty of ways
of seeing this geometrically. For instance, we have $\sin(\pi +
\theta) = -\sin \theta$ for all angles $\theta$.

Suppose now that, instead of measuring the signed area, we are
interested in measuring the unsigned area. The {\em unsigned area}
between the graph of a function $f$ and the $x$-axis on an interval
$[a,b]$ is given by

$$\int_a^b |f(x)| \, dx$$

Equivalently, we break the interval $[a,b]$ into subintervals such
that $f \ge 0$ or $f \le 0$ on each subinterval. Then we calculate the
magnitude of the integral on each subinterval and add these
magnitudes.

In the case of the sine function, we can partition $[0,2\pi]$ at
$\pi$, to get the subintervals $[0,\pi]$ and $[\pi,2\pi]$. On
$[0,\pi]$, the integral is $[-\cos x]_0^\pi$, which simplifies to
$2$. On $[\pi,2\pi]$, the integral is $-2$, and its magnitude is
$2$. The total magnitude of the integral is thus $2 + 2$, and we know
that $2 + 2 = 4$. Hence, the unsigned area between the graph of $\sin$
and the $x$-axis on $[0,2\pi]$ is $4$. 

\subsection{Area between two graphs}

\includegraphics[width=4in]{areabetweengraphs.png}

Suppose $f$ and $g$ are two continuous functions. To measure the
signed area between the graphs of $f$ and $g$ between the points $a$
and $b$, we compute the integral

$$\int_a^b [f(x) - g(x)] \, dx$$

Here, the subintervals where $f$ is bigger than $g$ make positive
contributions and the subintervals where $g$ is bigger than $f$ make
negative contributions. If we are interested in the unsigned area,
whereby we want positive contributions regardless of which function is
bigger, we consider the integral

$$\int_a^b |f(x) - g(x)| \, dx$$

To compute this, we break up the interval $[a,b]$ into subintervals
based on whether $f$ or $g$ is smaller (the overtaking can happen at
points where $f(x) = g(x)$). We then compute the integral of $f - g$
(or $g - f$, depending on which is bigger) on each subinterval and add
up the magnitudes.

For instance, consider the unsigned area between the graphs of $f(x) =
2x/\pi$ and $g(x) = \sin x$ on the interval $[-\pi/2,\pi/2]$. We see
that $f(x) = g(x)$ at $-\pi/2,0,\pi/2$. On $(-\pi/2,0)$, $f(x) >
g(x)$, and on $(0,\pi/2)$, $g(x) > f(x)$. Thus, the integral is:

$$\int_{-\pi/2}^0 (2x/\pi - \sin x) \, dx + \int_0^{\pi/2} (\sin x - 2x/\pi) \, dx$$

\includegraphics[width=3in]{sineandlinefilling.png}

We can calculate and simplify both these integrals. Note that instead
of computing indefinite integrals for both separately, we can note
that the two functions are negatives of each other, so if we compute
an antiderivative for the first, the antiderivative for the second is
its negative. We get:

$$[x^2/\pi + \cos x]_{-\pi/2}^0 + [-\cos x - x^2/\pi]_0^{\pi/2}$$

Both parts are $1 - \pi/4$, and we thus get $2 - \pi/2$. Since $\pi$
is approximately $3.14$, this is approximately $0.43$.

Why are the two integrals the same? This can be seen geometrically
from the fact that both $f$ and $g$ are odd, so the picture from
$-\pi/2$ to $0$ is the same as the picture from $0$ to $\pi/2$,
subjected to a half-turn about the origin. Thus, the magnitude of the
two areas is the same.

\subsection{Areas bounded by graphs of different functions}

\includegraphics[width=3in]{sinecosinegraphs.png}

Sometimes, the bounding curves for an area come from different
functions. In this case, it makes sense to split up the interval of
integration into subintervals so that we are dealing with only one
function in each subinterval. For instance, consider computing the
area of the region between the $x$-axis and the graphs of $\sin$ and
$\cos$ on the interval $[0,\pi/2]$. On the interval $[0,\pi/4]$, this
is the definite integral of the $\sin$ function, and on the interval
$[\pi/4,\pi/2]$, this is the definite integral of the $\cos$
function. The total area is the sum of the values of these two
definite integrals.

As we can see, both integrals are $1 - 1/\sqrt{2}$, and the total
integral is $2 - \sqrt{2}$, which is approximately $0.59$.

Why are the two integrals the same? We can see graphically that the
two areas being measured are mirror images of each other about the
line $x = \pi/4$. This is because for any angle $\theta$, $\cos (\pi/2
- \theta) = \sin \theta$.

{\em For the second midterm, you are responsible only for the material
till this point.}

\section{Area computations as slicing, and other methods}

This way of thinking about area computations will turn out to be
useful for the subsequent topic, which is volume computations. It also
makes it possible to compute areas of shapes oriented somewhat
differently from before.

\subsection{Vertical slicing}

So far, the situations where we've been computing areas are: area
between the graph of a function and the $x$-axis, area bounded between
graph of a function, the $x$-axis, and two vertical lines, area
between the graphs of two functions, area bounded by the graphs of two
functions and two vertical lines.

In all these situations, the region $\Omega$ whose area we need to
compute has the property that the intersection of $\Omega$ with any
vertical line is either empty or a line segment. Regions of this kind
are sometimes called Type I regions. For Type I regions, the general
formula for the unsigned area is:

$$\int \text{(Length of the line segment as a function of $x$)} \, dx$$

This process can be thought of as {\em vertical slicing}. We are
dividing the area that we want to measure into vertical slices, and
then integrating the length along the perpendicular axis (which is
horizontal).

\subsection{Horizontal slicing}

Horizontal slicing is a lot like vertical slicing, but works for
regions where the role of vertical and horizontal is replaced.

Horizontal slicing works for regions $\Omega$ which have the property:
the intersection of $\Omega$ with any horizontal line is either empty
or a line segment. Regions of this type are sometimes called Type II
regions. The formula for the area of a Type II region is

$$ \int \text{(Length of the line segment as a function of $y$)} \, dy$$

This process can be thought of as {\em horizontal slicing}. We are
dividing the area that we want to measure into horizontal slices, and
the integrating the length along the perpendicular axis (which is
vertical).

Thus, we have seen two processes of breaking up an area into slices:
vertical slicing (where we integrate the lengths along a horizontal
axis) and horizontal slicing (where we integrate the lengths along a
vertical axis).

Notice that both these procedures are variants of the same basic
procedure: choose two mutually perpendicular directions, such that all
lines in one direction have intersection with the region that is
either empty or a line segment. Then, integrate the length of the line
segment along the perpendicular direction.

Note also that the extreme case of both these occurs in
rectangles. Here, whether we use horizontal or vertical slicing, we
are integrating a constant function.

\subsection{Regions whose area can be computed by integration in multiple ways}

Consider the region bounded by the line $y = 4$, $y = x^2$, and the
$y$-axis. This is both a Type I and a Type II region, so we can
determine its area by vertical slicing as well as by horizontal
slicing. Let's first compute the area by vertical slicing.

\includegraphics[width=4in]{paraboliccup.png}

By vertical slicing, the interval is $[0,2]$, and the lower and upper
functions are $x^2$ and $4$ respectively. Thus, the length of the line
segment in each vertical slice is $4 - x^2$. The area is thus:

$$\int_0^2 (4 - x^2) \, dx = [4x - (x^3/3)]_0^2 = 8 - 8/3 = 16/3$$

We could also integrate using horizontal slicing. For this, we express
$x$ in terms of $y$. We get $x = \sqrt{y}$, with $y \in
[0,4]$. Measuring the area between this and the $y$-axis, we get:

$$\int_0^4 \sqrt{y} \, dy = [y^{3/2}/(3/2)]_0^4 = 8/(3/2) = 16/3$$

When we later introduce the concept of {\em inverse function}, we will
notice that what we've just done is moved from integrating one
function to integrating its inverse function. We'll also see a
relationship between this and integration by parts next quarter.

\section{Areas of regions given by inequalities}

Suppose a region of the plane is defined by a set of inequalities. In
other words, the region is defined as the set of all points in the
plane that satisfy a given system of inequalities. How do we find its area?

The first step is to identify the {\em bounding} lines/curves for this
region. The bounding lines are typically the lines given by the case
where equality holds instead of inequality. Once we have found these
boundary curves, we can then try to use horizontal or vertical slicing
to determine the area. In some cases, it makes sense to divide the
region into sub-regions so that it is easy to tackle each sub-region
separately by slicing.

Another complication is that the boundary curves may not be graphs of
functions. Often, they may be graphs of relations, i.e., the set of
points $(x,y)$ satisfying $F(x,y) = 0$ for some two-variable function
$F$. In these cases, we try to break it up into functions. We consider
some examples.

\subsection{The example of the circular disk}

Consider the region $1 \le x^2 + y^2 \le 2$. In other words, we are
looking at the set of points $(x,y)$ such that $x^2 + y^2 \in
[1,2]$. We easily see graphically that this region is bounded on the
inside by the circle $x^2 + y^2 = 1$ and on the outside by $x^2 + y^2
= 2$. The region is called a {\em circular annulus}. To find the area
of the annulus, we thus need to subtract the area of the disk $x^2 +
y^2 \le 1$ from the area of the disk $x^2 + y^2 \le 2$.

Now, it so happens that we know formulas for the areas of these disks:
they are $\pi$ and $2\pi$ respectively, so the difference of areas is
$2\pi - \pi = \pi$. If we did not know these formulas, we would need
to break up the circle into graphs of functions $\pm \sqrt{r^2 -
x^2}$. Unfortunately, integrating these functions requires a
trigonometric substitutions, so illustrating this idea would take us
too far afield.

\end{document}
