\documentclass[10pt]{amsart}

%Packages in use
\usepackage{fullpage, hyperref, vipul, enumerate}

%Title details
\title{Take-home class quiz solutions: due Friday February 1: Limits}
\author{Math 195, Section 59 (Vipul Naik)}
%List of new commands

\begin{document}
\maketitle

\section{Performance review}

$27$ people took this $14$-question quiz. The score distribution was
as follows:

\begin{itemize}
\item Score of $8$: $2$ people.
\item Score of $9$: $3$ people.
\item Score of $10$: $2$ people.
\item Score of $11$: $4$ people.
\item Score of $12$: $5$ people.
\item Score of $13$: $5$ people.
\item Score of $14$: $6$ people.
\end{itemize}

The question wise answers and performance review were as follows:

\begin{enumerate}
\item Option (A): All $27$ people.
\item Option (D): $22$ people.
\item Option (C): $26$ people.
\item Option (C): $20$ people.
\item Option (B): $21$ people.
\item Option (C): $23$ people.
\item Option (B): $26$ people.
\item Option (C): $26$ people.
\item Option (E): $21$ people.
\item Option (D): $21$ people.
\item Option (A): $18$ people.
\item Option (E): $22$ people.
\item Option (A): $19$ people.
\item Option (B): $24$ people.
\end{enumerate}

\section{Solutions}

\begin{enumerate}

\item We call a function $f$ left continuous on an open interval $I$
  if, for all $a \in I$, $\displaystyle \lim_{x \to a^-} f(x) =
  f(a)$. Which of the following is an example of a function that is
  left continuous but not continuous on $(0,1)$? If all are examples,
  please select Option (E).

  \begin{enumerate}[(A)]
  \item $f(x) := \left \lbrace\begin{array}{rl}x, & 0 < x \le 1/2 \\ 2x, & 1/2 < x < 1 \\\end{array}\right.$
  \item $f(x) := \left \lbrace\begin{array}{rl}x, & 0 < x < 1/2 \\ 2x, & 1/2 \le x < 1 \\\end{array}\right.$
  \item $f(x) := \left \lbrace\begin{array}{rl}x, & 0 < x \le 1/2 \\ 2x - (1/2), & 1/2 < x < 1 \\\end{array}\right.$
  \item $f(x) := \left \lbrace\begin{array}{rl}x, & 0 < x < 1/2 \\ 2x - (1/2), & 1/2 \le x < 1 \\\end{array}\right.$
  \item All of the above
  \end{enumerate}

  {\em Answer}: Option (A)

  {\em Explanation}: Note that in all four cases, the two pieces of
  the function are continuous. Thus, the relevant questions are: (i) do
  the two definitions agree at the point where the definition changes
  (in all four cases here, $1/2$)? and (ii) is the point (in all cases,
  $1/2$) where the definition changes included in the left or the
  right piece?

  For options (C) and (D), the definitions on the left and right piece
  agree at $1/2$. Namely the function $x$ and $2x - (1/2)$ both take the
  value $1/2$ at the domain point $1/2$. Thus, options (C) and (D)
  both define continuous functions (in fact, the same continuous
  function).

  This leaves options (A) and (B). For these, the left definition $x$
  and the right definition $2x$ do not match at $1/2$: the former
  gives $1/2$ and the latter gives $1$. In other words, the function
  has a jump discontinuity at $1/2$. Thus, (ii) becomes relevant: is
  $1/2$ included in the left or the right definition?

  For option (A), $1/2$ is included in the left definition, so $f(1/2)
  = 1/2 = \displaystyle \lim_{x \to 1/2^-} f(x)$. On the other hand, $\displaystyle \lim_{x \to
  1/2^+} f(x) = 1$. Thus, the $f$ in option (A) is left continuous but
  not right continuous.

  For option (B), $1/2$ is included in the right definition, so
  $f(1/2) = 1$ and $f$ is right continuous but not left continuous at
  $1/2$.

  {\em Performance review}: All $27$ got this.

  {\em Historical note (last time)}: $47$ out of $49$ people got this
  correct. $1$ person each chose (B) and (C).

  
\item Suppose $f$ and $g$ are functions $(0,1)$ to $(0,1)$ that are
  both left continuous on $(0,1)$. Which of the following is {\em not}
  guaranteed to be left continuous on $(0,1)$? Please see Option (E)
  before answering.

  \begin{enumerate}[(A)]
  \item $f + g$, i.e., the function $x \mapsto f(x) + g(x)$
  \item $f - g$, i.e., the function $x \mapsto f(x) - g(x)$
  \item $f \cdot g$, i.e., the function $x \mapsto f(x)g(x)$
  \item $f \circ g$, i.e., the function $x \mapsto f(g(x))$
  \item None of the above, i.e., they are all guaranteed to be left
  continuous functions
  \end{enumerate}

  {\em Answer}: Option (D)

  {\em Explanation}: We need to construct an explicit example, but we
  first need to do some theoretical thinking to motivate the right
  example. The full reasoning is given below.

  {\em Motivation for example}: Left hand limits split under addition,
  subtraction and multiplication, so options (A)-(C) are guaranteed to
  be left continuous, and are thus false. This leaves the option $f
  \circ g$ for consideration. Let us look at this in more detail.

  For $c \in (0,1)$, we want to know whether:

  $$\displaystyle \lim_{x \to c^-} f(g(x)) \stackrel{?}{=} f(g(c))$$

  We do know, by assumption, that, as $x$ approaches $c$ from the
  left, $g(x)$ approaches $g(c)$. However, we do not know whether
  $g(x)$ approaches $g(c)$ from the left or the right or in
  oscillatory fashion. If we could somehow guarantee that $g(x)$
  approaches $g(c)$ from the left, then we would obtain that the above
  limit holds. However, the given data does not guarantee this, so (D)
  is false.

  We need to construct an example where $g$ is {\em not} an increasing
  function. In fact, we will try to pick $g$ as a decreasing function,
  so that when $x$ approaches $c$ from the left, $g(x)$ approaches
  $g(c)$ from the right. As a result, when we compose with $f$, the
  roles of left and right get switched. Further, we need to construct
  $f$ so that it is left continuous but not right continuous.

  {\em Explanation with example}: Consider the case where, say:

  $$f(x) := \left \lbrace\begin{array}{rl}1/3,& 0 < x \le 1/2 \\ 2/3, & 1/2 < x < 1 \\\end{array}\right.$$

  and

  $$g(x) := 1 - x$$

  Note that both functions have range a subset of $(0,1)$.

  Composing, we obtain that:

  $$f(g(x)) = \left \lbrace\begin{array}{rl}2/3,& 0 < x < 1/2\\ 1/3, & 1/2 \le x < 1 \\\end{array}\right.$$

  $f$ is left continuous but not right continuous at $1/2$, whereas $f
  \circ g$ is right continuous but not left continuous at $1/2$.

  {\em Performance review}: $22$ out of $27$ got this. $2$ chose (E),
  $1$ each chose (C), $1$ chose (B).

  {\em Historical note (last time)}: $20$ out of $49$ people got this
  correct. $26$ people chose (E), $2$ chose (C), $1$ chose (B).


\item Which of these is the correct interpretation of $\displaystyle
  \lim_{x \to c} f(x) = L$ in terms of the definition of limit? Please
  see Option (E) before answering.

  \begin{enumerate}[(A)]
  \item For every $\alpha > 0$, there exists $\beta > 0$ such that if
    $0 < |x - c| < \alpha$, then $|f(x) - L| < \beta$.
  \item There exists $\alpha > 0$ such that for every $\beta > 0$, and
    $0 < |x - c| < \alpha$, we have $|f(x) - L| < \beta$.
  \item For every $\alpha > 0$, there exists $\beta > 0$ such that if
    $0 < |x - c| < \beta$, then $|f(x) - L| < \alpha$.
  \item There exists $\alpha > 0$ such that for every $\beta > 0$ and
    $0 < |x - c| < \beta$, we have $|f(x) - L| < \alpha$.
  \item None of the above
  \end{enumerate}

  {\em Answer}: Option (C)

  {\em Explanation}: $\alpha$ plays the role of $\varepsilon$ and $\beta$
  plays the role of $\delta$.

  {\em Performance review}: $26$ out of $27$ got this. $1$ chose (B).

  {\em Historical note (last time)}: $44$ out of $49$ people got this
  correct. $3$ people chose (A), $2$ chose (B).

\item Suppose $f:\R \to \R$ is a function. Which of the following says
  that $f$ does not have a limit at any point in $\R$ (i.e., there is
  no point $c \in \R$ for which $\displaystyle \lim_{x \to c} f(x)$
  exists)? If all, please select Option (E).

  \begin{enumerate}[(A)]
  \item For every $c \in \R$, there exists $L \in \R$ such that for
    every $\varepsilon > 0$, there exists $\delta > 0$ such that for all
    $x$ satisfying $0 < |x - c| < \delta$, we have $|f(x) - L| \ge
    \varepsilon$.
  \item There exists $c \in \R$ such that for every $L \in \R$, there
    exists $\varepsilon > 0$ such that for every $\delta > 0$, there exists
    $x$ satisfying $0 < |x - c| < \delta$ and $|f(x) - L| \ge \varepsilon$.
  \item For every $c \in \R$ and every $L \in \R$, there exists
    $\varepsilon > 0$ such that for every $\delta > 0$, there exists $x$
    satisfying $0 < |x - c| < \delta$ and $|f(x) - L| \ge \varepsilon$.
  \item There exists $c \in \R$ and $L \in \R$ such that for
    every $\varepsilon > 0$, there exists $\delta > 0$ such that for all
    $x$ satisfying $0 < |x - c| < \delta$, we have $|f(x) - L| \ge
    \varepsilon$.
  \item All of the above.
  \end{enumerate}

  {\em Answer}: Option (C)

  {\em Explanation}: Our statement should be that {\em every} $c$ has
  no limit. In other words, for {\em every} $c$ and {\em every} $L$,
  it is {\em not} true that $\displaystyle \lim_{x \to c} f(x) =
  L$. That's exactly what option (C) says.

  {\em Performance review}: $20$ out of $27$ got this. $4$ chose (A),
  $1$ each chose (B), (D), and (E).

  {\em Historical note (last time)}: $20$ out of $49$ people got this
  correct. $15$ chose (B), $7$ chose (E), $4$ chose (D), $3$ chose
  (A).


\item In the usual $\varepsilon-\delta$ definition of limit for a
  given limit $\displaystyle \lim_{x \to c} f(x) = L$, if a given
  value $\delta > 0$ works for a given value $\varepsilon > 0$, then
  which of the following is true? Please see Option (E) before
  answering.

  \begin{enumerate}[(A)]
  \item Every smaller positive value of $\delta$ works for the same
    $\varepsilon$. Also, the given value of $\delta$ works for every
    smaller positive value of $\varepsilon$.
  \item Every smaller positive value of $\delta$ works for the same
    $\varepsilon$. Also, the given value of $\delta$ works for every
    larger value of $\varepsilon$.
  \item Every larger value of $\delta$ works for the same
    $\varepsilon$. Also, the given value of $\delta$ works for every
    smaller positive value of $\varepsilon$.
  \item Every larger value of $\delta$ works for the same
    $\varepsilon$. Also, the given value of $\delta$ works for every
    larger value of $\varepsilon$.
  \item None of the above statements need always be true.
  \end{enumerate}

  {\em Answer}: Option (B)

  {\em Explanation}: This can be understood in multiple ways. One is
  in terms of the prover-skeptic game. A particular choice of $\delta$
  that works for a specific $\varepsilon$ also works for larger
  $\varepsilon$s, because the function is already ``trapped'' in a
  smaller region. Further, smaller choices of $\delta$ also work
  because the skeptic has fewer values of $x$.

  Rigorous proofs are being skipped here, but you can review the
  formal definition of limit notes if this stuff confuses you.

  {\em Performance review}: $21$ out of $27$ got this. $3$ chose (E),
  $2$ chose (D), $1$ chose (C).

  {\em Historical note (last time)}: $31$ out of $49$ people got this
  correct. $6$ each chose (A) and (E), $5$ chose (C), $1$ chose (D).

\item Which of the following is a correct formulation of the statement
  $\displaystyle \lim_{x \to c} f(x) = L$, in a manner that avoids the use of
  $\varepsilon$s and $\delta$s? Please see Option (E) before answering.

  \begin{enumerate}[(A)]
  \item For every open interval centered at $c$, there is an open
    interval centered at $L$ such that the image under $f$ of the open
    interval centered at $c$ (excluding the point $c$ itself) is
    contained in the open interval centered at $L$.
  \item For every open interval centered at $c$, there is an open
    interval centered at $L$ such that the image under $f$ of the open
    interval centered at $c$ (excluding the point $c$ itself) contains
    the open interval centered at $L$.
  \item For every open interval centered at $L$, there is an open
    interval centered at $c$ such that the image under $f$ of the open
    interval centered at $c$ (excluding the point $c$ itself) is
    contained in the open interval centered at $L$.
  \item For every open interval centered at $L$, there is an open
    interval centered at $c$ such that the image under $f$ of the open
    interval centered at $c$ (excluding the point $c$ itself) contains
    the open interval centered at $L$.
  \item None of the above.
  \end{enumerate}

  {\em Answer}: Option (C)

  {\em Explanation}: The ``open interval centered at $L$'' describes
  the ``$\varepsilon > 0$'' part of the definition (where the open
  interval is the interval $(L - \varepsilon, L + \varepsilon)$). The ``open
  interval centered at $c$'' describes the ``$\delta > 0$'' part of
  the definition (where the open interval is the interval $(c -
  \delta, c + \delta)$). $x$ being in the open interval centered at
  $c$ (except the case $x = c$) is equivalent to $0 < |x - c| <
  \delta$, and $f(x)$ being in the open interval centered at $L$ is
  equivalent to $|f(x) - L| < \varepsilon$.

  {\em Performance review}: $23$ out of $27$ got this. $3$ chose (D),
  $1$ chose (B).

  {\em Historical note (last time)}: $24$ out of $49$ people got this
  correct. $12$ chose (A), $8$ chose (D), $3$ chose (E), $2$ chose (B).


\item Consider the function:

  $$f(x) := \left \lbrace\begin{array}{rl} x, & x \text{ rational }\\1/x, & x \text{ irrational }\\\end{array}\right.$$

  What is the set of all points at which $f$ is continuous? 

  \begin{enumerate}[(A)]
  \item $\{ 0, 1 \}$
  \item $\{ -1,1 \}$
  \item $\{-1,0 \}$
  \item $\{ -1,0,1 \}$
  \item $f$ is continuous everywhere
  \end{enumerate}
  {\em Answer}: Option (B)

  {\em Explanation}: In this interesting example, instead of a {\em
  left} versus {\em right} split, we are splitting the domain into
  rationals and irrationals. For the overall limit to exist at $c$, we
  need that: (i) the limit for the function as defined for rationals
  exists at $c$, (ii) the limit for the function as defined for
  irrationals exists at $c$, and (iii) the two limits are equal.

  Note that regardless of whether the point $c$ is rational or
  irrational, we need {\em both} the rational domain limit and the
  irrational domain limit to exist and be equal at $c$. This is
  because rational numbers are surrounded by irrational numbers and
  vice versa -- both rational numbers and irrational numbers are dense
  in the reals -- hence at any point, we care about the limits
  restricted to the rationals as well as the irrationals.

  The limit for rationals exists for all $c$ and equals the value
  $c$. The limit for irrationals exists for all $c \ne 0$ and equals
  the value $1/c$. For these two numbers to be equal, we need $c =
  1/c$. Solving, we get $c^2 = 1$ so $c = \pm 1$.

  {\em Performance review}: $26$ out of $27$ got this. $1$ chose (A).

  {\em Historical note (last time)}: $39$ out of $49$ people got this
  correct. $7$ chose (D), $2$ chose (A), $1$ chose (E).

\item The graph $y = f(x)$ of a function $f$ defined on all reals has
  a horizontal asymptote $y = c$ as $x$ approaches $+\infty$. Which of
  the following is the correct definition of this?

  \begin{enumerate}[(A)]
  \item For every $a \in \R$, there exists $\delta > 0$ such that for
    all $x$ satisfying $0 < |x - c| < \delta$, we have $f(x) > a$.
  \item For every $a \in \R$, there exists $\varepsilon > 0$ such that
    for all $x$ satisfying $x > a$, we have $|f(x) - c| < \varepsilon$.
  \item For every $\varepsilon > 0$, there exists $a \in \R$ such that
    for all $x$ satisfying $x > a$, we have $|f(x) - c| < \varepsilon$.
  \item For every $\delta > 0$, there exists $a \in \R$ such that for
    all $x$ satisfying $0 < |x - c| < \delta$, we have $f(x) > a$.
  \item For every $\varepsilon > 0$, there exists $\delta > 0$ such
    that for all $x$ satisfying $0 < |x - c| < \delta$, we have $|f(x)
    - c| < \varepsilon$.
  \end{enumerate}

  {\em Answer}: Option (C)

  {\em Explanation}: The neighborhood of $c$ (picked by the skeptic)
  is the interval $(c - \varepsilon, c + \varepsilon)$, and it is
  parametrized by its radius $\varepsilon$. The neighborhood of
  $+\infty$ (picked by the prover) is the interval $(a,\infty)$, and
  it is parametrzed by its lower endpoint $a$. The skeptic then picks
  $x$ in the neighborhood specified by the prover, i.e., $f(x) > a$,
  and then they check whether $f(x)$ is in the chosen neighborhood of
  $c$.

  {\em Performance review}: $26$ out of $27$ got this. $1$ chose (B).

  {\em Historical note (last time)}: $35$ out of $46$ got this correct. $4$
  each chose (B) and (E). $3$ chose (A).

\item Which of the following is the correct definition of
  $\displaystyle \lim_{x \to c^-} f(x) = -\infty$ (in words: the left
  hand limit of $f$ at $c$ is $-\infty$)?

  \begin{enumerate}[(A)]
  \item For every $a \in \R$, there exists $\delta > 0$ such that for
    all $x$ satisfying $0 < |x - c| < \delta$, we have $f(x) > a$.
  \item For every $a \in \R$, there exists $\delta > 0$ such that for
    all $x$ satisfying $0 < x - c < \delta$, we have $f(x) > a$.
  \item For every $a \in \R$, there exists $\delta > 0$ such that for
    all $x$ satisfying $0 < x - c < \delta$, we have $f(x) < a$.
  \item For every $a \in \R$, there exists $\delta > 0$ such that for
    all $x$ satisfying $0 < c - x < \delta$, we have $f(x) > a$.
  \item For every $a \in \R$, there exists $\delta > 0$ such that for
    all $x$ satisfying $0 < c - x < \delta$, we have $f(x) < a$.
  \end{enumerate}

  {\em Answer}: Option (E)

  {\em Explanation}: The neighborhood of $-\infty$ chosen by the
  skeptic is $(-\infty, a)$, and it is parameterized by its upper
  endpoint $a$. The prover picks the parameter $\delta$ for the left
  side $\delta$ ``half-neighborhood'' of $c$, namely $(c - \delta,
  c)$. The skeptic then picks $x$ in this half-neighborhood, and they
  then check whether $f(x) \in (-\infty,a)$. Translating the interval
  conditions into inequality notation, we get the definition as stated.

  {\em Performance review}: $21$ out of $27$ got this. $3$ chose (C),
  $2$ chose (D), $1$ chose (B).

  {\em Historical note (last time)}: $37$ out of $46$ got this correct. $3$
  each chose (B), (C), and (D).


\item Suppose $f$ is a function defined on all of $\R$ and $c \in
  \R$. Which of the following is the correct $\varepsilon-\delta$
  definition for the statement ``$f$ is differentiable at $c$''?
  

  \begin{enumerate}[(A)]
  \item For every $L \in \R$, there exists $\varepsilon > 0$ such that
    for every $\delta > 0$, there exists $x$ such that $0 < |x - c| <
    \delta$ and $|f(x) - f(c) - L(x - c)| \ge |x - c|\varepsilon$.
  \item For every $L \in \R$, there exists $\varepsilon > 0$ such that
    for every $\delta > 0$, there exists $x$ such that $0 < |x - c| <
    \delta$ and $|f(x) - f(c) - L(x - c)| < |x - c|\varepsilon$.
  \item There exists $L \in \R$ such that for every $\varepsilon > 0$,
  there exists $\delta > 0$ such that for every $x$ satisfying $0 < |x
  - c| < \delta$, we have $|f(x) - f(c) - L(x - c)| \ge |x - c|\varepsilon$
  \item There exists $L \in \R$ such that for every $\varepsilon > 0$,
  there exists $\delta > 0$ such that for every $x$ satisfying $0 < |x
  - c| < \delta$, we have $|f(x) - f(c) - L(x - c)| < |x - c|\varepsilon$
  \item There exists $L \in \R$ such that there exists $\varepsilon > 0$
    such that for every $\delta > 0$, there exists $x$ such that $0 <
    |x - c| < \delta$ and $|f(x) - f(c) - L(x - c)| < |x - c|\varepsilon$.
  \end{enumerate}

  {\em Answer}: Option (D)

  {\em Explanation}: We would like to say that there exists $L \in \R$
  (where $L$ will be the claimed value of $f'(c)$) such that:

  $$\lim_{x \to c} \frac{f(x) - f(c)}{x - c} = L$$

  To do this, we must say that for every $\varepsilon > 0$, there
  exists $\delta > 0$ such that for all $x$ satisfying $0 < |x - c| <
  \delta$, we have:

  $$\left|\frac{f(x) - f(c)}{x - c} - L\right| < \varepsilon$$

  Rewriting the final inequality, we get option (D).

  {\em Performance review}: $21$ out of $27$ got this. $3$ chose (B),
  $1$ each chose (A) and (C), $1$ left the question blank.

  {\em Historical note (last time)}: $17$ out of $44$ got this. $13$ chose (A),
  $12$ chose (B), $1$ each chose (C) and (E).


\item Suppose $f$ is a function defined on all of $\R$ and $c \in
  \R$. Which of the following is the correct $\varepsilon-\delta$
  definition for the statement ``$f$ is not differentiable at $c$''?

  \begin{enumerate}[(A)]
  \item For every $L \in \R$, there exists $\varepsilon > 0$ such that
    for every $\delta > 0$, there exists $x$ such that $0 < |x - c| <
    \delta$ and $|f(x) - f(c) - L(x - c)| \ge |x - c|\varepsilon$.
  \item For every $L \in \R$, there exists $\varepsilon > 0$ such that
    for every $\delta > 0$, there exists $x$ such that $0 < |x - c| <
    \delta$ and $|f(x) - f(c) - L(x - c)| < |x - c|\varepsilon$.
  \item There exists $L \in \R$ such that for every $\varepsilon > 0$,
  there exists $\delta > 0$ such that for every $x$ satisfying $0 < |x
  - c| < \delta$, we have $|f(x) - f(c) - L(x - c)| \ge |x - c|\varepsilon$
  \item There exists $L \in \R$ such that for every $\varepsilon > 0$,
  there exists $\delta > 0$ such that for every $x$ satisfying $0 < |x
  - c| < \delta$, we have $|f(x) - f(c) - L(x - c)| < |x - c|\varepsilon$
  \item There exists $L \in \R$ such that there exists $\varepsilon > 0$
    such that for every $\delta > 0$, there exists $x$ such that $0 <
    |x - c| < \delta$ and $|f(x) - f(c) - L(x - c)| < |x - c|\varepsilon$.
  \end{enumerate}

  {\em Answer}: Option (A)

  {\em Explanation}: We would like to say that for every $L \in \R$,
  the statement:

  $$\lim_{x \to c} \frac{f(x) - f(c)}{x - c} = L$$

  is false. To do this, we must say that there exists $\varepsilon > 0$
  such that the difference quotient (i.e., the expression on the left)
  cannot be trapped within the interval $(L - \varepsilon,L +
  \varepsilon)$. In other words, for every $\delta > 0$, there is some
  value of $x$ such that $0 < |x -c| < \delta$ and:

  $$\left|\frac{f(x) - f(c)}{x - c} - L\right| \ge \varepsilon$$

  Rewriting the final inequality, we get option (A).

  {\em Performance review}: $18$ out of $27$ got this. $7$ chose (C),
  $2$ chose (D).

  {\em Historical note (administered in an earlier year)}: $5$ out of
  $15$ people got this correct. $4$ people chose (C), $2$ people chose
  (D), $2$ people chose (E), $1$ person chose (B), and $1$ person left
  the question blank.

\item Suppose $f:\R \to \R$ is a function. Identify which of these
  definitions is {\em not} correct for $\displaystyle \lim_{x \to c}
  f(x) = L$, where $c$ and $L$ are both finite real numbers. If all
  are correct, please select Option (E).

  \begin{enumerate}[(A)]
  \item For every $\varepsilon > 0$, there exists $\delta > 0$ such
    that if $x \in (c - \delta, c + \delta) \setminus \{ c \}$, then
    $f(x) \in (L - \varepsilon, L + \varepsilon)$.
  \item For every $\varepsilon_1 > 0$ and $\varepsilon_2 > 0$, there exist
    $\delta_1 > 0$ and $\delta_2 > 0$ such that if $x \in (c -
    \delta_1,c+\delta_2)\setminus \{ c \}$, then $f(x) \in (L -
    \varepsilon_1, L + \varepsilon_2)$.
  \item For every $\varepsilon_1 > 0$ and $\varepsilon_2 > 0$, there exists
    $\delta > 0$ such that if $x \in (c - \delta, c + \delta)
    \setminus \{ c \}$, then $f(x) \in (L - \varepsilon_1, L + \varepsilon_2)$.
  \item For every $\varepsilon > 0$, there exist $\delta_1 > 0$ and
    $\delta_2 > 0$ such that if $x \in (c - \delta_1, c + \delta_2)
    \setminus \{ c \}$, then $f(x) \in (L - \varepsilon, L + \varepsilon)$.
  \item None of these, i.e., all definitions are correct.
  \end{enumerate}

  {\em Answer}: Option (E)

  {\em Explanation}: Although the usual $\varepsilon-\delta$ definition
  uses centered intervals, i.e., intervals centered at the points $c$
  and $L$, this is not a necessary aspect of the definition. So,
  instead of taking centered intervals $(c - \delta, c + \delta)$ or
  $(L - \varepsilon,L + \varepsilon)$, we could consider open intervals that
  have different amounts on the left and on the right. Thus, all four
  definitions are correct.

  {\em Performance review}: $22$ out of $27$ got this. $3$ chose (D),
  $2$ chose (C).

  {\em Historical note (last time)}: $33$ out of $41$ got this. $3$ chose
  (C). $2$ each chose (B) and (D), $1$ left the question blank.


\item In the usual $\varepsilon-\delta$ definition of limit, we find that
  the value $\delta = 0.2$ for $\varepsilon = 0.7$ for a function $f$ at
  $0$, and the value $\delta = 0.5$ works for $\varepsilon = 1.6$ for a
  function $g$ at $0$. What value of $\delta$ {\em definitely} works
  for $\varepsilon = 2.3$ for the function $f + g$ at $0$?

  \begin{enumerate}[(A)]
  \item $0.2$
  \item $0.3$
  \item $0.5$
  \item $0.7$
  \item $0.9$
  \end{enumerate}

  {\em Answer}: Option (A)

  {\em Explanation}: We choose the {\em smaller} of the $\delta$s to
  guarantee that {\em both} $f$ and $g$ are within their respective
  $\varepsilon$-distances of the targets -- $0.7$ in the case of $f$ and
  $1.6$ in the case of $g$. Now, the triangle inequality guarantees
  that $f + g$ is within $2.3$ of its proposed limit.

  {\em Performance review}: $19$ out of $27$ got this. $6$ chose (D),
  $1$ each chose (C) and (E).

  {\em Historical note (last time)}: $36$ out of $41$ got this. $3$ chose (D),
  $1$ each chose (C) and (E).

\item The sum of limits theorem states that $\displaystyle \lim_{x \to c} [f(x) +
  g(x)] = \displaystyle \lim_{x \to c} f(x) + \displaystyle \lim_{x \to c} g(x)$ {\em if} the
  right side is defined. One of the choices below gives an example
  where the left side of the equality is defined and finite but the right side
  makes no sense. Identify the correct choice.

  \begin{enumerate}[(A)]
  \item $f(x) := 1/x$, $g(x) := -1/(x + 1)$, $c = 0$.
  \item $f(x) := 1/x$, $g(x) := (x - 1)/x$, $c = 0$.
  \item $f(x) := \arcsin x$, $g(x) := \arccos x$, $c = 1/2$.
  \item $f(x) := 1/x$, $g(x) = x$, $c = 0$.
  \item $f(x) := \tan x$, $g(x) := \cot x$, $c = 0$.
  \end{enumerate}

  {\em Answer}: Option (B)

  {\em Explanation}: $f + g$ is the constant function $1$, so it has a
  limit. On the other hand, both $f$ and $g$ have one-sided limits of
  $\pm \infty$.

  For options (A), (D), and (E), one of the function $f$ and $g$ has a
  finite limit, and the other has an infinite or undefined limit, and
  the sum has an infinite or undefined limit. Option (C) is a case
  where $f$, $g$, and $f + g$ all have finite limits.

  {\em Performance review}: $24$ out of $27$ got this. $2$ chose (A),
  $1$ chose (E).

  {\em Historical note (last time)}: $36$ out of $41$ got this. $2$ each chose
  (A) and (E), $1$ chose (C).

\end{enumerate}
\end{document}