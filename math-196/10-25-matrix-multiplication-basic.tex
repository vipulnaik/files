\documentclass[10pt]{amsart}

%Packages in use
\usepackage{fullpage, hyperref, vipul, enumerate}

%Title details
\title{Diagnostic in-class quiz: due Friday October 25: Matrix multiplication (basic)}
\author{Math 196, Section 57 (Vipul Naik)}
%List of new commands

\begin{document}
\maketitle

Your name (print clearly in capital letters): $\underline{\qquad\qquad\qquad\qquad\qquad\qquad\qquad\qquad\qquad\qquad}$

{\bf PLEASE DO {\em NOT} DISCUSS ANY QUESTIONS}

This quiz tests for basic comprehension of the setup for matrix
multiplication. It corresponds to the material from Sections 1-6
(excluding Section 4) of the {\tt Matrix multiplication and inversion}
notes, and also to Section 2.3 of the book.

\begin{enumerate}
\item {\em Do not discuss this!}: Suppose $A$ and $B$ are (not
  necessarily square) matrices. Then, which of the following describes
  correctly the relationship between the existence and value of the
  (alleged) matrix product $AB$ and the existence and value of the
  (alleged) matrix product $BA$?

  \begin{enumerate}[(A)]
  \item $AB$ is defined if and only if $BA$ is defined, and if so,
    they are equal.
  \item $AB$ is defined if and only if $BA$ is defined, but they need
    not be equal.
  \item If $AB$ and $BA$ are both defined, then $AB = BA$. However, it
    is possible for one of $AB$ and $BA$ to be defined and the other
    to not be defined.
  \item It is possible for only one of $AB$ and $BA$ to be defined. It
    is also possible for both $AB$ and $BA$ to be defined, but to not
    be equal to each other.
  \end{enumerate}

  \vspace{0.1in}
  Your answer: $\underline{\qquad\qquad\qquad\qquad\qquad\qquad\qquad}$
  \vspace{0.1in}
\item {\em Do not discuss this!}: Suppose $A$ and $B$ are matrices
  such that both $AB$ and $BA$ are defined. Which of the following
  correctly describes what we know about $AB$ and $BA$?

  \begin{enumerate}[(A)]
  \item Both $AB$ and $BA$ are square matrices and have the same
    dimensions, i.e., in both $AB$ and $BA$, the number of rows equals
    the number of columns, and further, the number of rows of $AB$
    equals the number of rows of $BA$.
  \item Both $AB$ and $BA$ are square matrices (the number of rows
    equals the number of columns) but they may not have the same
    dimensions: the number of rows in $AB$ need not equal the number of
    rows in $BA$.
  \item $AB$ and $BA$ need not be square matrices but both must have
    the same dimensions: the number of rows in $AB$ equals the number
    of rows in $BA$, and the number of columns in $AB$ equals the
    number of columns in $BA$.
  \item $AB$ and $BA$ need not be square matrices and they need not
    have the same row count or the same column count, i.e., the number
    of rows in $AB$ need not equal the number of rows in $BA$, and the
    number of columns in $AB$ need not equal the number of columns in
    $BA$.
  \end{enumerate}

  \vspace{0.1in}
  Your answer: $\underline{\qquad\qquad\qquad\qquad\qquad\qquad\qquad}$
  \vspace{0.1in}

\item {\em Do not discuss this!}: Suppose $A$ and $B$ are matrices
  such that both $AB$ and $A + B$ are defined. Which of the following
  correctly describes what we know about $A$ and $B$?

  \begin{enumerate}[(A)]
  \item Both $A$ and $B$ are square matrices and have the same
    dimensions, i.e., in both $A$ and $B$, the number of rows equals
    the number of columns, and further, the number of rows of $A$
    equals the number of rows of $B$.
  \item Both $A$ and $B$ are square matrices (the number of rows
    equals the number of columns) but they may not have the same
    dimensions: the number of rows in $A$ need not equal the number of
    rows in $B$.
  \item $A$ and $B$ need not be square matrices but both must have
    the same dimensions: the number of rows in $A$ equals the number
    of rows in $B$, and the number of columns in $A$ equals the
    number of columns in $B$.
  \item $A$ and $B$ need not be square matrices and they need not have
    the same row count or the same column count, i.e., the number of
    rows in $A$ need not equal the number of rows in $B$, and the
    number of columns in $A$ need not equal the number of columns in
    $B$.
  \end{enumerate}

  \vspace{0.1in}
  Your answer: $\underline{\qquad\qquad\qquad\qquad\qquad\qquad\qquad}$
  \vspace{0.1in}

\item {\em Do not discuss this!}: Suppose $A$ is a $p \times q$ matrix
  and $B$ is a $q \times r$ matrix. The product matrix $AB$ is a $p
  \times r$ matrix. Using the convention of matrices as linear
  transformations via their action by multiplication on column vectors,
  what is the appropriate interpretation of the matrix product in
  terms of composing linear transformations?

  \begin{enumerate}[(A)]
  \item $A$ corresponds to a linear transformation $T_A$ from $\R^p$
    to $\R^q$, and $B$ corresponds to a linear transformation $T_B$
    from $\R^q$ to $\R^r$. The product $AB$ therefore corresponds to a
    linear transformation from $\R^p$ to $\R^r$ that is the composite
    of the two linear transformations, with $T_A$ applied first (to
    the domain) and then $T_B$ ($T_B$ being applied to the
    intermediate space obtained after applying $T_A$).
  \item $A$ corresponds to a linear transformation $T_A$ from $\R^p$
    to $\R^q$, and $B$ corresponds to a linear transformation $T_B$
    from $\R^q$ to $\R^r$. The product $AB$ therefore corresponds to a
    linear transformation from $\R^p$ to $\R^r$ that is the composite
    of the two linear transformations, with $T_B$ applied first (to
    the domain) and then $T_A$ ($T_A$ being applied to the
    intermediate space obtained after applying $T_B$).
  \item $A$ corresponds to a linear transformation $T_A$ from $\R^q$
    to $\R^p$, and $B$ corresponds to a linear transformation $T_B$
    from $\R^r$ to $\R^q$. The product $AB$ therefore corresponds to a
    linear transformation from $\R^r$ to $\R^p$ that is the composite
    of the two linear transformations, with $T_A$ applied first (to
    the domain) and then $T_B$ ($T_B$ being applied to the
    intermediate space obtained after applying $T_A$).
  \item $A$ corresponds to a linear transformation $T_A$ from $\R^q$
    to $\R^p$, and $B$ corresponds to a linear transformation $T_B$
    from $\R^r$ to $\R^q$. The product $AB$ therefore corresponds to a
    linear transformation from $\R^r$ to $\R^p$ that is the composite
    of the two linear transformations, with $T_B$ applied first (to
    the domain) and then $T_A$ ($T_A$ being applied to the
    intermediate space obtained after applying $T_B$).
  \end{enumerate}

  \vspace{0.1in}
  Your answer: $\underline{\qquad\qquad\qquad\qquad\qquad\qquad\qquad}$
  \vspace{0.1in}

\item {\em Do not discuss this!}: Suppose $A$, $B$, and $C$ are
  matrices. Which of the following is true?

  \begin{enumerate}[(A)]
  \item If $ABC$ is defined, then so are $BCA$ and $CAB$.
  \item If $ABC$ and $BCA$ are both defined, then so is
    $CAB$. However, it is possible to have a situation where $ABC$ is
    defined but $BCA$ and $CAB$ are not defined.
  \item It is possible to have a situation where $ABC$ and $BCA$ are
    both defined but $CAB$ is not defined.
  \end{enumerate}

  \vspace{0.1in}
  Your answer: $\underline{\qquad\qquad\qquad\qquad\qquad\qquad\qquad}$
  \vspace{0.1in}

\end{enumerate}
\end{document}
