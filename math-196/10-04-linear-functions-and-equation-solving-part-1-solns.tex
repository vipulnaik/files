\documentclass[10pt]{amsart}

%Packages in use
\usepackage{fullpage, hyperref, vipul, enumerate}

%Title details
\title{Take-home class quiz solutions: due Friday October 4: Linear functions and equation-solving (part 1)}
\author{Math 196, Section 57 (Vipul Naik)}
%List of new commands

\begin{document}
\maketitle

\section{Performance review}

27 people took this 9-question quiz. The score distribution was as
follows:

\begin{itemize}
\item Score of 1: 2 people
\item Score of 3: 1 person
\item Score of 4: 2 people
\item Score of 5: 4 people
\item Score of 6: 3 people
\item Score of 7: 8 people
\item Score of 8: 5 people
\item Score of 9: 2 people
\end{itemize}

The mean score was about 6.1.

The question-wise answers and performance review are below:

\begin{enumerate}
\item Option (D): 20 people
\item Option (C): 16 people
\item Option (B): 21 people
\item Option (A): 23 people
\item Option (E): 9 people
\item Option (D): 19 people
\item Option (C): 26 people
\item Option (C): 11 people
\item Option (A): 20 people
\end{enumerate}

\section{Solutions}

This quiz covers some basics involving linear functions and
equation-solving (notes at {\tt Linear functions: a primer} and {\tt
  Equation-solving with a special focus on the linear case}). The quiz
tests for the following:

\begin{itemize}
\item What it means to be (affine) linear, and in particular, the
  significance of the intercept as an additional parameter to track.
\item The distinction between behavior relative to the variables (the
  inputs) and behavior relative to the parameters.
\item Using the linear paradigm to study functional forms that are not
  themselves linear.
\item A small taste of dealing with measurement uncertainty to obtain
  upper and lower bounds (not covered in the notes, so this is where
  your famed ability to think out of the box should manifest).
\item Solving ``triangular'' systems of equations.
\end{itemize}

\begin{enumerate}
\item A function $f$ of $3$ variables $x$, $y$, $z$ defined everywhere is
  (affine) linear in the variables. (The ``affine'' is to indicate
  that the intercept may be nonzero). Based on the above information
  and some input-output pairs for $f$, we would like to determine $f$
  uniquely. What is the minimum number of input-output pairs that we
  would need in order to achieve this?

  \begin{enumerate}[(A)]
  \item $1$
  \item $2$
  \item $3$
  \item $4$
  \item $5$
  \end{enumerate}

  {\em Answer}: Option (D)

  {\em Explanation}: The general expression for a linear function $f$
  of the variables $x$, $y$, and $z$ is:

  $$f(x,y,z) := ax + by + cz + d$$

  There are four unknown parameters here (one coefficient for each
  variable, and one parameter $d$ for the intercept). Our goal is to
  determine uniquely the values of the parameters. Since there are
  four parameters, we need four equations to determine them
  uniquely. Note that the equations themselves are linear. We should
  choose our four inputs in a manner that there are no linear
  dependencies between them (what this means will become clearer as we
  study more linear algebra).

  {\em Performance review}: $20$ out of $27$ got this. $6$ chose (C),
  $1$ chose (E).

  {\em Historical note (last time; however, discussion was not
    permitted for this question last time)}: $7$ out of $29$ people
  got this. $20$ chose (C), $2$ chose (B).

\item Which of the following gives an example of a function $F$ of
  three variables $x,y,z$ whose third-order mixed partial derivative
  $F_{xyz}$ is zero everywhere, but for which none of the second-order
  mixed partial derivatives $F_{xy}$, $F_{xz}$, $F_{yz}$ is zero
  everywhere?

  \begin{enumerate}[(A)]
  \item $\sin(xy) - z^2$
  \item $\cos(x^2 + y^2) - \sin(y^2 + z^2)$
  \item $e^{xy} + (y - z)^2 + 3xz$
  \item $x^2 + y^2 + z^2$
  \item $xyz$
  \end{enumerate}

  {\em Answer}: Option (C)

  {\em Explanation}: Look for functions of the form:

  $$F(x,y,z) = f(x,y) + g(y,z) + h(x,z)$$

  where $f$, $g$, and $h$ are all nonzero and none of them is
  additively separable.

  The only option fitting this description is Option (C).

  As for the other options:

  \begin{itemize}
  \item Option (A): Both $F_{xz}$ and $F_{yz}$ are zero. This is
    because neither $x$ nor $y$ interacts with $z$.
  \item Option (B): $F_{xz}$ is zero, because there is no interaction
    between $x$ and $z$.
  \item Option (D): $F$ is completely additively separable in terms of
    $x$, $y$, and $z$, so all the second-order mixed partials
    $F_{xy}$, $F_{xz}$, and $F_{yz}$ are zero.
  \item Option (E): $F_{xyz} = 1$.
  \end{itemize}

  {\em Performance review}: $16$ out of $27$ got this. $7$ chose (E),
  $3$ chose (B), $1$ chose (A).

  {\em Historical note (last time)}: $20$ out of $29$ people got this. $8$
  chose (E), $1$ chose (D).

\item Consider a function of the form $F(x,y) := Ca^xb^y$ where
  $C,a,b$ are all positive reals that serve as parameters and $x,y$
  are restricted to the positive reals. We wish to study $F$ using the
  paradigm of linear functions. What is the best way of doing this?

  \begin{enumerate}[(A)]
  \item Express $\ln(F(x,y))$ in terms of $\ln x$ and $\ln y$
  \item Express $\ln(F(x,y))$ in terms of $x$ and $y$
  \item Express $F(x,y)$ in terms of $\ln x$ and $\ln y$
  \item Express $\ln(F(x,y))$ in terms of $a^x$ and $b^y$
  \item Express $F(x,y)$ in terms of $a^x$ and $b^y$
  \end{enumerate}

  {\em Answer}: Option (B)

  {\em Explanation}: We take logarithms to get:

  $$\ln(F(x,y)) = \ln C + x \ln a + y \ln b$$

  Note that $a$, $b$, and $C$ are positive constants. Hence, $\ln C$,
  $\ln a$, and $\ln b$ are also constants. Thus, $\ln(F(x,y))$ is a
  linear function of $x$ and $y$.

  {\em Performance review}: $21$ out of $27$ got this. $4$ chose (A),
  $2$ chose (D).

  {\em Historical note (last time)}: $26$ out of $29$ got this. $3$ chose (A).

\item Consider a function of the form $F(x,y) := Cx^ay^b$ where
  $C,a,b$ are all positive reals that serve as parameters and $x,y$
  are restricted to the positive reals. We wish to study $F$ using the
  paradigm of linear functions. What is the best way of doing this?

  \begin{enumerate}[(A)]
  \item Express $\ln(F(x,y))$ in terms of $\ln x$ and $\ln y$
  \item Express $\ln(F(x,y))$ in terms of $x$ and $y$
  \item Express $F(x,y)$ in terms of $\ln x$ and $\ln y$
  \item Express $\ln(F(x,y))$ in terms of $x^a$ and $y^b$
  \item Express $F(x,y)$ in terms of $x^a$ and $y^b$
  \end{enumerate}

  {\em Answer}: Option (A)

  {\em Explanation}: We take logarithms to get:

  $$\ln(F(x,y)) = \ln C + a \ln x + b \ln y$$

  Note that $C$ is a positive constant, so $\ln C$ is a constant.
  $\ln(F(x,y))$ is a linear function of $\ln x$ and $\ln y$.

  {\em Performance review}: $23$ out of $27$ got this. $1$ each chose
  (B), (C)., and (D), $1$ left the question blank.

  {\em Historical note (last time)}: $28$ out of $29$ got this. $1$
  chose (B).

\item (**) {\em This is a hard question!} The population in the island
  of Andrognesia as a function of time is believed to be an
  exponential function. On January 1, 1984, the population was
  measured to be $3 * 10^5$ with a measurement error of up to $10^5$
  on either side, i.e., the population was measured to be between $2*
  10^5$ and $4 * 10^5$. On January 1, 1998, the population was
  measured to be $1.2 * 10^6$ with a measurememt error of up to $4 *
  10^5$ on either side, i.e., the population was measured to be
  between $8 * 10^5$ and $1.6 * 10^6$. If the population is an
  exponential function of time (i.e., the increment in population per
  year is a fixed proportion of the population that year), what is the
  {\bf range of possible values} of the population measured on January
  1, 2012? {\em Hint: Think of the umbral versus penumbral region for
    an eclipse}

  \begin{enumerate}[(A)]
  \item Between $3.2 * 10^6$ and $6.4 * 10^6$
  \item Between $3.2 * 10^6$ and $1.28 * 10^7$
  \item Between $1.6 * 10^6$ and $3.2 * 10^6$
  \item Between $1.6 * 10^6$ and $6.4 * 10^6$
  \item Between $1.6 * 10^6$ and $1.28 * 10^7$
  \end{enumerate}

 {\em Answer}: Option (E)

  {\em Explanation}: Note first that $2012 - 1998 = 1998 - 1984 = 14$.

  The key idea is that the lowest estimate occurs
  if the 1998 population was measured as low as possible {\em and} the
  rate of population growth estimated using the 1984 and 1998
  populations is as low as possible. The lowest possible rate of
  growth we can measure occurs if we choose the highest possible 1984
  value and the lowest possible 1998 value. Picking these, we obtain
  that the population estimate for 1984 is $4 * 10^5$ and the
  population estimate for 1998 is $8 * 10^5$. Since the
  multiplicative growth of the population depends on the time elapsed,
  the total population in 2012 will be the solution $x$ to:

  $$\frac{x}{8 * 10^5} = \frac{8 * 10^5}{4 * 10^5}$$

  which solves to $x = 1.6 * 10^6$.

  Similarly, the highest estimate will occur if we take the highest
  estimate possible for the 1998 population and the lowest estimate
  possibly for the 1984 population.

  This relates to the idea of linear models as follows. Consider a
  plot of the logarithm of the population with respect to time. Since
  the growth is exponential, this should be a linear plot. If we knew
  the precise values of the populations in 1984 and 1998, we could fit
  a straight line through those and use that to determine the
  population in 2012. Uncertainty regarding the values of the
  population in 1984 and 1998, however. means that instead of having
  points in the graph, we have an interval (represented by a vertical
  line segment) for the time coordinate value of 1984 and another
  interval (represented by another vertical line segment) for the time
  coordinate value of 1998.

  The upper end estimate is obtained by making a line through the
  lower end of the 1984 estimate range and the upper end of the 1998
  estimate range. The lower end estimate is obtained by making a line
  through the upper end of the 1984 estimate range and the lower end
  of the 1998 estimate range.

  {\em Performance review}: $9$ out of $27$ got this. $14$ chose (A),
  $3$ chose (D), $1$ chose (C).

  {\em Historical note (last time)}: $6$ out of $29$ got
  this. $15$ chose (A), $6$ chose (C), $1$ each chose (B) and (D).

\item Suppose, according to our model, a particular function $f(x,y)$
  is of the form $f(x,y) = a_1 + a_2x + a_3y + a_4x^2y^2$ where
  $a_1,a_2,a_3,a_4$ are parameters. Our goal is to determine the
  values of the parameters $a_1,a_2,a_3,a_4$. We do this by collecting
  a number of (input,output) pairs for the function $f$ and then
  setting up equations in terms of the parameters using the
  (input,output) pairs. What can we say about the nature of $f$ and
  the nature of the system of equations that we will need to solve?
  {\em Note that ``nonlinear'' as used here simply means that the
    expression is not guaranteed to be linear, though it may turn out
    to be linear in some cases. Similarly, ``non-polynomial'' means
    not guaranteed to be polynomial, though it may turn out to be
    polynomial in some cases.}

  \begin{enumerate}[(A)]
  \item $f$ is a linear function of $x$ and $y$, hence we need to
    solve a linear system of equations to determine the parameters
    $a_1,a_2,a_3,a_4$.
  \item $f$ is a nonlinear polynomial function of $x$ and $y$, hence
    we need to solve a nonlinear polynomial system of equations to
    determine the parameters $a_1,a_2,a_3,a_4$.
  \item $f$ is a linear function of $x$ and $y$. However, we need to
    solve a nonlinear polynomial system of equations to determine the
    parameters $a_1,a_2,a_3,a_4$.
  \item $f$ is a nonlinear polynomial function of $x$ and
    $y$. However, we need to solve a linear system of equations to
    determine the parameters $a_1,a_2,a_3,a_4$.
  \item $f$ is a nonlinear polynomial function of $x$ and
    $y$. However, we need to solve a non-polynomial system of
    equations to determine the parameters $a_1,a_2,a_3,a_4$.
  \end{enumerate}

  {\em Answer}: Option (D)

  {\em Explanation}: $f$ is polynomial in $x$ and $y$, and it is not
  linear because one of its terms is $x^2y^2$ (note that it may {\em
    happen} to be linear if $a_4 = 0$, but we do not know this in
  advance).

  However, $f$ is linear in the parameters, hence the system of
  equations that we get from input-output pairs is a linear system of
  equations.

  {\em Performance review}: $19$ out of $27$ got this. $6$ chose (B),
  $1$ each chose (C) and (E).

  {\em Historical note (last time)}: $24$ out of $29$ got this. $4$ chose (B),
  $1$ chose (A).
\item Consider the system of equations:

  \begin{eqnarray*}
    x^2 - x & = & 2 \\
    y^2 + xy & = & x + 13 \\
  \end{eqnarray*}

  What is the number of solutions to this system for real $x$ and $y$?

  \begin{enumerate}[(A)]
  \item $0$
  \item $2$
  \item $4$
  \item $6$
  \item $8$
  \end{enumerate}
  \vspace{0.6in}
  
  {\em Answer}: Option (C)

  {\em Explanation}: Solving the first equation, we get:

  $$(x - 2)(x + 1) = 0$$

  Thus, we have $x = 2$ or $x = -1$.

  For each choice of $x$, we need to solve the second equation by
  plugging in that value of $x$.

  For the choice $x = 2$, we have:

  $$y^2 + 2y = 15$$

  This simplifies to:

  $$(y - 3)(y + 5) = 0$$

  Thus, $y = 3$ or $y = -5$, so we get the solutions $x = 2, y = 3$
  and $x = 2, y = -5$.

  For the choice $x = -1$, we get:

  $$y^2 - y = 12$$

  This gives:

  $$(y - 4)(y + 3) = 0$$

  Thus, $y = 4$ or $y = -3$, so we get the solutions $x = -1, y = 4$
  and $x = -1, y= -3$.

  Overall, we have four solutions:

  \begin{itemize}
  \item $x = 2, y = 3$
  \item $x = 2, y = -5$
  \item $x = -1, y = 4$
  \item $x = -1, y = -3$
  \end{itemize}

  {\em Performance review}: $26$ out of $27$ got this. $1$ chose (D).

  {\em Historical note (last time)}: $28$ out of $29$ got this. $1$
  chose (D).

\item Consider the system of equations:

  \begin{eqnarray*}
    x^2 - x & = & 2 \\
    y^2 + xy & = & x + 13 \\
    z^2 & = & xy \\
  \end{eqnarray*}

  What is the number of solutions to this system for real $x$, $y$,
  and $z$?

  \begin{enumerate}[(A)]
  \item $0$
  \item $2$
  \item $4$
  \item $6$
  \item $8$
  \end{enumerate}

  {\em Answer}: Option (C)

  {\em Explanation}: Recall from the preceding question that we have
  the following solutions to the first two equations:

  \begin{itemize}
  \item $x = 2, y = 3$
  \item $x = 2, y = -5$
  \item $x = -1, y = 4$
  \item $x = -1, y = -3$
  \end{itemize}

  For each of these, our goal is to find the corresponding $z$-values
  that work. Note that if $xy$ is positive, there are two
  $z$-values. If $xy = 0$, there is a unique $z$-value, and if $xy$ is
  negative, there are no $z$-values.

  Of the four possible $(x,y)$-value pairs, only two give positive
  products. The other two give negative products. In both the positive
  product cases, we get $2$ values of $z$, so we overall get four
  solutions as listed below:

  \begin{itemize}
  \item $x = 2, y = 3, z = \sqrt{6}$
  \item $x = 2, y = 3, z = -\sqrt{6}$
  \item $x = -1, y = -3, z = \sqrt{3}$
  \item $x = -1, y = -3, z = -\sqrt{3}$
  \end{itemize}

  {\em Performance review}: $11$ out of $27$ got this. $10$ chose (B),
  $3$ chose (D), $2$ chose (E), $1$ chose (A).

  {\em Historical note (last time)}: $25$ out of $29$ got this. $2$ chose (D),
  $1$ each chose (B) and (E).
\item Consider the system of equations:

  \begin{eqnarray*}
    x^2 - x & = & 2 \\
    y^2 + xy & = & x + 13 \\
    z^2 & = & x^2 - y^2 \\
  \end{eqnarray*}

  What is the number of solutions to this system for real $x$, $y$,
  and $z$?

  \begin{enumerate}[(A)]
  \item $0$
  \item $2$
  \item $4$
  \item $6$
  \item $8$
  \end{enumerate}

  {\em Answer}: Option (A)

  {\em Explanation}: The solutions to the first two equations are:

  \begin{itemize}
  \item $x = 2, y = 3$
  \item $x = 2, y = -5$
  \item $x = -1, y = 4$
  \item $x = -1, y = -3$
  \end{itemize}

  In all cases, $x^2 - y^2 < 0$. Thus, there is no $z$-value that
  works in any of the cases. Thus, there are no solutions to this
  system.

  {\em Performance review}: $20$ out of $27$ got this. $5$ chose (E),
  $1$ each chose (C) and (D).

  {\em Historical note (last time)}: $22$ out of $29$ got this. $3$ chose (D),
  $2$ chose (C), $1$ each chose (B) and (E).
\end{enumerate}

\end{document}
