\documentclass[10pt]{amsart}

%Packages in use
\usepackage{fullpage, hyperref, vipul, enumerate}

%Title details
\title{Class quiz solutions: November 2: Integration}
\author{Math 152, Section 55 (Vipul Naik)}
%List of new commands

\begin{document}
\maketitle

\section{Performance review}

This quiz actually happened on November 7. $11$ people took this
quiz. The score distribution was as follows:

\begin{itemize}
\item Score of $0$: $2$ people
\item Score of $1$: $2$ people
\item Score of $2$: $4$ people
\item Score of $3$: $2$ people
\item Score of $4$: $1$ person
\end{itemize}

The mean score was $1.8$. The problem wise solutions and scores:

\begin{enumerate}
\item Option (B): $7$ people
\item Option (A): $7$ people
\item Option (C): $1$ person
\item Option (B): $5$ people
\item Option (E): Nobody
\end{enumerate}

Overall, performance on Questions 1 and 3 was not as good as it had
been last year, with Question 3 the most prominent of the lot. I think
this may have been because I explained something very similar to the
questions in class last year.
\section{Solutions}

\begin{enumerate}

\item Suppose $f$ and $g$ are both functions on $\R$ with the property
  that $f''$ and $g''$ are both everywhere the zero function. For
  which of the following functions is the second derivative {\em not
  necessarily} the zero function everywhere?

  \begin{enumerate}[(A)]
  \item $f + g$, i.e., the function $x \mapsto f(x) + g(x)$
  \item $f \cdot g$, i.e., the function $x \mapsto f(x)g(x)$
  \item $f \circ g$, i.e., the function $x \mapsto f(g(x))$
  \item All of the above, i.e., the second derivative need not be
    identically zero for any of these functions.
  \item None of the above, i.e., for all these functions, the second
    derivative is the zero function.
  \end{enumerate}

  {\em Answer}: Option (B)

  {\em Explanation}: $f$ and $g$ are both polynomial functions of
  degree at most $1$, i.e., they are both constant or linear. A sum of
  two such functions is again of the same type (i.e., constant or
  linear). A composite of two such functions is also of the same type
  (i.e., constant or linear). On the other hand, a product of two such
  functions need not be of that type, e.g., $x$ times $x$ is $x^2$.

  Note that the question can also be solved without explicitly using
  the actual form of $f$ and $g$, i.e., by just computing $(f + g)''$,
  $(f \cdot g)''$, and $(f \circ g)''$. However, this is somewhat more
  time-consuming.

  {\em Performance review}: $7$ out of $11$ got this correct. $3$
  chose (E) and $1$ chose (D).

  {\em Historical note (last year)}: $14$ out of $15$ people got this
  correct. $1$ person chose (D).

\item Suppose $f$ and $g$ are both functions on $\R$ with the property
  that $f'''$ and $g'''$ are both everywhere the zero function. For
  which of the following functions is the third derivative {\em
  necessarily} the zero function everywhere?

  \begin{enumerate}[(A)]
  \item $f + g$, i.e., the function $x \mapsto f(x) + g(x)$
  \item $f \cdot g$, i.e., the function $x \mapsto f(x)g(x)$
  \item $f \circ g$, i.e., the function $x \mapsto f(g(x))$
  \item All of the above, i.e., the third derivative is identically
    zero for all of these functions.
  \item None of the above, i.e., the third derivative is not
    guaranteed to be the zero function for any of these.
  \end{enumerate}

  {\em Answer}: Option (A)

  {\em Explanation}: Clearly, $(f + g)''' = f''' + g'''$, so if $f'''
  = g''' = 0$, then $(f + g)''' = 0$.

  Counterexamples for the others: for products, take $f(x) = g(x) =
  x^2$. For composites, the same counterexample works. What's
  different from the previous problem is that while a composite of
  linear polynomials is linear, a composite of quadratic polynomials
  has degree $4$.

  {\em Performance review}: $7$ out of $11$ got this correct. $2$
  chose (B) and $2$ chose (D).

  {\em Historical note (last year)}: $12$ out of $15$ people got this
  correct. $3$ people chose (D).

\item Suppose $f$ is a function on an interval $[a,b]$, that is
  continuous except at finitely many interior points $c_1 < c_2 <
  \dots < c_n$ ($n \ge 1)$, where it has jump discontinuities (hence,
  both the left-hand limit and the right-hand limit exist but are not
  equal). Define $F(x) := \int_a^x f(t) \, dt$. Which of the following
  {\bf is true}?

  \begin{enumerate}[(A)]
  \item $F$ is continuously differentiable on $(a,b)$ and the
    derivative equals $f$ wherever $f$ is continuous.
  \item $F$ is differentiable on $(a,b)$ but the derivative is not
    continuous, and $F' = f$ on the entire interval.
  \item $F$ has one-sided derivatives on $(a,b)$ and the left-hand
    derivative of $F$ at any point equals the left-hand limit of $f$
    at that point, while the right-hand derivative of $F$ at any point
    equals the right-hand limit of $f$ at that point.
  \item $F$ has one-sided derivatives on all points of $(a,b)$ except
    at the points $c_1, c_2, \dots, c_n$; it is continuous at all
    these points but does not have one-sided derivatives.
  \item $F$ is continuous at all points of $(a,b)$ except at the
  points $c_1, c_2, \dots, c_n$.
  \end{enumerate}

  {\em Answer}: Option (C)

  {\em Explanation}: On each interval $[c_i,c_{i+1}]$, the function is
  differentiable on the interior and has one-sided derivatives at the
  endpoints, which equal the corresponding one-sided limits of
  $f$. Piecing together the intervals, we get the desired result.

  {\em Performance review}: $1$ out of $11$ got this correct. $4$ each
  chose (B) and (E), $2$ chose (D).

  {\em Historical note (last year)}: $8$ out of $15$ people got this
  correct. $5$ people chose (B) and $2$ people chose (D).

  {\em Action point}: We'll review this in class next time.

\item (**) For a continuous function $f$ on $\R$ and a real number
  $a$, define $F_{f,a}(x) = \int_a^x f(t) \, dt$. Which of the
  following is {\bf true}?

  \begin{enumerate}[(A)]
  \item For every continuous function $f$ and every real number $a$,
    $F_{f,a}$ is an antiderivative for $f$, and every antiderivative
    of $f$ can be obtained in this way by choosing $a$ suitably.
  \item For every continuous function $f$ and every real number $a$,
    $F_{f,a}$ is an antiderivative for $f$, but it is not necessary
    that every antiderivative of $f$ can be obtained in this way by
    choosing $a$ suitably. (i.e., there are continuous functions $f$
    where not every antiderivative can be obtained in this way).
  \item For every continuous function $f$, every antiderivative of $f$
    can be written as $F_{f,a}$ for some suitable $a$, but there may
    be some choices of $f$ and $a$ for which $F_{f,a}$ is not an
    antiderivative of $f$.
  \item There may be some choices for $f$ and $a$ for which $F_{f,a}$
    is not an antiderivative for $f$, and there may be some choices of
    $f$ for which there exist antiderivatives that cannot be written
    in the form $F_{f,a}$.
  \item None of the above.
  \end{enumerate}

  {\em Answer}: Option (B)

  {\em Explanation}: The first clause: for every continuous function
  $f$ and every real number $a$, $F_{f,a}$ is an antiderivative of $f$
  is just a restatement of Theorem 5.3.5, which we covered. This
  already whittles our options down to (A) and (B). To see why (B) is
  true, imagine a situation where $F_{f,0}$ does not take all real
  values, e.g., it is an increasing function with horizontal
  asymptotes at $-1$ and $1$. We have $F_{f,0} - F_{f,a} = F_{f,0}(a)$
  (by properties of integrals). Thus, there is no way we can choose a
  value of $a$ for which $F_{f,a} = F_{f,0} + 5$.

  In more intuitive terms, the problem is that whereas for getting all
  antiderivatives, we should be able to add arbitrary constants, there
  could be cases where the definite integral between two points cannot
  be made to include the set of all constants.

  An explicit functional example (unfamiliar to you at this stage) is
  where $f(x) = 1/(x^2 + 1)$. Then $F_{f,0} = \arctan$ is bounded
  between $-\pi/2$ and $\pi/2$. Thus, say the function $20 + \arctan
  x$ cannot be realized as $F_{f,a}$ for any $a$.

  {\em Performance review}: $5$ out of $11$ got this correct. $5$
  chose (C), $1$ chose (A).

  {\em Historical note (last year)}: $5$ out of $15$ people got this
  correct. $5$ people chose (A), $4$ people chose (C), and $1$ person
  chose (D).

  {\em Action point}: We will return to this in Math 153 when we study
  improper integrals.

\item (**) Suppose $F$ is a differentiable function on an open interval
  $(a,b)$ and $F'$ is not a continuous function. Which of these
  discontinuities can $F'$ have?

  \begin{enumerate}[(A)]
  \item A removable discontinuity (the limit exists and is finite but
    is not equal to the value of the function)
  \item An infinite discontinuity (one or both the one-sided limits is
    infinite)
  \item A jump discontinuity (both one-sided limits exist and are
    finite, but not equal)
  \item All of the above
  \item None of the above
  \end{enumerate}
  
  {\em Answer}: Option (E)

  {\em Explanation}: This is hard -- perhaps part of a future
  challenge problem, so won't say more more. Briefly, the only kinds
  of discontinuities allowed are oscillatory discontinuities, of the
  kind seen with the derivative of $x^2\sin (1/x)$ at $0$.

  {\em Performance review}: Nobody got this correct. $3$ each chose
  (B), (C), (D), and $1$ chose (A).

  {\em Historical note (last year)}: Nobody got this correct.
\end{enumerate}

\end{document}