\documentclass{amsart}
\usepackage{fullpage,hyperref,vipul}
\title{Mean-value theorem, mean, median, mode, Gini}
\author{Math 152, Section 55 (Vipul Naik)}

\begin{document}
\maketitle

{\bf Corresponding material in the book}: Section 5.9

{\bf Difficulty level}: Medium.

{\bf What students should definitely get}: The statement of the
mean-value theorem for integrals, how it relates to the mean-value
theorem that we saw earlier.

{\bf What students should eventually get}: The notions of mean,
median, mode, how they differ, and how they all apply to the Gini
setup.


In this lecture (actually got split up into two lectures), we look at
the (first) mean-value theorem for integrals, which is Theorem 5.9.1
of the book. We are omitting, for now, the second mean-value theorem
for integrals (Theorem 5.9.3 of the book), though you might benefit
from reading the statement of this result in the book. After covering
the theorem, we discuss ideas of mean, median, and mode, which come up
in statistics, and how the concepts of differentiation and integration
relate to these. The discussion of concepts used in statistics is not
in your syllabus, i.e., you will not be tested on this knowledge in
your midterms or final.


\section{Statistics discussion}
\subsection*{The Gini coefficient setup}

Recall the setup that we had for the Gini coefficient. We arranged our
huge population in increasing order of income. Then, for $x \in
[0,1]$, we defined $f(x)$ as the fraction of the income earned by the
bottom $x$ fraction of the population. With reasonable assumptions and
using continuous approximations, we obtained that $f$ is continuous
and increasing, $f(0) = 0$, $f(1) = 1$, and $f(x) \le x$ for all $x
\in [0,1]$. These were the observations that were necessary for doing
the homework problems.

Another observation that was not necessary for doing the homework
problems, but is nonetheless true, is that about the significance of
$f'$. $f'(x)$ measures the fractional contribution of a person at
level $x$ (i.e., with $x$ fraction of the population earning
less). More precisely, we have:

$$f'(x) = \frac{\text{Income of person at level $x$}}{\text{Mean income}}$$

The reason why we need to normalize by mean income is that we have
normalized things to $[0,1]$. Here are some corollaries:

\begin{enumerate}
\item $f'$ is itself increasing, so $f$ is concave up. In other words,
  people at a higher income level earn more. (In a degenerate case,
  $f'$ may be constant in an interval, and $f$ linear on that
  interval. This is when multiple people earn the same income. Unless
  otherwise stated, we'll assume no degeneracy).
\item There is a $c$ such that $f'(c) = 1$. This follows from the
  mean-value theorem. In other words, there is a person who earns the
  mean income.
\end{enumerate}

\subsection*{Positions of interest}

What are all the positions $x$ and values $f(x)$ of interest? Here are
some of them:

\begin{enumerate}
\item The value $x$ such that the person at level $x$ earns the {\em
  mean income}. Mathematically, this means that $f'(x) = 1$. Note that
  the existence of this value is guaranteed by the mean-value theorem
  while its uniqueness is guaranteed by the fact that the function is
  concave up.
\item The value $1/2$. $f'(1/2)$ is the {\em median income}. $f(1/2)$
  is the fraction of total income earned by the {\em bottom half} of the
  population.
\item The {\em break-even point}, i.e., the value $x$ such that $f(x)
  = 1/2$. This is the level $x$ at which the bottom fraction earns the
  same total income as the remaining top fraction. The break-even
  point is always bigger than $1/2$ because of the concave up nature
  of the function. The income earned at this point (given by $f'(x)$)
  may also be of interest. The existence of a break-even point is
  guaranteed by the intermediate-value theorem and its uniqueness is
  guaranteed by the fact that $f$ is increasing.
\item The {\em Pareto point}, i.e., the value $x$ such that $f(x) = 1
  - x$. This is greater than $1/2$, but less than the break-even
  point. The income earned at this point may also be of interest. (You
  proved the existence and uniqueness of this point in your homework).
\end{enumerate}

\subsection*{Mean versus median? Mode? Looking at the third derivative}

Is the mean income greater than the median income? Equivalently, is
the value $x$ for which $f'(x) = 1$ greater than $1/2$? There is no
clear-cut answer. It turns out that the answer depends on whether the
distribution of incomes is skewed more toward lower incomes or toward
higher incomes.

A third statistical concept that comes up is that of the {\em
mode}. Roughly speaking, the mode is the region where there is maximum
clustering of incomes.

We thus want mathematical tools that will help answer the questions:
(a) how can we compare mean and median? (b) how can we define mode in
this situation?

The answer, interestingly, has something to do with the third
derivative.

\subsection*{The first, second and third derivatives}

You might remember that, when discussing how to graph functions to
understand them better, one useful technique we discussed was to graph
the function as well as its first and second derivative (and perhaps
higher derivatives as well). Let us put this technique to use here.

Note that the graph of $f$ measures the {\em cumulative income} earned
by certain fractions of the population. This is good for some
purposes, but for other purposes, we are interested in individual
incomes. Though the graph of $f$ contains this information, it is
hidden in that graph. To see the information on individual incomes
better, we consider the graph of $f'$.

As discussed above, the first derivative of $f$, denoted $f'$, is the
ratio of the income of the person at level $x$ to the mean income. We
know that $f'$ is a continuous and increasing function on $[0,1]$. We
also know that $f'(0) \ge 0$ and that there is some $c \in (0,1)$ such
that $f'(c) = 1$. Thus, $f'(0) \le 1$ and $f'(1) \ge 1$. We cannot say
anything more conclusive.

Thus, $f'$ is a continuous increasing function on $[0,1]$ with $0 \le
f'(0) \le 1$ and $f'(1) \ge 1$. The fact that $f'$ is increasing
corresponds to the fact that $f$ is concave up. The value $f'(1/2)$ is
the median income, and the point $c$ where $f'(c) = 1$ is the point
where the mean income is attained. We can see that the graph of $f'$,
subject to the given constraints, could be of many kinds. In
particular, the median may occur before the mean or it may occur after
the mean.

One advantage of drawing the graph of $f'$ is that, compared to the
graph of $f$, we can focus more in-depth on the way $f'$ increases. We
see that $f''$ measures the rate at which income increases (relative
to mean income) as we move from the poorest to the richest. However,
we also see that there are many unanswered questions. Where is $f'$
concave up and concave down? Where does it rise most quickly and where
does it rise most slowly? We see that the answers to these questions
depend on $f'''$. In the regions where $f'''$ is positive, $f'$ is
concave up, which means that the gain in income by moving to the right
increases as we move to the right. In the regions where $f'''$ is
negative, $f'$ is concave down, which means that the gain in income by
moving to the right decreases as we move to the right.

We see that if $f'''$ is positive throughout, that means that the
relative gain in income for every slight increase in position goes up
as we go from poorer to richer people. This means that the growth of
$f'$ is initially sluggish and picks up pace later. Such situations
typically correspond to larger values of the mean.

On the other hand, if $f'''$ is negative, that means that the relative
gain in income for every slight increase in position goes down as we
go from poorer to richer people. In other words, a small step up in
the relative ranking means more in income gain terms for poor people
than the same small step means for rich people. In this case, the
growth of $f'$ is sluggish for rich people and large for poor
people. These situations correspond to the mean occurring relatively
early.

A final question of interest is about the modal income. What is the
income range that most people have? This corresponds to:

\begin{itemize}
\item The parts where the graph of $f$ is closest to linear, i.e.,
\item The parts where the graph of $f'$ is closest to horizontal, i.e.,
\item The parts where the graph of $f''$ attains its minimum values.
\item (Probably) the parts where $f''' = 0$ and $f^{(4))} > 0$. 
\end{itemize}

In other words, the modal segment is the segment where people's income
is changing as little as possible with $x$.

\subsection*{The peril of numbers}

Before you entered the world of functions and calculus, the only type
of mathematical object you dealt with was a number. But once you
entered the world of functions and calculus, you saw yourself dealing
regularly with mathematical objects that were more complicated than
mere numbers: for instance, sets of numbers, functions, collections of
functions, points in the plane (which are ordered pairs of numbers)
and so on. Some of these objects are so complicated that it is not
possible to describe them using one or two or three numbers.

For instance, we saw that a partition of the interval $[a,b]$ is given
by an increasing finite sequence of numbers starting at $a$ and ending
at $b$. Unfortunately, the finite sequence may be arbitrarily
large. How do we compare different partitions? We saw two ideas for
comparing partitions: (a) The notion of {\em finer} partition, whereby
one refines the other. Unfortunately, given two partitions, it isn't
necessary that either one be finer than the other. (b) The notion of
the {\em norm} of a partition, which measures the size of the largest
part. We can use the norm to compare two partitions. Unfortunately, a
partition with smaller norm may not always behave like a {\em smaller}
partition as far as the upper and lower sums of a particular function
are concerned, as you discovered in the midterm.

So, one powerful idea is to use single numbers that measure {\em size}
for complicated objects and reflect some underlying reality of those
objects that is empirically useful. The drawback with that idea is
that when we look only at that single number, we lose a lot of
information about the original object. We may not be able to answer
every question that comes up.

The distribution of incomes is another such complicated construct. It
is described, as we saw, by this function $f:[0,1] \to [0,1]$. But a
function cannot be described by a single number. So, instead we ask
for single numbers that we can obtain from the function that measure
some empirically useful reality about the function. One such number,
which tries to measure the {\em extent of inequality}, is the Gini
coefficient. But one problem with the Gini coefficient is that it only
measures total inequality, and is not sensitive to inequalities within
subpopulations. For instance, if everybody earns roughly the same
income and a few people at the top earn a much much larger income, the
Gini coefficient is close to $1$, even though in some sense there is
not much inequality among most people. In other words, the Gini
coefficient is sensitive to {\em huge outliers} in the high-income
direction.

That is why it is useful to have a number of different size measures
that we can use, and to look at all of them. For instance, the
break-even point and Pareto point are useful single numbers that give
some intuition about the skew in the distribution of incomes. The
median income or the level at which the mean is attained are also
useful numbers. When you learn statistics, you will learn many other
single numbers that capture useful information about aggregates and
distributions. Keep in mind that for any single measure that you
choose, there will always be examples of distributions where that
measure does not seem to capture what you would like it to intuitively
capture.

\subsection*{Averages and compositional effects}

As some fun unwinding, here is a trick question. Suppose you have two
countries $A$ and $B$. Is it possible that the mean income in both $A$
and $B$ goes down, but the mean of no {\em individual} in either
country goes down, and in fact, there are individuals whose mean
income goes {\em up}?

Yes, it is possible. Suppose the mean income in country $A$ is $100$
money units and the mean income in country $B$ is $400$ money
units. Imagine that there is a person in country $A$ earning an income
of $200$ money units who chooses to migrate to country $B$ and gets
her income boosted to $300$ money units. Assume that nobody else
migrates, and nobody else's income changes.

The mean income of country $A$ has gone down, because a person earning
above the mean left the country. The mean income of country $B$ has
{\em also} goes down, because it just took in a person earning less
than the mean income. The net effect is that both countries see a
decline in their mean, but no individual is worse off -- and at least
one individual is better off! This is just one reason why {\em group
averages and aggregates are not always reflective of
individuals}. What we have described here is an example of a {\em
compositional effect} -- changes in group compositions affecting
averages that reflect the opposite of what is happening at the
individual level.

Of course, the group averages might still be useful in their own
right, but the statistical error would be to {\em deduce things about
individuals using group averages without taking into account
compositional effects and the fluidity of group boundaries}.

Here are some other examples of the same phenomenon:

\begin{enumerate}
\item Inter-sectoral migration: In rapidly industrializing nations
  such as China, agricultural productivity and industrial productivity
  are both rising about $5\%$ per year. Yet, overall productivity is
  rising by something like $8\%$ How is this happening? This is
  because the industrial sector is much more productive than the
  agricultural sector. As agricultural productivity increases, less
  people are needed in agriculture, and so people move from the
  (comparatively less productive) agricultural sector to the
  (comparatively more productive) industrial sector. This shifting of
  people from a less productive to a more productive sector itself
  causes an increase in productivity independent of the increase in
  productivity within each sector. Here, agriculture plays the role of
  the poorer nation $A$ and industry plays the role of the richer
  nation $B$.
\item Inter-level migration in calculus: Imagine that one of you, who
  is doing badly in the 150s, drops down to the 130s, which are a
  cakewalk for you. Then, the average mathematical skill of the 150s
  students increases, the average mathematical skill of the 130s
  student increases, yet there may probably be a net {\em decrease} in
  the overall average mathematical skill of the population, if your
  mathematical skills decline after you're no longer subjected to the
  rigors of the 150s.
\end{enumerate}


\end{document}