\subsection{The Lazard correspondence for small global class}

Suppose $L$ is a nilpotent Lie ring. We say that $L$ is a {\em Lazard
  Lie ring of small global class type} if $L$ is powered over all
primes less than {\em or equal to} its nilpotency class. {\em TODO: Is
  there a better word for this?} Particular cases of interest for this
are:

\begin{itemize}
\item The case where $L$ is a rationally powered nilpotent Lie
  ring. Rationally powered nilpotent Lie rings are called {\em Malcev
    Lie rings}.
\item The case where $L$ is a Lie ring whose additive group is a (not
  necessarily finite) $p$-group for some prime number $p$, and $L$ has
  nilpotency class strictly less than $p$. 
\end{itemize}

Analogously, define {\em Lazard Lie group of small global class type}
as a nilpotent group of class that is powered over all primes less
than or equal to its nilpotency class.

The Lazard correspondence for small global class type is a
correspondence:

Lazard Lie rings of small global class type $\leftrightarrow$ Lazard
Lie groups of small global class type

The correspondence is nice in a number of ways, including:

\begin{itemize}
\item We get element-to-element bijections between the Lazard Lie ring
  and Lazard Lie group.
\item The bijection corresponds subrings to subgroups and ideals to
  normal subgroups.
\item The bijection preserves automorphism groups. In fact, it is an
  equivalence of categories, so it preserves homomorphisms and
  endomorphisms.
\item If we have an exact sequence of Lazard Lie rings of small global
  class type, we get a corresponding exact sequence of their Lazard
  Lie groups of small global class type.
\item The element-to-element bijection is an isomorphism when
  restricted to the additive groups of abelian subrings and their
  images (abelian subgroups). In particular, it is an isomorphism
  restricted to any cyclic subring/subgroup. Thus, it also preserves
  all powering, divisibility, and torsion structure.
\item The group multiplication can be described using the Lie ring
  operations via the {\em Baker-Campbell-Hausdorff formula}.
\item The Lie operations can be expressed in terms of the group
  multiplication via {\em inverse Baker-Campbell-Hausdorff formulas}.
\end{itemize}

The Baker-Campbell-Hausdorff formula is an infinite series formula,
but it can be truncated to give a finite formula for any fixed class.

\subsection{Baker-Campbell-Hausdorff formula}

In this section, we provide a little background regarding the
Baker-Campbell-Hausdorff formula.

The Baker-Campbell-Hausdorff formula is an infinite formula of the form:

$$x + y + t_2(x,y) + t_3(x,y) + \dots+ t_n(x,y) + \dots$$

where each $t_i(x,y)$ is a sum of iterated Lie brackets involving $x$
and $y$ of homogeneous degree $i$ (in other words, each of the
iterated Lie brackets used in the summation involves a product of $i$
terms). The first few terms are described below:

\begin{tabular}{|l|l|}
  \hline
  Term & Expression\\\hline
  $t_2(x,y)$ & $\frac{1}{2}[x,y]$ \\\hline
  $t_3(x,y)$ & $ \frac{1}{12}[x,[x,y]] - \frac{1}{12}[y,[x,y]]$ \\\hline
  $t_4(x,y)$ & $-\frac{1}{24}[y,[x,[x,y]]]$\\\hline
\end{tabular}

The following are some general facts about these terms:

\begin{itemize}
\item All primes appearing as factors of the denominators of the
  expressions used for $t_n(x,y)$ are less than or equal to $n$.
\item $t_n(x,y)$ is symmetric in $x$ and $y$ for $n$ odd and is skew
  symmetric in $x$ and $y$ for $n$ even. This can be verified by
  noting that $t_n(-y,-x) = -t_n(x,y)$ (from the original defining
  properties of the formula) and then using the linearity and other
  properties of $t_n$.
\end{itemize}

We now explain how we can use these to construct the Lazard
correspondence explicitly. It is helpful to break down the Lazard
correspondence into subcorrespondences. For each positive integer $c$,
consider the correspondence:

\begin{quote}
Lie rings of nilpotency class exactly $c$ and powered over all primes
less than or equal to $c$ $\leftrightarrow$ Groups of nilpotency class
exactly $c$ and powered over all primes less than or equal to $c$
\end{quote}

The subcorrespondence works as follows. We first truncate the
Baker-Campbell-Hausdorff formula to the part of degree less than or
equal to $c$. In other words, we define:

\begin{equation*}
  x + y + \sum_{i=2}^c t_i(x,y)
\end{equation*}

We now interpret this formula as follows. Given any Lie ring $L$ of
nilpotency class exactly $c$ and powered over all primes less than or
equal to $c$, we define a group with the same underlying set as $L$
and with the group product as defined by the Baker-Campbell-Hausdorff formula. Explicitly:

\begin{equation*}
  \text{Group product } xy = x + y + \sum_{i=2}^c t_i(x,y)
\end{equation*}

Our assumptions about $L$ being powered over all primes less than or
equal to $c$, and the previously mentioned fact that the denominators
must involve primes less than or equal to the degree, tells us that
the formula on the right side must make sense.

We would need to check that this defines an associative operation and
that the operation admits an identity element and inverses.

For some purposes, it is useful to treat the Lie ring and the group as
set-theoretically different. If this is the goal, then we would
introduce, for each $x \in L$, a symbol $\exp(x)$ (alternatively
written as $e^x$) to which $x$ corresponds. Let $G$ be the set of such
symbols, with $\exp:L \to G$ the map $x \mapsto \exp(x)$. Let $\log$
be the inverse map to $\exp$.

We would then write the above formula as:

\begin{equation*}
   \exp(x)\exp(y) = \exp(x + y + \sum_{i=2}^c t_i(x,y))
\end{equation*}

Or equivalently:

\begin{equation*}
   \log(\exp(x)\exp(y)) = x + y + \sum_{i=2}^c t_i(x,y)
\end{equation*}

With some exceptions, we will generally identify the Lie ring and the
group with each other, and therefore we will {\em not} use $\log$ and
$\exp$.

The above gives a complete account of how to perform the Lazard
correspondence in the direction from Lie rings to groups. There is
also a Lazard correspondence in the reverse direction -- from groups
to Lie rings. This uses {\em inverse Baker-Campbell-Hausdorff
  formulas} that express the Lie ring operations (addition and the Lie
bracket) in terms of the group multiplication.

\subsection{Class two case reduces to the Baer correspondence}

We provide an illustration of the proof of the correspondence in class
two. In class two, the correspondence is called the {\em Baer
  correspondence}, which we discussed in detail in section
\ref{sec:baer-correspondence}. For convenience, the truncated
Baker-Campbell-Hausdorff formula and inverse Baker-Campbell-Hausdorff
formula are given below:

\begin{equation*}
  \text{Group product } xy := x + y + \frac{1}{2}[x,y]
\end{equation*}

The formula for the Lie ring addition in terms of the group
multiplication is:

\begin{equation*}
  \text{Lie ring sum } x + y := \frac{xy}{\sqrt{[x,y]_{\text{group}}}}
\end{equation*}

The formula for the Lie bracket is:

\begin{equation*}
  \text{Lie bracket} [x,y]_{\text{Lie}} = [x,y]_{\text{group}}
\end{equation*}

The equality of the group commutator and the Lie bracket is a special
feature of being in class two. It is not true for higher class.

\subsection{The class three Lazard correspondence}

{\em TODO: Insert full formulas and proofs of associativity in class
  three, including appropriate references}.

\subsection{What the Lazard correspondence is getting us towards}

In a preceding section (Section \ref{sec:group-powering}), we proved a
number of results relating to powering, divisibility, and torsion for
groups, with a special focus on the nilpotent case. In the next
section, (Section \ref{sec:lie-ring-powering}), we proved some similar
statements for Lie rings. The statements for Lie rings were a lot
easier to prove than the corresponding statements for groups.

With the statements for groups, we typically needed a proof by
induction on the lower or upper central series. The reason we needed
induction is that the group commutator map is bilinear or multilinear
only in the extreme scenario where we have as many inputs as the class
of the group. The commutator map for two elements is bilinear (i.e., a
homomorphism in each coordinate holding the other coordinate constant)
only in groups of class two. For class three or higher, it is
not. Thus, when making arguments related to powering and divisibility,
we needed to work in these extreme situations and then induct. In
contrast, any iterated Lie bracket for Lie rings is additive in each
coordinate, regardless of nilpotency assumptions.

The Lazard correspondence is convenient in so far as it provides an
alternative to the tedious inductive procedure: it allows us to direct
convert the group to a Lie ring, prove results for the Lie ring, and
then convert back. Admittedly, the results in Section
\ref{sec:group-powering} are nontrivial only in the infinite case, so
using the Lazard correspondence (if applicable) to simplify their
proofs is not of direct use for finite $p$-group theory. There are
many other results, however, that are specific to finite groups and
where the Lazard correspondence can be helpful. {\em TODO: Provide
  references}.

The proofs we use are often of an ``upper triangular'' sort in the
sense that instead of doing the proofs inductively, we could write a
complicated formula, dependent on the nilpotency class, that would
explicitly exhibit what we want. The Lazard correspondence, where
applicable, simply does the work for us beforehand by already
providing some basic formulas (such as the Baker-Campbell-Hausdorff
formula and its inverse) that can then be used to obtain alternative
derivations.

An example might help make this point clear. Theorem
\ref{powering-lcs} states that if $G$ is a nilpotent group that is
powered over a prime $p$, then all the lower central series members of
$G$ are also powered over $p$. The proof relies on an inductive
approach to constructing the $p^{th}$ root, albeit the proof as
presented is so indirect that it would require some work to
reformulate it constructively. Once we do the constructive
reformulation, it is possible to obtain an explicit formula (dependent
on the nilpotency class) for the $p^{th}$ root of $[x,y]$ in terms of
known commutators that involve $x$, $y$, and $p^{th}$ roots of
expressions in terms of $x$ and $y$.

The Lazard correspondence, if applicable, would give us an alternative
route to the formula. We first write the group commutator in terms of
the Lie operations using the Baker-Campbell-Hausdorff formula. We then
take that Lie ring expression and divide by $p$. Now, we rewrite the
Lie ring expression using the group operations by means of the inverse
Baker-Campbell-Hausdorff formula. Simplify. The final expression we
get should be the same as the expression we got using the earlier
inductive proof.

Another important thing worth noting here is that even though the
Lazard correspondence method relies crucially on the existence of the
Lazard correspondence (i.e., we are powered over all primes less than
or equal to the nilpotency class, and not just powered over $p$), the
final formula it gives us only requires powering over $p$. The
important idea is that {\em formulas obtained using the Lazard
  correspondence may have validity even for situations where the
  Lazard correspondence itself does not make sense}. Our main
application and generalization of the Lazard correspondence is of
precisely this sort.

\subsection{Group commutator formula for the Lazard correspondence}

In the usual Lazard correspondence, there is an explicit formula for
the group commutator of two elements in terms of Lie brackets. In
fact, there is an infinite series, which can be obtained from the
Baker-Cambell-Hausdorff formula series, whose truncations give the
formula for the group commutator. Let's see how this is obtained.

$$[x,y]_{\text{group}} = (yx)_{\text{group}}^{-1}(xy)$$

Inverses in the group correspond to taking the negative in the Lie
ring, so this becomes:

$$[x,y]_{\text{group}} = (-(yx))(xy)$$

We now need to apply the Baker-Campbell-Hausdorff formula to expand
each piece.

We have:

\begin{eqnarray*}
  yx = y + x + t_2(y,x) + t_3(y,x) + \dots + t_c(x,y)\\
  xy = x + y + t_2(x,y) + t_3(x,y) + \dots + t_c(x,y)\\
\end{eqnarray*}

We thus get: the following expression for $[x,y]_{\text{group}}$.

$$-(y + x + t_2(y,x) + t_3(y,x) + \dots) + (x + y + t_2(x,y) + t_3(x,y) + \dots) + t_2(-(yx),xy) + t_3(-yx,xy) + \dots + t_c(-(yx),xy)$$

Based on the symmetry and skew symmetry properties deduced in the
preceding section, we obtain the following, where $c'$ is the largest
even number less than or equal to $c$:

$$[x,y]_{\text{group}} = 2(t_2(x,y) + t_4(x,y) + \dots + t_{c'}(x,y)) + t_2(-(yx),xy) + t_3(-(yx),xy) + \dots + t_c(-(yx),xy)$$

It is also the case that $t_c(-(yx),xy) = 0$. This is because when we
expand these out, all the degree $c$ terms in the product are iterated
products with each piece equal to $(x + y)$ or $-(x + y)$, and all
higher degree terms are anyway zero. Thus, the group commutator
simplifies to:

$$[x,y]_{\text{group}} = 2(t_2(x,y) + t_4(x,y) + \dots + t_{c'}(x,y)) + t_2(-(yx),xy) + t_3(-(yx),xy) + \dots + t_{c-1}(-(yx),xy)$$

Denote this formula of $x$ and $y$ by $M_c(x,y)$.

We are now in a position to prove a lemma.

\begin{lemma}\label{commutator-denominators}
  In the formula $M_c(x,y)$ for the group commutator
  $[x,y]_{\text{group}}$ in terms of Lie ring operations (addition and
  Lie bracket) for the Lazard correspondence for $3$-local class $c$,
  all prime divisors of the denominators are less than or equal to $c
  - 1$.
\end{lemma}


\begin{proof}
  We make cases:

  \begin{itemize}
  \item $c = 2$: In this case, we can work the formula out and we get
    that $[x,y]_{\text{group}} = [x,y]$. This satisfies the condition,
    since there are no prime factors of the denominator.
  \item $c$ is odd: In this case, the formula above shows that we use
    terms only up to $t_{c - 1}$, and do not use $t_c$. Thus, we can
    only get the primes less than or equal to $c - 1$.
  \item $c$ is even and greater than $2$: In this case, $c$ is
    composite. We know from the formula that we only use primes less
    than or equal to $c$. Since $c$ is composite, this means we only
    use primes less than or equal to $c - 1$.
  \end{itemize}
\end{proof}

It is also possible to construct an infinite series expression whose
truncations give the group commutator formulas for various choices of
$3$-local nilpotency class.

The importance of these observations is as follows. We can make sense
of the formula for $M_c(x,y)$ for certain Lie rings that are {\em not}
Lazard Lie rings. Specifically, we can make sense of this formula for
Lie rings of $3$-local nilpotency class $c$ which are uniquely
divisible by primes strictly {\em less} than $c$, but not by $c$
itself. This case is of interest when $c$ itself is a prime number.

\subsection{The general Lazard correspondence allowing for $3$-local nilpotency class}

{\em TODO: Fill this in later} (carefully!)

\section{New stuff}

\subsection{Lazard correspondence up to isoclinism: the small global class case}

Suppose $L$ is a nilpotent Lie ring and $G$ is a nilpotent group such
that the following hold:

\begin{itemize}
\item $L$ is powered over all primes {\em strictly} less than its
  nilpotency class.
\item $G$ is powered over all primes {\em strictly} less than its
  nilpotency class.
\item $L$ and $G$ have the same nilpotency class.
\end{itemize}
  
A {\em Lazard correspondence up to isoclinism} between $L$ and $G$ is
the following data:

\begin{itemize}
\item An isomorphism $\zeta$ from $\operatorname{Inn}L$ to the Lazard
  Lie ring of $\operatorname{Inn}(G)$.
\item An isomorphism $\varphi$ from the Lazard Lie group of $L'$ to
  $G'$.
\item A compatibility condition between the Lie bracket map for $L$
  and the commutator map for $G$, described below.
\end{itemize}

Denote by $\gamma_G$ the map $\operatorname{Inn}(G) \times
\operatorname{Inn}(G) \to G'$ obtained from the group commutator map.

We can now define a function $\hat{M}_c:\operatorname{Inn}(L) \times
\operatorname{Inn}(L) \to L'$ as follows: for elements $x,y$ of
$\operatorname{Inn}(L)$, first lift them to $L$, then apply $M_c$ to
the lifts, and get an element of $L'$. Recall that $M_c$ is the
formula that outputs the putative group commutator (which doesn't
actually exist!) in terms of Lie brackets. The compatibility condition
now states that:

$$\varphi \circ \hat{M}_c = \gamma_G \circ (\zeta \times \zeta)$$

Explicitly:

$$\varphi(\hat{M}_c(x,y)) = \gamma_G(\zeta(x),\zeta(y))$$

Note that the leeway of being off by one in the formula for $M_c$
(proved in Lemma \ref{commutator-denominators}) is crucial for us to
make sense of this: we are guaranteed that the formula makes sense for
$L$ because we are guaranteed unique divisibility by all primes less
than $c$.

\subsection*{The edge case of genuine interest}

The only subcase of genuine interest here is where the nilpotency
class $c$ is a prime number, which we call $p$. In this case, what we
have is a potential for a Lazard correspondence up to isoclinism
between Lie rings of class $p$ that are uniquely divisible by all
primes less than $p$, and groups of class $p$ that are uniquely
divisibly by all primes less than $p$. The interesting case is the
edge case that would not be covered by the ordinary Lazard
correspondence. For finite Lie rings and groups, this involves:

finite $p$-Lie rings of class exactly $p$ up to isoclinism
$\leftrightarrow$ finite $p$-groups of class exactly $p$ up to
isoclinism

\subsection{Basic facts about the Lazard correspondence up to isoclinism}

We begin with a simple lemma:

\begin{lemma}
  Suppose $L_1$ and $L_2$ are isoclinic Lie rings and $G$ is a
  group. Then, $L_1$ is in Lazard correspondence up to isoclinism with
  $G$ if and only if $L_2$ is in Lazard correspondence up to
  isoclinism with $G$.
\end{lemma}

Other results to be lemmized:

\begin{itemize}
\item If $G_1$ and $G_2$ are isoclinic groups, then a Lazard Lie ring
  $L$ being in Lazard correspondence up to isoclinism with $G_1$ is
  equivalent to $L$ being in Lazard correspondence up to isoclinism
  with $G_2$: The reason is that we can ``compose'' an isoclinism of
  Lie rings with a Lazard correspondence up to isoclinism, and
  conversely, we can take the ``quotient'' of two Lazard
  correspondences up to isoclinism to get isoclinisms of Lie rings.
\item If $L_1$ and $L_2$ are Lie rings that are both Lazard
  correspondent up to isoclinism to a group $G$, then $L_1$ and $L_2$
  are isoclinic to each other as Lie rings: The reason is that we can
  compose the Lazard correspondence up to isoclinism from $L_1$
  to $G$ with the inverted Lazard correspondence from $L_2$ to $G$ to
  get an isoclinism of Lie rings from $L_1$ to $L_2$.
\item If $G_1$ and $G_2$ are groups that are both Lazard correspondent
  up to isoclinism to a Lie ring $L$, then $G_1$ and $G_2$ are
  isoclinic to each other as groups: The reason is that we can compose
  the inverted Lazard correspondence up to isoclinism from $L$ to
  $G_1$ with the Lazard correspondence up to isoclinism from $L$ to
  $G_2$ and get an isoclinism of groups from $G_1$ to $G_2$.
\item If $L$ is a Lazard Lie ring and $G$ is its Lazard Lie group,
  then $L$ is also in Lazard correspondence up to isoclinism with $G$:
  We take the isomorphisms induced by the Lazard correspondence, and
  note that they definitionally satisfy the conditions for giving a
  Lazard correspondence up to isoclinism.
\end{itemize}

{\em TODO: I will insert proofs of (some of) these later}.

In other words, the Lazard correspondence up to isoclinism is only a
correspondence up to isoclinism on both sides. Explicitly, it is a
correspondence between equivalence classes up to isoclinism of Lie
rings on the one hand and of groups on the other.

Further, in cases where the ordinary Lazard correspondence is
applicable, the Lazard correspondence up to isoclinism simply clubs
together equivalence classes of groups up to isoclinism on the one
hand and equivalence classes of Lie rings up to isoclinism on the
other hand.

\subsection{The hard task: what we need to show}

We have so far established a {\em potential} for a
correspondence. What we need to show is that this potential is actually
realized. We need to show two things:

\begin{itemize}
\item {\em Existence from Lie rings to groups}: For any Lie ring $L$
  of class $c$ that is powered over all primes strictly less than $c$,
  there is a group $G$ of class $c$ that is powered over all primes
  strictly less than $c$, and a Lazard correspondence up to isoclinism from $L$ to $G$.
\item {\em Existence from groups to Lie rings}: For any group $G$ of
  class $c$ that is powered over all primes strictly less than $c$,
  there is a Lie ring $L$ of class $c$ that is powered over all primes
  strictly less than $c$, and a Lazard correspondence up to isoclinism
  from $L$ to $G$.
\end{itemize}

In the next section, we prove these two facts. It turns out that we do
not need to do any heavy lifting for the proof, because all the heavy
lifting is already done by cohomology theory. Specifically, it will
turn out that the short exact sequence for second cohomology arising
from the universal coefficients theorem is exactly what we need to
demonstrate our result.

\section{The Lazard correspondence and extension theory}

The Lazard correspondence between groups and Lie rings, where it does
apply, is very nice. It establishes an {\em equivalence of categories}
that allows us to easily go back and forth between standard concepts
in the language of groups and of Lie rings. Our interest here is in
figuring out how this relates to cohomology theory.

\subsection{In the good Lazard case}

Suppose $c$ is a natural number. Suppose $G$ is a group of nilpotency
class at most $c - 1$ that is powered over all primes less than or
equal to $c$. Note in particular that $G$ is powered over all primes
dividing $c$. Suppose $A$ is an abelian group that is also powered
over all primes less than or equal to $c$.

We can consider all the group extensions with central subgroup $A$ and
quotient group $G$. Each such extension group is of class at most
$c$. Further, by Lemma \ref{powering-extension-group}, each such
extension group is also powered over all primes less than or equal to
$c$. Thus, each such extension group is a Lazard Lie group of small
global class type. In other words, all the elements of $H^2(G;A)$
correspond to Lazard Lie groups of small global class type.

$G$ itself is a Lazard Lie group of small global class type. Thus, it
has a Lazard Lie ring. Suppose that $L$ is the Lazard Lie ring of
$G$. View $A$ as an abelian Lie ring. We can then consider the second
cohomology group $H^2_{\text{Lie}}(L;A)$ of Lie ring extensions with
$A$ as the central subring and $L$ as the quotient Lie ring. Each
extension again satisfies the conditions for being a Lazard Lie ring.

The following observations are easy to check:

\begin{itemize}
\item The groups $H^2_{\text{Lie}}(L;A)$ are $H^2(G;A)$ are isomorphic
  as groups. In fact, there is a canonical isomorphism between them.
\item Under this isomorphism, the element-to-element bijection matches
  each Lie ring extension to its Lazard Lie group, with the Lazard
  correspondence compatible with the short exact sequences.
\end{itemize}

\subsection{The case we are interested in}

Suppose $p$ is a prime number. Suppose $G$ is a group of nilpotency
class exactly $p - 1$ that is powered over all primes {\em strictly
  less than} $p$. Suppose $A$ is an abelian group that is also powered
over all primes strictly less than $p$.

In this case, the group extensions that correspond to elements of
$H^2(G;A)$ have nilpotency class $p - 1$ or $p$. If the class is $p -
1$, the extension group is guaranteed to be a Lazard Lie group. If the
class is $p$, however, the extension group is not guaranteed to be a Lazard Lie
group.

$G$ is still a Lazard Lie group, so we can take its Lazard Lie
ring. Call this Lie ring $L$. The Lie ring extensions that correspond
to elements of $H^2_{\text{Lie}}(L;A)$ may give Lie rings of class $p
- 1$ or $p$. If the class is $p - 1$, the extension Lie ring is
guaranteed to be a Lazard Lie ring. If the class is $p$, however, the
extension Lie ring is not guaranteed to be a Lazard Lie ring.

Although it will turn out that the groups $H^2(G;A)$ and
$H^2_{\text{Lie}}(L;A)$ are still isomorphic, it is no longer the case
that there exists a canonical isomorphism between $H^2(G;A)$ and
$H^2_{\text{Lie}}(L;A)$. What {\em is} true is that there exists a
canonical isomorphism between the ``extensions up to isoclinism.'' To
understand what that means, we now turn to the cohomology
interpretation of extensions up to isoclinism.

In order to prove this, we need to take a short detour into the theory
of tensor products and exterior products.

\newpage

\section{Exterior and tensor products}

\subsection{Powering is inherited by Schur multiplier, Schur covering group, and exterior square}

\begin{lemma}\label{schurcoverpipower}
  Suppose $G$ is a $\pi$-powered nilpotent group of nilpotency class
  $c$. The following are true.

  \begin{enumerate}
  \item The Schur multiplier $M(G)$ of $G$ is $\pi$-powered.
  \item Every Schur covering group of $G$ is $\pi$-powered.
  \item The exterior square of $G$ is $\pi$-powered.
  \end{enumerate}
\end{lemma}

\begin{proof}
  Proof of (1): This follows from \cite{HiltonMislinRoitberg}, Theorem
  2.9. {\em TODO: Add in more details about how to translate between
    that language and this}.

  Proof of (2): Any Schur covering group arises as a central extension of $G$
  with central subgroup $M(G)$ and quotient group $G$. Thus, by Lemma
  \ref{powering-extension-group}, the Schur covering group is also
  $\pi$-powered.

  Proof of (3): The exterior square of $G$ is isomorphic to the derived
  subgroup of any Schur covering group of $G$. By (2) and Theorem
  \ref{powering-lcs}, the exterior square of $G$ is also
  $\pi$-powered.
\end{proof}

Note that the corresponding results are not true for $\pi$-divisible
groups. Specifically, the group $G = UT(3,\Q)/\Z$ (where the $\Z$
being factored out is central) is divisible by all primes, since
$UT(3,\Q)$ is divisible by all primes. However, the Schur multiplier
of this group is $\Z$, which is not divisible by any prime.

Similarly, we have:

\begin{lemma}\label{schurcoverpipower-lie}
  Suppose $L$ is a $\pi$-powered nilpotent Lie ring of nilpotency
  class $c$. The following are true.

  \begin{enumerate}
  \item The Schur multiplier $M(L)$ of $L$ is $\pi$-powered.
  \item Every Schur covering group of $L$ is $\pi$-powered.
  \item The exterior square of $L$ is $\pi$-powered.
  \end{enumerate} 
\end{lemma}

\begin{proof}
  Proof of (1): This follows from the way Lie ring cohomology is defined.

  Proof of (2): Similar to proof of (2) in the preceding lemma.

  Proof of (3): Similar to proof of (3) in the preceding lemma.
\end{proof}

\subsection{Rationally powered case}

For this subsection, we restrict attention to the case where $G$ is a
rationally powered nilpotent group. The advantage of this is that all
groups involved will be Malcev Lie groups and hence Lazard Lie
groups. The complications arising in other situations will be
discussed in the next section.

Suppose $G$ is a rationally powered nilpotent group and $L$ is its
Lazard Lie ring. $L$ is a nilpotent $\Q$-Lie algebra. Notationally, $L
= \log G$ and $G = \exp L$.

\begin{itemize}
\item Consider the category of central extensions of $G$ with
  homoclinism {\em where the extension group is also rationally
    powered nilpotent}. This is a subcategory of the category of all
  central extensions of $G$ with homoclinism. Any Schur covering group
  of $G$ would be an initial object in the bigger category, hence also
  in the subcategory if it is in the subcategory. By Lemma
  \ref{schurcoverpipower}, the Schur covering groups are in the
  subcategory.
\item Corresponding, consider the category of central extensions of
  $L$ with homoclinism {\em where the extension Lie ring is also
  rationally powered}. By reasoning analogous to the group case, and
  using Lemma \ref{schurcoverpipower-lie}, we obtain that the Schur
  covering Lie rings of $L$ are initial objects in this category.
\item The two categories under consideration are in Lazard
  correspondence with each other, hence their sets of initial objects
  are also in Lazard correspondence. In particular, this means that if
  $K$ is a Schur covering group for $G$, then $\log K$ is a Schur
  covering Lie ring for $L$. Similarly, if $M$ is a Schur covering Lie
  ring for $L$, then $\exp M$ is a Schur covering group for $G$.
\end{itemize}

\subsection{The free powered case}

For $\pi$ a set of primes and $d$ and $c$ positive integers, denote by
$FNG(\pi,d,c)$ the free $\pi$-powered class $c$ group on $d$
generators. {\em This is not standard notation, and we are using it
  only to make our statements in this section compact.} Similarly,
denote by $FNL(\pi,d,c)$ the free $\pi$-powered class $c$ Lie ring on
$d$ generators.

Suppose $\pi$ is the set of all primes $\le c$. Let $\mathcal{L} =
FNL(\pi,d,c)$ and $\mathcal{G} = FNG(\pi,d,c)$. Denote by $\Q
\mathcal{L}$ and $\Q \mathcal{G}$ respectively the minimal rationally powered
Lie ring and group containing $\mathcal{L}$ and $\mathcal{G}$ respectively. Then, the
following are true.

\begin{lemma}
  With notation as in the preceding paragraph, the following are true:

  \begin{enumerate}
  \item $\Q \mathcal{L}$ is the free rationally powered Lie ring on $d$
    generators and with class $c$.
  \item $\Q \mathcal{G}$ is the free rationally powered group on $d$ generators
    and with class $c$.
  \item $\Q \mathcal{L}$ and $\Q \mathcal{G}$ are in Lazard correspondence with the Lazard
    correspondence inducing a bijection between their respective
    freely generating sets.
  \item $\mathcal{L}$ and $\mathcal{G}$ are in Lazard correspondence, and the inclusion of
    $\mathcal{L}$ in $\Q \mathcal{L}$ is Lazard correspondent with the inclusion of $G$ in
    $\Q \mathcal{G}$.
  \end{enumerate}
\end{lemma}

\begin{proof}
  The proofs of (1), (2) and (3) are straightforward. {\em TODO: Quote
    standard references if possible}.

  Proof of (4): The key point here is that the
  Baker-Campbell-Hausdorff formula uses denominators all within
  $\pi$. If the initial element of $\Q L$ is in $L$, that means it can
  be obtained from the freely generating set through $\pi$-powered Lie
  ring operations. Applying the Baker-Campbell-Hausdorff formula keeps
  us within $\pi$-powered operations, so we land within $G$.
\end{proof}

We prove another lemma before proceeding to the main theorem.

\begin{lemma}
  With notation as above, the following are true:

  \begin{enumerate}
  \item $\Q(M(L)) = M(\Q L)$. In other words, taking the Schur
    multiplier commutes with taking the minimal rationally powered
    completion.
  \item $\Q(M(G)) = M(\Q G)$.
  \item $\Q(L \wedge L) = (\Q L) \wedge (\Q L)$.
  \item $\Q(G \wedge G) = (\Q G) \wedge (\Q G)$.
  \item If $I$ is an ideal in $L$ and $H$ is the corresponding normal
    subgroup of $G$, then $\Q(I \wedge L) = (\Q I) \wedge (\Q L)$ and
    $\Q(H \wedge G) = (\Q H) \wedge (\Q G)$.
  \end{enumerate}
\end{lemma}

\begin{proof}
  Proof of (2): This follows from \cite{HiltonMislinRoitberg}, Theorem
  2.9. Note that in the language of that text, we are localizing at
  the empty set.

  Proof of (1): This is the Lie algebra analogue to (2).

  
\end{proof}
\begin{theorem}
  With the notation as above, the following are true:

  \begin{enumerate}
  \item The Schur multipliers $M(L)$ and $M(G)$ are isomorphic, with a
    canonical isomorphism induced by the Lazard correspondence between
    $L$ and $G$.
  \item The exterior squares $L \wedge L$ and $G \wedge G$ are
    isomorphic, with a canonical isomorphism induced by the Lazard
    correspondence between $L$ and $G$.
  \item Suppose $I$ is an ideal of $L$ and $H$ is the corresponding
    normal subgroup of $G$. Then, $I \wedge L$ is in Lazard
    correspondence with $H \wedge G$, and the induced maps $I \wedge L
    \to L \wedge L$ and $H \wedge G \to G \wedge G$ are also in Lazard
    correspondence.
  \end{enumerate}
\end{theorem}

Finally, we have this theorem:

\begin{theorem}
  Suppose $L$ and $G$ are as above, with $I$ an ideal in $L$ and $H$
  the Lazard correspondent normal subgroup of $G$. The, the following
  are true:

  \begin{enumerate}
  \item $(L/I) \wedge (L/I) \cong (L \wedge L)/(\operatorname{Im}(I \wedge L))$. 
  \item $(G/H) \wedge (G/H) \cong (G \wedge G)/(\operatorname{Im}(H \wedge G))$.
  \item $(L/I) \wedge (L/I)$ and $(G/H) \wedge (G/H)$ are Lazard
    correspondent.
  \end{enumerate}
\end{theorem}

\begin{proof}
  Parts (1) and (2) follow from the definition.

  (3) follows from (1) and (2) and the preceding theorem.
\end{proof}

\subsection{The final result that we want}

The result that we want to show follows from the fact that the group
that we are trying to get can be written in the form $G/H$ and the Lie
ring in the form $L/I$ for the preceding theorem.

\newpage

\subsection{Extensions up to isoclinism: theory}

The universal coefficients theorem gives a short exact sequence for
computing the second cohomology group:

$$0 \to \operatorname{Ext}^1_{\mathbb{Z}}(G^{\text{ab}},A) \to H^2(G;A) \to \operatorname{Hom}(H_2(G;\mathbb{Z}),A) \to 0$$

There is an analogous short exact sequence for Lie ring extensions:

$$0 \to \operatorname{Ext}^1_{\mathbb{Z}}(L^{\text{ab}},A) \to H^2_{\text{Lie}}(L;A) \to \operatorname{Hom}(H_{2,\text{Lie}}(L;\mathbb{Z}),A) \to 0$$

The following are obvious from the definition of cohomology: 

\begin{itemize}
\item The fibers of the map $H^2(G;A) \to
  \operatorname{Hom}(H_2(G;\mathbb{Z}),A)$ correspond to equivalence
  classes up to isoclinism of group extensions with central subgroup
  $A$ and quotient group $G$.
\item The fibers of the map $H^2_{\text{Lie}}(L;A) \to
  \operatorname{Hom}(H_{2,\text{Lie}}(L;\mathbb{Z}),A)$ correspond to
  equivalence classes up to isoclinism of Lie ring extensions with
  central subring $A$ and quotient Lie ring $L$.
\end{itemize}

\subsection{Extensions up to isoclinism and the Lazard correspondence}

{\em TODO: Provide background for the terminology here, some of which
is not too standard. Relevant references}:

\begin{itemize}
\item R. K. Dennis. {\em In search of new homology functors having a
  close relationship to K theory}. Cornell University Preprint.
\item Van Kampen theorems for diagrams of spaces by Ronald Brown and
  Jean-Louis Loday, Topology, Volume 26,Number 3, (Year 1987)
\item The non-abelian tensor product of finite groups is finite by
  Graham Ellis, Journal of Algebra, ISSN 00218693, Volume 111, Page 203
  - 205(Year 1987)
\item A non-abelian tensor product of Lie algebras by Graham J. Ellis,
  Glasgow Journal of Math, Volume 33, Page 101 - 120(Year 1991)
\end{itemize}

\begin{lemma}
  Suppose $L$ is a Lazard Lie ring with the property that any central
  extension (with what kind of central subgroup?) with quotient $L$ is
  a Lazard Lie ring. Then, if $G$ is the Lazard Lie group of $L$, we
  have the following:

  \begin{itemize}
  \item There is a canonical isomorphism between $H_2(L;\mathbb{Z})$
    and $H_2(G;\mathbb{Z})$.
  \item There is a canonical Lazard correspondence beween $L \wedge
    L$ and $G \wedge G$ that relates the maps $L \wedge L \to [L,L]$
    and $G \wedge G \to [G,G]$ as Lazard correspondent.
  \end{itemize}  
\end{lemma}

\begin{proof}
  {\em TODO: This proof is correct but needs some more elaboration to
  be accessible}.

  For any abelian group $A$, we have a Lazard correspondence:

  Lie rings that are central extensions with central subring $A$ (as
  an abelian Lie ring) and quotient $L$ $\leftrightarrow$ Groups that
  are central extensions with central subgroup $A$ and quotient $G$

  The Schur multiplier is defined in terms of a universal property
  with respect to abelian groups $A$ that arise as the base of such
  central extensions, hence, we get that the Schur multipliers are
  Lazard correspondent, and since they are abelian, they are
  isomorphic as abelian groups.

  Thus, we also get Lazard correspondences between the Schur covering
  Lie rings of $L$ and the Schur covering groups of $G$. We can use
  any of these to obtain the Lazard correspondence between $L \wedge
  L$ (which is the derived subring of any Schur covering Lie ring) and
  $G \wedge G$ (which is the derived subgroup of the corresponding
  Schur covering group). Explicitly, the following short exact
  sequences are in Lazard correspondence:

  Short exact sequence of Lie rings:

  $$0 \to H_2(L;\mathbb{Z}) \to L \wedge L \to [L,L] \to 0$$

  Short exact sequence of groups:

  $$1 \to H_2(G;\mathbb{Z}) \to G \wedge G \to [G,G] \to 1$$
\end{proof}

We next show that the condition that the extensions be Lazard is not
necessary.

\begin{lemma}
  If $L$ is the Lazard Lie ring of $G$, then $L \wedge L$ is the
  Lazard Lie ring of $G \wedge G$, and this is compatible with the Lie
  bracket/commutator homomorphism to $[L,L]$ (respectively
  $[G,G]$). In other words, the Lazard correspondence relates the map
  $L \wedge L \to [L,L]$ to the map $G \wedge G \to [G,G]$.
\end{lemma}

\begin{proof}
  {\em TODO: Improve this. This is the heart of the
  proof, so I want to do it well}.

  The previous lemma already establishes this result in the case that
  the extensions themselves are Lazard Lie rings and groups
  respectively. We would like to show that that condition is
  unnecessary.

  We note that if we start with $L$ a free Malcev $\Q$-Lie algebra,
  then the previous lemma applies. However, the algebraic expressions
  that we work out for free Malcev $\Q$-Lie algebras apply {\em
  everywhere that they make sense}, so this shows the result
  everywhere.
\end{proof}

Next:

\begin{lemma}
  If $L$ is the Lazard Lie ring of a Lazard Lie group $G$, then their
  Schur multipliers are canonically isomorphic as abelian groups,
  i.e.,:

  $$H_2(G;\mathbb{Z}) \cong H_{2,\operatorname{Lie}}(L;\mathbb{Z})$$
\end{lemma}

\begin{proof}
  $H_2(G;\mathbb{Z})$ is the kernel of the natural map $G \wedge G \to
  [G,G]$ given by the commutator operation, while
  $H_{2,\operatorname{Lie}}(L;\mathbb{Z})$ is the kernel of the
  natural map $L \wedge L \to [L,L]$ given by the Lie bracket
  operation. By the lemma above, these maps are Lazard correspondent,
  so their kernels are Lazard correspondent. But the kernels are {\em
    abelian}, so they are isomorphic abelian groups. This proves our
  result.
\end{proof}

Finally:

\begin{lemma}
  If $L$ is the Lazard Lie ring of a Lazard Lie group $G$, and $A$ is
  any abelian group, there is a canonical isomorphism (and hence, a
  set-theoretic correspondence) between 

  \begin{itemize} 
  \item the set of Lie ring central extensions up to isoclinism of
    extensions with central subring $A$ and quotient ring $L$,
    identified with the group
    $\operatorname{Hom}(H_2(L;\mathbb{Z}),A)$; and
  \item the set of group central extensions up to isoclinism with
    central subgroup $A$ and quotient group $G$, identified with the
    group $\operatorname{Hom}(H_2(L;\mathbb{Z}),A)$.
  \end{itemize}
\end{lemma}

\begin{proof}
  This follows from the preceding lemma.
\end{proof}

\subsection{Moving from extensions up to isoclinism to all extensions}

The preceding sections lay out a theory of how to identify extensions
up to isoclinism at the Lie ring level with extensions up to
isoclinism at the group level. We now turn to the question of how to
identify groups up to isoclinism with Lie rings up to isoclinism.

The idea is to simply look at the group as an extension with its full
center as the central subgroup and its inner automorphism group as the
corresponding quotient group.

\section{The adjoint action}

\subsection{The Lazard correspondence situation}

Suppose $G$ is a Lazard Lie group and $L$ is its Lazard Lie
ring. Then, there is a set bijection between $G$ and $L$, and hence
the action of $G$ on itself by conjugation gives rise to an action of
$G$ on $L$ by automorphisms. There is another way to understand this action.

Suppose $x \in L$. Then, $\ad x$ is an inner derivation of
$L$. Consider the function $\exp(\ad x)$ defined as the exponential of
$\ad x$. Note that this is a finite sum because of the nilpotency
conditions on $L$. Then, it turns out that $\exp(\ad x)$ is an
automorphism and in fact coincides with the automorphism obtained as
conjugation by the group element corresponding to $x$. This was proved
in Lemma 3.3 of Glauberman's {\em Partial extensions} paper. {\em
TODO: Cross-check, insert original reference}.

\subsection{Our generalization}

We need to construct an action of $G$ on $L$ by Lie ring automorphisms
that mimics the ``action by conjugation'' that we used in the
past. Note that there is no element-to-element correspondence between
the elements of $G$ and $L$, so the naive method doesn't work. A
slight modification does. Note that it suffices to construct an action
of $G/Z(G)$ on $L$ by Lie ring automorphisms.

For any element of $G/Z(G)$, consider the corresponding element of
$L/Z(L)$, and let $x$ be a representative of it in $L$. Then, the
automorphism we seek is $\exp(\ad x)$.

To see that this is an automorphism, we first state the following
lemma, which appears as Proposition 2.5(b) in {\em Limits of abelian
subgroups of finite $p$-groups} by Alperin and Glauberman.

\begin{lemma}
  Suppose $R$ is a non-associative (i.e., not necessarily associative)
  ring and $d$ is a derivation on $R$ and $n$ is a natural number such
  that:
  
  \begin{itemize}
  \item $d^n$ is the zero map.
  \item $d^i(u)d^j(v) = 0$ for all $u,v \in R$ and $i + j \ge n$.
  \item $R$ is uniquely divisible by all numbers $1,2,\dots,n-1$.
  \end{itemize}
  
  Then we can define:

  $$\exp(d) := \sum_{k=0}^{n-1} \frac{d^k}{k!}$$
  
  Then:

  \begin{enumerate}
  \item $\exp(d)$ is an automorphism.
  \item $\exp(-d)$ is also an automorphism and is the inverse of
  $\exp(d)$.
  \end{enumerate}
\end{lemma}

\begin{proof}
  {\em TODO: Insert proof when expanding this into a thesis document}.
\end{proof}

We now use the lemma to obtain that $\exp(\ad x)$ is an automorphism
in the case of Lie rings that are just one class higher than those
eligible for the Lazard correspondence.

\begin{lemma}
  Suppose $L$ is a Lie ring such that $L$ is uniquely divisible by all
  primes less than or equal to $c - 1$, and $L/Z(L)$ is a Lazard Lie
  ring for class $c - 1$ (i.e., its $3$-local nilpotency class is at
  most $c - 1$, and it is uniquely divisible by all primes less than
  or equal to $c - 1$). Then, for any $x \in L$, $\exp(\ad x)$ is an
  automorphism of $L$.
\end{lemma}

\begin{proof}
  Note that since $L/Z(L)$ has $3$-local nilpotency class at most $c -
  1$, this means that $L$ has $3$-local nilpotency class at most
  $c$. Hence, any Lie product of length $c + 1$ or higher that
  involves at most three distinct elements must be zero.

  We verify that $\ad x$ satisfies the conditions for being the
  derivation $d$ in the preceding lemma, setting $n = c$:

  \begin{itemize}
  \item $(\ad x)^c = 0$: To see this, note that $(\ad x)^cy$ is a Lie
    product of length $c + 1$ involving at most two distinct elements,
    hence is zero for all $y \in L$.
  \item $(\ad x)^i(u)(\ad x)^j(v) = 0$ for all $i,j$ with $i + j \ge
    c$: This is a Lie product of length $c + 2$ involving at most
    three distinct elements, hence is zero for all $u,v \in L$.
  \item $L$ is uniquely divisible by all numbers $1,2,\dots,n - 1$:
    This is given to us.
  \end{itemize}

  Thus, the conditions apply, and we get an automorphism as desired.
\end{proof}

Our next claim makes this a group action.

\begin{lemma}
  Suppose $L$ is a Lie ring such that $L$ is uniquely divisible by all
  primes less than or equal to $c - 1$, and $L/Z(L)$ is a Lazard Lie
  ring for class $c - 1$ (i.e., its $3$-local nilpotency class is at
  most $c - 1$, and it is uniquely divisible by all primes less than
  or equal to $c - 1$). Let $K$ be the Lazard Lie group of
  $L/Z(L)$. For an element $a \in K$, define the automorphism induced
  by $a$ as follows: let $x \in L$ be such that its image mod $Z(L)$
  is Lazard correspondent to $a$. Then, the automorphism corresponding
  to $a$ is $\exp(\ad x)$.

  This defines a group action of $K$ on $L$.
\end{lemma}

\begin{proof}
  The preceding lemma already showed that each $\exp(\ad x)$ is an
  automorphism. What we need to show next is that the composition rule
  for the automorphisms is compatible with the multiplication in $K$.

  To see this, we need to imagine everything happening inside the ring
  $\operatorname{End}_{\Z}(L)$ of additive group endomorphisms of
  $L$. By one of the original interpretations of the
  Baker-Campbell-Hausdorff formula, it is true that in an associative
  ring, if $a,b$ are nilpotent elements, then:

  $$\exp(a)\exp(b) = \exp(\text{Baker-Campbell-Hausdorff formula of } a,b)$$
  
  The Baker-Campbell-Hausdorff formula is how we define multiplication
  in $K$, so this is essentially the proof. {\em TODO: Make this
  clearer}. {\em TODO: Add this original interpretation of the
  Baker-Campbell-Hausdorff formula at the point of original
  introduction of the formula}
\end{proof}

From all this, we obtain that if a Lie ring $L$ and a group $G$ are
Lazard correspondent up to isoclinism, then we have a homomorphism
(note that $G/Z(G)$ is the group $K$ of the preceding lemma:

$$G/Z(G) \to \operatorname{Aut}(L)$$

and hence also a homomorphism:

$$G \to \operatorname{Aut}(L)$$

\begin{lemma}
  Suppose a finite Lie ring $L$ and a finite group $G$ are Lazard
  correspondent up to isoclinism. Suppose $c$ is a positive
  integer. Let $m_1$ be the number of orbits in $L$ under the
  $G$-action that have size $c$, and let $m_2$ be the number of
  conjugacy classes in $G$ of size $c$. Then, $m_1$ is nonzero if and
  only if $m_2$ is nonzero, and if so, $m_1/m_2 = |L|/|G|$.
\end{lemma}

\begin{proof}
  {\em TODO: insert proof}.
\end{proof}

\newpage

\section{Some examples}

\subsection{$2$-groups of class two}


\newpage
\section{Applications}

\subsection{Conversion of questions about abelian subgroups of maximum order}

In Glauberman's papers ({\em TODO: insert references}) on finding
abelian subgroups of maximum order, one of the techniques used was to
pass from a group $G$ to its Lazard Lie ring. However, this would only
give results for $p$-groups of class up to $p - 1$. In order to give
results for class $p$, it is necessary to consider separately the Lie
rings for $G/Z(G)$ and $G'$ and consider the commutator map between
them.

The isoclinism results help shed new light on these constructions by
Glauberman. Specifically, they guarantee that for a $p$-group of class
$p$, it will always be possible to find a Lie ring $L$ such that
$L/Z(L)$ is the Lazard Lie ring for $G/Z(G)$ and $L'$ is the Lazard
Lie ring for $G'$, with a compatibility between the Lie bracket map
and the comutator map via the group commutator formula. $L$ is not, of
course, the Lazard Lie ring for $G$ because that concept doesn't make
sense. However, abelian subrings of maximum order in $L$ still
correspond to abelian subgroups of maximum order in $G$, because these
both contain the respective centers. Our new approach does not give
any result stronger than those found in Glauberman's papers (at least
{\em prima facie}). However, they do help simplify some of the
presentations of results.

\subsection{Kirillov orbit method}

This builds off of the paper:

{\em The orbit method for profinite groups and a p-adic analogue of
Brown's theorem} by Mitya Boyarchenko and Maria Sabitova, Israel
Journal of Math, Volume 165, Page 67 - 91(Year 2008).

In their paper, Boyarchenko and Sabitova describe the abstract
Kirillov orbit method as a method to compute the degrees of
irreducible representations for a finite $p$-group $G$ of $3$-local
class less than $p$ (their method is more general, but we'll restrict
attention to this case for now).

The method is as follows. Let $L$ be the Lazard Lie ring of $G$. We
have an action of $G$ on itself by conjugation, which, under the set
identification of $G$ on $L$, gives an action of $G$ on $L$. Note that
the orbits under this action correspond to the conjugacy classes of
$G$.

Instead of looking at these orbits, we consider the induced action of
$G$ on the Pontryagin dual of $L$ as an abelian group, i.e., the
one-dimensional characters of $L$. Note that if $L$ is a modulo over
$\mathbb{Z}/p^k\mathbb{Z}$, this can be thought of (up to suitable
choice of identification with roots of unity) as the dual module over
that ring. Now, look at the orbits for the action of $G$ on this
dual. The following are true:

\begin{itemize}
\item There is a canonical bijection between these orbits and the
  irreducible representations of $G$ over the complex numbers.
\item The size of each orbit is the square of the corresponding degree
  of irreducible representation.
\item There is a formula computing the character of the irreducible
  representation. This is called the {\em Kirillov character formula}.
\end{itemize}

In a previous section, we noted that in the situation of a Lazard
correspondence up to isoclinism, we still have an action that
generalizes the action by conjugation:

$$G \to \operatorname{Aut}(L)$$

We showed earlier that the orbit sizes in $L$ are the same as the
conjugacy class sizes in $G$, adjusted for the relative sizes of $L$
and $G$.

Fr convenience, we restrict attention to the case that $G$ and $L$ have the same size.

The analogous result to the abstract Kirillov orbit method now states
the following: there is a (possibly no longer canonical) bijection
between the irreducible representations of $G$ and the orbits for the
induced action of $G$ on the Pontryagin dual of $L$. Further, we can
arrange this bijection so that the orbit size corresponding to a
representation is the square of the degree of that representation.


\begin{lemma}
  Suppose $G$ is a finite group and $L$ is a finite Lie ring, that are
  Lazard correspondent up to isoclinism. Suppose for simplicity that
  both $G$ and $L$ have the same order. Then, the algebra of
  $G$-invariant $\C$-valued functions on $L$ is isomorphic to the
  algebra of $G$-invariant $\C$-valued functions on $G$.
\end{lemma}

\begin{proof}
  Both algebras are just direct products of copies of $\C$. The number
  of copies needed for $G$-invariant $\C$-valued functions on $L$ is
  the number of orbits in $L$ under the action of $G$. The number of
  copies needed for $G$-invariant $\C$-valued functions on $G$ is the
  number of conjugacy classes in $G$. We've shown that these two
  numbers are equal (we in fact showed a stronger result that the
  orbit size multiset and conjugacy class size multiset are equal).
\end{proof}

Note that this is a considerably weaker statement than the statement
of Theorem 1.1(ii) of Boyarchenko and Sabitova's paper, because that
statement is about $\Z$-algebras, whereas our weaker statement is only
about $\C$-algebras. In fact, there is no hope of the statement being
true for $\Z$-algebras, and that is one of the disadvantages of
working up to isoclinism. The reason is similar to the reason why
isoclinic groups have isomorphic group algebras over $\mathbb{C}$ but
not over $\mathbb{Z}$ or $\mathbb{Q}$.

\subsection{What I hope to show}

I hope to mimic Boyaarchenko and Sabitova's proof in order to show the
following:

\begin{conjecture}
  Suppose $G$ is a finite group and $L$ is a finite Lie ring, that are
  Lazard correspondent up to isoclinism. Suppose for simplicity that
  both $G$ and $L$ have the same order. Consider the set action of $G$
  on $L$. Now, consider the induced set action of $G$ on the
  Pontryagin dual of $L$. The multiset of orbit sizes for this action
  is precisely the same as the multiset of degrees of irreducible
  representations of $G$.
\end{conjecture}

Note what we {\em don't} expect to get:

\begin{itemize}
\item We don't expect to get a canonical bijection between the orbits
  and the irreducible representations.  
\item We don't expect to get an explicit character formula, because
  too much information is lost when going up to isoclinisms to allow us
  to recover the character.
\end{itemize}
