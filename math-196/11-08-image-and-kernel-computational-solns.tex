\documentclass[10pt]{amsart}

%Packages in use
\usepackage{fullpage, hyperref, vipul, enumerate}

%Title details
\title{Diagnostic in-class quiz solutions: due Friday November 8
  (delayed to Monday November 11): Image and kernel (computational)}
\author{Math 196, Section 57 (Vipul Naik)}
%List of new commands

\begin{document}
\maketitle

\section{Performance review}

26 people took this 3-question quiz. The score distribution was as follows:

\begin{itemize}
\item Score of 0: 2 people
\item Score of 1: 5 people
\item Score of 2: 8 people
\item Score of 3: 11 people
\end{itemize}

The mean score was 2.08.

The question-wise answers and performance review were as follows:

\begin{enumerate}
\item Option (E): 20 people
\item Option (A): 15 people
\item Option (B): 19 people
\end{enumerate}

{\em Note}: Performance on these questions was notably better than
last year, even though last year these questions were in the 'please
feel free to discuss'' category. Seems like people are understanding
the image and kernel better this year!
\section{Solutions}

{\bf PLEASE DO NOT DISCUSS {\em ANY} QUESTIONS.}

\begin{enumerate}

\item {\em Do not discuss this!}: Consider the linear transformation
  $\operatorname{Avg}:\R^2 \to \R^2$ defined as:

  $$\operatorname{Avg} = \left[ \begin{matrix} x \\ y \\\end{matrix}\right] \mapsto \left[ \begin{matrix} (x + y)/2 \\ (x + y)/2 \\\end{matrix}\right]$$

  What can we say about the kernel and image of $\operatorname{Avg}$?
  Note that in our descriptions of the kernel and the image below, we
  use $x$ to denote the first coordinate of the vector and $y$ to
  denote the second coordinate of the vector.

  {\em Note}: One way you can do that is to write the matrix for
  $\operatorname{Avg}$, but in this particular situation, it's easiest
  to just do things directly.

  \begin{enumerate}[(A)]
  \item The kernel is the zero subspace and the image is all of $\R^2$
  \item The kernel is the line $y = x$ and the image is also the line
    $y = x$
  \item The kernel is the line $y = x$ and the image is the line $y =
    -x$
  \item The kernel is the line $y = -x$ and the image is also the line
    $y = -x$
  \item The kernel is the line $y = -x$ and the image is the line $y = x$
  \end{enumerate}

  {\em Answer}: Option (E)

  {\em Explanation}: The kernel must satisfy that both coordinates of
  the output are zero, so $(x + y)/2 = 0$ and $(x + y)/2 = 0$. The
  solution is $y = -x$.

  The image must satisfy that both coordinates are equal, so it is the
  line $y = x$.

  Explicitly, the matrix in question is:

  $$\left[\begin{matrix} 1/2 & 1/2 \\ 1/2 & 1/2 \\\end{matrix}\right]$$

  The rref for this is:

  $$\left[\begin{matrix} 1 & 1 \\ 0 & 0 \\\end{matrix}\right]$$

  The kernel is generated by the vector $\left[\begin{matrix} -1 \\ 1
      \\\end{matrix}\right]$. The image is generated by
  $\left[\begin{matrix} 1/2 \\ 1/2 \\\end{matrix}\right]$.

  {\em Performance review}: 20 out of 26 got this. 5 chose (D), 1 chose (B).

  {\em Historical note (last time)}: $10$ out of $26$ got this. $7$ chose (B),
  $6$ chose (D), $3$ chose (C).

\item {\em Do not discuss this!}: Consider the {\em average of other
  two} linear transformation $\nu:\R^3 \to \R^3$ given as follows:

  $$\nu = \left[ \begin{matrix} x \\ y \\ z \\ \end{matrix}\right] \mapsto \left[ \begin{matrix} (y + z)/2 \\ (z + x)/2 \\ (x + y)/2 \\ \end{matrix} \right]$$

  What can we say about the kernel and image of $\nu$?

  Note that in our descriptions of the kernel and the image below, we
  use $x$ to denote the first coordinate of the vector, $y$ to denote
  the second coordinate of the vector, and $z$ to denote the third
  coordinate of the vector.

  {\em Note}: This can both be reasoned directly (without any
  knowledge of linear algebra) or alternatively it can be done by
  writing the matrix of $\nu$ and computing its rank, image, and
  kernel.

  \begin{enumerate}[(A)]
  \item The kernel is the zero subspace and the image is all of $\R^3$
  \item The kernel is the line $x = y = z$ (one-dimensional) and the
    image is the plane $x + y + z = 0$ (two-dimensional)
  \item The kernel is the plane $x + y + z = 0$ (two-dimensional) and
    the image is the line $x = y = z$ (one-dimensional)
  \item The kernel is the plane $x = y = z$ (two-dimensional) and the
    image is the line $x + y + z = 0$ (one-dimensional)
  \item The kernel is the line $x + y + z = 0$ (one-dimensional) and
    the image is the plane $x = y = z$ (two-dimensional)
  \end{enumerate}

  {\em Answer}: Option (A)

  {\em Explanation}: We can check that if all three outputs are zero,
  then $x = y = z = 0$. Alternatively, we can verify that the matrix:

  $$\left[\begin{matrix} 0 & 1/2 & 1/2 \\ 1/2 & 0 & 1/2 \\ 1/2 & 1/2 & 0 \\\end{matrix}\right]$$

  has full rank $3$. Its inverse is the matrix:

  $$\left[\begin{matrix} -1 & 1 & 1 \\ 1 & -1 & 1 \\ 1 & 1 & -1 \\\end{matrix}\right]$$

  Intuitively, $x$ is the sum of the second and third output minus the
  first output, and so on.

  {\em Performance review}: 15 out of 26 got this. 6 chose (C), 5
  chose (B).

  {\em Historical note (last time)}: $4$ out of $26$ got this. $11$ chose (E),
  $7$ chose (D), $2$ each chose (B) and (C).

\item {\em Do not discuss this!}: Consider the {\em difference of
  other two} linear transformation $\mu:\R^3 \to \R^3$ given by:

  $$\mu = \left[ \begin{matrix} x \\ y \\ z \\ \end{matrix}\right] \mapsto \left[ \begin{matrix} y - z \\ z - x \\ x - y \\ \end{matrix} \right]$$

  What can we say about the kernel and image of $\mu$?

  Note that in our descriptions of the kernel and the image below, we
  use $x$ to denote the first coordinate of the vector, $y$ to denote
  the second coordinate of the vector, and $z$ to denote the third
  coordinate of the vector.

  {\em Note}: This can both be reasoned directly (without any
  knowledge of linear algebra) or alternatively it can be done by
  writing the matrix of $\mu$ and computing its rank, image, and
  kernel.

  \begin{enumerate}[(A)]
  \item The kernel is the zero subspace and the image is all of $\R^3$
  \item The kernel is the line $x = y = z$ (one-dimensional) and the
    image is the plane $x + y + z = 0$ (two-dimensional)
  \item The kernel is the plane $x + y + z = 0$ (two-dimensional) and
    the image is the line $x = y = z$ (one-dimensional)
  \item The kernel is the plane $x = y = z$ (two-dimensional) and the
    image is the line $x + y + z = 0$ (one-dimensional)
  \item The kernel is the line $x + y + z = 0$ (one-dimensional) and
    the image is the plane $x = y = z$ (two-dimensional)
  \end{enumerate}

  {\em Answer}: Option (B)

  {\em Explanation}: The kernel must satisfy that all the three output
  coordinates are $0$. This means that $y - z = 0$, $z - x = 0$, and
  $x - y = 0$, so we get $x = y = z$. The image must satisfy that the
  sum of the three coordinates is zero, so it lies in the plane $x + y
  + z = 0$. A little more effort can show that it is equal to the
  entire plane.

  The matrix for the linear transformation $\mu$ is:

  $$\left[\begin{matrix} 0 & 1 & -1 \\ -1 & 0 & 1 \\ 1 & -1 & 0 \\\end{matrix}\right]$$

  We can row reduce this to get the rref:

  $$\left[\begin{matrix} 1 & 0 & -1 \\ 0 & 1 & -1 \\ 0 & 0 & 0 \\\end{matrix}\right]$$

  The rref shows that the linear transformation has rank two. The
  third variable is non-leading, so the kernel is generated by the vector:

  $$\left[\begin{matrix} 1 \\ 1 \\ 1 \\\end{matrix}\right]$$

  Explicitly, the kernel is the line $x = y = z$.

  For computing the image: the first two variables are leading
  variables, so the image of the linear transformation is the space
  spanned by the first two columns of the original matrix, i.e., it is
  the subspace spanned by the vectors:

  $$\left[\begin{matrix} 0 \\ -1 \\ 1 \\\end{matrix}\right], \left[\begin{matrix} 1 \\ 0 \\ -1\\\end{matrix}\right]$$

  We want to express this as a plane. The appropriate approach to
  determining the equation of the plane is to take the cross product
  of the vectors, albeit that is a construct specific to three
  dimensions. Alternatively, we can solve another linear system. Doing
  this from sratch is somewhat beyond our current scope, but it is
  relatively easy to figure out the plane from the collection of
  options presented.

  {\em Performance review}: 19 out of 26 got this. 4 chose (D), 2
  chose (A), 1 chose (E).

  {\em Historical note (last time)}: $8$ out of $26$ got this. $15$ chose (D),
  $2$ chose (C), and $1$ chose (E).

\end{enumerate}

\end{document}
