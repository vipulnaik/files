\documentclass[10pt]{amsart}

%Packages in use
\usepackage{fullpage, hyperref, vipul, enumerate}

%Title details
\title{Class quiz: November 11: Whoppers}
\author{Math 152, Section 55 (Vipul Naik)}
%List of new commands

\begin{document}
\maketitle

Your name (print clearly in capital letters): $\underline{\qquad\qquad\qquad\qquad\qquad\qquad\qquad\qquad\qquad\qquad}$

\begin{enumerate}
\item Suppose $g:\R \to \R$ is a continuous function such that
  $\lim_{x \to 0} g(x)/x^2 = A$ for some constant $A \ne 0$. What is
  $\lim_{x \to 0} g(g(x))/x^4$? 

  \begin{enumerate}[(A)]
  \item $A$
  \item $A^2$
  \item $A^3$
  \item $A^2g(A)$
  \item $g(A)/A^2$
  \end{enumerate}

  
  Your answer: $\underline{\qquad\qquad\qquad\qquad\qquad\qquad\qquad}$
  

\item Which of the following statements is {\bf always true}? {\em
  Exact replica of a past question}.

  \begin{enumerate}[(A)]

  \item The range of a continuous nonconstant function on a closed
    bounded interval (i.e., an interval of the form $[a,b]$) is a
    closed bounded interval (i.e., an interval of the form $[m,M]$).
  \item The range of a continuous nonconstant function on an open
    bounded interval (i.e., an interval of the form $(a,b)$) is an
    open bounded interval (i.e., an interval of the form $(m,M)$).
  \item The range of a continuous nonconstant function on a closed
    interval that may be bounded or unbounded (i.e., an interval of
    the form $[a,b]$, $[a,\infty)$, $(-\infty,a]$, or
    $(-\infty,\infty)$) is also a closed interval that may be bounded
    or unbounded.
  \item The range of a continuous nonconstant function on an open
    interval that may be bounded or unbounded (i.e., an interval of
    the form $(a,b)$,$(a,\infty)$, $(-\infty,a)$, or
    $(-\infty,\infty)$), is also an open interval that may be bounded
    or unbounded.
  \item None of the above.
  \end{enumerate}

  
  \vspace{0.05in}
  Your answer: $\underline{\qquad\qquad\qquad\qquad\qquad\qquad\qquad}$
  \vspace{0.05in}
  
\item Suppose $f$ is a continuously differentiable function on $\R$
  and $c \in \R$. Which of the following implications is {\bf false}?
  {\em Similar to a past question}.

  \begin{enumerate}[(A)]
  \item If $f$ has mirror symmetry about $x = c$, $f'$ has half turn
    symmetry about $(c,f'(c))$.
  \item If $f$ has half turn symmetry about $(c,f(c))$, $f'$ has
    mirror symmetry about $x = c$.
  \item If $f'$ has mirror symmetry about $x = c$, $f$ has half turn
    symmetry about $(c,f(c))$.
  \item If $f'$ has half turn symmetry about $(c,f'(c))$, $f$ has
    mirror symmetry about $x = c$.
  \item None of the above, i.e., they are all true.
  \end{enumerate}

  
\vspace{0.05in}
  Your answer: $\underline{\qquad\qquad\qquad\qquad\qquad\qquad\qquad}$
\vspace{0.05in}
  

\item Consider the function $f(x) := \lbrace\begin{array}{rl} x, & 0
  \le x \le 1/2 \\ x - (1/5), & 1/2 < x \le 1 \\\end{array}$. Define by
  $f^{[n]}$ the function obtained by iterating $f$ $n$ times, i.e.,
  the function $f \circ f \circ f \circ \dots \circ f$ where $f$
  occurs $n$ times. What is the smallest $n$ for which $f^{[n]} =
  f^{[n + 1]}$? {\em Similar to a question on the previous midterm.}

  \begin{enumerate}[(A)]
  \item $1$
  \item $2$
  \item $3$
  \item $4$
  \item $5$
  \end{enumerate}

  
\vspace{0.05in}
  Your answer: $\underline{\qquad\qquad\qquad\qquad\qquad\qquad\qquad}$
\vspace{0.05in}
  



\item With $f$ as in the previous question, what is the set of points
  in $(0,1)$ where $f \circ f$ is not continuous?

  \begin{enumerate}[(A)]
  \item $0.5$ only
  \item $0.5$ and $0.7$
  \item $0.5$, $0.7$, and $0.9$
  \item $0.7$ and $0.9$
  \item $0.9$ only
  \end{enumerate}

  
\vspace{0.05in}
  Your answer: $\underline{\qquad\qquad\qquad\qquad\qquad\qquad\qquad}$
\vspace{0.05in}
  

\item Consider the graph of the function $f(x) := x\sin(1/(x^2 -
  1))$. What can we say about the vertical and horizontal asymptotes?
  

  \begin{enumerate}[(A)]
  \item The graph has vertical asymptotes at $x = +1$ and $x = -1$
    and horizontal asymptote (in both directions) $y = 0$.
  \item The graph has vertical asymptotes at $x = +1$ and $x = -1$
    and horizontal asymptote (in both directions) $y = 1$.
  \item The graph has no vertical asymptotes and horizontal
    asymptote (in both directions) $y = 0$.
  \item The graph has no vertical asymptotes and horizontal
    asymptote (in both directions) $y = 1$.
  \item The graph has no vertical or horizontal asymptotes.
  \end{enumerate}

  
\vspace{0.05in}
  Your answer: $\underline{\qquad\qquad\qquad\qquad\qquad\qquad\qquad}$
\vspace{0.05in}
  

\item Suppose $f$ and $g$ are increasing functions from $\R$ to
  $\R$. Which of the following functions is {\em not} guaranteed to be
  an increasing functions from $\R$ to $\R$? {\em An exact replica of
  a past question.}

  \begin{enumerate}[(A)]

  \item $f + g$
  \item $f \cdot g$
  \item $f \circ g$
  \item All of the above, i.e., none of them is guaranteed to be increasing.
  \item None of the above, i.e., they are all guaranteed to be increasing.
  \end{enumerate}

  
\vspace{0.05in}
  Your answer: $\underline{\qquad\qquad\qquad\qquad\qquad\qquad\qquad}$
\vspace{0.05in}
  

\item Suppose $F$ and $G$ are continuously differentiable functions on
  all of $\R$ (i.e., both $F'$ and $G'$ are continuous). Which of the
  following is {\bf not necessarily true}? {\em Exact replica of a
  previous question.}

  \begin{enumerate}[(A)]
  \item If $F'(x) = G'(x)$ for all integers $x$, then $F - G$ is a
    constant function when restricted to integers, i.e., it takes the
    same value at all integers.
  \item If $F'(x) = G'(x)$ for all numbers $x$ that are not integers,
    then $F - G$ is a constant function when restricted to the set of
    numbers $x$ that are not integers.
  \item If $F'(x) = G'(x)$ for all rational numbers $x$, then $F - G$
    is a constant function when restricted to the set of rational
    numbers.
  \item If $F'(x) = G'(x)$ for all irrational numbers $x$, then $F -
    G$ is a constant function when restricted to the set of irrational
    numbers.
  \item None of the above, i.e., they are all necessarily true.
  \end{enumerate}

  
\vspace{0.05in}
  Your answer: $\underline{\qquad\qquad\qquad\qquad\qquad\qquad\qquad}$
\vspace{0.05in}
  

\item Consider the four functions $\sin(\sin x)$, $\sin(\cos x)$,
  $\cos(\sin x)$, and $\cos(\cos x)$. Which of the following
  statements are true about their periodicity?

  \begin{enumerate}[(A)]
  \item All four functions are periodic with a period of $2\pi$.
  \item All four functions are periodic with a period of $\pi$.
  \item $\sin(\sin x)$ and $\sin(\cos x)$ have a period of $\pi$,
    whereas $\cos(\sin x)$ and $\cos(\cos x)$ have a period of $2\pi$.
  \item $\cos(\sin x)$ and $\cos(\cos x)$ have a period of $\pi$,
    whereas $\sin(\sin x)$ and $\sin(\cos x)$ have a period of $2\pi$.
  \item $\sin(\sin x)$ has a period of $2\pi$, the other three
    functions have a period of $\pi$.
  \end{enumerate}
  
  \vspace{0.05in}
  Your answer: $\underline{\qquad\qquad\qquad\qquad\qquad\qquad\qquad}$
  \vspace{0.05in}
\end{enumerate}
\end{document}
