\section{Applications and related results}\label{sec:applications}

The Lazard correspondence up to isoclinism can be used as an
analytical framework for the study of previous extensions of the
Lazard correspondence. Some of these are discussed in this
section. For simplicity, we restrict our statements to the global
Lazard correspondence up to isoclinism.

\subsection{Relation with past work of Glauberman}\label{sec:rel-gg08}

A special case of the Lie bracket-like map for groups, described in
Section \ref{sec:lie-bracket-like-map}, was described by Glauberman in
Section 5 of his 2008 paper \cite{GG08}. Theorem 5.1 of the paper
demonstrated that the map is alternating and bilinear when viewed as a
map from the Lazard Lie ring of the inner automorphsim
group. Explicitly, the statement of Theorem 5.1 was as follows:

\begin{theorem}[Theorem 5.1 of \cite{GG08}]
  Suppose $S$ has nilpotence class at most $p$ and $Z=Z(S)$. Then the following assertions hold.
  \begin{enumerate}
    \item Both $S/Z$ and $S'$ have nilpotence class at most $p-1$.
    \item Define addition (and bracket multiplication) on $S/Z$ and
      $S'$ as in Theorem 2.1 (of \cite{GG08}). Then there exists an
      alternating bi-additive function $f$ from $(S/Z)\times(S/Z)$
      into $S'$ such that, for all $u, v$ in $S$,
      $$f(uZ,vZ)=0\text{ if and only if }uv=vu.$$
      Moreover, the image of $f$ generates $S'$ as an additive group.
  \end{enumerate}
\end{theorem}

Part (1) of the theorem is straightforward, and is a special case of
Lemma \ref{lemma:global-class-one-more}. Part (2) of the Theorem
describes the map $\omega_S^{\text{Lie}}$ described in Section
\ref{sec:lie-bracket-like-map}. The statement of part (2) follows
indirectly from the discussion in Section \ref{sec:lcuti} as
follows. The commutator map defines a map $(S/Z) \wedge (S/Z) \to S'$
from the exterior square of $S/Z$ as a group. The Lie bracket map
$\omega_S^{\text{Lie}}$ correspondingly defines a Lie ring map
$\log(S/Z) \wedge \log (S/Z) \to \log(S')$. Here, $\log(S/Z) \wedge
\log(S/Z)$ denotes the exterior square of $\log(S/Z)$ {\em as a Lie
  ring}. Essentially, this follows from Theorem
\ref{thm:global-lazard-correspondence-preserves-schur-multipliers}
where $L = \log(S/Z)$ and $G = S/Z$ and part of the work done in the
proof of Theorem \ref{thm:glcuti-extensions-implies-glcuti}.

The main way that our results are more general than Glauberman's is
that our results establish the existence of a Lie ring that is a
counterpart to the {\em whole} group ($S$ in Glauberman's theorem)
whose Lie bracket map realizes the map $\log(S/Z) \wedge \log (S/Z)
\to \log(S')$. The crucial additional ingredient in our proof is the
use of the surjectivity of the universal coefficient theorem short
exact sequence (described in Section \ref{sec:ses-uct-lie}, and
applied in Theorem \ref{thm:glcuti-extensions-implies-glcuti}) to
demonstrate the existence of an appropriate Lie ring extension.

\subsection{Correspondences between subgroups and subrings}\label{sec:lcuti-sub}

The Lazard correspondence up to isoclinism is a correspondence between
some equivalence classes {\em up to isoclinism} of groups and some
equivalence classes {\em up to isoclinism} of Lie rings. This fact
immediately constrains the utility of the correspondence to the study
of attributes that are invariant under isoclinism. Explicitly, any
attribute of a group that we wish to study using the Lazard
correspondence up to isoclinism should be an attribute that is
invariant under isoclinisms of groups. Keeping this in mind, we apply
the Lazard correspondence up to isoclinism to relate subgroups of the
group and subrings of the Lie ring.

In Section \ref{sec:isoclinism-correspondence-some-subgroups}, we
described some aspects of the subgroup structure that are invariant
under isoclinism. In particular, we noted there that we have a
correspondence between subgroups containing the center for the two
groups. We discussed the similar situation for Lie rings in Section
\ref{sec:isoclinism-correspondence-some-subrings}.

In Section \ref{sec:global-lazard-correspondence-subgroups}, we
described how the global Lazard correspondence gives a correspondence
between some subgroups of the group and some subrings of the Lie
ring. Explicitly, the correspondence is between the global Lazard Lie
subgroups of the group and the global Lazard Lie subrings of the Lie
ring.

The global Lazard correspondence up to isoclinism combines the above
ideas to give a correspondence between some subgroups of the group and
some subrings of the Lie ring. We now describe this correspondence.

Suppose $c$ is a positive integer. Denote by $\pi_c$ the set of primes
that are less than or equal to $c$. Suppose $G$ is a $\pi_c$-powered
group of class $(c + 1)$ and $L$ is a $\pi_c$-powered Lie ring of
class $(c + 1)$, and suppose $G$ and $L$ are in global class $(c + 1)$
Lazard correspondence. We have a correspondence:

\begin{center}
  $\pi_c$-powered subgroups of $G$ containing $Z(G)$ $\leftrightarrow$
  $\pi_c$-powered Lie subrings of $L$ containing $Z(L)$
\end{center}

Further, if a subgroup $H$ of $G$ and a subring $M$ of $L$ correspond
to each other by the correspondence above, then $H$ and $M$ are in
global class $c$ Lazard correspondence {\em up to isoclinism}. Even in
the case that $H$ is a global Lazard Lie group or $M$ is a global
Lazard Lie ring, $H$ and $M$ are not necessarily in global Lazard
correspondence up to isomorphism.

The correspondence restricts to a correspondence between normal
subgroups and ideals:

\begin{center}
  $\pi_c$-powered normal subgroups of $G$ containing $Z(G)$
  $\leftrightarrow$ $\pi_c$-powered ideals of $L$ containing $Z(L)$
\end{center}

We can generalize the correspondence somewhat. For any $d \le c$,
denote by $\pi_d$ the set of all primes less than or equal to $d$. We
have a correspondence:

\begin{center}
  $\pi_d$-powered subgroups of $G$ of class at most $(d + 1)$
  containing $Z(G)$ $\leftrightarrow$ $\pi_d$-powered Lie subrings of
  $L$ of class at most $(d + 1)$ containing $Z(L)$
\end{center}

This also restricts to a correspondence between normal subgroups and ideals:

\begin{center}
  $\pi_d$-powered normal subgroups of $G$ of class at most $(d + 1)$
  containing $Z(G)$ $\leftrightarrow$ $\pi_d$-powered ideals of $L$
  of class at most $(d + 1)$ containing $Z(L)$
\end{center}

\subsection{Correspondence between abelian subgroups and abelian subrings}

In Section \ref{sec:isoclinism-abelian-subgroups} (respectively,
Section \ref{sec:isoclinism-abelian-subrings}), we noted that for two
groups (respectively, two Lie rings) that are isoclinic, the abelian
subgroups (respectively, abelian subrings) that contain the center are
in correspondence. We now state similar results describing a
correspondence between a group and a Lie ring that are in Lazard
correspondence up to isoclinism. Suppose $c$ is a positive
integer. Denote by $\pi_c$ the set of primes that are less than or
equal to $c$. Suppose a group $G$ is in global class $(c + 1)$ Lazard
correspondence up to isoclinism with a Lie ring $L$. Then, the
following hold:

\begin{itemize}
\item The Lazard correspondence up to isoclinism establishes a
  correspondence between abelian subgroups of $G$ containing $Z(G)$
  and abelian subrings of $L$ containing $Z(L)$.
\item The Lazard correspondence up to isoclinism establishes a
  correspondence between the abelian subgroups of $G$ that are
  self-centralizing and the abelian subrings of $L$ that are
  self-centralizing.
\item In the case that $G$ and $L$ are both finite, the Lazard
  correspondence up to isoclinism establishes a correspondence between
  abelian subgroups of maximum order in $G$ and abelian subrings of
  maximum order in $L$.
\item For each of the correspondences above, normal subgroups
  correspond with ideals.
\item If $G$ and $L$ are both finite, then each of the correspondences
  above preserves index.
\end{itemize}

\subsection{Normal subgroups that are global Lazard Lie groups}

We begin with a lemma. We omit the proof because the lemma is
straightforward.

\begin{lemma}
  The following are equivalent for a group $G$.

  \begin{enumerate}
  \item Any two elements of $G$ that are conjugate to each other
    commute.
  \item Every element of $G$ is contained in an abelian normal
    subgroup of $G$.
  \item The normal closure of every element of $G$ is abelian.
  \item $G$ is a union of abelian normal subgroups.
  \item For all $x,y \in G$, $[[x,y],y] = 1$, i.e., $G$ satisfies a
    2-Engel condition.
  \end{enumerate}
\end{lemma}

Groups that satisfy the equivalent conditions of the lemma are termed
{\em Levi groups} or {\em 2-Engel groups}. We can now state the next
lemma.

\begin{lemma}
  Suppose $G$ is a nilpotent group of nilpotency class two. Then, $G$
  is a Levi group.
\end{lemma}

We can now state an important result relating the nilpotency class of
a group and the nilpotency class of normal closures of elements. The
result follows from the above lemma and induction on the nilpotency
class.

\begin{lemma}\label{lemma:union-of-normal}
  Suppose $G$ is a nilpotent group of nilpotency class $c + 1$ where
  $c \ge 1$. Then, the normal closure of any element of $G$ is a
  nilpotent group of nilpotency class at most $c$. In other words, $G$
  is a union of normal subgroups each of which has nilpotency class at
  most $c$.
\end{lemma}

\begin{proof}
  We prove the claim by induction on $c$. The base case $c = 1$
  follows from the preceding lemma. We proceed to demonstrate the
  inductive step, assuming $c \ge 2$.

  Let $x \in G$ and let $H$ be the normal closure of $x$ in
  $G$.

  Denote by $\overline{x}$ the image of $x$ in $G/Z(G)$. It is easy to
  verify that the normal closure of $\overline{x}$ in $G/Z(G)$ is the
  image of $H$ in $G/Z(G)$. Denote this by $\overline{H}$. By
  assumption, $G/Z(G)$ has class $c$. Thus, by the inductive
  hypothesis, $\overline{H}$ has class at most $c - 1$. Therefore, $H$
  has class at most $c$.
\end{proof}

The relevance of this result to the global Lazard correspondence is as
follows.

\begin{lemma}\label{lemma:union-of-lazard}
  Suppose $c$ is a positive integer, $\pi_c$ is the set of all primes
  less than or equal to $c$, and $G$ is a $\pi_c$-powered group of
  nilpotency class at most $(c + 1)$. Then, the following hold:

  \begin{enumerate}
  \item Every element of $G$ is contained in a normal subgroup of $G$
    that is a global class $c$ Lazard Lie group. Equivalently, $G$ is
    a union of normal subgroups that are global class $c$ Lazard Lie
    groups.
  \item $G$ is a union of normal subgroups that are global class $c$
    Lazard Lie groups such that all the subgroups contain the center
    of $G$.
  \end{enumerate}
\end{lemma}

\begin{proof}
  {\em Proof of (1)}: We will show that for every element $g \in G$,
  $g$ is contained in global class $c$ Lazard Lie subgroup of $G$. By
  the preceding lemma (Lemma \ref{lemma:union-of-normal}), the normal
  closure of $g$ in $G$ is a group of nilpotency class $c$. Denote
  this normal closure as $H$. Then, by Theorem
  \ref{thm:pi-powered-envelope}, the subgroup $\sqrt[\pi_c]{H}$ is a
  $\pi_c$-powered normal subgroup of $G$. Thus, $\sqrt[\pi_c]{H}$ is a
  global class $c$ Lazard Lie group that is a normal subgroup of $G$.

  {\em Proof of (2)}: We can replace each of the normal subgroups
  obtained for part (1) by its product with the center of $G$.
\end{proof} 

In Section \ref{sec:lcuti}, we showed that for a group $G$ satisfying
the hypotheses of the lemma above, we can find a Lie ring $L$ such
that $G$ is in global class $(c + 1)$ Lazard correspondence up to
isoclinism with $L$. In Section \ref{sec:lcuti-sub}, we showed that
the $\pi_c$-powered normal subgroups of $G$ that contain the center
$Z(G)$ are in Lazard correspondence up to isoclinism with the
$\pi_c$-powered ideals of $L$ that contain the center $Z(L)$. In
particular, this means that there is a correspondence:

\begin{center}
  Normal subgroups of $G$ containing the center that are global class
  $c$ Lazard Lie groups $\leftrightarrow$ Ideals of $L$ containing the
  center that are global class $c$ Lazard Lie rings
\end{center}

Note, however, that even though the objects on both sides of the
correspondence are in the domain of the global Lazard correspondence,
the correspondence itself is only a global Lazard correspondence {\em
  up to isoclinism}.

This raises the following question:

\begin{quote}
  Given a $\pi_c$ powered group $G$ of nilpotency class $(c + 1)$, is
  it possible to choose a $\pi_c$-powered Lie ring $L$ such that,
  under the above correspondence, each of the corresponding objects
  are in global Lazard correspondence (not just up to isoclinism)?
\end{quote}

In general, the answer to this question is {\em no}. We can see
examples even for $2$-groups of class two, such as the case where $c =
2$ and $G = D_8$. Our conclusion can be deduced from the discussion in
Section \ref{sec:bcuti-ex}.

\subsection{Adjoint action}\label{sec:lcuti-adjoint}

Many aspects of the relationship between inner automorphisms and inner
derivations described in Section \ref{sec:lazard-adjoint} continue to
be valid, with suitable modification, for the global Lazard
correspondence up to isoclinism (and also for the $3$-local Lazard
correspondence up to isoclinism). We will state our results for the
global Lazard correspondence up to isoclinism, and mention at the end
why the results generalize to the $3$-local Lazard correspondence up
to isoclinism.

Suppose $c$ is a positive integer and $\pi_c$ is the set of all primes
less than or equal to $c$. Suppose $G$ is a $\pi_c$-powered group of
class at most $c + 1$ and $L$ is a $\pi_c$-powered Lie ring of class
at most $c + 1$ such that $G$ and $L$ are in global class $(c + 1)$
Lazard correspondence up to isoclinism.

The adjoint action of $G$ on $L$ is defined as follows:

$$\operatorname{Ad}: G \to \operatorname{Aut}(L)$$

For any $u \in G$, define $\operatorname{Ad}_u$ as follows. Denote by
$\overline{u}$ the image of $u$ in $G/Z(G)$. Denote by $x$ an element
of $L$ such that the image of $x$ in $L/Z(L)$ corresponds to the
element $\overline{u}$ under the global class $c$ Lazard
correspondence between $L/Z(L)$ and $G/Z(G)$. We define
$\operatorname{Ad}_u$ as the following automorphism of $L$:

$$\operatorname{Ad}_u = \exp(\operatorname{ad}_x)$$

where $\exp$ is understood to mean the actual power series of $\exp$,
with the addition and multiplication happening inside
$\operatorname{End}_\Z(L)$, the ring of endomorphisms of the
underlying additive group of $L$. Explicitly:

$$\operatorname{Ad}_u = 1 + \operatorname{ad}_x + \frac{\operatorname{ad}_x^2}{2!} + \dots \frac{\operatorname{ad}_x^c}{c!}$$

Or even more explicitly:

$$\operatorname{Ad}_u(g) = g + [x,g] + \frac{1}{2!}[x,[x,g]] + \dots + \frac{1}{c!}[x,[x,\dots[x,g]\dots]]$$

where the $x$ appears $c$ times in the last iterated Lie bracket.

It can easily be verified that $\operatorname{Ad}_u$ is an
automorphism of $L$ (this follows, for instance, by noting that
$\operatorname{ad}_x$ is a derivation of $L$ satisfying the conditions
of Proposition 2.5 in Alperin and Glauberman's paper \cite{AG98}). It
can also be verified that $\operatorname{Ad}_{uv} =
\operatorname{Ad}_u\operatorname{Ad}_v$, making $\operatorname{Ad}$ a
homomorphism. Note that both these verifications use only three
elements at a time:

\begin{itemize}
\item The verification that $\operatorname{Ad}_u$ is an automorphism
  requires us to consider the effect of $\operatorname{Ad}_u$ on an
  arbitrary Lie product $[g,h]$, and therefore involves three
  elements: $x$, $g$, and $h$.
\item The verification that $\operatorname{Ad}_{uv} =
  \operatorname{Ad}_u\operatorname{Ad}_v$ requires us to consider an
  arbitrary element $g \in L$ and elements (say $x$ and $y$) that
  correspond to $u$ and $v$ when considered modulo the
  center. Therefore, this involves three elements.
\end{itemize}

Thus, the proofs generalize to the $3$-local case.

\subsection{Adjoint group and unitriangular matrix group}

In Section \ref{sec:adjoint-lazard}, we noted that for a nilpotent
associative ring $N$ of class $c$, its associated Lie ring and adjoint
group $1 + N$ are both nilpotent. We further noted that if the
additive group of $N$ is $\pi_c$-powered where $\pi_c$ is the set of
primes less than or equal to $c$, then $N$ (as a Lie ring) is in
global class $c$ Lazard correspondence with the adjoint group $1 + N$.

The result has an analogue for the global class $(c + 1)$ Lazard
correspondence up to isoclinism, as follows. Suppose $N$ is a
nilpotent associative ring of nilpotency class $c + 1$ (i.e., all
products of length $c + 2$ or more are zero). Suppose further that the
additive group of $N$ is $\pi_c$-powered. Then, the Lie ring $N$ and
the adjoint group $1 + N$ are in global class $(c + 1)$ Lazard
correspondence up to isoclinism.

We can therefore also obtain an analogue of the result described in
Section \ref{sec:unitriangular-lazard}. Explicitly, this says the
following: if the additive group of a commutative associative unital
ring $R$ is $\pi_c$-powered, then $NT(c + 2,R)$, viewed as a Lie ring,
is in global class $(c + 1)$ Lazard correspondence up to isoclinism
with the group $UT(c + 2,R)$.

%\newpage

\section{Possible extensions}\label{sec:possible-extensions}

\subsection{Relaxing the $\pi_c$-powered assumption on the whole group}

It is possible to reframe the existence result of the global class $(c
+ 1)$ Lazard correspondence in a manner that replaces the assumption
that the group itself is $\pi_c$-powered by the assumption that the
inner automorphism group and derived subgroup are
$\pi_c$-powered. Similarly, we can replace the assumption that the Lie
ring itself is $\pi_c$-powered by the assumption that the inner
derivation Lie ring and derived subring are both $\pi_c$-powered. We
can ahow both results either by making modifications to the proofs or
by first passing from the group to an isoclinic group that is
$\pi_c$-powered.

Note that we {\em do} need to make the assumption of $\pi_c$-powering
for {\em both} the inner automorphism group {\em and} the derived
subgroup. If we assume only that the inner automorphism group is
$\pi_c$-powered, it does follow from that that the derived subgroup is
$\pi_c$-divisible, but the derived subgroup need not be
$\pi_c$-torsion-free and therefore need not be $\pi_c$-powered. For
instance, for $c \ge 2$, consider the group $G = UT(c + 2,\Q)/\Z$,
where the subgroup $\Z$ being factored out is inside the central
subgroup $\Q$. $G$ is a global class $(c + 1)$ group and
$\operatorname{Inn}(G)$ is $\pi_c$-powered (in fact, it is rationally
powered), but $G'$ is not $\pi_c$-powered (in fact, it is not powered
over any prime, because it contains $\Q/\Z$ as a subgroup).

\subsection{$3$-local Lazard correspondence up to isoclinism: proof of existence}

In Section \ref{sec:lcuti}, we {\em defined} the Lazard correspondence
up to isoclinism in the $3$-local setting, but we {\em proved
  existence} only in the global setting. We expect the results to hold
in the $3$-local setting. Explicitly, we expect the following results
described in the outline to hold:

\begin{itemize}

\item For a Lie ring $L$, if both $\operatorname{Inn}(L)$ and $L'$ are
  ($3$-local) Lazard Lie rings, then we can find a group $G$ such that
  $L$ is in ($3$-local) Lazard correspondence up to isoclinism with
  $G$.
\item For a group $G$, if both $\operatorname{Inn}(G)$ and $G'$ are
  ($3$-local) Lazard Lie groups, then we can find a Lie ring $L$ such
  that $L$ is in Lazard correspondence up to isoclinism with $G$.
\end{itemize}

We believe that proofs analogous to those presented in Section
\ref{sec:lcuti} of this thesis can be used to show the above. However,
executing these proofs would require us to define a number of
intermediate objects more generally, making the exercise of
generalizing the proofs more difficult.

\subsection{More results about primes appearing in denominators}\label{sec:prime-denominator-more}

The literature on the Baker-Campbell-Hausdorff formula and the Lazard
correspondence includes a number of bounds on primes that appear in
denominators in these formulas. Some of the relevant literature is
discussed below.

\begin{itemize}
\item Theorem C of Easterfield's paper \cite{Easterfield} provides
  bounds on the exponents of primes appearing in formulas for
  commutators between powers of elements. The paper does not
  explicitly discuss the Lazard correspondence or the
  Baker-Campbell-Hausdorff formula, but the results are closely
  related, and the relationship is elucidated further by Glauberman in
  his paper \cite{Partialextensions} on partial extensions of the
  Lazard correspondence.
\item The paper \cite{Lazardeffective}, a paper describing a
  computationally effective version of the Lazard correspondence,
  provides bounds on the exponents of primes in denominators for the
  formula. The bounds for the inverse Baker-Campbell-Hausdorff formula
  are in Section 6 of the paper.
\item Thomas Weigel's doctoral dissertation \cite{Weigel} contains
  strong bounds on the primes that appear in the denominators for
  formulas for $M_d$ and $h_{2,d}$ where $d \le 2c - 2$. These results
  are related to the results we describe in the Appendix, Section
  \ref{appsec:bch-prime-power-divisor-bound}.
\end{itemize}

\subsection{Potential extension to a Lazard correspondence up to $n$-isoclinism}\label{sec:potential-extension-n-isoclinism}

In the Appendix, Section \ref{appsec:homologism-theory}, we describe
the notions of {\em isologism} and {\em homologism} for
groups. Similar concepts can be defined for Lie rings (and in fact,
for more general varieties of algebras).  The corresponding
generalization of the Schur multiplier is an abelian group termed the
{\em Baer invariant}. The paper \cite{Baerinvariantsandisolosigms} by
Leedham-Green and McKay is an important source of results about
isologisms.

A particular form of isologism of interest to us is $n$-isoclinism for
a positive integer $n$. The concepts of $n$-isoclinism and
$n$-homoclinism are described in the Appendix, Section
\ref{appsec:n-homoclinism} and also in \cite{Hekster} (for groups) and
\cite{Moghaddametal} (for Lie rings). Recall that an isoclinism
between groups is an equivalence between their commutator
structures. The commutator structure is precisely the structure that
becomes trivial in all abelian groups. Thus, we can think of
isoclinism as ``equivalence modulo the subvariety of abelian groups.''
In a similar vein, $n$-isoclinism is an equivalence between the
$(n+1)$-fold commutator structures, and we can think of it as
``equivalence modulo the subvariety of groups of nilpotency class
$n$.'' The corresponding generalization of the Schur multiplier is an
abelian group termed the $n$-nilpotent multiplier. The $n$-nilpotent
multiplier of a group $G$ is denoted $M^{(n)}(G)$, and we use similar
notation for Lie rings. Note that the Schur multiplier is the
$n$-nilpotent multiplier for the case $n = 1$.

It may be possible to generalize the ``global class $(c + 1)$ Lazard
correspondence up to isoclinism'' to a ``global class $(c + n)$ Lazard
correspondence up to $n$-isoclinism'' for some values of $n >
1$. Weigel's results, alluded to in Section
\ref{sec:prime-denominator-more}, suggest that it may be possible to
    {\em define} the notion for some values of $n > 1$ (dependent on
    $c$). 

However, there are important parts of the theory developed in Sections
\ref{sec:schur-multiplier-and-second-cohomology} and
\ref{sec:schur-multiplier-and-second-cohomology-lie} that do not
generalize in the expected manner. For instance, the paper
\cite{Baerinvariantsandisolosigms} by Leedham-Green and McKay suggests
that the approach that we have used in this thesis cannot be used to
show existence. In particular, instead of the universal coefficient
theorem short exact sequence described in Section \ref{sec:ses-uct},
we obtain a long exact sequence. Specifically, the analogue of right
exactness (the surjectivity of the right map) fails. For a more
detailed discussion of the failure of surjectivity, see Section 2 of
the paper by Leedham-Green and McKay.

\subsection{Glauberman's partial extension}

In his 2007 paper \cite{Partialextensions}, George Glauberman described
a generalization of the Lazard correspondence. His Theorem A and
Theorem B are restated below.

\begin{theorem}[Theorem A of \cite{Partialextensions}]
  Suppose $p$ is a prime and $S$ is a finite $p$-group. Then $[x,y]$
  (in the Lie bracket sense) is well defined whenever $x$ and $y$ are
  elements of (possibly different) normal subgroups of $S$ of
  nilpotence class at most $p - 1$.

  In addition, suppose $A$ and $B$ are normal subgroups of $S$ of
  nilpotence class at most $p - 1$. Define $+$ and $[ \ , \ ]$ on $A$
  and $B$ as in the Lazard correspondence. Then:

  \begin{enumerate}[(i)]
  \item for each $u$ in $A$ and $v$ in $B$, the elements $[u,v]$ and
    $[v,u]$ lie in $A \cap B$, and $[v,u] = [u,v]^{-1}$.
  \item for each $u,u'$ in $A$ and $v$ in $B$,

    $$[u + u',v] = [u,v] + [u',v] \qquad \text{and} \qquad [[u,u'],v] = [[u,v],u'] + [u,[u',v]]$$
  \end{enumerate}
\end{theorem}

\begin{theorem}[Theorem B of \cite{Partialextensions}]
  Suppose $S$ is a finite $p$-group generated by a set $\mathcal{S}$
  of normal subgroups $N$ of $S$ having nilpotence class at most $p -
  1$. Let $\mathfrak{U}$ be the set-theoretic union of the elements of
  $\mathcal{S}$. For each $N$ in $\mathcal{S}$, define $+$ on $N$ by
  Lazard's definition. For each $u,v \in \mathfrak{U}$, define $[u,v]$
  as in Theorem A.

  Let $E = \operatorname{End}(\mathcal{S})$ be the set of all mappings
  $\phi$ from $\mathcal{U}$ to $\mathcal{U}$ such that, for each $N$
  in $\mathcal{S}$,

  $\phi$ maps $N$ into $N$ and induces an endomorphism of $N$ under
  $+$.

  Define addition and multiplication on $E$ by

  $$(\phi + \phi')(x) = \phi(x) + \phi'(x) \qquad \text{and} \qquad (\phi \phi')(x) = \phi(\phi'(x))$$

  For each $v \in \mathfrak{U}$, define a mapping $\operatorname{ad}
  v$ on $\mathfrak{U}$ by

  $$(\operatorname{ad} v)(u) = [u,v]$$

  Then:

  \begin{enumerate}[(i)]
  \item $\operatorname{ad} v$ lies in $E$ for each $v$ in
    $\mathfrak{U}$.
  \item for each $N$ in $\mathcal{S}$ and each $v,w \in N$,
    $\operatorname{ad}(v + w) = \operatorname{ad} v +
    \operatorname{ad} w$.
  \item for $v,w \in \mathfrak{U}$:

    $$[\operatorname{ad} v, \operatorname{ad} w] = \operatorname{ad}[w,v] = -\operatorname{ad}[v,w]$$
    
    $$\operatorname{ad} v = \operatorname{ad} w \iff v \equiv w \pmod{Z(S)}$$
  \item The additive subgroup $L(\mathcal{S})$ of $E$ spanned by
    mappings $\operatorname{ad} v$ for $v$ in $\mathfrak{U}$ is a Lie
    subring of $E$, and
  \item for $L(\mathcal{S})$ as in part (iv), each element $\phi$ of
    $L(\mathcal{S})$ satisfies

    $$\phi([u,v]) = [\phi(u),v] + [u,\phi(v)] \ \forall u,v \in \mathfrak{U}$$
  \end{enumerate}
\end{theorem}

In the special case that $S$ is a finite $p$-group of nilpotency class
$p$, the existence of a collection $\mathcal{S}$ satisfying the
hypotheses of the theorems is guaranteed by Lemma
\ref{lemma:union-of-lazard}. We can also deduce that $L(\mathcal{S})$
is isomorphic to the Lazard Lie ring $\log(\operatorname{Inn}(S))$,
regardless of the choice of $\mathcal{S}$. We also know that in this
case there exists a Lie ring $N$ that is in global class $p$ Lazard
correspondence up to isoclinism with $S$, and therefore, that
$\operatorname{Inn}(N) \cong L(\mathcal{S})$.

Thus, in the case that $S$ is a finite $p$-group of nilpotency class
greater than $p$ but admitting such a collection of normal subgroups
$\mathcal{S}$, the Lie ring $L(\mathcal{S})$ can be thought of as our
attempt to define $\log(\operatorname{Inn}(S))$, even though the
latter does not exist.\footnote{In private correspondence, George
  Glauberman shared an example of a finite $p$-group $S$ of class
  greater than $p$ for which the isomorphism type of $L(\mathcal{S})$
  is dependent on the choice of $\mathcal{S}$, i.e., different choices
  of $\mathcal{S}$ may yield different isomorphism types for
  $L(\mathcal{S})$. The example has not yet been published.} This
raises the question of whether we can define a generalization of the
Lazard correspondence up to isoclinism that would guarantee the
existence of a Lie ring $N$ such that we can think of $S$ and $N$ as
being related via that appropriate generalization, and such that
$\operatorname{Inn}(N) \cong L(\mathcal{S})$.

\subsection{Possible generalization of the Kirillov orbit method}\label{sec:kirillov-orbit-method}

The Kirillov orbit method is a method used to compute the degrees of
the irreducible representations of a finite Lazard Lie group. The
following is the procedure for computing the degrees of irreducible
representations of a finite group $G$:

\begin{itemize}
\item Denote by $L$ the Lazard Lie ring corresponding to $G$.
\item Denote by $\hat{L}$ the Pontryagin dual to $L$, viewed only as
  an additive group. Note that $\hat{L}$ is isomorphic to $L$, but
  there is no natural isomorphism.
\item The natural action of $G$ on $L$ (called the {\em adjoint
  representation}, and described in Section \ref{sec:lazard-adjoint})
  induces a natural action of $G$ on $\hat{L}$ (called the {\em
    coadjoint representation}). The orbits under this action correspond
  to the irreducible representations. Moreover, the size of any orbit
  is the {\em square} of the degree of the irreducible representation
  to which it corresponds. Note that this is combinatorially
  consistent with the fact that the sum of squares of the degrees of
  irreducible representations of $G$ equals the order of $G$ (because
  the order of $G$ equals the order of $L$, and this in turns equals
  the order of $\hat{L}$).
\end{itemize}

For a detailed discussion of the method, see the papers
\cite{GonzalezSanchez}, \cite{BoyarchenkoSabitova}, and
\cite{enumeratingcharacters}.

In the Appendix, Section \ref{appsec:isoclinism-extra-proofs}, we show
that if two finite groups are isoclinic, they have the same
proportions of degrees of irreducible representations, and in
particular, if they have the same order, then they have the same
multiset of degrees of irreducible representations, and therefore they
have isomorphic group algebras over the field of complex numbers.

In particular, suppose $G_1$ and $G_2$ are isoclinic and both are
finite $p$-groups of the same order that are Lazard Lie groups. Denote
by $L_1$ the Lazard Lie ring of $G_1$ and denote by $\hat{L_1}$ the
Pontryagin dual to $L_1$. In order to find the degrees of irreducible
representations of $G_1$, we consider its coadjoint representation
(action on $\hat{L_1}$). Note that this action factors through the
group $G_1/Z(G_1) \cong \operatorname{Inn}(G_1)$. In particular, since
$G_2$ is isoclinic to $G_1$, we can also view the coadjoint action as
an action of $G_2$ on $\hat{L_1}$. Moreover, the sizes of the orbits
here are the squares of the degrees of irreducible representations of
$G_1$, and hence also of $G_2$.

This has an important implication, namely, that if our goal behind
using the Kirillov orbit method is solely to find the degrees of the
irreducible representations rather than determine the actual
irreducible representations, then we can use the Lazard Lie ring of
any isoclinic group. This suggests that we might be able to generalize
the method to the situation of the Lazard correspondence up to
isoclinism.

In Section \ref{sec:lcuti-adjoint}, we noted that if $G$ and $L$ are
in global class $(c + 1)$ Lazard correspondence up to isoclinism, then
we can define an adjoint action of $G$ on $L$. We can use this to
obtain a {\em coadjoint action} of $G$ on the Pontryagin dual
$\hat{L}$.

This leads to the following conjecture.

\begin{conjecture}[Kirillov orbit method for correspondence up to isoclinism]
  Suppose $p$ is a prime number, $G$ is a finite $p$-group of
  nilpotency class $p$, and $L$ is a finite $p$-Lie ring of nilpotency
  class $p$ such that $G$ and $L$ are in global class $p$ Lazard
  correspondence up to isoclinism, and such that $G$ and $L$ {\em have
    the same order}. Consider the coadjoint action of $G$ on the
  Pontryagin dual $\hat{L}$ described above. The sizes of the orbits
  for this coadjoint action are the squares of the degrees of
  irreducible representations.
\end{conjecture}

Note that, even if the conjecture were true, the method would be
weaker than the actual Kirillov orbit method, because the actual
Kirillov orbit method can be used to {\em find explicitly the
  characters of the irreducible representations}. However, this method
can only provide the degrees of the irreducible representations. It
cannot reveal the characters themselves because the characters are not
invariant under isoclinism. In fact, this method can {\em only} reveal
isoclinism-invariant information.
\subsection{Other possibilities: the use of multiplicative Lie rings}

In \cite{Ellis93}, Graham Ellis defined {\em multiplicative Lie
  rings}, which have been further considered in
\cite{homologyofmultiplicativeLierings}. The theory of multiplicative
Lie rings is powerful enough to encapsulate both the theory of groups
and the theory of Lie rings. This theory is, however, relatively
non-standard and not sufficiently well-developed in the literature, so
we do not use this framework in the document. It would be a
potentially interesting exercise to reformulate the results and proofs
presented here in the language of multiplicative Lie rings.

%% Some information about
%% multiplicative Lie rings and how they can be used to obtain an
%% alternative perspective on the results here can be found in the
%% Appendix, Section \ref{appsec:multiplicative-lie-rings}.

