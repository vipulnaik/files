\documentclass[10pt]{amsart}

%Packages in use
\usepackage{fullpage, hyperref, vipul, enumerate}

%Title details
\title{Class quiz solutions: Wednesday February 6: Multivariable limit computations}
\author{Math 195, Section 59 (Vipul Naik)}
%List of new commands

\begin{document}
\maketitle

\section{Performance review}

$26$ people took this quiz. The score distribution was as follows:

\begin{itemize}
\item Score of $0$: $1$ person.
\item Score of $1$: $2$ people.
\item Score of $2$: $9$ people.
\item Score of $3$: $10$ people.
\item Score of $4$: $4$ people.
\end{itemize}

The question-wise answers and performance review were as follows:

\begin{enumerate}
\item Option (A): $18$ people
\item Option (D): $24$ people
\item Option (B): $9$ people
\item Option (B): $15$ people
\end{enumerate}
\section{Solutions}

\begin{enumerate}
\item Consider the function $f(x,y) := x \sin(1/(x^2 + y^2))$, defined
  on all points other than the point $(0,0)$. What is the limit of the
  function at $(0,0)$?

  \begin{enumerate}[(A)]
  \item $0$
  \item $1/\sqrt{2}$
  \item $1$
  \item The limit is undefined, because the expression becomes unbounded around $0$.
  \item The limit is undefined, because the expression is oscillatory
    around $0$.
  \end{enumerate}

  {\em Answer}: Option (A)

  {\em Explanation}: We know that $\sin(1/(x^2 + y^2))$ is bounded in
  $[-1,1]$, and $x \to 0$ as we approach the origin. We thus get the
  product of something approaching $0$ and something bounded, which is
  therefore $0$.

  {\em Performance review}: $18$ out of $26$ got this. $7$ chose (E),
  $1$ chose (D).

  {\em Historical note (last time)}: $8$ out of $22$ people got this
  correct. $12$ chose (E). $1$ each chose (C) and (D).

 Some failed to note the $x$ on the outside. Others thought that the
  behavior of $\sin(1/(x^2 + y^2))$ is like the behavior of $1/(x^2 +
  y^2)$ near the origin. This is not true. Near the origin, $1/(x^2 +
  y^2)$ approaches $\infty$ but $\sin$ of the same quantity is
  oscillatory. Note also that $\sin$ cannot be stripped off because
  the input to it is not going to $0$.
\item The typical $\varepsilon-\delta$ definition of limit in two
  dimensions makes use of open disks centered at the points on the
  domain and range side, where the open disk is the interior region
  bounded by a circle centered at the point. Which other geometric
  shapes can we use instead of a circle of specified radius centered
  at the point?

  \begin{enumerate}[(A)]
  \item A square of specified side length centered at the point
  \item An equilateral triangle of specified side length centered at the point
  \item A regular hexagon of specified side length centered at the point
  \item Any of the above
  \item None of the above
  \end{enumerate}

  {\em Answer}: Option (D)

  {\em Explanation}: Basically, any shape that is bounded both from
  inside and from outside by a circle will do.

  {\em Performance review}: $24$ out of $26$ got this. $1$ each chose
  (B) and (E).

  {\em Historical note (last time)}: $12$ out of $22$ people got this
  correct. $6$ chose (A), $4$ chose (E).

\item Here's a quick recap of the limit definition for a function of a
  vector variable. We say that $\displaystyle \lim_{\mathbf{x} \to \mathbf{c}}
  f(\mathbf{x}) = L$ if for every $\varepsilon > 0$ there exists
  $\delta > 0$ such that for all $\mathbf{x}$ satisfying $0 <
  |\mathbf{x} - \mathbf{c}| < \delta$, we have $|f(\mathbf{x}) - L| <
  \varepsilon$. We define $|\mathbf{x} - \mathbf{c}|$ as the Euclidean
  norm of $\mathbf{x} - \mathbf{c}$ where the Euclidean norm of a
  vector is the square root of the sum of the squares of its
  coordinates.

  We could replace the Euclidean norm by other measurements. For
  instance, we could use:

  (i) The {\em sum} of the absolute values of the coordinates of
  $\mathbf{x} - \mathbf{c}$.

  (ii) The {\em maximum} of the absolute values of the coordinates of
  $\mathbf{x} - \mathbf{c}$.

  (iii) The {\em minimum} of the absolute values of the coordinates of
  $\mathbf{x} - \mathbf{c}$.

  For any of (i) - (iii), we could replace $|\mathbf{x} - \mathbf{c}|$
  in our current definition of limit with that notion. The question
  is: for which of the replacements will our new notion of limit be
  the same as the old one? The deeper idea here is that limit depends
  upon a concept of what it means for two points to be close. So
  another way of phrasing the question is: which of the notions
  (i)-(iii) capture the same notion of closeness as the usual
  Euclidean distance?

  \begin{enumerate}[(A)]
  \item All of (i), (ii), and (iii).
  \item (i) and (ii) but not (iii).
  \item (i) and (iii) but not (ii).
  \item Only (i).
  \item None of (i), (ii), or (iii).
  \end{enumerate}

  {\em Answer}: Option (B)

  {\em Explanation}: The sum of the absolute values of the coordinates
  of a vector is small if and only if all the absolute values of the
  coordinates are small. Similarly, the maximum of the absolute values
  of the coordinates is small if and only if all the coordinates are
  small. On the other hand, the minimum could be small even if some of
  the coordinates are very large. For this reason, the minimum does
  not capture the correct notion of ``closeness'' -- in the notion of
  closeness it gives, very far-away points can appear close merely
  because they are close in one coordinate.

  {\em Performance review}: $9$ out of $26$ people got this. $16$
  chose (C), $1$ chose (D).

\item Suppose $f$ is a function of two variables $x,y$ and is defined
  on the whole $xy$-plane. Consider three conditions: (i) $f$ is
  continuous on the whole $xy$-plane, (ii) for every fixed value $x =
  x_0$, the function $y \mapsto f(x_0,y)$ is continuous in $y$ for all
  $y \in \R$, (iii) for every fixed value $y = y_0$, the function $x
  \mapsto f(x,y_0)$ is continuous in $x$ for all $x \in \R$, (iv) the
  function $t \mapsto f(p(t),q(t))$ is continuous for all $t \in \R$
  whenever $p$ and $q$ are both constant or linear functions (in other
  words, the restriction of $f$ to any straight line in $\R^2$ is
  continuous).

  Which of the following correctly describes the implications between
  (i), (ii), (iii), and (iv)?

  \begin{enumerate}[(A)]
  \item (i) implies both (ii) and (iii), and (ii) and (iii) together imply (iv).
  \item (i) implies (iv), and (iv) implies both (ii) and (iii).
  \item (iv) implies (ii) and (iii), and (ii) and (iii) together imply (i).
  \item (iv) implies (i), and (i) implies both (ii) and (iii).
  \item (ii) and (iii) together imply (iv), and (iv) implies (i).
  \end{enumerate}
 
  {\em Answer}: Option (B)

  {\em Explanation}: Continuous implies continuous in every direction,
  linear and curved, hence (i) implies (iv) (we can also think of this
  as a result of composition of continuous functions being
  continuous). (iv) implies (ii) and (iii) because (ii) and (iii) are
  continuous from specific linear directions, namely the directions
  parallel to the coordinate axes.

  That the reverse implicationa fail is covered in the lecture notes
  and in videos.

  {\em Performance review}: $15$ out of $26$ got this. $4$ chose (A),
  $3$ each chose (D) and (E). $1$ left the question blank.
\end{enumerate}

\end{document}