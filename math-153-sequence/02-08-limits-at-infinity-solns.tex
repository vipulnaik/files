\documentclass[10pt]{amsart}

%Packages in use
\usepackage{fullpage, hyperref, vipul, enumerate}

%Title details
\title{Take-home class quiz solutions: due February 8: Limits at infinity and improper integral}
\author{Math 153, Section 55 (Vipul Naik)}
%List of new commands

\begin{document}
\maketitle

\section{Performance review}

$11$ people took this $11$-question quiz. The score distribution was
as follows:

\begin{itemize}
\item Score of $2$: $1$ person
\item Score of $4$: $1$ person
\item Score of $6$: $2$ people
\item Score of $7$: $1$ person
\item Score of $8$: $1$ person
\item Score of $9$: $1$ person
\item Score of $10$: $1$ person
\item Score of $11$: $3$ people
\end{itemize}

The mean score was about $7.7$.

The question wise answers and performance were as follows:

\begin{enumerate}
\item Option (B): $10$ people
\item Option (B): $10$ people
\item Option (E): $7$ people
\item Option (D): $6$ people
\item Option (B): $5$ people
\item Option (D): $8$ people
\item Option (C): $10$ people
\item Option (A): $11$ people
\item Option (C): $7$ people
\item Option (D): $7$ people
\item Option (A): $4$ people
\end{enumerate}

\section{Solutions}

\begin{enumerate}
\item If $\lim_{x \to \infty} f(x) = L$ for some finite $L$, this tells
  us that the graph of $f$ has a:

  \begin{enumerate}[(A)]
  \item vertical asymptote
  \item horizontal asymptote
  \item vertical tangent
  \item horizontal tangent
  \item vertical cusp
  \end{enumerate}

  {\em Answer}: Option (B)

  {\em Explanation}: Just by definition.

  {\em Performance review}: $10$ out of $11$ got this correct. $1$
  chose (D).

  {\em Historical note (last year)}: $23$ out of $26$ people got this
  correct. $1$ person each chose (A), (C), and (D).

  {\em Action point}: Everybody should get this correct!
\item If $\lim_{x \to \infty} f(x) = L$ and $\lim_{x \to \infty} f'(x)
  = M$, where both $L$ and $M$ are finite, then:

  \begin{enumerate}[(A)]
  \item $L = 0$ but $M$ need not be zero
  \item $M = 0$ but $L$ need not be zero
  \item Both $L$ and $M$ must be zero.
  \item Neither $L$ nor $M$ need be zero.
  \item At least one of $L$ and $M$ must be zero, but it could be
    either one.
  \end{enumerate}

  {\em Answer}: Option (B)

  {\em Explanation}: If $M$ were finite and nonzero, $f$ would have to
  go to $+\infty$ if $M$ were positive and to $-\infty$ if $M$ were
  negative, because $f$ would be growing/decaying at a rate that was
  bounded from both above and below by linear functions.

  Consider the following $\epsilon-\delta$ definition of limit at
  $\infty$: $\lim_{x \to \infty} f(x) = L$ if for all $\epsilon > 0$,
  there exists $a \in \R$ such that for all $x > a$, $|f(x) - L| <
  \epsilon$. 

  {\em Performance review}: $10$ out of $11$ got this correct. $1$
  chose (D).

  {\em Historical note (last year)}: $17$ out of $26$ people got this
  correct. $6$ people chose (D), $2$ people chose (E), and $1$ person
  chose (C).
\item What is the smallest $a$ that can be picked for the function $f
  = \arctan$ with $L$ being its limit at $\infty$ and $\epsilon =
  \pi$?
  \begin{enumerate}[(A)]
  \item $\sqrt{3}$
  \item $1$
  \item $0$
  \item $-1$
  \item There is no smallest $a$. Any $a \in \R$ will do.
  \end{enumerate}

  {\em Answer}: Option (E)

  {\em Explanation}: The function $\arctan$ has range
  $(-\pi/2,\pi/2)$, which is within the interval $(\pi/2 - \pi, \pi/2
  + \pi)$. Thus, for all real numbers, the value of the $\arctan$
  function is within the specified range.

  {\em Performance review}: $7$ out of $11$ got this correct. $2$
  chose (B), $1$ chose (C) and $1$ chose (D).

  {\em Historical note (last year)}: $14$ out of $26$ people got this
  correct. $5$ people chose (B), $3$ chose (C), $2$ each chose (A) and
  (D).

\item What is the smallest $a$ that can be picked for the function $f
  = \arctan$ with $L$ being its limit at $\infty$ and $\epsilon =
  \pi/6$?
  \begin{enumerate}[(A)]
  \item $1/2$
  \item $1/\sqrt{3}$
  \item $1$
  \item $\sqrt{3}$
  \item $2$
  \end{enumerate}

  {\em Answer}: Option (D)

  {\em Explanation}: We need $a$ such that for $x > a$, $\arctan x \in
  (\pi/2 - \pi/6, \pi/2 + \pi/6) = (\pi/3,2\pi/3)$. The right value of
  $a$ is thus $\tan(\pi/3) = \sqrt{3}$.

  {\em Performance review}: $6$ out of $11$ got this correct. $2$
  chose (E), $1$ each chose (A), (B), and (C).

  {\em Historical note (last year)}: $10$ out of $26$ people got this
  correct. $11$ people chose (B), indicating that either they took
  $\tan(\pi/3)$ wrong, or they computed $\tan(\pi/6)$ and did not
  perform the subtraction step $\pi/2 - \pi/6$. $4$ people chose (A),
  $1$ person chose (C), and $1$ person left the question blank.
\item Suppose $f(x) := p(x)/q(x)$ is a rational function in reduced form
  (i.e., the numerator and denominator are relatively prime) and
  $\lim_{x \to c} f(x) = \infty$. Which of the following can you
  conclude about $f$?
  \begin{enumerate}[(A)]
  \item $x - c$ divides $p(x)$, and the largest $r$ such that $(x -
    c)^r$ divides $p(x)$ is even.
  \item $x - c$ divides $q(x)$, and the largest $r$ such that $(x -
    c)^r$ divides $q(x)$ is even.
  \item $x - c$ divides $p(x)$, and the largest $r$ such that $(x -
    c)^r$ divides $p(x)$ is odd.
  \item $x - c$ divides $q(x)$, and the largest $r$ such that $(x -
    c)^r$ divides $q(x)$ is odd.
  \item $x - c$ does not divide either $p(x)$ or $q(x)$.
  \end{enumerate}

  {\em Answer}: Option (B)

  {\em Explanation}: We need $x - c$ to divide $q(x)$ for the
  denominator to blow up as $x \to c$. The power needs to be even to
  get the {\em same} sign of infinity for both left-sided and
  right-sided approach. $1/x^2$ at $c = 0$ is one example.

  {\em Performance review}: $5$ out of $11$ got this correct. $5$
  chose (C), $1$ chose (A).

  {\em Historical note (last year)}: $9$ out of $26$ people got this
  correct. $7$ chose (A), $4$ chose (C), $3$ chose (D), $2$ chose (E),
  and $1$ left the question blank.

  {\em Action point}: Please review this solution, make sure you
  understand it, and if you were convinced of another answer, debug
  the reasoning or examples that misled you.

\item Suppose $f(x) := p(x)/q(x)$ is a rational function in reduced
  form (i.e., the numerator and denominator are relatively prime) and
  $\lim_{x \to c^-} f(x) = \infty$ and $\lim_{x \to c^+} f(x) =
  -\infty$. Which of the following can you conclude about $f$?
  \begin{enumerate}[(A)]
  \item $x - c$ divides $p(x)$, and the largest $r$ such that $(x -
    c)^r$ divides $p(x)$ is even.
  \item $x - c$ divides $q(x)$, and the largest $r$ such that $(x -
    c)^r$ divides $q(x)$ is even.
  \item $x - c$ divides $p(x)$, and the largest $r$ such that $(x -
    c)^r$ divides $p(x)$ is odd.
  \item $x - c$ divides $q(x)$, and the largest $r$ such that $(x -
    c)^r$ divides $q(x)$ is odd.
  \item $x - c$ does not divide either $p(x)$ or $q(x)$.
  \end{enumerate}

  {\em Answer}: Option (D)

  {\em Explanation}: We need $x - c$ to divide $q(x)$ for the
  denominator to blow up as $x \to c$. The power needs to be even to
  get {\em opposite} signs of infinity for left-sided and
  right-sided approach. $-1/x$ at $c = 0$ is one example.

  {\em Performance review}: $8$ out of $11$ got this correct. $1$ each
  chose (A), (C), and (E).

  {\em Historical note (last year)}: $9$ out of $26$ people got this
  correct. $8$ people chose (C), $5$ people chose (E), $2$ people
  chose (A), $1$ chose (B), and $1$ left the question blank.

  {\em Action point}: Please review this solution, make sure you
  understand it, and if you were convinced of another answer, debug
  the reasoning or examples that misled you.

  Suppose $F$ is a function of two real variables, say $x$
  and $t$, so $F(x,t)$ is a real number for $x$ and $t$ restricted to
  suitable open intervals in the real number. Suppose, further, that
  $F$ is jointly continuous (whatever that means) in $x$ and $t$.

  Define $f(t) := \int_0^\infty F(x,t) \, dx$. Here, while doing the
  integration, $t$ is treated as a constant. $x$, the variable of
  integration, is being integrated on $[0,\infty)$.
    
  Suppose further that $f$ is defined and continuous for $t$ in
  $(0,\infty)$. {\em Note that similar computations we did in the
    midterm review session involved integration from $-\infty$ to
    $\infty$}.

  In the next few questions, you are asked to compute the function $f$
  explicitly given the function $F$, for $t \in (0,\infty)$.

\item $F(x,t) := e^{-tx}$. Find $f$.

  \begin{enumerate}[(A)]
  \item $f(t) = e^{-t}/t$
  \item $f(t) = e^t/t$
  \item $f(t) = 1/t$
  \item $f(t) = -1/t$
  \item $f(t) = -t$
  \end{enumerate}

  {\em Answer}: Option (C)

  {\em Explanation}: The integral becomes
  $[-e^{-tx}/t]_0^\infty$. Plugging in at $\infty$ gives $0$ and
  plugging in at $0$ gives $-1/t$. Since the value at $0$ is being
  subtracted, we eventually get $1/t$.

  Note that the answer must be positive for the simple reason that we
  are integrating a positive function from left to right across an
  interval.

  {\em Performance review}: $10$ out of $11$ got this correct. $1$
  chose (A).

  {\em Historical note (last year)}: $17$ out of $25$ people got this
  correct. $4$ chose (A), $3$ chose (D), and $1$ chose (E).

\item $F(x,t) := 1/(t^2 + x^2)$. Find $f$.

  \begin{enumerate}[(A)]
  \item $f(t) = \pi/(2t)$
  \item $f(t) = \pi/t$
  \item $f(t) = 2\pi/t$
  \item $f(t) = \pi t$
  \item $f(t) = 2\pi t$
  \end{enumerate}

  {\em Answer}: Option (A)

  {\em Explanation}: We get $[(1/t)\arctan(x/t)]_0^\infty$. The
  evaluation at $\infty$ gives $\pi/(2t)$ and the evaluation at $0$
  gives $0$. Subtracting, we get $\pi/(2t)$.

  {\em Performance review}: Everybody got this correct.

  {\em Historical note (last year)}: $17$ out of $25$ got this correct. $5$
  chose (B), $2$ chose (D), $1$ chose (C).

\item $F(x,t) := 1/(t^2 + x^2)^2$. Find $f$.

  \begin{enumerate}[(A)]
  \item $f(t) = \pi/t^3$
  \item $f(t) = \pi/(2t^3)$
  \item $f(t) = \pi/(4t^3)$
  \item $f(t) = \pi/(8t^3)$
  \item $f(t) = 3\pi/(8t^3)$
  \end{enumerate}

  {\em Answer}: Option (C)

  {\em Explanation}: Put in $\theta = \arctan(x/t)$. Substitute, and
  we get $(1/t^3) \int_0^{\pi/2} \cos^2\theta \,
  d\theta$. Integrating, we get $[\theta/2t^3 +
  \sin(2\theta)/4t^3]_0^{\pi/2}$. The trigonometric part vanishes
  between limits, and we are left with $\pi/(4t^3)$

  {\em Performance review}: $7$ out of $11$ got this correct. $3$
  chose (B), $1$ chose (D).

  {\em Historical note (last year)}: $15$ out of $25$ people got this
  correct. $5$ chose (B), $2$ chose (A), $1$ each chose (D) and (E),
  $1$ left the question blank.
\item $F(x,t) = \exp(-(tx)^2)$. Use that $\int_0^\infty \exp(-x^2) \, dx=
  \sqrt{\pi}/2$. Find $f$.

  \begin{enumerate}[(A)]
  \item $f(t) = t^2\sqrt{\pi}/2$
  \item $f(t) = t\sqrt{\pi}/2$
  \item $f(t) = \sqrt{\pi}/2$
  \item $f(t) = \sqrt{\pi}/(2t)$
  \item $f(t) = \sqrt{\pi}/(2t^2)$
  \end{enumerate}

  {\em Answer}: Option (D)

  {\em Explanation}: Put $u = tx$, get a $1/t$ on the outside, giving
  $(1/t) \int_0^\infty \exp(-u^2) \, du$.

  {\em Performance review}: $7$ out of $11$ got this correct. $2$
  chose (A), $2$ chose (E).

  {\em Historical note (last year)}: $12$ out of $25$ people got the question
  correct. $6$ chose (E), $3$ chose (C), $2$ each chose (A) and (B).

\item In the same general setup as above (but with none of these
  specific $F$s), which of the following is a {\em sufficient}
  condition for $f$ to be an increasing function of $t$?

  \begin{enumerate}[(A)]
  \item $t \mapsto F(x_0,t)$ is an increasing function of $t$ for
    every choice of $x_0 \ge 0$.
  \item $x \mapsto F(x,t_0)$ is an increasing function of $x$ for
    every choice of $t_0 \in (0,\infty)$.
  \item $t \mapsto F(x_0,t)$ is a decreasing function of $t$ for
    every choice of $x_0 \ge 0$.
  \item $x \mapsto F(x,t_0)$ is a decreasing function of $x$ for
  every choice of $t_0 \in (0,\infty)$.
  \item None of the above.
  \end{enumerate}

  {\em Answer}: Option (A)

  {\em Explanation}: If $F$ is increasing in $t$ for every value of
  $x_0$, then that means that as $t$ gets bigger, the function $F$
  being integrated gets bigger everywhere in $x$, i.e., if $t_1 <
  t_2$, then $F(t_1,x_0) < F(t_2,x_0)$ for every $x_0 \ge 0$. The
  integral for the larger value $t_2$ must therefore also be
  bigger. (We looked at this stuff in Section 5.8 of the book).

  {\em Performance review}: $4$ out of $11$ got this correct. $4$
  chose (B), $1$ each chose (C), (D), and (E).

  {\em Historical note (last year)}: $4$ out of $25$ got the question
  correct. $10$ chose (B), $5$ chose (E), $3$ chose (C), $2$ chose
  (D), and $1$ left the question blank.

  (A) was the ``obvious'' choice -- people may have tried to seek more
  subtletly in the question than it had.

  {\em Action point}: This should not trip {\em anybody} in the future.

\end{enumerate}

\end{document}