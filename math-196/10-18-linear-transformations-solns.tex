\documentclass[10pt]{amsart}

%Packages in use
\usepackage{fullpage, hyperref, vipul, enumerate}

%Title details
\title{Diagnostic in-class quiz solutions: due Friday October 18: Linear transformations}
\author{Math 196, Section 57 (Vipul Naik)}
%List of new commands

\begin{document}
\maketitle

\section{Performance review}

27 people took this 5-question quiz. The score distribution was as follows:

\begin{itemize}
\item Score of 1: 2 people
\item Score of 2: 4 people
\item Score of 3: 5 people
\item Score of 4: 8 people
\item Score of 5: 8 people
\end{itemize}

The mean score was 3.59 out of 5.

The question-wise answers and performance review were as follows:

\begin{enumerate}
\item Option (B): 24 people
\item Option (B): 22 people
\item Option (D): 19 people
\item Option (E): 19 people
\item Option (D): 13 people
\end{enumerate}
\section{Solutions}

{\bf PLEASE DO {\em NOT} DISCUSS ANY QUESTIONS.}

The quiz covers basics related to linear transformations (notes titled
{\tt Linear transformations}, corresponding section in the book
Section 2.1). Explicitly, the quiz covers:

\begin{itemize}
\item Representation of a linear transformation using a matrix, and
  identifying the domain and co-domain in terms of the row and column
  counts of the matrix.
\item Relationship between injectivity, surjectivity, rank, row count,
  and column count.
\item Relationship between the entries of the matrix and the images of
  the standard basis vectors under the corresponding linear
  transformation.
\end{itemize}

The question are fairly easy if you understand the material. But it's
important that you be able to answer them, otherwise what we study
later will not make much sense.

\begin{enumerate}
\item {\em Do not discuss this!}: Which of the following correctly
  describes a $m \times n$ matrix?

  \begin{enumerate}[(A)]
  \item There are $m$ rows, and each row gives a vector with $m$
    coordinates. There are $n$ columns, and each column gives a vector
    with $n$ coordinates.
  \item There are $m$ rows, and each row gives a vector with $n$
    coordinates. There are $n$ columns, and each column gives a vector
    with $m$ coordinates.
  \item There are $n$ rows, and each row gives a vector with $m$
    coordinates. There are $m$ columns, and each column gives a vector
    with $n$ coordinates.
  \item There are $n$ rows, and each row gives a vector with $n$
    coordinates. There are $m$ columns, and each column gives a vector
    with $m$ coordinates.
  \end{enumerate}

  {\em Answer}: Option (B)

  {\em Explanation}: This should be obvious by looking at the
  matrix. For instance, a $2 \times 3$ matrix is of the form:

  $$\left[\begin{matrix} a & b & c \\ d & e & f \\\end{matrix}\right]$$

  {\em Performance review}: 24 out of 27 got this. 3 chose (A).

\item {\em Do not discuss this!}: For a $p \times q$ matrix $A$, we can
  define a linear transformation $T_A$ by $T_A(\vec{x}) :=
  A\vec{x}$. What type of linear transformation is $T_A$?

  \begin{enumerate}[(A)]
  \item $T_A$ is a linear transformation from $\R^p$ to $\R^q$
  \item $T_A$ is a linear trnasformation from $\R^q$ to $\R^p$
  \item $T_A$ is a linear transformation from $\R^{\max\{p,q \}}$ to $\R^{\min\{p,q\}}$
  \item $T_A$ is a linear transformation from $\R^{\min\{p,q\}}$ to $\R^{\max\{p,q\}}$
  \end{enumerate}

  {\em Answer}: Option (B)

  {\em Explanation}: The way matrix-vector multiplication works, a $p
  \times q$ matrix multiplies with a $q \times 1$ vector to give a $p
  \times 1$ vector. Thus the linear transformation takes as input a
  $q$-dimensional vector and gives as output a $p$-dimensional vector,
  and is hence a transformation from $\R^q$ to $\R^p$.

  {\em Performance review}: 22 out of 27 got this. 3 chose (A), 2 chose (C).

\item {\em Do not discuss this!}: With the same notation as for the
  preceding question, which of the following is true?

  \begin{enumerate}[(A)]
  \item If $p < q$, $T_A$ must be injective
  \item If $p > q$, $T_A$ must be injective
  \item If $p = q$, $T_A$ must be injective
  \item If $p < q$, $T_A$ cannot be injective
  \item If $p > q$, $T_A$ cannot be injective
  \end{enumerate}

  {\em Answer}: Option (D)

  {\em Explanation}: For the linear transformation $T_A$ to be
  injective, the matrix $A$ needs to have full column rank $q$,
  because that is what it means for there to be no non-leading
  variables and for the solution to therefore be unique if it exists
  (see the lecture notes for more). In other words, we require that
  the matrix have rank $q$. However, we know that the rank of a matrix
  is at most equal to the minimum of the number of rows and the number
  of columns. Thus, if $p < q$, the matrix cannot have full column
  rank, and the linear transformation cannot be injective.

  Intuitively, the linear transformation goes from $\R^q$ to $\R^p$,
  so in order for it to be injective, the target space should be at
  least as big as the domain. Thus, $p < q$ is incompatible with
  injectivity.

  {\em Performance review}: 19 out of 27 got this. 4 chose (E), 3
  chose (C), 1 left the question blank.

\item {\em Do not discuss this!}: With the same notation as for the
  previous two questions, which of the following is true?

  \begin{enumerate}[(A)]
  \item If $p < q$, $T_A$ must be surjective
  \item If $p > q$, $T_A$ must be surjective
  \item If $p = q$, $T_A$ must be surjective
  \item If $p < q$, $T_A$ cannot be surjective
  \item If $p > q$, $T_A$ cannot be surjective
  \end{enumerate}

  {\em Answer}: Option (E)

  {\em Explanation}: For the linear transformation $T_A$ to be
  surjective, the matrix $A$ needs to have full row rank $p$, because
  we want the system to always be consistent and therefore we want
  that there should be no zero rows in the rref. We also know that the
  rank of the matrix is at most equal to the minimum of the number of
  rows and number of columns. Therefore, if $p > q$, $A$ cannot have
  full row rank and $T_A$ cannot be surjective.

  {\em Performance review}: 19 out of 27 got this. 5 chose (D), 2
  chose (C).
\item {\em Do not discuss this!}: With the same notation as for the
  last three questions, which of the following is true?

  \begin{enumerate}[(A)]
  \item The rows of $A$ are the images under $T_A$ of the standard
    basis vectors of $\R^p$.
  \item The columns of $A$ are the images under $T_A$ of the standard
    basis vectors of $\R^p$.
  \item The rows of $A$ are the images under $T_A$ of the standard
    basis vectors of $\R^q$.
  \item The columns of $A$ are the images under $T_A$ of the standard
    basis vectors of $\R^q$.
  \end{enumerate}

  {\em Answer}: Option (D)

  {\em Explanation}: See the lecture notes for details. Note, however,
  that dimension considerations can get the answer here
  immediately. $T_A$ is a map from $\R^q$ to $\R^p$, so only Options
  (C) and (D) make sense from the domain perspective. Further, the
  rows of $A$ are $q$-dimensional whereas the columns are
  $p$-dimensional, so only the columns have the right dimension, thus
  making (D) the only legitimate option.

  {\em Performance review}: 13 out of 27 got this. 11 chose (D), 2
  chose (C), 1 chose (A).
\end{enumerate}
\end{document}
