\documentclass[10pt]{amsart}

%Packages in use
\usepackage{fullpage, hyperref, vipul, enumerate}

%Title details
\title{Diagnostic in-class quiz: due Friday November 8: Image and kernel (computational)}
\author{Math 196, Section 57 (Vipul Naik)}
%List of new commands

\begin{document}
\maketitle

Your name (print clearly in capital letters): $\underline{\qquad\qquad\qquad\qquad\qquad\qquad\qquad\qquad\qquad\qquad}$

{\bf PLEASE DO NOT DISCUSS {\em ANY} QUESTIONS.}

The questions here test for an understanding of the ideas covered in
the lecture notes titled {\tt Image and kernel of a linear
  transformation}. However, the format of presentation of the
questions in the quiz differs somewhat from that used in typical
linear algebra problems, so you need to think a bit before plugging
and chugging. The corresponding section of the book is Section 3.1.

All these questions can be solved without using any part of the
``toolkit'' of linear algebra, but they can be understood better and
more deeply using the ideas and methods of linear algebra.

\begin{enumerate}

\item {\em Do not discuss this!}: Consider the linear transformation
  $\operatorname{Avg}:\R^2 \to \R^2$ defined as:

  $$\operatorname{Avg} = \left[ \begin{matrix} x \\ y \\\end{matrix}\right] \mapsto \left[ \begin{matrix} (x + y)/2 \\ (x + y)/2 \\\end{matrix}\right]$$

  What can we say about the kernel and image of $\operatorname{Avg}$?
  Note that in our descriptions of the kernel and the image below, we
  use $x$ to denote the first coordinate of the vector and $y$ to
  denote the second coordinate of the vector.

  {\em Note}: One way you can do that is to write the matrix for
  $\operatorname{Avg}$, but in this particular situation, it's easiest
  to just do things directly.

  \begin{enumerate}[(A)]
  \item The kernel is the zero subspace and the image is all of $\R^2$
  \item The kernel is the line $y = x$ and the image is also the line
    $y = x$
  \item The kernel is the line $y = x$ and the image is the line $y =
    -x$
  \item The kernel is the line $y = -x$ and the image is also the line
    $y = -x$
  \item The kernel is the line $y = -x$ and the image is the line $y = x$
  \end{enumerate}

  \vspace{0.1in}
  Your answer: $\underline{\qquad\qquad\qquad\qquad\qquad\qquad\qquad}$
  \vspace{0.1in}

\item {\em Do not discuss this!}: Consider the {\em average of other
  two} linear transformation $\nu:\R^3 \to \R^3$ given as follows:

  $$\nu = \left[ \begin{matrix} x \\ y \\ z \\ \end{matrix}\right] \mapsto \left[ \begin{matrix} (y + z)/2 \\ (z + x)/2 \\ (x + y)/2 \\ \end{matrix} \right]$$

  What can we say about the kernel and image of $\nu$?

  Note that in our descriptions of the kernel and the image below, we
  use $x$ to denote the first coordinate of the vector, $y$ to denote the second
  coordinate of the vector, and $z$ to denote the third coordinate of
  the vector.

  {\em Note}: This can both be reasoned directly (without any
  knowledge of linear algebra) or alternatively it can be done by
  writing the matrix of $\nu$ and computing its rank, image, and
  kernel.

  \begin{enumerate}[(A)]
  \item The kernel is the zero subspace and the image is all of $\R^3$
  \item The kernel is the line $x = y = z$ (one-dimensional) and the
    image is the plane $x + y + z = 0$ (two-dimensional)
  \item The kernel is the plane $x + y + z = 0$ (two-dimensional) and
    the image is the line $x = y = z$ (one-dimensional)
  \item The kernel is the plane $x = y = z$ (two-dimensional) and the
    image is the line $x + y + z = 0$ (one-dimensional)
  \item The kernel is the line $x + y + z = 0$ (one-dimensional) and
    the image is the plane $x = y = z$ (two-dimensional)
  \end{enumerate}

  \vspace{0.1in}
  Your answer: $\underline{\qquad\qquad\qquad\qquad\qquad\qquad\qquad}$
  \vspace{0.1in}

\item {\em Do not discuss this!}: Consider the {\em difference of
  other two} linear transformation $\mu:\R^3 \to \R^3$ given by:

  $$\mu = \left[ \begin{matrix} x \\ y \\ z \\ \end{matrix}\right] \mapsto \left[ \begin{matrix} y - z \\ z - x \\ x - y \\ \end{matrix} \right]$$

  What can we say about the kernel and image of $\mu$?

  Note that in our descriptions of the kernel and the image below, we
  use $x$ to denote the first coordinate of the vector, $y$ to denote
  the second coordinate of the vector, and $z$ to denote the third
  coordinate of the vector.

  {\em Note}: This can both be reasoned directly (without any
  knowledge of linear algebra) or alternatively it can be done by
  writing the matrix of $\mu$ and computing its rank, image, and
  kernel.

  \begin{enumerate}[(A)]
  \item The kernel is the zero subspace and the image is all of $\R^3$
  \item The kernel is the line $x = y = z$ (one-dimensional) and the
    image is the plane $x + y + z = 0$ (two-dimensional)
  \item The kernel is the plane $x + y + z = 0$ (two-dimensional) and
    the image is the line $x = y = z$ (one-dimensional)
  \item The kernel is the plane $x = y = z$ (two-dimensional) and the
    image is the line $x + y + z = 0$ (one-dimensional)
  \item The kernel is the line $x + y + z = 0$ (one-dimensional) and
    the image is the plane $x = y = z$ (two-dimensional)
  \end{enumerate}

  \vspace{0.1in}
  Your answer: $\underline{\qquad\qquad\qquad\qquad\qquad\qquad\qquad}$
  \vspace{0.1in}
\end{enumerate}

\end{document}
