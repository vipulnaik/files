\documentclass[10pt]{amsart}

%Packages in use
\usepackage{fullpage, hyperref, vipul, enumerate}

%Title details
\title{Take-home class quiz: due Friday November 1: Matrix multiplication and inversion: abstract behavior prediction}
\author{Math 196, Section 57 (Vipul Naik)}
%List of new commands

\begin{document}
\maketitle

Your name (print clearly in capital letters): $\underline{\qquad\qquad\qquad\qquad\qquad\qquad\qquad\qquad\qquad\qquad}$

{\bf PLEASE FEEL FREE TO DISCUSS {\em ALL} QUESTIONS.}

This quiz tests for {\em abstract behavior prediction} related to the
structure of matrices defined based on the operations of matrix
multiplication and inversion. It is based on part of the {\tt Matrix
  multiplication and inversion} notes and is related to Sections 2.3
and 2.4. It does not, however, test all aspects of that material.

To understand this abstract behavior, we will consider {\em
  nilpotent}, {\em invertible}, and {\em idempotent} matrices.

\begin{enumerate}

\item Suppose $A$ and $B$ are $n \times n$ matrices such that $B$ is
  invertible. Suppose $r$ is a positive integer. What can we say that
  $(BAB^{-1})^r$ definitely equals?

  \begin{enumerate}[(A)]
  \item $A^r$
  \item $BA^rB^{-1}$
  \item $B^rA^rB^{-r}$
  \item $B^rAB^{-r}$
  \item $BAB^{-1-r}$
  \end{enumerate}

   \vspace{0.1in}
   Your answer: $\underline{\qquad\qquad\qquad\qquad\qquad\qquad\qquad}$
   \vspace{0.1in}

\item Suppose $A$ and $B$ are $n \times n$ matrices ($n$ not too
  small) such that $(AB)^2 = 0$. What is the smallest $r$ for which we
  can conclude that $(BA)^r$ is definitely $0$?

  \begin{enumerate}[(A)]
  \item $1$
  \item $2$
  \item $3$
  \item $4$
  \item $5$
  \end{enumerate}

  \vspace{0.1in}
  Your answer: $\underline{\qquad\qquad\qquad\qquad\qquad\qquad\qquad}$
  \vspace{0.1in}

\item Suppose $n > 1$. A $n \times n$ matrix $A$ is termed {\em
  nilpotent} if there exists a positive integer $r$ such that $A^r$ is
  the zero matrix. It turns out that if $A$ is nilpotent, then $A^n =
  0$. Which of the following describes correctly the relationship
  between being invertible and being nilpotent for $n \times n$
  matrices?

  \begin{enumerate}[(A)]
  \item A matrix is nilpotent if and only if it is invertible.
  \item Every nilpotent matrix is invertible, but not every invertible matrix is nilpotent.
  \item Every invertible matrix is nilpotent, but not every nilpotent matrix is invertible.
  \item An invertible matrix may or may not be nilpotent, and a
    nilpotent matrix may or may not be invertible.
  \item A matrix cannot be both nilpotent and invertible.
  \end{enumerate}

  \vspace{0.1in}
  Your answer: $\underline{\qquad\qquad\qquad\qquad\qquad\qquad\qquad}$
  \vspace{0.1in}

\item Suppose $A$ and $B$ are $n \times n$ matrices. Which of the
  following is true? Please see Option (E) before answering.

  \begin{enumerate}[(A)]
  \item $AB$ is nilpotent if and only if $A$ and $B$ are both nilpotent.
  \item $AB$ is nilpotent if and only if at least one of $A$ and $B$ is nilpotent.
  \item If both $A$ and $B$ are nilpotent, then $AB$ is nilpotent, but
    $AB$ being nilpotent does not imply that both $A$ and $B$ are
    nilpotent.
  \item If $AB$ is nilpotent, then both $A$ and $B$ are
    nilpotent. However, both $A$ and $B$ being nilpotent does not
    imply that $AB$ is nilpotent.
  \item None of the above.
  \end{enumerate}

  \vspace{0.1in}
  Your answer: $\underline{\qquad\qquad\qquad\qquad\qquad\qquad\qquad}$
  \vspace{0.1in}

\item Suppose $A$ and $B$ are $n \times n$ matrices. Which of the
  following is true? Please see Option (E) before answering.

  \begin{enumerate}[(A)]
  \item $AB$ is invertible if and only if $A$ and $B$ are both invertible.
  \item $AB$ is invertible if and only if at least one of $A$ and $B$ is invertible.
  \item If both $A$ and $B$ are invertible, then $AB$ is invertible, but
    $AB$ being invertible does not imply that both $A$ and $B$ are
    invertible.
  \item If $AB$ is invertible, then both $A$ and $B$ are
    invertible. However, both $A$ and $B$ being invertible does not
    imply that $AB$ is invertible.
  \item None of the above.
  \end{enumerate}

  \vspace{0.1in}
  Your answer: $\underline{\qquad\qquad\qquad\qquad\qquad\qquad\qquad}$
  \vspace{0.1in}

\item Suppose $A$ and $B$ are $n \times n$ matrices. Which of the
  following is true? We call a $n \times n$ matrix {\em idempotent} if
  it equals its own square. Please see Option (E) before answering.

  \begin{enumerate}[(A)]
  \item $AB$ is idempotent if and only if $A$ and $B$ are both idempotent.
  \item $AB$ is idempotent if and only if at least one of $A$ and $B$ is idempotent.
  \item If both $A$ and $B$ are idempotent, then $AB$ is idempotent, but
    $AB$ being idempotent does not imply that both $A$ and $B$ are
    idempotent.
  \item If $AB$ is idempotent, then both $A$ and $B$ are
    idempotent. However, both $A$ and $B$ being idempotent does not
    imply that $AB$ is idempotent.
  \item None of the above.
  \end{enumerate}

  \vspace{0.1in}
  Your answer: $\underline{\qquad\qquad\qquad\qquad\qquad\qquad\qquad}$
  \vspace{0.1in}

\end{enumerate}
\end{document}
