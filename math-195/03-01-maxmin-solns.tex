\documentclass[10pt]{amsart}

%Packages in use
\usepackage{fullpage, hyperref, vipul, enumerate}

%Title details
\title{Take-home class quiz solutions: due Friday March 1: Max-min values: one-variable recall} 
\author{Math 195, Section 59 (Vipul Naik)}
%List of new commands

\begin{document}
\maketitle

\section{Performance review}

$25$ people took this $6$-question quiz. The score distribution was as
follows:

\begin{itemize}
\item Score of $2$: $2$ people
\item Score of $3$: $4$ people
\item Score of $4$: $7$ people
\item Score of $5$: $8$ people
\item Score of $6$: $4$ people
\end{itemize}

The question wise answers and performance review were as follows:

\begin{enumerate}
\item Option (D): $18$ people
\item Option (B): $10$ people
\item Option (B): $25$ people
\item Option (B): $15$ people
\item Option (D): $24$ people
\item Option (A): $16$ people
\end{enumerate}

\section{Solutions}

\begin{enumerate}

\item Suppose $f$ is a function defined on a closed interval
  $[a,c]$. Suppose that the left-hand derivative of $f$ at $c$ exists
  and equals $\ell$. Which of the following implications is {\bf true
  in general}?

  \begin{enumerate}[(A)]
  \item If $f(x) < f(c)$ for all $a \le x < c$, then $\ell < 0$.
  \item If $f(x) \le f(c)$ for all $a \le x < c$, then $\ell \le 0$.
  \item If $f(x) < f(c)$ for all $a \le x < c$, then $\ell > 0$.
  \item If $f(x) \le f(c)$ for all $a \le x < c$, then $\ell \ge 0$.
  \item None of the above is true in general.
  \end{enumerate}

  {\em Answer}: Option (D)

  {\em Explanation}: If $f(x) \le f(c)$ for all $a \le x < c$, then
  all difference quotients from the left are nonnegative. The limiting
  value, which is the left-hand derivative, is thus also
  nonnegative. See the lecture notes for more details.

  {\em The other choices}: Options (A) and (B) predict the wrong
  sign. Option (C) is incorrect because even though the difference
  quotients are all strictly positive, their limiting value could be
  $0$. For instance, $\sin x$ on $[0,\pi/2]$ or $x^3$ on $[-1,0]$.

  {\em Performance review}: $18$ out of $25$ got this. $4$ chose (E),
  $3$ chose (C).

  {\em Historical note 1}: $10$ out of $18$ people got this. $4$
  chose (E), $3$ chose (B), $1$ chose (C).

  {\em Historical note 2}: This question appeared in a 152 quiz, which
  $15$ people took. At the time, $8$ people got this correct. $5$
  people chose option (B) and $2$ people chose option (E). It is
  likely that the people who chose option (B) made a sign computation
  error.

\item Suppose $f$ is a continuous function defined on an open interval
  $(a,b)$ and $c$ is a point in $(a,b)$. Which of the following
  implications is {\bf true}?
  \begin{enumerate}[(A)]

  \item If $c$ is a point of local minimum for $f$, then there is a
    value $\delta > 0$ and an open interval $(c - \delta, c + \delta)
    \subseteq (a,b)$ such that $f$ is non-increasing on $(c -
    \delta,c)$ and non-decreasing on $(c,c+\delta)$.
  \item If there is a value $\delta > 0$ and an open interval $(c -
    \delta, c + \delta) \subseteq (a,b)$ such that $f$ is
    non-increasing on $(c - \delta,c)$ and non-decreasing on
    $(c,c+\delta)$, then $c$ is a point of local minimum for $f$.
  \item If $c$ is a point of local minimum for $f$, then there is a
    value $\delta > 0$ and an open interval $(c - \delta, c + \delta)
    \subseteq (a,b)$ such that $f$ is non-decreasing on $(c -
    \delta,c)$ and non-increasing on $(c,c+\delta)$.
  \item If there is a value $\delta > 0$ and an open interval $(c -
    \delta, c + \delta) \subseteq (a,b)$ such that $f$ is
    non-decreasing on $(c - \delta,c)$ and non-increasing on
    $(c,c+\delta)$, then $c$ is a point of local minimum for $f$.
  \item All of the above are true.
  \end{enumerate}

  {\em Answer}: Option (B).

  {\em Explanation}: Since $f$ is continuous, being non-increasing on
  $(c - \delta, c)$ implies being non-increasing on $(c -
  \delta,c]$. Similarly on the right side. In particular, this means
  that $f(c) \le f(x)$ for all $x \in (c - \delta, c + \delta)$,
  establishing $c$ as a point of local minimum.

  {\em Performance review}: $10$ out of $25$ got this. $12$ chose (A),
  $2$ chose (E), $1$ chose (D).

  {\em Historical note 1}: $7$ out of $18$ people got this
  correct. $5$ chose (A), $3$ each chose (C) and (D).

  {\em The other choices}: Options (C) and (D)
  have the wrong kind of increase/decrease. Option (A) is wrong,
  though counterexamples are hard to come by. The reason Option (A) is
  wrong is the core of the reason that the first-derivative test does
  not always work: the function could be oscillatory very close to the
  point $c$, so that even though $c$ is a point of local minimum, the
  function does not steadily become non-increasing to the left of
  $c$. The example discussed in the lecture notes is $|x|(2 +
  \sin(1/x))$.

  {\em Historical note 2}: This question appeared in a Math 152 quiz. At
  the time, $5$ out of $15$ people got this correct. $5$ people chose
  (A), which is the converse of the statement. $2$ people chose (D)
  and $1$ person each chose (C) and (E). Thus, most people got the
  sign/direction part correct but messed up on which way the
  implication goes.

\item Consider all the rectangles with perimeter equal to a fixed
  length $p > 0$. Which of the following {\bf is true} for the unique
  rectangle which is a square, compared to the other rectangles?

  \begin{enumerate}[(A)]
  \item It has the largest area and the largest length of diagonal.
  \item It has the largest area and the smallest length of diagonal.
  \item It has the smallest area and the largest length of diagonal.
  \item It has the smallest area and the smallest length of diagonal.
  \item None of the above.
  \end{enumerate}

  {\em Answer}: Option (B)

  {\em Explanation}: We can see this easily by doing calculus, but it
  can also be deduced purely by thinking about how a square and a long
  thin rectangle of the same perimeter compare in terms of area and
  diagonal length.

  {\em Performance review}: All $25$ got this.

  {\em Historical note 1}: $13$ out of $18$ people got this
  correct. $3$ chose (E) and $2$ chose (A).

  {\em Historical note 2}: This question appeared in a past 152 quiz,
  and everybody got this correct.

  {\em Historical note3}: This question appeared in an earlier 151
  final, and $31$ out of $33$ people got it correct.

\item Suppose the total perimeter of a square and an equilateral
  triangle is $L$. (We can choose to allocate all of $L$ to the
  square, in which case the equilateral triangle has side zero, and we
  can choose to allocate all of $L$ to the equilateral triangle, in
  which case the square has side zero). Which of the following
  statements {\bf is true} for the sum of the areas of the square and
  the equilateral triangle? (The area of an equilateral triangle is
  $\sqrt{3}/4$ times the square of the length of its side).
  \begin{enumerate}[(A)]
  \item The sum is minimum when all of $L$ is allocated to the square.
  \item The sum is maximum when all of $L$ is allocated to the square.
  \item The sum is minimum when all of $L$ is allocated to the
    equilateral triangle.
  \item The sum is maximum when all of $L$ is allocated to the
    equilateral triangle.
  \item None of the above.
  \end{enumerate}

  {\em Answer}: Option (B)

  {\em Quick explanation}: The problem can also be solved
  using the rough heuristic that works for these kinds of problems:
  the maximum occurs when everything is allocated to the most
  efficient use, but the minimum typically occurs somewhere in
  between.

  {\em Full explanation}: Suppose $x$ is the part allocated to the
  square. Then $L - x$ is the part allocated to the equilateral
  triangle. The total area is:

  $$A(x) = x^2/16 + (\sqrt{3}/4)(L - x)^2/9$$

  Differentiating, we obtain:

  $$A'(x) = \frac{x}{8} - \frac{\sqrt{3}}{18} (L - x) = x \left(\frac{1}{8} + \frac{\sqrt{3}}{18} \right) - \frac{\sqrt{3}}{18}L$$

  We see that $A'(x) = 0$ at

  $$x = \frac{L(\sqrt{3}/18)}{(1/8) + (\sqrt{3}/18)}$$

  This number is indeed within the range of possible values of $x$.

  Further, $A'(x) > 0$ for $x$ greater than this and $A'(x) < 0$ for
  $x$ less than this. Thus, this point is a local minimum and the
  maximum must occur at one of the endpoints. We plug in $x = 0$ to
  get $(\sqrt{3}/36)L^2$ and we plug in $x = L$ to get $L^2/16$. Since
  $1/16 > \sqrt{3}/36$, we obtain the the maximum occurs when $x = L$,
  which means that all the perimeter goes to the square.

  {\em Performance review}: $15$ out of $25$ got this. $6$ chose (C),
  $2$ each chose (D) and (E).

  {\em Historical note 1}: $7$ out of $18$ people got this
  correct. $7$ chose (E), $2$ each chose (C) and (D).

  {\em Historical note 2}: This question appeared in a Math 152 quiz,
  and at the time $9$ out of $15$ people got this correct. $2$ people
  each chose (E) and (C), $1$ person chose (D), and $1$ person left
  the question blank.

  {\em Historical note 3}: This question appeared in a 152 midterm
  last year, and $20$ of $29$ people got it correct. In that midterm,
  option (E) wasn't there, so things became a little easier.


\item Suppose $x$ and $y$ are positive numbers such as $x + y =
  12$. For {\bf what values} of $x$ and $y$ is $x^2y$ maximum?

  \begin{enumerate}[(A)]
  \item $x = 3$, $y = 9$
  \item $x = 4$, $y = 8$
  \item $x = 6$, $y = 6$
  \item $x = 8$, $y = 4$
  \item $x = 9$, $y = 3$
  \end{enumerate}

  {\em Answer}: Option (D).

  {\em Quick explanation}: This is a special case of the general
  Cobb-Douglas situation where we want to maximize $x^a(C - x)^b$. The
  general solution is to take $x = Ca/(a + b)$, i.e., to take $x$ and
  $C - x$ in the proportion of $a$ to $b$.

  {\em Full explanation}: We need to maximize $f(x) := x^2(12 - x)$,
  subject to $0 < x < 12$. Differentiating, we get $f'(x) = 3x(8 -
  x)$, so $8$ is a critical point. Further, we see that $f'$ is
  positive on $(0,8)$ and negative on $(8,12)$, so $f$ attains its
  maximum (in the interval $(0,12)$) at $8$.

  {\em Performance review}: $24$ out of $25$ got this. $1$ chose (E).

  {\em Historical note 1}: $11$ out of $18$ people got this
  correct. $5$ chose (E), $2$ chose (C).

  {\em Historical note 2}: This question appeared in a Math 152 quiz. At
  the time, $12$ out of $15$ people got this correct. $2$ people chose
  (E) and $1$ person chose (B). Of the people who got this correct,
  some seem to have computed the numerical values and others seem to
  have used calculus. Some who did not show any work may have used the
  general result of the Cobb-Douglas situation.

\item Consider the function $p(x) := x^2 + bx + c$, with $x$
  restricted to integer inputs. Suppose $b$ and $c$ are integers. The
  minimum value of $p$ is attained either at a single integer or at
  two consecutive integers. Which of the following is a {\bf
  sufficient condition} for the minimum to occur at two consecutive
  integers?

  \begin{enumerate}[(A)]
  \item $b$ is odd
  \item $b$ is even
  \item $c$ is odd
  \item $c$ is even
  \item None of these conditions is sufficient.
  \end{enumerate}

  {\em Answer}: Option (A)

  {\em Explanation}: The graph of $f$ is symmetric about the
  half-integer axis value $-b/2$. It is an upward-facing parabola. For
  odd $b$, it attains its minimum among integers at the two
  consecutive integers $-b/2 + 1/2$ and $-b/2 - 1/2$. When $b$ is
  even, the minimum is attained uniquely at $-b/2$, which is itself an
  integer. $c$ being odd or even tells us nothing.

  {\em Performance review}: $16$ out of $25$ got this. $4$ chose (E),
  $2$ each chose (B) and (D), $1$ chose (C).

  {\em Historical note 1}: $2$ out of $18$ people got this
  correct. $8$ chose (E), $5$ chose (B), $3$ chose (C).

 {\em Historical note 2}: This question appeared in a Math 152 quiz. At
  the time, $4$ out of $15$ people got this correct. $8$ people chose
  (E) and $3$ people chose (B).

\end{enumerate}

\end{document}