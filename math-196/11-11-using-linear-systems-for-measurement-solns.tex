\documentclass[10pt]{amsart}

%Packages in use
\usepackage{fullpage, hyperref, vipul, enumerate}

%Title details
\title{Take-home class quiz solutions: due Monday November 11: Using linear systems for measurement}
\author{Math 196, Section 57 (Vipul Naik)}
%List of new commands

\begin{document}
\maketitle

\section{Performance review}

27 people took this 4-question quiz. The score distribution was as
follows:

\begin{itemize}
\item Score of 0: 2 people
\item Score of 2: 9 people
\item Score of 3: 9 people
\item Score of 4: 7 people
\end{itemize}

The mean score was about 2.7.

The question-wise answers and performance review are as follows:

\begin{enumerate}
\item Option (E): 21 people
\item Option (D): 18 people
\item Option (B): 17 people
\item Option (D): 17 people
\end{enumerate}

\section{Solutions}

{\bf PLEASE FEEL FREE TO DISCUSS {\em ALL} QUESTIONS.}

The purpose of this quiz is two-fold. First, many of the ideas you saw
early on in the course (in the second and third week) may be on the
verge of fading out. Drawing from the best research on {\em spaced
  repetition} (see for instance
\url{http://en.wikipedia.org/wiki/Spaced_repetition}) it's high time
we tried recalling some of that stuff. But with a twist, because we
can now use some concepts from later topics we've seen to refine our
past understanding.

The second purpose is to prepare you for what we hope to eventually
get to: a deep and rich understanding of linear algebra as it's {\em
  used}: in linear regressions, computing correlations, and more fancy
applications like factor analysis and principal component
analysis. The third question, in particular, relates to the central
idea behind linear regression (specifically, ordinary least squares
regression). The questions also relate, albeit not very directly, to the
broad ideas behind factor analysis and principal component analysis.

\vspace{0.5in}

For the questions here, assume two dimensions of a person's general
cognitive ability: verbal and mathematical. Denote by $g_v$ the
person's general verbal ability and by $g_m$ the person's general
mathematical ability.

Various ability tests can be devised that aim to test for the person's
abilities. However, it is not possible to construct a test that {\em
  solely} measures $g_v$ or {\em solely} measure $g_m$. Different
tests measure $g_v$ and $g_m$ to different extents. For instance, an
ordinary numerical computation test might measure mostly $g_m$. On the
other hand, a test similar to the quizzes in this course might measure
both $g_v$ and $g_m$ a fair amount, given how much you have to read to
answer the quiz questions.

Of course, the score on a given test could depend on a lot of factors
other than general abilities. Some of them could be systematic: a
student with poor mathematical abilities in general may have ``trained
for the test.'' As an example, using a calculus test to test for
general mathematical ability might mean that people who have happened
to take calculus do a lot better than people who haven't, but have
similar general mathematical ability. Some are more ephemeral, such as
students guessing answers, mood fluctuations, and other
context-specific factors that affect scores.

For simplicity, we will assume that there are no systemic factors
other than general verbal and general mathematical ability that the
test is measuring. For even greater simplicity, assume that the
(expected) test score is linear in $g_v$ and $g_m$ with zero
intercept, i.e, the score is of the form $w_vg_v + w_mg_m$ where $w_v$
and $w_m$ are real numbers that serve as {\em weights}. Note that this
assumes that there is no {\em interaction} between the verbal and
mathematical skills in determining the scores.

The assumption may or may not be realistic. For instance, a question
(such as those on this quiz!) that requires a lot of reading {\em and}
strong math skills would probably have an expected score formula that
is {\em multiplicative} in $g_v$ and $g_m$: having zero or near-zero
verbal ability means you will be unable to do the question, even if
your mathematical ability is awesome. Similarly, having zero or
near-zero mathematical ability means you will be unable to do the
question, even if your verbal ablity is awesome. Multiplicatively
separable functions are better suited to capture this sort of
dependence. However, even if the test has questions of this sort, we
can take logs on test scores and make them additively separable, so
the additive model may still work well.

The assumption of {\em linearity} goes further, but this too might be
realistic.

My goal is to use one or more tests in order to determine the true
values of a student's $g_v$ and $g_m$. Another formulation is that my
goal is to determine the vector:

$$\vec{g} = \left[ \begin{matrix} g_v \\ g_m \\\end{matrix}\right]$$
\begin{enumerate}
\item I administer two tests to a student. The student's score $s_1$
  on the first test is $2g_v + 3g_m$ while the score $s_2$ on the
  second test is $3g_v + 5g_m$. How do I recover $g_v$ and $g_m$ from
  $s_1$ and $s_2$?

  \begin{enumerate}[(A)]
  \item $g_v = 2s_1 + 3s_2$, $g_m = 3s_1 + 5s_2$
  \item $g_v = 2s_1 - 3s_2$, $g_m = 3s_1 - 5s_2$
  \item $g_v = 5s_1 + 3s_2$, $g_m  = 3s_1 + 2s_2$
  \item $g_v = 5s_1 - 3s_2$, $g_m = 3s_1 - 2s_2$
  \item $g_v = 5s_1 - 3s_2$, $g_m = -3s_1 + 2s_2$
  \end{enumerate}

  {\em Answer}: Option (E)

  {\em Explanation}: We have the following:

  $$\left[ \begin{matrix} s_1 \\ s_2 \\\end{matrix}\right] = \left[ \begin{matrix} 2 & 3 \\ 3 & 5 \\\end{matrix}\right] \left[ \begin{matrix} g_v \\ g_m \\\end{matrix}\right]$$

  We thus have:

  $$\left[ \begin{matrix} g_v \\ g_m \\\end{matrix}\right] = \left[ \begin{matrix} 2 & 3 \\ 3 & 5 \\\end{matrix}\right]^{-1} \left[\begin{matrix} s_1 \\ s_2 \\\end{matrix}\right]$$

  Recall that, for a general $2 \times 2$ matrix, the inverse is given by:

  $$\left[ \begin{matrix} a & b \\ c & d \\\end{matrix}\right]^{-1} = \frac{1}{ad - bc} \left[ \begin{matrix} d & -b \\ -c & a \\\end{matrix}\right]$$

  In this case, the determinant $ad - bc$ equals $(2)(5) - (3)(3) =
  1$, so that the inverse is:

  $$\left[ \begin{matrix} 2 & 3 \\ 3 & 5 \\\end{matrix}\right]^{-1} = \left[ \begin{matrix} 5 & -3 \\ -3 & 2 \\\end{matrix}\right]$$

  Plugging in, we get:

  $$\left[ \begin{matrix} g_v \\ g_m \\\end{matrix}\right] = \left[ \begin{matrix} 5 & -3 \\ -3 & 2 \\\end{matrix}\right] \left[\begin{matrix} s_1 \\ s_2 \\\end{matrix}\right]$$

  This simplifies to:

  \begin{eqnarray*}
    g_v & = & 5s_1 - 3s_2\\
    g_m & = & -3s_1 + 2s_2\\
  \end{eqnarray*}

  {\em Performance review}: 21 out of 27 got this. 3 chose (A), 2
  chose (D), 1 chose (C).

  {\em Historical note (last time)}: $23$ out of $25$ people got tihs. $1$ each
  chose (B) and (C).
\item In order to combat the problem of uncertainty about my model, I
  decide to administer more than two tests. I administer a total of
  $n$ tests. The score on the $i^{th}$ test is $s_i = w_{i,v}g_v
  + w_{i,m}g_m$. The score vector $\vec{s}$ has coordinates $s_i,
  1 \le i \le n$.

  If there is no measurement error and the student's actual score in
  each test equals the student's expected score, then we have a system
  of $n$ simultaneous linear equations in $2$ variables.

  Let $W$ be the matrix:

  $$\left[\begin{matrix} w_{1,v} & w_{1,m} \\ w_{2,v} & w_{2,m} \\ \cdot & \cdot \\ \cdot & \cdot \\ \cdot & \cdot \\ w_{n,v} & w_{n,m} \\\end{matrix}\right]$$

  Assume that all entries of $W$ are positive, i.e., each test tests
  to a nonzero extent for both verbal and mathematical ability.

  Our goal is to ``solve for'' $\vec{g}$ the following vector
  equation:

  $$W\vec{g} = \vec{s}$$

  What is the necessary and sufficient condition on $W$ so that the
  equation has at most one solution for $\vec{g}$ for each $\vec{s}$?
  If $\vec{s}$ arises from an actual $\vec{g}$, i.e., it is a true
  score vector, then note that there will be a solution.

  \begin{enumerate}[(A)]
  \item All the ratios $w_{i,v}:w_{i,m}$ are the same.
  \item All the ratios $w_{i,v}:w_{i,m}$ are different.
  \item At least two of the ratios $w_{i,v}:w_{i,m}$ are the same.
  \item At least two of the ratios $w_{i,v}:w_{i,m}$ are different.
  \end{enumerate}

  {\em Answer}: Option (D)

  {\em Explanation}: Recall that for a unique solution, the
  coefficient matrix needs to have full column rank, which in this
  case means rank $2$. If all the ratios $w_{i,v}:w_{i,m}$ are the
  same, then the rank is $1$. On the other hand, if two of the ratios
  differ, then those two rows alone give rank $2$, and the remaining
  rows cannot affect the rank any more.

  {\em Performance review}: 18 out of 27 got this. 6 chose (B), 2
  chose (A), 1 chose (C).

  {\em Historical note (last time)}: $14$ out of $25$ people got this. $4$ each
  chose (A) and (C). $3$ chose (B).

\item Use notation as in the previous question. Suppose that there is
  some measurement error, so that instead of getting the true score
  vector $\vec{s}$, I have a somewhat distorted score vector
  $\vec{t}$. How do I go about recovering my ``best guess'' for
  $\vec{s}$ from $\vec{t}$?

  \begin{enumerate}[(A)]
  \item Find the closest vector to $\vec{t}$ in the kernel of the
    linear transformation corresponding to $W$.
  \item Find the closest vector to $\vec{t}$ in the image of the
    linear transformation corresponding to $W$.
  \item Find the farthest vector from $\vec{t}$ in the kernel of the
    linear transformation corresponding to $W$.
  \item Find the farthest vector from $\vec{t}$ in the image of the
    linear transformation corresponding to $W$.
  \end{enumerate}

  {\em Answer}: Option (B)

  {\em Explanation}: Note that when talking of the kernel and image of
  $W$, we want to solve $W\vec{g} = \vec{s}$. Instead of $\vec{s}$, we
  have a distorted vector $\vec{t}$. Our best guess for $\vec{s}$ is
  the vector closest to $\vec{t}$ for which the equation can be
  solved, i.e., the vector closest to $\vec{t}$ in the image of the
  linear transformation corresponding to $W$.

  {\em Performance review}: 17 out of 27 got this. 5 chose (A), 4
  chose (C), 1 chose (D).

  {\em Historical note (last time)}: $14$ out of $25$ people got this. $8$
  chose (C), $3$ chose (A).

\item Suppose I want to introduce a {\em new} test that tests for both
  verbal and mathematical ability with expected score of the form
  $w_vg_v + w_mg_m$, but the values $w_v$ and $w_m$ are currently
  unknown. My strategy is as follows. I find two students. I
  administer a bunch of tests with {\em known} $w_v$ and $w_m$ values
  to those students. I use those tests to find the $g_v$ and $g_m$
  values for both students. Then, I administer the new test to both
  students and try to determine the values of $w_v$ and $w_m$. Assume no measurement error.

  Of course, I want the matrix $W$ of the {\em known} tests to satisfy
  the condition of Question 2. What additional criteria would I wish
  of the two students I use for this in order to correctly determine
  $g_v$ and $g_m$? Note that it will not be possible to be sure of
  this in advance, but one can still pick student pairs who are more
  likely to satisfy the criterion and thus avoid waste of effort.

  \begin{enumerate}[(A)]
  \item The students should have the same $g_v + g_m$ value.
  \item The students should have different $g_v + g_m$ values.
  \item The students should have the same $g_v:g_m$ ratio.
  \item The students should have different $g_v:g_m$ ratios.
  \end{enumerate}

  {\em Answer}: Option (D)

  {\em Explanation}: The reasoning is similar to that used for Question 2.

  Let $g_{v,1}$ and $g_{m,1}$ be the $g_V$ and $g_m$ values for the
  first student. Let $g_{v,2}$ and $g_{m,2}$ be the $g_v$ and $g_m$
  values for the second student. We want to find the weights $w_v$ and
  $w_m$ for the test. Explicitly, this means solving the following
  equation for unknowns $w_v$ and $w_m$:

  $$\left[\begin{matrix} g_{v,1} & g_{m,1} \\ g_{v,2} & g_{m,2} \\\end{matrix}\right]\left[\begin{matrix} w_v \\ w_m \\\end{matrix}\right] = \left[\begin{matrix} \text{score of first student} \\ \text{score of second student} \\\end{matrix}\right]$$

  In order to have a unique solution, we need the coefficient matrix
  to have full column rank $2$, i.e., we need the students' $g_v:g_m$
  ratios to differ.

  {\em Performance review}: 17 out of 27 got this. 9 chose (C), 1 chose (B).

  {\em Historical note (last time)}: $14$ out of $25$ got this. $9$ chose (C),
  $1$ each chose (A) and (B).
\end{enumerate}
\end{document}
