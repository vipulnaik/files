\documentclass[10pt]{amsart}

%Packages in use
\usepackage{fullpage, hyperref, vipul,graphicx}

%Title details
\title{Cohomology trees}
\author{Vipul Naik}

%List of new commands
\newcommand{\Skew}{\operatorname{Skew}}

\begin{document}
\maketitle


\section{Cohomology trees}

\subsection{Idea}

Suppose $G$ is a group and $A_2,A_3,A_4,\dots, A_n$ are abelian
groups. Define:

$$\mathcal{H}^2(G;A_2,\dots,A_n)$$

as the collection of all extensions of the form:

$$K_1 \ge K_2 \ge \dots K_n \ge K_{n+1}= 1$$

where $K_1/K_2 = G$, $K_i/K_{i+1} \cong A_i$ for $2 \le i \le n$, and
finally, $[K_1,K_i] \le K_{i+1}$ for $i \ge 2$. In other words, the
series is {\em almost} a central series with the exception of how it
starts: it is not necessary that $[K_1,K_1] \le K_2$.

Isomorphism is defined with the usual commuting diagrams.

This is a multi-group generalization of cohomology.

We can see that as a set we can compute it as follows:

$$\mathcal{H}^2(G;A_2,A_3,\dots,A_n) = \bigsqcup_{L \in H^2(G;A_2)} \mathcal{H}^2(L;A_3,A_4,\dots,A_n)$$

Here $H^2(G;A_2)$ is the second cohomology group for trivial group action of $G$ on $A_2$.

This immediately suggests a tree method for computing this
structure. Namely, the root of the tree represents the string
$(G;A_2,A_3,\dots,A_n)$. This branches out into children of the form
$\mathcal{H}^2(L;A_3,A_4,\dots,A_n)$, with one child for each $L \in
H^2(G;A_2)$. Each of these children further has children. For each
$L$, the children are parametrized by $M \in H^2(L;A_3)$, and the
strings for this are $\mathcal{H}^2(M;A_4,\dots,A_n)$. We proceed this
way till the string has length $1$, and these are the leaves of the tree.

The leaves of the tree represent the set of all possible extensions.

\subsection{Symmetries of the tree}

Many leaves of the tree give the same group under abstract
isomorphism. There could be three possible reasons:

\begin{itemize}
\item The isomorphism traces back to the first branching out point,
  and is explained by the action of automorphism groups on the second
  cohomology group at the branch-out point (i.e., at that point, the
  extensions are pseudo-congruent but not congruent).
\item The isomorphism traces back to the first branching out point,
  but is not explained by a pseudo-congruence of extensions.
\item The isomorphism arises otherwise.
\end{itemize}

Let's consider an example. Consider $G = A_2 = A_3 = \Z_2$. We are
interested in the object:

$$\mathcal{H}^2(\Z_2;\Z_2,\Z_2)$$

It's easy to see that this will classify {\em all} groups of order 8,
because any group of order 8 has a central series where the factor
groups are all $\Z_2$.

We note that:

\begin{itemize}
\item The children are $\mathcal{H}^2(V_4;\Z_2)$ and
  $\mathcal{H}^2(\Z_4;\Z_2)$. Note that once the string length becomes
  $2$, then $\mathcal{H}^2$ becomes $H^2$, so the children are
  $H^2(V_4;\Z_2)$ and $H^2(\Z_4;\Z_2)$.
\item The children of the node $H^2(V_4;\Z_2)$ are as follows: one
  copy of $E_8$, three copies of $\Z_4 \times \Z_2$ (permuted
  transitively by the automorphism action), one copy of $Q_8$, the
  quaternion group, and three copies of $D_8$ (permuted transitively
  by the automorphism action).
\item The children of the node $H^2(\Z_4;\Z_2)$ are as follows: one
  copy of $\Z_4 \times \Z_2$ and one copy of $\Z_8.$
\end{itemize}

\includegraphics[width=4in]{cohomologytreesorder8groups.png}

Note that $\Z_4 \times \Z_2$ occurs four times as a leaf. Three of
these are equivalent under an automorphism action at the branch-out
point. The fourth is not equivalent to these, and represents a
fundamentally different choice of lower central series for the group.

\subsection{Fun aside: weighting}

Not sure this will get anywhere, but it can be done and may be useful for some other purpose:

\begin{itemize}
\item Assign a weight of $1$ to the root.
\item At each branching, split the weight on the parent node equally
  to all children (uniform distribution within each group).
\item We get a weighting on leaves.
\item For a given isomorphism class of group, add up the weights on
  all leaves with that isomorphism class.
\item We thus get a probability distribution on the finite set of all
  groups arising as central extensions.
\end{itemize}

In particular, for groups of order $p^n$, view then as
$\mathcal{H}^2(\Z_p;\Z_p,\Z_p,\dots,\Z_p)$ where the total (including
the first) number of $\Z_p$s is $n$. This gives a canonical
probability distribution on the set of isomorphism classes of groups
of order $p^n$.

Apply this to $2^3 = 8$. We get:

\begin{itemize}
\item The root has weight $1$
\item The children $H^2(V_4;\Z_2)$ and $H^2(\Z_4;\Z_2)$ each have weight $1/2$.
\item Each child of $H^2(V_4;\Z_2)$ has weight $(1/8)(1/2) =
  1/16$. Thus $E_8$ and $Q_8$ have weight $1/16$, the three copies of
  $\Z_4 \times \Z_2$ have combined weight $3/16$, and so do the three
  copies of $D_8$.
\item Each child of $H^2(\Z_4;\Z_2)$ has weight $1/4$: so $\Z_4 \times \Z_2$ has
  weight $1/4$ and so does $\Z_8$.
\end{itemize}

The overall probability distribution is:

\begin{itemize}
\item $\Z_8$: $1/4$
\item $\Z_4 \times \Z_2$: $7/16$ (adding up its appearance in both subtrees)
\item $D_8$: $3/16$
\item $Q_8$: $1/16$
\item $E_8$: $1/16$
\end{itemize}

\subsection{Why cohomology trees?}

Cohomology trees are particularly suited for the nilpotent situation, because:

\begin{itemize}
\item The inductive definition meshes well with the notion of nilpotency class
\item Central series are critical to the idea of nilpotency
\item We are always dealing with trivial group actions
\item We are always dealing with at most one non-abelian group at any given time
\end{itemize}

\subsection{Cohomology trees for Lie ring extensions}

We could do the same cohomology business for Lie ring extensions. We
would need to replace group cohomology with Lie ring cohomology at the
crucial place where we are using $H^2$.

However, it is worth noting that the cohomology tree approach applied
to $\mathcal{H}^2(\Z_p;\Z_p,\Z_p,\dots\Z_p)$ will {\em not} yield all
Lie rings of order $p^n$, but only the nilpotent Lie rings of order
$p^n$. This is exactly what we want, so not a problem. However, it
does point a difference with groups, because for groups, any group of
prime power order is nilpotent.

\section{Our broad goals}

At the classification level, we would like to compare the {\em group cohomology tree}:

$$\mathcal{H}^2_{\operatorname{Grp}}(A_1;A_2,\dots,A_n)$$

with the {\em Lie ring cohomology tree}:

$$\mathcal{H}^2_{\operatorname{Lie}}(A_1;A_2,\dots,A_n)$$

where $A_1,A_2,\dots,A_n$ are all abelian groups, treated as abelian
Lie rings (trivial Lie bracket) for the Lie ring cohomology
tree. (Another way of thinking of this is to write it as
$\mathcal{H}^2_{\_}(1;A_1,A_2,\dots,A_n)$, which is the same thing but
automatically enforces abelianness).

A case of particular interest is when $A_1 = A_2 = \dots = A_n =
\Z_p$. In this case, the leaves of the tree include all groups
(respectively, nilpotent Lie rings) of order $p^n$, possibly with
repetitions.

How similar are the trees?


\subsection{Restrict to a single prime}

Cohomology groups for coprime group actions are trivial, etc., so in
practice, we can restrict attention from finite groups to the case of
finite $p$-groups for a single prime $p$.

\subsection{Lazard isomorphism}

A particular case where Lazard would be applicable is if $A_i$ are all
finite $p$-groups for the same prime $p$ and $n < p$.

In this case, there is an isomorphism between the trees that in fact
induces the Lazard correspondence between every node and its
identified node.

\subsection{What happens when the Lazard isomorphism fails?}

Suppose $n \ge p$. In this case the Lazard isomorphism fails all the
way down. However, note that there is a substantial portion of the
trees unto which the isomorphism does work. Specifically, it works to
depth $p - 1$ (i.e., $p -2$ branchings). The interesting
questions are as follows:

\begin{itemize}
\item Can the isomorphism obtained to depth $p - 1$ be extended to a
  {\em tree} isomorphism all the way through?
\item Can we choose this tree isomorphism such that the identification
  of nodes happens between a group and a Lie ring whose additive group
  is $1$-isomorphic to it?
\end{itemize}

\subsection{case $p = 2$ and depth two}

In this case, we are looking at:

$$\mathcal{H}^2_{\operatorname{Grp}}(A_1;A_2) = H^2(A_1;A_2)$$

versus:

$$\mathcal{H}^2_{\operatorname{Lie}}(A_1;A_2) = H^2_{\operatorname{Lie}}(A_1;A_2)$$

It turns out that we have:

$$H^2_{\operatorname{Lie}}(A_1;A_2) \cong H^2_{sym}(A_1;A_2) \oplus \operatorname{Hom}(\bigwedge^2(G),A)$$

where the direct summation is by splitting the information into the
``extension as an abelian group'' part and the ``Lie bracket
description'' part.

We also have a short exact sequence (at least in the finite case!):

$$0 \to H^2_{sym}(A_1;A_2) \to H^2(A_1;A_2) \to \operatorname{Hom}(\bigwedge^2A_1,A_2) \to 0$$

In cases of interest, this splits:

$$H^2(A_1;A_2) \cong H^2_{sym}(A_1;A_2) \oplus \bigwedge^2(G;A)$$

Under cases of interest to us, we can obtain an automorphism-invariant
splitting, where we choose the complement as the subgroup comprising
cohomology classes represented by cyclicity-preserving 2-cocycles.

In these cases, we thus get a tree isomorphism where each node is
identified with a node with which it is $1$-isomorphic.

\section{Exploration of the class three situation}

We consider the class three situation, which basically involves
considering a cohomology tree of the form:

$$\mathcal{H}^2(A_1;A_2,A_3)$$

versus:

$$\mathcal{H}^2_{\text{Lie}}(A_1;A_2,A_3)$$

where $A_1$, $A_2$, and $A_3$ are all finite $p$-groups. We consider
the general theory and also note how the Lazard situation (which in
this case would be guaranteed if $p \ge 5$) is nicer than the more
general situation.

\subsection{The key parallelism idea}

The key idea is that, as we proceed to build the cohomology tree on
the group side, we try at every stage to break up the cohomology group
as an internal direct sum of (weakly) symmetric cohomology classes
(corresponding to abelian group extensions) and a suitable choice of
complement. In the Lazard case, this suitable choice of complement can
be represented canonically by cocycles that are alternating
bihomomorphisms (or something like that). In other cases, we have to
make do with other alternatives such as the cyclicity-preserving
subgroup, or sometimes, we throw up our hand and say that there is no
natural splitting.

On the Lie ring side, we try to do a similar internal direct sum
splitting. We then try to use the splittings done on both sides to
obtain identifications between the nodes of the cohomology trees.

The trick is then to patch these splittings with each other. This is
trickier than it sounds.

Overall, we will notice that our approach does more than just
re-establish the Lazard correspondence. It should find a chain of
intermediate groups as well. Recall that the Lazard correspondence relates:

Groups of nilpotency class $\le p - 1$ $\leftrightarrow$ Abelian groups (with an additional Lie bracket structure)

We can do better. We can, in the Lazard situation, form a descending
nilpotency class chain:

Group of nilpotency class $c$ $\leftrightarrow$ Group of nilpotency class $c - 1$ $\leftrightarrow$ $\dots$ Group of nilpotency class $1$ = Abelian group

Further, in the non-Lazard situation, we can diagnose exactly {\em
what step} the class reduction fails at. It may happen that a class
three group can be brought down to a class two group which fails to go
down to an abelian group. Or, it may happen that it's not possible to
reduce to a class two group in the first place.

Let's now return to the class three situation to see how all this is
illustrated in that situation.

\subsection{The first cut: the class two top}

On the Lie ring side, we have:

$$H^2_{\text{Lie}}(A_1;A_2) = H^2_{\text{sym}}(A_1;A_2) \oplus \operatorname{Hom}(\bigwedge^2A_1,A_2)$$

The direct sum decomposition is explained as follows. To describe a
Lie ring extension, we need to define the abelian group extension for
the additive structure, and {\em separately} describe the Lie bracket
as an alternating bihomomorphism from $A_1$ to $A_2$. The two direct
summands correspond to these separate pieces of information.

On the group side, we have a short exact sequence:

$$0 \to H^2_{sym}(A_1;A_2) \to H^2(A_1;A_2) \stackrel{\Skew}{\to} \operatorname{Hom}(\bigwedge^2A_1,A_2) \to 0$$

The goal is to find a natural splitting for this short exact sequence,
and hence an isomorphism between $H^2_{\text{Lie}}(A_1;A_2)$ and
$H^2(A_1;A_2)$.

In the Lazard case, we choose a natural splitting by choosing cocycles
that are halves of the alternating bihomomorphisms. In some of the
non-Lazard cases, we can choose a complement that comprises the
cohomology classes represented by cyclicity-preserving $2$-cocycles.

What the isomorphism does (more concretely) is the following: for any
group extension $L \in H^2(A_1;A_2)$, we have found an abelian group
$M$ with a Lie bracket on it, and a bijection from $M$ to $L$ under
which the Lie bracket on $M$ becomes the commutator map on $L$. The
group $M$ is basically obtained by ``projecting'' the cohomology class
describing $L$ onto the $H^2_{sym}(A_1;A_2)$ part.

\subsection{The next cut}

Let's say we have picked a group extension $L \in H^2(A_1;A_2)$, i.e.,
we have chosen one of the child nodes of the root, namely
$\mathcal{H}^2(L;A_3) = H^2(L;A_3)$. 

Note now that $L$ is a (possibly) non-abelian group. Thus, we need to
be a little more careful in describing $H^2(L;A_3)$. Note that because
$L \in H^2(A_1;A_2)$, we are given an explicit identification of $A_2$
with a central subgroup of $L$. We abuse notation and call this
subgroup $A_2$.

Note that since $A_2$ is in the center of $L$, it is true that $xy =
yx$ for all $x \in A_2$ and $y \in L$. In particular, any
$2$-coboundary $f$ from $L$ to $A_3$ has the property that $f(x,y) =
f(y,x)$ for $x \in A_2$ and $y \in L$.

Further, if a $2$-cocycle from $L$ to $A_3$ has the property that
$f(x,y) = f(y,x)$ for all $x \in A_2$ and $y \in L$, then the
corresponding extension of $L$ on top of $A_3$ has the property that
the inverse image of $A_2$ in the extension (which is $A_2$ sitting
atop $A_3$) is central in the whole extension, and thus the whole
extension in particular has nilpotency class two.

Define $Z^2_{sym,A_2}(L;A_3)$ as the group of $2$-cocycles $f$ from
$L$ to $A_3$ satisfying $f(x,y) = f(y,x)$ for all $x \in A_2$ and $y
\in L$. As already noted, $B^2(L;A_3) \le Z^2_{sym,A_2}(L;A_3)$, and
we can hence define a quotient $H^2_{sym,A_2}(L;A_3)$ which is a
subgroup of $H^2(L;A_3)$.

Recall the short exact sequence in class two:

$$0 \to H^2_{sym}(A_1;A_2) \to H^2(A_1;A_2) \stackrel{\Skew}{\to} \operatorname{Hom}(\bigwedge^2A_1,A_2) \to 0$$

We would like to mimic this in class three. The right mimic is to
consider, instead of alternating bilinear maps from $A_1 \times A_1$
to $A_2$, to consider alternating bilinear maps from $A_2 \times A_1$
to $A_3$. How did we get this? First, we are interested in the
commutator map as it happens when one of the elements is in $A_2$ mod
$A_3$, i.e., is in the piece that become s$A_2$ when we go mod
$A_3$. We care about this element mod $A_3$ only. Further, because
this element is already in $A_2$, we care only about the top part (mod
$A_1,A_2$) of the other eleemnt. 

The sequence we get is:

$$0 \to H^2_{sym,A_2}(L;A_3) \to H^2(L;A_3) \stackrel{\Skew}{\to} \operatorname{Hom}(A_2 \wedge A_1,A_3) \to 0$$

From the definitions, we can verify that it is left and middle
exact. Right exactness is a little harder and may be isn't always
true. So what we really have is:

$$0 \to H^2_{sym,A_2}(L;A_3) \to H^2(L;A_3) \stackrel{\Skew}{\to} J \to 0$$

where $J$ is a suitable subgroup of $\operatorname{Hom}(A_2 \wedge
A_1,A_3)$ that depends on $L$.

What we would ideally like to do is find a natural complement or
equivalently a natural splitting of this short exact sequence, so
that:

$$H^2(L;A_3) = H^2_{sym,A_2}(L;A_3) \oplus J$$

In the situation of $2$-divisible groups, the way this would work is
that we would perform a halving to reverse the skew map.

\subsection{Recombination}

Let's say we start with a group $M \in H^2(L;A_3)$. We use the direct
sum decomposition of $H^2(L;A_3)$ to project $M$ down to its
$H^2_{sym,A_2}(L;A_3)$ component. Call that $N$.

Now, we can rethink $N$ as follows. Consider the subgroup $A_2$ in $L$
sitting atop $A_3$. That defines an extension group. Further, we know
that this extension group is central in $N$ because $N \in
H^2_{sym,A_2}(L;A_3)$. Call that group $K$. $N$ can be identified with
an element of $H^2(A_1;K)$.

Overall, we have a bijection between two bunches of disjoint unions of
groups:

$$\bigsqcup_{L \in H^2(A_1;A_2)} H^2_{sym,A_2}(L;A_3) \leftrightarrow \bigsqcup_{K \in H^2_{sym}(A_2;A_3)} H^2(A_1;K)$$

and we are simply moving along this bijection from left to right.

Now that we are in $H^2(A_1;K)$, we are in yet another class two situation and we try to decompose this:

$$H^2(A_1;K) = H^2_{sym}(A_1;K) \oplus \text{complement}$$

Further, we need to ensure that this ``canonical'' direct sum
decomposition is compatible with the direct sum decomposition we chose
for $H^2(A_1;A_2)$. More precisely, the surjection $K \to A_2$ induces
a map $H^2(A_1;K) \to H^2(A_1;A_2)$ and we should get some kind of
commuting diagram relating the direct sum decomposition.

The upshot is that if we now project $N$ (viewed as an element of
$H^2(A_1;K)$) down to $H^2_{sym}(A_1;K)$, we get a genuinely abelian
group, say $P$. We have thus obtained a two-step procedure:

Class three group $M$ $\leadsto$ Class two group $N$ $\leadsto$
abelian group $P$.

\subsection{This seems to work for $p=3$, but shouldn't. What gives?}

\end{document}
