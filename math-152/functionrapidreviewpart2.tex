\documentclass{amsart}
\usepackage{fullpage,hyperref,vipul,graphicx}
\title{Functions: a rapid review (part 2)}
\author{Math 152, Section 55 (Vipul Naik)}

\begin{document}
\maketitle

{\bf Difficulty level}: Easy to moderate. Some of the symmetry
concepts (half turn symmetry and mirror symmetry) and some of the
proof techniques are likely to be new to students. The rest should be
straightforward.

{\bf Covered in class?}: This will roughly correspond to material
covered on Wednesday September 28. Most of the trickier aspects of this
will be covered in class, but many small points will be omitted due to
time constraints. Hence, it is recommended that you read through these
notes either before or after lecture.

{\bf Corresponding material in the book}: Sections 1.6, 1.7. However,
some ideas (including mirror symmetry and half turn symmetry) are not
covered in the book. In some other cases, ideas covered later in the
book are introduced at this early stage since you are already familiar
with calculus.

{\bf Corresponding material in homework problems}: Homework 1, routine
problems 7--8, advanced problems 2--3.
 
{\bf Things that students should get immediately}: Definitions of
various notions of pointwise combination of functions, scalar
multiples of functions, and compositions of functions. Definitions of
even, odd, and periodic functions.

{\bf Things that students should get with effort}: Definitions and
graphical interpretations of half turn symmetry and mirror
symmetry. Proof techniques related to showing even, odd, and periodic.

\section*{Executive summary}

For first time reading, skip to the next section.

Words ...

\begin{enumerate}
\item Given two functions $f$ and $g$, we can define pointwise
  combinations of $f$ and $g$: the sum $f + g$, the difference $f -
  g$, the product $f \cdot g$, and the quotient $f/g$. For the sum,
  difference, and product, the domain is the intersection of the
  domains of $f$ and $g$. For the quotient, the domain is the
  intersection of the domain of $f$ and the set of points where $g$
  takes a nonzero value.
\item Given a function $f$ and a real number $\alpha$, we can consider
  the scalar multiple $\alpha f$.
\item Given two functions $f$ and $g$, we can try talking of the
  composite function $f \circ g$. This is defined for those points in
  the domain of $g$ whose image lies in the domain of $f$.
\item One interesting kind of symmetry that we often see in the graph
  of a function is {\em mirror symmetry} about a vertical line. This
  means that the graph of the function equals its reflection about the
  vertical line. If the vertical line is $x = c$ and the function is
  $f$, this is equivalent to asserting that $f(x) = f(2c - x)$ for all
  $x$ in the domain, or equivalently, $f(c + h) = f(c - h)$ whenever
  $c + h$ is in the domain. In particular, the domain itself must be
  symmetric about $c$.
\item A special case of mirror symmetry is the case of an {\em even
  function}. An even function is a function with mirror symmetry about
  the $y$-axis. In other words, $f(x) = f(-x)$ for all $x$ in the
  domain. (Even also implies that the domain should be symmetric about $0$).
\item Another interesting kind of symmetry that we often see in the
  graph of a function is {\em half-turn symmetry} about a point on the
  graph. This means that the graph equals the figure obtained by
  rotating it by an angle of $\pi$ about that point. A point $(c,d)$
  is a point of half-turn symmetry if $f(x) + f(2c - x) = 2d$ for all
  $x$ in the domain. In particular, the domain itself must be
  symmetric about $c$. If $f$ is defined at $c$, then $d = f(c)$.
\item A special case of half-turn symmetry is an odd function, which
  is a function having half-turn symmetry about the origin. By
  definition, the domain of an odd function is symmetric about $\R$. An
  odd function, if defined at $0$, takes the value $0$ at $0$.
\item A function $f$ defined on $\R$ is periodic if there exists $h >
  0$ such that $f(x+ h) = f(x)$ for every $x \in \R$. If there is a
  smallest $h > 0$ satisfying this, such a $h$ is termed the {\em
  period}. Constant functions are periodic but have no period. The
  sine and cosine functions are periodic with period $2\pi$.
\end{enumerate}

Actions ...

\begin{enumerate}
\item To prove that a function is periodic, try to find a $h$ that
  {\em works} for every $x$. To prove that a function is periodic but
  has no period, try to show that there are arbitrarily small $h > 0$
  that work.
\item To prove that a function is even or odd, just try proving the
  corresponding equation for all $x$. Nothing but algebra.
\item If a function is defined for the positive or nonnegative reals
  and you want to extend the definition to negatives to make it even
  or odd, extend it so that the formula is preserved. So define $f(-x)
  = f(x)$, for instance, to make it even.
\end{enumerate}

\section{Ways of creating new functions from old}\label{newfunctionsfromold}

\subsection{Pointwise combinations of functions}

Suppose $f,g: \R \to \R$ are functions. By the way, before we proceed,
a clarification on notation and terminology. When I say $f:A \to B$ is
a function, then the {\em domain} of the function is $A$. However, the
{\em range} of the function need not be $B$. All that notation means
is that the range of the function is a {\em subset} of $B$. It might
be equal to $B$, but there's no guarantee. And the reason why we allow
this kind of latitude is that it makes it a lot easier to write things
down if we do not need to calculate the exact range all the time. And,
by the way, the set $B$ is termed the {\em co-domain}.

What does $f + g$ mean? So the first thing you might say is: ``How can
we add functions? I thought we could add only numbers.'' And the
answer is that we don't yet know how to make sense of it, but once we
do, it seems intuitive.

So remember, to define $f + g$ as a function, we need to describe
where it sends $x$. So $(f + g)(x)$ is defined as follows:

\begin{equation*}
  (f + g)(x) := f(x) + g(x)
\end{equation*}

This is the sum of the two functions, and if you stick around in the
world of mathematics, you'll hear people say that the sum is defined
{\em pointwise}. What this really means is that to add two functions,
what we do is add the {\em values} of the functions at each {\em
point}.

Here's a picture showing two functions $f$ and $g$ and their sum $f +
g$. Note that for each vertical line, the height of the $f + g$-point
is the sum of the heights of the $f$-point and the $g$-point:

\includegraphics[width=3in]{sumoffunctions.png}

Now, I assumed that the functions are both defined from $\R$ to
$\R$. And so, what is the domain of the function $f + g$? Well, it is
$\R$ again, because as you can see from the definition, since you can
evaluate $f$ and $g$ at a point, you can also evaluate $f + g$ at that
point. And, by the way, I use the word {\em point} where I actually
mean {\em real number} -- secretly, I'm thinking of real numbers as
points on the number line.

What if $f$ is a function defined on a smaller domain (i.e., a subset
of $\R$) and $g$ on another smaller domain (i.e., another subset of
$\R$)? In that case, $f + g$ is defined on the {\em intersection} of
$\operatorname{dom}(f)$ and $\operatorname{dom}(g)$. Why the
intersection? Because to evaluate $f + g$ at a point, you need to
evaluate $f$ at the point {\em and} $g$ at the point, and then add
those values. And to be able to evaluate {\em both}, the input should
be in the domain of both functions.

We similarly define:

\begin{align*}
  (f - g)(x) & := f(x) - g(x) \\
  (f \cdot g)(x) & := f(x)g(x) \\
  (f/g)(x) & := f(x)/g(x) \\
\end{align*}

For the case of the difference and product, the domain is the
intersection of the domains. For the ratio, or quotient, we need to be
a little more careful: the domain of the new function is inside the
intersection of the domains of $f$ and $g$, but there's a caveat: we
need to exclude points at which $g = 0$.

\subsection{Scalar multiples of functions}

Suppose $f$ is a function and $\alpha$ is a real number. The function
$\alpha f$ is defined as:

\begin{equation*}
  (\alpha f)(x) := \alpha f(x)
\end{equation*}

For instance, $2f$ is the function that sends $x$ to $2f(x)$, while
$-f$ is the function sending $x$ to $-f(x)$.

\subsection{Composition of functions}

Suppose $f,g:\R \to \R$ are functions. Then $f \circ g$ is defined as
the following function:

\begin{equation*}
  (f \circ g)(x) := f(g(x))
\end{equation*}

$f \circ g$ is termed the {\em composition} of the functions $f$ and
$g$. Orally, we say ``$f$ composed with $g$''. Note that the function
written on the right is the one the we apply {\em first}, so in the
case of function composition, we work from right to left. This can be
potentially confusing.

We can also define the composite of two functions when their domains
are subsets of $\R$. The domain of the composite $f \circ g$ is that
subset of $\operatorname{dom}(g)$ whose image under $g$ lies inside
$\operatorname{dom}(f)$. There is a more precise way of expressing
this, but it will take us too far afield, so we will skip it.

\subsection{Are there other ways of creating new functions?}

Yes, but we will see them later. The most significant of these are
differentiation, integration, and taking inverse functions.

\subsection{Why do ways of creating new functions from old matter?}

First, of course, ways of creating new functions from old help us
create new functions from old. However, just as new food recipes are
of little interest to those unenthusiastic about cooking and eating,
new function recipes may seem pointless to those unenthusiastic about
playing with functions. There is a deeper reason.

The point is that these ways of creating new functions from old are
{\em already in use when we think of and create new functions}. By
{\em explicitly identifying} the various recipes used to create new
functions from old, we hope to get a better mental model of functions
that {\em already exist}. Both pointwise combination and composition
are implicitly used all the time without our even knowing it. Making
them explicit is like writing down an explicit recipe for a dish that
we've already been cooking and eating.

We will better be able to understand a new phenomenon for {\em all
functions} when we are able to break the process of such understanding
into two steps: (i) understanding the phenomenon for a list of basic
building block functions, (ii) understanding how the phenomenon
interacts with the recipes for creating new functions from old. For instance:

\begin{enumerate}
\item In order to learn how to differentiate functions, we do two
  things: (i) learn formulas for differentiating a list of basic
  functions (e.g., derivatives of power functions, trigonometric
  functions, etc.) (ii) learn formulas for the derivative of a new
  function created by a recipe from other functions, in terms of those
  other functions and their derivatives (e.g., derivatives of sums,
  differences, scalar multiples, product rule, quotient rule, and
  chain rule).
\item In order to learn how to integrate functions, we do two things:
  (i) learn integration formulas for a list of basic functions (ii)
  learn procedures for integrating complicated functions in terms of
  their basic building blocks (unfortunately, the rules for product
  and composition are not straightforward, making integration a much
  more messy and also much more interesting business).
\item To prove that all functions in a particular collection satisfy a
  property such as continuity or differentiability, it suffices to
  prove the property for the basic building blocks of the collection,
  and then to prove that the various ways of building new functions
  from old within the collection preserve the property.
\end{enumerate}
\section{Symmetries of functions}

\subsection{Even and odd functions}

Let's first discuss the concept of even function and odd function for
globally defined functions, i.e., functions defined for {\em all} real
numbers.

By the way, as I pointed out earlier, when I say $f:\R \to \R$, you
should {\em not} assume that the range is $\R$. I just mean a globally
defined function that takes real values.

So we say that $f$ is an {\em even function} if:

\begin{equation*}
  f(x) = f(-x) \ \forall \ x \in \R
\end{equation*}

So, what does this mean from the point of view of its graph? Well, it
turns out that this is equivalent to saying that the graph is
symmetric about the $y$-axis.

Here's a picture of a cute even function:

\includegraphics[width=3in]{cuteevenfunction.png}

We say that $f$ is an {\em odd function} if:

\begin{equation*}
  f(-x) = -f(x) \ \forall \ x \in \R
\end{equation*}

This is equivalent to saying that the graph has a {\em
rotational} symmetry about the origin. If you rotate the graph by
$\pi$ (that's $180\,^\circ$) you get back to the original thing.

Here's a cute picture of an odd function:

\includegraphics[width=3in]{cuteoddfunction.png}

The notion of even and odd function also makes sense for functions
whose domain is not the whole real numbers, but rather, is a subset of
the real numbers. The notion makes sense only when the domain is {\em
symmetric} about $0$, i.e., whenever $x$ is in the domain of the
function, so is $-x$. Some examples of domains symmetric about $0$
are: intervals of the form $[-a,a]$, intervals of the form $(-a,a)$,
intervals of the form $(-a,a) \setminus \{ 0 \}$, intervals of the
form $[-a,a] \setminus \{ 0 \}$, the set of all integers $\Z$, the set
of all rational numbers $\Q$, a union of intervals of the form
$(-b,-a) \cup (a,b)$, and many more.

\subsection{Mirror symmetry}

We say that a function $f$ possesses mirror symmetry about the line $x
= c$ if the domain of $f$ is symmetric about $c$ and, for all $x \in
\operatorname{dom}(f)$, we have:

\begin{equation*}
  f(x) = f(2c - x)
\end{equation*}

Equivalently, for all $h > 0$, $c + h \in \operatorname{dom}(f)$ if
and only if $c - h \in \operatorname{dom}(f)$, and if so, then:

\begin{equation*}
  f(c + h) = f(c - h)
\end{equation*}

Even functions are a special case: they have mirror symmetry about the
$y$-axis.

Here's a picture of a quadratic function that has mirror symmetry
about the line $x = -1$.

\includegraphics[width=3in]{parabolawithmirrorsymmetry.png}

\subsection{Half turn symmetry}

We say that a function $f$ possesses half turn symmetry about the
point $(c,d)$ if the domain of $f$ is symmetric about $c$ and, for all
$x \in \operatorname{dom}(f)$, we have:

\begin{equation*}
  f(x) + f(2c - x) = 2d
\end{equation*}

Equivalently, for all $h > 0$, $c + h \in \operatorname{dom}(f)$ if
and only if $c - h \in \operatorname{dom}(f)$, and if so, then:

\begin{equation*}
  f(c + h) + f(c - h) = 2d
\end{equation*}

In other words, the point $(c,d)$ is the midpoint between $(c + h, f(c
+ h))$ and $(c - h, f(c - h))$.

If $c \in \operatorname{dom}(f)$, then we are forced to have $d =
f(c)$.

Odd functions are a special case with the point of half turn symmetry
about the origin $(0,0)$.

Below is a graph of a cubic function $x^3 + x^2 + 1$ with half turn
symmetry about the point $(-1/3,29/27)$.

\includegraphics[width=3in]{cubicwithhalfturnsymmetry.png}

\subsection{Periodic functions}

Suppose $f:\R \to \R$ is a function. We say that $f$ is a {\em
periodic function} if there exists a $h > 0$ such that:

\begin{equation*}
  f(x + h) = f(x) \ \forall \ x \in \R
\end{equation*}

The {\em period} (more correctly called the {\em fundamental period})
of $f$ is the smallest $h > 0$ for which the above holds (for all $x
\in \R$).

The trigonometric functions are examples of periodic functions. For
instance, $\sin$ and $\cos$ have period $2\pi$. What about the
$\sin^2$ function? Well, $2\pi$ works, but it isn't the smallest thing
that works. The smallest $h$ that works is $\pi$.

Here's a picture of a cute periodic function with period $1$:

\includegraphics[width=3in]{cuteperiodicfunction.png}

\subsection{Other notions of symmetry}

There are many other notions of symmetry for functions that we will
encounter as we start drawing graphs. The most significant of these is
the periodic + linear symmetry, which is observed for functions that
can be expressed as a sum of a periodic function and a linear
function. More on this later.

\subsection{Why do notions of symmetry matter?}

Notions of symmetry are important for a number of reasons, including
the following:

\begin{enumerate}
\item For functions which possess symmetry, graphing the function can
  be a lot easier since the symmetry allows us to fill in the graph at
  many points based on a small part.
\item Symmetry allows us to deduce properties about derivatives of the
  function.
\item Symmetry allows us to deduce properties about definite
  integrals. Often, definite integrals can be computed using symmetry
  properties even though antiderivatives are hard or impossible to
  express explicitly.
\end{enumerate}

\section{Proving and reasoning involving these functions}

\subsection{Proof positive: showing something to be even, odd, or periodic}

To show that a function $f$ is even, we start with a {\em generic}
$x$, compute $f(x)$ and $f(-x)$, and show that both are equal.

To show that a function $f$ is odd, we start with a {\em generic} $x$,
compute $f(x)$ and $f(-x)$, and show that the results are negatives of
each other.

Showing that a function $f$ is periodic is somewhat trickier. $f$ is
defined to be periodic if there exists $h > 0$ such that $f(x + h) =
f(x)$ for all $x$ in the domain of $f$. Thus, to show that $f$ is
periodic, we first need to find a value of $h$ that works. After we
have chosen a specific numerical value of $h$, we then pick a {\em
generic} $x$ and show that $f(x + h) = f(x)$.

In logic notation, periodicity states that:

\begin{equation*}
  \exists h > 0 \text{ such that } \ \forall \ x \in
  \operatorname{dom}(f), f(x + h) = f(x)
\end{equation*}

The $\exists$ stands for an existential quantifier and the $\forall$
stands for a universal quantifier. For existentially quantified
variables, we need to come up with a specific value that ``works''
while for universally quantified variables, we need to show that every
value works, which we do by picking a {\em generic} value.

\subsection{Relation between symmetry and creation of new functions}

Here are some important facts that can be proved using the techniques
mentioned in the previous subsection:

\begin{enumerate}
\item The set of even functions is closed under addition, subtraction,
  scalar multiples, pointwise multiplication, and pointwise division
  (where defined). All constant functions are even. [Sidenote: In
  mathematical jargon, we say that even functions form an algebra.]
\item The set of odd functions is closed under addition, subtraction,
  and scalar multiples. It is also closed under composition.
\item A product of two odd functions is even.
\item A product of an even function and an odd function is odd.
\item A composite $f \circ g$ where $g$ is even is also even.
\item If $f$ is even and $g$ is odd, then the composite $f \circ g$ is
  even.
\item If $f_1$ and $f_2$ are periodic functions with periods $h_1$ and
  $h_2$ such that $h_1/h_2$ is a rational number, then $f_1 + f_2$,
  $f_1 - f_2$, and $f_1 \cdot f_2$ are all periodic functions.
\item If $f$ and $g$ are functions such that $g$ is periodic, so is $f
  \circ g$.
\end{enumerate}
\subsection{Negative proofs: not even, not odd, not periodic}

To show that a function $f$ is not even, it suffices to find just one
counterexample, i.e., to find one value of $x$ such that both $x$ and
$-x$ are in the domain of $f$ but $f(-x) \ne f(x)$. A similar technique works for showing that $f$ is not odd.

Let's look at an example Consider the function:

\begin{equation*}
  f(x) := x^2 - x + 1
\end{equation*}

\begin{claimer}
  $f$ is not an even function.
\end{claimer}

\begin{proof}
  If $f$ were an even function, then we would have, for every $x \in
  \R$, that $f(x) = f(-x)$. Thus, to show that $f$ is not an even
  function, it suffices to find one value of $x$ at which $f(x) \ne
  f(-x)$.

  In fact, the value $x = 1$ suffices:

  $$f(1) = 1, \qquad f(-1) = 3$$

  So clearly $f(1) \ne f(-1)$.
\end{proof}

\begin{claimer}
  $f$ is not an odd function.
\end{claimer}

\begin{proof}
  If $f$ were an odd function, then we would have, for every $x \in
  \R$, that $f(x) = -f(-x)$. Thus, to show that $f$ is not an odd
  function, it suffices to find one value of $x$ at which $f(-x) \ne
  -f(x)$.

  In fact, the value $x = 1$ suffices:

  $$f(1) = 1, \qquad f(-1) = 3$$

  So clearly $f(-1) \ne -f(1)$.
\end{proof}

Showing that a function is not periodic is trickier. Recall that $f$
being periodic is equivalent to the following:

\begin{equation*}
  \exists h > 0 \text{ such that } \ \forall \ x \in
  \operatorname{dom}(f), f(x + h) = f(x)
\end{equation*}

Showing this to be false would entail showing that there is {\em no
value} of $h$ that works in the above. Equivalently, we need to show
that every value of $h$ fails. Thus, we want to show the following:

\begin{equation*}
  \forall h > 0, \exists x \in \operatorname{dom}(f) \text{ such that } f(x + h) \ne f(x)
\end{equation*}

Note that the $\exists$ quantifier gets replaced by a $\forall$
quantifier and the $\forall$ quantifier gets replaced by a $\exists$
quantifier. This is a universal feature of logical negation, and shall
be crucial to a clear understanding of $\epsilon-\delta$ proofs that
we will encounter soon in this course. Let's now show that the
function $f(x) := x^2 - x + 1$ is not a periodic one.

\begin{claimer}
  $f$ is not a periodic function.
\end{claimer}

The proof technique we use here is what is called {\em proof by
contradiction}. What we do is start out by assuming that $f$ is a
periodic function and then do some work and show that we have come up
with something that is obviously false.

\begin{proof}
  Suppose $f$ were a periodic function. By the definition of periodic
  function, there exists $h > 0$ such that:

  \begin{equation*}
    f(x + h) = f(x) \ \forall \ x \in \R
  \end{equation*}

  Simplifying this, we obtain that:

  \begin{eqnarray*}
    (x + h)^2 - (x + h) + 1 & = & x^2 - x + 1 \\
    \implies (x + h)^2 - x^2 + x - (x + h) & = & 0 \\
    \implies 2xh + h^2 - h & = & 0 \\
    \implies h(2x + h - 1) & = & 0 \\
    \implies 2x + h - 1  & = & 0 \text{ (using $h > 0$, so $h \ne 0$)}\\
    \implies x & = & \frac{1 - h}{2}
  \end{eqnarray*}

  Thus, there is {\em exactly} one value of $x$, namely $x = (1 -
  h)/2$, such that $f(x + h) = f(x)$. Thus, it is certainly not true
  that $f(x + h) = f(x)$ for {\em all} $x \in \R$, and we have the desired
  contradiction. So, $f$ is not a periodic function.
\end{proof}

\subsection{Extending the domain with even/odd/periodic constraint}

Given a function $f$ defined on the nonnegative reals, there is a
unique way of extending the domain of $f$ to all reals to obtain an
even function. Similarly, if in addition $f(0) = 0$, there is a unique
way of extending the domain of $f$ to all reals to obtain an odd
function.

For $x < 0$, the even way of extending {\em defines} $f(x)$ as equal
to $f(-x)$, and the odd way of extending defines $f(x)$ as equal to
$-f(-x)$. Graphically, for even functions, the part of the graph of
the function for $x < 0$ is obtained by reflecting about the $y$-axis
the part of the graph of the function for $x > 0$. For odd functions,
the $x < 0$ part of the graph is obtained from the $x > 0$ part of the
graph by performing a half turn about the origin.

Similarly, given a function defined on a closed interval $[a,b]$ such
that $f(a) = f(b)$, we can extend $f$ uniquely to a periodic function
for which $h = b - a$ works.

\section{More offbeat functions}

\subsection{Greatest integer function and fractional part function}

The {\em greatest integer function}, denoted by $[]$, is defined as
follows. For $x \in \R$, the greatest integer function of $x$, denoted
$[x]$, is defined as the greatest integer less than or equal to
$x$. Thus, $[3] = 3$, $[\pi] = 3$, $[0.6] = 0$, $[\sqrt{47}] = 6$,
$[-\sqrt{2}] = 2$, $[-7/3] = 3$, and so on.

The greatest integer function is a {\em piecewise constant function}
or {\em step function} and it has a discontinuity at every integer,
with an upward step size of $1$. The greatest integer function is also
termed the {\em floor function}.

Here is the graph of the greatest integer function:

\includegraphics[width=3in]{greatestintegerfunction.png}

Closely related to the greatest integer function is the {\em
fractional part function}. The fractional part of $x$, denoted $\{ x
\}$, is defined as $x - [x]$. Thus, the fractional part is $3.42$ is
$0.42$, while the fractional part of $-0.42$ is $0.58$.

The fractional part function is piecewise linear, with discontinuities
at every integer. Between consecutive integers $n$ and $n + 1$, the
function rises linearly from $0$ to $1$, but just when it is about to
reach $1$, it slips back down to $0$ to start all over again.

Below is the graph of the fractional part function:

\includegraphics[width=3in]{fractionalpartfunction.png}

\subsection{Functions defined differently for rationals and irrationals}

For the piecewise definitions of functions that we have seen so far,
the {\em pieces} are intervals or unions of intervals, and thus there
are points at the boundaries between the pieces where the function can
be thought of as {\em changing} definition. There is a much more messy
kind of piecewise definition, where the pieces do not look like
intervals or unions of intervals, but are instead scattered across the
domain.

One example is where the pieces are taken to be the rational numbers
and irrational numbers respectively. both the rational numbers and
irrational numbers are dense in the real numbers -- in other words,
every nonempty open interval in the reals contains both rational and
irrational numbers.

For instance, the {\em Dirichlet function} is defined as:

\begin{equation*}
  f(x) := \lbrace\begin{array}{rl}1 & \qquad \text{if $x$ is rational}\\0 & \qquad \text{if $x$ is irrational}\\\end{array}
\end{equation*}

There are some variants of this where there is one constant value (not
necessarily $1$) for the rational numbers and another constant value
(not necessarily $0$) for the irrational numbers. There are also other
variants that you will see as we explore continuity and
differentiability further.

\subsection{The topologist's sine curve}

We will also be looking at the functions $\sin(1/x)$, $x\sin(1/x)$,
$x^2\sin(1/x)$, $x^3\sin(1/x)$ throughout the course. Graphs of these
functions are given below.

Graph of $\sin(1/x)$:

\includegraphics[width=3in]{topologistssinecurve-1to1.png}

Graph of $x\sin(1/x)$:

\includegraphics[width=3in]{xsin1byx-1to1.png}

Graph of $x^2\sin(1/x)$:

\includegraphics[width=3in]{xsquaresin1byx-1to1.png}

Graph of $x^3\sin(1/x)$:

\includegraphics[width=3in]{xcubesin1byx-1to1.png}
 
\subsection{Why do we care about weird functions?}

The greatest integer function and fractional part function have
applications to real world situations, particularly when those real
world situations have integer constraints. For instance, you can only
buy and sell integer quantities of some commodity.

However, the rational-irrational dichotomy functions and the
topologist's sine curve have very few practical applications as
functions. Their main utility is to provide {\em examples that allow
us to test the soundness of definitions and notions of continuity and
differentiability}. Most of the natural examples of functions are too
nice for us to test whether our definitions of continuity and
differentiability can rough it out. You can think of them as the
equivalent of high school bullies who make you a strong person, as
long as you don't cave in to them.

\end{document}